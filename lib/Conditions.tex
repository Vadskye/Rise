\chapter{Conditions}\label{Conditions}

%\condition{ability Damaged} The creature has temporarily lost 1 or more attribute score points. Lost points return at a rate of 1 per day unless noted otherwise by the condition dealing the damage. A creature with Strength \minus10 falls to the ground and is helpless. A creature with Dexterity \minus10 is paralyzed. A creature with Constitution \minus10 is dead. A creature with Intelligence, Wisdom, or Charisma \minus10 is unconscious. Ability damage is different from penalties to attribute scores, which go away when the conditions causing them go away.

%\condition{ability Drained} The creature has permanently lost 1 or more attribute score points. The creature can regain these points only through magical means. A creature with Strength \minus10 falls to the ground and is helpless. A creature with Dexterity \minus10 is paralyzed. A creature with Constitution \minus10 is dead. A creature with Intelligence, Wisdom, or Charisma \minus10 is unconscious.

\condition{blinded} A blinded creature cannot see. It automatically fails at actions which depend on vision, including simply seeing the locations of other objects and creatures (but see \pcref{Awareness}). It is \severelyimpaired at any vision-related attacks and checks, even if it knows the location of its targets. Finally, it is \defenseless.

%\condition{blown Away} Depending on its size, a creature can be blown away by winds of high velocity. A creature on the ground that is blown away is knocked down and rolls 1d4 \mtimes 10 feet, taking 1d4 points of nonlethal damage per 10 feet. A flying creature that is blown away is blown back 2d6 \mtimes 10 feet and takes 2d6 points of nonlethal damage due to battering and buffering.

\condition{bloodied} At or below half hit points. Bloodied creatures take a \minus5 penalty to Fortitude and Mental defense.

\condition{charmed} A charmed creature is mentally influenced to like another creature.
It always sees the words and actions of the creature that charmed it in the most favorable way, as a close friend or trusted ally.
A charmed creature cannot be controlled like an automaton, but can be persuaded to take particular actions with the Persuasion skill (see \pcref{Persuasion}).
It treats the creature that charmed it as a friend (a \plus10 relationship modifier) for the purpose of Persuasion checks.

%\condition{checked} Prevented from achieving forward motion by an applied force, such as wind. Checked creatures on the ground merely stop. Checked flying creatures move back a distance specified in the description of the effect.

\condition{confused} A confused creature is unable to independently control its actions. \confusionexplanation

\condition{crouching} A crouching creature gains a \plus2 bonus to physical defenses against ranged attacks.
However, it takes a \minus2 penalty to physical accuracy with melee attacks and physical defenses against melee attacks, and moves at half speed.

\condition{dazed} A dazed creature cannot act during the movement phase.

%\condition{dazzled} The creature is unable to see well because of
%overstimulation of the eyes. A dazzled creature has a 20\% miss chance on all
%attack rolls and takes a \minus4 penalty to visual Awareness checks. He is also unable to see with darkvision.

\condition{dead} A dead creature's soul leaves its body. Dead creatures cannot benefit from normal or magical healing, but they can be restored to life via magic (see \pcref{Resurrecting the Dead}). A dead body decays normally unless magically preserved.

\condition{deafened} A deafened creature cannot hear. It automatically fails at actions which depend on hearing. It is \impaired at any hearing-related attacks and checks, as well as when casting any spell with verbal components.

\condition{defenseless} A defenseless creature is unable to defend itself in melee combat. It takes a \minus5 penalty to physical defenses against melee attacks. Any creature not capable of using a weapon or shield to defend itself, such as most unarmed creatures, is defenseless.

\condition{disabled} A disabled creature has no hit points remaining.
It is both \staggered and \bloodied.

\condition{disoriented} During each movement phase, a disoriented creature is forced to

\condition{dominated} A charmed creature is mentally compelled to obey another creature.
It obeys the commands of the creature of the dominated it unquestioningly, as an automaton.
If it does not understand the language of the creature that dominated it, it still attempts to obey as much as possible, and simple commmands (such as ``attack'' or ``follow'') can usually be communicated successfully.

\condition{dying} A dying creature is unconscious and near death. See \pcref{Dying}.

\condition{encumbered} An encumbered creature has its motion restricted by armor or weight. It may be unable to use certain class features and abilities which require free motion. See \pcref{Encumbrance} for details.

\condition{entangled} An entangled creature is ensnared in a net or other physical restraint.
It moves at half speed, cannot sprint or charge, and is \impaired with physical attacks and checks.

\condition{exhausted} An exhausted creature cannot sprint or charge, moves at half speed, and is \impaired with attacks and checks. After 1 hour of complete rest, an exhausted creature becomes fatigued. A fatigued creature becomes exhausted by doing something else that would normally cause fatigue.

\condition{fascinated} A fascinated creature can take no actions. It remains in place, giving its total attention to some object, creature, or effect. It takes a \minus5 penalty to skill checks made as reactions, such as Awareness checks. If the creature notices any threat against it, such as an approaching enemy, it is no longer fascinated.

\condition{fatigued} A fatigued creature can neither sprint nor charge, and moves at half speed. If the fatigue does not have a set duration, doing anything that would normally cause fatigue causes the fatigued creature to become exhausted. After 8 hours of complete rest, fatigued creatures are no longer fatigued.

\condition{frightened} A frightened creature flees by any means necessary if it is within 30 feet of the source of its fear. If unable to flee, it may fight, but is \severelyimpaired with attacks and checks while doing so.

\condition{grappled} A grappled creature is in wrestling or some other form of hand-to-hand struggle with one or more attackers. While grappled, you suffer certain penalties and restrictions, as described below.

\begin{itemize}
    \item You must use a free hand (or equivalent limbs) to grapple, preventing you from taking any actions which would require having two free hands. If you cannot free a hand, you suffer a \minus10 penalty to accuracy with all physical attacks, including grapple attacks, until you have a free hand.
    \item You take a \minus4 penalty to physical defenses against creatures you are not grappling with.
    \item You take a \minus4 penalty to accuracy with weapons that are not light, since they are too large and cumbersome to be used effectively in a grapple.
    \item Spellcasting is extremely difficult. You cannot cast spells with somatic components. Using a spell-like ability or casting a spell without somatic components requires a Concentration check with a DC equal to 20 \add double spell level.
    \item You cannot move normally (but see Move the Grapple, below).
\end{itemize}

Other than the restrictions listed above, you can act normally. You can also try to move the grapple, escape the grapple, or pin your opponent. See \pref{Grapple} for more information.

\condition{helpless} A helpless creature is completely at an opponent's mercy. Its physical defenses are equal to 10 \add its size modifier. Paralyzed, bound, and unconscious creatures are helpless. Helpless creatures can be killed instantly by a coup de grace (see \pcref{Coup de Grace}).

\condition{incorporeal} Having no physical body. Incorporeal creatures are immune to all nonmagical attack forms. They can be harmed only by other incorporeal creatures, magic weapons, spells, spell-like effects, or supernatural effects.

\condition{ignited} An ignited creature has been set on fire. It takes 1d6 fire damage at the end of each round, and is \impaired with attacks and checks. As a move action, an ignited creature can make a DC 15 Dexterity check to put out the flames. This action requires a free hand. Dropping prone as part of the action gives a \plus5 bonus on this check.

\condition{immobilized} An immobilized creature can't move out of the space it was in when it became immobilized. Immobilized flying creatures that have the ability to hover can maintain their initial altitude. All other flying creatures subjected to this condition descend at a rate of 20 feet per round until they reach the ground, taking no falling damage.

\condition{impaired} An impaired creature has a 20\% chance to fail when it attempts some actions. The actions that are impaired are defined in the ability which impairs the creature. For example, a creature affected by the \spell{bane} spell suffers a 20\% chance of failure with all attacks and checks.

\condition{invisible} An invisible creature or object cannot be seen. Other creatures are \defenseless against an invisible creature. Attackers suffer a 50\% miss chance even if they know the location of the invisible creature. See \pcref{Awareness} and \pcref{Stealth}, for how to identify invisible creatures.

%\condition{knocked Down} Depending on their size, creatures can be knocked down by winds of high velocity. Creatures on the ground are knocked prone by the force of the wind. Flying creatures are instead blown back 1d6 \mtimes 10 feet.

\condition{nauseated} A nauseated creature moves at half speed, and is unable to act during the action phase.

\condition{negative levels}[negative level] A creature with a negative level takes a \minus1 penalty to accuracy, defenses, and checks. Additionally, the creature's maximum and current hit points are reduced by an amount equal to its level. If the creature has at least as many negative levels as it has levels, it dies.

\condition{panicked} A panicked creature must flee by any means necessary from the source of its fear. If unable to flee, it must do nothing other than take the total defense action every round. It may only stop fleeing if it is at least 1,000 feet from the source of its fear, or believes the source of its fear is unable to find or affect it for other reasons (such as if the creature is across a vast chasm).

\condition{paralyzed} A paralyzed creature is unable to take physical actions. It has effective Dexterity and Strength scores of \minus10 and is helpless, but can take purely mental actions. This can cause flying creatures to crash, swimming creatures to drown, and so on. A creature can move through a space occupied by a paralyzed creature -- ally or otherwise. Each square occupied by a paralyzed creature, however, counts as 2 squares.

\condition{partially blinded} A partially blinded creature has difficulty seeing. It loses any special vision properties it has, such as darkvision or low-light vision. It is \impaired at any vision-related attacks and checks.

\condition{petrified} A petrified creature has been turned to stone. It is neither alive nor dead, but is unable to take actions, and its body is an inanimate statue. If the statue is broken or damaged before the creature is restored to its original state, the creature has equivalent damage or deformities.

\condition{pinned} A pinned creature is held completely immobile in a grapple. Like a helpless creature, its physical defenses are equal to 10 \add its size modifier. Unlike a helpless creature, a pinned creature cannot be killed instantly by a coup de grace.

\condition{prone} The creature is on the ground. A prone creature takes a \minus4 penalty to accuracy with physical melee attacks and physical defenses. It gains a \minus4 bonus to physical defenses against ranged attacks.

Standing up is a move action that generally requires one free hand.

\condition{severely Impaired} A severely impaired creature has a 50\% chance to fail at a particular kind of action. The type of action that is impaired is defined in the ability which impairs the creature. For example, a creature with a severe visual impairment suffers a 50\% chance of failure at all tasks that depend on sight.

\condition{severely impaired} A severely impaired creature has a 50\% chance to fail when it attempts some actions. The affected actions are defined in the ability which impairs the creature.

\condition{shaken} A shaken creature is \impaired with all actions as long as it is within 30 feet of the source of its fear.

If the source of fear is a creature and is rendered \helpless, surrenders, or is otherwise unable to fight, this effect is broken.

\condition{sickened} A sickened creature moves at half speed.

\condition{slowed} A slowed creature cannot act during the movement phase, and moves at half speed.

\parhead{Stable} A creature who was dying but who has stopped losing hit points and still has critical damage is stable. The creature is no longer dying, but is still unconscious. See \pcref{Stable}.

\condition{staggered} A staggered creature cannot act during the movement phase. A creature with 0 hit points is staggered.

\condition{stunned} A stunned creature cannot take actions.

\condition{taunted} A taunted creature is compelled to attack the creature that taunted it.
It is \impaired with all attacks that do not directly affect the taunting creature.
If the taunting creature is rendered \helpless, surrenders, or is otherwise unable to fight, this effect is broken.

\condition{unaware} An unaware creature does not know that it is being attacked. Successful physical attacks against an unaware creature automatically threaten critical hits. After being attacked, an unaware creature typically stops being unaware of future attacks, even if cannot see or identify its attacker.

\condition{unconscious} An unconscious creature is helpless.

\condition{unencumbered} An unencumbered creature does not have its motion restricted by armor or weight. See \pcref{Encumbrance} for details.
