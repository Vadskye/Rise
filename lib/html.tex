% This package uses a different parsing argument for PDF and HTML output
\usepackage[tex4ht,bookmarks=true,breaklinks=true,colorlinks=true,linkcolor=MidnightBlue,bookmarksopen=true,bookmarksopenlevel=1,bookmarksdepth=1]{hyperref}

\ExplSyntaxOn

\renewcommand{\chaptermark}[1]{\markboth{\chaptername \: \thechapter \: - \: #1}{}}
\renewcommand{\sectionmark}[1]{\markright{#1}}

%make bigger section headers
% \titleformat*{\section}{\hspace*{-0.5em}\LARGE\bfseries}
\titleformat{\section}
    {\LARGE\bfseries}   % The style of the section title
    {}                             % a label prefix
    {0em}                          % How much space exists between the prefix and the title
    {\hspace{-0.5em}}    % A code representing the section
\titleformat{\subsection}
    {\large\bfseries}   % The style of the section title
    {}                             % a label prefix
    {0em}                          % How much space exists between the prefix and the title
    {\hspace{-0.5em}\vspace{-0.75em}}    % A code representing the section
    [\hspace{-0.5em}\titlerule]
\titleformat{\subsubsection}
    {\bfseries}
    {}
    {0em}
    {\hspace{-0.5em}}

\DeclareDocumentEnvironment{spellcontent}{}{}{}

\DeclareDocumentEnvironment{monsterstatistics}{}{}{}

% The HTML versions of mdframed environments need dumb hacks to fix parsing issues
\newmdenv[
    style=colorenv,
    backgroundcolor=LightCyan,
]{spelltargetinginfomdframed}
\DeclareDocumentEnvironment{spelltargetinginfo}{}
{
    \begin{spelltargetinginfomdframed}
}
{
    \end{spelltargetinginfomdframed}
}

\newmdenv[
    style=colorenv,
    backgroundcolor=LightCyan,
]{augmenttargetinginfomdframed}
\DeclareDocumentEnvironment{augmenttargetinginfo}{}
{
    \begin{augmenttargetinginfomdframed}
}
{
    \end{augmenttargetinginfomdframed}
}

\newmdenv[
    style=colorenv,
    backgroundcolor=Lavender,
]{spelleffectsmdframed}
\DeclareDocumentEnvironment{spelleffects}{}
{
    \begin{spelleffectsmdframed}
}
{
    \end{spelleffectsmdframed}
}

\newmdenv[
    style=colorenv,
    backgroundcolor=Lavender,
]{augmenteffectsmdframed}
\DeclareDocumentEnvironment{augmenteffects}{}
{
    \begin{augmenteffectsmdframed}
}
{
    \end{augmenteffectsmdframed}
}

\newmdenv[
    style=colorenv,
    backgroundcolor=Gainsboro,
    leftline=true,
    rightline=true,
]{spellfootermdframed}
\DeclareDocumentEnvironment{spellfooter}{}
{
    \begin{spellfootermdframed}
}
{
    \end{spellfootermdframed}
}

\newmdenv[
    style=colorenv,
    backgroundcolor=Gainsboro,
    leftline=true,
    rightline=true,
    bottomline=true,
    topline=true,
]{monsterfootermdframed}
\DeclareDocumentEnvironment{monsterfooter}{}
{
    \begin{monsterfootermdframed}
}
{
    \end{monsterfootermdframed}
}


% The HTML version of spelltwocol has to put its tabularx inside of a figure
\DeclareDocumentCommand{\spelltwocol}{m m}
{
    \renewcommand{\arraystretch}{0.5}
    \begin{figure}
    \begin{tabularx}{\linewidth}{>{\lcol}X~l}
        #1 & #2
    \end{tabularx}
    \end{figure}
}

% The HTML version of termlink can't use mbox
% args:
%   #1: reference name of term, if different from display name
%   #2: display name of term
%   #3: hyperlink prefix
\DeclareDocumentCommand{\termlink}{o m m}
{
    \IfValueTF{#1}
    {
        % Enable this to test for missing glossary definitions
        % \pref{#1}
        % Enable this to test for unnecessary glossary entries
        % \label{#1}

        % \mbox causes inconsistent but bizarre "missing space" issues
        % when used in HTML generation, so we omit it from this file
        \hyperlink{#3:#1}{\textbf{#2}}
    }
    {
        % Enable this to test for missing glossary definitions
        % \pref{#2}
        % Enable this to test for unnecessary glossary entries
        % \label{#2}

        \hyperlink{#3:#2}{\textbf{#2}}
    }
}

% The HTML versions of pref/pcref don't want to reference pages at all
\DeclareDocumentCommand{\pcref}{o m}
{
    \IfValueTF{#1}
    {
        \hyperref[#1]{#2}
    }
    {
        \hyperref[#2]{#2}
    }
}
\newcommand{\pref}[1]{\hyperref[#1]{here}}
\newcommand{\tref}[1]{Table~\ref{cap:#1}:~\hyperref[cap:#1]{#1}}
\newcommand{\trefnp}[1]{Table~\ref{cap:#1}:~\hyperref[cap:#1]{#1}}
\newcommand{\featpref}[1]{\hyperref[feat:#1]{here}}
\newcommand{\mitempref}[1]{\hyperref[item:#1]{here}}
\newcommand{\featpcref}[1]{\hyperref[feat:#1]{#1}}

\ExplSyntaxOff
