\chapter{Magic Items}

Magic items are objects that have been imbued with magical energy. They can take almost any form, and their potential uses are only as limited as the magic that created them.

\section{Magic Item Types}
Magic items are divided into three broad categories:
\begin{itemize}
    \item Apparel items provide access to their abilities while worn.
        A \mitem{flaming burst full plate} and a \mitem{ring of protection} are apparel items.
    \item Implements provide access to their abilities when wielded.
        A \mitem{flaming longsword} and a \mitem{wand of fire} are implements.
    \item Tools provide access to their abilities when used in some way.
        A \mitem{bag of holding} is a tool.
\end{itemize}

\section{Using Magic Items}

To use a magic item, it must be activated, although sometimes activation simply means wearing or holding the item.

\subsection{Daily Item Activations}
In general, you can activate any combination of magic items you possess a number of times per day half your level or half your Willpower, whichever is higher.
Once you have used up your activations for the day, you can't activate any more magic items until the next day, though you can continue to use items that don't require activation (such as most magic weapons) normally.

\subsection{Activation Methods}
Magic items can be activated in one of four ways:

\begin{itemize}
    \item Command word
    \item Specific action
    \item Spell completion
    \item Triggered
\end{itemize}

These methods are described below.

\parhead{Command Word} A character must speak a special word defined by the item to activate it.
Unless otherwise stated, activating a command word magic item is a standard action.

\parhead{Specific Action} A character must perform a specific action defined by the item to activate it.
For example, a creature might need to drink the item, wrap its cloak around itself, or perform some other task.
The activation time for such items can be varied.
Unless otherwise stated, activating a specific action magic item is a standard action.

\parhead{Spell Completion} A character must complete a spell defined by the item to activate it.
Unless otherwise stated, activating a spell completion magic item is a standard action.
Both verbal and somatic components may be required, as appropriate to the spell to be completed.

In order to activate a spell completion item, a character must be able to cast spells of that level and she must have that spell on his spell list.

\parhead{Triggered} A creature must fulfill some triggering condition defined by the item to activate it.
For example, a triggered magic item might activate when a character strikes a foe, is damaged, or is affected by a particular kind of magic.
Unless otherwise stated, activating a triggered item is an immediate action.
It is done completely mentally, requiring no physical action, so it can be done even while paralyzed.
Some triggered items activate automatically, without requiring an action of any kind.

\section{Magic Item Effects}
Most magic items either provide numerical bonuses or emulate the effect of a spell in some way.

\subsection{Removing Magic Items}
Unless otherwise noted, magic items that have effects on the creature using the item must continue to be worn or held as long as the effect lasts.
If a magic item is removed before its effect's duration is up, the effect ends.
Items which are consumed when used or which do not affect their user are unaffected by this rule.

\subsection{Special Attacks}

If a magic item requires a successful special attack to have its full effect, the accuracy is listed in the item's description.
Typically, the accuracy is equal to the item's power.

\section{Item Description Format}

Each general type of magic item gets an overall description, followed by descriptions of specific items.

General descriptions include notes on activation, random generation, and other material.
The AD, hardness, hit points, and break DC are given for typical examples of some magic items.
The AD assumes that the item is unattended and includes a \minus10 penalty for the item's effective Dexterity of \minus10.
If a creature holds the item, use the creature's Dexterity in place of the \minus10 penalty.

Some individual items, notably those that simply store spells and nothing else, don't get full-blown descriptions.
Reference the spell's description for details, modified by the form of the item (potion, scroll, wand, and so on).
Assume that the spell is cast at the minimum level required to cast it.

Items with full descriptions have their abilities detailed, and each of the following topics is covered in notational form at the end of the description.

\parhead{Base Power} The next item in a notational entry gives the base power of the item.
An item's power can affect how strong its abilities are, as well as its special defenses when unattended and its effective spellpower for the purpose of effects like \spell{dispel magic}.
An item's power is equal to its base power or the level of the creature using it \add the number of legend points that creature has, whichever is higher.

For potions and scrolls, the item's base power must be at least twice the level of the spell contained.
Generally, an item's base power is the same as the minimum spellpower required to create the item.

\parhead{Aura} Most of the time, a Spellcraft check will reveal the school of magic associated with a magic item and the strength of the aura an item emits.
This information (when applicable) is given at the beginning of the item's notational entry.
See the Spellcraft skill for details.

\parhead{Requirements} The qualifications that must be met to create the item, %described in \pcref{Creating Magic Items}.

\parhead{Market Price} This gold piece value, given following the word ``Price,'' represents the price someone should expect to pay to buy the item.
The market price for an item that can be constructed with an item creation feat is usually equal to the base price plus the price for any components.

\parhead{Cost to Create} The next part of a notational entry is the cost in gp to create the item, given following the word ``Cost.'' This information appears only for items with components which make their market prices higher than their base prices.
The cost to create includes the costs derived from the base cost plus the costs of the components.

Items without components do not have a ``Cost'' entry.
For them, the market price and the base price are the same.
The cost in gp is 1/2 the market price.

\parhead{Weight} The notational entry for many wondrous items ends with a value for the item's weight.
When a weight figure is not given, the item has no weight worth noting (for purposes of determining how much of a load a character can carry).

\section{Apparel}

\subsection{Apparel Item Limitations}

There are a wide variety of magic items, but there is a limit to how many apparel items a character can wear at any given time.
For humanoid-shaped creatures, there are four core areas on the body that can hold magic items: the arms, the head, the torso, and the legs.
A creature can wear only one item in the hands, arms, and legs, but it can use two torso items, provided that they are not identical (such as two belts).
In addition, it can wear a suit of armor and two magic rings (one on each hand), for a total of eight item slots.
Any additional items worn do not function.

Creatures with non-humanoid body structures may have different arrangement of items.
For example, a wolf would be able to wear items on its head, its torso, and on two different sets of legs.
Regardless of body shape or size, most creatures may not have more than eight item slots, and many have fewer.
A list of common items and their placement on the body is given below.

\begin{itemize}
    \item Arm items:
        \begin{itemize}
            \item Bracers, bracelets, gauntlets, and gloves.
        \end{itemize}
    \item Armor:
        \begin{itemize}
            \item Body armor, shields
        \end{itemize}
    \item Head items:
        \begin{itemize}
            \item Hats, headbands, and helmets
        \end{itemize}
    \item Leg items:
        \begin{itemize}
            \item Boots and shoes.
        \end{itemize}
    \item Rings
    \item Torso items:
        \begin{itemize}
            \item Amulets, belts, cloaks, mantles, necklaces, robes, shirts, and vests.
        \end{itemize}
\end{itemize}

Of course, a character may carry or possess as many items of the same type as he wishes.
However, additional equipped items beyond those listed above have no effect.

A rare few apparel items can be ``worn'' without taking up space on a character's body.
The description of an item indicates when it has this property.

\begin{dtable}
    \lcaption{General Apparel Items}
    \begin{dtabularx}{\columnwidth}{>{\lcol}X l l l}
        \tb{Special Ability}              & \tb{Cost} & \tb{Item Level} & \tb{Location} \\
        \hline
        Ring of Protection \plus1         & 100 gp    & 2nd             & Ring          \\
        Ring of Energy Resistance, Lesser & 200 gp    & 3rd             & Ring          \\
        Boots of Elvenkind                & 500 gp    & 4th             & Legs          \\
        Ring of Protection \plus2         & 500 gp    & 4th             & Ring          \\
        Belt of Constitution              & 2,000 gp  & 8th             & Torso         \\
        Circlet of Wisdom                 & 2,000 gp  & 8th             & Head          \\
        Cloak of Charisma                 & 2,000 gp  & 8th             & Torso         \\
        Gauntlets of Strength             & 2,000 gp  & 8th             & Arms          \\
        Gloves of Dexterity               & 2,000 gp  & 8th             & Arms          \\
        Headband of Intellect             & 2,000 gp  & 8th             & Head          \\
        Boots of Mobility                 & 1,000 gp  & 6th             & Legs          \\
        Ring of Sustenance                & 1,000 gp  & 6th             & Ring          \\
        Ring of Energy Resistance         & 2,000 gp  & 8th             & Ring          \\
        Ring of Protection \plus3         & 2,500 gp  & 8th             & Ring          \\
        Boots of Speed                    & 5,000 gp  & 10th            & Legs          \\
        Belt of Constitution, Greater     & 12,000 gp & 12th            & Torso         \\
        Circlet of Wisdom, Greater        & 12,000 gp & 12th            & Head          \\
        Cloak of Charisma, Greater        & 12,000 gp & 12th            & Torso         \\
        Gauntlets of Strength, Greater    & 12,000 gp & 12th            & Arms          \\
        Gloves of Dexterity, Greater      & 12,000 gp & 12th            & Arms          \\
        Headband of Intellect, Greater    & 12,000 gp & 12th            & Head          \\
        Ring of Protection \plus4         & 12,500 gp & 12th            & Ring          \\
        Boots of Teleportation, Greater   & 60,000 gp & 16th            & Legs          \\
        Ring of Protection \plus5         & 62,500 gp & 16th            & Ring          \\
    \end{dtabularx}
\end{dtable}

\subsection{Armor Overview}

Magic body armor and shields protect the wearer to a greater extent than their nonmagical equivalents.
All magic armor has an enhancement bonus to improve your hit points and ability to resist attacks.
In addition to an enhancement bonus, magic armor may have special abilities or be made of an unusual material.

Body armor is always created so that even if the type of armor comes with boots or gauntlets, these pieces can be switched for other magic boots or gauntlets.

\subsubsection{Armor Enhancement Bonuses}\label{Armor Enhancement Bonuses}

Magic armor can have enhancement bonuses ranging from \plus1 to \plus5.
Each \plus1 of enhancement bonus grants temporary hit points equal the item's power, and grants you an additional defensive legend point each day.
If you stop using the armor, you lose the temporary hit points and the legend points.

These bonuses can only be gained once per day, regardless of the number of items you use.
If you use both magic body armor and a magic shield, or change between different pieces of armor, use only the highest bonus that applies.
If you change from a weaker magical item to a stronger magical item, you gain bonuses equal to the difference between the two items.

\subsubsection{Prices}\label{Armor Prices}
The prices of enhancement bonuses to armor are listed in \trefnp{Magic Armor Prices}, and the prices of special abilities are listed on \trefnp{Magic Armor Special Abilities}.
If armor has a special ability, the price of the special ability is added to the price of the armor.
The number of special abilities on the armor cannot exceed the enhancement bonus of the armor.
Additionally, the price of all special abilities cannot exceed twice the price of the enhancement bonus on the armor.

\begin{dtable}
    \lcaption{Magic Armor Prices}
    \begin{dtabularx}{\columnwidth} {>{\ccol}X c c}
        \tb{Enhancement Bonus} & \tb{Base Price} & \tb{Item Level} \\
        \hline
        \plus1 armor/shield       & 100 gp             & 2nd                \\
        \plus2 armor/shield       & 500 gp             & 4th                \\
        \plus3 armor/shield       & 2,500 gp           & 8th                \\
        \plus4 armor/shield       & 12,500 gp          & 12th               \\
        \plus5 armor/shield       & 62,500 gp          & 16th               \\
    \end{dtabularx}
\end{dtable}

\parhead{Base Power for Armor and Shields} The base power of a magic shield or magic armor with a special ability is given in the item description.
For an item with only an enhancement bonus, the base power is equal to three times the enhancement bonus.
If an item has both an enhancement bonus and a special ability, the higher of the two base powers must be met.

\parhead{Shields} Shield enhancement bonuses do not act as accuracy or damage bonuses when the shield is used in a bash.
However, a shield can be enhanced as a weapon.

\parhead{Hardness and Hit Points} Each \plus1 of enhancement bonus adds 2 to an armor or shield's hardness and \plus10 to its hit points.

\parhead{Activation} A character benefits from magic armor and shields in exactly the way a character benefits from nonmagical armor and shields - by wearing them.
Special abilities on body armor are usually activated if the character is struck or damaged, while special abilities on shields are usually activated if the character avoids an attack.

\subsection{Armor Special Abilities}\label{Armor Special Abilities}

\begin{dtable}
    \lcaption{Magic Armor Special Abilities}
    \begin{dtabularx}{\columnwidth}{>{\lcol}X l l l}
        \tb{Special Ability}      & \tb{Cost}  & \tb{Item Level} & \tb{Location} \\
        \hline
        Feather \minus1           & 100 gp     & 2nd  & Body, Shield \\
        Energy Resistance, Lesser & 200 gp     & 3rd  & Armor        \\
        Invulnerability, Lesser   & 200 gp     & 3rd  & Armor        \\
        Feather \minus2           & 500 gp     & 4th  & Body, Shield \\
        Bashing                   & 800 gp     & 5th  & Shield       \\
        Flaming Burst             & 800 gp     & 5th  & Body, Shield \\
        Freezing Burst            & 800 gp     & 5th  & Body, Shield \\
        Glamered                  & 800 gp     & 5th  & Armor        \\
        Shocking Burst            & 800 gp     & 5th  & Body, Shield \\
        Ghost Touch               & 2,000 gp   & 8th  & Body, Shield \\
        Energy Resistance         & 2,000 gp   & 8th  & Armor        \\
        Feather \minus3           & 2,500 gp   & 8th  & Body, Shield \\
        Bashing, Greater          & 2,800 gp   & 9th  & Shield       \\
        Spell Resistance          & 5,000 gp   & 10th & Armor        \\
        Feather \minus4           & 12,500 gp  & 12th & Body, Shield \\
        Invulnerability           & 30,000 gp  & 14th & Armor        \\
        Spell Reflecting          & 60,000 gp  & 16th & Shield       \\
        Feather \minus5           & 62,500 gp  & 16th & Body, Shield \\
        Invulnerability, Greater  & 140,000 gp & 18th & Armor        \\
    \end{dtabularx}
\end{dtable}

\begin{magicitemdef}{Bashing}[Passive]{1}[Shield]{Transmutation [Augment]}{as shield}
    \magicitemability{Passive} This shield deals damage with shield bash attacks as if it was two size categories larger than normal (so a Medium-sized light shield deals 1d6 damage, and a Medium-sized heavy shield deals 1d8 damage).
\end{magicitemdef}

\begin{magicitemdef}{Bashing, Greater}[2800]{3}[Shield]{Transmutation [Augment]}{as shield}
    \magicitemability{Passive} This shield functions like a \magicitem{bashing} shield.
    \magicitemability{Triggered}[Immediate action] When you successfully shield bash a foe, you can activate this item. If you do, you gain a \plus2 bonus to your physical defenses against that foe for 5 rounds.

    After you activate this ability, the shield is lighter and seems to move of its own volition to block attacks.
    This effect lasts for 5 rounds, during which time you cannot activate this ability again.
\end{magicitemdef}

\begin{magicitemdef}{Energy Resistance, Lesser}[Active]{1}[Body]{Abjuration [Shielding]}{as armor}
    \magicitemability{Triggered}[Immediate action] When you take \glossterm{energy damage}, you can activate this item. If you do, you reduce the damage by an amount equal to the item's power.

    After you activate this ability, the armor sheds light as a torch.
    The color of the light depends on the energy damage resisted: green for acid, blue for cold, yellow for electricity, and red for fire.
    The light lasts for 5 rounds, during which time you cannot activate this ability again.
\end{magicitemdef}

\begin{magicitemdef}{Energy Resistance}[Passive]{3}[Body]{Abjuration [Shielding]}{as armor}
    \magicitemability{Passive} You have damage reduction against \glossterm{energy damage} equal to the item's power.
    Whenever you resist energy with this item, it sheds light as a torch for 5 rounds.
    The color of the light depends on the energy damage resisted: green for acid, blue for cold, yellow for electricity, and red for fire.
\end{magicitemdef}

\begin{magicitemdef}{Feather}{1}[Body, Shield]{Transmutation (Augment)}{as armor}
    \magicitemability{Passive} This armor has a reduced armor check penalty.
    The price depends on the penalty reduction, as shown in the table below.

    Its base power is equal to three times the item's penalty reduction.
    To craft the item, you must have a number of ranks in the relevant Craft skill equal to the item's base power \add 4.
\end{magicitemdef}

\begin{dtable}
    \lcaption{Feather Armor}
    \begin{dtabularx}{\columnwidth} {l X X}
        \tb{Bonus} & \tb{Base Price} & \tb{Item Level} \\
        \hline
        \minus1    & 100 gp          & 2nd             \\
        \minus2    & 500 gp          & 4th             \\
        \minus3    & 2,500 gp        & 8th             \\
        \minus4    & 12,500 gp       & 12th            \\
        \minus5    & 62,500 gp       & 16th            \\
    \end{dtabularx}
\end{dtable}

%\mitemreq{Transmutation}{Augment}{1}{varies}{Craft (as armor) varies}


\begin{magicitemdef}{Flaming Burst}[Active]{3}[Body, Shield]{Evocation [Fire]}{as armor}
    \magicitemability{Triggered}[Immediate action] When you are hit or missed by a melee attack, you can activate this item.
    Body armor triggers if the attack hits, and shields trigger if the attack misses.
    If you activate the item, the attacking creature takes 1d8 fire damage per two item power.

    After you activate this ability, the item is wreathed in flame, causing it to shed light as a torch.
    This effect lasts for 5 rounds, during which time you cannot activate this ability again.
\end{magicitemdef}

\begin{magicitemdef}{Freezing Burst}[Active]{3}[Body, Shield]{Evocation [Cold]}{as armor}
    \magicitemability{Triggered}[Immediate action] When you are struck or missed by a melee attack, you can activate this item.
    Body armor triggers if the attack hits, and shields trigger if the attack misses.
    If you activate the item, the attacking creature takes 1d8 cold damage per two item power.

    After you activate this ability, the item radiates frigid cold, causing it to snuff out torches and other small fires within a 5 foot radius.
    This effect lasts for 5 rounds, during which time you cannot activate this ability again.
\end{magicitemdef}

\begin{magicitemdef}{Ghost Touch}[Passive]{2}[Body, Shield]{Conjuration [Teleportation]}{as armor}[This armor seems almost translucent.]
    \magicitemability{Passive} You apply the full bonus granted by this armor against the attacks of incorporeal creatures.
    It can be picked up, moved, and worn by incorporeal creatures at any time.
    Incorporeal creatures gain this armor's bonus against both corporeal and incorporeal attacks, and they can still pass freely through solid objects.
\end{magicitemdef}

\begin{magicitemdef}{Glamered}[Active]{2}[Body]{Illusion [Glamer]}{as armor}
    \magicitemability{Active}[Standard action] If you trace the symbol of a mask on your chest, this armor appears to change shape and form to assume the appearance of a normal set of clothing.
    You may choose the design of the clothing.
    The armor retains all its properties (including weight and sound) when glamered.

    The armor remains disguised until you trace the symbol of the mask in the reverse direction, at which point it regains its normal appearance.
\end{magicitemdef}

\begin{magicitemdef}{Invulnerability, Lesser}[Active]{2}[Body]{Abjuration [Shielding]}{as armor}
    \magicitemability{Active}[Standard action] If you strike your chest with a weapon or other hard object, you gain physical damage reduction equal to the item's power for 5 rounds.
    This allows you to ignore the first points of physical damage you take each round.
    Adamantine weapons ignore this damage reduction and negate it for 1 round.
\end{magicitemdef}

\begin{magicitemdef}{Invulnerability}[Passive]{4}[Body]{Abjuration [Shielding]}{as armor}
    \magicitemability{Passive} You have physical damage reduction equal to the item's power.
    This allows you to ignore the first points of physical damage you take each round.
    Adamantine weapons ignore this damage reduction and negate it for 1 round.
\end{magicitemdef}

\begin{magicitemdef}{Invulnerability, Greater}[Passive]{6}[Body]{Abjuration [Shielding]}{as armor}
    \magicitemability{Passive} You have damage reduction against all attacks equal to the item's power.
    This allows you to ignore the first points of damage you take each round.
\end{magicitemdef}

\begin{magicitemdef}{Shocking Burst}[Active]{3}[Body, Shield]{Evocation [Electricity]}{as armor}
    \magicitemability{Triggered}[Immediate action] When you are struck or missed by a melee attack, you can activate this item.
    Body armor triggers if the attack hits, and shields trigger if the attack misses.
    If you activate the item, the attacking creature takes 1d8 electricity damage per two item power.

    After you activate this ability, the item crackles with electrical energy, causing you to radiate flashing light as a torch.
    This effect lasts for 5 rounds, during which time you cannot activate this ability again.
\end{magicitemdef}

\begin{magicitemdef}{Spell Reflecting}[Active]{7}[Shield]{Abjuration [Shielding]}{as armor}[This shield's surface is completely reflective, allowing it to act as a mirror.]
    \magicitemability{Triggered}[Immediate action] When you are targeted by a spell or spell-like ability, you can activate this shield to reflect the spell back at its caster exactly like the \spell{spell turning} spell.

    After you activate this ability, the shield's surface becomes dully metallic instead of reflective.
    This effect lasts for 5 rounds, during which time you cannot activate this ability again.
\end{magicitemdef}

\begin{magicitemdef}{Spell Resistance}[Active]{3}[Body]{Abjuration [Shielding]}{as armor}
    \magicitemability{Active}[Standard action] If you crouch low and strike the ground with your fist, you gain spell resistance equal to 10 \add the item's power.
    This spell resistance lasts as long as you remain crouching, and for 5 rounds thereafter (maximum 5 minutes).
    You can move at half speed while crouching.

    To affect a creature with spell resistance using a spell, a caster must make an attack with an accuracy equal to its spellpower.
    If the attack beats the creature's spell resistance, the spell works normally.
    Otherwise, the spell has no effect on the creature.
\end{magicitemdef}

\subsection{Arms}

\begin{magicitemdef}{Bracers of Archery}[Passive]{2}[Arms]{Transmutation (Augment)}{leather or metal}
    \magicitemability{Passive} You are proficient with bows and crossbows.
\end{magicitemdef}

\begin{magicitemdef}{Bracers of Armor}[Passive]{1}[Arms]{Abjuration [Shielding]}{leather or metal}
    \magicitemability{Passive} You gain a \plus2 bonus to Armor defense.
    The protection from these bracers is treated as body armor, and does not stack with any other body armor you wear.
    Since this armor is made of magical force, incorporeal creatures can't bypass it the way they do normal armor.
\end{magicitemdef}

\begin{magicitemdef}{Bracers of Repulsion}[Active]{2}[Arms]{Evocation [Telekinesis]}{leather or metal}
    \magicitemability{Triggered}[Immediate action] When a foe damages you with a melee attack, you can activate this item.
    If you do, you can make a shove attack against the attacking creature at the end of the round.
    Your accuracy is equal to the item's power \add the damage its attack dealt to you.

    After you activate this item, barely visible fields of telekinetic energy surround the bracers.
    This effect lasts for 5 rounds, during which time you cannot activate this ability again.
\end{magicitemdef}

\begin{magicitemdef}{Gauntlets of Strength}[Passive]{2}[Arms]{Transmutation [Augment]}{bone or metal}
    \magicitemability{Passive} You gain a \plus2 bonus to Strength, up to a maximum Strength equal to the item's power \add 2.
\end{magicitemdef}

\begin{magicitemdef}{Gauntlets of Strength, Greater}[Passive]{4}[Arms]{Transmutation [Augment]}{bone or metal}
    \magicitemability{Passive} You gain a \plus4 bonus to Strength, up to a maximum Strength equal to the item's power \add 4.
\end{magicitemdef}

\begin{magicitemdef}{Gauntlet of the Ram}[Active]{2}[Arms]{Evocation [Telekinesis]}{bone or metal}
    \magicitemability{Triggered}[Immediate action] When you make a successful unarmed attack with this gauntlet against a foe, you can activate it to immediately make a shove attack against the struck foe.
    You gain a bonus on the shove attack equal to the damage you dealt.
    In addition, you do not need to move with the foe to push it backwards.

    After you activate this ability, the gauntlet grows small ram horns.
    This effect lasts for 5 rounds, during which time you cannot activate this ability again.
\end{magicitemdef}

\begin{magicitemdef}{Gloves of Dexterity}[Passive]{2}[Arms]{Transmutation [Augment]}{leather or textiles}
    \magicitemability{Passive} You gain a \plus2 bonus to Dexterity, up to a maximum Dexterity equal to the item's power \add 2.
\end{magicitemdef}

\begin{magicitemdef}{Gloves of Dexterity, Greater}[Passive]{4}[Arms]{Transmutation [Augment]}{leather or textiles}
    \magicitemability{Passive} You gain a \plus4 bonus to Dexterity, up to a maximum Dexterity equal to the item's power \add 4.
\end{magicitemdef}

\begin{magicitemdef}{Gloves of the Flame}[Active]{1}[Arms]{Evocation [Fire]}{leather or textiles}
    \magicitemability{Active}[Standard action] By flicking your fingers, you can create fire in your hand.
    You can immediately throw this fire at a creature or object within \rngclose range.
    If you do, you make an attack with an accuracy equal to the item's power against the target's Reflex defense.
    If the attack succeeds, the target takes 1d10 fire damage per two item power.
    A failed attack deals half damage.
\end{magicitemdef}

\begin{magicitemdef}{Greatreach Bracers}[Active]{1}[Arms]{Transmutation [Augment}{bone or metal}
    \magicitemability{Active}[Swift action] You can activate this item to increase your \glossterm{reach} by 5 feet for 1 round.

    After you activate this ability, the bracers visually grow in size, though not in weight or encumbrance.
    This effect lasts for 5 rounds, during which time you cannot activate this ability again.
\end{magicitemdef}

\begin{magicitemdef}{Puppeteer's Glove}[Passive]{2}[Arms]{Illusion [Figment]}{leather or textiles}
    \magicitemability{Triggered}[Immediate action] When you cast a Figment spell or use an Figment spell-like ability, you can activate this glove.
    If you do, you can concentrate on the spell or spell-like ability as a swift action by controlling the figment with your glove.
    If you are unable to control the figment with the glove, such as if you use the gloved hand for any other purpose, you lose your concentration on the figment.
    You must retain line of sight and line of effect to the figment to control it.

    After you activate this ability, the tips of the glove grow faintly visible strings.
    The strings extend four inches from the glove before vanishing into nothingness.
    This effect lasts for five rounds, during which time you cannot activate this ability again.
\end{magicitemdef}

\begin{magicitemdef}{Throwing Gloves}[Passive]{2}[Arms]{Transmutation [Augment]}{leather or textiles}
    \magicitemability{Passive} You can throw any item as if it was designed to be thrown, granting you a \plus4 bonus to accuracy.
    This does not improve your ability to throw items already designed to be thrown, such as darts.
\end{magicitemdef}

\subsection{Head}

\begin{magicitemdef}{Circlet of Perception}[Passive]{2}[Head]{Transmutation [Augment]}{metal}
    \magicitemability{Passive} You gain a \plus2 bonus to Perception, up to a maximum Perception equal to the item's power \add 2.
\end{magicitemdef}

\begin{magicitemdef}{Circlet of Perception, Greater}[Passive]{4}[Head]{Transmutation [Augment]}{metal}
    \magicitemability{Passive} You gain a \plus4 bonus to Perception, up to a maximum Perception equal to the item's power \add 4.
\end{magicitemdef}

\begin{magicitemdef}{Headband of Intellect}[Passive]{2}[Head]{Transmutation [Augment]}{metal}
    \magicitemability{Passive} You gain a \plus2 bonus to Intelligence, up to a maximum Intelligence equal to the item's power \add 2.
\end{magicitemdef}

\begin{magicitemdef}{Hat of Disguise}[Active]{2}[Head]{Illusion [Glamer]}{textiles}
    \magicitemability{Active}[Standard action] If you spin this hat on your head once, you can change your appearance for 1 hour.
    This functions as the \spell{disguise self} ritual.
\end{magicitemdef}

\begin{magicitemdef}{Headband of Intellect, Greater}[Passive]{4}[Head]{Transmutation [Augment]}{metal}
    \magicitemability{Passive} You gain a \plus4 bonus to Intelligence, up to a maximum Intelligence equal to the item's power \add 4.
\end{magicitemdef}

\begin{magicitemdef}{Mask of Water Breathing}[Active]{2}[Head]{Transmutation [Augment]}{textiles}
    \magicitemability{Active}[Standard action] If you speak a command word while wearing this mask, you gain the ability to breathe water for 1 hour.
    This does not prevent you from breathing air, and does not grant you the ability to breathe other liquids.
\end{magicitemdef}

\subsection{Legs}

\begin{magicitemdef}{Boots of Earth's Embrace}[Active]{3}[legs]{Transmutation [Augment]}{leather or textiles}
    \magicitemability{Triggered}[Immediate action] When you are affected by a forced movement effect (such as a shove attack) while standing on solid ground, you can activate these boots.
    If you do, you are not subject to the forced movement.
    This does not prevent any other effects from the attack, such as damage.

    After you activate this ability, the boots appear to be made from solid rock.
    This effect lasts for 5 rounds, during which time you cannot activate this ability again.
\end{magicitemdef}

\begin{magicitemdef}{Boots of Elvenkind}[500]{1}[Legs]{Transmutation [Augment]}{leather or textiles}
    \magicitemability{Passive} You gain a \plus4 bonus to Stealth checks.
\end{magicitemdef}

\begin{magicitemdef}{Boots of Levitation}[Active]{3}[Legs]{Evocation [Telekinesis]}{leather or textiles}
    \magicitemability{Active}[Standard action] By lifting and planting one leg in mid-air, as if climbing an invisible stair, you can gain the benefit of the \spell{levitate} spell on yourself for 5 rounds.
\end{magicitemdef}

\begin{magicitemdef}{Boots of Mobility}[1000]{1}[Legs]{Transmutation [Augment]}{leather or textiles}
    \magicitemability{Passive} You gain a \plus4 bonus to Acrobatics and Athletics checks.
\end{magicitemdef}

\begin{magicitemdef}{Boots of Speed}[Active]{4}[Legs]{Transmutation [Temporal]}{leather or textiles}
    \magicitemability{Active}[Standard action] If you stomp your foot on the ground three times, you gain the effects of the \spell{haste} spell.
    This benefit lasts as long as you move continuously without taking any other action, and for 5 rounds thereafter (maximum 5 minutes).
\end{magicitemdef}

\begin{magicitemdef}{Boots of Striding and Springing}[Passive]{2}[Legs]{Transmutation [Augment], Transmutation [Augment]}{leather or textiles}
    \magicitemability{Passive} You gain a \plus10 foot bonus to your land speed.
    A high land speed increases your ability to jump, as described in \pcref{Jump Modifiers}.
\end{magicitemdef}

\begin{magicitemdef}{Boots of Swift Passage}[Active]{3}[Legs]{Conjuration [Teleportation]}{leather or textiles}
    \magicitemability{Active}[Move action] If you take three rapid steps in the shape of a triangle, you can teleport up to 30 feet to a location you can see, as the \spell{dimension slide} spell.
\end{magicitemdef}

\begin{magicitemdef}{Boots of Teleportation}[Active]{5}[Legs]{Conjuration [Teleportation]}{leather or textiles}
    \magicitemability{Active}[Standard action] If you click your heels together three times, you can teleport up to 1,000 feet to a location you can specify, as the \spell{dimension door} spell.
\end{magicitemdef}

% teleport, +2 level for minute -> standard
\begin{magicitemdef}{Boots of Teleportation, Greater}[Active]{8}[Legs]{Conjuration [Teleportation]}{leather or textiles}
    \magicitemability{Active}[Standard action] If you click your heels together three times, you can teleport up to 100 miles to a location you can specify, as the \spell{teleport} ritual.
\end{magicitemdef}

%800 for endure elements, *3/4 for warming only, +800 for snow+ice travelling
%(like 1st level ritual?)
\begin{magicitemdef}{Boots of the Winterlands}[1400]{2}[Legs]{Evocation/Transmutation [Augment]}{leather or textiles}
    \magicitemability{Passive} You can travel across snow and ice without slipping or suffering movement penalties for the terrain.
    In addition, the boots keep you warm, protecting you in conditions as cold as \minus50 Fahrenheit.
\end{magicitemdef}

\begin{magicitemdef}{Sandals of Sprinting}[500]{1}[Legs]{Transmutation [Augment]}{leather or textiles}
    \magicitemability{Passive} You gain a \plus4 bonus to Sprint checks.
\end{magicitemdef}

\begin{magicitemdef}{Winged Boots}[Active]{4}[Legs]{Transmutation [Augment]}{leather or textiles}
    \magicitemability{Active}[Standard action] If you tap your boots together in midair, you can cause wings to sprout from them.
    If you are unencumbered, you can fly with a 30 foot speed and average maneuverability, as the \spell{fly} spell.
    See \pcref{Flying}, for more details.
    This effect lasts for 5 rounds.
\end{magicitemdef}

\subsection{Rings}

\parhead{Physical Description} Rings have no appreciable weight.
Although exceptions exist that are crafted from glass or bone, the vast majority of rings are forged from metal - usually precious metals such as gold, silver, and platinum.
A typical ring has AD 13, 10 hit points, hardness 10, and a break DC of 25.

\begin{comment}
\begin{dtable}
    \lcaption{Rings}
    \begin{dtabularx}{\columnwidth}{>{\lcol}X l}
        Ring & Market Price \\
        \hline
        Protection \plus1 & 2,000 gp \\
        Feather falling & 2,200 gp \\
        Climbing & 2,500 gp \\
        Jumping & 2,500 gp \\
        Sustenance & 2,500 gp \\
        Swimming & 2,500 gp \\
        Mind shielding & 8,000 gp \\
        Protection \plus2 & 8,000 gp \\
        Climbing, improved & 10,000 gp \\
        Jumping, improved & 10,000 gp \\
        Swimming, improved & 10,000 gp \\
        Energy resistance, minor & 12,000 gp \\
        Protection \plus3 & 18,000 gp \\
        Energy resistance, major & 28,000 gp \\
        Protection \plus4 & 32,000 gp \\
        Energy resistance, greater & 44,000 gp \\
        Protection \plus5 & 50,000 gp \\
    \end{dtabularx}
\end{dtable}
\end{comment}

\begin{magicitemdef}{Energy Resistance, Lesser}[Active]{1}[Ring]{Abjuration [Shielding]}{bone, metal, or jewelry}
    \magicitemability{Triggered}[Immediate action] When you take \glossterm{energy damage}, you can activate this item. If you do, you reduce the damage by an amount equal to the item's power.

    After you activate this ability, the ring sheds light as a torch.
    The color of the light depends on the energy damage resisted: green for acid, blue for cold, yellow for electricity, and red for fire.
    The light lasts for 5 rounds, during which time you cannot activate this ability again.
\end{magicitemdef}

\begin{magicitemdef}{Energy Resistance}[Passive]{3}[Ring]{Abjuration [Shielding]}{bone, metal, or jewelry}
    \magicitemability{Passive} You have damage reduction against \glossterm{energy damage} equal to the item's power.
    Whenever you resist energy with this item, it sheds light as a torch for 5 rounds.
    The color of the light depends on the energy damage resisted: green for acid, blue for cold, yellow for electricity, and red for fire.
\end{magicitemdef}

\begin{magicitemdef}{Heroic Vengeance}[Active]{3}[Ring]{Abjuration [Retributive]}{bone, metal, or jewelry}
    \magicitemability{Triggered}[Immediate action] When a foe within \rngmed range rolls a natural 20 on an attack against you, you can activate this item. If you do, the attacking creature takes 1d8 divine damage per item power.
\end{magicitemdef}

\begin{magicitemdef}{Sustenance, Lesser}[Passive]{1}[Ring]{Conjuration/Transmutation [Creation, Temporal]}{bone, metal, or jewelry}
    \magicitemability{Passive} You continuously gain nourishment, and no longer need to eat or drink.
    The ring must be worn for 24 hours before it begins to work.
\end{magicitemdef}

\begin{magicitemdef}{Sustenance}[Passive]{3}[Ring]{Conjuration/Transmutation [Creation, Temporal]}{bone, metal, or jewelry}
    \magicitemability{Passive} You continuously gain nourishment, and no longer need to eat or drink.
    In addition, you need only one-quarter your normal amount of sleep (or similar activity, such as elven trance) each day.

    The ring must be worn for 24 hours before it begins to work.
\end{magicitemdef}

\begin{magicitemdef}{Protection}{1}[Ring]{Abjuration [Shielding]}{bone, metal, or jewelry}
    \magicitemability{Passive} This ring has an enhancement bonus to improve your defenses.
    Each \plus1 of enhancement bonus grants temporary hit points equal the item's power, and grants you an additional defensive legend point each day.
    If you stop using the ring, you lose the temporary hit points and the legend points.
    The benefits of this ring function in the same way as enhancement bonuses on magic armor, and they not stack with those bonuses.

    These bonuses can only be gained once per day, regardless of the number of items you use.
    If you wear two rings, or change between different rings, use only the highest bonus that applies.
    If you change from a weaker magical item to a stronger magical item, you gain bonuses equal to the difference between the two items.

    \spellspecial The price of the ring depends on its enhancement bonus, as shown in the table below.
    Its base power is equal to three times its enhancement bonus.
    To craft the item, you must have a number of ranks in Craft (bone, metal, or jewelry) equal to the item's base power \add 4.
\end{magicitemdef}

\begin{dtable}
    \lcaption{Ring of Protection}
    \begin{dtabularx}{\columnwidth} {l X X}
        \tb{Bonus} & \tb{Base Price} & \tb{Item Level} \\
        \hline
        \plus1 & 100 gp    & 2nd  \\
        \plus2 & 500 gp    & 4th  \\
        \plus3 & 2,500 gp  & 8th  \\
        \plus4 & 12,500 gp & 12th \\
        \plus5 & 62,500 gp & 16th \\
    \end{dtabularx}
\end{dtable}

\subsection{Torso}

\begin{magicitemdef}{Amulet of Mighty Fists}{1}[Torso]{Transmutation [Augment]}{bone or jewelry}
    \magicitemability{Passive} This amulet has an enhancement bonus to improve your natural attacks and unarmed strikes.
    You gain a bonus to damage on natural attacks and unarmed strikes equal to this amulet's enhancement bonus.
    In addition, each \plus1 of enhancement bonus grants you an additional offensive legend point each day.
    If you stop using the amulet, you lose the damage bonus and legend points.
    The benefits of this ring function in the same way as enhancement bonuses on magic weapons, and they not stack with those bonuses.

    These legend points can only be gained once per day, regardless of the number of items you use.
    If you both wear an amulet and wield a magic weapon, or change between different amulets, use only the highest bonus that applies.
    If you change from a weaker magical item to a stronger magical item, you gain bonuses equal to the difference between the two items.

    \spellspecial The price of the amulet depends on its enhancement bonus, as shown in the table below.
    Its base power is equal to three times its enhancement bonus.
    To craft the item, you must have a number of ranks in Craft (jewelry) equal to the item's base power \add 4.
\end{magicitemdef}

\begin{dtable}
    \lcaption{Amulet of Mighty Fists}
    \begin{dtabularx}{\columnwidth}{l X X}
        \tb{Bonus} & \tb{Base Price} & \tb{Item Level} \\
        \hline
        \plus1 & 200 gp & 3rd \\
        \plus2 & 1000 gp & 6th \\
        \plus3 & 5,000 gp & 10th \\
        \plus4 & 25,000 gp & 13th \\
        \plus5 & 125,000 gp & 17th \\
    \end{dtabularx}
\end{dtable}

\begin{magicitemdef}{Amulet of the Planes}[Active]{6}[Torso]{Conjuration [Teleportation, Planar]}{bone, jewelry}
    By holding this amulet in one hand and concentrating on a specific plane for 1 minute, you can create the effects of a \ritual{plane shift} ritual.
    Activating the item successfully requires a DC 15 Knowledge (planes) check.
    If you fail, your activation of the amulet has no effect (but still consumes an item use).
    If you roll a 1 and fail, you and any creatures with you are transported to a random plane.
    Each time you successfully activate this item in the same day, the DC of the check increases by 5.
\end{magicitemdef}

\begin{magicitemdef}{Amulet of Nondetection}[Passive]{3}[Torso]{Abjuration [Shielding]}{bone, jewelry}
    \magicitemability{Passive} You gain the benefits of the \spell{nondetection} ritual.
    If a divination is attempted against you, the caster must make a spellpower check against a DC equal to 15 \add the item's power.
\end{magicitemdef}

\begin{magicitemdef}{Belt of Constitution}[Passive]{2}[Torso]{Transmutation [Augment]}{leather or textiles}
    \magicitemability{Passive} You gain a \plus2 bonus to Constitution, up to a maximum Constitution equal to the item's power \add 2.
\end{magicitemdef}

\begin{magicitemdef}{Belt of Constitution, Greater}[Passive]{4}[Torso]{Transmutation [Augment]}{leather or textiles}
    \magicitemability{Passive} You gain a \plus4 bonus to Constitution, up to a maximum Constitution equal to the item's power \add 4.
\end{magicitemdef}

\begin{magicitemdef}{Belt of Dwarvenkind}[Passive]{3}[Torso]{Divination/Transmutation [Augment]}{leather or textiles}
    \magicitemability{Passive} You gain a \plus2 bonus to Constitution, up to a maximum Constitution equal to the item's power \add 2.
    In addition, you gain dwarven characteristics.
    You gain a \plus4 bonus to social checks when dealing with dwarves, but take a \minus2 penalty with all other creatures.
    You also gain the benefits of the Stonecunning feat, regardless of whether you meet the prerequisites.
\end{magicitemdef}

\begin{magicitemdef}{Healing Belt}[Active]{1}[Torso]{Vivimancy [Positive]}{leather or textiles}
    \magicitemability{Active}[Standard action] If you grab this belt in one hand and touch a creature with the other, the touched creature is healed for 1d10 damage per two item power.
    If you heal yourself, you only need one hand free to grab the belt.
\end{magicitemdef}

\begin{magicitemdef}{Heroic Recovery}[Active]{1}[Torso]{Abjuration/Vivimancy [Positive]}{leather or textiles}
    \magicitemability{Triggered}[Immediate action] When you roll a natural 20 on an attack roll, you can activate this item. If you do, you heal 1d10 damage per two item power.
\end{magicitemdef}

\begin{magicitemdef}{Torc of Willpower}[Passive]{2}[Torso]{Transmutation [Augment]}{metal or jewelry}
    \magicitemability{Passive} You gain a \plus2 bonus to Willpower, up to a maximum Willpower equal to the item's power \add 2.
\end{magicitemdef}

\begin{magicitemdef}{Torc of Willpower, Greater}[Passive]{4}[Torso]{Transmutation [Augment]}{metal or jewelry}
    \magicitemability{Passive} You gain a \plus4 bonus to Willpower, up to a maximum Willpower equal to the item's power \add 4.
\end{magicitemdef}

% +1 level for extra thick fog in the first round
\begin{magicitemdef}{Vanishing Cloak}[Active]{2}[Torso]{Conjuration [Creation, Fog]}{textiles}
    \magicitemability{Active}[Standard action] If you wrap this cloak around yourself with one hand, you can create a cloud of fog centered on you, as the \spell{fog cloud} spell.
    The effect lasts for 5 rounds.
    For the first round of the effect, the fog is unusually thick, blocking all sight beyond 10 feet.
    This can allow you to hide unobserved.
\end{magicitemdef}

\begin{magicitemdef}{Vanishing Cloak, Greater}[Active]{4}[Torso]{Conjuration [Creation, Fog, Teleportation]}{textiles}
    \magicitemability{Active}[Standard action] If you wrap this cloak around yourself with one hand, you can teleport up to 100 feet away to a location you can see, as the \spell{dimension slide} spell.
    In addition, you create a cloud of fog centered on you, as the \spell{fog cloud} spell.
    You can choose whether the cloud appears at your original location, or at your location after teleporting.
    For the first round of the effect, the fog is unusually thick, blocking all sight beyond 10 feet.
    This can allow you to hide unobserved.
\end{magicitemdef}

\section{Implements}
Magical implements must be wielded to gain their effects.

\subsection{Weapons Overview}

Magic weapons improve a character's combat abilities.
All magic weapons have an enhancement bonus to improve your damage and ability to hit.
In addition to an enhancement bonus, magic weapons may have special abilities or be made of an unusual material.

\subsubsection{Weapon Enhancement Bonuses}\label{Weapon Enhancement Bonuses}

Magic weapons can have enhancement bonuses ranging from \plus1 to \plus5.
You gain a bonus to damage on physical attacks using a magic weapon equal to the weapon's enhancement bonus.
In addition, each \plus1 of enhancement bonus grants you an additional offensive legend point each day.

These legend points can only be gained once per day, regardless of the number of weapons you use.
If you use multiple weapons at once, or change between different weapons, use only the highest number of legend points that applies.
If you change from a weaker magical weapon to a stronger magical weapon, you gain legend points equal to the difference between the two enhancement bonuses.

\subsubsection{Weapon Prices}\label{Weapon Prices}
The prices of enhancement bonuses to weapons are listed in \trefnp{Magic Weapon Prices}, and the prices of special abilities are listed on \trefnp{Magic Weapon Special Abilities}.
If a weapon has a special ability, the price of the special ability is added to the price of the weapon.
The number of special abilities on the weapon cannot exceed the enhancement bonus of the weapon.
Additionally, the price of all special abilities cannot exceed twice the price of the enhancement bonus on the weapon.

\begin{dtable}
    \lcaption{Magic Weapon Prices}
    \begin{dtabularx}{\columnwidth} {>{\ccol}X c c}
        \tb{Enhancement Bonus} & \tb{Base Price} & \tb{Item Level} \\
        \hline
        \plus1 weapon          & 200 gp          & 3rd             \\
        \plus2 weapon          & 1,000 gp        & 6th             \\
        \plus3 weapon          & 5,000 gp        & 10th            \\
        \plus4 weapon          & 25,000 gp       & 13th            \\
        \plus5 weapon          & 125,000 gp      & 17th            \\
    \end{dtabularx}
\end{dtable}

\parhead{Base Power for Weapons} The base power of a magic weapon with a special ability is given in the item description.
For an item with only an enhancement bonus, the base power is three times the enhancement bonus.
If an item has both an enhancement bonus and a special ability, the higher of the two base powers is used.

\parhead{Hardness and Hit Points} Each \plus1 of enhancement bonus adds 2 to a weapon's hardness and \plus10 to its hit points.

\parhead{Ranged Weapons and Ammunition} The enhancement bonus from a ranged weapon does not stack with the enhancement bonus from ammunition.
Only the higher of the two enhancement bonuses applies.
Special abilities are applied from both sources, as long as they are not identical.
If conflicting special abilities exist, the special ability on the ammunition takes precedence.

Magic ammunition loses its magic after being fired, whether it hits or misses.

\parhead{Light Generation} Some magic weapons shed light equivalent to a light spell (bright light in a 20-foot radius, shadowy light in a 40-foot radius).
The light on such weapons cannot normally be shut off.
Some of the specific weapons detailed below always or never glow, as defined in their descriptions.

\parhead{Activation} Usually, a character benefits from a magic weapon in the same way a character benefits from a mundane weapon - by attacking with it.
Special abilities on weapons are usually activated if the character strikes a foe with the weapon.

\parhead{Magic Weapons and Critical Hits} Some weapon qualities and some specific weapons have an extra effect on a critical hit.
These special effects function against creatures not subject to critical hits.
When fighting against such creatures, roll for critical hits as you would against any other creature subject to critical hits.
On a successful critical roll, apply the special effect, but do not multiply the weapon's regular damage.

\subsection{Weapon Special Abilities}\label{Weapon Special Abilities}

\begin{dtable*}
    \lcaption{Magic Weapon Special Abilities}
    \begin{dtabularx}{\textwidth}{l >{\lcol}X l l}
        \tb{Special Ability} & \tb{Description} & \tb{Cost} & \tb{Item Level} \\
        \hline
        Bane               & Add special ability that only functions against certain creatures   & Special    & Special \\
        Morphing           & Weapon transforms into similar weapon                               & 200 gp     & 3rd     \\
        Entangling         & Entangle struck foe                                                 & 400 gp     & 4th     \\
        Flaming            & Ignite struck foe                                                   & 400 gp     & 4th     \\
        Forceful           & Knock back struck foe                                               & 400 gp     & 4th     \\
        Thundering         & Deafen struck foe and those nearby                                  & 400 gp     & 4th     \\
        Freezing           & Fatigue struck foe                                                  & 400 gp     & 4th     \\
        Surestrike, Lesser & Roll critical confirmation twice                                    & 400 gp     & 4th     \\
        Defending          & Trade accuracy and damage for AC                                      & 500 gp     & 4th     \\
        Cleaving           & Extra strike after dropping foe                                     & 800 gp     & 5th     \\
        Returning          & Weapon returns after being thrown                                   & 1,000 gp   & 6th     \\
        Poisoning          & Quickly coat weapon in duplicated poison                            & 1,600 gp   & 7th     \\
        Shocking           & Stagger struck foe                                                  & 1,600 gp   & 7th     \\
        Vampiric           & Lick weapon to regain hit points                                    & 1,600 gp   & 7th     \\
        Surestrike         & Reroll missed attacks                                               & 4,000 gp   & 9th     \\
        Thieving           & Absorb struck objects into weapon                                   & 4,000 gp   & 9th     \\
        Returning, Greater & Weapon returns immediately after being thrown                       & 5,000 gp   & 10th    \\
        Heartseeking       & Automatically score critical hit after striking target repeatedly   & 12,000 gp  & 12th    \\
        Poisoning, Greater & Quickly coat weapon in potent duplicated poison                     & 24,000 gp  & 13th    \\
        Soulreaving        & Weapon strikes the soul for delayed damage instead of normal damage & 60,000 gp  & 16th    \\
        Vorpal             & Sever foe's head in a single blow                                   & 140,000 gp & 18th    \\
    \end{dtabularx}
\end{dtable*}

\begin{magicitemdef}{Bane}{2}{Transmutation [Augment]}{as weapon}
    \spellspecial A bane weapon excels at attacking a specific type of creature.
    Any weapon special ability can be designated as a ``bane'' ability, causing it to only function against a specific kind of creature.
    In exchange, the ability costs half the normal price in raw materials to add to the weapon.
    A list of possible foes is described on the following table.
\end{magicitemdef}

\begin{dtable}
    \lcaption{Bane Creature Types}
    \begin{dtabularx}{\columnwidth}{>{\lcol}X >{\lcol}X}
        \tb{Designated Foe}     & \tb{Designated Foe}     \\
        \hline
        Aberrations             & Animals                 \\
        Constructs              & Dragons                 \\
        Elementals              & Fey                     \\
        Giants                  & Humanoids, civilized    \\
        Humanoids, savage       & Magical beasts          \\
        Monstrous humanoids     & Oozes                   \\
        Outsiders, inner planes & Outsiders, outer planes \\
        Plants                  & Undead                  \\
        Vermin                  &                         \\
    \end{dtabularx}
\end{dtable}

\begin{magicitemdef}{Cleaving}[Active]{1}[Melee]{Transmutation [Augment]}{as weapon}
    \magicitemability{Triggered}[Immediate action] When you make a melee attack with this weapon that knocks a creature unconscious or kills it, you can activate this weapon to take an extra strike, as the Cleave feat (\featpcref{Cleave}).
    If you already have the Cleave feat, activating this weapon instead grants a \plus5 bonus to accuracy and damage on your cleave attack.

    After you activate this ability, the weapon sheds blood red light as a torch.
    This effect lasts for 5 rounds, during which time you cannot activate this ability again.
\end{magicitemdef}

\begin{magicitemdef}{Defending}[Passive]{2}[Melee]{Abjuration [Shielding]}{as weapon}
    \magicitemability{Passive} You can use the legend points granted by this weapon's enhancement bonus as defensive legend points, in addition to using them as offensive legend points.
    This stacks with any defensive legend points granted by armor, but not with other offensive legend points granted by weapons.
\end{magicitemdef}

\begin{magicitemdef}{Disorienting}[Active]{4}[Melee, Ranged]{Enchantment [Compulsion]}{as weapon}
    \magicitemability{Triggered}[Immediate action] When you strike a foe with this weapon, you can activate it to make the struck creature \disoriented for 5 rounds.
    This is a Mind effect.

    After you activate this ability, the weapon cackles gleefully.
    It cackles again every time you strike a foe with it.
    This effect lasts for 5 rounds, during which time you cannot activate this ability again.
\end{magicitemdef}

\begin{magicitemdef}{Flaming}[Active]{2}[Melee, Ranged]{Evocation [Fire]}{as weapon}
    \magicitemability{Active}[Standard action] If you speak a command word, you can ignite this weapon in flames and make a single attack with it. If this attack hits, the target takes 1d8 fire damage per two item power in addition to the damage from your attack.

    If the attack misses, the weapon continues to flame for up to 5 rounds.
\end{magicitemdef}

\impdescription{Flaming}{400}{4th}{Faint Evocation (Energy) [Fire]}{2nd}{Immediate (triggered)} When you strike a foe with this weapon, you can engulf the struck creature in flames.
If you do, it is \ignited for 5 rounds.

When you activate this ability, the weapon is wreathed in flames, causing damage you deal with it to be treated as fire damage in addition to its other types.
This effect lasts for 5 rounds, during which time you cannot activate the item's abilities.

\mitemreqdesc{Evocation}{Energy}{Fire}{1}{2nd}{Craft (as weapon) 6}

\impdescription{Freezing}{400}{4th}{Faint Evocation (Energy) [Cold]}{2nd}{Immediate (triggered)} When you strike a foe with this weapon, you can unleash an icy blast from the weapon.
If you do, your foe is \fatigued for 5 rounds.

When you activate this ability, the weapon radiates chilling cold, causing damage you deal with it to be treated as cold damage in addition to its other types.
This effect lasts for 5 rounds, during which time you cannot activate the item's abilities.

\mitemreqdesc{Evocation}{Energy}{Cold}{1}{2nd}{Craft (as weapon) 6}

\impdescription{Forceful}{400}{4th}{Faint Evocation (Control)}{4th}{Immediate (triggered)} When you strike a foe with this weapon, you can activate the weapon to immediately make a shove attempt with a circumstance bonus equal to the damage you dealt with the attack.
You do not have to move with your foe to knock it back the full distance.

After you activate this ability, the weapon feels heavier in your hands.
This effect lasts for 5 rounds, during which time you cannot activate the item's abilities.

\mitemreq{Evocation}{Control}{1}{2nd}{Craft (as weapon) 6}

%As 5th level spell because it feels right.
Based losely on Discern
%Vulnerability, although this feels like Awareness, so True Seeing?
\impdescription{Heartseeking}{12,000}{12th}{Moderate Divination (Awareness)}{10th}{Immediate (triggered)} When you strike the same foe with this weapon for multiple rounds in a row, you can suddenly perceive a critical weakness in your foe's defenses.
You must strike the foe for a number of consecutive rounds equal to the critical multiplier of the weapon you are using.
If you activate the item, the final hit automatically becomes a confirmed critical hit.
This has no effect on creatures immune to critical hits.

When you activate this ability, you gain a \plus4 bonus to confirm critical hits.
This effect lasts for 5 rounds, during which time you cannot activate the item's abilities.

\mitemreq{Divination}{Awareness}{5}{10th}{Craft (as weapon) 14}

\impdescription{Morphing}{200}{6th}{Faint Transmutation (Alteration)}{4th}{Standard (specific action)} A morphing weapon can transform into any other weapon from its weapon group.
To transform a morphing weapon, you must grab it with both hands and strike it against your knee or other hard object, as if breaking it, while visualizing its new form (a standard action).
It remains transformed until you transform it again.

\mitemreq{Transmutation}{Alteration}{1}{2nd}{Craft (as weapon) 6}

%price as 2nd level spell?
\impdescription{Poisoning}{1,600}{7th}{Faint Conjuration/Transmutation (Creation, Temporal)}{4th}{Swift (specific action) and standard (specific action)} A poisoning weapon can conjure poisons to cover the striking surface of the weapon.
The poison must first be inserted into a small slot in the hilt of the weapon (a standard action).
Once a poison is present in the slot, you can coat the weapon with a dose of the poison by pressing a small button on the hilt (a swift action).
After a poison has been used, it takes 5 rounds for the weapon to create more poison, during which time the weapon cannot be activated.
Only liquid poisons worth 100 gp per dose or less can be duplicated in this way.

The poison within the weapon is kept fresh magically, decaying at a rate of one minute per day.
The weapon can be emptied by pressing a second button to open its slot and pouring the poison out (a standard action).
You can freely insert and remove poison from the weapon, but coating the weapon in poison costs an activation.

\mitemreq{Conjuration, Transmutation}{2}{4th}{Craft (as weapon) 8}

%price as 5th level spell? similar to 20 uses of a one-shot item worth 1000 gp, so sure
\impdescription{Poisoning, Greater}{24,000}{13th}{Moderate Conjuration/Transmutation (Creation, Temporal)}{10th}{Swift (specific action) and standard (specific action)} This ability functions like the \magicitem{poisoning} weapon ability, except that it can duplicate liquid poisons of up to 1,000 gp per dose.
In addition, up to five different poisons can be stored within the weapon.
When you coat the weapon with poison, you may choose which poison to use.

\mitemreq{Conjuration, Transmutation}{5}{10th}{Craft (as weapon) 14}

\impdescription{Returning}{1,000}{6th}{Faint Conjuration (Translocation) [Teleportation]}{4th}{\x} After being thrown or fired, a returning weapon teleports back to the creature that threw or fired it.
It returns to the thrower just before the creature's next turn (and is therefore ready to use again in that turn).

Catching a returning weapon when it comes back is a free action.
If you can't catch it, the weapon drops to the ground in the square from which it was thrown.

\mitemreqdesc{Conjuration}{Translocation}{Teleportation}{2}{4th}{Craft (as weapon) 8}

\impdescription{Returning, Greater}{5,000}{10th}{Faint Conjuration (Translocation) [Teleportation]}{6th}{\x} This ability functions like the \magicitem{returning} ability, except that the weapon teleports back to the creature that threw or fired it immediately after the attack is resolved, allowing the creature to make multiple attacks in the same round with the same thrown weapon.

\mitemreqdesc{Conjuration}{Translocation}{Teleportation}{3}{6th}{Craft (as weapon) 10}

%As 7th level spell? I have no idea.
There aren't actually any spells this can be based on.
\impdescription{Soulreaving}{60,000}{16th}{Strong Necromancy (Soul)}{16th}{\x and standard (specific action)} This ghostly, translucent weapon strikes directly at the target's soul.
It ignores all damage reduction, but it does not deal hit point damage.
In fact, a creature struck by the weapon does not feel the attack at all.
Damage that would be dealt by the weapon can be delayed indefinitely.
While the damage is delayed, it cannot be cured or dispelled.

In order to convert the delayed damage into real damage, the wielder must stab themselves through the heart with the weapon as a standard action.
This deals no damage to the wielder, but any creatures that have been dealt damage by the weapon immediately take lethal damage equal to the delayed damage the weapon has stored up for them.
Any such damage dealt in excess of the creature's hit points is converted directly into critical damage.

A soulreaver weapon has no effect on objects or constructs.
While wielded, it has physical form only for its wielder, making it impossible to disarm.
While not in use, it can be picked up and touched normally.

\mitemreq{Necromancy}{Soul}{7}{14th}{Craft (as weapon) 18}

\impdescriptiondc{Shocking}{1,600}{7th}{Faint Evocation (Energy) [Electricity]}{4th}{Immediate (triggered)}{Fortitude (level \add 2)} When you strike a foe with this weapon, you can unleash an powerful electrical jolt from the weapon.
If you do, make a Fortitude attack.
If you succeed, your foe is \staggered for 5 rounds.

When you activate this ability, the weapon crackles with electrical energy, causing damage you deal with it to be treated as electrical damage in addition to its other types.
This effect lasts for 5 rounds, during which time you cannot activate the item's abilities.

\mitemreqdesc{Evocation}{Energy}{Electricity}{2}{4th}{Craft (as weapon) 8}

\impdescription{Surestrike, Lesser}{400}{4th}{Faint Divination (Knowledge)}{2nd}{Immediate (triggered)} When you threaten a critical hit with this weapon, you can activate it to receive a brief glimpse of the future, showing you how to wound your foe deeply.
If you do, you may roll the threat confirmation twice and take whichever roll you prefer.

After you activate this ability, you see shadowy glimpses of alternate futures out of the corner of your eyes.
This effect lasts for 5 rounds, during which time you cannot activate the item's abilities.

\mitemreq{Divination}{Knowledge}{1}{2nd}{Craft (as weapon) 6}

\impdescription{Surestrike}{4,000}{9th}{Faint Divination (Knowledge)}{8th}{Immediate (triggered)} When you miss an attack with this weapon, you can activate it to reroll the attack roll.
You must take the second result.

When you activate this ability, you see shadowy glimpses of alternate futures superimposed over objects and creatures you see.
This effect lasts for 5 rounds, during which time you cannot activate the item's abilities.

\mitemreq{Divination}{Knowledge}{3}{6th}{Craft (as weapon) 10}

\impdescription{Thieving}{4,000}{9th}{Faint Transmutation (Alteration)}{4th}{Immediate (trigger) and standard (specific action)} When you strike an object with this weapon, if the object is at least one size category smaller than the weapon, you may activate the weapon.
If you do, the object is absorbed into the weapon, leaving no trace that it ever existed.
Striking an attended object requires a successful disarm attempt.

You can retrieve objects from the weapon by running your hand along the length of the striking surface of the weapon (a standard action).
If you do, the last item absorbed by the weapon appears in your hand.
You may freely retrieve objects from within the weapon, but absorbing objects costs an activation.

The weapon can hold no more than three objects at once.
If you attempt to absorb an object while the weapon is full, the attempt fails.

\mitemreq{Transmutation}{Alteration}{2}{4th}{Craft (as weapon) 8}

\impdescriptiondc{Thundering}{400}{4th}{Faint Evocation (Energy) [Sonic]}{2nd}{Immediate (triggered)}{Fortitude (level \add 1)} When you strike a foe with this weapon in melee, you can detonate a deafening roll of thunder.
If you do, make a Fortitude attack against the struck foe and all other creatures within a \areasmall radius of you.
A successful attack deafens a creature for 5 rounds.
You are immune to the deafening effect.

After you activate this ability, the weapon emits non-damaging thunderous echoes whenever it strikes a solid object or creature.
This effect lasts for 5 rounds, during which time you cannot activate the item's abilities.

\mitemreqdesc{Evocation}{Energy}{Sonic}{4}{8th}{Craft (as weapon) 8}

\impdescription{Vampiric}{1,600}{7th}{Faint Necromancy (Life)}{4th}{Move (specific action)} If you lick the striking part of this weapon (a move action), you regain hit points equal to the damage dealt by the weapon on its last successful attack.
If the weapon has not dealt damage in the past round, you regain no hit points.

\mitemreq{Necromancy}{Life}{2}{4th}{Craft (as weapon) 8}

%8th level - 9th level for Power Word Kill, down to 8th because the trigger is
%so hard to pull off
\impdescription{Vorpal}{140,000}{18th}{Strong Transmutation (Augment)}{18th}{Immediate (triggered)} If you roll a 20 with this weapon and confirm the critical hit, you can instantly decapitate your foe.
If you do, it dies immediately.
This has no effect on creatures without a discernable head, creatures unaffected by the loss of a single head, or creatures whose head you cannot reach.

\mitemreq{Transmutation}{Augment}{9}{18th}{Craft (as weapon) 22}

\begin{comment}
\subsection{Rods}

Rods are scepterlike devices that have unique magical powers and do not usually have charges.
Anyone can use a rod.

\parhead{Physical Description} Rods weigh approximately 5 pounds.

They range from 2 feet to 3 feet long and are usually made of iron or some other metal.
(Many, as noted in their descriptions, can function as light maces or clubs due to their sturdy construction.)

These sturdy items have AC 9, 10 hit points, hardness 10, and a break DC of 27.

\parhead{Activation} Details relating to rod use vary from item to item.
See the individual descriptions for specifics.
\end{comment}

\subsection{Staffs}

A staff is a long shaft, usually made of wood, that enhances a spellcaster's power.
All magic staffs have an enhancement bonus to improve your spell damage and ability to hit.
In addition to an enhancement bonus, magic staffs may have special abilities or be made of an unusual material.

\subsubsection{Staff Enhancement Bonuses}\label{Staff Enhancement Bonuses}

Magic staffs can have enhancement bonuses ranging from \plus1 to \plus5.
You gain a bonus to damage with damaging spells equal to the staff's enhancement bonus.
In addition, each \plus1 of enhancement bonus grants you an additional offensive legend point each day.

These legend points can only be gained once per day, regardless of the number of staffs you use.
If you use multiple staffs at once, or change between different staffs, use only the highest number of legend points that applies.
If you change from a weaker magical staff to a stronger magical staff, you gain legend points equal to the difference between the two enhancement bonuses.


\parhead{Staff Prices} Enhancement bonuses on staffs are three times as expensive as wands, but staffs otherwise use the same pricing rules as wands.

\begin{dtable}
    \caption{Staff Prices}
    \begin{dtabularx}{\columnwidth} {>{\ccol}X c c}
        \tb{Enhancement Bonus} & \tb{Base Price} & \tb{Item Level}\\
        \hline
        \plus1 staff & 150 gp    & 3rd  \\
        \plus2 staff & 750 gp    & 5th  \\
        \plus3 staff & 3,750 gp  & 9th  \\
        \plus4 staff & 18,750 gp & 13th \\
        \plus5 staff & 93,750 gp & 17th \\
    \end{dtabularx}
\end{dtable}

\parhead{Physical Description} A typical staff is 4 feet to 7 feet long and 2 inches to 3 inches thick, weighing about 5 pounds.
Most staffs are wood, but a rare few are bone, metal, or even glass.
(These are extremely exotic.) Staffs often have a gem or some device at their tip or are shod in metal at one or both ends.
Staffs are often decorated with carvings or runes.
A typical staff is like a walking stick, quarterstaff, or cudgel.
It has AC 7, 10 hit points, hardness 5, and a break DC of 24.

\parhead{Activation} Staffs use the same activation method as wands.

\subsection{Holy Symbols}
A holy symbol is a small object that enhances a divine spellcaster's power.
Holy symbols function exactly like wands (see below), except that they enhance all schools of magic at once.

\parhead{Holy Symbol Prices} Enhancement bonuses on holy symbols are three times as expensive as wands, but holy symbols otherwise use the same pricing rules as wands.

\begin{dtable}
    \caption{Holy Symbol Prices}
    \begin{dtabularx}{\columnwidth} {>{\ccol}X c c}
        \tb{Enhancement Bonus} & \tb{Base Price} & \tb{Item Level}\\
        \hline
        \plus1 holy symbol & 150 gp & 3rd \\
        \plus2 holy symbol & 750 gp & 5th \\
        \plus3 holy symbol & 3,750 gp & 9th \\
        \plus4 holy symbol & 18,750 gp & 13th \\
        \plus5 holy symbol & 93,750 gp & 17th \\
    \end{dtabularx}
\end{dtable}

\parhead{Physical Description} A typical holy symbol is a no larger than 4 inches in each dimension and can be easily held in the palm of a hand.
Most holy symbols are metal, but they can be made from wood, bone, or even more exotic materials, depending on the deity they symbolize.

Many holy symbols are designed to be worn as an amulet in addition to being held in the hand.
When worn in this way, the holy symbol occupies a torso body slot.

\parhead{Activation} Holy symbols use the same activation method as wands.

\begin{dtable}
    \begin{dtabularx}{\columnwidth}{X l l}
        \tb{Special Ability} & \tb{Cost} & \tb{Item Level} \\
        \hline
        Channeling & 2,000 & 6th \\
        Channeling, Greater & 8,000 & 12th \\
    \end{dtabularx}
\end{dtable}

\impdescription{Channeling}{1,500}{7th}{Faint Evocation (Channeling)}{6th}{Immediate (triggered)} When you channel energy, you can activate this holy symbol to inflict or heal an extra d6 points of damage.
If you do not have the channel energy ability, this ability does not affect you.

\mitemreq{Evocation}{Channeling}{2}{4th}{Craft (as holy symbol) 8}

\impdescription{Channeling, Greater}{10,000}{11th}{Moderate Evocation (Channeling)}{12th}{Immediate (triggered)} This holy symbol functions like a \magicitem{channeling} holy symbol, except that it increases the damage by 2d6 instead.

\mitemreq{Evocation}{Channeling}{4}{8th}{Craft (as holy symbol) 12}

\subsection{Wands}

A wand is a thin baton that enhances a spellcaster's power.
Wands always provide an enhancement bonus to spellpower with a particular school of magic.
In addition to an enhancement bonus, wands may have special abilities or be made of an unusual material.

\parhead{Wand Prices} The prices of enhancement bonuses to wands are listed in \trefnp{Magic Wands}, and the prices of special abilities are listed on \trefnp{Wand Special Abilities}.
If a wand has a special ability, the price of the special ability is added to the price of the wand.

\parhead{Special Ability Limitations} The number of special abilities on the wand cannot exceed the enhancement bonus of the wand.
Additionally, the price of all special abilities cannot exceed twice the price of the enhancement bonus on the wand.

\subparhead{Multiple Schools} Some rare wands provide bonuses to two schools.
The enhancement bonus on a wand costs twice as much if it provides bonuses to two schools.
A wand cannot provide an enhancement bonus to more than two schools.

\begin{dtable}
    \lcaption{Magic Wands}
    \begin{dtabularx}{\columnwidth} {>{\ccol}X c c}
        \tb{Enhancement Bonus} & \tb{Base Price} & \tb{Item Level}\\
        \hline
        \plus1 wand & 50 gp & 1st \\
        \plus2 wand & 250 gp & 3rd \\
        \plus3 wand & 1,250 gp & 7th \\
        \plus4 wand & 6,250 gp & 10th \\
        \plus5 wand & 31,250 gp & 14th \\
    \end{dtabularx}
\end{dtable}

\parhead{Physical Description} A typical wand is 6 inches to 12 inches long and about 1/4 inch thick, and often weighs no more than 1 ounce.
Most wands are wood, but some are bone.
A rare few are metal, glass, or even ceramic, but these are quite exotic.
Occasionally, a wand has a gem or some device at its tip, and most are decorated with carvings or runes.
A typical wand has AC 7, 5 hit points, hardness 5, and a break DC of 16.

\begin{dtable}
    \lcaption{Wand Special Abilities}
    \begin{dtabularx}{\columnwidth}{>{\lcol}X l l}
        \tb{Special Ability} & \tb{Cost} & \tb{Item Level} \\
        \hline
        Enlarging & 400 gp & 4th \\
        Flaming & 400 gp & 4th \\
        Freezing & 400 gp & 4th \\
        Shocking & 1,600 gp & 7th \\
    \end{dtabularx}
\end{dtable}
%Wand special abilities: priced as close range, immediate action trigger.
%Trigger needs to be immediate; otherwise, a caster casting a quickened spell
%could potentially unleash four separate effects simultaneously.
\impdescription{Enlarging}{400}{4th}{Faint Universal}{2nd}{Immediate (triggered)} When you cast a spell, you can activate this wand to double the range of the spell.

When you activate this ability, the wand doubles in length.
This effect lasts for 5 rounds, during which time you cannot activate the item.

\mitemreq{No school}{}{1}{2nd}{Craft (as wand) 6}

\impdescription{Flaming}{400}{4th}{Faint Evocation (Energy) [Fire]}{2nd}{Immediate (triggered)} When you cast a spell, you can activate this wand to ignite a single creature affected by the spell for 5 rounds.

An ignited creature is vulnerable, causing it to take a \minus2 penalty to attacks, defenses, and checks.
In addition, at the end of each of its turns, it takes d6 damage from the fire.
If the creature takes a move action, it can attempt a DC 10 Dexterity check to put out the flames.
This action requires a free hand.
Dropping prone as part of the action gives a \plus5 bonus on this check.

When you activate this ability, the wand is wreathed in flame, causing it to cast light as a torch.
This effect lasts for 5 rounds, during which time you cannot activate the item.

\mitemreqdesc{Evocation}{Energy}{Fire}{1}{2nd}{Craft (as wand) 6}

\impdescription{Freezing}{400}{4th}{Faint Evocation (Energy) [Cold]}{2nd}{Immediate (triggered)} As you cast a spell, you can activate this wand to fatigue a single creature affected by the spell for 5 rounds.
A fatigued creature can neither sprint nor charge and is vulnerable, giving it a \minus2 penalty to attacks, defenses, and checks.

When you activate this ability, the wand radiate frigid cold, causing it to snuff out torches and other small fires within a 5 foot radius of it.
This effect lasts for 5 rounds, during which time you cannot activate the item's abilities.

\mitemreqdesc{Evocation}{Energy}{Cold}{1}{2nd}{Craft (as wand) 6}

\impdescriptiondc{Shocking}{1,600}{7th}{Faint Evocation (Energy) [Electricity]}{8th}{Immediate (triggered)}{Fortitude (level \add 2)} As you cast a spell, you can activate this wand.
If you do, make a Fortitude attack against a single creature affected by the spell.
A successful attack causes the target to be staggered for 5 rounds.

A staggered creature may take a single move action or standard action each round, but not both.
She cannot take full-round actions, but she may take swift actions.
In addition, she is vulnerable, causing her to take a \minus2 penalty on attacks, defenses, and checks.

When you activate this ability, you crackle with electrical energy, causing you to radiate flashing light as a torch.
This effect lasts for 5 rounds, during which time you cannot activate the item's abilities.

\mitemreqdesc{Evocation}{Energy}{Electricity}{2}{4th}{Craft (as wand) 8}

\section{Tools}

\subsection{Scrolls}
A scroll is a spell (or collection of spells) that has been stored in written form.
A spell on a scroll can be used only once.
The writing vanishes from the scroll when the spell is activated.
Using a scroll is basically like casting a spell.

\parhead{Physical Description} A scroll is a heavy sheet of fine vellum or high-quality paper.
An area about 8 1/2 inches wide and 11 inches long is sufficient to hold one spell.
The sheet is reinforced at the top and bottom with strips of leather slightly longer than the sheet is wide.
A scroll holding more than one spell has the same width (about 8 1/2 inches) but is an extra foot or so long for each extra spell.
Scrolls that hold three or more spells are usually fitted with reinforcing rods at each end rather than simple strips of leather.
A scroll has AC 9, 1 hit point, hardness 0, and a break DC of 8.

To protect it from wrinkling or tearing, a scroll is rolled up from both ends to form a double cylinder.
(This also helps the user unroll the scroll quickly.) The scroll is placed in a tube of ivory, jade, leather, metal, or wood.
Most scroll cases are inscribed with magic symbols which often identify the owner or the spells stored on the scrolls inside.
The symbols often hide magic traps.

\parhead{Activation} To activate a scroll, a spellcaster must read the spell written on it.
Doing so involves several steps and conditions.

\parhead{Decipher the Writing} The writing on a scroll must be deciphered before a character can use it or know exactly what spell it contains.
This requires a \ritual{read magic} ritual or a successful Spellcraft check (DC 20 \add spell level).

Deciphering a scroll to determine its contents does not activate its magic unless it is a specially prepared cursed scroll.
A character can decipher the writing on a scroll in advance so that he or she can proceed directly to the next step when the time comes to use the scroll.

\parhead{Activate the Spell} Activating a scroll requires reading the spell from the scroll.
The character must be able to see and read the writing on the scroll.
Activating a scroll spell requires no material components or focus.
(The creator of the scroll provided these when scribing the scroll.) Note that some spells are effective only when cast on an item or items.
In such a case, the scroll user must provide the item when activating the spell.
Activating a scroll spell is subject to disruption just as casting a normally prepared spell would be.
Using a scroll is like casting a spell for purposes of arcane spell failure chance.

To have any chance of activating a spell scroll, the scroll user must meet the following requirements.
\begin{itemize}
    \item The user's spellpower must be at least equal to the scroll's spellpower.
    \item The user must have the spell on his or her spell list.
        The spell list must be of the same magic type as the scroll (arcane, divine, or nature).
    \item The user must have the minimum casting attribute required to cast spells of the scroll's spell level.
        For arcane magic, the minimum attribute is equal to the spell's level.
        For divine and nature magic, the minimum attribute is equal to half the spell's level.
\end{itemize}

\parhead{Determine Effect} A spell successfully activated from a scroll works exactly like a spell cast the normal way.
The spellpower of a spell cast from a scroll is equal to twice the spell's level.

Once a scroll has been activated, the writing disappears from it, leaving behind only faint traces with no magical power.

\parhead{Scroll Levels} Some spells are acquired by multiple classes at different levels.
Use the entry on the table appropriate to the scribing of each individual scroll.

\begin{dtable}
    \lcaption{Scrolls and Potions}
    \begin{dtabularx}{\columnwidth}{X l l}
        \tb{Common Spells\fn{1}} & \tb{Market Price} & \tb{Item Level} \\
        \hline
        1st-Level & 10 gp & 1st \\
        2nd-Level & 40 gp & 1st \\
        3rd-Level & 100 gp & 2nd \\
        4th-Level & 250 gp & 3rd \\
        5th-Level & 600 gp & 5th \\
        6th-Level & 1500 gp & 7th \\
        7th-Level & 3000 gp & 9th \\
        8th-Level & 7000 gp & 11th \\
        9th-Level & 15000 gp & 12th \\
        \tb{Paladin Spells} & \tb{Market Price\fn{2}} & \tb{Item Level} \\
        1st-Level & 40 gp & 1st \\
        2nd-Level & 100 gp & 2nd \\
        3rd-Level & 250 gp & 3rd \\
        4th-Level & 600 gp & 5th \\
    \end{dtabularx}
    1 Includes arcane, divine, and nature spells.
    \\
    2 Scrolls and potions based on paladin spells cost as much as a spell of one level higher because of their rarity.
    The cost to create them is no different than normal, and players attempting to sell such items will find it difficult to find a buyer.
\end{dtable}

\subsection{Potions and Oils}

A potion is a magic liquid that produces its effect when imbibed.
Magic oils are similar to potions, except that oils are applied externally rather than imbibed.
A potion or oil can be used only once.
It can duplicate the effect of a spell that has a casting time of a standard action or less.

Potions are like spells cast upon the imbiber.
The character taking the potion doesn't get to make any decisions about the effect  - the caster who brewed the potion has already done so.
The drinker of a potion is both the effective target and the effective caster of the effect.

The person applying an oil is the effective caster, but the object is the target.

\parhead{Physical Description} A typical potion or oil consists of 1 ounce of liquid held in a ceramic or glass vial fitted with a tight stopper.
The stoppered container is usually no more than 1 inch wide and 2 inches high.
The vial has AC 13, 1 hit point, hardness 1, and a break DC of 12.
Vials hold 1 ounce of liquid.

\parhead{Identifying Potions} In addition to the standard methods of identification, PCs can sample from each container they find to attempt to determine the nature of the liquid inside.
An experienced character learns to identify potions by memory -- for example, the last time she tasted a liquid that reminded her of almonds, it turned out to be a potion of cure moderate wounds.

\parhead{Activation} Drinking a potion or applying an oil requires no special skill.
The user merely removes the stopper and swallows the potion or smears on the oil.
The following rules govern potion and oil use.

Drinking a potion or using an oil on an item of gear is a standard action.
The potion or oil takes effect immediately.

A creature must be able to swallow a potion or smear on an oil.
Because of this, incorporeal creatures cannot use potions or oils.

Any corporeal creature can imbibe a potion.
The potion must be swallowed, or in some other way ingested.
Any corporeal creature can use an oil.

A character can carefully administer a potion to an unconscious creature as a full-round action, trickling the liquid down the creature's throat.
Likewise, it takes a full-round action to apply an oil to an unconscious creature.
Exceptionally large objects or creatures require a greater time expenditure.

\parhead{Potion Descriptions} The spellpower for a standard potion is equal to twice the spell level of the spell in the potion.
Common potions refer to potions of spells on the unrestricted arcane, divine, and nature lists.
Any other spells, such as cleric domain spells and restricted arcane spells, are considered ``uncommon''.

\parhead{Potion Mishaps} Extraordinarily powerful potions can be dangerous to imbibe.
Whenever you use a potion, if the spellpower of the potion exceeds your level, roll 1d20 and subtract the difference between your level and the potion's spellpower.
Compare the result to \trefnp{Potion Mishaps}.

\begin{dtable}
    \lcaption{Potion Mishaps}
    \begin{dtabularx}{\columnwidth}{l X}
        \tb{Mishap Result} & \tb{Mishap Effect} \\
        \hline
        11\plus & No additional effects \\
        6--10 & User also takes damage equal to the potion's spellpower \\
        1--5 & As above, and the user is nauseated for 1 round \\
        0 or below & As above, and the potion does not have its normal effect \\
    \end{dtabularx}
\end{dtable}

\subsection{Rituals}

\begin{dtable}
    \caption{Ritual Costs}
    \begin{dtabularx}{\columnwidth}{X l l l }
        \tb{Ritual Level} & \tb{Cost to Perform} & \tb{Cost to Scribe} & \tb{Item Level} \\
        \hline
        1st-Level & 5 gp & 50 gp & 1st \\
        2nd-Level & 20 gp & 200 gp & 3rd \\
        3rd-Level & 50 gp & 500 gp & 4th \\
        4th-Level & 125 gp & 1250 gp & 7th \\
        5th-Level & 300 gp & 3000 gp & 9th \\
        6th-Level & 750 gp & 7500 gp & 11th \\
        7th-Level & 1500 gp & 15000 gp & 12th \\
        8th-Level & 3500 gp & 35000 gp & 14th \\
        9th-Level & 7500 gp & 75000 gp & 16th \\
    \end{dtabularx}
\end{dtable}

\subsection{Wondrous Items}

Wondrous items are items which are inherently magical in some way.

\tooldescription{Answerstone}{2,000}{8th}{Faint Divination (Knowledge)}{6th}{Standard (specific action)} If you shake this stone and ask a simple yes or no question (a standard action), you receive an answer of either ``yes'', ``no'', or ``unclear''.
The answer has a 75\% chance of being correct.
If the answerstone is incorrect, it will always answer ``unclear''.
The answerstone will always answer ``unclear'' if asked questions about events more than thirty minutes into the future or past, farther than ten miles from its current location.
It can only answer questions about observable quantities (including spells).

If you attempt to use an answerstone when you have no uses of magic items remaining, it will give random answers, including answers other than ``yes'', ``no'', and ``unknown''.
Although answerstones are usually made of stone, they can also be made of other sturdy materials.

\mitemreq{Divination}{Knowledge}{3}{6th}{Craft (bone, metal, or stone)}

\tooldescription{Bag of Holding}{Varies}{see text}{Varied Conjuration (Translocation) [Planar]}{}{\x} This appears to be a common cloth sack about 2 feet by 4 feet in size.
The bag of holding opens into a nondimensional space: Its inside is larger than its outside dimensions.
Regardless of what is put into the bag, it weighs a fixed amount.
This weight, and the limits in weight and volume of the bag's contents, depend on the bag's type, as shown on the table below.

\begin{dtable*}
    \begin{dtabularx}{\textwidth}{l l X X X X}
        \tb{Bag} & \tb{Bag Weight} & \tb{Weight Limit} & \tb{Space Limit} & \tb{Base Price} & \tb{Item Level}\\
        \hline
        Type I & 15 lb.
        & 250 lb.
        & 5 ft.
        radius & 750 gp & 5th \\
        Type II & 20 lb.
        & 500 lb.
        & 10 ft.
        radius & 1,500 gp & 7th \\
        Type III & 25 lb.
        & 1,000 lb.
        & 15 ft.
        radius & 3,000 & 9th \\
        Type IV & 30 lb.
        & 1,500 lb.
        & 20 ft.
        radius & 6,000 & 10th \\
    \end{dtabularx}
\end{dtable*}

If the bag is overloaded, or if sharp objects pierce it from the outside, the bag ruptures and is ruined.
All contents are lost forever.
If a bag of holding is turned inside out, its contents spill out, unharmed, but the bag must be put right before it can be used again.
If living creatures are placed within the bag, they can survive for up to 10 minutes, after which time they suffocate.
Retrieving a specific item from a bag of holding is a move action - unless the bag contains more than an ordinary backpack would hold, in which case retrieving a specific item is a full-round action.

If a bag of holding is placed within a portable hole a rift to the Astral Plane is torn in the space: Bag and hole alike are sucked into the void and forever lost.
If a portable hole is placed within a bag of holding, it opens a gate to the Astral Plane: The hole, the bag, and any creatures within a 10-foot radius are drawn there, destroying the portable hole and bag of holding in the process.

\mitemreqdesc{Conjuration}{Translocation}{Planar}{2}{4th}{Craft (textiles) 8}

\tooldescription{Bag of Tricks}{Varies}{see text}{Varied Conjuration (Summoning)}{}{\x} This small sack appears normal and empty.
However, anyone reaching into the bag feels a small, fuzzy ball.
If the ball is removed and tossed up to 20 feet away, it turns into an animal.
The animal serves the character who drew it from the bag for 1 minute (or until slain or ordered back into the bag), at which point it disappears.
It can follow any of the commands described in the Creature Handling skill.
Each of the three kinds of a bag of tricks produces a different set of animals.
Use the following tables to determine what animals can be drawn out of each.

Animals produced are always random, and only one may exist at a time.
If a new animal is drawn from the bag, the previous animal is immediately unsummoned.

\begin{dtable}
    \lcaption{Bag of Tricks Types}
    \begin{dtabularx}{\columnwidth}{l X X}
        \tb{Bag Type} & \tb{Base Price} & \tb{Item Level}\\
        \hline
        Gray & 200 gp & 3rd \\
        Rust & 2,000 gp & 8th \\
        Tan & 12,000 gp & 12th \\
    \end{dtabularx}
\end{dtable}

\begin{dtable}
    \lcaption{Bag of Tricks Animals}
    \begin{dtabularx}{\columnwidth}{l X X X}
        \tb{Roll} & \tb{Gray Bag} & \tb{Rust Bag} & \tb{Tan Bag} \\
        \hline
        1-20   & Dire rat   & Dire weasel   & Snake, giant constrictor \\
        21-40  & Cat        & Lion          & Dire lion \\
        41-60  & Badger     & Dire badger   & Dire wolverine \\
        61-80  & Owl        & Giant owl     & Griffon \\
        81-100 & Wolf       & Dire wolf    & Rhinoceros \\
    \end{dtabularx}
\end{dtable}

\mitemreq{Conjuration}{Summoning}{1 (gray), 3 (rust), or 5 (tan)}{2 (gray), 6 (rust), or 10 (tan)}{Craft (textiles) 6 (gray), 10 (rust), or 14 (tan)}

\tooldescription{Bead of Force}{600}{5th}{Moderate Evocation (Control) [Force]}{10th}{Standard (specific action)} This small black sphere appears to be a lusterless pearl.
By holding it in your hand and throwing it at an object or creature, you can surround the struck target with a \spell{resilient sphere}, as the spell.
The bead is treated as a thrown weapon with a range increment of 20 feet.
Your Reflex attack bonus for the \spell{resilient sphere} effect is equal to your level \add 5.

If you miss your target, the bead will still activate if it strikes a solid object, potentially wasting its effect.
After being activated, the bead is destroyed.

\mitemreqdesc{Evocation}{Control}{Force}{5}{10th}{Craft (jewelry) 14}

\tooldescription{Bottle of Air}{800}{5th}{Faint Conjuration (Creation)}{4th}{Standard (specific action)} This item appears to be a normal glass bottle with a cork.
By uncorking the bottle (a standard action), you can cause the bottle to continually fill itself with clean air for 1 hour.
It will remain filled with clean air even if taken into an airless environment or environment with dangerous gases, allowing you to breathe from the bottle.
Liquid can still fill the bottle, and the bottle will continuously bubble as air is created until the liquid is removed.

The bottle can even be shared by multiple creatures who pass it around.
Breathing out of the bottle is a standard action, but a creature that does so can then act for as long as it can hold its breath.

\mitemreq{Conjuration}{Creation}{2}{4th}{Craft (ceramics) 8}

\tooldescription{Candle of Truth}{100}{2nd}{Faint Enchantment (Inhibition) [Mind-Affecting]}{6th}{Standard (specific action)} By lighting this white tallow candle, you can prevent creatures within a \areamed radius of the candle from being able to lie, as the \ritual{zone of truth} ritual.
The zone lasts as long as the candle remains lit, which is normally 1 hour.
If the candle is snuffed, the zone immediately ends.
Relighting the candle does not recreate the zone.

%no good craft skill
\mitemreq{Enchantment}{Inhibition}{Mind-Affecting}{2}{6th}{Craft (alchemy) 10}

\tooldescription{Friendstone}{3,000}{9th}{Strong Conjuration/Divination (Scrying, Translocation) [Teleportation]}{14th}{Standard (specific action)} This item appears to be a glossy, smooth stone about three inches in diameter.
It can be used to summon a willing creature from a great distance.
By pressing the stone to your forehead and speaking your full name, you can activate the stone to make it attune to you, allowing you to be summoned by a creature using the stone.
This replaces the stone's attunement to any other creature.

By grasping the stone in your hand and throwing it to the ground, you can activate cause the stone to transport the creature attuned to the stone to the stone's location.
The creature is granted a brief glimpse of the stone's surroundings, as if looking out through the stone, and may refuse the summoning.
If the creature refuses, or if it is physically impossible for the creature to appear, your activation of the stone has no effect.
If the creature accepts, the stone is destroyed and the creature is unerringly teleported into the stone's location.

This item is usually made of stone, but can also be made from glass.

\mitemreq{Conjuration/Divination}{Scrying, Translocation}{7}{14th}{Craft (ceramics or stone) 18th}

\tooldescription{Witch's Broom}{60,000}{16th}{Strong Divination/Transmutation (Communication, Imbuement)}{14th}{Standard (specific action)} This broom can fly as directed by its owner, as the \ritual{overland flight} ritual.
Riding the broom is like riding a mount, except that you do not need to control it and some actions (such as taking cover behind the mount) are infeasible due to the shape of the broom.

In addition to riding the broom, you attune to the broom as a standard action by holding it firmly by the handle and concentrating.
If you have attuned to the broom, you can command it to fly on its own by specifying a location or a direction and distance.
The broom understands all locations that you know.
If the broom is within \rngext range of you and can hear your commands, it will obey your commands after 1 round.
Both attuning to the broom and issuing a command to the broom count as activating the broom, but simply riding the broom does not.

\mitemreq{Divination/Transmutation}{Communication, Imbuement}{7}{14th}{Craft (wood) 18}

\section{Special Materials}

Some materials are inherently magical.
Items made from such materials gain special properties automatically.

\subsection{Material Types}
There are three main types of materials used to create items: metal, wood, and leather.
In order to create an item with a special material, the item must normally be made with a material of the appropriate material type.
In the case of items that are made from multiple materials, such as polearms, the primary functional part of the item determines which type of special material can be used to create it.
For example, the material type for a weapon is generally determined by its striking surface.

\subsection{Special Material Descriptions}

\begin{dtable}
    \lcaption{Special Materials}
    \begin{dtabularx}{\columnwidth}{l l X l}
        \tb{Material Name} & \tb{Material Type} & \tb{Effect} & \tb{Price} \\
        \hline
        Adamantine & Metal & Grant or overcome physical damage reduction & \\
        Cold Iron & Metal & Effective against supernatural creatures & \\
        Darkwood & Wood & Extraordinarily light & \\
        Dragonbone & Metal & Inherently magical & \\
        Dragonhide & Leather & Resist energy & \\
        Dragonscale & Metal & Resist energy & \\
        Ironwood & Metal & Metallic wood & \\
        Mithral & Metal & Extraordinarily light & \\
        Silvered & Metal & Effective against supernatural creatures & \\
    \end{dtabularx}
\end{dtable}

\parhead{Adamantine}
Adamantine is a rare metal that is among the hardest substances known.

Adamantine weapons ignore the hardness of creatures and objects.

Adamantine armor grants its wearer physical damage reduction equal to half your level.
This damage reduction is overcome by adamantine weapons.

Adamantine items of any kind have double the hit points and hardness of an equivalent item of the same type.

\parhead{Cold Iron}

Cold iron is iron that has been magically smelted without the use of heat.

Cold iron items are especially effective against some supernatural creatures, as indicated in the creature descriptions.

\parhead{Darkwood}
Darkwood is a rare magic wood that is very light.

Darkwood weapons have the Finesse weapon property, allowing you to use Dexterity to attack with them instead of Strength.

Darkwood shields have their armor check penalty reduced by 2.

Darkwood items of any kind weigh half as much as an equivalent item of the same type.

\parhead{Dragonbone}

Dragon bones can be used in place of metal when creating weapons and armor.

Dragonbone weapons grant an enhancement bonus to physical accuracy and damage equal to a third of your level (to a maximum of \plus5 at 15th level).

Dragonbone armor grants its wearer spell resistance.
To affect a creature with spell resistance using a spell, a caster must make an attack with an accuracy equal to its spellpower.
If the attack beats the creature's spell resistance, the spell works normally.
Otherwise, the spell has no effect on the creature.

\parhead{Dragonhide}

Dragon hide can be used as leather when creating weapons and armor.

Dragonhide armor grants damage reduction equal to your level against the energy type used by the dragon the hide is taken from.

\parhead{Dragonscale}

Dragon scales can be used in place of metal when creating armor.

Dragonscale armor grants damage reduction equal to your level against the energy type used by the dragon the hide is taken from.

\parhead{Ironwood}
Ironwood is a magic wood created by the \spell{ironwood} ritual.
It has been magically hardened, giving it the strength of iron.
Ironwood can be used in place of metal when creating weapons and armor.

Ironwood items have no special properties except that they are made from wood instead of metal.

\parhead{Mithral}
Mithral is a rare metal that is very light, but has the strength of iron.

Mithral weapons have the Finesse weapon property, allowing you to use Dexterity to attack with them instead of Strength.

Mithral armor has its armor check penalty reduced by 2, and its arcane spell failure reduced by 10.

Mithral items of any kind weigh half as much of an equivalent item of the same type.

\parhead{Silvered}

Silvered items have silver infused into them in the process of their creation.
Only metal items can be silvered.

Silvered items are especially effective against some supernatural creatures, as indicated in the creature descriptions.

\section{Magic Item Rules}

\subsection{Magic Item Auras}

Magic items radiate magical auras which can be detected with the Spellcraft skill (see \pcref{Spellcraft}).
Each item describes the auras that can be detected on it, including the strength, school, and descriptors, as appropriate.

\subsection{Damaging Magic Items}

A magic item is normally unharmed by attacks unless it is unattended or is specifically targeted by an effect.
A magic item's special defenses are all equal to 10 \add the spellpower of the item.
The only exceptions to this are intelligent magic items, which apply their Charisma and Intelligence to their Mental defense.

Magic items, unless otherwise noted, take damage as nonmagical items of the same sort.
A damaged magic item continues to function, but if it is broken, its magic ceases to function until it is repaired.
If it is destroyed, all its magical power is lost.

\subsection{Repairing Magic Items}

A magic item which is broken (but not destroyed) can be repaired for 10\% of the value of the item.

\subsection{Intelligent Items}

Some magic items, particularly weapons, have an intelligence all their own.
Only permanent magic items (as opposed to those with a single use or those with charges) can be intelligent.
(This means that potions, scrolls, and wands, among other items, are never intelligent.)

In general, fewer than 1\% of magic items have intelligence.

\subsection{Cursed Items}

Some items are cursed - incorrectly made, or corrupted by outside forces.
Cursed items might be particularly dangerous to the user, or they might be normal items with a minor flaw, an inconvenient requirement, or an unpredictable nature.
Many cursed items are difficult to identify and remove, requiring the use of rituals such as \spell{remove curse}.
