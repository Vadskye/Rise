\chapter{Classes}

Your character's class represents the things your character has chosen to train in.
This choice determines a great deal about your character's abilities.

\section{Class Introductions}

There are eleven classes in Rise.
\begin{itemize}
    \item Barbarians are mighty warriors who can enter a deadly battlerage.
    \item Clerics are divine spellcasters who draw power from their veneration of a deity or ideal.
    \item Druids are nature spellcasters who draw power from their veneration of the natural world.
    \item Fighters are highly disciplined warriors who excel in physical combat of any variety.
    \item Monks are agile masters of ``\ki'' who hone their personal abilities to strike down foes and perform supernatural feats.
    \item Paladins are divinely empowered warriors whose devotion to an alignment grants them the ability to discern and smite their foes.
    \item Rangers are skilled hunters who bridge the divide between nature and civilization.
    \item Rogues are exceptionally skillful characters known for their ability to strike at their foe's weak points in combat.
    \item Sorcerers are arcane spellcasters with an intuitive and flexible understanding of magic.
    \item Spellwarped wield a unique blend of martial skill and narrowly focused magical abilities.
    \item Wizards are arcane spellcasters with a highly studied and deep understanding of magic.
\end{itemize}

\subsection{Class Description Format}

\parhead{Alignment}
Some classes require specific alignments (see \pcref{Alignment}).
Most classes allow characters of any alignment.

\parhead{Class Skills}
These are skills that members of this class are typically good at (see \pcref{Skills}).

\subsubsection{Base Class Features}

Abilities contained within this heading only apply to characters with the current class as a base class.
A character can normally have only one base class.
Except in unusual circumstances, a character's base class is the class that the character took at 1st level.

\parhead{Skill Points}
This is the number of skill points that members of this class get.

\parhead{Defenses}
Each class grants bonuses to defenses the class specializes in.
If the class has a good base defense progression, it grants a \plus4 bonus to that defense.
If the class has an average base defense progression, it grants a \plus2 bonus to that defense.
These bonuses apply regardless of the attribute or base progression used to determine the defense.

These bonuses do not stack with other defense bonuses granted by base classes.
If a character has multiple base classes, use the highest bonuses that apply to each defense.

\parhead{Weapon and Armor Proficiencies}
These are the types of equipment that members of this class are trained in using.

\subsection{Base Progressions}

The tables summarizing the abilities granted by each class contain information about the combat prowess and defenses that each class provides.

\parhead{Combat Prowess}
This measures how skilled a character is in combat.
A character can use his combat prowess to determine his Armor and Maneuver defenses, and his accuracy with physical attacks.
There are three progressions: Good, Average, or Poor.
The effects of each progression are described on \trefnp{Base Progressions}.

A high combat prowess can grant additional attacks, as described in \pcref{Multiple Attacks}.

\parhead{Base Defense}\label{Base Defense Progressions}
This measures how resistant members of the class are to unusual kinds of attacks.
There are three kinds of special defenses.
Your Fortitude defense represents your ability to resist attacks to your body, like poisons and diseases.
Your Reflex defense represents your ability to avoid attacks, such as pit traps or explosions.
Your Mental defense represents your ability to resist mental influence, like fearsome creatures and enchantment spells.
There are three progressions: Good, Average, or Poor.
The effects of each progression are described on \trefnp{Base Progressions}.

\begin{dtable}
    \lcaption{Base Progressions}
    \setlength\tabcolsep{0.45em}%
    \begin{dtabularx}{\columnwidth}{l l X}
        \tb{Progression} & \tb{Combat Prowess} & \tb{Base Defense Bonus}        \\
        \hline
        Good                & Class level \add 2             & Five-quarters class level  \\
        Average             & Four-fifths class level \add 2 & Class level                \\
        Poor                & Two-thirds class level \add 1  & Three-quarters class level \\
    \end{dtabularx}
\end{dtable}

\begin{dtable}
    \lcaption{Base Defense Progression Bonuses}
    \begin{dtabularx}{\columnwidth}{l X X X}
        \tb{Level} & \tb{Good} & \tb{Average} & \tb{Poor} \\
        \hline
        \alldefenseprogressionrow{1}  \\
        \alldefenseprogressionrow{2}  \\
        \alldefenseprogressionrow{3}  \\
        \alldefenseprogressionrow{4}  \\
        \alldefenseprogressionrow{5}  \\
        \alldefenseprogressionrow{6}  \\
        \alldefenseprogressionrow{7}  \\
        \alldefenseprogressionrow{8}  \\
        \alldefenseprogressionrow{9}  \\
        \alldefenseprogressionrow{10} \\
        \alldefenseprogressionrow{11} \\
        \alldefenseprogressionrow{12} \\
        \alldefenseprogressionrow{13} \\
        \alldefenseprogressionrow{14} \\
        \alldefenseprogressionrow{15} \\
        \alldefenseprogressionrow{16} \\
        \alldefenseprogressionrow{17} \\
        \alldefenseprogressionrow{18} \\
        \alldefenseprogressionrow{19} \\
        \alldefenseprogressionrow{20} \\
    \end{dtabularx}
\end{dtable}

\begin{dtable}
    \lcaption{Combat Prowess Progression Bonuses}
    \begin{dtabularx}{\columnwidth}{l l X X}
        \tb{Level} & \tb{Good} & \tb{Average} & \tb{Poor} \\
        \hline
        \allbabprogressionrow{1}  \\
        \allbabprogressionrow{2}  \\
        \allbabprogressionrow{3}  \\
        \allbabprogressionrow{4}  \\
        \allbabprogressionrow{5}  \\
        \allbabprogressionrow{6}  \\
        \allbabprogressionrow{7}  \\
        \allbabprogressionrow{8}  \\
        \allbabprogressionrow{9}  \\
        \allbabprogressionrow{10} \\
        \allbabprogressionrow{11} \\
        \allbabprogressionrow{12} \\
        \allbabprogressionrow{13} \\
        \allbabprogressionrow{14} \\
        \allbabprogressionrow{15} \\
        \allbabprogressionrow{16} \\
        \allbabprogressionrow{17} \\
        \allbabprogressionrow{18} \\
        \allbabprogressionrow{19} \\
        \allbabprogressionrow{20} \\
    \end{dtabularx}
\end{dtable}

\parhead{Class Features}
The class features that a character gets for being a member of the class.

\section{Barbarian}

\begin{dtable}
    \lcaption{Barbarian Progression}
    \begin{dtabularx}{\columnwidth}{>{\ccol}p{\levelcol} >{\ccol}p{\babcolgood} *{3}{>{\ccol}p{\savecol}} >{\lcol}X}
        \tb{Level} & \tb{Combat Prowess} & \tb{Fort} & \tb{Ref} & \tb{Ment} & \tb{Special} \\
        \hline
        \barbarianprogressionrow{1}  & Rage \plus2, damage reduction, grit \\
        \barbarianprogressionrow{2}  & Channeled rage                \\
        \barbarianprogressionrow{3}  & Uncanny dodge                 \\
        \barbarianprogressionrow{4}  & Fast movement                 \\
        \barbarianprogressionrow{5}  & Rage \plus3                   \\
        \barbarianprogressionrow{6}  & Channeled rage                \\
        \barbarianprogressionrow{7}  & Tireless rage                          \\
        \barbarianprogressionrow{8}  & Larger than life              \\
        \barbarianprogressionrow{9}  & Improved uncanny dodge        \\
        \barbarianprogressionrow{10} & Channeled rage, rage \plus4   \\
        \barbarianprogressionrow{11} & Chaotic rage                 \\
        \barbarianprogressionrow{12} & Fury of the storm                  \\
        \barbarianprogressionrow{13} & Indomitable will              \\
        \barbarianprogressionrow{14} & Channeled rage                \\
        \barbarianprogressionrow{15} & Rage \plus5                   \\
        \barbarianprogressionrow{16} & Larger than belief            \\
        \barbarianprogressionrow{17} & Mighty resilience             \\
        \barbarianprogressionrow{18} & Channeled rage                \\
        \barbarianprogressionrow{19} & Deathless rage                \\
        \barbarianprogressionrow{20} & Rage \plus6
    \end{dtabularx}
\end{dtable}

\classbasics{Alignment} Any nonlawful.

\classbasics{Class Skills}
\begin{itemize}
    \item \subparhead{Strength} Climb, Jump, Sprint, Swim.
    \item \subparhead{Dexterity} Balance, Ride, Tumble.
    \item \subparhead{Perception} Awareness, Creature Handling, Survival.
    \item \subparhead{Other} Intimidate.
\end{itemize}

\subsection{Base Class Features}
A character with barbarian as a base class gains the following abilities.

\classbasics{Skill Points} 10.

\classbasics{Defenses} \plus4 Fortitude, \plus2 Reflex.

\cf{Bbn}{Weapon and Armor Proficiency}
A barbarian is proficient with simple weapons, any four other weapon groups, light armor, medium armor, and shields.

\cf{Bbn}{Damage Reduction}[Ex]
A barbarian has the ability to shrug off some amount of injury from attacks.
He has \glossterm{damage reduction} against physical damage equal to his barbarian level.

\cf{Bbn}{Grit}[Ex]
The barbarian halves all \glossterm{critical damage} he takes (to a minimum of 1 damage).

\subsection{Class Features}
All barbarians have the following abilities.

\cf{Bbn}{Rage}[Ex]
Twice per day, a barbarian can fly into a rage as a free action.
Raging has the following benefits and drawbacks:
\begin{itemize}
    \item \plus2 bonus to damage with physical attacks.
    \item \plus2 bonus to Fortitude and Mental defense.
    \item 2 temporary hit points per Willpower.
        These extra hit points gained from raging are lost before any other hit points (see \pcref{Temporary Hit Points}).
    \item \minus2 to physical defenses (Armor, Maneuver, Reflex).
    \item Unable to take any action that requires patience or concentration, such as casting spells.
    \item If the barbarian does not spend a swift round to sustain the rage, it ends at the end of the round.
    \item At the end of each round, if the barbarian did not attack a creature or object, he takes nonlethal damage equal to his barbarian level.
\end{itemize}

A rage typically lasts for up to 5 rounds.
At the end of the rage, the barbarian takes nonlethal damage equal to his barbarian level.
If the barbarian has any temporary hit points remaining at the end of his rage, the nonlethal damage is dealt to those hit points before they go away.
In addition, he becomes \fatigued until he rests for 5 minutes.
The barbarian cannot enter a rage while he is fatigued from his previous rage.

The bonuses granted by a barbarian's rage increase with level.
This is called the barbarian's rage bonus.
At 5th level, and every 5 levels thereafter, the bonus to physical damage and the bonus to Fortitude and Mental defenses increases by \plus1.
In addition, the number of hit points gained per Willpower increases by 1, and he can use his rage one additional time per day.
His penalty to physical defenses while raging remains the same.

\cf{Bbn}[2nd]{Channeled Rage}
The barbarian gains the ability to channel his rage to gain new abilities.
He chooses a single channeled rage from the list below.
Whenever the barbarian enters a rage, he may gain the benefits of one channeled rage he knows.
Some channeled rages require a minimum barbarian level, as indicated before the name of the ability.
At his 6th barbarian level, and every four barbarian levels thereafter, the barbarian gains an additional channeled rage.

All channeled rages are extraordinary abilities unless otherwise noted.

\subcf{Athletic Rage}
The barbarian adds his rage bonus to his Climb, Jump, Sprint, and Swim checks.

\subcf{Rapid Rage}
The barbarian gains a \plus10 foot bonus to land speed.

\subcf{Savage Rage}
The barbarian gains the unarmed warrior ability (see Unarmed Warrior, \pref{Mnk:Unarmed Warrior}), increasing his power with unarmed attacks (1d6 damage for a Medium barbarian).

\subcf{Wary Rage}
The barbarian only suffers a \minus1 penalty to physical defenses for raging.

\subcf{Willful Rage}
The barbarian adds his rage bonus to his Mental defense.

\subcf{6th -- Destructive Rage}
When attacking, the barbarian ignores an amount of hardness equal to his Strength.

\subcf{6th -- Furious Styles}
The barbarian can initiate or change combat styles as part of the swift action he uses to sustain his rage.

\subcf{6th -- Overwhelming Rage}
Overwhelmed foes the barbarian threatens increase their overwhelm penalties by 1.

\subcf{6th -- Terrifying Rage}
Whenever the barbarian makes a physical melee attack, he may also make a special attack against the target's Mental defense.
His accuracy is equal to his Willpower.
Success means the target is \shaken for 5 rounds.
\norepeatnotes

\subcf{10th -- Critical Rage}
The barbarian gains a bonus equal to his Perception on attacks made to confirm critical hits. Whenever the barbarian scores a critical hit while raging, he extends his rage by 2 rounds.

\subcf{10th -- Overpowering Rage}
The barbarian adds his rage bonus to his accuracy with combat maneuvers.
Once per round, when the barbarian successfully performs a combat maneuver, he gains temporary hit points equal to his Willpower.

\subcf{10th -- Taunting Rage}
Once per round, the barbarian can make a special attack against the Mental defense of a foe within \rngmed range of him.
His accuracy is equal to his Willpower.
If the attack succeeds, the foe is \taunted for 5 rounds.
\norepeatnotes.

\subcf{10th -- Endless Rage}
The barbarian's rage lasts for an additional 5 rounds.

\subcf{10th -- Terrifying Rage, Improved}
This channeled rage functions like terrifying rage, except that it affects all foes the barbarians threatens.
The barbarian must have the terrifying rage ability to choose this ability.

\subcf{14th -- Spellbreaker Rage (Su)}
The barbarian gains spell resistance equal to 10 \add his Constitution.
To affect the barbarian with a spell, a caster must make an attack with an accuracy equal to its spellpower.
If the attack beats the barbarian's spell resistance, the spell works normally.
Otherwise, the spell has no effect on the barbarian.

\subcf{14th -- Taunting Rage, Improved}
This channeled rage functions like taunting rage, except that it affects all foes the barbarians threatens.
The barbarian must have the taunting rage rage ability to choose this ability.

\subcf{14th -- Whirlwind Rage}
Whenever the barbarian is threatened by at least five creatures, he gains a physical damage bonus equal to his rage bonus.

\subcf{18th -- Endless Rage}
If the barbarian takes damage and does not receive healing during a given round, that round does not count against the duration of his rage.

%\subcf{18th -- Hurricane Rage} This channeled rage functions like whirlwind rage, except that the damage bonus is equal to the number of foes the barbarian threatens, up to a maximum bonus equal to twice his rage bonus.
%The barbarian must have the whirlwind rage ability to choose this ability.

\subcf{18th -- Invulnerable Rage}
The barbarian doubles his damage reduction.
%This replaces his rage bonus to Strength.

\subcf{18th -- Mindless Rage}
The barbarian becomes immune to mind-affecting spells and effects.

\cf{Bbn}[3rd]{Uncanny Dodge}[Ex]
A barbarian can react to danger before his senses would normally allow him to do so.
He reduces his overwhelm penalties by 1.
If his overwhelm penalty is reduced to 0, he is not considered to be overwhelmed.
In addition, he is not \unaware when attacked by surprise.

\cf{Bbn}[4th]{Fast Movement}[Ex]
The barbarian increases his land speed by 10 feet while \unencumbered.

\cf{Bbn}[7th]{Tireless Rage}[Ex]
The barbarian does not take damage when his rage ends, and is not \fatigued at the end of his rage.
This can allow him to rage multiple times without resting.

\cf{Bbn}[8th]{Larger than Life}[Ex]
A barbarian holds the strength of a giant in the body of a man (or woman).
The barbarian is treated as being one size category larger than he actually is for all purposes except his physical space and reach, and the weapons he wields.
Although he uses weapons of the same size as normal, his weapons deal damage as if they were one size category larger, including natural weapons and unarmed strikes.
The benefits of this class feature stack with the effects of spells and abilities that increase the barbarian's size category.

\cf{Bbn}[9th]{Improved Uncanny Dodge}[Ex]
The barbarian reduces his overwhelm penalties by 2.
This does not stack with the effects of uncanny dodge.
If his overwhelm penalty is reduced to 0, he is not considered to be overwhelmed.

\cf{Bbn}[11th]{Chaotic Rage}[Ex]
The barbarian gains the ability to change channeled rage abilities at will, without consuming an additional use of his rage ability.
He may not change channeled rages in this way more than once per round.

\cf{Bbn}[12th]{Fury of the Storm}[Ex]
A barbarian cannot be overwhelmed.
He does not suffer overwhelm penalties, regardless of the number of enemies threatening him.

\cf{Bbn}[13th]{Indomitable Will}[Ex]
The barbarian becomes immune to compulsion spells and effects.

\cf{Bbn}[16th]{Larger than Belief}[Ex]
The barbarian's larger than life ability improves.
He is treated as being two size categories larger than he actually is.

\cf{Bbn}[17th]{Mighty Resilience}[Ex]
The barbarian cannot take more than half his maximum hit points in damage during a single round.
Any excess damage is ignored.

\cf{Bbn}[19th]{Deathless Rage}[Ex]
While raging, the barbarian ignores all penalties from critical damage.
This can prevent him from dying from critical damage.
However, if his critical damage exceeds twice his maximum hit points, the barbarian immediately dies.

\subsubsection{Ex-Barbarians}
A barbarian who becomes lawful loses his ability to rage, and cannot gain more levels as a barbarian.
He retains all his other class features.
If he stops being lawful, he regains his ability to rage and take barbarian levels.

\section{Cleric}
\begin{dtable}
    \lcaption{Cleric Progression}
    \begin{dtabularx}{\columnwidth}{>{\ccol}p{2em} >{\ccol}p{\babcolavg} *{3}{>{\ccol}p{\savecol}} >{\lcol}X}
        \tb{Level} & \tb{Combat Prowess} & \tb{Fort} & \tb{Ref} & \tb{Ment} & \tb{Special} \\
        \hline
        \clericprogressionrow{1}  & Devotion \plus2, domain gifts, spells, rituals \\
        \clericprogressionrow{2}  & Domain invocation                          \\
        \clericprogressionrow{3}  & Domain invocation                          \\
        \clericprogressionrow{4}  & Devotion feat                              \\
        \clericprogressionrow{5}  & Devotion \plus3                            \\
        \clericprogressionrow{6}  & Domain aspect                              \\
        \clericprogressionrow{7}  & \x                                         \\
        \clericprogressionrow{8}  & Domain aspect                              \\
        \clericprogressionrow{9}  & Expanded devotion                          \\
        \clericprogressionrow{10} & Devotion \plus4, greater domain invocation \\
        \clericprogressionrow{11} & \x                                         \\
        \clericprogressionrow{12} & Greater domain invocation                  \\
        \clericprogressionrow{13} & \x                                         \\
        \clericprogressionrow{14} & Devotion feat                              \\
        \clericprogressionrow{15} & Devotion \plus5                            \\
        \clericprogressionrow{16} & Domain mastery                             \\
        \clericprogressionrow{17} & \x                                         \\
        \clericprogressionrow{18} & Domain mastery                             \\
        \clericprogressionrow{19} & Endless devotion                           \\
        \clericprogressionrow{20} & Devotion \plus6                            \\
    \end{dtabularx}
\end{dtable}

\classbasics{Alignment} If the cleric worships a deity, his alignment must be within one step of his deity's (that is, it may be one step away on either the lawful-chaotic axis or the good-evil axis, but not both).
He may not be neutral unless his deity's alignment is also neutral.

\classbasics{Class Skills}
\subparhead{Intelligence} Heal, Knowledge (arcana, local, religion, the planes), Linguistics.
\subparhead{Perception} Sense Motive, Spellcraft.
\subparhead{Other} Bluff, Intimidate, Persuasion.

\subsection{Base Class Features}
A character with cleric as a base class gains the following abilities.

\classbasics{Skill Points} 5.

\classbasics{Defenses} \plus2 Fortitude, \plus4 Mental.

\cf{Clr}{Weapon and Armor Proficiency}
 Clerics are proficient with simple weapons, any two other weapon groups, light and medium armor, and shields.

\cf{Clr}{Rituals}
Clerics can perform rituals to create unique magical effects (see \pcref{Rituals}).
A cleric begins play with a ritual book containing one divine ritual of his choice (see \pcref{Divine Rituals}).

\cf{Clr}{Domain Gifts}[Su]
A cleric's abilities are shaped by his domains.
He gains the domain gifts of both of his domains.
Domain gifts are not activated.
The gifts offered by each domain are listed at \pcref{Domain Gifts}.

\subsection{Class Features}
All clerics have the following abilities.

\cf{Clr}{Domains}
A cleric chooses two domains, which represent his personal spiritual inclinations.
If he has a deity, he must choose his domains from among those his deity offers.
A cleric's choice of domains has broad effects on the cleric's spellcasting and supernatural abilities.
The domains are listed below.

\begin{itemize}
    \item{Air}
    \item{Chaos}
    \item{Death}
    \item{Destruction}
    \item{Earth}
    \item{Evil}
    \item{Fire}
    \item{Good}
    \item{Knowledge}
    \item{Law}
    \item{Magic}
    \item{Protection}
    \item{Strength}
    \item{Travel}
    \item{Trickery}
    \item{Vitality}
    \item{War}
    \item{Water}
\end{itemize}

\cf{Clr}{Spells}
A cleric casts divine spells using his devotion.
A cleric's spellpower is normally equal to his divine power.
See the Divine Power and Devotion abilities, below.

The number of spells a cleric knows is given on \trefnp{Cleric Spells Known}.
The cleric may learn spells from both the divine spell list (see \pcref{Divine Spells}) and his domain spell lists (see \pcref{Cleric Domains}).
Sometimes these domain spells are spells that are normally available on the divine spell list, but often they are only accessible by the domain.

The number of spells a cleric can cast per day is given on \trefnp{Cleric Spell Slots}.

In order to regain his spell slots for the day, the cleric must dismiss all his active spells and spend 1 hour performing a ritual, worshipping, or quietly contemplating.
The cleric cannot regain spell slots in this way more than once per day.

A cleric can't cast spells of an alignment opposed to his own or his deity's (if he has one).
Spells associated with particular alignments are indicated by the chaos, evil, good, and law descriptors in their spell descriptions.

\begin{dtable}
    \lcaption{Cleric Spell Slots}
    \begin{dtabularx}{\columnwidth}{>{\ccol}X *{9}{>{\ccol}p{\spellcol}}}
        & \multicolumn{9}{c}{\tb{---{}---{}---{}---{}---{}---{}---{}---Spell Level---{}---{}---{}---{}---{}---{}---{}---}} \\
\hline
        \tb{Level} & \tb{1st} & \tb{2nd} & \tb{3rd} & \tb{4th} & \tb{5th} & \tb{6th} & \tb{7th} & \tb{8th} & \tb{9th} \\
        1st  & 3 & \x & \x & \x & \x & \x & \x & \x & \x \\
        2nd  & 4 & \x & \x & \x & \x & \x & \x & \x & \x \\
        3rd  & 5 & \x & \x & \x & \x & \x & \x & \x & \x \\
        4th  & 6 & 3  & \x & \x & \x & \x & \x & \x & \x \\
        5th  & 6 & 4  & \x & \x & \x & \x & \x & \x & \x \\
        6th  & 6 & 5  & 3  & \x & \x & \x & \x & \x & \x \\
        7th  & 6 & 6  & 4  & \x & \x & \x & \x & \x & \x \\
        8th  & 6 & 6  & 5  & 3  & \x & \x & \x & \x & \x \\
        9th  & 6 & 6  & 6  & 4  & \x & \x & \x & \x & \x \\
        10th & 6 & 6  & 6  & 5  & 3  & \x & \x & \x & \x \\
        11th & 6 & 6  & 6  & 6  & 4  & \x & \x & \x & \x \\
        12th & 6 & 6  & 6  & 6  & 5  & 3  & \x & \x & \x \\
        13th & 6 & 6  & 6  & 6  & 6  & 4  & \x & \x & \x \\
        14th & 6 & 6  & 6  & 6  & 6  & 5  & 3  & \x & \x \\
        15th & 6 & 6  & 6  & 6  & 6  & 6  & 4  & \x & \x \\
        16th & 6 & 6  & 6  & 6  & 6  & 6  & 5  & 3  & \x \\
        17th & 6 & 6  & 6  & 6  & 6  & 6  & 6  & 4  & \x \\
        18th & 6 & 6  & 6  & 6  & 6  & 6  & 6  & 5  & 3  \\
        19th & 6 & 6  & 6  & 6  & 6  & 6  & 6  & 6  & 4  \\
        20th & 6 & 6  & 6  & 6  & 6  & 6  & 6  & 6  & 6  \\
    \end{dtabularx}
\end{dtable}

\begin{dtable}
    \lcaption{Cleric Spells Known}
    \centering
    \begin{dtabularx}{\columnwidth}{>{\ccol}X *{9}{>{\ccol}p{\spellcol}}}
        & \multicolumn{9}{c}{\tb{---{}---{}---{}---{}---{}---{}---{}---Spell Level---{}---{}---{}---{}---{}---{}---{}---}} \\
\hline
        \tb{Level} & \tb{1st} & \tb{2nd} & \tb{3rd} & \tb{4th} & \tb{5th} & \tb{6th} & \tb{7th} & \tb{8th} & \tb{9th} \\
        1st  & 1 & \x & \x & \x & \x & \x & \x & \x & \x \\
        2nd  & 2 & \x & \x & \x & \x & \x & \x & \x & \x \\
        3rd  & 3 & \x & \x & \x & \x & \x & \x & \x & \x \\
        4th  & 3 & 1 & \x & \x & \x & \x & \x & \x & \x \\
        5th  & 4 & 2 & \x & \x & \x & \x & \x & \x & \x \\
        6th  & 4 & 2 & 1 & \x & \x & \x & \x & \x & \x \\
        7th  & 4 & 3 & 2 & \x & \x & \x & \x & \x & \x \\
        8th  & 4 & 3 & 2 & 1 & \x & \x & \x & \x & \x \\
        9th  & 4 & 3 & 3 & 2 & \x & \x & \x & \x & \x \\
        10th & 4 & 3 & 3 & 2 & 1 & \x & \x & \x & \x \\
        11th & 4 & 3 & 3 & 3 & 2 & \x & \x & \x & \x \\
        12th & 4 & 3 & 3 & 3 & 2 & 1 & \x & \x & \x \\
        13th & 4 & 3 & 3 & 3 & 3 & 2 & \x & \x & \x \\
        14th & 4 & 3 & 3 & 3 & 3 & 2 & 1 & \x & \x \\
        15th & 4 & 3 & 3 & 3 & 3 & 3 & 2 & \x & \x \\
        16th & 4 & 3 & 3 & 3 & 3 & 3 & 2 & 1 & \x \\
        17th & 4 & 3 & 3 & 3 & 3 & 3 & 2 & 2 & \x \\
        18th & 4 & 3 & 3 & 3 & 3 & 3 & 2 & 2 & 1 \\
        19th & 4 & 3 & 3 & 3 & 3 & 3 & 2 & 2 & 2 \\
        20th & 4 & 3 & 3 & 3 & 3 & 3 & 2 & 2 & 2
    \end{dtabularx}
\end{dtable}

\begin{dtable!*}
    \lcaption{Deities}
    \begin{dtabularx}{\textwidth}{X l X}
        \tb{Deity} & \tb{Alignment} & \tb{Domains} \\
        \hline
        Guftas, horse god of justice & Lawful good & Good, Law, Strength, Travel \\
        Lucied, paladin god of justice & Lawful good & Destruction, Good, Protection, War \\
        Simor, fighter god of protection & Lawful good & Good, Protection, Strength, Vitality \\
        Rucks, monk god of pragmatism & Neutral good & Good, Law, Protection, Travel \\
        Vanya, centaur god of nature & Neutral good & Good, Strength, Travel, Wild \\
        Brushtwig, pixie god of creativity & Chaotic good & Chaos, Good, Trickery, Wild \\
        Chavi, god of stories & Chaotic good & Chaos, Knowledge, Trickery, Travel \\
        Ivan Ivanovitch, bear god of strength & Chaotic good & Chaos, Strength, War, Wild \\
        Krunch, barbarian god of destruction & Chaotic good & Destruction, Good, Strength, War \\
        Sir Cakes, dwarf god of freedom & Chaotic good & Chaos, Good, Strength, Vitality \\
        Raphael, monk god of retribution & Lawful neutral & Death, Law, Protection, Travel \\
%Atana, halfling god of ???
        Declan, god of fire & True neutral & Destruction, Fire, Knowledge, Magic \\
%Kalten, half-orc god of smiths
        Kurai, shaman god of nature & True neutral & Air, Earth, Fire, Water \\
        Clockwork, elf god of time & Chaotic neutral & Chaos, Magic, Trickery, Travel \\
        Murdoc, god of mercenaries & Chaotic neutral & Destruction, Knowledge, Travel, War\\
        Tak, orc god of war & Lawful evil & Law, Strength, Trickery, War \\
        Theodolus, sorcerer god of ambition & Neutral evil & Evil, Knowledge, Magic, Trickery \\
        Daeghul, demon god of slaughter & Chaotic evil & Destruction, Evil, Magic, War \\
%Ribo, halfling god of trickery & Chaotic neutral & Chaos, Trickery, Water
    \end{dtabularx}
\end{dtable!*}

\cf{Clr}{Divine Power}[Su]
The strength of a cleric's spells and abilities are determined by his divine power.
Normally, his divine power is equal to his cleric level.
A cleric's devotion can increase his divine power -- see the Devotion ability below.

\cf{Clr}{Devotion}[Su]
A cleric's divine power varies depending on the extent to which he calls on his devotion to create effects.
A cleric has a devotion pool with a maximum of eight devotion points in it.
If the cleric's devotion pool is at least three-quarters full, he gains a \plus2 bonus to his divine power.
This bonus increases by \plus1 at his 5th cleric level and every 5 cleric levels thereafter.
If the cleric's devotion pool has no points remaining, he takes a \minus5 penalty to his divine power (minimum 1).

After regaining spells for the day, a cleric's devotion pool is full.
Each time the cleric casts a non-domain spell, he loses one point from his devotion pool.
He also loses a point from his devotion pool if he acts against his deity or other spiritual inclinations.

The cleric can refill his devotion pool by spending an hour in prayer, supplication, or contemplation.
In addition, whenever the cleric performs a significant service to his deity or other spiritual inclinations, he may regain one devotion point.
Extraordinary services may allow the cleric to regain more devotion points.

\cf{Clr}[2nd]{Domain Invocation}[Su]
As a standard action, a cleric can spend a devotion point to invoke divine power.
He gains the domain invocations offered by one of his domains.
At 3rd level, he gains a domain invocation offered by another one of his domains.

All domain invocations affect a single creature within \rngmed range and require a special attack against a defense.
The cleric's accuracy with domain invocations is equal to his divine power.
If the attack succeeds, a domain invocation heals or inflicts 1d10 damage per two divine power.
If the attack fails, the invocation heals or inflicts half damage.

\cf{Clr}[4th]{Devotion Feat}[Su]
The cleric chooses a feat that he meets the prerequisites for.
As long as his devotion pool is at least half full, he gains that feat as a bonus feat.
At each level, the cleric may change this feat to a different feat he qualifies for.

At 14th level, the cleric gains a second devotion feat.
The cleric may not normally use these devotion feats as prerequisites for other feats or abilities.
However, he he may use one devotion feat to meet a prerequiste for his second devotion feat.

\cf{Clr}[6th]{Domain Aspect}[Su]
The cleric gains a domain aspect from one of his domains.
Domain aspects do not require an action to activate.
Options for domain aspects are listed at \pcref{Domain Aspects}.

At his 8th cleric level, the cleric gains an additional domain aspect from one of his domains.

\cf{Clr}[9th]{Expanded Devotion}[Su]
The cleric's devotion pool stores a maximum of twelve devotion points.

\cf{Clr}[12th]{Greater Domain Invocations}[Su]
The cleric gains the ability to invoke the power of one of his domains even more effectively.
This consumes two devotion points.
A cleric's accuracy when invoking a greater domain is equal to his divine power.
Greater domain invocations are described at \pcref{Greater Domain Invocations}.

At his 14th cleric level, the cleric gains an additional greater domain invocation from one of his domains.

\cf{Clr}[16th]{Domain Mastery}[Su]
The cleric gains a domain mastery from one of his domains.
Options for domain masteries are listed at \pcref{Domain Masteries}.

At his 18th cleric level, the cleric gains an additional domain mastery from one of his domains.

\cf{Clr}[19th]{Endless Devotion}[Su]
The cleric's devotion pool stores a maximum of twenty devotion points.

\subsection{Cleric Domain Abilities}

\subsubsection{Domain Gifts}\label{Domain Gifts}
% If a domain gift involves an attack, the accuracy is equal to the cleric's divine power.

\subcf{Air}
The cleric adds the Jump skill (see \pcref{Jump}) to his cleric class skill list, and gains a bonus equal to his divine power on Jump checks.
In addition, he does not take \glossterm{falling damage}.
\subcf{Chaos}
The cleric is immune to \glossterm{compulsion} effects.
In addition, he rolls twice for all \glossterm{random effects} and chooses his preferred result.
\subcf{Death}
The cleric halves all critical damage he takes (to a minimum of 1 damage).
\subcf{Destruction}
When making physical attacks, the cleric ignores an amount of hardness and damage reduction equal to his divine power.
\subcf{Earth}
The cleric gains a \plus10 bonus to Maneuver defense while standing on solid ground.
\subcf{Evil}
The cleric gains \glossterm{damage reduction} against physical damage from non-evil sources equal to half his divine power.
\subcf{Fire}
The cleric gains \glossterm{damage reduction} against fire and cold damage equal to twice his divine power.
\subcf{Good}
The cleric gains \glossterm{damage reduction} against physical damage from non-good sources equal to half his divine power.
\subcf{Knowledge}
The cleric adds all Knowledge skills to his cleric class skill list.
In addition, he gains four skill points which must be spent on Knowledge skills.
\subcf{Law}
The cleric is immune to \glossterm{compulsion} and \glossterm{delusion} effects.
\subcf{Magic}
The cleric may spend a devotion point in place of any spell slot to cast spells granted by this domain.
\subcf{Protection}
Allies adjacent to the cleric gain damage reduction against physical damage equal to half his divine power.
\subcf{Strength}
The cleric adds Athletics, Climb, Sprint, and Swim to his cleric class skill list.
In addition, he gains two skill points which must be spent on Strength-based skills.
\subcf{Travel}
The cleric adds Knowledge (geography), Sprint, and Survival to his cleric class skill list.
In addition, he gains a \plus10 foot bonus to his land speed.
\subcf{Trickery}
The cleric adds Bluff, Disguise, and Stealth to his cleric class skill list.
In addition, he may spend a devotion point as an immediate action to reroll any skill check from those three skills.
\subcf{Vitality}
The cleric gains a bonus equal to his divine power on Heal checks.
\subcf{War}
The cleric gains Weapon Focus with his deity's favored weapon group as a bonus feat.
If he does not have a deity, he gains it in a weapon group related to his alignment or ideals.
\subcf{Water}
The cleric adds Swim to his cleric class skill list, gains a bonus equal to half his divine power on Swim checks, and halves the penalties he takes for fighting underwater.
\subcf{Wild}
The cleric adds Creature Handling, Knowledge (nature), and Survival to his cleric class skill list.

\subsubsection{Domain Invocations}\label{Domain Invocations}

\subcf{Air -- Reflex}
The target takes elecricity damage.
\subcf{Chaos -- Mental}
This invocation randomly heals or inflicts damage.
The cleric chooses the target after rolling to determine the effect.
\subcf{Death -- Fortitude}
The target takes divine damage.
If this attack deals critical damage, the target is instantly killed.
This is a death effect.
\subcf{Destruction -- Fortitude}
The target takes sonic damage.
\subcf{Earth -- Reflex}
The target takes bludgeoning damage if it is on the ground.
\subcf{Evil -- Mental}
This invocation does not heal or inflict damage.
If the attack succeeds, and the target is good, it is \staggered for 5 rounds.
\subcf{Fire -- Reflex}
The target takes fire damage.
\subcf{Good -- Mental}
This invocation does not heal or inflict damage.
If the attack succeeds, and the target is evil, it is \dazed for 5 rounds.
\subcf{Knowledge -- Special}
The target must make a Knowledge check.
If its check result beats your attack result, it is healed.
Otherwise, it takes damage.
This invocation heals or inflicts 1d8 damage per two divine power instead of the normal value.
\subcf{Law -- Mental}
This invocation does not heal or inflict damage.
If the attack succeeds, and the target is chaotic, it is \immobilized for 5 rounds.
\subcf{Magic -- Mental}
If the target can cast spells, it is healed.
Otherwise, it takes divine damage.
This invocation heals or inflicts 1d8 damage per two divine power instead of the normal value.
\subcf{Protection -- Fortitude}
This invocation does not heal or inflict damage.
The target gains 1d10 temporary hit points per two divine power.
\subcf{Strength -- Special}
The target must make a Strength check.
If its check result beats your attack result, it is healed.
Otherwise, it takes divine damage.
This invocation heals or inflicts 1d8 damage per two divine power instead of the normal value.
\subcf{Travel -- Reflex}
The target is healed.
In addition, it gains a \plus10 foot bonus to its movement speed for 1 round.
This invocation heals 1d8 damage per two divine power instead of the normal value.
\subcf{Trickery -- Mental}
If the attack succeeds, the target is \disoriented for 1 round.
\subcf{Vitality -- Fortitude}
The target takes divine damage or is healed, as you choose.
\subcf{War -- Fortitude}
This invocation affects all creatures within a \areasmall radius of you instead of the normal target.
The targets take divine damage.
This invocation deals 1d6 damage per two divine power instead of the normal value.
\subcf{Water -- Fortitude}
The target takes nonlethal physical damage from water in its mouth and lungs.
In addition, if the attack succeeds, the target is unable to speak for 1 round.
\subcf{Wild -- Fortitude}
The target takes divine damage.
If the target is an animal or plant, the cleric may choose to heal it instead.

\subsubsection{Domain Aspects}\label{Domain Aspects}

\subcf{Air -- Glide}
The cleric gains a glide speed equal to his land speed.
See \pcref{Gliding}, for more details.
\subcf{Chaos -- Chaotic Retribution}
Whenever a lawful creature within 30 feet of you attacks you, make an attack against its Mental defense.
Success means it takes 1d6 damage per two divine power.
\subcf{Death -- Lifedrinker}
Whenever the cleric kills a creature with a death effect, he gains temporary hit points equal to his divine power for a number of rounds equal to the creature's level.
\subcf{Destruction -- Beacon of Destruction}
All enemies within a \areamed emanation of the cleric have their damage reduction and hardness (if any) reduced by an amount equal to half the cleric's divine power.
\subcf{Earth -- Tremorsense}
The cleric gains the tremorsense ability with a range of 50 feet.
If he is touching a surface, he can automatically pinpoint the location of anything within 50 feet that is in contact with the surface, including inanimate objects.
\subcf{Evil -- Unholy Retribution}
Whenever a good creature within 30 feet of you attacks you, make an attack against its Mental defense.
Success means it takes 1d6 damage per two divine power.
\subcf{Fire -- Friendly Fire}
All of the cleric's fire spells and abilities deal only half damage to his allies.
\subcf{Good -- Holy Retribution}
Whenever an evil creature within 30 feet of you attacks you, make an attack against its Mental defense.
Success means it takes 1d6 damage per two divine power.
\subcf{Knowledge -- Knowledge Mastery}
The cleric may choose a number of Knowledge skills equal to his Intelligence (minimum 1).
He may take 10 with those skills if he is not in danger or rushed.
\subcf{Law -- Certain Retribution}
Whenever an chaotic creature within 30 feet of you attacks you, make an attack against its Mental defense.
Success means it takes 1d6 damage per two divine power.
\subcf{Magic -- Magic Feat}
The cleric gains a bonus magic feat or metamagic feat.
\subcf{Protection -- Faithful Shield}
The cleric may maintain concentration on Shielding spells as a swift action.
\subcf{Strength -- Strength of Will}
The cleric may use his Strength in place of his cleric level to determine his divine power.
\subcf{Travel -- Rapid Traveller}
The cleric gains a \plus30 foot bonus to his speed in all movement modes, up to a maximum of double his original speed.
\subcf{Trickery -- Legendary Liar}
The cleric gains Legendary Liar as a bonus feat, even if he does not meet the prerequisites.
\subcf{Vitality -- Vital Spirit}
The cleric reduces his \glossterm{critical damage penalties} by an amount equal to his divine power.
\subcf{War -- Combat Feat}
The cleric gains a combat feat of his choice.
\subcf{Water -- Water Breathing} The cleric may breathe and speak normally while underwater, as the \spell{water breathing}
ritual.
He also takes no penalties to melee attacks underwater.
\subcf{Wild -- Favored Terrain}
The cleric gains a favored terrain, as the ranger class feature (see \pcref{Rgr:Favored Terrain}).

\subsubsection{Greater Domain Invocations}\label{Greater Domain Invocations}

\subcf{Air -- Mantle of Air}
As a swift action, the cleric can surround himself in a mantle of air for 5 rounds.
Thrown and projectile weapons have a 50\% chance to miss him while this effect is active.
Unusually large weapons, such as a giant's boulders, may suffer a decreased miss chance as appropriate to their size.
\subcf{Chaos -- Invoke Chaos}
This power functions like the Chaos channeled domain power, except that it randomly generates negative energy, positive energy, or both.
If both effects are generated, the cleric may exclude creatures separately from each effect.

\subcf{Death -- Invoke Death}
This power functions like the Death channeled domain power, except that any creature brought to 0 hit points by this effect immediately dies.
This is a death effect.
\subcf{Destruction -- Tide of Destruction}
The cleric channels destructive energy as the Destruction channeled domain power, except that any creature damaged by the effect is also filled with a destructive resonance for 5 rounds.
The first time each round that each subject takes damage, that damage is increased by half the cleric's level.
\subcf{Earth -- Mantle of Earth}
As a swift action, the cleric can surround himself in a mantle of earth for 5 rounds.
The mantle grants him physical damage reduction equal to his cleric level.
This allows him to ignore the first points of damage he would take each round.
If he is struck by an adamantine weapon, he cannot use his damage reduction for 1 round.
\subcf{Evil -- Invoke Evil}
The cleric channels negative energy, except that it heals evil creatures.

\subcf{Fire -- Mantle of Fire}
As a swift action, the cleric can surround himself in a mantle of fire for 5 rounds.
He gains the effect of a \spell{fire shield} spell, with a spellpower equal to his cleric level.
\subcf{Good -- Invoke Good}
The cleric channels positive energy, except that it deals divine damage to evil creatures.

\subcf{Knowledge -- See the Truth} As a swift action, the cleric can gain the benefit of the \spell{true seeing}
spell until the end of his turn.
\subcf{Law -- Invoke Law}
This power functions like the Law channeled domain power, except that the attack automatically succeeds against chaotic creatures.
\subcf{Magic -- Invoke Magic}
This power functions like the Magic channeled domain power, except that affected spellcasters receive a \plus5 bonus to spellpower on the next spell they cast.
If this bonus is not used within 5 rounds, it is wasted.

\subcf{Protection -- Invoke Sanctuary} This power functions like the Protection channeled domain power, except that each subject also receives the benefit of a \spell{sanctuary}
spell for 5 rounds.
If a subject attacks, the \spell{sanctuary} is broken for that creature, but not for any other subject.

\subcf{Strength -- Invoke Strength}
As a swift action, the cleric can add his cleric level as an bonus to his Strength until the end of his turn.
\subcf{Travel -- Invoke Speed}
As a swift action, the cleric can double his movement speed with all forms of movement until the end of his turn.
In addition, he does not treat squares threatened by blocking creatures as difficult terrain.
\subcf{Trickery -- Swift Invisibility} As a swift action, the cleric can gain the benefit of the \spell{invisibility}
spell until the end of his turn.
\subcf{Vitality -- }
\subcf{War -- Warmaster's Boon}
The cleric can use this power as part of casting a spell that affects a single creature other than himself.
The spell also affects the cleric.
This lasts for the normal duration of the spell or for a number of rounds equal to the cleric's domain attribute, whichever is shorter.
\subcf{Water -- Aquatic Globe}
The cleric creates water out of thin air in an immobile \areamed radius spread centered on his original location for 5 rounds.
Everything within the area is underwater.
After 1 round, the sphere grows to fill an \arealarge radius spread.
At the end of the duration, the water evaporates, leaving no trace that it was ever there.
\subcf{Wild -- Wild Aspect}
When the cleric gains this ability, he chooses one wild aspect ability, as if he were a were a druid of a level equal to his cleric level (see Wild Aspect, \pref{Drd:Wild Aspect}).
When he uses this ability, he may embody that wild aspect.
This effect lasts as long as that wild aspect would normally last.

\subsubsection{Domain Masteries}\label{Domain Masteries}

\subcf{Air -- Flight}
The cleric gains a fly speed (good maneuverability) equal to his land speed.
He may remain flying for up to 5 rounds at a time.
After that, he must land for 1 round before he can fly again.
See \pcref{Flying}, for more details.
\subcf{Chaos -- Avatar of Luck}
Once per round, the cleric can gain a \plus1d6 bonus to any check or physical attack.
He may declare the use of the ability after failing the roll, but before any additional effects are resolved, potentially making it succeed where it would have failed.
\subcf{Death -- Deathfeeder}
The cleric constantly radiates a \areamed radius emanation of death.
Whenever a creature dies within the area, the cleric gains the benefits of the \spell{death knell} spell as if it had been cast on the creature.
\subcf{Destruction -- Ruinbringer}
The cleric's attacks and spells ignore all damage reduction and hardness (but not damage immunity).
\subcf{Earth -- Earth Glide}
The cleric gains the earth glide ability, as an earth elemental.
\subcf{Evil -- Avatar of Evil} The cleric continuously gains the benefits of the \spell{protection from good}
spell, with a spellpower equal to his cleric level.
If the effect is dispelled or suppressed, he can resume it as a swift action.
\subcf{Fire -- Flaming Soul}
The cleric gains the fire subtype, making him immune to fire but giving him a 50\% vulnerability to cold damage.
In addition, whenever he deals fire damage to a creature, the creature is \ignited for 5 rounds.
\subcf{Good -- Avatar of Good} The cleric continuously gains the benefits of the \spell{protection from evil}
spell, with a spellpower equal to his cleric level.
If the effect is dispelled or suppressed, he can resume it as a swift action.
\subcf{Knowledge -- Combat Insight}
The cleric gains a \plus2 bonus to accuracy, checks, and special defenses against non-humanoid creatures he has identified with a successful Knowledge check.
\subcf{Law -- Avatar of Order}
Once per round, if the cleric rolls less than a 10 on a d20, he may treat the result as if it were a 10, potentially causing him to succeed where he would have failed.
You must declare the use of this ability before any additional effects from the roll are resolved.
\subcf{Magic -- Spellfeeder}
The cleric gains spell resistance equal to 10 \add cleric level or Intelligence.
To affect the cleric with a spell, a caster must make an attack with its spellpower.
If the attack beats the cleric's spell resistance, the spell works normally.
Otherwise, the spell has no effect on the cleric.

In addition, whenever the cleric resists a spell with his spell resistance, he regains a spell slot of a level up to one lower than the level of the resisted spell.
\subcf{Protection -- Martyr's Gift}
The cleric constantly radiates a \areamed radius emanation of protective energy.
Whenever a creature within the area takes damage, the cleric can choose to take half of that damage instead, as the \spell{share pain} spell.
\subcf{Strength -- Might of the Gods}
The cleric gains the larger than life ability, as the barbarian class feature (see Larger than Life, \pref{Bbn:Larger than Life}).
\subcf{Travel -- Perfect Stride}
The cleric gains perfect stride, as the ranger class feature.
He constantly acts as if he were under the effect of a \spell{freedom} spell, except that it does not allow him to act normally underwater.
\subcf{Trickery -- Exemplar of Deceit} The cleric continuously gains the benefits of the \spell{nondetection}
spell, with a spellpower equal to his cleric level, except that it also protects him from any spells or effects which would prevent him from lying or reveal his lies.
If the effect is dispelled or suppressed, he can resume it as a swift action.
\subcf{Vitality -- }
\subcf{War -- Warmaster's Favor} The cleric continuously gains the benefits of the \spell{greater divine favor}
spell, with a spellpower equal to his cleric level.
If the effect is dispelled or suppressed, he can resume it as a swift action.
\subcf{Water -- Water's Flow}
As a swift action, the cleric can transform himself into a rushing flow of water with a volume roughly equal to his normal volume until the end of his turn.
In this form, he may move wherever water could go, but he cannot take other actions, such as jumping, attacking, or casting spells.
His speed is halved when moving uphill and doubled when moving downhill.
He may move through squares occupied by creatures or threatened by blocking enemies without penalty.
He may return to his normal form as a free action.
\par If the water is split, he may reform from anywhere the water has reached, to as little as a single ounce of water.
If not even an ounce of water exists contiguously, his body reforms from the largest available parts of water, cut into pieces of appropriate size.
This usually causes the cleric to die.
\subcf{Wild -- Natural Casting}
Whenever the cleric is in a natural environment, he gains the natural casting ability, as the druid class feature (see Natural Casting, \pref{Drd:Natural Casting}).
He can use this ability at up to Close range.

\subsubsection{Ex-Clerics}
A cleric who grossly violates the code of conduct required by his god loses all spells and supernatural cleric class features.
He cannot thereafter gain levels as a cleric of that god until he atones (see the \spell{atonement} spell description).

\section{Druid}
\begin{dtable}
    \lcaption{Druid Progression}
    \begin{dtabularx}{\columnwidth}{>{\ccol}p{\levelcol} >{\centering}p{\babcolavg} *{3}{>{\ccol}p{\savecol}} >{\ccol}X}
        \tb{Level} & \tb{Combat Prowess} & \tb{Fort} & \tb{Ref} & \tb{Ment} & \tb{Special} \\
        \hline
        \druidprogressionrow{1}  & Attunement \plus2, natural casting, rituals, spells \\
        \druidprogressionrow{2}  & Wild aspect                             \\
        \druidprogressionrow{3}  & Wild speech                             \\
        \druidprogressionrow{4}  & Wild aspect                             \\
        \druidprogressionrow{5}  & Attunement \plus3, wild speech (plants) \\
        \druidprogressionrow{6}  & Wild aspect (x2)                        \\
        \druidprogressionrow{7}  & Charming wild speech                    \\
        \druidprogressionrow{8}  & Wild aspect                             \\
        \druidprogressionrow{9}  & Natural casting (Close)                 \\
        \druidprogressionrow{10} & Attunement \plus4                       \\
        \druidprogressionrow{11} & Wild speech (elements)                  \\
        \druidprogressionrow{12} & Natural aspect (x3)                     \\
        \druidprogressionrow{13} & A thousand faces                        \\
        \druidprogressionrow{14} & Natural aspect, timeless body           \\
        \druidprogressionrow{15} & Attunement \plus5                       \\
        \druidprogressionrow{16} & Natural aspect                          \\
        \druidprogressionrow{17} & Dominating wild speech                  \\
        \druidprogressionrow{18} & Natural aspect (x4)                     \\
        \druidprogressionrow{19} & Natural casting (Medium)                \\
        \druidprogressionrow{20} & Attunement \plus6, natural aspect       \\
    \end{dtabularx}
\end{dtable}

\classbasics{Alignment} Neutral good, lawful neutral, neutral, chaotic
neutral, or neutral evil.

\classbasics{Class Skills}
\subparhead{Strength} Climb, Jump, Sprint, Swim.
\subparhead{Dexterity} Balance, Ride, Stealth.
\subparhead{Intelligence} Heal, Knowledge (geography, nature).
\subparhead{Perception} Awareness, Creature Handling, Survival.
\subparhead{Other} Intimidate.

\subsection{Base Class Features}
A character with druid as a base class gains the following abilities.

\classbasics{Skill Points} 10.

\classbasics{Defenses} \plus4 Fortitude, \plus2 Mental.

\cf{Drd}{Weapon and Armor Proficiency}
Druids are proficient with simple weapons, any one other weapon group, scimitars, sickles, and slings.
In addition, druids are proficient with light armor, medium armor and shields.
However, a druid cannot use metal armor; see the Metal Abhorrence ability, below.

\cf{Drd}{Natural Casting}[Ex]
Whenever the druid casts a nature spell with an area that originate from her, such as most cone or line spells, she may cause the spell to originate from any location within 10 feet of her.
All other aspects of the spell are unchanged.

For example, a druid casting \spell{burning hands} could create a cone of fire originating from 10 feet to her right.
The cone would extend 20 feet out from that point, as normal.
If the druid directed the cone back towards her, she could potentially be affected by the spell.

At 9th level, this ability's range improves to \rngclose.
At 19th level, this ability's range improves to \rngmed.

\cf{Drd}{Rituals}
Druids can perform rituals to create unique magical effects (see \pcref{Rituals}).
A druid begins play with a ritual book containing one nature ritual of her choice (see \pcref{Nature Rituals}).

\cf{Drd}{Druidic Language}
Druids know Druidic, a secret language known only to druids, in addition to their normal languages.
Druids are forbidden to teach this language to nondruids.
Druidic has its own alphabet.

\subsection{Class Features}
All druids have the following abilities.

\cf{Drd}{Metal Abhorrence}
The oaths that druids swear as part of their initiation prohibit them from wearing armor made of metal.
A druid who wears prohibited armor or carries a prohibited shield is unable to cast druid spells or use any of her supernatural class abilities while doing so and for 24 hours thereafter.
(A druid may also wear wooden armor that has been altered by the \spell{ironwood} ritual so that it functions as though it were steel. See the ritual description.)

\cf{Drd}{Spells}
A druid casts nature spells using her attunement with nature.
A druid's spellpower is normally equal to her nature power.
See the Nature Power and Attunement abilities, below.

At 1st level, the druid knows one 1st level spell.
At every level theareafter, she learns one additional spell.
The spell can be of any level, up to a maximum of half the druid's class level.
A druid's spells are drawn from the spells on the nature spell list (see \pcref{Nature Spells}).

Druids have a limit on the number of spells they can cast, as given on \trefnp{Druid Spell Slots}.
Attuning to a natural environment for an hour, as described above, restores all spell slots the druid has expended.

\begin{dtable}
    \lcaption{Druid Spell Slots}
    \centering
    \begin{dtabularx}{\columnwidth}{>{\ccol}X *{9}{>{\ccol}p{\spellcol}}}
        & \multicolumn{9}{c}{\tb{---{}---{}---{}---{}---{}---{}---{}---Spell Level---{}---{}---{}---{}---{}---{}---{}---}} \\
        \hline
        \tb{Level} & \tb{1st} & \tb{2nd} & \tb{3rd} & \tb{4th} & \tb{5th} & \tb{6th} & \tb{7th} & \tb{8th} & \tb{9th} \\
        1st  & 1 & \x & \x & \x & \x & \x & \x & \x & \x \\
        2nd  & 2 & \x & \x & \x & \x & \x & \x & \x & \x \\
        3rd  & 3 & \x & \x & \x & \x & \x & \x & \x & \x \\
        4th  & 3 & 1  & \x & \x & \x & \x & \x & \x & \x \\
        5th  & 3 & 2  & \x & \x & \x & \x & \x & \x & \x \\
        6th  & 3 & 3  & 1  & \x & \x & \x & \x & \x & \x \\
        7th  & 3 & 3  & 2  & \x & \x & \x & \x & \x & \x \\
        8th  & 3 & 3  & 3  & 1  & \x & \x & \x & \x & \x \\
        9th  & 3 & 3  & 3  & 2  & \x & \x & \x & \x & \x \\
        10th & 3 & 3  & 3  & 3  & 1  & \x & \x & \x & \x \\
        11th & 3 & 3  & 3  & 3  & 2  & \x & \x & \x & \x \\
        12th & 3 & 3  & 3  & 3  & 3  & 1  & \x & \x & \x \\
        13th & 3 & 3  & 3  & 3  & 3  & 2  & \x & \x & \x \\
        14th & 3 & 3  & 3  & 3  & 3  & 3  & 1  & \x & \x \\
        15th & 3 & 3  & 3  & 3  & 3  & 3  & 2  & \x & \x \\
        16th & 3 & 3  & 3  & 3  & 3  & 3  & 3  & 1  & \x \\
        17th & 3 & 3  & 3  & 3  & 3  & 3  & 3  & 2  & \x \\
        18th & 3 & 3  & 3  & 3  & 3  & 3  & 3  & 3  & 1  \\
        19th & 3 & 3  & 3  & 3  & 3  & 3  & 3  & 3  & 3  \\
        20th & 3 & 3  & 3  & 3  & 3  & 3  & 3  & 3  & 3  \\
    \end{dtabularx}
\end{dtable}

\cf{Drd}{Nature Power}[Su]
The strength of a druid's spells and abilities are determined by her connection to nature.
Normally, her nature power is equal to her druid level.
Attuning to nature can increase a druid's nature power -- see the Attunement ability below.

\cf{Drd}{Attunement}[Su]
Whenever the druid is in a natural environment, she may dismiss all her active spells and spend one hour attuning to the natural world.
If she does so, she regains all her spell slots.
In addition, she gains a \plus2 bonus to her spellpower for as long as she stays in a natural environment, and for five minutes after leaving a natural environment. This bonus increases by \plus1 at her 5th druid level and every 5 druid levels thereafter.

\cf{Drd}[2nd]{Wild Speech}[Su]
The druid learns how to communicate with animals.
As a standard action, the druid can choose a type of animal, such as owl or wolf.
She gains the ability to speak to and understand animals of that type for 5 minutes.
A druid can use this ability a number of times per day equal to her Perception or half her druid level, whichever is higher.

This ability doesn't make the animals any more friendly or cooperative than normal.
Furthermore, wary and cunning animals are likely to be terse and evasive, while the more stupid ones make inane comments.
If an animal is friendly toward the druid, she may be able to convince it to do some favor or service.

\cf{Drd}[2nd]{Wild Aspect}[Su]
The druid gains the ability to embody an aspect of an animal.
She chooses one wild aspect from the list below.
Some wild aspects have minimum druid levels, as indicated in the title of the aspect.
At her 4th druid level, and every even druid level thereafter, the druid gains an additional wild aspect.

Unless otherwise noted, embodying a wild aspect is a standard action.
Embodying a wild aspect costs a spell slot of any level.
Once embodied, a wild aspect persists until the druid embodies a new aspect or dismisses the aspect.
If the druid embodies a new wild aspect, the previous aspect ends immediately.
All wild aspects can be dismissed as a swift action.

All wild aspects are supernatural abilities unless otherwise noted.

The descriptions below describe the effects of the aspect.
With many aspects, the druid's appearance also changes to match the aspect, but this is not described.
Different druids change in different ways.
For example, one druid might gain unusually large eyes when embodying the low-light vision aspect, while another might change her irises into slits, like a cat, when embodying the same aspect.
The changes made are up to the druid, but cannot be used to gain an additional substantive benefit beyond the effects given in the description of the aspect.

At 6th level, and every 4 levels thereafter, the druid gains the ability to embody an additional wild aspect at the same time.

\subcf{Animal Affinity}
The druid gains a \plus2 bonus to Creature Handling and Ride checks.
\subcf{Armaments of the Bear}
The druid's mouth and hands transform, allowing her to perform bite and claw attacks.
The bite attack deals 1d8 damage for a Medium druid, and the claws deal 1d6 damage.
(See \pcref{Natural Weapons}, for details about natural weapons.)
\subcf{Constrict}
The druid's body transforms, allowing her to perform a constrict attack.
The attack deals 1d10 damage for a Medium druid, but it can only be used against a foe she is grappling with.
(See \pcref{Natural Weapons}, for details about natural weapons).
\subcf{Gore}
The druid's head transforms, allowing it to perform a gore attack.
The attack deals 1d8 damage for a Medium druid.
(See \pcref{Natural Weapons}, for details about natural weapons.)
\subcf{Lope}
The druid gains the ability to move on all four limbs.
When doing so, she increases her speed by 20 feet, but she cannot use her hands for anything except moving.
When not moving on four legs, her ability to use her hands is unchanged.
\subcf{Low-light Vision}
The druid gains low-light vision.
She treats sources of light as if they had double their normal illumination range.
\subcf{Talons}
The druid's feet transform, allowing her to perform a talon attack.
The attack deals 1d6 damage for a Medium druid.
(See \pcref{Natural Weapons}, for details about natural weapons.)
\subcf{Woodland Stride}
The druid may move through any sort of undergrowth (such as natural thorns, briars, overgrown areas, and similar terrain) at her normal speed and without taking damage or suffering any other impairment.
However, plants magically manipulated to impede motion still affect her.
\subcf{4th -- Bear's Endurance}
The druid gains a \plus2 bonus to Constitution.
This cannot increase her Constitution above her nature power.
\subcf{4th -- Bull's Strength}
The druid gains a \plus2 bonus to Strength.
This cannot increase her Strength above her nature power.
\subcf{4th -- Cat's Grace}
The druid gains a \plus2 bonus to Dexterity.
This cannot increase her Dexterity above her nature power.
\subcf{4th -- Fox's Cunning}
The druid gains a \plus2 bonus to Intelligence.
This cannot increase her Intelligence above her nature power.
\subcf{4th -- Mule's Tenacity}
The druid gains a \plus2 bonus to Willpower.
This cannot increase her Willpower above her nature power.
\subcf{4th -- Owl's Insight}
The druid gains a \plus2 bonus to Perception.
This cannot increase her Perception above her nature power.
\subcf{6th -- Climb}
The druid gains a climb speed equal to her base land speed.
\subcf{6th -- Darkvision}
The druid gains \glossterm{darkvision} out to 50 feet, allowing her to see in complete darkness.
If she already has darkvision, she increases its range by 50 feet.
\subcf{6th -- Enhanced Natural Weapons}
The druid's natural weapons gain a \glossterm{enhancement bonus} equal to one quarter of her nature power.
This functions like an enhancement bonus on a weapon, increasing her damage and offensive legend points per day (see \pcref{Weapon Enhancement Bonuses}).
\subcf{6th -- Glide}
The druid grows wings, granting her a glide speed equal to her land speed.
See \pcref{Gliding}, for more details.
\subcf{6th -- Shrink}
The druid shrinks by a size category.
This functions like the \spell{reduce person} spell.
This is a sizing effect.
\subcf{6th -- Slither}
The druid gains a climb speed equal to half her base land speed.
She does not need to use her hands to climb in this way.
See \pcref{Climbing}, for more details.
\subcf{6th -- Venom Immunity}
The druid gains immunity to all poisons.
\subcf{8th -- Natural Grab}
If the druid hits with a natural attack, she may attempt to grapple her foe as an immediate action.
\subcf{8th -- Natural Knockback}
If the druid hits with a natural attack, she may attempt to shove her foe as an immediate action.
She cannot move with the struck creature to push it back farther.
\subcf{8th -- Natural Trip}
If the druid hits with a natural attack, she may attempt to trip her foe as an immediate action.
See \pcref{Trip}, for more details.
\subcf{8th -- Scent}
The druid gains the scent ability, granting her a \plus10 bonus to scent-based Awareness checks (see \pcref{Awareness}).
\subcf{8th -- Wolfpack}
Overwhelmed foes the druid threatens increase their overwhelm penalties by 1.
\subcf{10th -- Grow}
The druid increases in size by one size category.
This functions as the \spell{enlarge person} spell.
This is a sizing effect.
\subcf{10th -- Limited Flight}
The druid grows wings, granting her a fly speed equal to her land speed with average maneuverability.
See \pcref{Flying}, for more details.
She can only fly for a number of rounds equal to 3 \add half her Constitution.
After that limit is reached, she must rest for 5 minutes.
\subcf{10th -- Swiftstrike}
The druid's attack speed increases.
When she makes a standard attack, she may make an additional strike.
This strike must be made with a natural weapon.
This effect does not stack with similar effects that grant extra strikes.
\subcf{10th -- Venom}
When the druid embodies this aspect, she transforms one of her natural weapons to become poisonous.
If she hits with that natural attack, she may inject poison into the struck creature as an immediate action.
At the end of each round, the druid makes a special attack vs. Fortitude against all creatures she has poisoned with this ability.
If the attack succeeds, the creature takes 1d6 physical damage per two druid levels and is \sickened for 5 rounds.
The poison lasts until the druid fails the attack twice.
%TODO poison stacking rules

\cf{Drd}[6th]{Wild Speech (Plants)}[Ex]
The druid can also converse with plants and plant creatures using her wild speech ability.
A regular plant's sense of its surroundings is limited, so it won't be able to give (or recognize) detailed descriptions of creatures or answer questions about events outside its immediate vicinity.

\cf{Drd}[7th]{Charming Wild Speech}[Su]
As a standard action that consumes a use of her wild speech ability, the druid can attempt to charm a creature she is speaking with using her wild speech ability.
If she succeeds at a special attack vs. Mental defense, the target is charmed.
The effect lasts for the duration of the conversation, and for 1 hour thereafter.
This ability is not mind-affecting, and can affect creatures or even objects of any kind that the druid can use her wild speech to converse with.
The attack automatically succeeds against non-intelligent objects.
The druid's accuracy with this ability is equal to her nature power.

A charmed creature or object regards the druid as a trusted friend and ally.
The druid cannot control the subject as if it were an automaton, but it perceives her words and actions in the most favorable way.
She can try to give the subject orders, but she must succeed at a Persuasion check to convince it to do anything it wouldn't ordinarily do.
(Retries are not allowed.) Treat the subject as a friend (a \plus10 relationship modifier) for the purpose of the Persuasion check.
An affected creature never obeys suicidal or obviously harmful orders, but it might be convinced that something very dangerous is worth doing.

\cf{Drd}[11th]{Wild Speech (Elements)}[Su]
The druid gains the ability to speak with one of the elements that make up the natural world with her wild speech ability.
When she gains this ability, she chooses whether she can speak with natural air, earth, fire, or water.
That choice cannot thereafter be changed.

\cf{Drd}[12th]{Natural Aspect}[Su]
The druid gains the ability to embody aspects of the natural world, including the elements, in addition to those of animals.
She adds the options below to the list of abilities she can gain with her wild aspect ability.

\begin{comment}

%\subcf{Aerial Combat} The druid gains a \plus2 bonus to physical attacks while either she or her target is airborne.
%\subcf{Aquatic Combat} The druid gains a \plus2 bonus to physical attacks while both she and her target are touching naturally occuring water (including rain, but not fog or ice).

%\subcf{Earthen Combat} The druid gains a \plus2 bonus to physical attacks with melee attacks while both she and her target are touching natural earth or stone.
%\subcf{Fiery Combat} The druid gains a \plus2 bonus to physical attacks while she or her target is ignited.
%\subcf{Earthen Might} The druid gains a \plus2 bonus to caster level while she is touching natural earth or stone.
\subcf{Heart of Air} The druid can breathe in any environment, and is immune to \spell{sickening cloud}
and similar effects.
In addition, she falls at half speed and takes no falling damage.
\subcf{Heart of Earth} The druid gains the effects of the \spell{stoneskin}
spell, with a spellpower equal to her druid level or Constitution.
\subcf{Heart of Fire} The druid gains the effects of the warm version of the \spell{fire shield}
spell, with a spellpower equal to her druid level or Constitution.
\subcf{Heart of the Sun}
The druid constantly radiates bright light out to a 100 foot radius (and shadowy illumination for an additional 100 feet).
The illumination is so bright that she becomes hard to look at.
Any creature attacking her from within the radius of bright light becomes dazzled for 5 rounds after the attack.
\subcf{Heart of Oak} The druid gains the effects of the \spell{stoneskin}
spell, with a spellpower equal to her druid level or Constitution, except that the damage reduction is overcome by fire instead of adamantine weapons.
\subcf{Heart of Water} The druid gains the effects of the \spell{freedom}
spell.
\subcf{15th -- Air Mantle}
 The druid is surrounded by a mantle of air.
Thrown and projectile weapons have a 50\% chance to miss her while this effect is active.
Unusually large weapons, such as a giant's boulders, may suffer a decreased miss chance as appropriate to their size.
\subcf{15th -- Aqueous Step}
 Wherever the druid moves, she leaves a path of animated water that can grab creatures.
Whenever a creature crosses the path, the druid makes a Reflex attack to trip the creature, causing it to fall prone and waste the rest of its movement.
Her accuracy is equal to her druid level \add her Constitution.
\subcf{15th -- Flaming Step}
Wherever the druid moves, she leaves a path of burning flame behind her that lasts for 1 round.
Whenever a creature crosses the path, the druid makes a Reflex attack to deal damage to the creature.
The attack deals 1d6 points of fire damage per two druid levels.
Her accuracy is equal to her druid level \add her Constitution.
A failed attack deals half damage.
\subcf{15th -- Lifegiving Step}
Wherever the druid moves, she leaves a path of small, living plants that entangle foes for 1 round.
Whenever a creature crosses the path, the druid makes a Reflex attack to entangle the creature, causing it to waste the rest of its movement.
Her accuracy is equal to her druid level \add her Constitution.
The plants appear on any surface, and will continue to grow if they can survive, though they may die quickly if they appear on inhospitable terrain.
\subcf{17th -- Flight}
The druid gains a fly speed equal to her land speed, with good maneuverability.
She may remain flying for up to 5 rounds at a time.
After that, she must land for 1 round before she can fly again.
See \pcref{Flying}, for more details.
\subcf{17th -- Flaming Soul}
The druid gains the fire subtype, making her immune to fire but giving her a 50\% vulnerability to cold damage.
In addition, whenever she deals fire damage to a creature, the creature is \ignited for 5 rounds.
\subcf{17th -- Sunblessed Rejuvenation}
The druid gains fast healing equal to her druid level as long as she remains in sunlight or touches a plant of her size or larger.
\subcf{17th -- Sunscour}
This aspect functions like the heart of the sun natural aspect, except that it also suppresses shadow effects and the visual components of illusions within the area of bright light.

\subcf{17th -- Water's Flow}
As a swift action, the druid can transform herself into a rushing flow of water with a volume roughly equal to her normal volume until the end of her turn.
In this form, she may move wherever water could go, but she cannot take other actions, such as jumping, attacking, or casting spells.
Her speed is halved when moving uphill and doubled when moving downhill.
She may move through squares occupied by creatures or threatened by blocking enemies without penalty.
She may return to her normal form as a free action.
\par If the water is split, she may reform from anywhere the water has reached, to as little as a single ounce of water.
If not even an ounce of water exists contiguously, her body reforms from the largest available parts of water, cut into pieces of appropriate size.
This usually causes the druid to die.

\end{comment}

\cf{Drd}[13th]{A Thousand Faces}[Su] The druid gains the ability to change her appearance at will, as if using the \spell{disguise self}
spell.
This affects the druid's body, but not her possessions.
It is not an illusory effect, but a minor physical alteration of the druid's appearance, within the limits described for the spell.

\cf{Drd}[14th]{Timeless Body}[Ex]
The druid no longer takes attribute penalties for aging and cannot be magically aged.
Any penalties she may have already incurred, however, remain in place.
Bonuses still accrue, and the druid still dies of old age when her time is up.

\cf{Drd}[17th]{Dominating Wild Speech}[Ex]
When the druid uses her charming wild speech ability, she can dominate the subject (as the \spell{dominate person}
spell) instead of charming it.
This ability is not mind-affecting, and can affect creatures of any kind that the druid can converse with.
The attack automatically succeeds against non-intelligent objects.

\subsection{Ex-Druids}
A druid who ceases to revere nature, changes to a prohibited alignment, or teaches the Druidic language to a nondruid loses all spells and supernatural druid class features.
She cannot thereafter gain levels as a druid until she atones (see the \spell{atonement} ritual).

\subsection{Variant Druids}

\subsubsection{Blighter}

Blighters draw power from nature, as do other druids. However, while other druids revere nature and draw power from it gently, blighters steal power from nature forcefully. Wherever a blighter goes, destruction and death surely follows.

\altcf{Attunement} As normal, except that when a blighter attunes to a natural environment, the terrain within a 100 foot radius is blighted.
This has several effects, all of which slowly take place over the course of the attunement process.
Every living thing in the area other than the blighter takes damage equal to the blighter's level once per ten minutes.
All inanimate plants of Huge size or smaller wither and die.
The earth becomes cracked and infertile, and any nutrients from the soil are destroyed.
This ability has no effect on artificial environments or materials, such as metal or worked stone.

In addition, the blighter's bonus to spellpower from attuning lasts for 1 hour, regardless of whether she is in a natural environment.

\altcf{Spells} As normal, except that a blighter adds all general Vivimancy arcane spells to her spell list.

\altcf[2nd]{Wild Speech} As normal, except that a blighter gains a \plus5 bonus to Intimidate against her wild speech targets, and a \minus5 penalty to Persuasion.

\altcf{Natural Casting} The blighter does not gain this ability.

\altcf[10th]{Blightcasting}

\altcf[20th]{Improved Blightcasting}

\subsubsection{Rotbringer}

While most druids seek to emulate and interact with animals, rotbringers focus on the power of fungi, decay, and regeneration.

\altcf{Attunement} As normal, except that when a rotbringer attunes to a natural environment, the terrain within a 100 foot radius decomposes.
This has several effects, all of which slowly take place over the course of the attunement process.
All organic objects of Huge size or smaller, such as plants and corpses, decompose.
This decomposition kills living plants.
All organic objects, regardless of size, are covered with various fungi.
This ability has no effect on artificial environments or materials, such as metal or worked stone.

If the rotbringer decomposes a Huge object while attuning, or a combination of smaller objects equivalent in size to a Huge object, she gains an bonus spell slot of her highest available spell level.
This extra spell slot lasts until it is used, or until the druid attunes again.

\altcf[2nd]{Wild Speech} The rotbringer gains the ability to speak with plants at 2nd level.
She gains the ability to speak with animals at 6th level, instead of at 2nd level.

\altcf[3rd]{Wild Aspect} The rotbringer does not gain this ability.

\altcf[3rd]{Rot Spell} The druid learns an additional spell slot and spell known.
The spell must be taken from the following list of spells.
The spell's level cannot exceed half her druid level.
If she already knows a spell from the list at every spell level she has access to, she may instead learn any nature spell (see \pcref{Nature Spells}).

At 5th level, and every odd level, the druid may learn a new spell.

\begin{dtable}
    \begin{dtabularx}{\columnwidth}{l X}
        \tb{Spell level} & \tb{Rotbringer Spells} \\
        1st & \spell{excrete slime}, \spell{lesser regeneration} \\
        2nd & \spell{fungal growth} \\
        3rd & \spell{rotburst} \\
        4th & \spell{poison} \\
        6th & \spell{regeneration} \\
        7th & \spell{greater rotburst} \\
    \end{dtabularx}
\end{dtable}

\altcf[7th]{Fungal Armor} The rotbringer becomes covered in fungus that protects her from attacks. She gains a \plus1 bonus to Armor and Fortitude defense.

This bonus increases by 1 at her 7th druid level, and every 4 druid levels thereafter.

\section{Fighter}
\begin{dtable}
    \lcaption{Fighter Progression}
    \begin{dtabularx}{\columnwidth}{>{\ccol}p{\levelcol} >{\ccol}p{\babcolgood} *{3}{>{\ccol}p{\savecol}} >{\lcol}X}
        \tb{Level} & \tb{Combat Prowess} & \tb{Fort} & \tb{Ref} & \tb{Ment} & \tb{Special} \\
        \hline
        \fighterprogressionrow{1}   & Advanced training, armor discipline   \\
        \fighterprogressionrow{2}   & Bonus feat                           \\
        \fighterprogressionrow{3}   & Weapon discipline                    \\
        \fighterprogressionrow{4}   & Adaptive combat feat                 \\
        \fighterprogressionrow{5}   & Combat discipline                    \\
        \fighterprogressionrow{6}   & Bonus feat                           \\
        \fighterprogressionrow{7}   & Improved armor discipline            \\
        \fighterprogressionrow{8}   & Adaptive combat feat                 \\
        \fighterprogressionrow{9}   & Improved weapon discipline           \\
        \fighterprogressionrow{10}  & Battlemaster, bonus feat             \\
        \fighterprogressionrow{11}  & Improved combat discipline           \\
        \fighterprogressionrow{12}  & Adaptive combat feat                 \\
        \fighterprogressionrow{13}  & Greater armor discipline             \\
        \fighterprogressionrow{14}  & Bonus feat, improved adaptive combat \\
        \fighterprogressionrow{15}  & Greater weapon discipline            \\
        \fighterprogressionrow{16}  & Adaptive combat feat                 \\
        \fighterprogressionrow{17}  & Greater combat discipline            \\
        \fighterprogressionrow{18}  & Bonus feat, improved battlemaster    \\
        \fighterprogressionrow{19}  & True discipline                      \\
        \fighterprogressionrow{20}  & Adaptive combat feat,
        greater adaptive combat    \\
    \end{dtabularx}
\end{dtable}

\classbasics{Alignment} Any.

\classbasics{Class Skills}
\subparhead{Strength} Climb, Jump, Sprint, Swim.
\subparhead{Dexterity} Balance, Escape Artist, Ride, Tumble.
\subparhead{Perception} Awareness.
\subparhead{Other} Bluff, Intimidate, Persuasion.

\subsection{Base Class Features}
A character with fighter as a base class gains the following abilities.

\classbasics{Skill Points} 10.

\classbasics{Defenses} \plus4 Fortitude, \plus2 Mental.

\cf{Ftr}{Weapon and Armor Proficiency}
 A fighter is proficient with simple weapons, any four other weapon groups,  all armor (heavy, medium, and light) and shields.

\cf{Ftr}{Advanced Training}[Ex]
A fighter treats his combat prowess as if it were 2 points higher than it actually is for the purpose of meeting feat prerequisites.

\subsection{Class Features}
All fighters have the following abilities.

\cf{Ftr}{Armor Discipline}
A fighter's training grants him additional capability in armor.
He must choose to improve his agility or his resilience in armor.
This applies to all armor discipline abilities the fighter has.
If he improves his agility, he reduces his armor check penalty by 2 and reduces his arcane spell failure by 5\% while wearing body armor.
If he improves his resilience, he gains a \plus1 bonus to Armor defense while wearing body armor.

\cf{Ftr}[2nd]{Bonus Feat}
The fighter gets a bonus combat-oriented feat.
This bonus feat must be drawn from the feats noted as combat feats on \tref{Combat Feats}.
A fighter must still meet all prerequisites for a bonus feat, including attribute and combat prowess minimums.
The fighter gains an additional bonus feat at his 6th fighter level and every four fighter levels thereafter (6th, 10th, 14th, and 18th).

\cf{Ftr}[3rd]{Weapon Discipline}
The fighter's training grants him additional capability when using his weapons.
He may choose a weapon group, or he may choose to train equally with all weapons.
If he chooses a weapon group, he gains a \plus1 bonus to accuracy with weapons from that group.

If he chooses not to focus on a specific group of weapons, he gains the ability to become proficient with any weapon group if he spends 1 hour training with a weapon from that group.
He may only keep this proficiency with one weapon group at a time; if he trains with a new weapon group, he loses his proficiency in the previous group.

\cf{Ftr}[4th]{Adaptive Combat Feats}
The fighter gains a flexible bonus feat which he can change periodically.
The fighter chooses a number of combat feats equal to half his fighter level or half his Intelligence, whichever is higher.
These feats comprise his adaptive feat pool.
The fighter gains one of the feats from his adaptive feat pool as a bonus feat.
By training for an hour, the fighter can change his current adapative combat feat to one of the other feats in his adaptive style feat pool.
He must meet the prerequisites for the new feat.

\par An adaptive style feat may be used normally as prerequisites for other feats or abilities.
However, if an adaptive style feat is used as a prerequisite, it cannot be changed until the fighter no longer needs to use it as a prerequisite, such as might happen if the fighter takes the feat as a normal feat or bonus feat.

\par In order to gain an adaptive style feat, it must be reasonably possible to do training related to the new feat.
For example, a fighter could not gain Weapon Focus in axes without at least one axe available to train with.

The fighter gains an additional adaptive combat feat at his 8th fighter level and every four fighter levels thereafter (8th, 12th, 16th, and 20th).
He may change all of his adaptive combat feats at once when he trains.

\cf{Ftr}[5th]{Combat Discipline}
The fighter can use his superior training and focus to keep fighting in the face of debilitating effects.
When a fighter is initially affected by one of the conditions listed on \trefnp{Combat Discipline Conditions}, he may use his combat discipline ability to instead suffer the mitigated condition one column to the right.
He can suppress the condition up to 5 rounds.

\par Using combat discipline takes no action, and can be done at any time, even when it isn't the fighter's turn.
A fighter may use this ability a number of times per day equal to his Willpower or half his fighter level, whichever is higher.
However, he cannot mitigate more than one condition at a time.
If the fighter attempts to mitigate a new condition, the old condition resumes its normal effect immediately.

\begin{dtable}
    \lcaption{Combat Discipline Conditions}
    \begin{dtabularx}{\columnwidth}{l *{3}{>{\lcol}X}}
        \tb{Original Condition} & \tb{Mitigated Condition} & \tb{Mitigated Condition} & \tb{Mitigated Condition} \\
        \hline
        Panicked              & Frightened        & Shaken & None \\
        Petrified             & Paralyzed         & Slowed & None \\
        Blinded               & Visually impaired & None   & \x   \\
        Confused              & Disoriented       & None   & \x   \\
        Exhausted             & Fatigued          & None   & \x   \\
        Nauseated             & Sickened          & None   & \x   \\
        Severely impaired     & Impaired          & None   & \x   \\
        Stunned               & Dazed             & None   & \x   \\
        Ability damage\fn{1}  & None              & \x     & \x   \\
        Ability penalty\fn{1} & None              & \x     & \x   \\
        %Entangled             & None              & \x     & \x   \\
        Deafened              & None              & \x     & \x   \\
        Fascinated            & None              & \x     & \x   \\
        Ignited\fn{2}         & None              & \x     & \x   \\
        Immobilized           & None              & \x     & \x   \\
        Negative level\fn{3}  & None              & \x     & \x   \\
        Slowed                & None              & \x     & \x   \\
        Vulnerable            & None              & \x     & \x   \\
    \end{dtabularx}
    1.  Mitigate up to half fighter level or half Constitution, whichever is greater. \\
    2.  Mitigates the impairment, but does not prevent the fighter from taking d6 fire damage per round until the fire is put out.  \\
    3.  Mitigate a single negative level. \\
\end{dtable}

\par A fighter can never use this ability more than once against a single source.
For example, if a fighter is confused by a \spell{confusion} spell, he can use this ability to become disoriented instead of confused, but he can't then expend a second use to stop being disoriented.
The lesser condition that this ability imposes may be cured or removed normally, but doing so does not affect the resurgence of the condition the fighter was originally afflicted with.
If a fighter uses this ability to mitigate or negate a condition which he must suffer as a sacrifice or cost to gain some benefit, he automatically forfeits the benefit he would have gained.

\cf{Ftr}[7th]{Improved Armor Discipline}
The fighter's training with his armor improves.
If he chose agility, he reduces his armor check penalty by 4 and decreases his arcane spell failure by 15\%.
This does not stack with the effects of armor discipline.
In addition, he treats all body armor as if it were one encumbrance category lighter than it is.
\par This ability means heavy armor is treated as medium armor, medium armor is treated as light armor, and light armor is treated as being unarmored.
This can remove the halving of the fighter's Dexterity bonus, if appropriate for the new encumbrance of the fighter's armor.
This allows the fighter to qualify for class features using the reduced armor encumbrance category.
For example, a fighter 9 / wizard 2 who reduces his encumbrance in light armor could cast spells without any arcane spell failure in light armor.

If the fighter chose resilience, he gains damage reduction against physical damage equal to his fighter level. This allows him to ignore the first points of damage he would take each round.

\cf{Ftr}[9th]{Improved Weapon Discipline}
The fighter's training in his chosen weapons improves.
He gains a \plus4 bonus to resist disarm attempts against using his chosen weapons.
If he chose a specific weapon group, he gains a \plus2 bonus to accuracy with weapons from that group.
This does not stack with effects of weapon discipline.
If he did not, he can apply all weapon group-specific feats he has to any weapon group that he trains with for 1 hour.
He retains this benefit for one week after the training.

\cf{Ftr}[10th]{Battlemaster}
The fighter can improve his allies' combat abilities.
As a standard action, he may grant the use of one of his combat feats to allies within \rngmed range of him who can see and hear him.
He can affect a number of allies equal to half his Intelligence (minimum 1).
Affected allies must meet combat prowess prerequisites for the granted feat, but they can ignore all other prerequisites.
The effect lasts as long as the fighter spends a standard action to maintain the effect, and for 5 rounds thereafter.
The fighter can use this ability a number of times per day equal to his Intelligence or half his fighter level, whichever is higher.

\cf{Ftr}[11th]{Improved Combat Discipline}
The fighter's ability to keep fighting despite negative influence improves.
He can reduce conditions by two steps instead of one when using combat discipline.
For example, a stunned fighter who used combat discipline would instead be staggered.
\par In addition, a fighter may use combat discipline to reduce any penalties he suffers to his accuracy, damage, checks, or defenses by 2, even if the source of the penalty is not listed on the combat discipline chart.
\par The fighter may also mitigate up to two conditions at once.

\cf{Ftr}[13th]{Greater Armor Discipline}
The fighter's training in his chosen armor becomes still greater.
If he chose agility, he reduces his armor check penalty by 6 and decreases his arcane spell failure by 30\% while wearing armor of any kind.
In addition, he treats all armor as if it were two encumbrance categories lighter than it actually is whenever doing so would be beneficial to him.
This does not stack with the benefits of armor discipline or improved armor discipline.

If the fighter chose resilience, he may apply his damage reduction against all damage, including from magical attacks.

\cf{Ftr}[14th]{Improved Adaptive Combat}
The fighter's ability to adapt to situations improves.
He need only spend 1 minute training to change his adaptive combat feats.

\cf{Ftr}[15th]{Greater Weapon Discipline}
The fighter's training in his chosen weapons becomes still greater.
He increases the critical threat range of his chosen weapons by 1.
This increase applies after and stacks with any other effects that affect critical threat range.
Thus, a fighter using the Heartseeker combat style (see \featpcref{Heartseeker}) would have a critical threat range of 18-20.

\cf{Ftr}[17th]{Greater Combat Discipline}
The fighter's ability to keep fighting despite influence becomes still greater.
When using combat discipline, he may ignore any condition listed on the combat discipline chart.
In addition, he may use combat discipline to be \severelyimpaired with all actions rather than suffer any non-damaging condition not listed on the chart.

\par The fighter may also mitigate up to three conditions at once.

\cf{Ftr}[18th]{Improved Battlemaster}
The fighter can improve his allies' combat abilities more effectively.
When using his battlemaster ability, he can grant two feats at once.
In addition, he can use his battlemaster ability as a swift action.

\cf{Ftr}[19th]{True Discipline}
The fighter's discipline in his chosen area is beyond equal.
He must choose either weapon discipline, armor discipline, or combat discipline.
Depending on which discipline he chooses, he gains a different bonus.
\subcf{True Weapon Discipline}
When the fighter makes a standard attack, he can make an additional strike.
This does not stack with any other effects which grant extra strikes.
\subcf{True Armor Discipline}
If the fighter chose agility, he no longer suffers armor check penalties or arcane spell failure with any armor.
He ignores the encumbrance of all armor, causing him to be treated as unarmored whenever doing so is beneficial to him.
In addition, he applies the defense bonus from any body armor he wears to his Reflex defense.

If the fighter chose resilience, he may apply his Constutition to his Reflex defense in place of his Dexterity while wearing armor of any kind.
In addition, he applies the defense bonus from any body armor he wears to his Fortitude defense.
\subcf{True Combat Discipline}
The fighter can use combat discipline to be \impaired with all actions instead of suffering any nondamaging negative effect with a duration.
He may also mitigate up to four conditions at once.

\cf{Ftr}[20th]{Greater Adaptive Combat}
The fighter's ability to react to situations is unparalleled.
As a swift action, he can exchange a single adaptive combat feat from his adapative feat pool.

\section{Monk}
\begin{dtable}
    \lcaption{Monk Progression}
    \begin{dtabularx}{\columnwidth}{>{\ccol}p{\levelcol} >{\ccol}p{\babcolavg} *{3}{>{\ccol}p{\savecol}} >{\lcol}X}
        \tb{Level} & \tb{Combat Prowess} & \tb{Fort} & \tb{Ref} & \tb{Ment} & \tb{Special} \\
        \hline
        \monkprogressionrow{1}  & \Ki strike \plus1, \ki ward, unarmed warrior  \\
        \monkprogressionrow{2}  & Channel \ki                                   \\
        \monkprogressionrow{3}  & Uncanny dodge                                 \\
        \monkprogressionrow{4}  & Flurry of blows, \ki strike \plus2            \\
        \monkprogressionrow{5}  & Channel \ki                                   \\
        \monkprogressionrow{6}  & \Ki augmentation                              \\
        \monkprogressionrow{7}  & Bodily perfection, \ki strike \plus3          \\
        \monkprogressionrow{8}  & Channel \ki                                   \\
        \monkprogressionrow{9}  & Improved uncanny dodge                        \\
        \monkprogressionrow{10} & Improved evasion, \Ki strike \plus4           \\
        \monkprogressionrow{11} & Channel \ki                                   \\
        \monkprogressionrow{12} & \Ki augmentation                              \\
        \monkprogressionrow{13} & Improved bodily perfection, \ki strike \plus5 \\
        \monkprogressionrow{14} & Channel \ki                                   \\
        \monkprogressionrow{15} & Timeless                                      \\
        \monkprogressionrow{16} & \Ki strike \plus6, perfect mind               \\
        \monkprogressionrow{17} & Channel \ki                                   \\
        \monkprogressionrow{18} & \Ki augmentation                              \\
        \monkprogressionrow{19} & Greater bodily perfection, \ki strike \plus7  \\
        \monkprogressionrow{20} & Channel \ki, true perfection                  \\
    \end{dtabularx}
\end{dtable}

\classbasics{Alignment} Any nonchaotic.

\classbasics{Class Skills}
\subparhead{Strength} Climb, Jump, Sprint, Swim.
\subparhead{Dexterity} Balance, Escape Artist, Ride, Stealth, Tumble.
\subparhead{Intelligence} Heal.
\subparhead{Perception} Awareness, Spellcraft, Survival.
\subparhead{Other} Bluff, Intimidate, Perform, Persuasion.

\subsection{Base Class Features}
A character with monk as a base class gains the following abilities.

\classbasics{Skill Points} 10.

\classbasics{Defenses} \plus2 Fortitude, \plus4 Reflex, \plus4 Mental.

\cf{Mnk}{Weapon and Armor Proficiency}
Monks are proficient with simple weapons, monk weapons, and any one other weapon group.
Monks are not proficient with any armor or shields.
When wearing armor, using a shield, or carrying a medium or heavy load, a monk loses the benefit of her enlightened defense, fast movement, and \ki abilities.

\cf{Mnk}{Ki Strike} A monk's fists, and all weapons she uses, gain a \plus1 \glossterm{enhancement bonus}.
This bonus increases by \plus1 at 4th level and every 3 levels thereafter.
This functions like an enhancement bonus on a weapon, increasing her damage and offensive legend points per day (see \pcref{Weapon Enhancement Bonuses}).

\subsection{Class Features}
All monks have the following abilities.

\cf{Mnk}{Enlightened Defense}[Ex]
A monk's \ki shields her body from attacks.
When \monkunencumbered, she gains a \plus2 bonus to Armor defense.
She loses this bonus when she is helpless.

\cf{Mnk}{Ki Power}[Su]
Many monk abilities depend on her \ki power.
A monk's \ki power is equal to her Willpower or her monk level, whichever is higher.

\cf{Mnk}{Unarmed Warrior}[Ex]
A monk's unarmed attacks are exceptionally deadly.
She gains the Improved Unarmed Strike feat as a bonus feat.
In addition, she deals damage with her unarmed strikes as if she were two size categories larger than her actual size (1d6 for a Medium creature, or 1d4 for a small creature).
For details about how to fight while unarmed, see \pcref{Unarmed Combat}.

\cf{Mnk}[2nd]{Manifest Ki}[Su]
The monk gains the ability to channel her \ki energy to temporarily enhance her abilities.
This ability can be used a number of times per day equal to her Willpower or half her monk level, whichever is higher.

She chooses one \ki manifestation from the list below.
Some \ki manifestations have minimum monk levels, as indicated in the title of the power.
At her 5th monk level, and every three monk levels thereafter, the monk learns an additional \ki manifestation.

Some \ki manifestations function like spells.
Unless otherwise noted, the monk's effective spellpower with these abilities is equal to her \ki power.
All \ki manifestations are supernatural abilities unless otherwise noted.

\subcf{Distant Blows}
As a swift action, the monk can empower her unarmed attacks with \ki, allowing her to strike distant foes.
Until the end of the round, she gains an additional five feet of reach with her unarmed attacks, extending her threatened area.
\subcf{Slow Fall}
As an immediate action, the monk can gain the benefits of the \spell{feather fall} spell.
She must be within reach of a solid object to use this ability.
\subcf{Surpass Limits}
As a swift action, the monk can surpass the physical limitations of her body.
Until the start of her next turn, she may use her \ki power in place of her Strength or Dexterity when making skill and ability checks.
\subcf{Wholeness of Body}
As a standard action, the monk can correct the flow of energy within her body.
She heals hit points equal to half her \ki power in d10s.
\subcf{5th -- Resist Energy}
As a standard action, the monk can gain the benefits of the \spell{greater resist energy} spell for 5 minutes.
\subcf{5th -- Speed Boost}
As a swift action, the monk can gain the benefits of the \spell{lesser haste} spell for 1 round.
\subcf{5th -- Stunning Fist}
As a swift action, the monk can imbue the next unarmed attack she makes that round with the ability to disrupt her foe's \ki.
If the attack deals damage, the monk makes a special attack against the Fortitude of the struck creature.
Her accuracy is equal to her \ki power.
Success means the target is \staggered for 1 round.
A critical success means the target is \stunned for 1 round instead.
\norepeatnotes
\subcf{8th -- Rapid Step}
As a move action, the monk can move up to her speed.
She cannot be followed or withdrawn from during this movement.
\subcf{8th -- Redirect Attack}
As an immediate action, when a foe misses the monk with a melee attack, the monk can redirect the attack.
Both the foe and the monk must threaten a third creature.
If the monk redirects the attack, the foe rolls the same attack against the third creature.
\subcf{11th -- Diamond Fists}
As a swift action, the monk can empower her unarmed attacks with incredible force.
Until the end of the round, she may use her \ki power in place of her normal modifiers to physical damage with her unarmed attacks.
In addition, she treats her unarmed strike as if it were an adamantine weapon for the purpose of overcoming damage reduction and hardness.
\subcf{11th -- Flash Step}
As a move action, the monk can slip between spaces, allowing her to teleport to anywhere she can see within 30 feet.
If her line of effect is blocked, even by an invisible barrier, or if this would somehow place her inside a solid object, the ability fails.
\subcf{14th -- Empty Step} As a swift action, the monk can step into the Ethereal Plane until the end of her turn, as the \spell{ethereal jaunt}
spell.
She may return as a free action.
\subcf{14th -- Quivering Palm}
As a standard action, the monk can make a single unarmed strike.
If the attack deals damage, the struck creature is \sickened by the disruption of the \ki within in its body for 5 rounds.
At any point during that time, the monk can will the struck target to die (a free action).
If she does, and the creature is \bloodied, she makes a special attack vs. Fortitude.
If the attack succeeds, the creature loses all its hit points and takes critical damage equal to her monk level, causing it to begin dying (see \pcref{Dying}).
\subcf{17th -- Moment of Perfection}
The monk can align herself with the universe to achieve a single moment of perfection.
As a swift action, she can add her monk level as an bonus to any single physical attack or opposed check.
Alternately, when she is physically attacked, she can use an immediate action to add her monk level to her physical defenses against the attack.
After using this ability, she must wait five minutes before she can use it again.
\subcf{17th -- Empty Body} As a move action, the monk can step into the Ethereal Plane for 5 rounds, as the \spell{ethereal jaunt}
spell.
For the duration of the effect, she may switch between the planes as a move action.

\cf{Mnk}[3rd]{Evasion}[Ex]
If the monk resists a Reflex attack that normally deals half damage when resisted, she instead takes no damage.
Evasion can be used only if a monk is \unencumbered.
A helpless monk does not gain the benefit of evasion.

\cf{Mnk}[3rd]{Uncanny Dodge}[Ex]
The monk can react to danger before her senses would normally allow her to do so.
The monk reduces her overwhelm penalties by 1.
If her overwhelm penalty is reduced to 0, she is not considered to be overwhelmed.
In addition, she is not \unaware when attacked by surprise.

\cf{Mnk}[4th]{Flurry of Blows}
When attacking unarmed, the monk can attack with multiple parts of her body simultaneously.
This allows her gain the benefits of dual wielding with her unarmed strike (see \pcref{Dual Wielding}).

%\cf{Mnk}[5th]{Still Mind}{Ex} The monk may add half her Perception to he]
%cf defense in place of half her Intelligence]
%If she has an Intelligence penalty, that penalty still applies.

\cf{Mnk}[6th]{Ki Augmentation}[Su]
The monk's control over her \ki improves, allowing her to permanently improve her abilities.
She chooses one \ki augmentation from the list below.
Some \ki manifestations have minimum monk levels, as indicated in the title of the power.
At her 12th and 18th monk levels, she gains an additional \ki augmentation.

\subcf{Distant \Ki}
The monk gains an additional five feet of reach with her unarmed attacks, extending her threatened area.
\subcf{Gentle \Ki}
The monk may choose to grant any weapon she wields the Nonlethal weapon property.
This can cause all damage she deals to be nonlethal.
\subcf{Forceful \Ki}
All weapons the monk wields gain the Forceful weapon property.
She can use any weapon to perform a shove attack, and gains a \plus2 bonus to accuracy with shove attacks.
\subcf{Impactful \Ki}
All weapons the monk wields gain the Impact weapon property.
Whenever she scores a critical hit, all damage dealt in excess of the target's hit points is dealt as critical damage.
\subcf{Perfect Sight}
The monk continuously gains the benefits of the \spell{see invisibility} spell.
\subcf{Perfect Speech}
The monk gains the ability to speak with and understand the speech of any creature.
This grants her no special ability to speak to or understand creatures that do not speak, such as animals, or to understand writing.
\subcf{Sharpened \Ki}
All weapons the monk wields gain the Keen weapon property.
She gains a \plus4 bonus to confirm critical threats.
\subcf{12th -- Perfect Motion}
The monk continuously gains the benefits of the \spell{freedom} spell.
\subcf{12th -- Perfect Soul}
The monk gains spell resistance equal to 10 \add her \ki power.
To affect the monk with a spell, a caster must make an attack with an accuracy equal to its spellpower.
If the attack beats the monk's spell resistance, the spell works normally.
Otherwise, the spell has no effect on the monk.

\cf{Mnk}[7th]{Bodily Perfection}[Ex]
The monk gains a \plus1 bonus to her Strength, Dexterity, and Constitution.

\cf{Mnk}[9th]{Improved Uncanny Dodge}[Ex]
The monk reduces her overwhelm penalties by 2.
This does not stack with the effects of uncanny dodge.
If her overwhelm penalty is reduced to 0, she is not considered to be overwhelmed.

\cf{Mnk}[10th]{Improved Evasion}[Ex]
The monk's evasion ability improves.
If the monk resists an attack against her Reflex defense, she ignores all effects of the attack, even if it would normally have effects on a miss.
Like evasion, improved evasion can be used only if a monk is \unencumbered.
A helpless monk does not gain the benefit of improved evasion.

\cf{Mnk}[13th]{Improved Bodily Perfection}
The bonus granted by the monk's bodily perfection ability increases to \plus2.

\cf{Mnk}[15th]{Timeless}[Ex]
The monk no longer takes penalties to her attribute scores for aging, and cannot be magically aged.
She also gains the benefits of being middle-aged if she did not already possess them, granting her a \plus1 bonus to her Intelligence, Perception, and Willpower.
Any aging penalties she has are removed.
The monk still dies of old age when her time is up.

\cf{Mnk}[16th]{Perfect Mind}[Ex]
The monk becomes immune to hostile mind-affecting effects.

\cf{Mnk}[19th]{Greater Bodily Perfection}
The bonus granted by the monk's bodily perfection ability increases to \plus3.

\cf{Mnk}[20th]{True Perfection}
The monk becomes inhumanly perfect.
If she rolls less than a 5 on any d20 roll, it is treated as a 5.
In addition, she is treated as an outsider rather than as a humanoid for the purpose of spells and magical effects whenever doing so is advantageous to her.

\subsubsection{Ex-Monks}
A monk who becomes chaotic loses her \ki powers, and cannot gain more levels as a monk.
She retains all her other class features.
If she stops being chaotic, she regains her \ki powers and ability to take monk levels.

\section{Paladin}
\begin{dtable}
    \lcaption{Paladin Progression}
    \begin{dtabularx}{\columnwidth}{>{\ccol}p{\levelcol} >{\ccol}p{\babcolgood} *{3}{>{\ccol}p{\savecol}} >{\lcol}X}
        \tb{Level} & \tb{Combat Prowess} & \tb{Fort} & \tb{Ref} & \tb{Ment} & \tb{Special} \\
        \hline
        \paladinprogressionrow{1}  & Divine invocation (smite), divine protection \\
        \paladinprogressionrow{2}  & Divine invocation                            \\
        \paladinprogressionrow{3}  & Divine gift                                  \\
        \paladinprogressionrow{4}  & Divine invocation               \\
        \paladinprogressionrow{5}  & Divine presence           \\
        \paladinprogressionrow{6}  & Divine invocation                            \\
        \paladinprogressionrow{7}  & Divine gift                                  \\
        \paladinprogressionrow{8}  & Divine invocation                            \\
        \paladinprogressionrow{9}  & Pass judgment                                \\
        \paladinprogressionrow{10} & Divine invocation, divine presence           \\
        \paladinprogressionrow{11} & Divine gift                                  \\
        \paladinprogressionrow{12} & Divine invocation                            \\
        \paladinprogressionrow{13} &                                              \\
        \paladinprogressionrow{14} & Divine invocation                            \\
        \paladinprogressionrow{15} & Divine gift, divine presence                 \\
        \paladinprogressionrow{16} & Divine invocation                            \\
        \paladinprogressionrow{17} & Martyr's glorious retribution                \\
        \paladinprogressionrow{18} & Divine invocation                            \\
        \paladinprogressionrow{19} & Divine gift                                  \\
        \paladinprogressionrow{20} & Divine invocation, divine presence           \\
    \end{dtabularx}
\end{dtable}

\classbasics{Alignment} Any other than true neutral.

\classbasics{Class Skills}
\subparhead{Dexterity} Ride.
\subparhead{Intelligence} Heal, Knowledge (local, religion).
\subparhead{Perception} Awareness, Intimidate, Sense Motive.
\subparhead{Other} Bluff, Intimidate, Persuasion.

\subsection{Base Class Features}
A character with paladin as a base class gains the following abilities.

\classbasics{Skill Points} 5.

\classbasics{Defenses} \plus4 Fortitude, \plus4 Mental.

\cf{Pal}{Weapon and Armor Proficiency}
 Paladins are proficient with simple weapons,  any two other weapon groups,  all types of armor (heavy, medium, and light), and with shields.
A paladin is also proficient with the favored weapon group of her deity.
If she does not follow a deity, she is proficient with any other weapon group of her choice.

\cf{Pal}{Divine Protection}[Su]
The paladin's force of belief manifests a divine protection around her.
She may add her Willpower to her physical defenses (Armor, Maneuver, and Reflex) in place of Dexterity or Constitution.

\subsection{Class Features}
All paladins have the following abilities.

\cf{Pal}{Devoted Alignment}[Su]
A paladin is devoted to a specific alignment.
She must choose one of her alignment components: good, evil, lawful, or chaotic.
The alignment she chooses is her devoted alignment.
A paladin's class features are affected by this choice.
She excels at slaying creatures with alignments opposed to her devoted alignment.
A paladin's devoted alignment cannot be changed without extraordinary repurcussions to the paladin.

\cf{Pal}{Divine Power}
Many paladin abilities depend on her divine power.
A paladin's divine power is equal to her Willpower or her paladin level, whichever is higher.

\cf{Pal}{Divine Invocation}[Su]
A paladin can invoke the power of her alignment to achieve incredible effects.
This ability can be used a number of times per day equal to her Willpower or half her paladin level, whichever is higher.
She gains the smite divine invocation.

At 2nd level, and every even level thereafter, the paladin gains an additional divine invocation.
Most divine invocations have minimum paladin levels, as indicated in the title of the ability.
Some divine invocations are also restricted to paladins with specific devoted alignments.
All divine invocations are supernatural abilities unless otherwise noted.
The paladin's accuracy with divine invocations is equal to her divine power.
If a divine invocation emulates a spell, the paladin's effective spellpower is equal to her divine power.

Divine powers marked with an asterisk are called smite powers.
Smite powers function like the smite divine invocation, except that they also have additional effects.
Unless otherwise noted, these additional effects only occur if the smited creature does not share the paladin's devoted alignment.

\parhead{Any Alignment}

\subcf{Smite}
As a swift action, the paladin may declare her next physical attack to be a smite attack.
She may use her divine power in place of her normal accuracy.
If she strikes a target that does not share her devoted alignment, her weapon deals maximum damage, and she deals bonus damage equal to her divine power.
If she strikes a creature who shares her devoted alignment, she deals no damage at all (not even normal weapon damage), but the use of the ability is still spent.
\subcf{2nd -- Bless} This invocation functions like the \spell{bless} spell.
\subcf{2nd -- Divine Favor} This invocation functions like the \spell{divine favor} spell.
\subcf{2nd -- Lay on Hands}
As a standard action, the paladin can make a touch attack against a creature.
If the creature's alignment is not opposed to her devoted alignment, the creature is healed.
If the creature's alignment is opposed to the paladin's devoted alignment, the creature takes divine damage.
This ability heals or inflicts 1d10 damage per two divine power.
The paladin must make a special attack to affect unwilling targets.
A failed attack heals or inflicts half damage.
\subcf{2nd -- Resounding Smite*}
The creature struck by this smite is knocked prone.
\subcf{4th -- Dispelling Smite*}
The creature struck by this smite is affected by \spell{dispel magic}.
\subcf{4th -- Penetrating Smite*}
The paladin makes a special attack against the Fortitude of the creature struck by this smite.
If the special attack succeeds, the struck creature loses its damage reduction for 5 rounds, including against this smite.
\subcf{6th -- Seeking Smite*}
This smite attack ignores any miss chances, such as from active cover or visual impairment.
The weapon must still be physically able to strike the target.
\subcf{8th -- Brilliant Smite*}
This smite attack targets the foe's Reflex defense instead of its Armor defense.
\subcf{8th -- Dazing Smite*}
The paladin makes a special attack against the Mental defense of the creature struck by this smite.
If the special attack succeeds, the struck creature is \dazed for 5 rounds.
\subcf{8th -- Mass Bless} This invocation functions like the \spell{mass bless} spell.
\subcf{10th -- Greater Divine Favor} This invocation functions like the \spell{greater divine favor} spell.
\subcf{10th -- Spellreaving Smite*}
All spells and magical effects on the creature struck by this smite are dispelled.
Spells and effects that cannot be removed by \spell{dispel magic} are unaffected.
The paladin must have the dispelling smite invocation to choose this invocation.
\subcf{12th -- Coercing Smite*}
The paladin makes a special attack against the Mental defense of the creature struck by this smite.
If the special attack succeeds, the struck creature must obey a \spell{suggestion}, of the paladin's choice.
This functions like the \spell{suggestion} spell, except that the creature must only obey the suggestion for 5 rounds.
The paladin must speak the suggestion aloud, but she need not speak in a language the subject understands.
The \spell{suggestion} lasts for 5 rounds.
This is a mind effect.
\subcf{12th -- Divine Might} This invocation functions like the \spell{divine might} spell.
\subcf{14th -- Disorienting Smite*}
The creature struck with this smite is \disoriented for 5 rounds.
\subcf{14th -- Staggering Smite*}
The creature struck with this smite is \staggered for 5 rounds.
\subcf{14th -- Immobilizing Smite*}
The creature struck with this smite is \immobilized for 5 rounds.
\subcf{20th -- Converting Smite*}
The paladin's smite shows her foe the error of its ways.
If the struck creature is \bloodied after the damage from the smite, and the paladin succeeds at a special attack vs. Mental defense, the struck creature's alignment changes.
It gains the paladin's devoted alignment for 1 week.
After that time, it can choose to return to its original alignment, or keep its new alignment permanently.
\subcf{20th -- Terrifying Smite*}
The paladin's smite fills her foe with fear.
The struck creature is \frightened for 5 rounds.

\parhead{Chaos Divine Invocations}
\subcf{6th -- Confusion} This invocation functions like the \spell{confusion}
spell.
\subcf{8th -- Break the Chains}
As a standard action, the paladin can break all shackles, bindings, and locks within a \arealarge radius of her.
Nonmagical objects are automatically broken.
To break magical objects, the paladin must make a special attack against a DC equal to 10 \add the object's spellpower.
\subcf{8th -- Freedom} This invocation functions like the \spell{freedom} spell.
\subcf{10th -- Free the Mind}
As a standard action, the paladin can dispel all magical enchantment and illusions affecting a creature within \rngmed range.
\subcf{10th -- Chaotic Pursuit}
As a move action, the paladin can teleport adjacent to a random enemy within \rngmed range of her.

The paladin can only teleport to enemies whose location she is aware of, and who have adjacent free space she can stand in.
She appears in the space closest to her original location.
A paladin who has other movement modes, such as flight, can also teleport next to enemies only accessible using that movement mode.
\subcf{12th -- Chaotic Redirection}
As an immediate action, when the paladin or any of her allies within \rngclose range is struck by a physical attack, the paladin can redirect the attack to a random creature within \rngclose range of the paladin, including the paladin.
The attack is made against that creature instead of its original target, using its original accuracy, and has its normal effects if it hits.
After using this invocation, the paladin cannot use it for 5 rounds.
\subcf{12th -- Discordant Chant} This invocation functions like the \spell{discordant song}
spell, except that the paladin must chant to maintain the effect instead of creating music.
\subcf{14th -- Mass Confusion} This invocation functions like the \spell{mass confusion}
spell.

\parhead{Good Divine Invocations}
\subcf{4th -- Challenging Smite*}
The paladin's smite compels her foe's attention.
For 5 rounds, the struck creature takes a \minus2 penalty to accuracy on attacks that do not include the paladin as a target.
\subcf{4th -- Shield Other} As a standard action, the paladin can choose to take half the damage that an ally within \rngmed range will take, as the \spell{share pain}
spell.
\subcf{6th -- Martyr's Shield}
As an immediate action, when an ally within \rngmed range would take critical damage, the paladin can take that damage as regular damage instead.
\subcf{10th -- Noble Pursuit}
As a move action, the paladin can teleport adjacent to an enemy within \rngmed range that attacked one of her allies within the past round.

The paladin can only teleport to enemies whose location she is aware of, and who have adjacent free space she can stand in.
She appears in the space closest to her original location.
A paladin who has other movement modes, such as flight, can also teleport next to enemies only accessible using that movement mode.

\parhead{Evil Divine Invocations}
\subcf{2nd -- Enfeeblement} This invocation functions like the \spell{enfeeblement}
spell.
\subcf{8th -- Enervation} This invocation functions like the \spell{enervation}
spell.
\subcf{8th -- Executing Smite*}
The paladin's smite takes her foe's life.
If the paladin succeeds on a special attack vs. Fortitude, the struck creature is affected by the \spell{death knell} spell.
\subcf{10th -- Agony} This invocation functions like the \spell{agony}
spell.
\subcf{10th -- Brutal Pursuit}
As a move action, the paladin can teleport adjacent to a \bloodied enemy within \rngmed range.

The paladin can only teleport to enemies whose location she is aware of, and who have adjacent free space she can stand in.
She appears in the space closest to her original location.
A paladin who has other movement modes, such as flight, can also teleport next to enemies only accessible using that movement mode.

\parhead{Law Divine Invocations}
\subcf{2nd -- Command} This invocation functions like the \spell{command}
spell.
\subcf{4th -- Hold Person} This invocation functions like the \spell{hold person}
spell.
\subcf{6th -- Read Mind} This invocation functions like the \spell{read mind}
spell.
\subcf{8th -- Hold Monster} This invocation functions like the \spell{hold monster}
spell.
\subcf{8th -- Retributive Shield} This invocation functions like the \spell{retributive shield}
spell.
\subcf{10th -- Certain Pursuit}
As a move action, the paladin can teleport adjacent to an enemy within \rngmed range.
For the next 5 rounds, or until that creature is defeated, she can only use this ability to teleport next to the same creature.

The paladin can only teleport to enemies whose location she is aware of, and who have adjacent free space she can stand in.
She appears in the space closest to her original location.
A paladin who has other movement modes, such as flight, can also teleport next to enemies only accessible using that movement mode.
\subcf{12th -- Prohibition} This invocation functions like the \spell{prohibition}
spell.
\subcf{16th -- See the Truth}
As a standard action, the paladin can unerringly dispel all magical enchantments and illusions within a \arealarge radius.
\subcf{18th -- Greater Prohibition} This invocation functions like the \spell{greater prohibition}
spell.

\cf{Pal}[2nd]{Discernment}[Su]
A paladin can discern truths about creatures she sees as a swift action.
When she uses this ability, she learns which creatures within a \arealarge cone have her devoted alignment.

The paladin may use her discernment ability a number of times per day equal to her Perception or half her paladin level (minimum 1), whichever is higher.

\cf{Pal}[3rd]{Divine Gift}
The paladin's devotion to her ideals is rewarded with a divine gift which improves her abilities.
She chooses a single divine gift from the list below.
At 7th level, and every 4 levels thereafter, she gains an additional divine gift.
Some divine gifts have minimum paladin levels, as indicated in the title of the ability.
All divine gifts are supernatural abilities unless otherwise noted.

\parhead{Any Alignment}
\subcf{Divine Health}
The paladin is immune to poison and disease.
\subcf{Unbending Dedication}
The paladin is immune to charm and domination effects.
\subcf{6th -- Divine Grace}
If the paladin resists an attack against her Fortitude, Reflex, or Mental defenses that normally deals half damage when resisted, she instead takes no damage.
\subcf{6th -- Shielded Senses}
The paladin is immune to visual and auditory effects, whenever that is beneficial to her.
\subcf{9th -- Implacable Resolve}
The paladin is immune to compulsion and inhibition effects.
\subcf{12th -- Improved Divine Grace}
If the paladin resists an attack against her Fortitude, Reflex, or Mental defenses that has any partial effect when resisted, she instead suffers no effect.
The paladin must have the divine grace gift in order to select this gift.

\parhead{Chaotic Divine Gifts}
\subcf{Chaotic Mind}
The paladin is unaffected by effects which detect truth, lies, or alignment.
Such effects never detect the paladin, just as if she was not there at all.
\subcf{7th -- Scrambled Senses}
The paladin is immune to speech effects, whenever that would be beneficial to her.
\subcf{7th -- Uncertain Fate}
Whenever the paladin would take 10, she instead rolls 2d20 and uses whichever roll she prefers.
\subcf{15th -- Freedom of Movement} The paladin continuously gains the benefit of the \spell{freedom}
spell.

\parhead{Evil Divine Gifts}
\subcf{3rd -- Malicious Mind}
The paladin is immune to charm effects.
%\subcf{6th -- Executioner} The paladin gains a \plus2 bonus to physical damage against \bloodied creatures.

\parhead{Good Divine Gifts}
\subcf{3rd -- Couragous Mind}
The paladin is immune to fear and negative morale effects.

\parhead{Lawful Divine Gifts}
\subcf{3rd -- Ordered Mind}
The paladin is immune to being bewildered or confused.
\subcf{7th -- Discern Lies}
The paladin may use her discernment ability to detect lies.
As a swift action, she may use her discernment to focus on a specific creature.
For the next 5 minutes, the paladin knows whenever that creature deliberately and knowingly lies.
This does not reveal the truth, uncover unintentional inaccuracies, or necessarily reveal evasions.
\subcf{15th -- Truthbearer}
The paladin is immune to unreal effects on her, such as phantasms.
Additionally, she automatically sees through figments.
Unlike normally seeing through figments, the paladin does not receive any indication that the figment would otherwise be there - the figment simply does not exist for the paladin.

\cf{Pal}[5th]{Divine Presence}[Su]
The paladin's presence alters the world around her.
She chooses a single divine presence from the list below.
Each divine presence affects a \arealarge radius emanation from the paladin, including herself.
She may choose to suppress or resume her divine presence as a swift action.

At 10th level, and every 5 levels thereafter, the paladin gains an additional divine presence.
She may have multiple divine presences active simultaneously, and suppress or resume them individually.
Most divine presences have minimum paladin levels, as indicated in the title of the ability.
All divine presences are supernatural abilities unless otherwise noted.
If a divine presence emulates a spell, the paladin's effective spellpower is equal to her paladin level or her Willpower.

\parhead{Any Alignment}
\subcf{Bolstering Aura}
Allies in the area gain a \plus2 bonus to Mental defense.
\subcf{Resilient Aura}
Allies in the area gain a \plus2 bonus to Fortitude defense.

\parhead{Chaotic Divine Presences}
\subcf{Surestrike Aura}
Whenever an ally in the area fails to confirm a critical threat, it can reroll the critical confirmation.
Any individual confirmation can only be rerolled once in this way.
\subcf{10th -- Aura of Free Movement}
Allies within the area gain a \plus10 foot bonus to movement speed.
\subcf{10th -- Aura of Mishap Avoidance}
Whenever an ally within the area rolls a 1 on an attack roll, it can reroll the attack with a \minus5 penalty.
Any individual attack can only be rerolled once in this way.
\subcf{15th -- Aura of Freedom} Allies within the area gain the benefit of the \spell{freedom}
spell.

\parhead{Evil Divine Presences}
\subcf{Executioner's Aura}
Bloodied foes within the area take a \minus2 penalty to defenses.
%\subcf{10th -- Debilitating Aura}
%All other creatures within the area are \vulnerable.

\parhead{Good Divine Presences}
\subcf{Defender's Aura}
Foes within the area take a \minus2 penalty to any attack which does not include the paladin as a target.
\subcf{10th -- Healing Aura}
At the end of each round, bloodied allies within the area heal hit points equal to half the paladin's level.
This healing cannot heal an ally above their bloodied value.

\parhead{Lawful Divine Presences}
\subcf{Aura of Certain Success}
Whenever an ally within the area rolls a 10 on an attack roll, it gets a \plus10 bonus to accuracy on that attack.
%\subcf{Surestrike Aura} Allies within the area automatically confirm critical hits.
\subcf{10th -- Aura of Certain Failure}
Whenever a foe within the area rolls a 10 on an attack roll, it takes a \minus10 penalty to accuracy that attack.
%\subcf{10th -- Enforcer's Aura}
%Any creature within the area that breaks the law is \vulnerable as long as it remains in the aura.
%The paladin must be aware of the creature's transgression for it to suffer the penalty.
\subcf{15th -- Aura of Truth}
All illusory figments and glamers are suppressed in the area.

\cf{Pal}[9th]{Pass Judgment}[Su]
The paladin gains the ability to pass judgment on those she deems unworthy.
Once per day as a swift action, she may pass judgment on a creature within 100 feet of her.
For the purpose of spells and effects, the creature is treated as if it had the alignment opposed to the paladin's devoted alignment.
This effect lasts for one day per paladin level, or until the paladin changes her mind about the subject.
This does not change the creature's actions or behavior, but the creature is subject to the paladin's smite attack, would detect as that alignment under the inspection of a \spell{detect alignment} spell, and so on.

No attack is required for this effect, and it cannot be dispelled, but a \spell{remove curse}, \spell{miracle}, or \spell{wish} spell can remove it.
The paladin can use this ability an additional time per day at her 10th paladin level and every third level thereafter.
A paladin should be careful when using this ability, as persecution of allies can lead overzealous paladins to fall.

\cf{Pal}[17th]{Martyr's Glorious Retribution}[Su]
If the paladin dies in the devoted service of her devoted alignment, she may choose to have her fallen body erupt in an immense burst of divine energy.
If she does, her body is almost completely consumed, preventing her from being raised with \spell{lesser resurrection} and similar effects that require an intact body.
This burst has two effects.
First, a \spell{sunburst} spell immediately takes effect over the area where the paladin fell.
Second, a \spell{storm of vengeance} spell begins to take effect, centered on the same area.
The spell lasts for 10 rounds, and the lightning strikes target the paladin's enemies.
Both of these effects harm only the paladin's foes, and do not harm her allies.
However, her allies' vision is still impeded by the \spell{storm of vengeance}.

\subsubsection{Ex-Paladins}
A paladin who ceases to follow her devoted alignment loses all supernatural paladin class features.
She may not gain any additional paladin levels.
She regains her abilities and advancement potential if she atones for her violations (see the \spell{atonement} ritual), as appropriate.

\section{Ranger}

\begin{dtable}
    \lcaption{Ranger Progression}
    \begin{dtabularx}{\columnwidth}{>{\ccol}p{\levelcol} >{\ccol}p{\babcolgood} *{3}{>{\ccol}p{\savecol}} >{\lcol}X}
        \tb{Level} & \tb{Combat Prowess} & \tb{Fort} & \tb{Ref} & \tb{Ment} & \tb{Special} \\
        \hline
        \rangerprogressionrow{1}  & Quarry \plus2, tenacious hunter, Track   \\
        \rangerprogressionrow{2}  & Fast movement, ranger lore               \\
        \rangerprogressionrow{3}  & Favored terrain, wild speech             \\
        \rangerprogressionrow{4}  & Low-light vision, tracking expert        \\
        \rangerprogressionrow{5}  & Quarry \plus3, ranger lore               \\
        \rangerprogressionrow{6}  & Free stride                              \\
        \rangerprogressionrow{7}  & Favored terrain, Guide                   \\
        \rangerprogressionrow{8}  & Darkvision, ranger lore                  \\
        \rangerprogressionrow{9}  & Hidden hunter                            \\
        \rangerprogressionrow{10} & Quarry \plus4                            \\
        \rangerprogressionrow{11} & Favored terrain (planar), ranger lore    \\
        \rangerprogressionrow{12} & Blindsense                               \\
        \rangerprogressionrow{13} & Terrain mastery                          \\
        \rangerprogressionrow{14} & Ranger lore, unerring hunter             \\
        \rangerprogressionrow{15} & Favored terrain (planar), quarry \plus5  \\
        \rangerprogressionrow{16} & Blindsight                               \\
        \rangerprogressionrow{17} & Ranger lore, terrain mastery             \\
        \rangerprogressionrow{18} & Hide in plain sight, ranger lore         \\
        \rangerprogressionrow{19} & Favored terrain (planar), perfect stride \\
        \rangerprogressionrow{20} & Quarry \plus6, truesight
    \end{dtabularx}
\end{dtable}

\classbasics{Alignment} Any.

\classbasics{Class Skills}
\subparhead{Strength} Climb, Jump, Sprint, Swim.
\subparhead{Dexterity} Balance, Escape Artist, Ride, Stealth, Tumble.
\subparhead{Intelligence} Heal, Knowledge (dungeoneering, geography, nature).
\subparhead{Perception} Awareness, Creature Handling, Survival.
\subparhead{Other} Bluff, Intimidate, Persuasion.

\subsection{Base Class Features}
A character with ranger as a base class gains the following abilities.

\classbasics{Skill Points} 15.

\classbasics{Defenses} \plus4 Fortitude, \plus4 Reflex, \plus2 Mental.

\cf{Rgr}{Weapon and Armor Proficiency}
 A ranger is proficient with simple weapons, any two weapon groups, light and medium armor, and shields.
He is also proficient with his choice of bows, crossbows, or thrown weapons.

\cf{Rgr}{Tenacious Hunter}[Ex]
The ranger adds his quarry bonus to his defenses against attacks that his quarry makes.
See the Quarry ability, below, for details.

\subsection{Class Features}
All rangers have the following abilities.

\cf{Rgr}{Quarry}[Ex]
As a swift action, a ranger may designate any foe he sees as his quarry.
A ranger gains a \plus2 bonus to damage with physical attacks and Awareness, Stealth, and Survival checks against his quarry.
However, while a ranger has designated a quarry, he takes a \minus2 penalty on the same rolls against any target other than his quarry.

A ranger may not normally have more than one quarry at once.
He may not designate a new quarry until he defeats his old quarry, or until he gives up on the quarry.
He may give up pursuing a quarry as a free action.
If he does, he is unable to designate a new quarry until he rests for 5 minutes.

Some special abilities allow the ranger to designate multiple creatures as a quarry.
Abilities which designate multiple creatures as a quarry always last for a specific amount of time.

If the ranger does not see his quarry for more than a week, it is no longer considered his quarry.

\par The amount by which a ranger's damage and skill checks increase against his quarry is called his quarry bonus.
The ranger's quarry bonus improves by \plus1 at 5th level and every 5 levels thereafter.
His penalties against targets other than his quarry remains the same.

\cf{Rgr}{Track}
A ranger gains Track as a bonus feat (see \featref{Track}).

\cf{Rgr}[2nd]{Fast Movement}[Ex]
The ranger increases his movement speed by 10 feet when \unencumbered.

\cf{Rgr}[2nd]{Ranger Lore}
The ranger gains an ability drawn from ancient ranger lore.
He chooses a single ranger lore from the list below.
Some ranger lores have minimum ranger levels, as indicated in the title of the ability.
At his 5th ranger level, and every three ranger levels thereafter, the ranger gains an additional ranger lore.

Some ranger lores depend on the ranger's lore power.
A ranger's lore power is equal to his Perception or his ranger level, whichever is higher.

All ranger lore abilities are extraordinary abilities unless otherwise noted.

\subcf{Combat Feat}
The ranger gains a combat feat for which he qualifies (see Feats).
This lore can be selected multiple times.

\subcf{Evasion}
If the ranger resists a Reflex attack that normally deals half damage when resisted, he instead takes no damage.
Evasion can be used only if a ranger is \unencumbered.
A helpless ranger does not gain the benefit of evasion.

\subcf{Favored Enemy}
The ranger increases his quarry bonus by \plus2 against creatures of a particular kind.
The possible creature options are listed on \trefnp{Favored Enemy Options}.
The ranger may select this lore multiple times, choosing a different favored enemy each time.

\begin{dtable}
    \lcaption{Favored Enemy Options}
    \begin{dtabularx}{\columnwidth}{X X}
        \hline
        Animals and vermin             & Humanoids (uncivilized)  \\
        Dragons                        & Oozes and plants         \\
        Fey                            & Outsiders (inner planes) \\
        Giants and monstrous humanoids & Outsiders (outer planes) \\
        Humanoids (civilized)          & Undead and constructs    \\
    \end{dtabularx}
\end{dtable}

\subcf{Rapid Healing}
The ranger naturally heals four times faster than normal.
He requires only two hours of rest to heal half his hit points.
This stacks with the benefits of accelerating recoveryy with the Heal skill (see \pcref{Accelerate Recovery}).

\subcf{Survivalist}
The ranger gains a Survival-based feat for which he qualifies as a bonus feat.

\subcf{Undead Destroyer}
The ranger may designate undead as a quarry.
If he does, he gains his quarry bonus against all undead for 5 minutes.
In addition, non-intelligent undead treat him as if he were undead.

\subcf{6th -- Barkskin}
The ranger gains a \plus1 bonus to Armor defense.

\subcf{6th -- Aberrant Hunter}
The ranger may designate aberrations as a quarry.
If he does, he gains his quarry bonus against all aberrations for 5 minutes.
In addition, he becomes immune to Mind effects from aberrations.

\subcf{6th -- Energy Adaptation}
The ranger continuously gains the benefits of the \spell{resist energy} spell, with a spellpower equal to his lore power.

\subcf{6th -- Fey Stalker}
The ranger may designate fey as a quarry.
If he does, he gains his quarry bonus against all fey.
In addition, he becomes immune to Mind effects from fey.

\subcf{6th -- Giantslayer}
The ranger may designate monstrous humanoids as a quarry.
If he does, he gains his quarry bonus against all monstrous humanoids for 5 minutes.
In addition, he gains a \plus8 bonus to Maneuver defense against attacks that would move him.

\subcf{6th -- Ooze Hunter}
The ranger may designate oozes as a quarry.
If he does, he gains his quarry bonus against all oozes for 5 minutes.
In addition, he becomes immune to the slime and engulf effects of oozes.

\subcf{9th -- Dragonslayer}
The ranger may designate dragons as a quarry.
If he does, he gains his quarry bonus against all dragons for 5 minutes.
In addition, he becomes immune to the breath attacks of dragons.

\subcf{9th -- Golem Breaker}
The ranger may designate constructs as a quarry.
If he does, he gains his quarry bonus against all constructs for 5 minutes.
In addition, his attacks ignore the hardness and damage reduction of constructs, and negate it for 1 round.

\subcf{9th -- Mageslayer}
The ranger may designate arcane spellcasters as a quarry.
If he does, he gains his quarry bonus against all arcane spellcasters for 5 minutes.
In addition, he gains spell resistance against arcane spells equal to 10 \add his lore power.

To affect the ranger with a spell, a caster must make an attack with an accuracy equal to its spellpower.
If the attack beats the ranger's spell resistance, the spell works normally.
Otherwise, the spell has no effect on the ranger.

\subcf{9th -- Master of the Hunt}
The ranger may use a standard action to share the benefits of his quarry ability with all allies who can see and hear him for 5 rounds.
His allies do not suffer penalties against targets other than the quarry.

\subcf{9th -- Scent}
The ranger gains the scent ability (see \pcref{Scent}).

\subcf{12th -- Camouflage}\label{Camouflage}
The ranger can use the Stealth skill to hide in any of his favored terrains, even if the terrain does not grant cover or concealment.

\subcf{12th -- Improved Evasion}
If the ranger resists an attack against his Reflex defense, he ignores all effects of the attack, even if it would normally have effects on a miss.
Like evasion, improved evasion can be used only if a ranger is \unencumbered.
A helpless ranger does not gain the benefit of improved evasion.
The ranger must have the evasion lore to choose this lore.

\subcf{12th -- Wild Aspect}
The ranger chooses any wild aspect from the list of wild aspects available to druids (see \pcref[Drd:Wild Aspect]{Wild Aspect}).
He continuously gains the benefits of that aspect.
His effective nature power for this ability is equal to his lore power.

\cf{Rgr}[3rd]{Favored Terrain}[Ex]
The ranger becomes particularly attuned to certain kinds of terrain.
He chooses one kind of terrain to select as a favored terrain from the list below.
Usually, rangers favor their home terrain, but a ranger may choose any kind of terrain that he has personally experienced at least once.
At his 7th ranger level, and every four ranger levels thereafter, the ranger gains an additional favored terrain.

\par While in a favored terrain, a ranger gains a \plus2 bonus to Awareness, Stealth, and Survival checks.
If he desires, he may leave no trace of his passage, causing attempts to track him to take a \minus20 penalty.
In addition, his experience with his favored terrain grants the ranger a single ability, regardless of whether he is currently in that terrain or not.
The options for favored terrains are listed below.
\subcf{Aquatic}
The ranger gains Skill Focus (Swim) as a bonus feat, and halves the penalties he takes for fighting underwater.
\subcf{Cold}
The ranger gains cold damage reduction equal to twice his ranger level.
This allows him to ignore the first points of cold damage he would take each round.
\subcf{Desert}
The ranger gains fire damage reduction equal to twice his ranger level.
This allows him to ignore the first points of fire damage he would take each round.
\subcf{Forest}
The ranger gains Skill Focus (Stealth) as a bonus feat.
\subcf{Mountains}
The ranger gains Skill Focus (Climb) as a bonus feat, and takes half damage from falling damage.
\subcf{Plains}
The ranger gains Skill Focus (Awareness) as a bonus feat.
\subcf{Swamp}
The ranger gains Perfect Health as a bonus feat.
\subcf{Underground}
The ranger gains Blind-Fight as a bonus feat.
\subcf{Urban}
The ranger gains Skill Focus (Persuasion) as a bonus feat.

\cf{Rgr}[3rd]{Wild Speech}[Su]
The ranger learns how to communicate with animals.
This ability functions like the druid ability of the same name (see Wild Speech, \pref{Drd:Wild Speech}).
A ranger can use this ability a number of times per day equal to his Perception or half his ranger level, whichever is higher.

\cf{Rgr}[4th]{Low-light Vision}[Ex]
The ranger's sight improves, allowing him to see in conditions of dim light more easily.
He gains low-light vision, allowing him to treat sources of light as if they had double their normal illumination range.
If he already has low-light vision, he doubles its benefit, allowing him to treat sources of light as if they had four times their normal illumination range.

\cf{Rgr}[4th]{Tracking Expert}[Ex]
The ranger's ability to track his foes improves.
He may always take 10 on Survival checks made to track, even if conditions would otherwise prevent this.
Additionally, he can move at his normal speed while following tracks without taking the normal \minus5 penalty.
He takes only a \minus10 penalty (instead of the normal \minus20) when moving at up to twice normal speed while tracking.

\cf{Rgr}[6th]{Free Stride}[Ex]
The ranger can move through any sort of natural terrain that slows or impedes movement at his normal speed without suffering any sort of impairment.
If a skill check, such as Climb or Swim, would normally be required to move through the terrain, this ability does not help.

\cf{Rgr}[7th]{Guide}[Ex]
Whenever the ranger is in his favored terrain, all allies that can see and hear the ranger gain his favored terrain bonuses in that terrain as well.

\cf{Rgr}[8th]{Darkvision}[Ex]
The ranger's sight improves again, and he gains the ability to see even when there is no light at all.
He gains darkvision out to 50 feet, allowing him to see in complete darkness.
If he already has darkvision, he increases its range by 50 feet.

\cf{Rgr}[11th]{Favored Terrain (Planar)}[Ex]
The ranger may choose any plane as a favored terrain in addition to his normal options whenever he gains a new favored terrain.
He is immune to any hostile planar effects from any plane he has chosen as favored terrain.
In addition, he gains a \plus2 bonus to Knowledge checks relating to the plane and is always treated as trained in Knowledge (planar) for the purpose of such checks.

\cf{Rgr}[9th]{Hidden Hunter}[Su]
The ranger becomes even more difficult for his quarry to detect.
He adds his quarry bonus to his Stealth checks against his quarry.
In addition, he continuously benefits from the effect of the \spell{nondectection} spell against all attempts that his quarry makes to detect him magically.
The effect uses a spellpower equal to his ranger level or Perception, whichever is higher.

\cf{Rgr}[12th]{Blindsense}[Ex]
The ranger's perceptions are so finely honed that he can sense his enemies without seeing them.
He gains the blindsense ability out to 50 feet.
This ability allows him to sense the presence and location of objects and foes within 50 feet without seeing them.
If he already has the blindsense ability, he increases its range by 50 feet.

\cf{Rgr}[13th]{Terrain Mastery}[Ex]
The ranger gains a greater degree of mastery over some of his favored terrains.
He chooses a single kind of terrain that he has already chosen as a favored terrain.
At his 17th ranger level, he chooses an additional kind of terrain to master.
\par While in that terrain, his bonuses on Awareness, Stealth, and Survival checks increase to \plus4.
In addition, he gains another ability based on that terrain that is constantly active, whether or not he is currently in the terrain.
The options for terrain masteries are given below.
\subcf{Aquatic}
The ranger gains a swim speed equal to his base land speed.
If he already has a swim speed, he increases his swim speed by 10 feet.
\subcf{Cold}
The ranger becomes immune to fatigue.
\subcf{Desert}
The ranger becomes immune to fatigue.
\subcf{Forest}
The ranger may use his wild speech ability to communicate with plants, as the druid ability.
\subcf{Mountains}
The ranger gains a climb speed equal to his land speed.
If he already has a climb speed, he increases his climb speed by 10 feet.
\subcf{Plains}
The ranger increases his land speed by 10 feet.
\subcf{Swamp}
The ranger becomes immune to nausea.
\subcf{Underground}
The ranger increases the range of his darkvision and blindsense by 50 feet.
\subcf{Urban}
The ranger can use the Stealth skill to hide behind creatures granting him active cover, just like he can hide behind passive cover.

\cf{Rgr}[14th]{Unerring Hunter}[Su]
The ranger's ability to hunt down his quarry improves to supernatural levels.
Once per day, the ranger may concentrate for a full round to duplicate the effects of the \spell{discern location} ritual targeted at his quarry.

\cf{Rgr}[16th]{Blindsight}[Ex]
The ranger gains the ability to ``see'' perfectly without his eyes in a 50 foot radius around him.
With this ability, he can fight just as well with his eyes closed as with them open.
If he already has the blindsight ability, he increases its range by 50 feet.

\cf{Rgr}[18th]{Hide in Plain Sight}[Ex]
While in any of his favored terrains, the ranger can use the Stealth skill to hide even while being observed, taking a \minus5 penalty to the Stealth check.
He still needs cover or concealment to hide.

If the ranger has the Camouflage ranger lore (see \pcref{Camouflage}), this allows the ranger to attempt to hide in almost any situation, as long as he is in one of his favored terrains.

\cf{Rgr}[19th]{Perfect Stride}[Su]
The ranger's ability to surpass obstacles becomes unparalleled.
He constantly acts as if he were under the effect of a \spell{freedom} spell, except that it does not allow him to act normally underwater.

\cf{Rgr}[20th]{Truesight}[Su]
The ranger's perceptions are accurate enough to defeat even powerful magic.
He gains the ability to see all things as they actually are, as the \spell{true seeing} spell, out to a range of 50 feet.

\section{Rogue}

\begin{dtable}
    \lcaption{Rogue Progression}
    \begin{dtabularx}{\columnwidth}{>{\ccol}p{\levelcol} >{\ccol}p{\babcolgood} *{3}{>{\ccol}p{\savecol}} >{\lcol}X}
        \tb{Level} & \tb{Combat Prowess} & \tb{Fort} & \tb{Ref} & \tb{Ment} & \tb{Special} \\
        \hline
        \rogueprogressionrow{1}  & Advanced training, sneak attack \plus1d6                  \\
        \rogueprogressionrow{2}  & Skill talent                           \\
        \rogueprogressionrow{3}  & Sneak attack \plus2d6, survival trick  \\
        \rogueprogressionrow{4}  & Combat trick                           \\
        \rogueprogressionrow{5}  & Skill exemplar, sneak attack \plus3d6  \\
        \rogueprogressionrow{6}  & Persistent sneak attack, skill talent  \\
        \rogueprogressionrow{7}  & Sneak attack \plus4d6, survival trick  \\
        \rogueprogressionrow{8}  & Combat trick                           \\
        \rogueprogressionrow{9}  & Skill exemplar, sneak attack \plus5d6  \\
        \rogueprogressionrow{10} & Skill talent   \\
        \rogueprogressionrow{11} & Sneak attack \plus6d6, survival trick  \\
        \rogueprogressionrow{12} & Combat trick                           \\
        \rogueprogressionrow{13} & Skill exemplar, sneak attack \plus7d6  \\
        \rogueprogressionrow{14} & Skill talent                           \\
        \rogueprogressionrow{15} & Sneak attack \plus8d6, survival trick  \\
        \rogueprogressionrow{16} & Combat trick                           \\
        \rogueprogressionrow{17} & Skill exemplar, sneak attack \plus9d6  \\
        \rogueprogressionrow{18} & Skill talent                           \\
        \rogueprogressionrow{19} & Sneak attack \plus10d6, survival trick \\
        \rogueprogressionrow{20} & Ambush master, combat trick            \\
    \end{dtabularx}
\end{dtable}

\classbasics{Alignment} Any.

\classbasics{Class Skills}
\subparhead{Strength} Climb, Jump, Sprint, Swim.
\subparhead{Dexterity} Balance, Escape Artist, Sleight of Hand, Stealth, Tumble.
\subparhead{Intelligence} Devices, Disguise, Knowledge (dungeoneering, local), Linguistics.
\subparhead{Perception} Awareness, Sense Motive.
\subparhead{Other} Bluff, Intimidate, Perform, Persuasion.

\subsection{Base Class Features}
A character with rogue as a base class gains the following abilities.

\classbasics{Skill Points} 15.

\classbasics{Defenses} \plus4 Reflex, \plus2 Mental.

\cf{Rog}{Weapon and Armor Proficiency}
Rogues are proficient with simple weapons, any two weapon groups, light armor, and bucklers.
They are also proficient with saps.

\cf{Rog}{Advanced Training}
A rogue treats her skill ranks as if they were 2 points higher than they actually are for the purpose of meeting feat prerequisites.

\subsection{Class Features}
All rogues have the following abilities.

\cf{Rog}{Sneak Attack}
If a rogue can catch an opponent when it is unable to defend himself effectively from her attack, she can strike a vital spot for extra damage.
She can choose to deal 1d6 points of extra damage if the target is unaware or is suffering overwhelm penalties from being surrounded by enemies (see \pcref{Overwhelm}).

This extra damage is only dealt the first time that the rogue makes a successful sneak attack against that particular creature in the encounter.
Additional sneak attacks against the same creature deal no additional damage.

The extra damage increases by 1d6 at her 3rd rogue level and every two rogue levels thereafter.

\par Ranged attacks can count as sneak attacks only if the target is within 30 feet.
A rogue can't strike with deadly accuracy from beyond that range.

With a sap (blackjack) or an unarmed strike, a rogue can make a sneak attack that deals nonlethal damage instead of lethal damage.
She cannot use a weapon that deals lethal damage to deal nonlethal damage in a sneak attack, not even with the usual \minus4 penalty.

A rogue can only sneak attack creatures with a discernible body structure -- oozes, incorporeal creatures, and some plants lack vital areas to attack.
Any creature that is immune to critical hits is not vulnerable to sneak attacks.
The rogue must be able to see the target well enough to pick out a vital spot and must be able to reach such a spot.
A rogue cannot sneak attack while striking the limbs of a creature whose vitals are beyond reach.

\cf{Rog}[2nd]{Skill Talent}[Ex]
The rogue's skills improve.
She gains an additional skill point, which she can place in any skill, and a bonus skill feat for which she qualifies.
At her 6th rogue level, and every four rogue levels thereafter, she gains an additional skill point and skill feat.

\cf{Rog}[3rd]{Survival Trick}
The rogue gains a trick to help her survive.
She chooses a single survival trick from the list below.
Some survival tricks have minimum rogue levels, as indicated in the title of the ability.
At her 7th rogue level, and every four rogue levels thereafter, the rogue gains an additional survival trick.

All survival tricks are extraordinary abilities unless otherwise noted.

\subcf{Evasion}
If the rogue resists an attack against her Reflex defense that normally deals half damage when resisted, she instead takes no damage.
Evasion can be used only if a rogue is \unencumbered.
A helpless rogue does not gain the benefit of evasion.

\subcf{Uncanny Dodge}
The rogue can react to danger before her senses would normally allow her to do so.
The rogue reduces her overwhelm penalties by 1.
If her overwhelm penalty is reduced to 0, she is not considered to be overwhelmed.
In addition, she is not \unaware when attacked by surprise.

\subcf{7th -- Defensive Roll}
As an immediate action, whenever the rogue takes damage from a physical attack, she can attempt to roll with the blow.
If the damage dealt is less than her Reflex defense, she takes half damage from the attack, and the damage is nonlethal.
This applies after all other effects that reduce damage, such as damage reduction.
However, the rogue also falls \prone.
The rogue must be aware of the attack to use this ability.

\subcf{7th -- Slippery Mind}
Whenever an attack for a Mind spell or effect beats the rogue's Mental defense by less than 5, she is affected normally at first.
One round later, the rogue is instead affected as if the attack had failed.
This does not help against instantaneous effects.

\subcf{11th -- Improved Evasion}
If the rogue resists an attack against her Reflex defense, she ignores all effects of the attack, even if it would normally have effects on a miss.
Like evasion, improved evasion can be used only if a rogue is \unencumbered.
A helpless rogue does not gain the benefit of improved evasion.
The rogue must have the evasion survival trick to choose this survival trick.

\subcf{11th -- Improved Uncanny Dodge}
The rogue reduces her overwhelm penalties by 2.
This does not stack with the effects of uncanny dodge.
If her overwhelm penalty is reduced to 0, she is not considered to be overwhelmed.
The rogue must have the uncanny dodge survival trick to choose this survival trick.

\cf{Rog}[4th]{Combat Tricks}
The rogue gains a combat trick to aid her and confound her foes.
She chooses a single combat trick from the list below.
Some combat tricks have minimum rogue levels, as indicated in the title of the ability.
At her 8th rogue level, and every four rogue levels thereafter, the rogue gains an additional combat trick.

Some combat tricks depend on a rogue's trick power.
A rogue's trick power is equal to her Intelligence or her rogue level, whichever is higher.

Tricks marked with an asterisk are called ambush attacks.
Ambush attacks only function on the first sneak attack the rogue makes against a particular creature in an encounter.

All combat tricks are extraordinary abilities unless otherwise noted.

\subcf{Combat Feat}
The rogue gains a combat feat for which she qualifies (see Feats).
This trick can be selected multiple times.

\subcf{Distracting Attack}
Whenever the rogue successfully sneak attacks a creature, the struck creature takes a penalty to Concentration checks until the end of the round.
The penalty is equal to the number of sneak attack dice the rogue rolled on the attack.
This penalty does not stack with itself.

\subcf{Dispelling Ambush (Su)*}
A creature damaged by this ambush attack is affected by \spell{dispel magic}.
The rogue's spellpower for this ability is equal to her trick power.
This is an ambush attack, and only works once per creature.

\subcf{Distant Precision}
The rogue can make sneak attacks from up to 100 feet away.

\subcf{Hamstring*}
A creature damaged by this ambush attack has its land speed halved for 5 rounds.
Despite the name, this can be used on creatures who do not have hamstrings.

\subcf{Merciful Blows}
The rogue suffers no penalty to physical attacks when attacking for nonlethal damage, and can deal her full sneak attack damage when attacking nonlethally.

%\subcf{Swift Poisoner}
%The rogue can apply poison to a weapon she is holding as a swift action.

\subcf{Tricky Maneuver}
When performing a maneuver against a creature she would be able to sneak attack, the rogue gains a bonus to attack equal to the number of sneak attack dice she would roll.
The benefits of this trick apply even against creatures immune to critical hits.

\subcf{8th -- Brutal Ambush*}
The rogue rolls d8s instead of d6s for her sneak attack dice on this ambush attack.

\subcf{8th -- Spellstealing Ambush*}
A creature damaged by this ambush attack is affected by \spell{spelltheft}.
The rogue's spellpower for this ability is equal to her trick power.

\subcf{8th -- Staggering Ambush*}
The rogue makes a special attack against the Fortitude defense of the creature struck by this ambush attack.
Her accuracy is equal to her trick power.
If the special attack succeeds, the struck creature is \staggered for 5 rounds.

\subcf{12th -- Assassination}
To use this ability, the rogue must spend a full round studying a creature within 100 feet of her who has not noticed her and who is not in combat.
If she make a melee sneak attack against that target within 1 round, her attack deals maximum damage, including her sneak attack damage.
If the target becomes aware of her presence before she attacks, this ability has no benefit.

\subcf{12th -- Confusing Ambush*}
The rogue makes a special attack against the Mental defense of the creature struck by this ambush attack.
Her accuracy is equal to her trick power.
If the special attack succeeds, the struck creature is \disoriented for 5 rounds.
If it critically succeeds, the struck creature is instead \confused for 5 rounds.

\subcf{12th -- Perfect Precision}
The rogue has no range limit on her sneak attacks.
She must have the distant precision combat trick to gain this trick.

\subcf{12th -- Spellreaving Ambush (Su)*}
All spells and magical effects on the creature struck by this ambush attack are dispelled.
Spells and effects that cannot be removed by \spell{dispel magic} are unaffected.
The rogue must have the dispelling ambush combat trick to choose this combat trick.

\subcf{16th -- Dazing Ambush*}
A creature damaged by this ambush attack is \dazed for 5 rounds.

\subcf{16th -- Deadly Ambush*}
The rogue makes a special attack against the Fortitude defense of the creature struck by this ambush attack.
Her accuracy is equal to her trick power.
If the special attack succeeds, the struck creature is \staggered for 5 rounds.
If it critically succeeds, the struck creature immediately dies.

\subcf{16th -- Paralyzing Ambush*}
The rogue makes a special attack against the Fortitude defense of the creature struck by this ambush attack.
Her accuracy is equal to her trick power.
If the special attack succeeds, the struck creature is \staggered for 5 rounds.
If it critically succeeds, the struck creature is \paralyzed for 5 rounds.

%\subcf{16th -- Opportunist} Once per round, the rogue can make an attack of opportunity against a creature that has just taken physical damage from another creature's attack.
%This attack counts as one of the rogue's attacks of opportunity for that round.

\subcf{20th -- Dual Ambush}
The rogue can apply the benefits of two ambush attacks to a single sneak attack.

\subcf{20th -- Lingering Ambush}
The effects of the rogue's ambush attacks last ten times longer than normal.
This has no effect on ambush attacks that have no duration.

\cf{Rog}[5th]{Skill Exemplar}[Ex]
The rogue gains a \plus5 bonus with a single skill of her choice.
At her 9th rogue level, and every four rogue levels thereafter, she may gain this bonus with an additional skill.

\cf{Rog}[6th]{Persistent Sneak Attack}[Ex]
The rogue learns how to strike vital spots more consistently.
She gains half her sneak attack dice (rounded down) when making sneak attacks against a creature she has already successfully dealt sneak attack damage to in the encounter.
For example, a 6th level rogue would deal 3d6 points of extra damage on her first sneak attack against a creature, and 1d6 points of damage on every subsequent sneak attack against the same creature.
This does not allow her to deliver ambush attacks to the same creature multiple times.

\cf{Rog}[20th]{Endless Sneak Attack}[Ex]
The rogue deals full sneak attack damage on her first successful sneak each round against the same creature, rather than only on her first sneak attack in the encounter against that creature.
This does not allow her to deliver ambush attacks to the same creature multiple times.

\section{Sorcerer}
\begin{dtable}
    \lcaption{Sorcerer Progression}
    \begin{dtabularx}{\columnwidth}{>{\ccol}p{\levelcol} >{\ccol}p{\babcolpoor} *{3}{>{\ccol}p{\savecol}} >{\lcol}X}
        \tb{Level} & \tb{Combat Prowess} & \tb{Fort} & \tb{Ref} & \tb{Ment} & \tb{Special} \\
        \hline
        \sorcererprogressionrow{1}  & Cantrip, spells, wild magic \plus2, wild tolerance    \\
        \sorcererprogressionrow{2}  & Arcane adaptation, cantrip            \\
        \sorcererprogressionrow{3}  & Focused casting                       \\
        \sorcererprogressionrow{4}  & Defensive spellblend                  \\
        \sorcererprogressionrow{5}  & Wild magic \plus3                     \\
        \sorcererprogressionrow{6}  & Arcane adaptation                     \\
        \sorcererprogressionrow{7}  &                                       \\
        \sorcererprogressionrow{8}  & Offensive spellblend                  \\
        \sorcererprogressionrow{9}  &                                       \\
        \sorcererprogressionrow{10} & Arcane adaptation, wild magic \plus4  \\
        \sorcererprogressionrow{11} &                                       \\
        \sorcererprogressionrow{12} & Improved defensive spellblend         \\
        \sorcererprogressionrow{13} &                                       \\
        \sorcererprogressionrow{14} & Arcane adaptation                     \\
        \sorcererprogressionrow{15} & Wild magic \plus5                     \\
        \sorcererprogressionrow{16} & Improved offensive spellblend         \\
        \sorcererprogressionrow{17} &                                       \\
        \sorcererprogressionrow{18} & Arcane adaptation                     \\
        \sorcererprogressionrow{19} &                                       \\
        \sorcererprogressionrow{20} & Spellblend mastery, wild magic \plus6 \\
    \end{dtabularx}
\end{dtable}

\classbasics{Alignment} Any.

\classbasics{Class Skills}
\subparhead{Intelligence} Knowledge (arcana, the planes).
\subparhead{Perception} Spellcraft.
\subparhead{Other} Bluff, Intimidate, Persuasion.

\subsection{Base Class Features}
A character with sorcerer as a base class gains the following abilities.

\classbasics{Skill Points} 5.

\classbasics{Defenses} \plus4 Mental.

\cf{Sor}{Weapon and Armor Proficiency}
Sorcerers are proficient with simple weapons  and one other weapon group.
They are not proficient with any type of armor or shield.
Armor of any type interferes with a sorcerer's arcane gestures, which can cause his spells with somatic components to fail.

\cf{Sor}{Wild Tolerance}
When the sorcerer fails a wild magic roll, the spell's normal effect happens in addition to its miscast effect.
See Wild Magic, below, for details.

\subsection{Class Features}
All sorcerers have the following abilities.

\cf{Sor}{Spells}
A sorcerer casts arcane spells using his Willpower.
The maximum spell level a sorcerer can learn or cast is equal to half his sorcerer level (minimum 1) or his Willpower, whichever is lower.
A sorcerer's spellpower is normally equal to his sorcerer level.
See the Wild Magic ability, below.

At 1st level, the sorcerer knows one 1st level spell.
At every level thereafter, the sorcerer learns one additional spell.
The spell can be of any level, up to a maximum of half the sorcerer's class level.
A sorcerer's spells are drawn from the \glossterm{unrestricted spells} on the arcane spell list (see \pcref{Arcane Spells}).

Sorcerers do not have a limit on the number of spells they can cast each day.
Their ability to cast spells is limited by their lack of control over their magic.
See Wild Magic, below, for details.

\cf{Sor}{Wild Magic}[Ex]\label{Wild Magic}
Every time a sorcerer casts a spell, except cantrips, he must make a \glossterm{wild magic roll}.
To make a wild magic roll, roll d20 \add half sorcerer level or half Willpower, whichever is higher.
The DC of the roll is equal to 10 \add the spell's level.
This roll is not an attack or check, and is not modified by abilities that affect attacks or checks.
On a natural 1, this roll always fails, regardless of other modifiers.

If the wild magic roll succeeds, the spell is cast with a \plus2 bonus to spellpower.
Failure means the spell is miscast instead (see \pcref{Miscasting}).
In addition, if the sorcerer fails a wild magic roll, he cannot cast spells of the same level as the miscast spell for 10 minutes per spell level.

At 5th level, and every 5 levels thereafter, the spellpower bonus from succeeding on a wild magic roll increases by \plus1.

\cf{Sor}{Cantrip}
Cantrips are minor spells which do not require effort to use.
A sorcerer chooses one cantrip from the list of cantrips on \pref{Cantrip List}.
He may use the cantrip at will.
Cantrips cannot be miscast, and are not affected by the sorcerer's wild magic ability.
For all other purposes, cantrips are treated as 0th level spells.

At his 2nd sorcerer level, the sorcerer learns a second cantrip of his choice.

\cf{Sor}[2nd]{Arcane Adaptation}
The sorcerer gains a new ability as he adapts to the magic that flows through him.
He chooses a single arcane adaptation from the list below.
Some arcane adaptations have minimum sorcerer levels, as indicated in the title of the ability.
At his 6th sorcerer level, and every four sorcerer levels thereafter, the sorcerer gains an additional arcane adaptation.

All arcane adaptations are extraordinary abilities unless otherwise noted.

\subcf{Arcane Bloodline}
The sorcerer gains a bloodline feat of his choice.
This arcane adaptation can be taken multiple times.
Each time, the sorcerer chooses a different feat.
\subcf{Cautious Magic}
When the sorcerer fails a wild magic roll, he may choose to supress the magical energy released.
If he does, neither the spell nor its miscast effect occurs.
In addition, he loses the ability to cast spells of the same level for half as long as he normally would.
\subcf{Rapid Recovery}
When the sorcerer fails a wild magic roll to cast a spell, the time required to regain the ability to cast spells of the spell's level is reduced by 5 minutes.
\subcf{Ritual Caster}
The sorcerer gains the Ritual Caster feat, even if he does not meet the prerequisites.
\subcf{6th -- Arcane Resilience}
The sorcerer gains damage reduction against arcane spells equal to his sorcerer level or Constitution, whichever is higher.
This allows him to ignore the first points of damage he takes each round.
This ability applies against the sorcerer's own miscast effects.
\subcf{6th -- Improved Focused Casting}
When the sorcerer casts a spell using his focused casting ability, he gains a bonus to spellpower with the spell as if he had succeeded on a wild magic roll.
\subcf{6th -- Wild Retargeting}
When the sorcerer targets a random creature with a spell's miscast effect, he may roll twice to determine which creature is affected.
He chooses which result is used.
\subcf{10th -- Spell Resistance}
The sorcerer gains spell resistance equal to 10 \add his sorcerer level or Constitution, whichever is higher.
To affect the sorcerer with a spell, a caster must make an attack with an accuracy equal to its spellpower.
If the attack beats the sorcerer's spell resistance, the spell works normally.
Otherwise, the spell has no effect on the sorcerer.

This ability does not protect the sorcerer from his own miscast effects.
\subcf{10th -- Wild Explosion}
When the sorcerer miscasts a spell that explodes when miscast, the damage affects the sorcerer and enemies within a \areasmall radius of the sorcerer.
It does not affect the sorcerer's allies within that area (other than himself).
\subcf{14th -- Certain Magic}
The sorcerer becomes unable to miscast for any reason except failing a wild magic roll.
If he would miscast a spell, such as if his concentration is broken while casting, the spell simply fails to take effect.
\subcf{18th -- Miscast Immunity}
The sorcerer becomes immune to explosive and retargeting miscast effects from spells he casts and spells cast by other creatures.
He no longer takes any damage from explosive miscasts.
He may choose to be unaffected by retargeting miscast effects that target him.
This does not protect him from localized miscasts.

\cf{Sor}[3rd]{Focused Casting}[Ex]
If the sorcerer spends one minute focusing on casting a spell, he does not need to make a wild magic roll.
This prevents him from gaining the bonus to spellpower from succeeding on a wild magic roll, but also prevents him from miscasting the spell.
If his concentration is broken while casting a spell in this way, the sorcerer automatically miscasts the spell and suffers consequences as if he had failed a wild magic roll.

\cf{Sor}[4th]{Defensive Spellblend}[Ex]
The sorcerer may combine his cantrips with his spells.
As a full-round action, the sorcerer may cast a spell that affects only himself.
The spell must have a casting time of 1 standard action or less.
If he does, he may also use a cantrip as part of the same action.
The cantrip need not target the sorcerer.
In exchange, he gains a \plus5 bonus to his wild magic roll.
The spell's level is used to determine the bonus to the sorcerer's wild magic roll.

\cf{Sor}[8th]{Offensive Spellblend}[Ex]
This ability functions like defensive spellblend, except that the spell need not only affect the sorcerer.

\cf{Sor}[12th]{Improved Defensive Spellblend}[Ex]
The sorcerer may combine two spells together.
As a full-round action, the sorcerer may cast two spells at once, resolving each spell's effects separately.
The spells cast in this way must have a casting time of 1 standard action or less, and must be at least three spell levels apart, such as a 1st-level spell and a 4th-level spell.
In addition, one of the two spells must affect only the sorcerer.
In exchange, he gains a \plus5 bonus to his wild magic roll.
The level of the higher level spell is used to determine the bonus to the sorcerer's wild magic roll.

\cf{Sor}[16th]{Improved Offensive Spellblend}[Ex]
This ability functions like improved defensive spellblend, except that neither spell need only affect the sorcerer.

\cf{Sor}[20th]{Spellblend Mastery}[Ex]
The sorcerer may use spellblends to combine spells that are only two spell levels apart, rather than three.

\subsection{Variant Sorcerer}

Warlocks are sorcerers that draw power from pacts with otherworldly creatures.

\subsubsection{Warlock}

Warlocks cast spells without training, like sorcerers.
However, while sorcerers have innate magical power, warlocks draw power from dark pacts they have made with demons, fae, or other otherworldly creatures.

\altcf{Wild Magic} The warlock does not gain this ability.

\altcf{Pact Magic} Every time the warlock casts a spell, he must make a pact magic roll.
To make a pact magic roll, roll d20 and add the spell's level.
The DC of the roll is equal to 10 \add 1 per two warlock levels.
Failure means the spell is cast normally.
Success means that the warlock's pact backfires, allowing the dark entities which gave him power to influence the world instead.

\section{Spellwarped}
\begin{dtable*}
    \lcaption{Spellwarped Progression}
    \begin{dtabularx}{\textwidth}{>{\ccol}p{\levelcol} >{\ccol}p{\babcolgood} *{2}{>{\ccol}p{4.5em}} >{\lcol}X}
        \tb{Level} & \tb{Combat Prowess} & \tb{Good Defense}\fn{1} & \tb{Normal Defenses}\fn{1} & \tb{Special} \\
        \hline
        \spellwarpedprogressionrow{1}  & Innate magic, spellwarped invocation, pool regeneration, spellwarp pool  \\
        \spellwarpedprogressionrow{2}  & Spellwarped body, surge of power                     \\
        \spellwarpedprogressionrow{3}  & Attuned senses, spellwarped aspect                   \\
        \spellwarpedprogressionrow{4}  & Resist magic, spellwarped invocation                 \\
        \spellwarpedprogressionrow{5}  & Manipulate magic                                     \\
        \spellwarpedprogressionrow{6}  & Spellwarped invocation                               \\
        \spellwarpedprogressionrow{7}  & Spellwarped aspect                                   \\
        \spellwarpedprogressionrow{8}  & Spellwarped invocation                               \\
        \spellwarpedprogressionrow{9}  & Spell resistance                                     \\
        \spellwarpedprogressionrow{10} & Spellwarped invocation                               \\
        \spellwarpedprogressionrow{11} & Spellwarped aspect                                   \\
        \spellwarpedprogressionrow{12} & Spellwarped invocation                               \\
        \spellwarpedprogressionrow{13} & Improved manipulate magic                            \\
        \spellwarpedprogressionrow{14} & Spellwarped invocation                               \\
        \spellwarpedprogressionrow{15} & Spellwarped aspect                                   \\
        \spellwarpedprogressionrow{16} & Spellwarped invocation                               \\
        \spellwarpedprogressionrow{17} & Mass surge of power                                  \\
        \spellwarpedprogressionrow{18} & Spellwarped invocation                               \\
        \spellwarpedprogressionrow{19} & Permanent surge of power, spellwarped aspect         \\
        \spellwarpedprogressionrow{20} & Spellwarped invocation                               \\
    \end{dtabularx}
    1 Each spellwarped has a good defense determined by his choice of innate magic.
\end{dtable*}

\classbasics{Alignment} Any.

\classbasics{Class Skills}
\subparhead{Intelligence} Knowledge (arcana).
\subparhead{Perception} Spellcraft.
\subparhead{Other} Bluff, Intimidate, Persuasion.

\classbasics{Special Class Skills}
A spellwarped gains additional class skills based on his choice of innate magic.
\subparhead{Alteration} Disguise, Escape Artist, Jump, Swim.
\subparhead{Pyromancy} Jump, Perform, Sprint, Tumble.
\subparhead{Telekinesis} Climb, Escape Artist, Devices, Sleight of Hand.
\subparhead{Temporal} Awareness, Sleight of Hand, Sprint, Tumble.

\subsection{Base Class Features}
A character with spellwarped as a base class gains the following abilities.

\classbasics{Skill Points} 5.

\classbasics{Defenses} \plus4 good defense, \plus2 other defenses.

\cf{Spl}{Weapon and Armor Proficiency}
A spellwarped is proficient with simple weapons, any two weapon groups, light and medium armor, and shields.

\cf{Spl}{Pool Regeneration}
Once per hour, the spellwarped regains half of his spellwarp points.
See the Spellwarped Pool ability, below, for details.

\subsection{Class Features}
All spellwarped have the following abilities.

\cf{Spl}{Innate Magic}[Su]
Each spellwarped draws his magical power from a particular kind of magic.
This is a choice made when the first level of the class is taken, and it cannot thereafter be changed.
The choices are listed below.
\subcf{Alteration}
The spellwarped can manipulate the physical forms of creatures.
His good defense is Fortitude, his key attribute is Willpower, and he treats Disguise, Escape Artist, and Jump as class skills.
An alteration spellwarped may be called an alterer, bodywarper, or shifter.
\subcf{Pyromancy}
The spellwarped can manipulate fire and heat.
His good defense is Fortitude, his key attribute is Willpower, and he treats Jump, Perform, and Tumble as class skills.
A pyromancy spellwarped may be called a pyromancer.
\subcf{Telekinesis}
The spellwarped can manipulate objects and creatures with his mind.
His good defense is Mental, his key attribute is Intelligence, and he treats Craft, Devices, and Sleight of Hand as class skills.
A telekinesis spellwarped may be called a telekine.
\subcf{Temporal}
The spellwarped can manipulate time.
His good defense is Reflex, his key attribute is Perception, and he treats Awareness, Sleight of Hand, and Tumble as class skills.
A temporal spellwarped may be called a temporalist or timewarper.

\cf{Spl}{Spellwarp Pool}[Su]
A spellwarped has the ability to tap into the latent magic within his body to generate magical effects.
He has a maximum number of spellwarp points equal to his Constitution or his spellwarped level (minimum 1), whichever is higher.
In addition, he gains a minor ability based on his choice of magic.
\subcf{Alteration -- Alter Appearance}
The spellwarped can change minor aspects of his appearance as a swift action -- removing a mole or lengthening his beard slightly.
This can grant him a \plus2 bonus to Disguise checks.
Major changes are not possible.
\subcf{Pyromancy -- Ember}
The spellwarped can snap his fingers as a swift action to create a small ember of flame in his hand for 5 minutes.
This ember casts light as a torch, and can deal 1 point of fire damage with a successful touch attack.
The ember can be dismissed as a swift action or extinguished as a move action.
\subcf{Telekinesis -- Object Manipulation}
The spellwarped can concentrate as a standard action to move objects within five feet of him telekinetically.
He can slowly lift or manipulate one object by up to one foot per round.
The object can weigh up to five pounds.
This level of control is insufficient to make skill checks or wield a weapon or shield effectively.
\subcf{Temporal -- Time Awareness}
The spellwarped always knows exactly what time it is, and can track the passage of time precisely without effort.

\cf{Spl}{Spellpower}[Su]
The strength of a spellwarped's spells and abilities are determined by his spellpower.
His spellpower is equal to his key attribute or his spellwarped level, whichever is higher.

\cf{Spl}{Spellwarped Invocation}
A spellwarped can invoke his innate magic to generate powerful effects by spending a spellwarp point.
He chooses a single invocation at 1st level from those available based on his choice of innate magic.

At his 4th spellwarped level, and every two spellwarped levels thereafter, he gains an additional invocation.
Some invocations have minimum spellwarped levels, as indicated in the title of the ability.
The list of invocations is given at \pcref{Spellwarped Invocations}.

All spellwarped invocations are supernatural abilities unless otherwise noted.
The spellwarped's accuracy with spellwarped invocations is equal to his spellpower.

\cf{Spl}[2nd]{Surge of Power}[Su]
The spellwarped can invoke a surge of magical power that allows him to embody his innate magic more fully for 5 rounds.
To invoke a surge of power, he must spend a spellwarp point as a a swift action.
The effect of his surge depends on his choice of innate magic, as described below.
\subcf{Alteration -- Alter Body}
The spellwarped enhances his physical ability.
He gains a \plus2 bonus to a physical attribute of his choice.
This bonus increases by 1 at 8th, 14th, and 20th spellwarped level.
This effect cannot increase the attribute higher than his spellpower.
\subcf{Pyromancy -- Flame Aura}
The spellwarped emanates an aura of fire for 5 rounds.
At the end of each round, creatures adjacent to him take fire damage equal to his spellpower.
\subcf{Telekinesis -- Kinetic Deflection}
The spellwarped reflexively deflects attacks away with his mind.
He gains a \plus2 bonus to his physical defenses (Armor, Maneuver, Reflex).
In addition, he may use his Intelligence to determine his physical defenses in place of his Dexterity or Constitution.
\subcf{Temporal -- Accelerate Movement}
The spellwarped accelerates his movement and reactions.
He gains a \plus2 bonus to his Reflex defense and a \plus10 foot bonus to his movement speed.
He also gains a bonus on his Sprint checks equal to his spellpower.
At 8th, 14th, and 20th spellwarped level, the defense bonus increases by 1 and the speed bonus increases by 10 feet.

\cf{Spl}[2nd]{Spellwarped Body}[Ex]
The spellwarped's body is fundamentally altered by exposure to magic.
He shows signs of the magic coursing through his body: strangely or inconsistently colored hair, natural skin markings which often resemble runes, and so on.
Anyone observing the spellwarped can make an Awareness or Spellcraft check with a DC equal to 20 \sub his spellwarped level to recognize that the character is a spellwarped.
Beating this DC by 10 allows the observer to determine the type of innate magic the spellwarped has.
In addition, the spellwarped gains an ability based on his innate magic.
\subcf{Alteration -- Augment Skin}
The spellwarped gains a \plus1 bonus to his Armor defense.
This bonus increases by 1 at his 10th and 20th spellwarped levels.
\subcf{Pyromancy -- Energy Resistance}
The spellwarped gains cold and fire damage reduction equal to twice his spellpower, allowing him to ignore the first points of cold or fire damage he takes each round.
\subcf{Telekinesis -- Tactile Telekinesis}
The spellwarped gains a \plus1 bonus to Strength and Dexterity-based checks.
This bonus increases by 1 at spellpower 5 and every 5 spellpower thereafter.
\subcf{Temporal -- Accelerate Mind}
The spellwarped gains a \plus2 bonus to Intelligence and Perception-based checks.
This bonus increases by 1 at spellpower 5 and every 5 spellpower thereafter.

\cf{Spl}[3rd]{Attuned Senses}[Su]
The spellwarped learns to recognize the telltale signs of his chosen magic.
He must concentrate as a standard action to use this ability, and he may do so any number of times per day.
\subcf{Alteration -- Perceive Alteration}
The spellwarped can discern the true form of all creatures within 50 feet of him for 1 round, ignoring any effects which magically alter their shapes.
This also grants him a \plus5 bonus to Awareness checks to see through disguises.
\subcf{Pyromancy -- Flame of Life}
The spellwarped can see the life-fire that lies within all living creatures, allowing him to clearly see all living creatures within 50 feet of him for 1 round.
This ability can reveal creatures hiding in concealment and defeat figments and glamers such as \spell{invisibility}, but does not reveal creatures hiding behind cover.
It also allows the spellwarped to see unusually warm objects, such as fires.
\subcf{Telekinesis -- Spatial Awareness}
The spellwarped can feel the forms of all objects and creatures around him, granting blindsense out to a 50 foot range for 1 round.
\subcf{Temporal -- Accelerated Search}
The spellwarped can accelerate his mind to immediately search everything within a 10 foot radius of him with the Awareness skill as a standard action.
Alternately, he may use this ability to read a book ten times as fast as normal.

\cf{Spl}[3rd]{Spellwarped Aspect}[Su]
The spellwarped gains a new ability based on his continued exposure to magical energy.
Most aspects are specific to particular kinds of innate magic, but some aspects can be taken by any spellwarped.
These aspects are listed under the General heading.

At his 7th spellwarped level, and every four spellwarped levels thereafter, the spellwarped gains an additional spellwarped aspect.
Some aspects require a minimum spellwarped level, as indicated in the title of the ability.
The full list of spellwarped aspects is given below.

\parhead{General}
\subcf{Spell Conduit}
The spellwarped may use his character level in place of his spellwarped level to determine his spellpower and the spellwarped invocations he has access to.
In addition, if he has the ability to cast spells, he may use his spellwarped spellpower in place of his normal spellpower from his casting class if it would be higher.
\subcf{7th -- Expanded Senses}
The range of the spellwarped's attuned senses ability doubles.
\subcf{11th -- Accelerated Recovery}
The spellwarped regains spellwarp points once per 10 minutes, rather than once per hour.
\subcf{11th -- Rapid Senses}
The spellwarped can constantly gain the benefit of his attuned senses ability.
He can toggle his enhanced senses on or off as a swift action.
If the ability does not have a duration, such as the temporal attuned senses ability, this aspect has no effect.

\parhead{Alteration}
\subcf{Damage Reduction}
The spellwarped gains damage reduction against his choice of piercing, slashing, or bludgeoning damage.
The amount of damage resisted is equal to half his spellpower, allowing him to ignore the first points of damage he takes each round.
If he is hit by an adamantine weapon, he cannot use his damage reduction for 1 round.
\subcf{7th -- Improved Damage Reduction}
The spellwarped's damage reduction applies against all forms of physical damage.
The spellwarped must have the damage reduction aspect to gain this aspect.
\subcf{7th -- Alter Movement}
The spellwarped gains his choice of the Legendary Balance, Legendary Climber, Legendary Leaper, or Legendary Swimmer feats, even if he does not meet the prerequisites.
He may select this aspect multiple times, choosing a different bonus feat each time.
\subcf{11th -- Alter Size}
When the spellwarped uses his surge of power, he can increase or decrease by a size category, as he chooses.
The size alteration lasts as long as his surge of power does.
This is a sizing effect, and does not stack with other sizing effects.
\subcf{15th -- Fast Healing}
While his surge of power is active, the spellwarped gains fast healing equal to half his spellpower, allowing him to heal damage each round.
This does not affect critical damage.

\parhead{Pyromancy}
\subcf{Improved Ember}
When the spellwarped uses his ember ability, he can strengthen the fire so that it illuminates up to a 40 foot radius with bright illumination.
He can also throw the ember up to 100 feet.
It burns for up to 5 rounds on its own before becoming extinguished.
\subcf{Intense Flames}
The spellwarped's attacks can ignore an amount of fire damage reduction equal to his spellpower.
\subcf{7th -- Flame Eater}
When the spellwarped resists fire damage with his spellwarped body ability, he gains temporary hit points equal to the damage resisted for 5 minutes.

\parhead{Telekinesis}
\subcf{Improved Object Manipulation}
When the spellwarped uses his object manipulation ability, he can affect objects within 10 feet, with a weight limit of up to two pounds per spellpower.
He has enough control to make checks with a DC of up to 10.
\subcf{7th -- Shieldbearer}
The spellwarped may wield shields, except tower shields, telekinetically.
The shield floats in his square, granting him its bonus to his physical defenses just as if he were wielding it.
He does not need a free hand to wield the shield and suffers no armor check penalty or arcane spell failure from it.
The shield follows him as he moves.
If it is forcibly removed from his square, he loses control over it and it falls to the ground.
\subcf{11th -- Mind Armory}
The spellwarped may control a number of weapons equal to half his Intelligence with his mind blade ability.
This does not allow him to make additional attacks per round, but he may attack interchangeably with any weapon he controls.
Each weapon threatens an area and contributes to overwhelm penalties, just as with his normal mind blade ability.

\parhead{Temporal}
\subcf{Evasion}
If the spellwarped resists a attack against his Reflex that normally deals half damage when resisted, he instead takes no damage.
Evasion can be used only if a spellwarped is \unencumbered.
A helpless spellwarped does not gain the benefit of evasion.
\subcf{Fast Movement}
The spellwarped gains a \plus10 foot bonus to movement speed.
\subcf{Uncanny Dodge}
The spellwarped not \unaware when attacked by surprise.
\subcf{7th -- Accelerate Attack}
While his surge of power is active, the spellwarped can make an additional strike at a \minus5 penalty when making a standard attack.
This does not stack with any other effects which grant extra strikes.
\subcf{11th -- Improved Uncanny Dodge}
The spellwarped reduces his overwhelm penalties by 2.
If his overwhelm penalty is reduced to 0, he is not considered to be overwhelmed.

\cf{Spl}[4th]{Resist Magic}[Ex]
The power of the magic with the spellwarped offers him some measure of protection against hostile magical effects.
He gains a \plus1 bonus to special defenses against spells and spell-like abilities.
This bonus increases by \plus1 at his 8th spellwarped level and every 4 spellwarped levels thereafter.

\cf{Spl}[5th]{Manipulate Magic}[Su]
The spellwarped can channel his innate magic to manipulate other forms of magic.
Using this ability costs a spellwarp point.

\subcf{Alteration -- Absorption}
As an immediate action, when the spellwarped makes successfully resists an attack against his Fortitude from a spell or spell-like ability, he may absorb the magic harmlessly into his body.
The spell has no effect on him, even if it would normally have an effect on a failed attack.
\subcf{Pyromancy -- Fuel the Flame}
As an immediate action, when the spellwarped is affected by a spell or spell-like ability, he may channel its energy into a burst of flame around him.
Creatures within a \areasmall radius of the spellwarped take fire damage equal to his spellpower.
The spell still has its normal effect on the spellwarped.
\subcf{Telekinesis -- Mind over Matter}
As an immediate action, when the spellwarped is subject to an attack against his Fortitude from a spell or spell-like ability, he may use his Mental defense instead.
\subcf{Temporal -- Accelerate Magic}
As a swift action, the spellwarped can halve the duration of any spell or spell-like ability affecting him.
This can end the effect immediately if it has less than one round remaining.
If this would reduce the duration by more than one day, the duration is instead reduced by one day.

\cf{Spl}[9th]{Spell Resistance}[Ex]
The magic within the spellwarped allows him to completely ignore other magic, granting him spell resistance equal to 10 \add his Constitution or spellwarped level, whichever is higher.
To affect the spellwarped with a spell, a caster must make an attack with its spellpower.
If the attack beats the spellwarped's spell resistance, the spell works normally.
Otherwise, the spell has no effect on the spellwarped.

\cf{Spl}[13th]{Improved Manipulate Magic}[Su]
The spellwarped can use his manipulate magic ability to affect any ally within \rngmed range of him.

\cf{Spl}[17th]{Mass Surge of Power}[Su]
The spellwarped can share the benefits of his surge of power with his allies.
When he uses his surge of power, he can also affect up to five additional creatures within \rngmed range of him.

\cf{Spl}[19th]{Permanent Surge of Power}[Su]
The spellwarped can maintain the full power of his innate magic without limit.
He can gain the effects of his surge of power indefinitely.
He may toggle the ability on or off as a swift action at will, without expending spellwarp points.
This does not allow him to activate his mass surge of power ability at will, and his allies only gain the benefits for 5 rounds.

\subsection{Spellwarped Invocations}\label{Spellwarped Invocations}

\subsubsection{Alteration Invocations}
\parhead{1st -- Lesser Reduction}
The spellwarped makes an special attack vs. Fortitude against a creature within \rngclose range.
A successful attack causes the creature to become one size category smaller for 2 rounds.
This has the following effects:
\begin{itemize}
    \item \minus10 ft. penalty to movement speed.
    \item \minus4 penalty to maneuver accuracy and defense.
    \item \plus1 bonus to other physical accuracy and defenses.
    \item \plus4 bonus to Stealth.
\end{itemize}
This is a size-affecting effect.
\parhead{4th -- Reduction}
This invocation functions like the lesser reduction invocation, except that the foe is reduced for 5 rounds.
\parhead{6th -- Purge}
The spellwarped can end any single effect or condition on him which requires a successful attack against his Fortitude.
If he identify the effects on him, such as with a Spellcraft check, he may freely choose which effect to end.
If he cannot differentiate the effects, choose the effect ended randomly.
\parhead{8th -- Body Bludgeon}
The spellwarped elongates and distorts a part of his body and strikes a foe with it.
The foe must be within his reach, as if he were wielding a reach weapon.
He must make a physical attack with that part of his body.
If he hits, he deals 1d10 bludgeoning damage per two spellpower \add half his Strength.
In addition, whether he hits or misses, he may make a shove attack against the creature.
He need not move with the creature to push it back.
\parhead{8th -- Enlargement} This invocation functions like the \spell{enlarge person}
spell, except that it can affect creatures of any type.
\parhead{10th -- Amorphous Body}
The spellwarped transforms his body into an amorphous form for 1 round.
In this form, he gains several benefits.
He gains a \plus20 bonus against grapple attacks, is immune to critical hits, takes no penalties for squeezing, and can move through spaces that are no more than two inches in width, though doing so forces him to move at half speed.
\parhead{10th -- Heal Wounds}
As a standard action, the spellwarped can spend two spellwarp points to remove his own injuries by transforming himself into a healthier version of his body.
He heals 1d10 points of damage per two spellpower.
This also removes any of the following conditions: blinded, diseased, exhausted, fatigued, nauseated, sickened, and poisoned.
\parhead{12th -- Baleful Polymorph} This attack functions like the \spell{baleful polymorph}
spell.
\parhead{14th -- Flight}
As a swift action, the spellwarped can spend two spellwarp points to grow wings to fly for 5 rounds.
His fly speed is equal to his base land speed, and his maneuverability is good.
See \pcref{Flying}, for more details.
At the end of the duration, the wings are subsumed back into his body.
This ability is draining to use, and the spellwarped must wait for five minutes after using it before he can use it again.
\parhead{14th -- Greater Amorphous Body}
This invocation functions like the amorphous form invocation, except that it costs two spellwarp points and lasts for 5 rounds.
\parhead{16th -- Bludgeon the Horde}
This attack functions like the body bludgeon attack, except that he may attack all foes within his reach, as if he were wielding a reach weapon.
He deals 1d6 bludgeoning damage per two spellpower \add half his Strength to each foe.
\parhead{18th -- }
\parhead{20th -- }

\subsubsection{Pyromancy Invocations}
\parhead{1st -- Lesser Ignite}
As a standard action, the spellwarped makes a Reflex attack to deal damage to a foe within \rngclose range.
This attack deals 1d6 points of fire damage \add 1 per spellpower.
A failed attack deals half damage.
\parhead{1st -- Weapon of Flame}
As a swift action, the spellwarped can create a weapon made of flame that lasts for 5 rounds.
The weapon is sized appropriately for him, and may take the form of any weapon he is proficient with.
He can attack with the weapon as if it were a normal weapon of its type, except that he adds half his Willpower to damage in place of half his Strength, and all damage dealt with the weapon is fire damage.
\par If the flame weapon leaves his hand, it is extinguished 1 round later.
\parhead{4th -- Ignite}
This attack functions like the lesser ignite attack, except that it deals 1d10 points of fire damage per two spellpower, and a successful attack also makes the target \ignited.
\parhead{6th -- Ignite Weapon}
As a swift action, the spellwarped can set one of his weapon on fire for 5 rounds.
During this time, the spellwarped can add half his Willpower to damage with the weapon he wields in place of half his Strength.
This bonus damage is fire damage.
%If ignites a weapon created using his weapon of flame invocation, he adds his full Charisma to damage with the weapon instead of half his Charisma.
\parhead{6th -- Fiery Protection}
As a standard action, the spellwarped can bestow fire and cold damage reduction equal to twice his spellpower on a creature within 30 feet of him.
The protection lasts for 1 hour.
\parhead{8th -- Conflagration}
As a standard action, the spellwarped can release a powerful explosion of flame.
He makes an attack against the Reflex defense of all enemies and objects within a \areamed radius spread of him.
If the attack succeeds against a target, it deals 1d6 fire damage per two spellpower.
A failed attack deals half damage.
\parhead{10th -- Fire Shield}
As a standard action, the spellwarped can wreath himself in flame for 5 rounds.
Any creature that hits him with its body or a melee weapon takes 1d6 fire damage per two spellpower.
Each individual creature can take this damage only once per round.
\parhead{10th -- Flameheart}
As a standard action, the spellwarped can become a being of pure fire for 1 round.
In this form, he is immune to physical damage and can pass through openings as small as one inch at no movement penalty.
However, he cannot attack normally or use any of his items, as they meld into his body.
He may invoke any of his spellwarped invocations normally.
In this form, he can make a touch attack as a standard action to deal 1d6 points of fire damage per spellpower.
\parhead{12th -- Firestride}
As a move action, the spellwarped can may teleport to any active flame of at least Tiny size within \rngmed range.
When he does so, he immolates himself and disappears into a pile of ash before stepping out from the flame unharmed.
An ordinary torch is sufficient flame to teleport to, but not a candle.
\parhead{14th -- Flight of the Phoenix}
As a swift action, the spellwarped can spend two spellwarp points to fly on wings of flame for 5 rounds.
His fly speed is equal to his base land speed, and his maneuverability is good.
See \pcref{Flying}, for more details.
At the end of the duration, the wings are extinguished.
This ability is draining to use, and the spellwarped must wait for five minutes after using it before he can use it again.
\parhead{14th -- Greater Flameheart}
This invocation functions like the flameheart invocation, except that it lasts for 5 rounds.
\parhead{16th -- }

\parhead{18th -- Phoenix Revival}
When the spellwarped takes critical damage, he may spend five spellwarp points as an immediate action, even if the critical damage would be sufficient to kill him.
If he does, he ignores the critical damage he just took and dissolves into a pile of ash for 5 rounds.
During this time, he can take no actions.
If the pile of ash remains intact after 5 rounds, the spellwarped is restored to his normal body, with zero hit points but with all critical damage healed.
However, if the pile of ash is dispersed, the spellwarped dies.
The ash cannot be harmed by fire damage, and if the pile of ash would take at least 10 points of fire damage during a round, the spellwarped returns one round sooner.
The spellwarped may take his normal actions immediately after being restored.
\parhead{20th -- Immolate}
As a standard action, the spellwarped makes a special attack vs. Fortitude against a foe within \rngclose range to consume it in flames from the inside out.
Success deals 1d10 points of fire damage per two spellpower.
Critical success kills the target instantly.
Failure deals half damage and leaves a bloodied creature with 0 hit points.

\subsubsection{Telekinesis Invocations}
\parhead{1st -- Lesser Crush}
As a standard action, the spellwarped makes a special attack vs. Fortitude against a creature within \rngclose range.
If his attack succeeds, he crushes it with telekinetic force for 1d6 points of physical damage \add 1 per spellpower.
A failed attack deals half damage.
\parhead{1st -- Mind Blade}
As a swift action, the spellwarped can telekinetically wield an unattended weapon within \rngclose range for 5 rounds.
The weapon must be a light or medium weapon appropriate for his size.
This allows him to attack with the weapon just as if he were holding it in one hand, except that he uses his Intelligence in place of his Strength.
In all other respects, this functions as if he were wielding the weapon normally, including contributing to overwhelm penalties.
The weapon floats in midair and threatens all squares adjacent to it.
As a move action, he may move the weapon up to 30 feet in any direction, even vertically.
If the weapon goes outside of \rngclose range, he loses control of it and it falls to the ground.
\parhead{4th -- Crush}
This invocation functions like the lesser crush attack, except that it deals 1d10 points of physical damage per two spellpower.
In addition, if his attack succeeds, the target is also \sickened for 5 rounds.
\parhead{4th -- Mighty Mind Blade}
This invocation functions like the mind blade invocation, except that the spellwarped may also use a heavy weapon appropriate for his size, allowing him to attack with it just as if he were holding it in two hands.
The spellwarped must have the mind blade invocation to select this invocation.
\parhead{6th -- Distant Manipulation}
As a standard action, the spellwarped can mentally exert influence at up to \rngclose range.
This allows him to take any standard action which he could normally take with his hands, using his Intelligence in place of his Strength or Dexterity, as appropriate.
He may take actions that require more than a standard action to complete by spending the same amount of time concentrating, spending one spellwarp point per two rounds that he spends concentrating.
\subcf{6th -- Dual Mind Blade}
This invocation functions like his mind blade invocation, except that the spellwarped may wield two weapons at once.
They must stay in the same space, and he may make two-weapon fighting attacks with the weapons, just as if he was wielding them with two hands.
The spellwarped must have the mind blade invocation to select this invocation.
\parhead{8th -- }
\parhead{10th -- Telekinetic Force} This invocation functions like the \spell{telekinetic force}
spell, using his Intelligence as his casting attribute.
\parhead{12th -- Strangle}
As a standard action, the spellwarped can make a special attack vs. Fortitude against a creature within \rngclose range to crush its windpipe.
This attack deals 1d10 damage per two spellpower.
If the target is \bloodied after the damage is dealt, it is nauseated for 1 round.
A failed attack deals half damage, and prevents the foe from being nauseated.
This invocation costs two spellwarp points.
\parhead{14th -- }
\parhead{16th -- }

\parhead{18th -- }
\parhead{20th -- Mass Strangle}
This invocation functions like the strangle invocation, except that it costs three spellwarp points and the spellwarped can affect any creatures within a \areasmall radius.

\subsubsection{Temporal Invocations}
\parhead{1st -- Lesser Timetheft}
As a standard action, the spellwarped can attempt to steal time.
He makes a special attack against the Mental defense of an adjacent creature to force it to skip an action.
If it is \bloodied, it skips a standard action, while if it is healthy, it skips a move action.
\parhead{4th -- Slow}
As a standard action, the spellwarped makes a special attack against the Mental defense of a creature within \rngclose range, making it \slowed for 5 rounds.
\parhead{4th -- Timetheft}
This invocation functions like lesser timetheft, except that the spellwarped regains a spellwarp point if the attack is successful.
\parhead{6th -- Flashstep}
As a standard action, the spellwarped can accelerate a creature within \rngclose range so much that he can seem to pause time for everyone but the subject.
This allows the target to immediately take a single move action.
During this move action, the target cannot be followed or withdrawn from, and may move through squares occupied by creatures or threatened by blocking enemies without penalty.
The target still suffers the effects of any environmental hazards.
If the spellwarped uses this invocation on himself, it only requires a move action to activate.
\parhead{6th -- Disjointed Time}
As a standard action, the spellwarped chaotically disrupt its local flow of time of a creature within \rngmed range.
The creature is \impaired with all actions for 5 rounds.
\parhead{8th -- Haste} This invocation functions like the \spell{haste} spell.
\parhead{8th -- Temporal Prison}
As a standard action, the spellwarped makes a special attack against the Mental defense of a single creature within \rngclose range to completely stop time for it for 5 rounds.
The affected creature can take no actions and cannot be moved, damaged, or even affected in any way until the effect ends.
The spellwarped may dismiss the effect as a swift action.

The spellwarped can only affect any individual creature with this ability once per 24 hours.
\parhead{10th -- Slow, Greater}
As a standard action, the spellwarped can spend two spellwarp points to make a creature within \rngmed range \slowed.
\parhead{10th -- Flashstep, Greater}
This invocation functions like the flashstep invocation, except that it costs two spellwarp points and can be used as a move action.
If the spellwarped uses this invocation on himself, it only requires a swift action to activate.
\parhead{12th -- Timetheft, Greater}
This invocation functions like the timetheft invocation, except that it costs two spellwarp points and does not require a special attack.
\parhead{12th -- Timestream}
The spellwarped manipulates time in a \arealarge, 10 ft.
wide line that extends out from him for 5 rounds.
All creatures and objects that pass through the line are \slowed for 1 round.
The spellwarped can exclude his allies from the effect.
The timestream is virtually invisible, requiring a DC 30 Awareness check to notice in a clear environment, though objects passing through the effect can make it more obvious.
\parhead{14th -- Inhuman Speed}
As a move action, the spellwarped can accelerate himself to immense speed, allowing him to move up to five times his speed.
During this time, he cannot be followed or withdrawn from, can move through squares occupied by enemies or threatened by blocking enemies without penalty, and can treat liquids as if they were solid ground.
\parhead{14th -- Mass Slow}
This invocation functions like the slow invocation, except that it costs two spellwarp points and affects up to five creatures within \rngclose range of the spellwarped.
\parhead{16th -- Time Reversal}
As a swift action, the spellwarped can spend a spellwarp point to create a ``time lock.'' The time lock persists for one round.
As a standard action, he can make a special attack against the Mental defense of a creature within \rngmed range to bring it backwards through time to the point at which the time lock was created.
An affected creature is perfectly restored to the point immediately after the time lock was created.
The effects of any actions that the creature took in the intervening time are undone, any damage it dealt or took is removed, it is restored to its original location, and the creature is restored in all other ways, just as if the intervening time had never occured.
The spellwarped cannot reverse time for himself in this way.
\parhead{16th -- Supreme Acceleration}
As a standard action, the spellwarped can spend three spellwarp points to accelerate himself so much that he can take an additional round of actions immediately.
During this round, all creatures he attacks are treated as \helpless, but he cannot perform a coup de grace or similar ability; such an act requires more care and precision than is possible with such immense speed.
After using this ability, he must wait 5 rounds before he can use it again.
\parhead{18th -- Mass Haste}
This invocation functions like the haste invocation, except that it costs two spellwarp points and affects up to five creatures within \rngclose range of the spellwarped.
\parhead{18th -- Time Stop}
As a standard action, the spellwarped can spend two spellwarp points to step into an alternate timestream, causing him to speed up so greatly that all other creatures seem frozen.
He can act for 1d3\plus1 rounds of apparent time.
During this time, all objects and creatures are frozen in place and are completely invulnerable to the spellwarped, though he may affect them with spellwarped invocations normally.
After using this ability, he must wait 5 rounds before he can use it again.
\parhead{20th -- Temporal Prison, Greater}
As a standard action, the spellwarped can spend two spellwarp points to completely stop time for a single creature for 5 rounds.
This functions like the temporal prison invocation, except that no attack is required.
\norepeatnotes

\section{Wizard}

\begin{dtable}
    \lcaption{Wizard Progression}
    \begin{dtabularx}{\columnwidth}{>{\ccol}p{\levelcol} >{\ccol}p{\babcolpoor} *{3}{>{\ccol}p{\savecol}} >{\lcol}X}
        \tb{Level} & \tb{Combat Prowess} & \tb{Fort} & \tb{Ref} & \tb{Ment} & \tb{Special} \\
        \hline
        \wizardprogressionrow{1}  & Cantrip, rituals, spells, prepared spells \plus2 \\
        \wizardprogressionrow{2}  & Cantrip, magic feat, specialization              \\
        \wizardprogressionrow{3}  & Arcane insight                                   \\
        \wizardprogressionrow{4}  & Cantrip sequencer                                \\
        \wizardprogressionrow{5}  & Arcane insight, prepared spells \plus3           \\
        \wizardprogressionrow{6}  & Magic feat                                       \\
        \wizardprogressionrow{7}  & Arcane insight                                   \\
        \wizardprogressionrow{8}  & Defensive sequencer                              \\
        \wizardprogressionrow{9}  & Arcane insight                                   \\
        \wizardprogressionrow{10} & Magic feat, prepared spells \plus4               \\
        \wizardprogressionrow{11} & Arcane insight                                   \\
        \wizardprogressionrow{12} & Contingency                                      \\
        \wizardprogressionrow{13} & Arcane insight                                   \\
        \wizardprogressionrow{14} & Magic feat                                       \\
        \wizardprogressionrow{15} & Arcane insight, prepared spells \plus5           \\
        \wizardprogressionrow{16} & Offensive sequencer                              \\
        \wizardprogressionrow{17} & Arcane insight                                   \\
        \wizardprogressionrow{18} & Magic feat                                       \\
        \wizardprogressionrow{19} & Arcane insight                                   \\
        \wizardprogressionrow{20} & Chain contingency, prepared spells \plus6        \\
    \end{dtabularx}
\end{dtable}

\classbasics{Alignment} Any.

\classbasics{Class Skills}
\subparhead{Intelligence} Knowledge (all kinds, taken individually), Linguistics.
\subparhead{Perception} Spellcraft.

\subsection{Base Class Features}
A character with wizard as a base class gains the following abilities.

\classbasics{Skill Points} 5.

\classbasics{Defenses} \plus4 Mental

\cf{Wiz}{Weapon and Armor Proficiency}
Wizards are proficient with simple weapons, but not with any type of armor or shield.
Armor of any type interferes with a wizard's movements, which can cause her spells with somatic components to fail.

\cf{Wiz}{Rituals}
Wizards can perform rituals to create unique magical effects (see \pcref{Rituals}).
A wizard begins play with a ritual book containing two arcane rituals of her choice (see \pcref{Arcane Rituals}).

\cf{Wiz}{Prepared Spellcasting}
A wizard may prepare specific spells ahead of time so they can be cast more powerfully.
After regaining spells for the day, a wizard may prepare a number of spell slots equal to half her Wizard level (minimum 1) or half her Intelligence, whichever is higher.
She must choose a single spell to assign to each prepared spell slot.
Once assigned in this way, the prepared spell slot can only be used to cast the chosen spell.
The spells cast using these prepared spell slots are called the wizard's prepared spells.
Whenever a wizard casts a prepared spell, she gains a \plus2 bonus to spellpower with that spell.

By resting and concentrating for 1 hour, a wizard may prepare spells again.
She may convert prepared spell slots into regular spell slots, or prepare more spell slots after casting prepared spells earlier in the day.

A wizard may prepare spells of any level, up to the maximum level of spells she is able to cast.
She may only prepare spell slots that she actually possesses.
However, she may use higher level spell slots to prepare lower level spells if she desires.

At 5th level, and every 5 levels thereafter, the spellpower bonus a wizard gains with her prepared spells increases by 1.

\subsection{Class Features}
All wizards have the following abilities.

\cf{Wiz}{Spells}
A wizard casts arcane spells using her Intelligence.
The maximum spell level a wizard can learn or cast is equal to half her wizard level (minimum 1) or her Intelligence, whichever is lower.
A wizard's spellpower is normally equal to her wizard level.
See the Prepared Spellcasting ability, below.

The number of spells a wizard knows is given on \trefnp{Wizard Spells Known}.
A wizard's spells are drawn from the \glossterm{unrestricted spells} on the arcane spell list (see \pcref{Arcane Spells}).

The number of spells a wizard can cast per day is given on \trefnp{Wizard Spell Slots}.

In order to regain her spells for the day, a wizard must dismiss all her active spells and rest for 8 hours.
This rest does not have to involve sleep, but most wizards get this rest when they sleep for the night.

\begin{dtable}
    \lcaption{Wizard Spell Slots}
    \centering
    \begin{dtabularx}{\columnwidth}{>{\ccol}X *{9}{>{\ccol}p{\spellcol}}}
        & \multicolumn{9}{c}{\tb{---{}---{}---{}---{}---{}---{}---{}---Spell Level---{}---{}---{}---{}---{}---{}---{}---}} \\
        \hline
        \tb{Level} & \tb{1st} & \tb{2nd} & \tb{3rd} & \tb{4th} & \tb{5th} & \tb{6th} & \tb{7th} & \tb{8th} & \tb{9th} \\
        1st  & 3 & \x & \x & \x & \x & \x & \x & \x & \x \\
        2nd  & 4 & \x & \x & \x & \x & \x & \x & \x & \x \\
        3rd  & 5 & \x & \x & \x & \x & \x & \x & \x & \x \\
        4th  & 6 & 3  & \x & \x & \x & \x & \x & \x & \x \\
        5th  & 6 & 4  & \x & \x & \x & \x & \x & \x & \x \\
        6th  & 6 & 5  & 3  & \x & \x & \x & \x & \x & \x \\
        7th  & 6 & 6  & 4  & \x & \x & \x & \x & \x & \x \\
        8th  & 6 & 6  & 5  & 3  & \x & \x & \x & \x & \x \\
        9th  & 6 & 6  & 6  & 4  & \x & \x & \x & \x & \x \\
        10th & 6 & 6  & 6  & 5  & 3  & \x & \x & \x & \x \\
        11th & 6 & 6  & 6  & 6  & 4  & \x & \x & \x & \x \\
        12th & 6 & 6  & 6  & 6  & 5  & 3  & \x & \x & \x \\
        13th & 6 & 6  & 6  & 6  & 6  & 4  & \x & \x & \x \\
        14th & 6 & 6  & 6  & 6  & 6  & 5  & 3  & \x & \x \\
        15th & 6 & 6  & 6  & 6  & 6  & 6  & 4  & \x & \x \\
        16th & 6 & 6  & 6  & 6  & 6  & 6  & 5  & 3  & \x \\
        17th & 6 & 6  & 6  & 6  & 6  & 6  & 6  & 4  & \x \\
        18th & 6 & 6  & 6  & 6  & 6  & 6  & 6  & 5  & 3  \\
        19th & 6 & 6  & 6  & 6  & 6  & 6  & 6  & 6  & 4  \\
        20th & 6 & 6  & 6  & 6  & 6  & 6  & 6  & 6  & 6  \\
    \end{dtabularx}
\end{dtable}

\begin{dtable}
    \lcaption{Wizard Spells Known}
    \begin{dtabularx}{\columnwidth}{>{\ccol}X *{9}{>{\ccol}p{\spellcol}}}
        & \multicolumn{9}{c}{\tb{---{}---{}---{}---{}---{}---{}---{}---Spell Level---{}---{}---{}---{}---{}---{}---{}---}} \\
        \hline
        \tb{Level} & \tb{1st} & \tb{2nd} & \tb{3rd} & \tb{4th} & \tb{5th} & \tb{6th} & \tb{7th} & \tb{8th} & \tb{9th} \\
        1st  & 1 & \x & \x & \x & \x & \x & \x & \x & \x \\
        2nd  & 2 & \x & \x & \x & \x & \x & \x & \x & \x \\
        3rd  & 3 & \x & \x & \x & \x & \x & \x & \x & \x \\
        4th  & 3 & 1  & \x & \x & \x & \x & \x & \x & \x \\
        5th  & 4 & 2  & \x & \x & \x & \x & \x & \x & \x \\
        6th  & 4 & 2  & 1  & \x & \x & \x & \x & \x & \x \\
        7th  & 4 & 3  & 2  & \x & \x & \x & \x & \x & \x \\
        8th  & 4 & 3  & 2  & 1  & \x & \x & \x & \x & \x \\
        9th  & 4 & 3  & 3  & 2  & \x & \x & \x & \x & \x \\
        10th & 4 & 3  & 3  & 2  & 1  & \x & \x & \x & \x \\
        11th & 4 & 3  & 3  & 3  & 2  & \x & \x & \x & \x \\
        12th & 4 & 3  & 3  & 3  & 2  & 1  & \x & \x & \x \\
        13th & 4 & 3  & 3  & 3  & 3  & 2  & \x & \x & \x \\
        14th & 4 & 3  & 3  & 3  & 3  & 2  & 1  & \x & \x \\
        15th & 4 & 3  & 3  & 3  & 3  & 3  & 2  & \x & \x \\
        16th & 4 & 3  & 3  & 3  & 3  & 3  & 2  & 1  & \x \\
        17th & 4 & 3  & 3  & 3  & 3  & 3  & 2  & 2  & \x \\
        18th & 4 & 3  & 3  & 3  & 3  & 3  & 2  & 2  & 1  \\
        19th & 4 & 3  & 3  & 3  & 3  & 3  & 2  & 2  & 2  \\
        20th & 4 & 3  & 3  & 3  & 3  & 3  & 2  & 2  & 2
    \end{dtabularx}
\end{dtable}

\cf{Wiz}{Cantrip}
Cantrips are minor spells which do not require effort to use.
A wizard chooses one cantrip from the list of cantrips on \pref{Cantrip List}.
She may use the cantrip at will.
Cantrips cannot be miscast.
For all other purposes, cantrips are treated as 0th level spells.
Specialist wizards must choose one of the cantrips granted by their specialist school.

At her 2nd wizard level, the wizard gains a second cantrip, which can be chosen from any non-prohibited school.

\cf{Wiz}[2nd]{Magic Feat}
The wizard gains a bonus magic feat or metamagic feat of her choice.
She must meet the prerequisites for the feat as normal.
At her 6th wizard level, and every 4 wizard levels thereafter, she gains an additional magic feat or metamagic feat.

\cf{Wiz}[2nd]{Specialization}
The wizard may choose to specialize in a particular school of magic.
Specialist wizards are able to learn all restricted spells from their chosen school, and gain additional spells known.
However, they can never access restricted spells from other schools, and must choose two other spell schools to ban.
A specialist wizard can never learn or cast spells or rituals from their banned schools.
Divination cannot be chosen as a banned school.

\cf{Wiz}[3rd]{Arcane Insight}[Ex]\label{Arcane Insight}
The wizard gains a greater understanding of magic.
A generalist wizard adds a restricted spell to her personal spell list.
The spell may be of any school, but she must still spend a spell known to learn it.

A specialist wizard gains an additional spell known.
The spell must be from her chosen school, including restricted spells.

In either case, the spell's level cannot exceed half her wizard level -- normally, the highest level of spells that she can cast.
At her 5th wizard level, and every odd wizard level thereafter, the wizard gains a new arcane insight.

\cf{Wiz}[4th]{Cantrip Sequencer}[Ex]
The wizard gains the ability to create a sequence of a spell and cantrip which she can cast together later.
To create an cantrip sequencer, the wizard must cast a spell which affects only herself and a cantrip, which may affect any target.
The spell must have a casting time of 1 standard action or less.
Neither has any effect immediately.
The wizard may later use a full-round action to cast both the spell and the cantrip at once, choosing the target of the cantrip at that time.
\par The wizard may initially have only one sequencer active at any time.
At her 8th wizard level, and every 4 wizard levels thereafter, she may keep an additional sequencer active at once.
If she creates a new sequencer, it replaces one of her previous sequencers.
She may choose which old sequencer is replaced.

\cf{Wiz}[8th]{Defensive Sequencer}[Ex]
The wizard gains the ability to create a sequence of two spells which she can cast rapidly later.
To create a defensive sequencer, the wizard must cast two spells, one of which affects only herself.
The spells must be at least three levels apart, and both must have a casting time of 1 standard action or less.
Neither has any effect immediately.
The wizard may later use a full-round action to cast both spells at once, resolving their effects separately.
The number of sequencers a wizard may have active at once is limited, as described in the cantrip sequencer ability.

\cf{Wiz}[12th]{Contingency}[Ex]
The wizard gains the ability to prepare a spell so it takes effect automatically if specific circumstances arise.
To prepare a contingency, the wizard must spend 1 minute preparing the spell, which consumes the a spell slot two levels higher than the spell's level.
During this casting time, the wizard specifies what circumstances cause the spell to take effect.

The contingency can be set to trigger in response to any circumstances that a typical human observing the wizard and her situation could detect.
For example, a wizard could specify ``when I fall at least 50 feet'' or ``when I become bloodied,'' but not ``when there is an invisible creature within 50 feet of me'' or ``when I am at 17 hit points or fewer.'' The more specific the required circumstances, the better -- vague requirements, such as ``when I am in danger,'' may cause the contingency to trigger unexpectedly or fail to trigger at all.
If a wizard attempts to specify multiple separate triggering conditions, such as ``when I take damage or when an enemy is adjacent to me,'' the contingency will randomly ignore all but one of the conditions.

The spell must have a casting time of 1 standard action or less, and it must target the wizard or have its area centered on the wizard.
Any spells which require decisions, such as \spell{dimension door}, must have those decisions made at the time the contingency is created.
The wizard cannot change those decisions when the contingency takes effect.

A wizard can have only one contingency active at a time.
If she creates another contingency, it replaces her old contingency.

\cf{Wiz}[16th]{Offensive Sequencer}[Ex]
The wizard gains the ability to create a sequence of two spells which she can cast rapidly later.
To create an offensive sequencer, the wizard must cast two spells.
The spells must be at least three levels apart, and both must have a casting time of 1 standard action or less.
One spell must be a damaging spell, and the other must not.
Neither has any effect immediately.
The wizard may later use a full-round action to cast both spells at once, resolving their effects separately.
The number of sequencers a wizard may have active at once is limited, as described in the cantrip sequencer ability.

\cf{Wiz}[20th]{Chain Contingency}[Ex]
The wizard may ready a sequencer in her contingency instead of a single spell.
This sequencer counts against her limit of available sequencers.

\section{Character Advancement}\label{Character Advancement}

As your character accomplishes challenges and defeats foes, he gains experience.
If your character has enough experience, he gains a level.
When you gain a level, you can increase your character's level in your current class or in any other class, and gain the benefits described for each class.
Rules for taking levels in multiple classes are described in \pref{Multiclass Characters}, below.

A character that increases in level gains additional benefits.
\begin{itemize}
    \item Every odd level, including 1st level, you gain a feat (see \pcref{Feats}).
    \item Every level after 1st level, you increase two different attributes of your choice by one.
    \item Every 5th level, you also increase all of your other attributes by one.
    \item At 4th level, and every 4 levels thereafter, you gain a \glossterm{legend point}.
\end{itemize}

If a character has multiple classes, these benefits are gained based on the total character level, not based on class level.
The experience required to reach a level, and the benefits gained at each level, are shown on \trefnp{Character Advancement}.

\begin{dtable}
    \lcaption{Character Advancement}
    \begin{dtabularx}{\columnwidth}{*{4}{>{\lcol}l} >{\lcol}X}
        \tb{Level} & \tb{XP} & \tb{Feats} & \tb{Attribute Increases} & \tb{Legend Points} \\
        \hline
        1st  & 0         & 1st  & \x              & \x  \\
        2nd  & 2,000     & \x   & \plus1 to two   & \x  \\
        3rd  & 5,000     & 2nd  & \plus1 to two   & \x  \\
        4th  & 9,000     & \x   & \plus1 to two   & 1st \\
        5th  & 15,000    & 3rd  & All gain \plus1 & \x  \\
        6th  & 23,000    & \x   & \plus1 to two   & \x  \\
        7th  & 35,000    & 4th  & \plus1 to two   & \x  \\
        8th  & 51,000    & \x   & \plus1 to two   & 2nd \\
        9th  & 75,000    & 5th  & \plus1 to two   & \x  \\
        10th & 105,000   & \x   & All gain \plus1 & \x  \\
        11th & 155,000   & 6th  & \plus1 to two   & \x  \\
        12th & 220,000   & \x   & \plus1 to two   & 3rd \\
        13th & 315,000   & 7th  & \plus1 to two   & \x  \\
        14th & 445,000   & \x   & \plus1 to two   & \x  \\
        15th & 635,000   & 8th  & All gain \plus1 & \x  \\
        16th & 890,000   & \x   & \plus1 to two   & 4th \\
        17th & 1,300,000 & 9th  & \plus1 to two   & \x  \\
        18th & 1,800,000 & \x   & \plus1 to two   & \x  \\
        19th & 2,550,000 & 10th & \plus1 to two   & \x  \\
        20th & 3,600,000 & \x   & All gain \plus1 & 5th \\
    \end{dtabularx}
\end{dtable}

\section{Multiclass Characters}\label{Multiclass Characters}
A character may add new classes as he or she progresses in level, thus becoming a multiclass character.
The class abilities from a character's different classes combine to determine a multiclass character's overall abilities.
Multiclassing improves a character's versatility at the expense of focus.

\subsection{Class And Level Features}
As a general rule, the abilities of a multiclass character are the sum
of the abilities of each of the character's classes.

\parhead{Level}
``Character level'' is a character's total number of levels.
It is used to determine when feats and attribute score boosts are gained, as noted on \tref{Character Advancement}.
Whenever a creature's ``level'' is specified, without reference to a particular class, the character level is used.

\par ``Class level'' is a character's level in a particular class.
For a character whose levels are all in the same class, character level and class level are the same.

\parhead{Hit Points}
The normal rules for determining hit points apply to multiclass characters.

\parhead{Combat Prowess}

\begin{enumerate}
    \item For each type of combat prowess progression the character has, sum the character's level in classes with that combat prowess progression.
    \item Determine the character's fastest type of combat prowess progression. Good is faster than Average, which is faster than Poor.
    \item Add (levels in Good progression) + (four-fifths levels in Average progression) + (two-thirds levels in Poor progression). Each part of this calculation rounded down separately.
    \item If the character's fastest combat prowess progression is Good or Average, add 2 to the result. If it is Poor, add 1.
    \item The result is the character's combat prowess.
\end{enumerate}

\parhead{Example}
Wissy, a 2nd-level rogue/2nd-level wizard, would have a combat prowess of 4.
\begin{enumerate}
    \item Wissy has 2 levels in the Average progression, and 2 levels in the Poor progression.
    \item Wissy's fastest progression type is Average.
    \item Four-fifths of 2 is 1, and two-thirds of 2 is 1, so the total is 2.
    \item Wissy's Average progression type adds 2 to the result, for a total of 4.
    \item Wissy's combat prowess is therefore 4.
\end{enumerate}

\parhead{Defenses}

For each defense, add the base defense bonuses granted by each class together.
The result is the character's total base defense bonus for that defense.

\parhead{Skills}
A multiclass character gains all class skills from all of his classes.
However, only the character's base class grants skill points.
If a character has multiple base classes, he must choose which base class grants skill points.

\parhead{Class Features}
A multiclass character gets all the class features of all his or her classes, but must also suffer the consequences of the special restrictions of all his or her classes.

\par In some cases, two classes can have virtually identical abilities.
Use the following guidelines to determine how abilities stack.
\begin{itemize}
    \item If two identical class features are not based on level and are not gained following a specific pattern, they do not stack.
    \item If two identical class features are not explicitly based on level, but both classes gain them in a predictable pattern, the levels of the two classes stack for determining when the next improvement to the class feature will be gained.
    \item If two identical class features are explicitly based on level, the levels of the two classes stack for determining the power of the ability.
    \item If two identical class features say how they stack, those rules trump any other rules.
\end{itemize}
These are some examples of how to use these guidelines.
\begin{itemize}
    \item Both a druid and a ranger gain wild speech.
        A druid/ranger who has wild speech from both classes has the same wild speech ability as a druid or ranger would.
    \item Both a barbarian and a rogue get uncanny dodge and improved uncanny dodge at the same level.
        A barbarian/rogue adds his barbarian and rogue levels together to determine when he acquires improved uncanny dodge.
\end{itemize}

\parhead{Weapon and Armor Proficiency}
Only a character's base class grants weapon and armor proficiencies.
If a character has multiple base classes, she must choose which base class grants weapon and armor profiencies.

\subsection{Spellcasters and Multiclassing}\label{Spellcasters and Multiclassing}
The character gains spells from all of his or her spellcasting classes separately and tracks his spells per day, spells known, and spellpower separately with each class.

Characters with magical ability gain a special benefit when multiclassing.
Such a character must choose a specific spellcasting class he has.
For every two levels that a character has in nonmagical classes, up to the number of levels he has in his chosen spellcasting class, he increases his spellcasting ability with that class.
This increases his spells per day, spells known, and spellpower as if he had gained a level in his chosen spellcasting class.
No class features or other abilities can be gained in this way.


For example, Gish, a 2nd level fighter / 2th level wizard, would have the spells per day, spells known, and spellpower of a 3rd level wizard.
If he gained two more fighter levels, his spellcasting ability would not increase.
