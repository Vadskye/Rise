% This package uses a different parsing argument for PDF and HTML output
\usepackage[pdftex,bookmarks=true,breaklinks=true,colorlinks=true,linkcolor=MidnightBlue,bookmarksopen=true,bookmarksopenlevel=1,bookmarksdepth=1]{hyperref}

\ExplSyntaxOn

\setcounter{secnumdepth}{0}

%make bigger section headers
% \titleformat*{\section}{\hspace*{-0.5em}\LARGE\bfseries}
\titleformat{\section}
    {\LARGE\bfseries}   % The style of the section title
    {Part \arabic{section}:}                             % a label prefix
    {0.75em}                          % How much space exists between the prefix and the title
    {\hspace{-0.5em}}    % A code representing the section
\titleformat{\subsection}
    {\large\bfseries}   % The style of the section title
    {\arabic{subsection}.}                             % a label prefix
    {0.75em}                          % How much space exists between the prefix and the title
    {\hspace{-0.5em}\vspace{-0.75em}}    % A code representing the section
    [\hspace{-0.5em}\titlerule]
\titleformat{\subsubsection}
    {\large\bfseries}
    {}
    {1em}
    {\hspace{-0.5em}}

% The HTML versions of mdframed environments need dumb hacks to fix parsing issues
\newmdenv[
    style=spellcontent,
    leftline=true,
    topline=true,
    rightline=true,
    bottomline=true,
]{spellcontent}


% The HTML versions of mdframed environments need dumb hacks to fix parsing issues
\newmdenv[
    backgroundcolor=LightCyan,
    style=colorenv,
    leftline=true,
    topline=true,
    rightline=true,
    bottomline=true,
    skipabove=4pt,
    innertopmargin=0.5em,
    innerbottommargin=0.25em,
    leftmargin=-0.5em,
]{monsterstatistics}

\newmdenv[
    style=colorenv,
    backgroundcolor=LightCyan,
    topline=true,
]{spelltargetinginfo}

\newmdenv[
    style=colorenv,
    backgroundcolor=LightCyan,
]{augmenttargetinginfo}

\newmdenv[
    style=colorenv,
    backgroundcolor=Lavender,
]{spelleffects}

\newmdenv[
    style=colorenv,
    backgroundcolor=Lavender,
]{augmenteffects}

\newmdenv[
    style=colorenv,
    backgroundcolor=Gainsboro,
    leftline=true,
    rightline=true,
]{spellfooter}

\newmdenv[
    style=colorenv,
    backgroundcolor=Gainsboro,
    leftline=true,
    rightline=true,
    bottomline=true,
    topline=true,
]{monsterfootermdframed}
\DeclareDocumentEnvironment{monsterfooter}{}
{
    \begin{monsterfootermdframed}
    \begin{fakehang}
}
{
    \end{fakehang}
    \end{monsterfootermdframed}
}

% The HTML version of spelltwocol has to put its tabularx inside of a figure
\DeclareDocumentCommand{\spelltwocol}{m m}
{
    \par\noindent
    \begin{tabularx}{\columnwidth}{@{}~>{\lcol}X~l~@{}}
        \renewcommand{\arraystretch}{0.5}
        #1 & #2
    \end{tabularx}
}

% The HTML version of termlink can't use mbox
% args:
%   #1: reference name of term, if different from display name
%   #2: display name of term
%   #3: hyperlink prefix
\DeclareDocumentCommand{\termlink}{o m m}
{
    \IfValueTF{#1}
    {
        % Enable this to test for missing glossary definitions
        % \pref{#1}
        % Enable this to test for unnecessary glossary entries
        % \label{#1}

        % TODO: remove \mbox before making major releases;
        % it fixes bizarre page break errors but makes text uglier
        \hyperlink{#3:#1}{\textbf{\mbox{#2}}}
    }
    {
        % Enable this to test for missing glossary definitions
        % \pref{#2}
        % Enable this to test for unnecessary glossary entries
        % \label{#2}

        \hyperlink{#3:#2}{\textbf{\mbox{#2}}}
    }
}

% The HTML versions of pref/pcref don't want to reference pages at all
\DeclareDocumentCommand{\pcref}{o m}
{
    \IfValueTF{#1}
    {
        #2,~page~\pageref{#1}
    }
    {
        #2,~page~\pageref{#2}
    }
}
\newcommand{\pref}[1]{page~\pageref{#1}}
\newcommand{\tref}[1]{Table~\ref{cap:#1}:~#1,~\pref{cap:#1}}
\newcommand{\trefnp}[1]{Table~\ref{cap:#1}:~#1}
\newcommand{\featpref}[1]{\pageref{feat:#1}}
\newcommand{\mitempref}[1]{\pageref{item:#1}}
\newcommand{\featpcref}[1]{#1,~page~\pageref{feat:#1}}

% small text for the feats table
\newenvironment{longtablewrapper}
{
    \onecolumn
    \small
    \rowcolors{1}{white}{tbrown}
    \renewcommand*{\arraystretch}{1.2}
    \setlength\LTleft{0pt}
    \setlength\LTright{0pt}
}
{\twocolumn}

\ExplSyntaxOff
