\section{Monsters}

\subsection{Dire Animals}
\subsubsection{Dire Bear}
\parhead{Level} 12
\parhead{Archetypes} Brute 6, Disabler 1, Tank 5
\parhead{Attributes} 
\parhead{Traits} (4+12+6) Natural Weapon I x2, Natural Weapon II x2, Natural Armor I, Natural Armor II x2, Senses I x2, 

\subsubsection{Dire Lion}
\parhead{Level} 10
\parhead{Archetypes} Brute 7, Disabler 2, Tank 1
\parhead{Attributes} 5, 6, 4, \minus8, 1, 0
\parhead{Traits} (4+10+5) Natural Weapon I x2, Natural Weapon II x2, Natural Armor I, Senses I x2, Pounce, Area Debuff (shaken), Widen Area Debuff x2, Attribute (Dex) x3, HV Increase, Attribute (Str) x2, Attribute (Con), Size Increase

\subsubsection{Dire Wolf}
\parhead{Level} 8
\parhead{Archetypes} Brute 6, Disabler 1, Tank 1
\parhead{Attributes} 7, 2, 4, \minus8, 1, 0
\parhead{Traits} (4+8+4) Natural Weapon I, Natural Weapon II x2, Natural Armor I, Natural Armor II x2, Improved Trip, Size Increase, Senses I x2, Attribute (Strength) x4, Attribute (Con) x1, HV Increase

\subsection{Animals}

\subsubsection{Bear, Black}
\parhead{Level} 3
\parhead{Archetypes} Brute 1, Disabler 1, Tank 1
\parhead{Attributes} 4/0/5/\minus8/2/0
\parhead{Traits} (2+5) Natural Weapon I x2, Natural Armor I, Attr (Str), Attribute (Con x2), Improved Grab

\subsubsection{Bear, Brown}
\parhead{Level} 6
\parhead{Archetypes} Brute 2, Disabler 1, Tank 3
\parhead{Attributes} 6/0/6/\minus8/3/0
\parhead{Traits} (2+9) Natural Weapon I x2, Natural Armor I, NA II, Attr (Str), Attribute (Con x3), Attr (Wis), Improved Grab, Size Increase (Large)

\section{Monster Design}

Creating a monster involves following a series of steps. These steps need not be followed in the order given, as long as all of them are done.

\begin{enumerate*}
    \item Choose a level for the monster
    \item Choose a creature type (Animal, Dragon, etc.)
    \item Choose one or more archetypes (Warrior, Scout, etc.)
    \item Assign attributes
    \item Choose traits (special abilities unique to monsters)
    \item Choose skills and feats
    \item Choose equipment (if any)
\end{enumerate*}

\subsection{Monster Level}
A monster's level affects many things. Like a PC, a monster's level affects its core statistics, such as hit points, base attack bonus, and saving throws. In addition, monsters gain a number of special abilities based on their level, like class features for PCs.

\subsection{Creature Types}
\parhead{HV} Hit points gained at each level.
\parhead{BAB} Base attack bonus progression (Good, Average, or Poor).
\parhead{Saving Throws} Saving throw progressions (Good, Average, or Poor).
\parhead{Attributes} The number of points the creature can spend on its attributes. If the creature has a special modifier to its attributes, such as the low Intelligence of animals, that is also listed here.
\parhead{Automatic Traits} Many creature types automatically grant certain traits. These are listed here. Individual creatures may not have these traits if appropriate.

\subsubsection{Aberration}
\parhead{HV} 5
\parhead{BAB} Average
\parhead{Saving Throws} Average Fort, Ment; poor Ref
\parhead{Attributes} 10 points
\parhead{Automatic Traits} Natural Weapon, Natural Armor

\subsubsection{Animal}
\parhead{HV} 5
\parhead{BAB} Average
\parhead{Saving Throws} Average Fort, Ref; poor Ment
\parhead{Attributes} 10 points, -8 Intelligence (max -5)
\parhead{Automatic Traits} Natural Weapon, Natural Armor, Senses (Low-light vision, Scent)

\subsection{Archetypes}
Archetypes are thematic representations of the powers a creature has and the role it plays in a typical combat. For example, a Warrior is more powerful in physical combat, while a Brute is more difficult to kill. A creature may have any number of archetypes.

Each archetype has two main effects. First, archetypes often change a creature's base progressions for attacks, saving throws, or hit points. Second, each archetype has an associated list of traits called a ``trait pool''. Creatures gain bonus traits from each of the archetypes they have, chosen from the traits in that archetype's trait pool.

For each archetype the creature has, it gains trait points equal to its level which can be spent on traits from that archetype's trait pool.

\subsubsection{Brute}
\parhead{BAB} Increases by one step
\parhead{Saving Throws} Fort increases by one step
\parhead{Trait Pool}
\begin{itemize}
    \item Attribute (Str, Con)
    \item Combat Feat (Power)
    \item Natural Weapon I, II
    \item Poison I, II
    \item Size Increase
    \item Trample
\end{itemize}

\subsubsection{Disabler}
\parhead{Trait Pool}
\begin{itemize}
    \item Improved Grab
\end{itemize}

\section{Monster Traits}
Traits are unique abilities that monsters gain. A monster has a number of trait points equal to its level, which it can use to acquire traits. Most traits cost a single trait point to acquire.

A monster with an archetype gains additional trait points equal to half its level (minimum 1). These trait points can only be spent on traits from that archetype's trait pool. If a creature has multiple archetypes, it gains and spends trait points separately for each archetype.

For example, a 5th level monster with the Scout and Warrior archetypes has nine trait points: five which can be spent on any trait, two for traits from the Scout trait pool, and two for traits from the Warrior trait pool.

Some traits can be taken multiple times. Unless otherwise stated, no trait can be taken more times than half your level (minimum 1).

Traits come in two main groups: active traits, which grant new attacks or special powers the monster can use, and passive traits, which improve the monster's statistics.

\subsection{Active Traits}

\subsubsection{Natural Grab}
Choose one of your natural weapons.
\featpre Natural Weapon
\featben When you hit a creature with your chosen natural weapon, you may make a grapple attempt as an immediate action. The creature must be at least one size category smaller than you.

\subsubsection{Natural Grab, Improved}
\featpre Natural Grab, Natural Weapon.
\featben You can use your natural grab ability on creatures of your size category or smaller.

\subsubsection{Natural Poison}
Choose one of your natural weapons.
\featpre Natural Weapon.
\featben As a swift action, you can ready your chosen natural weapon to inflict poison. The next time you deal damage with it, the struck creature must make a Fortitude save or be afflicted with poison. The DR for the save is equal to 10 \add 1/2 your level \add Constitution.
\par The poison deals 1 damage each round to an attribute of your choice (other than Constitution). It continues until the subject makes two successful Fortitude saves against the poison.

\subsubsection{Natural Poison, Extended}
\featpre Natural Poison, Natural Weapon
\featben Your poison lasts until the subject makes three successful Fortitude saves.

\subsubsection{Natural Poison, Improved}
\featpre Natural Poison, Natural Weapon.
\featben Your poison deals 1d4 damage per round to an attribute other than Constitution, or 1 damage per round to Constitution.

\subsubsection{Trample}
\featpre Large size or larger.
\featben As a full-round action, you can move up to twice your speed in a straight line and literally run over any creature at least one size category smaller than you without being impeded. Any creature whose space is completely covered by your own at any point during this movement takes damage.

This attack deals 1d10 damage \add your Strength for a Large creature, and the save DR is equal to 10 \add your level \add Strength.

\subsection{Passive Traits}

\subsubsection{Attribute}
Choose an attribute.
\featben You gain a \plus1 inherent bonus to the appropriate attribute.
\parhead{Special} This trait can be taken any number of times. Its effects stack. You can choose a different attribute or the same attribute each time. However, your bonus with a single attribute cannot exceed half your level (minimum 1).

\subsubsection{Damage Reduction}
\featben You gain damage reduction equal to your HV \add Constitution against any of the damage types listed below. This damage reduction can overcome by your choice of a different one of these types.
\begin{itemize}
    \item Aligned damage, single (chaotic, good, evil, lawful)
    \item Energy damage, single (acid, cold, electricity, or fire)
    \item Physical damage, single (bludgeoning, slashing, piercing, nonmagical)
\end{itemize}
%How to resist life, solar damage? How to have multiple ``overcome by''?
\parhead{Advanced} For an additioanl trait point, your damage reduction can instead resist any of the following damage types.
\begin{itemize}
    \item Energy damage, all
    \item Physical damage, all
    \item Spell damage
\end{itemize}

\subsubsection{Natural Armor}
\featben You gain a \plus2 inherent bonus to your natural armor modifier.
\parhead{Special} You can take this trait multiple times. Each time, you gain an additional \plus1 inherent bonus to your natural armor.

\subsubsection{Natural Weapon}
\featben You gain a natural weapon from the following list. The natural weapon deals damage appropriate for your size. The damage listed is for Medium creatures.
\begin{itemize}
    \item Bite (d8)
    \item Claws (d6 main, d6 flurry)
    \item Slam (d8)
\end{itemize}

\subsubsection{Natural Weapon, Improved}
Choose a natural weapon.
\featpre Natural Weapon.
\featben You increase the damage die of one of your natural weapons by one category.
\parhead{Special} You can take this trait multiple times. Each time, you can apply it to a different natural weapon or to the same natural weapon. You cannot apply this trait to the same weapon more times than half your Constitution. 

\subsubsection{Sense}
\featben You gain one of the following senses.
\begin{itemize}
    \item Blindsense 50 ft.
    \item Darkvision 50 ft.
    \item Low-light vision
    \item Scent
    \item Tremorsense 50 ft.
\end{itemize}

\parhead{Special} This trait can be taken multiple times. Each time, you may select a different sense.
\parhead{Advanced} For an additional trait point, you can instead gain one of the following senses.
\begin{itemize}
    \item Blindsight 30 ft.
\end{itemize}

\subsubsection{Sense, Improved}
Choose one of your senses from the Sense trait with a range.
\featpre Sense.
\featben You double the range of your chosen sense.

\subsubsection{Size Increase}
\featpre Medium size or smaller, Attribute (Strength) x2.
\featben Your size increases by one category.
\parhead{Special} You can take this trait multiple times. You must have Attribute (Strength) x2 to become Large, x4 to become Huge, x6 to become Gargantuan, and x8 to become Colossal.

\subsection{Templates}
\subsubsection{Dire}
\featpre Animal type
\featben Base attack bonus progression improves by 1 step. HV improves by 1.
