
\spellsection{Align Weapon}{2}
\begin{spellheader}
    \spelldesc{You enhance a weapon while bringing it closer to your ideals.}
\end{spellheader}
\begin{spellcontent}
    \begin{spelltargetinginfo}
        \spelltwocol{\spelltgt{One weapon or group of fifty projectiles}}{\spellrng{\rngtouch}}
    \end{spelltargetinginfo}
    \begin{spelleffects}
        \spelleffect The target weapon become good, evil, law, or chaos-aligned, as you choose. Attacks with the weapon against creatures of the opposite alignment gain a magic bonus to attack and damage equal to 2 \add 1 per 6 caster levels. If the weapon already had an alignment, it is suppressed for the duration of the spell.
    \end{spelleffects}
\end{spellcontent}
\begin{spellfooter}
    \spellinfo{Evocation/Transmutation (Augment, Channeling) [see text]}*{Chaos, Evil, Good, Law}
    \spellnotes This spell cannot change the alignment of intelligent items. When you make a weapon good, evil, lawful, or chaotic, this spell is a good, evil, lawful, or chaotic spell, respectively.
\end{spellfooter}

\spellsection{Animate Objects}{5}
\begin{spellheader}
    \spelldesc{You imbue inanimate objects with mobility and a semblance of life.}
\end{spellheader}
\begin{spellcontent}
    \begin{spelltargetinginfo}
        \spelltwocol{One Small object/level in the area}{\spellrng{\rngmed}}
    \end{spelltargetinginfo}
    \begin{spelleffects}
        \spelleffect Each animated object immediately attacks whomever or whatever you initially designate. Your control of the objects is limited to simple commands (``Attack,'' ``Defend,'' ``Stop,'' and so forth).
        \par An animated object can be of any nonmagical material. You may animate one Small or smaller object or an equivalent number of larger objects per caster level. A Medium object counts as two Small or smaller objects, a Large object as four, a Huge object as eight, a Gargantuan object as sixteen, and a Colossal object as thirty-two. You can give the objects new commands as a move action, as normal for directing an active spell.
        \spelldur{\durshort}
    \end{spelleffects}
\end{spellcontent}
\begin{spellfooter}
    \spellinfo{Transmutation (Animation)}{Chaos, Transmutation}
    \spellnotes This spell cannot animate objects carried or worn by a creature. This spell can be made permanent with a \spell{permanency} ritual.
\end{spellfooter}

\spellsection{Animate Plants}{5}
\begin{spellheader}
    \spelldesc{You imbue inanimate plants with mobility and a semblance of life.}
\end{spellheader}
\begin{spellcontent}
    \begin{spelltargetinginfo}
        \spelltgts{Up to one Small plant/level in the area; see text}
    \end{spelltargetinginfo}
    \begin{spelleffects}
        \spelleffect This spell functions like \spell{animate objects}, except that you animate plants instead of inanimate objects.
    \end{spelleffects}
\end{spellcontent}
\begin{spellfooter}
    \spellinfo{Transmutation (Animation)}{Nature, Plant}
    \spellsr{No}
    \spellnotes \spell{Animate plants} cannot affect plant creatures, nor does it affect nonliving vegetable material.
\end{spellfooter}


\spellsection{Circle of Death}{6}
\begin{spellheader}
    \spelldesc{You snuff out the life force of your weakened foes by flooding them with negative energy.}
\end{spellheader}
\begin{spellcontent}
    \begin{spelltargetinginfo}
        \spelltwocol{\spelltgts{Several bloodied living creatures}}{\spellrng{\rngmed}}
    \end{spelltargetinginfo}
    \begin{spelleffects}
        \spelleffect This spell affects all \bloodied living creatures in the area, starting with the creature with the lowest level, until it affects a total number of levels equal to twice your caster level. Among creatures with equal levels, those closest to the burst's point of origin are affected first. No creature whose level is greater than half your caster level can be affected, and levels that are not sufficient to affect a creature are wasted. Healthy creatures are not affected by this spell, and do not count against its level limit.
        \begin{spellattack}{Magic vs. Fortitude}
            \spellsuccess If the target is bloodied, it is reduced to 0 hit points and takes critical damage equal to your caster level, causing it to begin dying.
        \end{spellattack}
        \spellmat{The powder of a crushed black pearl with a minimum value of 100 gp.}
    \end{spelleffects}
\end{spellcontent}
\begin{spellfooter}
    \spellinfo{Necromancy (Vitalism) [Death, Negative]}{Death, Divine}
\end{spellfooter}

\spellsection{Clenched Fist}{9}
\begin{spellheader}
    \spelldesc{You create a floating, disembodied hand made of magical force that strikes your foe.}
\end{spellheader}
\begin{spellcontent}
    \begin{spelltargetinginfo}
        \spelltwocol{\spelltgt{One creature}}{\spellrng{\rngmed}}
    \end{spelltargetinginfo}
    \begin{spelleffects}
        \spelleffect This spell creates a hand, as \spell{interposing hand}, except that the hand attacks its target instead of protecting you from it.
        \begin{spellattacktriggered}{During each action phase, make a physical attack against Armor defense. Your attack bonus is equal to your caster level \add your casting attribute.}
            \spellsuccess 2d10 force damage \add half casting attribute.

            If the target is \bloodied after the damage is dealt, you make an additional attack.
            \begin{spellattack}{Magic vs. Fortitude}
                \spellsuccess The target is \dazed for 1 round.
            \end{spellattack}
        \end{spellattack}
        \spelldur{\durshort \dismissable}
    \end{spelleffects}
\end{spellcontent}
\begin{spellfooter}
    \spellinfo{Evocation (Control) [Force]}{Evocation, Strength}
    \spellnotes As \spell{interposing hand}. The hand attacks during the action phase, regardless of when you direct it to attack a target.
\end{spellfooter}

\spellsection{Crushing Hand}{8}
\begin{spellheader}
    \spelldesc{You create a floating, disembodied hand made of magical force that crushes your foe in its grasp.}
\end{spellheader}
\begin{spellcontent}
    \begin{spelltargetinginfo}
        \spelltwocol{\spelltgt{One creature}}{\spellrng{\rngmed}}
    \end{spelltargetinginfo}
    \begin{spelleffects}
        \spelleffect This spell creates a hand, as \spell{interposing hand}, except that the hand grapples its target instead of protecting you from it.
        \begin{spellattack}{Caster level \add casting attribute vs. Maneuver defense}
            \spellsuccess The target is grappled. It takes 2d6 bludgeoning damage \add half your casting attribute.
        \end{spellattack}
        \spelldur{\durshort \dismissable}
    \end{spelleffects}
\end{spellcontent}
\begin{spellfooter}
    \spellinfo{Evocation (Control) [Force]}{Evocation}
    \spellnotes As \spell{interposing hand}. The hand attacks during the action phase, regardless of when you direct it to attack a target.
\end{spellfooter}

\spellsection{Grasping Hand}{6}
\begin{spellheader}
\end{spellheader}
\begin{spellcontent}
    \begin{spelltargetinginfo}
        \spelltwocol{\spelltgt{One creature or object}}{\spellrng{\rngmed}}
    \end{spelltargetinginfo}
    \begin{spelleffects}
        \spelleffect This spell creates a hand, as \spell{interposing hand}, except that the hand grapples its target instead of protecting you from it.
        \begin{spellattack}{Caster level \add casting attribute vs. Maneuver defense}
            \spellsuccess The target is grappled. The hand deals no damage.
        \end{spellattack}
        \spelldur{\durshort \dismissable}
    \end{spelleffects}
\end{spellcontent}
\begin{spellfooter}
    \spellinfo{Evocation (Control) [Force]}{Arcane}
    \spellnotes As \spell{interposing hand}. The hand attacks during the action phase, regardless of when you direct it to attack a target.
\end{spellfooter}

\spellsection{Interposing Hand}{2}
\begin{spellheader}
    \spelldesc{You create a massive hand from thin air that blocks your foe's attacks.}
\end{spellheader}
\begin{spellcontent}
    \begin{spelleffects}
        \spelleffect This spell creates a floating, disembodied hand made of magical force. Each round, as a swift action, you can direct the hand to protect you from a target. If you do not direct the hand, it remains motionless.

        The hand provides you with active cover from the target. Each physical attack the target makes against you has a 20\% chance to strike the hand instead. In addition, if the target is Large size or smaller, it moves at half speed while moving towards you.

        The hand is 10 feet long and about that wide with its fingers outstretched. It has 3 hit points per caster level, an Armor defense of 10 \add half your caster level, and a Maneuver defense of 10 \add your caster level \add your casting attribute. Most effects that don't affect objects do not affect the hand.
        \spelldur{\durshort \dismissable}
    \end{spelleffects}
\end{spellcontent}
\begin{spellfooter}
    \spellinfo{Evocation (Control) [Force]}{Arcane}
    \spellnotes The hand can move up to 50 feet per round. Since the hand is directed by you, its ability to interact with invisible or concealed creatures is no better than yours. Its special defenses are the same as your special defenses. If the hand goes out of range of you, it winks out.
\end{spellfooter}

\spellsection{Cloudkill}{7}
\begin{spellheader}
    \spelldesc{You conjure a yellowish green fog bank that obscures vision and slowly poisons creatures inside.}
\end{spellheader}
\begin{spellcontent}
    \begin{spelltargetinginfo}
        \spelltwocol{\spellzone{\areamed radius cylinder}}{\spellrng{\rngmed}}
    \end{spelltargetinginfo}
    \begin{spelleffects}
        \spelleffect Fog in the area, as \spell{fog cloud}, except that the fog is mobile and poisonous.

        \par The fog moves away from you at 10 feet per round, rolling along the surface of the ground. Figure out the cloud's new area each round based on its new point of origin, which is 10 feet farther away from the point of origin where you cast the spell.
        \begin{spelltriggeredattack}{At the end of every round, make a Magic vs. Fortitude attack against everything in the area.}
            \spellsuccess 
        \end{spellattacktriggered}
    \end{spelleffects}
\end{spellcontent}
\begin{spellfooter}
    \spellinfo{Conjuration (Creation) [Fog, Poison]}{Arcane}
    \spellsr{No}
    \spellnotes As \spell{fog cloud}.

    Holding one's breath doesn't help against the poison, but creatures immune to poison are unaffected. Because the vapors are heavier than air, they sink to the lowest level of the land, even pouring down den or sinkhole openings. This spell cannot penetrate liquids, nor can it be cast underwater.
\end{spellfooter}

\spellsection{Combat Transformation}{7}
\begin{spellheader}
    \spellcmp{Verbal, Somatic, and Material}
\end{spellheader}
\begin{spellcontent}
    \begin{spelltargetinginfo}
        \spelltgt{You}
    \end{spelltargetinginfo}
    \begin{spelleffects}
        \spelleffect You gain 
        \spelleffect You gain a \plus3 enhancement bonus to Strength, Dexterity, Constitution, and Fortitude defense. This bonus increases to \plus4 at 14th caster level, and to \plus5 at 20th caster level. In addition, you gain proficiency with any weapons you hold (except exotic weapons).
        \spelldur{\durshort \dismissable}
    \end{spelleffects}
\end{spellcontent}
\begin{spellfooter}
    \spellinfo{Transmutation (Augment)}{Arcane}
    \spelldesc{You become a virtual fighting machine -- stronger, tougher, faster, and more skilled in combat. Your mind-set changes so that you relish combat instead of casting spells.}
    \spellnotes If you cast a spell or use a spell activation or spell completion magic item, the spell immediately ends.
    \spellmat{A potion of \spell{totemic power} (which costs 40 gp), which you drink (and whose effects are subsumed by the spell effects).}
\end{spellfooter}

\spellsection{Vestments of the Mage}{2}
\begin{spellheader}
    \spelldesc{You imbue a set of armor with magical power, preventing it from interfering with your spellcasting.}
    \begin{spelltargetinginfo}
        \spelltwocol{\spelltgt{One nonmagical armor or shield}}{\spellrng{Touch}}
    \end{spelltargetinginfo}
\end{spellheader}
\begin{spellcontent}
    \begin{spelleffects}
        \spelleffect The armor or shield's chance of arcane spell failure decreases by 10\% as long as you are wearing or using it. If any other creature wears the armor, it receives no benefit from this spell.
        \spelldur{\durext \dismissable}
    \end{spelleffects}
\end{spellcontent}
\begin{spellfooter}
    \spellinfo{Transmutation (Imbuement)}{Arcane}
    \spellnotes This decrease is considered an enhancement bonus.
\end{spellfooter}%complicated

\spellsection{Creeping Doom}{7}
\begin{spellheader}
    \spelldesc{You summon uncountable hordes of centipedes to overwhelm your foes.}
    \spelltime{Full-round action}
    \spellrng{\rngmed}
\end{spellheader}
\begin{spellcontent}
    \spelleffect This spell creates one centipede swarm per two caster levels. They must all be adjacent at least one other swarm. You may summon the centipede swarms so that they share the area of other creatures. The swarms remain stationary, attacking any creatures in their area, unless you command the creeping doom to move (a standard action). As a standard action, you can command any number of the swarms to move toward any prey within range of you. Any swarm out of range of you remains stationary, attacking any creatures in its area.
    \spelldur{\durmed}
\end{spellcontent}
\begin{spellfooter}
    \spellinfo{Conjuration (Summoning)}{Nature}
    \spellsr{No}
\end{spellfooter}

\spellsection{Control Water}{2}
\begin{spellheader}
    \spelldesc{You manipulate elemental forces to control water around you.}
\end{spellheader}
\begin{spellcontent}
    \begin{spelltargetinginfo}
        \spelltwocol{\spellzone{One 5 ft. cube/caster level}}{\spellrng{\rngfar}}
    \end{spelltargetinginfo}
    \begin{spelleffects}
        \spelleffect Depending on the version you choose, this spell raises or lowers water.
        \par \subspell{Lower Water} This causes water or similar liquid to reduce its depth by as much as 2 feet per caster level (to a minimum depth of 1 inch). The water is lowered within a squarish depression whose sides are up to caster level \mtimes 10 feet long. In extremely large and deep bodies of water, such as a deep ocean, the spell creates a whirlpool that sweeps ships and similar craft downward, putting them at risk and rendering them unable to leave by normal movement for the duration of the spell.
        \par \subspell{Raise Water} This causes water or similar liquid to rise in height, just as the lower water version causes it to lower. Boats raised in this way slide down the sides of the hump that the spell creates. If the area affected by the spell includes riverbanks, a beach, or other land nearby, the water can spill over onto dry land.
        \spelldur{\durmed \dismissable}
    \end{spelleffects}
\end{spellcontent}
\begin{spellfooter}
    \spellinfo{Evocation (Control) [Water]}{Nature, Water}
    \spellnotes With either version, you may reduce one horizontal dimension by half and double the other horizontal dimension.
\end{spellfooter}


\spellsection{Crushing Despair}{3}
\begin{spellheader}
    \spelldesc{You fill a number of creatures with sadness and gloom.}
\end{spellheader}
\begin{spellcontent}
    \begin{spelltargetinginfo}
        \spellburst{\areamed cone}
        \spelltgts{All creatures in the area}
    \end{spelltargetinginfo}
    \begin{spelleffects}
        The target is \vulnerable.
        \spelldur{\durmed}
    \end{spelleffects}
\end{spellcontent}
\begin{spellfooter}
    \spellinfo{Ench (Emotion) [Mind-Affecting, Morale]}{Arcane}
\end{spellfooter}


\spellsection{Daylight}{2}
\begin{spellheader}
    \spelldesc{You infuse an object with the power of the sun, causing it to illuminate a large area.}
\end{spellheader}
\begin{spellcontent}
    \begin{spelltargetinginfo}
        \spelltwocol{\spelltgt{Object touched}}{\spellrng{\rngtouch}}
    \end{spelltargetinginfo}
    \begin{spelleffects}
        \spelleffect The object touched sheds light as bright as full daylight in a \arealarge radius, and dim light for an additional 50 feet beyond that. Creatures that take penalties in bright light also take them while within the radius of this magical light. Despite its name, this spell is not the equivalent of sunlight for the purposes of creatures that are damaged or destroyed by bright light.
        \par If \spell{daylight} is cast on a small object that is then placed inside or under a light-proof covering, the spell's effects are blocked until the covering is removed.
        \spelldur{\durlong \dismissable}
    \end{spelleffects}
\end{spellcontent}
\begin{spellfooter}
    \spellinfo{Illus (Figment) [Light]}{Divine}
    \spellsr{No}
    \spellnotes \spell{Daylight} brought into an area of magical darkness (or vice versa) is temporarily negated, so that the otherwise prevailing light conditions exist in the overlapping areas of effect.
\end{spellfooter}

\spellsection{Banishment}{5}
\begin{spellheader}
    \spelldesc{You force extraplanar creatures back to their home plane.}
\end{spellheader}
\begin{spellcontent}
    \begin{spelltargetinginfo}
        \spelltwocol{\spelltgt{One extraplanar creature}}{\spellrng{\rngmed}}
    \end{spelltargetinginfo}
    \begin{spelleffects}
        \spellspecial This spell functions as \dismissal, except that it deals \spelldmg{5}{
        \begin{spellattack}{Magic vs. Will}
            \spellsuccess 5d6 damage \add d6 per four caster levels above 10th.
            \spellcritical As above, and the target is teleported to a random location on its home plane. If it is already on its home plane, this has no effect. There is a 10\% chance of sending the target to a random plane other than its home plane.
            \spellfailure Half damage, and no additional effects.
        \end{spellattack}
        \spellspecial You can concentrate on this spell as a standard action. If you do, you may affect another creature each round. You cannot target the same creature twice with a single casting of this spell.
    \end{spelleffects}
\end{spellcontent}
\begin{spellfooter}
    \spellinfo{Abjuration/Conjuration (Interdiction, Translocation) [Planar, Teleportation]}*{Arcane, Divine}
    \spellsr{No}
\end{spellfooter}

\spellsection{Disrupting Weapon}{4}
\begin{spellheader}
    \spelldesc{You imbue a weapon with positive energy, making it deadly to undead.}
    \spellrng{\rngclose}
    \spellfocus{One melee weapon}
\end{spellheader}
\begin{spellcontent}
    \spelltgr{The focus weapon strikes a \bloodied undead creature for the first time in a round}
    \begin{spelltargetinginfo}
        \spelltgt{The struck creature}
    \end{spelltargetinginfo}
    \begin{spelleffects}
        \begin{spellattack}{Magic vs. Fortitude}
            \spellsuccess The target creature is utterly destroyed.
        \end{spellattack}
    \end{spelleffects}
\end{spellcontent}

\spellsection{Elemental Swarm}{9}
\begin{spellheader}
    \spelltwocol{\spelllimit{\arealarge radius}}{\spellrng{\rngmed}}
\end{spellheader}
\begin{spellcontent}
    \spelleffect This spell opens a portal to an Elemental Plane and summons elementals from it. A druid can choose the plane (Air, Earth, Fire, or Water); a cleric opens a portal to the plane matching his domain.
    \par When the spell is complete, 2d4 Large elementals appear. Five minutes later, 1d4 Huge elementals appear. Five minutes after that, one greater elemental appears. All creatures initially appear wherever you desire within the spell's area. Once these creatures appear, they serve you for the duration of the spell.
    \par The elementals obey you explicitly and never attack you, even if someone else manages to gain control over them. You do not need to concentrate to maintain control over the elementals. You can dismiss them singly or in groups at any time.
    \spelldur{\durlong \dismissable}
\end{spellcontent}
\begin{spellfooter}
    \spellinfo{Conjuration (Summoning) [see text]}{Air, Earth, Fire, Nature, Water}
    \spellsr{No}
    \spellnotes When you use a summoning spell to summon an air, earth, fire, or water creature, it is a spell of that type.
\end{spellfooter}


\spellsection{Wall of Stone}{5}
\begin{spellheader}
    \begin{spelltargetinginfo}
        \spelltwocol{\spellzone{\arealarge solid wall, 5 ft. high \shapeable}}{\spellrng{\rngmed}}
    \end{spelltargetinginfo}
\end{spellheader}
\begin{spellcontent}
    \begin{spelleffects}
        \spelleffect This spell creates a wall made of stone atop an existing rock surface. A 5-foot square of wall has 60 hit points, and hardness 8. The wall is four inches thick.
        \par Unlike most walls, a \spell{wall of stone} need not be vertical. It need not rest entirely on solid ground, as long as it is solidly supported by existing stone.
        \par A wall of stone can be crudely shaped to add crenellations, battlements, and so forth.
    \end{spelleffects}
\end{spellcontent}
\begin{spellfooter}
    \spellinfo{Conjuration/Transmutation (Alteration, Creation) [Earth, Wall]}{Arcane, Earth, Nature}
    \spellsr{No}
    \spellnotes Once created, the stone is nonmagical, and can be destroyed like any other stone.
\end{spellfooter}

\spellsection{Giant Vermin}{4}
\begin{spellheader}
\end{spellheader}
\begin{spellcontent}
    \begin{spelltargetinginfo}
        \spelltwocol{\spelltgts{Up to three vermin}}{\spellrng{\rngclose}}
    \end{spelltargetinginfo}
    \begin{spelleffects}
        \spelleffect You turn three normal-sized centipedes, two normal-sized spiders, or a single normal-sized scorpion into Large-sized forms. Only one type of vermin can be transmuted (so a single casting cannot affect both a centipede and a spider), and all must be grown to the same size.
        \par Any giant vermin created by this spell do not attempt to harm you, but your control of such creatures is limited to simple commands (``Attack," ``Defend," ``Stop," and so forth). Orders to attack a certain creature when it appears or guard against a particular occurrence are too complex for the vermin to understand. Unless commanded to do otherwise, the giant vermin attack whoever or whatever is near them.
        \spelldur \durmed
    \end{spelleffects}
\end{spellcontent}
\begin{spellfooter}
    \spellinfo{Transmutation (Polymorph)}{Nature, Wild}
\end{spellfooter}

\spellsection{Invest Magic}{4}
\begin{spellheader}
\end{spellheader}
\begin{spellcontent}
    \begin{spelltargetinginfo}
        \spelltwocol{\spelltgt{One creature}}{\spellrng{\rngclose}}
    \end{spelltargetinginfo}
    \begin{spelleffects}
        \spelleffect All weapons and armor that the target wields gain a \plus3 enhancement bonus for as long as it wields them. This bonus increases to \plus4 at 14th caster level, and to \plus5 at 20th caster level.
        \spelldur \durshort
    \end{spelleffects}
\end{spellcontent}
\begin{spellfooter}
    \spellinfo{Transmutation (Augment)}{Arcane, Divine, War}
\end{spellfooter}


\spellsection{Lifebreaker Curse}{7}
\begin{spellheader}
    \spelldesc{You permanently cripple your foe's life force.}
\end{spellheader}
\begin{spellcontent}
    \begin{spelltargetinginfo}
        \spelltwocol{\spelltgt{One living creature}}{\spellrng{\rngmed}}
    \end{spelltargetinginfo}
    \begin{spelleffects}
        \begin{spellattack}{Magic vs. Will}
            \spellsuccess 7d10 life damage \add d10 per four caster levels above 14th. The target's maximum hit points are reduced by the amount of life damage it takes from this attack (to a minimum of 1).
            \spellfailure As above, but half damage, and the target's maximum hit points are not reduced.
        \end{spellattack}
        \spelldur Permanent
    \end{spelleffects}
\end{spellcontent}
\begin{spellfooter}
    \spellinfo{Necromancy (Life) [Curse]}{Necromancy}
    \spellnotes If this spell is cast multiple times on the same target, the reduction of maximum hit points does not stack. Only the largest reduction applies.

    \cursespellnotes
\end{spellfooter}

\spellsection{Limited Wish}{7}
\begin{spellheader}
\end{spellheader}
\begin{spellcontent}
    \spellinfo{Universal}{Arcane}
    \spellcmp{Verbal, Somatic, and Material}
    \begin{spelltargetinginfo}
        \spelltwocol{\spelltgteffarea{See text}}{\spellrng{See text}}
    \end{spelltargetinginfo}
    \begin{spelleffects}
        \spelleffect A limited wish lets you create nearly any type of effect. For example, a limited wish can do any of the following things.
        \begin{itemize}
            \item Duplicate any general sorcerer/wizard spell of 6th level or lower, provided the spell is not of a school prohibited to you.
            \item Duplicate any general sorcerer/wizard spell of 5th level or lower, even if it's of a prohibited school.
            \item Duplicate any other spell of 4th level or lower, provided the spell is not of a school prohibited to you.
            \item Duplicate any other spell of 3rd level or lower, even if it's of a prohibited school.
            \item Undo the harmful effects of many spells, such as geas/quest or insanity.
            \item Produce any other effect whose power level is in line with the above effects, such as a single creature automatically hitting on its next attack or taking a \minus5 penalty to its defenses for 5 rounds.
        \end{itemize}
        \par When casting a limited wish, you do not specify the exact spell or effect you wish to duplicate. Instead, you make a wish, describing what you want to have happen, and make a DC 15 Wisdom check. If the check fails, your intent is redirected or perverted in some way. For example, a \spell{limited wish} to turn a foe to stone would normally mimic the \spell{flesh to stone} effect of the \spell{transmute flesh and stone} spell. However, if the Wisdom check failed, your foe might gain the benefit of a \spell{stoneskin} spell instead.
        \par When a limited wish spell duplicates a spell with a material component that costs more than 1,000 gp, you must provide that component (in addition to the 1,000 gp cost for this spell).
        \spellmat{A diamond worth no less than 1,000 gp (see above).}
        \spelldur See text
    \end{spelleffects}
\end{spellcontent}
\begin{spellfooter}
\end{spellfooter}

\spellsection{Link Vitality}{3}
\begin{spellheader}
\end{spellheader}
\begin{spellcontent}
    \begin{spelltargetinginfo}
        \spelltwocol{\spelltgts{Two living creatures in the area}}{\spellrng{\rngclose}}
    \end{spelltargetinginfo}
    \begin{spelleffects}
        \spellatk{Magic vs. Will}
        \spellspecial This spell has no effect unless the attack succeeds against both targets.
        \spellsuccess Whenever one target gains or loses hit points, the other target also gains or loses the same amount of hit points.
        \spelldur \durshort
    \end{spelleffects}
\end{spellcontent}
\begin{spellfooter}
    \spellinfo{Necromancy (Life)}{Necromancy}
    \spellnotes The loss of hit points caused by this spell is not damage, and is not affected by damage reduction or other abilities which affect damage.
\end{spellfooter}

\spellsectioncomma{Link Vitality}{Mass}{6}
\begin{spellheader}
\end{spellheader}
\begin{spellcontent}
    \begin{spelltargetinginfo}
        \spelltwocol{\spelltgts{Up to five living creatures}}{\spellrng{\rngclose}}
    \end{spelltargetinginfo}
    \begin{spelleffects}
        \spellspecial This spell has no effect unless the attack succeeds against at least two targets.
        \spellsuccess Whenever one target gains or loses hit points, all other affected targets also gain or lose the same amount of hit points.
        \spelldur \durshort
    \end{spelleffects}
\end{spellcontent}
\begin{spellfooter}
    \spellinfo{Necromancy (Life)}{Arcane}
\end{spellfooter}

\spellsection{Living Projectile}{3}
\begin{spellheader}
    \spelldesc{You telekinetically fling an ally at great speed towards your foe.}
    \begin{spelltargetinginfo}
        \spelltwocol{\spelltgt{One willing creature}[Primary]}{\spellrng{\rngclose}}
    \end{spelltargetinginfo}
    \begin{spelleffects}
        \begin{spelltargetinginfo}
            \spelltgt{One creature or object}[Secondary]
        \end{spelltargetinginfo}
    \end{spelleffects}
\end{spellheader}
\begin{spellcontent}
    \spelleffect The primary target is moved adjacent to the secondary target, gains the benefit of the \spell{ablate impact} spell for 1 round, and is knocked prone.
    \spellatk{Caster level \add casting attribute vs. Armor defense (secondary target)}
    \begin{spellmargin}
        \spellsuccess The secondary target takes 3d10 damage \add d10 per four caster levels above 6th. The primary target takes half this amount of damage.
        \spellfailure As above, but half damage to the secondary target. The damage taken by the primary target is not halved.
    \end{spellmargin}
    \spelldur 1 round
\end{spellcontent}
\begin{spellfooter}
    \spellinfo{Abjuration/Evocation (Control, Shielding)}{Arcane}
    \spellsr{Yes (primary target), No (secondary target)}
\end{spellfooter}



\spellsection{Locate Entity}{6}
\begin{spellheader}
\end{spellheader}
\begin{spellcontent}
    \begin{spelltargetinginfo}
        \spellrng{\rngext \rngunrestricted}
    \end{spelltargetinginfo}
    \begin{spelleffects}
        \spelleffect This spell functions as \spell{locate object}, except that it can also detect creatures, as \spell{locate creature}. When you cast this spell, you choose to locate an object or creature, following the restrictions stated in the respective location spells.
        \spelldur \durmed
    \end{spelleffects}
\end{spellcontent}
\begin{spellfooter}
    \spellinfo{Divination (Awareness) [Detection]}{Arcane, Knowledge}
    \spellsr{No}
    \spellnotes A detection spell can penetrate barriers, but 1 foot of stone, 1 inch of common metal, a thin sheet of lead, or 3 feet of wood or dirt blocks it.
\end[spellfooter}

\spellsection{Locate Creature}{2}
\begin{spellheader}
    \spellrng{\rnglong}
\end{spellheader}
\begin{spellcontent}
    \spelleffect You sense the direction of a well-known or clearly visualized creature if it is within the spell's range. You can search for general creatures based on visual characteristics (such as ``pointy ears'' or ``looking human''), in which case you locate the nearest one of its kind if more than one is within the range. Attempting to find a certain creature requires a specific and accurate mental image of a distinguishing visual characteristic, such as its clothes or face; if the image is not close enough to the actual creature, the spell fails.
    \spelldur \durmed
\end{spellcontent}
\begin{spellfooter}
    \spellinfo{Divination (Awareness) [Detection]}{Arcane, Divine}
    \spellnotes A detection spell can penetrate barriers, but 1 foot of stone, 1 inch of common metal, a thin sheet of lead, or 3 feet of wood or dirt blocks it.
\end{spellfooter}

\spellsectioncomma{Locate Creature}{Greater}{4}
\begin{spellheader}
\end{spellheader}
\begin{spellcontent}
    \spellinfo{Divination (Awareness) [Detection]}{Arcane, Divine, Knowledge}
    \spellrng{\rngext}
    \spellsr{No}
    \spelleffect As \spell{locate creature}, except that it detects creatures within \rngext range. In addition, you detect all appropriate creatures within the range, rather than only the nearest creature.
    \spelldur \durmed \dismissable
\end{spellcontent}
\begin{spellfooter}
\end{spellfooter}

\spellsection{Locate Object}{1}
\begin{spellheader}
    \spellrng{\rnglong}
\end{spellheader}
\begin{spellcontent}
    \spelleffect You sense the direction of a well-known or clearly visualized object if it is within the spell's range. You can search for general items, in which case you locate the nearest one of its kind if more than one is within the range. Attempting to find a certain item requires a specific and accurate mental image; if the image is not close enough to the actual object, the spell fails.
    \spellnotes A detection spell can penetrate barriers, but 1 foot of stone, 1 inch of common metal, a thin sheet of lead, or 3 feet of wood or dirt blocks it.
    \spelldur \durmed \dismissable
\end{spellcontent}
\begin{spellfooter}
    \spellinfo{Divination (Awareness) [Detection]}{Arcane, Divine}
\end{spellfooter}

\spellsectioncomma{Locate Object}{Greater}{3}
\begin{spellheader}
    \spellrng{\rngext}
\end{spellheader}
\begin{spellcontent}
    \spellsr{No}
    \spelleffect This spell functions like \spell{locate object}, except that it detects objects within \rngext range. In addition, you detect all appropriate objects within the range, rather than only the nearest object. 
    \spelldur \durmed \dismissable
\end{spellcontent}
\begin{spellfooter}
    \spellinfo{Divination (Awareness) [Detection]}{Arcane, Divine, Knowledge}
\end{spellfooter}


\spellsection{Magic Circle against Alignment}{5}
\begin{spellheader}
    \spellrng{Touch}
    \spellfocus{One creature}
    \spellemanation{\areasmall radius from the focus}
    \spellspecial When you cast this spell, choose an alignment (chaotic, good, lawful, or evil).
\end{spellheader}
\begin{spellcontent}
    \begin{spelltargetinginfo}
        \spelltgts{All creatures in the area}
    \end{spelltargetinginfo}
    \begin{spelleffects}
        \spelleffect Summoned creatures which have the chosen alignment cannot enter the area.
        \spelleffect The target is protected from the chosen alignment, as \spell{protection from alignment}.
        \spelldur \durshort \dismissable
    \end{spelleffects}
\end{spellcontent}
\begin{spellfooter}
    \spellinfo{Abjuration (Interdiction) [Barrier, Good]}{Arcane, Chaos, Divine, Evil, Good, Law}
\end{spellfooter}

\spellsection{Magic Fang}{2}
\begin{spellheader}
\end{spellheader}
\begin{spellcontent}
    \begin{spelltargetinginfo}
        \spelltwocol{\spelltgt{One creature}}{\spellrng{\rngclose}}
    \end{spelltargetinginfo}
    \begin{spelleffects}
        \spelleffect One of the target's natural weapons gains a \plus2 enhancement bonus to attack and damage. \spellbonusscalingdescription
        \spelldur \durshort
    \end{spelleffects}
\end{spellcontent}
\begin{spellfooter}
    \spellinfo{Transmutation (Augment)}{Nature}
    \spellnotes This spell can be cast multiple times on the same creature. Each time, you can choose a new natural weapon. It does not change an unarmed strike's damage from nonlethal damage to lethal damage.

    This spell can be made permanent with a \spell{permanency} ritual.
\end{spellfooter}

\spellsectioncomma{Magic Fang}{Greater}{4}
\begin{spellheader}
\end{spellheader}
\begin{spellcontent}
    \begin{spelltargetinginfo}
        \spelltwocol{\spelltgt{One creature}}{\spellrng{\rngclose}}
    \end{spelltargetinginfo}
    \begin{spelleffects}
        \spelleffect All of the target's natural weapons gain a \plus3 enhancement bonus to attack and damage. This bonus increases to \plus4 at 14th caster level, and to \plus5 at 20th caster level.
        \spelldur \durshort
    \end{spelleffects}
\end{spellcontent}
\begin{spellfooter}
    \spellinfo{Transmutation (Augment)}{Nature}
    \spellnotes This spell can be made permanent with a \spell{permanency} ritual.
\end{spellfooter}

\spellsection{Magic Vestment}{1}
\begin{spellheader}
\end{spellheader}
\begin{spellcontent}
    \begin{spelltargetinginfo}
        \spelltwocol{\spelltgt{One suit of armor or shield}}{\spellrng{\rngclose}}
    \end{spelltargetinginfo}
    \begin{spelleffects}
        \spelleffect The target gains a \plus2 enhancement bonus, increasing the defense bonus it provides. \spellbonusscalingdescription
        \spelldur \durmed
    \end{spelleffects}
\end{spellcontent}
\begin{spellfooter}
    \spellinfo{Transmutation (Augment)}{Arcane, Divine}
    \spellnotes An outfit of regular clothing counts as armor that grants no armor defense bonus for the purpose of this spell.
\end{spellfooter}

\spellsection{Magic Weapon}{2}
\begin{spellheader}
\end{spellheader}
\begin{spellcontent}
    \begin{spelltargetinginfo}
        \spelltwocol{\spelltgt{One weapon or fifty projectiles (in a single group)}}{\spellrng{\rngclose}}
    \end{spelltargetinginfo}
    \begin{spelleffects}
        \spelleffect The target gains a \plus2 enhancement bonus to attack and damage. \spellbonusscalingdescription
        \spelldur \durshort
    \end{spelleffects}
\end{spellcontent}
\begin{spellfooter}
    \spellinfo{Transmutation (Augment)}{Arcane, Divine}
    \spellnotes You can't cast this spell on a natural weapon, such as an unarmed strike (instead, see \spell{magic fang}). A monk's unarmed strike is considered a weapon, and thus it can be enhanced by this spell.
    \par If you use this spell to enhance projectiles, the projectiles must be of the same kind, and they have to be together (in the same quiver or other container). Projectiles, but not thrown weapons, lose their transmutation when used. (Treat darts and shuriken as projectiles, rather than as thrown weapons, for the purpose of this spell.)
\end{spellfooter}

\begin{spellsection}{Phantasmal Maze}{5}
    \begin{spellheader}
        \spelldesc{You manipulate a foe's perceptions, causing it to believe that it is trapped in a labyrinth.}
    \end{spellheader}
    \begin{spellcontent}
        \begin{spelltargetinginfo}
            \spelltwocol{\spelltgt{One creature}}{\spellrng{\rngmed}}
        \end{spelltargetinginfo}
        \begin{spelleffects}
            \begin{spellattack}{Magic vs. Will}
                \spellsuccess The target perceives itself to be banished to an extradimensional labyrinth of force planes, as the \spell{maze} spell. It cannot see or hear anything to the contrary, causing it to be treated as if blinded and deafened for most purposes. However, it can still see and hear itself. Typically, this means the target moves in a random direction each round to escape the maze.

                If the target encounters any physical resistance in its movements or takes any damage, you must make another attack to maintain the effect. Failure means the target disbelieves the phantasm, ending the spell.
            \end{spellattack}
            \spelldur \durmed
        \end{spelleffects}
    \end{spellcontent}
    \begin{spellfooter}
        \spellinfo{Illus (Phantasm)}{Arcane, Trickery}
    \end{spellfooter}
\end{spellsection}

\begin{spellsection}{Phantasmal Wound}{2}
    \begin{spellheader}
        \spelldesc{You manipulate a foe's perceptions, causing it to believe that it is grievously wounded.}
    \end{spellheader}
    \begin{spellcontent}
        \begin{spelltargetinginfo}
            \spelltwocol{\spelltgt{One creature}}{\spellrng{\rngmed}}
        \end{spelltargetinginfo}
        \begin{spelleffects}
            \begin{spellattack}{Magic vs. Will}
                \spellsuccess The target is \sickened.

                As long as the target is \bloodied, it also perceives itself to have no hit points remaining. It is \staggered, and may try to heal itself or take other appropriate actions. If its hit points are altered, such as by damage or healing, the creature disbelieves the effect automatically. It is immune to this effect for the rest of the duration, though it remains sickened.
            \end{spellattack}
            \spelldur \durshort
        \end{spelleffects}
    \end{spellcontent}
    \begin{spellfooter}
        \spellinfo{Illus (Phantasm)}{Arcane}
    \end{spellfooter}
\end{spellsection}

\begin{spellsection}{Prohibition}[Greater]{9}
    \begin{spellheader}
    \end{spellheader}
    \begin{spellcontent}
        \spellemanation{\arealarge radius centered on you}
        \spelleffect You loudly declare a prohibited action, as \spell{prohibition}.
        \spelldur \durshort
    \end{spellcontent}
    \begin{spellsubcontent}
        \begin{spelltargetinginfo}
            \spelltwocol{\spelltgr{A creature breaks the rule}}{\spelltgt{Triggering creature}}
        \end{spelltargetinginfo}
        \begin{spelleffects}
            \begin{spellattack}{Magic vs. Will}
                \spellsuccess \spelldamage{9}{}[d6]. You know a creature broke the rule, but not which creature.
                \spellcritical As above, and the creature is \stunned for 1 round, preventing it from completing its action.
                \spellfailure As above, but half damage.
            \end{spellattack}
        \end{spelleffects}
    \end{spellsubcontent}
    \begin{spellfooter}
        \spellinfo{Abjuration/Divination (Communication, Interdiction)}{Abjuration, Law}
        \spellnotes Mindless creatures are given no special insight into the rule. Any individual creature can only take damage for breaking the rule once per round.
    \end{spellfooter}
\end{spellsection}

\begin{spellsection}{Resilient Sphere}{5}
    \begin{spellheader}
        \spelldesc{You trap a foe in a globe of shimmering force, removing it from the battle.}
        \spellrng{\rngmed}
    \end{spellheader}
    \begin{spellcontent}
        \begin{spelltargetinginfo}
            \spelltwocol{\spelltgt{One creature or object (Large or smaller)}}{\spellrng{\rngmed}}
        \end{spelltargetinginfo}
        \begin{spelleffects}
            \begin{spellattack}{Magic vs. Reflex}
                \spellsuccess An immobile, invisible spherical prison appears around the target. You choose whether the walls made of bars of force or perfect sheets of force. The bars are an inch wide, with one inch gaps between them.
            \end{spellattack}
            \spelldur \durshort \dismissable
        \end{spelleffects}
    \end{spellcontent}
    \begin{spellfooter}
        \spellinfo{Evocation (Control) [Force]}{Evocation}
        \spellnotes Force effects can be destroyed by \spell{disintegrate}.
    \end{spellfooter}
\end{spellsection}

\begin{spellsection}{Retributive Brilliance}{5}
    \begin{spellheader}
    \end{spellheader}
    \begin{spellcontent}
        \begin{spelltargetinginfo}
            \spellquicktargeting{One creature}{\rngclose}
        \end{spelltargetinginfo}
        \begin{spelleffects}
            \spelleffect The target is surrounded by a faintly shimmering field. If the target is attacked, the field retaliates.
            \spelldur \durshort or until discharged
        \end{spelleffects}
    \end{spellcontent}
    \begin{spellsubcontent}
        \begin{spelltargetinginfo}
            \spelltgr{A creature attacks the target with a melee weapon}
            \spelltwocol{\spelltgt{The attacking creature}}{\spellrng{\rngclose}}
        \end{spelltargetinginfo}
        \begin{spelleffects}
            \begin{spellattack}{Magic vs. Reflex}
                \spellsuccess The target is \dazzled for 5 rounds.

                If the target is \bloodied, it is also \blinded for 5 rounds.
                \spellfailure The target is dazzled for 5 rounds.
            \end{spellattack}
        \end{spelleffects}
    \end{spellsubcontent}
    \begin{spellfooter}
        \spellinfo{Abjuration/Illus (Figment, Shielding)}{Arcane}
    \end{spellfooter}
\end{spellsection}

\begin{spellsection}{Scintillating Pattern}{8}
    \begin{spellheader}
    \end{spellheader}
    \begin{spellcontent}
        \begin{spelltargetinginfo}
            \spellzone{\arealarge radius centered on you}
            \spelltgts{All enemies in the area}
        \end{spelltargetinginfo}
        \begin{spelleffects}
            \spelleffect The area is brightly illuminated, and dim illumination extends for an additional 50 feet.
            \spelleffect The target is \bewildered.
            \spelldur \durshort
        \end{spelleffects}
    \end{spellcontent}
    \begin{spellfooter}
        \spellinfo{Ench/Illus (Compulsion, Figment) [Light, Mind-Affecting, Sight-Dependent]}{Arcane}
        \spelldesc{You create a massive spread of colorful lights that spin and whirl in a complex pattern that bewilders your foes.}
        \spellnotes Creatures unable to see the lights are not bewildered.
    \end{spellfooter}
\end{spellsection}

\begin{spellsection}{Stampede}{9}
    \begin{spellheader}
        \spelldesc{You summon a stampede of bison that trample your foes before disappearing as quickly as they arrived.}
        \begin{spelltargetinginfo}
            \spellburst{100 ft. line, 20 ft. wide}
            \spelltgts{Everything in the area}
            \spelltime{Full-round action}
        \end{spelltargetinginfo}
    \end{spellheader}
    \begin{spellcontent}
        \begin{spelleffects}
            \begin{spellattack}{Magic vs. Reflex}
                \spellsuccess \spelldamage{9}{bludgeoning}[d6]
                \spellfailure Half damage, and the target is not knocked prone.
            \end{spellattack}
            \spelldur \durshort \dismissable
        \end{spelleffects}
    \end{spellcontent}
    \begin{spellfooter}
        \spellinfo{Conjuration (Summoning)}{Nature, Wild}
        \spellsr{No}
    \end{spellfooter}
\end{spellsection}

\begin{spellsection}{Shadow Body}{7}
    \begin{spellheader}
        \begin{spelltargetinginfo}
            \spelltgt{You}
        \end{spelltargetinginfo}
    \end{spellheader}
    \begin{spellcontent}
        \begin{spelleffects}
            \spelleffect Your body and all your equipment are subsumed by your shadow. As a living shadow, you blend perfectly into any other shadow and vanish in darkness. You appear as an unattached shadow in areas of full light.
            \par You can move at your normal speed, on any surface, including walls and ceilings, as well as across the surfaces of liquids -- even up the face of a waterfall.
            \par You become perfectly flat, potentially allowing you to move into locations you would not normally be able to move into.
            \par While in your shadow body, you gain physical damage reduction, reducing the physical damage you take each round by 14 \add 1 per caster level above 14th. If you take solar damage, you cannot use your damage reduction for 1 round. You are immune to ability damage, disease, drowning, and poison. You take only half damage from energy attacks (acid, cold, electricity, and fire).
            \par While affected by this spell, you can be detected by spells that read thoughts, life, or presences (including true seeing), or if you make suspicious movements in lighted areas.
            \par You cannot harm anyone physically or manipulate any objects, but you can use your spells normally. Doing so may attract notice, but if you remain in a shadowed area, you get a \plus15 enhancement bonus on your Hide check to remain unnoticed.
            \spelldur \durmed \dismissable
        \end{spelleffects}
    \end{spellcontent}
    \begin{spellfooter}
        \spellinfo{Illus/Transmutation (Polymorph, Shadow)}{Arcane}
    \end{spellfooter}
\end{spellsection}

\begin{spellsection}{Share Pain}[Forced]{3}
    \begin{spellheader}
        \begin{spelltargetinginfo}
            \spelltwocol{\spelltgts{Two creatures}}{\spellrng{\rngmed}}
        \end{spelltargetinginfo}
        \begin{spelleffects}
            \spellspecial When you cast this spell, you choose which target will be protected.
        \end{spelleffects}
    \end{spellheader}
    \begin{spellcontent}
        \spellatk{Magic vs. Will}
        \begin{spellmargin}
            \spellsuccess The targets share damage, as \spell{share pain}.
        \end{spellmargin}
        \spelldur \durlong \dismissable
    \end{spellcontent}
    \begin{spellfooter}
        \spellinfo{Abjuration/Necromancy (Life, Shielding)}{Arcane, Divine, Evil}
    \end{spellfooter}
\end{spellsection}

\begin{spellsection}{Shillelagh}{1}
    \begin{spellheader}
        \begin{spelltargetinginfo}
            \spelltwocol{\spelltgt{One nonmagical oak club or quarterstaff}}{\spellrng{Touch}}
        \end{spelltargetinginfo}
    \end{spellheader}
    \begin{spellcontent}
        \begin{spelleffects}
            \spelleffect As long as you wield it, the target is enhanced, as \spell{magic weapon}. In addition, it deals damage as if it were one size category larger. If you stop wielding the weapon, this spell has no effect on it.
            \spelldur \durshort
        \end{spelleffects}
    \end{spellcontent}
    \begin{spellfooter}
        \spellinfo{Transmutation}{Nature}
        \spellnotes A typical club or quarterstaff wielded by a Medium creature would deal d8 damage under this spell.
    \end{spellfooter}
\end{spellsection}

\begin{spellsection}{Wish}{9}
    \begin{spellheader}
    \end{spellheader}
    \begin{spellcontent}
        \begin{spelltargetinginfo}
            \spelltwocol{\spelltgteffarea{See text}}{\spellrng{See text}}
            \spellcmp{Verbal, Somatic, and Material}
        \end{spelltargetinginfo}
        \spellatk{See text}
        \spelleffect This spell is the mightiest spell a wizard or sorcerer can cast. By simply speaking your desires aloud, you can alter reality to better suit you.
        \par Even wish, however, has its limits.
        \par A wish can produce any one of the following effects.
        \begin{itemize}
            \item Duplicate any general wizard or sorcerer spell of 8th level or lower, provided the spell is not of a school prohibited to you.
            \item Duplicate any general wizard or sorcerer spell of 7th level or lower even if it's of a prohibited school.
            \item Duplicate any other spell of 6th level or lower, provided the spell is not of a school prohibited to you.
            \item Duplicate any other spell of 5th level or lower even if it's of a prohibited school. 
            \item Undo the harmful effects of many other spells, such as geas/quest or insanity.
            \item Create a nonmagical item of up to 10,000 gp in value.
            \item Create a magic item, or add to the powers of an existing magic item.
            \item Remove injuries and afflictions. A single wish can aid one creature per caster level, and all subjects are cured of the same kind of affliction. For example, you could heal all the damage you and your companions have taken, or remove all poison effects from everyone in the party, but not do both with the same wish. A wish can never restore the experience point loss from casting a spell or the level or Constitution loss from being resurrected.
            \item Revive the dead. A wish can bring a dead creature back to life by duplicating a resurrection spell. A wish can revive a dead creature whose body has been destroyed, but the task takes two wishes, one to recreate the body and another to infuse the body with life again.
            \item Transport travelers. A wish can lift one creature per caster level from anywhere on any plane and place those creatures anywhere else on any plane regardless of local conditions. You must make a Will attack to affect unwilling targets.
            \item Undo misfortune. A wish can undo a single recent event. The wish forces a reroll of any roll made within the last round (including your last turn). Reality reshapes itself to accommodate the new result. For example, a wish could undo a foe's successful critical hit (either the attack roll or the critical roll), a friend's failed attack, and so on. The reroll, however, may be as bad as or worse than the original roll. You must make a Will attack to affect an unwilling target.
        \end{itemize}
        \par When casting a wish, you do not specify the exact spell or effect you wish to duplicate. Instead, you make a wish, describing what you want to have happen, and make a DC 20 Wisdom check. If the check fails, your intent is redirected or perverted in some way. For example, a wish to turn a foe to stone would normally mimic the flesh to stone effect of the transmute flesh and stone spell. However, if the Wisdom check failed, your foe might gain the benefit of a \spell{stoneskin} spell instead.
        \par You may try to use a wish to produce greater effects than these, but doing so is dangerous. The DC of the Wisdom check increases to 25, and the negative consequences for failing the check increase in proportion to the potency of the effect you try to create.
        \spellmat{10,000gp of diamonds. In addition, when a \spell{wish} duplicates a spell with a material component that costs more than 10,000 gp, you must provide that component.}
    \end{spellcontent}
\end{spellsection}

\begin{spellsection}{Telekinesis}{6}
    \begin{spellheader}
        \spelldesc{You move objects or creatures by concentrating on them.}
    \end{spellheader}
    \begin{spellcontent}
        \begin{spelltargetinginfo}
            \spellrng{\rngmed}
        \end{spelltargetinginfo}
        \begin{spelleffects}
            \spellspecial Each round, you choose one of three effects.
            \spelleffect[Sustained Force] As \spell{telekinetic force}.
            \spelleffect[Combat Maneuver] As \spell{telekinetic maneuver}.
            \spelleffect[Violent Thrust] As \spell{telekinetic thrust}. Using this effect ends the spell.
            \spelldur Concentration or until discharged
        \end{spelleffects}
    \end{spellcontent}
    \begin{spellfooter}
        \spellinfo{Evocation (Control)}{Evocation}
    \end{spellfooter}
\end{spellsection}

\begin{spellsection}{Telekinetic Force}{4}
    \begin{spellheader}
    \end{spellheader}
    \begin{spellcontent}
        \spelltwocol{\spelltgr{End of every round}}{\spelldur Concentration}
        \begin{spelltargetinginfo}
            \spelltwocol{\spelltgt{One creature or object}}{\spellrng{\rngmed}}
        \end{spelltargetinginfo}
        \begin{spelleffects}
            \begin{spellattack}{Magic vs. Will}
                \spellsuccess You can take a standard action using the target as if you were holding it in your hands. Your effective Strength is equal to your Charisma, and your effective Dexterity is equal to your Intelligence. You can move the target up to thirty feet per round.
            \end{spellattack}
        \end{spelleffects}
    \end{spellcontent}
    \begin{spellfooter}
        \spellinfo{Evocation (Control)}{Evocation}
        \spellnotes If a target resists your attempt to control it, it (and its equipment) is immune to any further attempts you make for the duration of the spell.
    \end{spellfooter}
\end{spellsection}

\begin{spellsection}{Telekinetic Maneuver}{3}
    \begin{spellheader}
    \end{spellheader}
    \begin{spellcontent}
        \begin{spelltargetinginfo}
            \spelltwocol{\spelltgt{One creature or object}}{\spellrng{\rngmed}}
        \end{spelltargetinginfo}
        \begin{spelleffects}
            \spelltwocol{\spelltgr{End of every round}}{\spelldur Concentration}
            \spellspecial Choose one of the following combat maneuvers: disarm, dirty trick, grapple, shove, or trip.
            \begin{spellattack}{Caster level \add casting attribute vs. Maneuver defense}
                \spellsuccess The target is affected by the chosen maneuver.
            \end{spellattack}
        \end{spelleffects}
    \end{spellcontent}
    \begin{spellfooter}
        \spellinfo{Evocation (Control)}{Evocation}
    \end{spellfooter}
\end{spellsection}

\begin{spellsection}{Telekinetic Thrust}{5}
    \begin{spellheader}
        \begin{spelltargetinginfo}
            \spelltwocol{\spelltgt{One creature or object}}{\spellrng{\rngmed}}
        \end{spelltargetinginfo}
    \end{spellheader}
    \begin{spellcontent}
        \begin{spelleffects}
            \begin{spellattack}{Magic vs. Will}
                \spellsuccess You throw the target in any direction a number of feet equal to 50 \add 5 per two caster levels above 10th. You can only throw it half as far vertically.

                If the target strikes a solid obstacle, such as another creature, both the target and the struck obstacle take 1d6 physical bludgeoning damage per 10 feet of movement the target had left to travel.
            \end{spellattack}
        \end{spelleffects}
    \end{spellcontent}
    \begin{spellfooter}
        \spellinfo{Evocation (Control)}{Evocation}
    \end{spellfooter}
\end{spellsection}


\begin{spellsection}{Touch of Idiocy}{2}
    \begin{spellheader}
        \spelldesc{With a touch, you reduce the target's mental faculties.}
    \end{spellheader}
    \begin{spellcontent}
        \begin{spelltargetinginfo}
            \spelltwocol{\spelltgt{One creature}}{\spellrng{5 feet}}
        \end{spelltargetinginfo}
        \begin{spelleffects}
            \begin{spellattack}{Magic vs. Reflex and Will}
                \spellsuccess[Reflex] The target takes a \minus4 penalty to its Intelligence, Wisdom, and Charisma. This penalty can't reduce any of these attributes below \minus9.
                \spellfailure[Will] As above, but the penalty is halved.
            \end{spellattack}
            \spelldur \durshort
        \end{spelleffects}
    \end{spellcontent}
    \begin{spellfooter}
        \spellinfo{Ench (Inhibition) [Mind-Affecting]}{Arcane}
        \spellnotes This spell's effect may make it impossible for the target to cast some or all of its spells, if its casting attribute drops below the minimum required to cast spells of that level.
    \end{spellfooter}
\end{spellsection}

\begin{spellsection}{Changestaff}{7}
    \begin{spellheader}
        \spelldesc{You plant your staff in the ground and transform it into a massive tree-like creature which obeys your every command.}
    \end{spellheader}
    \begin{spellcontent}
        \begin{spelltargetinginfo}
            \spelltwocol{\spelltgt{Your touched staff}}{\spellrng{\rngtouch}}
            \spellcmp{Verbal, Somatic, and Material}
            \spelltime{Full-round action}
        \end{spelltargetinginfo}
        \begin{spelleffects}
            \spelleffect Your staff turns into a creature that looks and fights just like a treant. The staff-treant defends you and obeys any spoken commands. However, it is not a true treant; it cannot converse with actual treants or control trees.

            If the staff-treant is reduced to 0 or fewer hit points, it crumbles to powder and the staff is destroyed. Otherwise, the staff returns to its normal form when the spell duration expires (or when the spell is dismissed), and it can be used as the material component for another casting of the spell. The staff-treant is always at full strength when created, despite any wounds it may have incurred the last time it appeared.
            \spelldur \durmed \dismissable
        \end{spelleffects}
    \end{spellcontent}
    \begin{spellfooter}
        \spellinfo{Transmutation [Alteration]}{Nature, Wild}
        \spellmat{The quarterstaff, which must be specially prepared. The staff must be a sound limb cut from an ash, oak, or yew, then cured, shaped, carved, and polished (a process requiring twenty-eight days). You cannot adventure or engage in other strenuous activity during the shaping and carving of the staff.}
    \end{spellfooter}
\end{spellsection}

\begin{spellsection}{Retrieve Ally}{2}
    \begin{spellheader}
        \spelldesc{You save your ally from danger by teleporting it next to you.}
    \end{spellheader}
    \begin{spellcontent}
        \begin{spelltargetinginfo}
            \spelltwocol{\spelltgt{One willing creture}}{\spellrng{\rngmed}}
        \end{spelltargetinginfo}
        \begin{spelleffects}
            \spelleffect The target teleports into a free space adjacent to you. You must have line of sight and line of effect to both the target and its destination. If you accidentally attempt to teleport the creature into an invalid location, the spell simply fails.
        \end{spelleffects}
    \end{spellcontent}
    \begin{spellfooter}
        \spellinfo{Conjuration [Teleportation]}{Conjuration}
    \end{spellfooter}
\end{spellsection}
