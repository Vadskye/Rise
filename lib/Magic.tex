\chapter{Magic}\label{Magic}
Magic comes in many forms, but it is most commonly wielded with spells. A spell is a one-time magical effect. There are three types of spells: arcane (cast by sorcerers and wizards), divine (cast by clerics), and nature (cast by druids). Cutting across these categories are the eight schools of magic. Each of the eight schools represents a different type of mastery over the world, based on fundamentally distinct principles.

\section{Casting Spells}\label{Casting Spells}
Whether a spell is arcane, divine, or natural, casting a spell works the same way.

\subsection{Casting Process}

\begin{itemize}
    \itemhead{Choose spell}: You must choose which spell to cast from among the spells you know. If a spell has multiple versions, you choose which version to use when you cast it.
    \itemhead{Pay spell slot}: If you use spell slots to cast spells, you must expend a spell slot of the spell's level or higher. If you do not have spells slots to spend, your attempt to cast the spell fails.
    \itemhead{Perform spell components}: Almost all spells have verbal and somatic components.
        \begin{itemize}
            \itemhead{Verbal components}\label{Verbal Components} involve speaking the spell's incantation loudly and clearly.
            \itemhead{Somatic components}\label{Somatic Components} involve using at least one hand to make gestures with magical significance. While casting a spell with somatic components, one hand is used to cast the spell, and cannot be used to defend yourself or take other actions.
        \end{itemize}
    \itemhead{Concentrate}: You must concentrate to cast spells. If you take damage or are otherwise distracted during a phase in which you attempt to cast a spell, you may miscast the spell. See \pcref{Concentration}, for details.
    \itemhead{Choose effects}: You make choices about the spell's effects as you finish casting the spell. This includes deciding which creatures to target, where the spell takes effect, and so on.
    \itemhead{Roll dice}: Finally, after making any necessary decisions about the spell's effects, you roll any dice required to determine how successful the spell is. This includes attack rolls, damage rolls, and so on.
\end{itemize}

\subsection{Spell Slots}
To cast a spell of a given level, most spellcasters must spend a spell slot of the appropriate level. If the spellcaster has no spell slots of the appropriate level, she may use a higher-level spell slot instead. The spell is still treated as if it were its actual level, not the level of the slot used to cast it.

Sorcerers do not use spell slots, and do not need to spend spell slots to cast spells.

\subsection{Focusing and Concentration}\label{Concentration}\label{Focus}\label{Focusing and Concentration}

Some actions require focusing, such as casting spells.
If you are damaged or distracted while taking an action that requires focus, your concentration may be broken.

\parhead{Concentration Checks}\label{Concentration Tests}\label{Concentration Checks} 

\parhead{Concentration Tests}\label{Concentration Tests} To test concentration, roll d20 \add (half the damage you have taken during the current round) \add (twice the level of the spell you are casting) \add (overwhelm penalties). If the result equals or exceeds your Mental defense, the test fails, and your concentration has been broken. Otherwise, the test succeeds.

\parhead{Casting a Spell} You must concentrate to cast spells. When you finish casting a spell, if you took any damage while casting it, make a Concentration test. Failure means you miscast the spell (see \pcref{Miscasting}), but you still lose the spell slot used to cast it.

\parhead{Focusing on Existing Spells} Many spells allow you to spend a standard action focusing to extend their effects. At the end of every round you focus, if you took any damage, make a Concentration test. Failure means your focus ends, but the spell may continue to have effects, as indicated in the spell description. Most spells do not allow you to resume focusing on them after your concentration is broken.

\parhead{Performing Rituals} You must focus to perform rituals. At the end of every round, if you took any damage, make a Concentration test. Failure means the ritual fails and has no effect. When a ritual fails, half of the ritual components are usually consumed during the ritual, the but remainder can be salvaged.

\parhead{Distracting Circumstances} In some circumstances, you need to test Concentration to cast spells or concentrate even if you haven't taken damage. Examples include being on a galloping horse, in a storm-tossed ship, or in an earthquake.

\parhead{Focus Limits} Focusing on a spell is mentally tiring. You can focus on a spell for up to 5 minutes without penalty. After 5 minutes, and every minute thereafter, you must test Concentration even if you haven't taken damage. If you fail, you lose your focus on the spell and become fatigued. The difficulty of the test increases by 2 for every additional minute of focus.

\subsection{Miscasting}\label{Miscasting}

If you start casting a spell and fail to complete it, such as if your concentration is broken or your armor interferes with your spellcasting, you miscast the spell. Each spell specifies what happens if you miscast it. In general, ranged spells that would affect a single target instead affect a random target, ranged spells which would affect an area instead originate from you, and other spells simply explode.

\subsection{Dismissing Spells}

As a swift action, you can dismiss any spells you cast that has lasting effects. This requires the same casting components (verbal and somatic) as casting the spell normally. The effects of a dismissed spell immediately end.

\subsection{Impossible Spell Effects}
If you try to cast a spell in circumstances that make the spell's effect impossible, the spell fails and has no effect. You still lose the spell slot used to cast it.

\section{Determining Spell Effects}

\subsection{Spellpower}

Both your magical attack bonus and the power of your spells is determined by your spellpower. Normally, your spellpower is equal to your level in your casting class. However, each class has a different way to increase its spellpower.

\begin{itemize}
    \item Clerics gain increased spellpower based on their devotion.
    \item Druids gain increased spellpower by attuning to natural environments.
    \item Sorcerers can use Willpower in place of their sorcerer level to determine their spellpower.
    \item Wizards can use Intelligence in place of their wizard level to determine their spellpower.
\end{itemize}

Effects that increase spellpower never increase spells per day or spells known. Only your class levels affect those values.

\parhead{Multiple Casting Classes} If you have levels in more than one spellcasting class, use the spellpower appropriate to the class that you are casting the spell from.

\parhead{Reducing Spellpower} You can voluntarily reduce the power of the spells you cast by using a lower spellpower. However, you cannot use a spellpower lower than the minimum spellpower required to cast the spell, which is equal to twice the spell's level.

\parhead{Spell Resistance} Some creatures have spell resistance, which is a special ability which allows them to resist spells. You can overcome spell resistance by making an attack with a bonus equal to your spellpower. See \pcref{Spell Resistance}, for details.

\subsection{Magical Attacks}

To affect an unwilling creature with a spell, you must make a magical attack. Your magic attack bonus is normally equal to your spellpower.

\subsubsection{Resisting a Spell} A creature that successfully resists a spell that has no obvious physical effects feels a hostile force or a tingle, but cannot deduce the exact nature of the attack without the use of the Spellcraft skill (see \pcref{Spellcraft}).

\subsubsection{Not Resisting a Spell} A creature can voluntarily forego its defenses and willingly accept a spell's result. However, a character with a special immunity to specific magical effects cannot suppress that quality.

\subsection{Targeting Spells}

\parhead{Midair Locations} A creature or object brought into being or transported to your location by a spell cannot appear floating in an empty space. It must arrive in an open location on a surface capable of supporting it.

\parhead{Targeting Inside Creatures} Creatures block line of effect to the inside of their own bodies. As a result, you cannot cast a spell that takes effect inside a creature unless you are also inside the creature. This restriction applies even if there is no physical barrier to the inside of the creature; you cannot detonate a \spell{fireball} inside a creature's mouth, even if it has its mouth open at the time.

\subsection{Special Spell Effects}

\subsubsection{Attacks}
Some spell descriptions refer to attacking. All offensive combat actions and special abilities, even those that don't damage opponents, are considered attacks. All spells that affect any unwilling creatures are considered attacks. If all creatures affected by a spell are willing, the spell is not considered an attack. Spells that damage objects or summon allies are not attacks because the spells themselves don't harm anyone.

\subsubsection{Resurrecting the Dead}\label{Resurrecting the Dead}

Several spells have the power to restore slain characters to life.

When a living creature dies, its soul departs its body, leaves the Material Plane, travels through the Astral Plane, and goes to abide on the plane where the creature's deity resides. If the creature did not worship a deity, its soul departs to the plane corresponding to its alignment. Bringing someone back from the dead means retrieving his or her soul and returning it to his or her body.

\subparhead{Negative Level} Any creature brought back to life gains a negative level. If the character was 1st level at the time of death, he or she takes 2 points of Constitution drain instead. This negative level or Constitution drain cannot be repaired by anything less than a \spell{wish} or \spell{miracle} spell, but it goes away naturally after a year has passed or after the character gains two new levels.

\subparhead{Preventing Revivification} Enemies can take steps to make it more difficult for a character to be returned from the dead. Except for \spell{true resurrection}, every ritual to raise the dead requires a body, so keeping or destroying the body is an effective deterrent. The \spell{soul bind} ritual prevents any sort of revivification unless the soul is first released.

\subparhead{Revivification against One's Will} A soul cannot be returned to life if it does not wish to be. A soul infallibly knows the name, alignment, and patron deity (if any) of the character attempting to revive it and may refuse to return on that basis.


\subsection{Combining Effects}
Spells or magical effects usually work as described, no matter how many other spells or magical effects happen to be operating in the same area or on the same recipient. Except in special cases, a spell does not affect the way another spell operates. Whenever a spell has a specific effect on other spells, the spell description explains that effect.

However, spells, feats, class features, and other effects that have very similar effects may not both help the subject. A character can only be increased so far beyond his or her normal limits; even layered with powerful magical effects, a commoner is no serious threat to a giant. The limitations on these effects are provided by the stacking rules described below.

\subsubsection{Stacking Effects}
Spells that provide bonuses or penalties usually do not stack with themselves. More generally, two enhancement bonuses don't stack even if they come from different spells; see \pcref{Stacking Rules}, for more details.

\parhead{Same Effect More than Once in Different Strengths} In cases when two or more identical spells are operating in the same area or on the same target, but at different strengths, only the best one applies. This is called overlapping.

\parhead{Same Effect with Differing Results} The same spell can sometimes produce varying effects if applied to the same recipient more than once. Usually, the last spell in the series trumps the others. None of the previous spells are actually removed or dispelled, but their effects become irrelevant while the final spell in the series lasts.

\parhead{One Effect Makes Another Irrelevant} Sometimes, one spell can render a later spell irrelevant. Both spells are still active, but one has rendered the other useless in some fashion.

\parhead{Multiple Mind Control Effects} Sometimes magical effects that affect a creature's mind render each other irrelevant, such as a spell that removes the subject's ability to act. Mental controls that don't remove the recipient's ability to act usually do not interfere with each other. If a creature is under the mental control of two or more creatures, it tends to obey each to the best of its ability, and to the extent of the control each effect allows. If the controlled creature receives conflicting orders simultaneously, the competing controllers must make opposed Charisma checks to determine which one the creature obeys.

\parhead{Spells with Opposite Effects} Spells with opposite effects apply normally, with all bonuses, penalties, or changes accruing in the order that they apply.

\parhead{Instantaneous Effects} Two or more spells with instantaneous durations work cumulatively when they affect the same target.

\section{Spell Descriptions}
The description of each spell is presented in a standard format. Each category of information is explained and defined below.

\subsection{Name}
The first line of every spell description gives the name by which the spell is generally known.

\subsection{Description}
Beneath the spell name is a brief description of the spell's effect. This description has no mechanical significance, and simply describes how the spell usually appears or is used.

\subsection{School/Schools}
The next line describes the schools of magic that the spell belongs to. Almost every spell belongs to at least one of eight schools of magic. A school of magic is a group of related spells that work in similar ways. They are described below.

Some spells belong to more than one school of magic. Treat these spells for all purposes as if they were a member of both schools simultaneously. If you are prohibited from casting spells from a certain school, you cannot cast a spell which belongs to that school, even if it also belongs to another school. Likewise, any benefits which apply to casting spells from a specific school apply normally. If you have abilities which apply when casting spells from both schools that make up a spell, the abilities do not stack.

A small number of spells (\spell{limited wish}, \spell{permanency}, \spell{prestidigitation}, and \spell{wish}) are universal, belonging to no school.

\subsubsection{Abjuration}
Abjuration spells manipulate the raw essence of magic. They often protect allies or ward off foes. Many abjurations spells also belong to another school of magic.

\subsubsection{Conjuration}
Conjuration spells transport and create objects and creatures to aid you. They can also transport you and your allies elsewhere.

\subsubsection{Divination}
Divination spells enable you to predict the future, gain or share knowledge, find hidden things, and foil deceptive spells.

\subsubsection{Enchantment}
Enchantment spells affect the minds of others, influencing or controlling their behavior or mental capabilities. Almost all enchantment spells are Mind spells, and many of them are Subtle as well.

\subsubsection{Evocation}
Evocation spells create and manipulate energy and forces or tap into divine or other powers to produce a desired end. In effect, they create energy or effects, but not physical objects, out of nothing. Many of these spells produce spectacular effects.

\subsubsection{Illusion}
Illusion spells deceive the senses of others. They conceal things that exist or cause people to perceive things that do not exist.

\subsubsection{Transmutation}
Transmutation spells change the properties of creatures and objects. They can grant new abilities, enhance existing abilities, change a target's form, or even alter the flow of time itself.

\subsubsection{Vivimancy}
Vivimancy spells manipulate the power of life and death, as well as souls. Spells involving positive and negative energy belong to this school.

\subsection{[Tags]}
Appearing on the same line as the school, when applicable, are tags which further categorizes the spell in some way. Some spells have more than one descriptor. Spell tags are described at \pcref{Spell Tags}.

\subsection{Level}
The next line of a spell description gives the spell's level, a number between 1 and 9 that defines the spell's relative power. This number is preceded by an abbreviation for the class whose members can cast the spell. The Level entry also indicates whether a spell is a domain spell and, if so, what its domain and its level as a domain spell are.

Names of spellcasting classes are abbreviated as follows: cleric Clr; druid Drd; sorcerer Sor; wizard Wiz.

The domains a spell can be associated with include Air, Chaos, Death, Destruction, Earth, Evil, Fire, Good, Knowledge, Law, Leadership, Magic, Nature, Protection, Strength, Travel, Trickery, Vitality, War, and Water.

\subsection{Components}
A spell's components are what you must do or possess to cast it. All spells have verbal and somatic components unless the spell description says otherwise. The Components entry in a spell description includes abbreviations that tell you what type of components it has. Specifics for material components and focuses are given at the end of the descriptive text.

\parhead{Verbal (V)} A verbal component is a spoken incantation. To provide a verbal component, you speak in a strong voice with a volume at least as loud as ordinary conversation.

A gag spoils the incantation (and thus the spell). A spellcaster who has been deafened has a 20\% chance to spoil any spell with a verbal component that he or she tries to cast. Likewise, a \spell{silence} spell imposes a 20\% chance of failure.

\parhead{Somatic (S)} A somatic component is a measured and precise movement of the hand. You must have at least one hand free to provide a somatic component. Touch range spells often include the act of touching the spell recipient as part of the somatic component.

\parhead{Material (M)} A material component is one or more physical substances or objects that are annihilated by the spell energies in the casting process.

\subsection{Casting Time}
All spells have a casting time of 1 standard action unless otherwise specified in the spell description. Some take 1 round or more, while a few require only a swift or immediate action.

If a spell takes more than a full-round action to cast, you must spend each round of the casting time taking a full-round action to cast the spell. The spell takes effect at the end of your last round of casting. These actions must be consecutive and uninterrupted, or the spell automatically fails.

You make all pertinent decisions about a spell (range, target, area, effect, version, and so forth) when the spell comes into effect.

\subsection{Range}
A spell's range indicates how far from you it can reach, as defined in the Range entry of the spell description. It indicates the maximum distance at which you can designate the spell's point of origin. The effect of a spell can extend beyond that range if it affects an area. A spell without a range simply affects the area specified in the spell's description; if it becomes relevant, you are considered to be its point of origin. Standard ranges include the following.
\parhead{Touch} You must touch a creature or object to affect it. Touching a creature requires a successful attack against its Reflex defense. A touch spell that deals damage can score a critical hit just as a weapon can. A touch spell threatens a critical hit on a natural roll of 20 and deals double damage on a successful critical hit. Some touch spells allow you to touch multiple targets. You can touch as many willing targets as you can reach as part of the casting, but all targets of the spell must be touched in the same round that you finish casting the spell. Normally, you can touch no more than six targets per round of casting.

If you have the ability to make multiple touch attacks, such as from the \spell{chill touch} spell, and you can make multiple attacks in a round, you can make a touch attack on each of those attacks.

\parhead{Close} The spell reaches as far as 30 feet.
\parhead{Medium} The spell reaches as far as 100 feet.
\parhead{Far} The spell reaches as far as 300 feet.
\parhead{Unlimited} The spell reaches anywhere on the same plane of existence.
\parhead{Range Expressed in Feet} Some spells have no standard range category, just a range expressed in feet.

\subsubsection{Unrestricted Ranges}

Some spells have an unrestricted range, as denoted by \rngunrestricted. A spell with an unrestricted range does not require line of sight or line of effect.

\subsection{Area}\label{Spell Area}

Some spells affect an area. Sometimes a spell description specifies a specially defined area, but usually an area falls into one of the categories defined below.

When casting an area spell, you select the point where the spell originates. The point of origin of a spell is always a grid intersection. When determining whether a given creature is within the area of a spell, count out the distance from the point of origin in squares just as you do when moving a character or when determining the range for a ranged attack. The only difference is that instead of counting from the center of one square to the center of the next, you count from intersection to intersection.

You can freely decrease a spell's area, provided that you decrease it uniformly across all of the spell's dimensions. For example, you can cast a \spell{fireball} that affects a 5 foot radius if you choose to do so, but you can't cast a \spell{fireball} with any shape other than a sphere.

You can count diagonally across a square, but remember that every second diagonal counts as 2 squares of distance. If the far edge of a square is within the spell's area, anything within that square is within the spell's area. If the spell's area only touches the near edge of a square, however, anything within that square is unaffected by the spell.

\subsubsection{Area Types}

\parhead{Burst} A burst spell has an immediate effect on all valid targets within an area.

\parhead{Emanation} An emanation spell has effects within an area for the duration of the spell. It emanates from a specific creature or object, rather than a location. If that creature or object moves, the emanation moves with it.

\parhead{Limit} A limit spell has an immediate effect within an area. It may affect specific targets of your choice, or it may create effects at locations within the area, but it will not affect the entire area at once.

\parhead{Zone} A zone spell has effects within an area for the duration of the spell. Unless otherwise noted, it does not move after being created.

\subsubsection{Area Shapes}

\parhead{Cone} A cone extends from the point of origin in a quarter-sphere, up to the given length.

\parhead{Cylinder} A cylinder extends out from the point of origin in a circle, up to the given radius. Cylinders also have a specific height. Unless otherwise specified, a cylinder's height is the same as its radius. Cylinders ignore obstacles that partially block line of effect, as long as there is a path around the obstacle that lies entirely within the spell's area.

\parhead{Line} A line extends from the point of origin in a straight line, up to the given length. Lines also have a specific width and height. Unless otherwise specified, a line-shaped spell affects an area 5 feet wide and 5 feet high. The affected squares are chosen such that they stay close to the chosen line as possible. All squares affected by a line must be contiguous, so every square is adjacent to another affected square, disregarding diagonals.

\subparhead{Shapeable Lines}\label{Shapeable} Some lines are shapeable, as denoted by \shapeable. A shapeable line can make 90 degree turns at any point in its path, which you can freely determine within the normal limitations.

\parhead{Sphere} A sphere extends from the point of origin in all directions. Any spell which only specifies a radius for its area is sphere-shaped.

\parhead{Wall} A wall is like a line, except that it has no width. Instead, it affects the boundary between squares. Walls can also be shapeable.

Walls can normally be created within occupied squares, but not within solid objects. Some walls are called solid walls, and cannot be created within occupied squares.

\parhead{Specific Shapes} Some spells specify a series of volumes that make up the area of the spell. Most commonly, the volumes are cubes. You may arrange the volumes as you want, with the restriction that each volume in the spell's area must be adjacent to one other volume in the spell's area.

\subsubsection{Area Sizes}

The area affected by many spells falls into one of three sizes. Each size defines the extent to which the spell extends out from its origin, whether as a radius or as a length. Some spells have specific sizes, as given in the spell description.

\parhead{Small} Small spells extend 10 feet from their point of origin.
\parhead{Medium} Medium spells extend 20 feet from their point of origin.
\parhead{Large} Large spells extend 50 feet from their point of origin.

\subsection{Targets}
Some spells have a target or targets. You cast these spells on creatures or objects, as defined by the spell itself. You must be able to see or touch the target, and you must specifically choose that target. You do not have to select your target until you finish casting the spell.

\subparhead{Multiple Targets} Most spells which have multiple targets also specify an area that the targets must reside in. If the spell says ``all creatures'', you do not have the ability to choose which creatures it affects; otherwise, you may pick and choose creatures within the area.

\subparhead{Redirecting a Spell} Some spells allow you to redirect the effect to new targets or areas after you cast the spell. Redirecting a spell is a swift action.

\subparhead{Targeting Restrictions} Many spells affect ``living creatures,'' which means all creatures other than constructs and undead. Creatures in the spell's area that are not of the appropriate type do not count against the creatures affected.

\subparhead{Willing Targets} Some spells restrict you to willing targets only. You can choose to be a willing target at any time. Unconscious creatures and objects are automatically considered willing, but a character who is conscious but immobile or helpless (such as one who is bound, cowering, grappling, paralyzed, pinned, or stunned) is not automatically willing.

\subparhead{Ray} Some effects are rays. You aim a ray as if using a ranged weapon, though typically you make a ranged touch attack rather than a normal ranged attack. As with a ranged weapon, you can fire into the dark or at an invisible creature and hope you hit something. You don't have to see the creature you're trying to hit, as you do with a targeted spell. Intervening creatures and obstacles, however, can block your line of sight or provide cover for the creature you're aiming at.

If a ray spell has a duration, it's the duration of the effect that the ray causes, not the length of time the ray itself persists.

If a ray spell deals damage, you can score a critical hit just as if it were a weapon. A ray spell threatens a critical hit on a natural roll of 20 and deals double damage on a successful critical hit.

\parhead{Line of Effect} A line of effect is a straight, unblocked path that indicates what a spell can affect. A line of effect is canceled by a solid barrier. It's like line of sight for ranged weapons, except that it's not blocked by fog, darkness, and other factors that limit normal sight.

You must have a clear line of effect to any target that you cast a spell on or to any space in which you wish to create an effect. You must have a clear line of effect to the point of origin of any spell you cast.

A burst, cone, cylinder, or emanation spell affects only an area, creatures, or objects to which it has line of effect from its origin (a spherical burst's center point, a cone-shaped burst's starting point, a cylinder's circle, or an emanation's point of origin).

An otherwise solid barrier with a hole of at least 1 square foot through it does not block a spell's line of effect. Such an opening means that the 5-foot length of wall containing the hole is no longer considered a barrier for purposes of a spell's line of effect.

\subsection{Duration}

A spell's Duration entry tells you how long the magical energy of the spell lasts.

\parhead{Focus} The spell lasts as long as you focus on it. Focusing to maintain a spell is a standard action. Anything that could break your concentration when casting a spell can also break your concentration while you're maintaining one, causing your focus to end. You can't cast a spell while focusing on another one.
\parhead{Timed Durations} Many durations are measured in rounds, minutes, hours, or some other increment. When the time is up, the magic goes away and the spell ends at the end of your turn.

Spells end after the given length of time has completely elapsed. For example, a spell that lasts 1 round ends at the end of the next round after it is cast, not at the end of the round in which it is cast.

If a spell's duration is variable, the duration is rolled secretly (the caster doesn't know how long the spell will last).
\subparhead{Short} The spell lasts for as long as you focus, plus 5 additional rounds.
\subparhead{Medium} The spell lasts for 5 minutes.
\subparhead{Long} The spell lasts for 1 hour.
\subparhead{Extreme} The spell lasts for 12 hours.
\subparhead{Other} Some spells, such as long-lasting rituals, last for a specific amount of time specified in the spell description.
\parhead{Permanent} The energy remains as long as the effect does. This means the spell is vulnerable to \spell{dispel magic}.

\parhead{Instantaneous} Spells without a listed duration are instantaneous. spell energy comes and goes the instant the spell is cast, though the consequences might be long-lasting.

\parhead{Subjects, Effects, and Areas} If the spell affects creatures directly, the result travels with the subjects for the spell's duration. If the spell creates or summons objects or creatures, they last for the duration, and are capable of moving outside the spell's initial area. Such effects can sometimes be destroyed prior to when its duration ends. If the spell creates an emanation, then the spell stays with that area for its duration. Creatures become subject to the spell when they enter the emanation and are no longer subject to it when they leave.

\parhead{Discharge} Occasionally a spells lasts for a set duration or until triggered or discharged.

\subsection{Spell Resistance}\label{Spell Resistance}
Spell resistance is like an additional defense against spells and spell-like abilities. When you cast a spell or use a spell-like ability that directly affects a creature or object with spell resistance, you must make a special attack with a bonus equal to your spellpower. If that attack beats a target's spell resistance, the spell or spell-like ability works normally on that target. Otherwise, it has no effect on the target. This does not prevent the spell from having its normal effect on other creatures or objects.

Spells which do not directly affect targets, such as \spell{summon monster I} or \spell{silent image}, do not allow spell resistance. In addition, Antimagic and Physical spells do not allow spell resistance.

Normally, creatures with spell resistance can choose to allow spells through their resistance. Some creatures cannot control their spell resistance, so an attack is always necessary to affect them. This is specified in the description of the creature's spell resistance.

\subsection{Effect}
This portion of a spell description details what the spell does and how it works. If one of the previous entries in the description included ``see text,'' this is where the explanation is found. There are several key parts of a spell which are also contained here.

\subsubsection{Damage}
This is the amount of damage the spell deals. Typically, the effect will specify who takes the damage. If no effect is specified, the spell damages all of its targets, or all creatures (but not objects) in the area. A spell with this entry is considered a damaging spell. A spell without this entry is not, even if it could be used to deal damage.

Spells can inflict many kinds of damage. Common damage types include acid, bludgeoning, cold, divine, electricity, fire, force, life, physical, piercing, slashing, solar, and sonic.

\parhead{Damaging Items} Unless the descriptive text for the spell specifies otherwise, all items carried or worn by a creature are assumed to survive a magical attack. If an item is not carried or worn and is not magical, it does not get any defenses. It simply is dealt the appropriate damage.

\subsubsection{Healing}
This is the amount of damage the spell heals. Typically, the effect will specify who receives the healing. If no effect is specified, the spell heals all of its targets, or all creatures (but not objects) in the area.

\subsubsection{Healthy Effect}
This is the effect a spell has on a healthy subject (above half hit points remaining).

\subsubsection{Bloodied Effect}
This is the effect a spell has on a bloodied subject (at or below half hit points remaining). If the spell has a duration, and a healthy creature becomes bloodied during the duration of the spell, it immediately suffers the bloodied effect of the spell. If the spell does not have a duration, any damage the subject takes after being affected by the spell does not change which effect the spell has.


\section{Spell Tags}\label{Spell Tags}

Most spells have tags that describe the spell's effects or nature. Many of these tags have no game effect by themselves, but they govern how the spell interacts with other spells, with special abilities, with unusual creatures, with alignment, and so on. They are described on \trefnp{Spell Tags}.

{
    \onecolumn
    \taburowcolors [2] 2{tbrown .. white}
    \begin{longtabu}{l X}
        \lcaption{Spell Tags} \\
        \thead{Tag} & \thead{Effect} \\
        Acid & \x \\
        Air & The spell has no effect in environments without air. \\
        Alteration & \x \\
        Antimagic & The spell attacks magic itself. It does not allow spell resistance. \\
        Auditory & Creatures that cannot hear the spell's effect are immune to the spell. \\
        Augment & \x \\
        Barrier & If you attempt to move the spell's area towards a force or creature it prohibits, you feel a discernible pressure. If you continue moving, the spell ends. \\
        Charm & \x \\
        Chaotic & \x \\
        Cold & \x \\
        Compulsion & The spell forcibly alters a creature's actions, but does not necessarily affect its opinions or personality. \\
        Creation & The spell creates a physical object. \\
        Curse & The spell cannot be dispelled. It can be removed with \spell{break enchantment} or \spell{remove curse}. \\
        Death & The spell only affects living creatures. A creature killed by a death effect cannot be returned to life by \spell{lesser resurrection}. \\
        Delusion & The spell alters a creature's opinions or personality, but does not necessarily affect its actions. \\
        Detection & The spell's area can penetrate physical objects. Unless otherwise specified, it is blocked by 1 foot of stone, 1 inch of common metal, a thin sheet of lead, or 3 feet of wood or dirt. \\
        Disease & \x \\
        Earth & \x \\
        Electricity & \x \\
        Evil & \x \\
        Fear & \x \\
        Figment & The spell creates a false sensation shared by everyone viewing or otherwise perceiving the figment. Figments cannot remove real sensations present in their area, but they can add additional false sensations. You can only create figments of sensations you understand; for example, you cannot create a figment which speaks in a language you do not understand.
        \par A figment's physical defenses are equal to 10 \add its size modifier. \\
        Fire & The spell has no effect underwater. It provides light as a torch. \\
        Flesh & The spell has no effect on creatures without flesh, such as ghosts or oozes. \\
        Fog & The spell has effect underwater. If at least 5 points of fire damage are dealt in a square within the spell's area, the fog is destroyed in that area. Unless otherwise specified, a moderate wind (11\add mph) disperses the fog in 5 rounds, and a strong wind (21\add mph) disperses the fog in 1 round. \\
        Force & If the spell is cast on the Material Plane, it also affects the Ethereal Plane. \\
        Glamer & \x \\
        Good & \x \\
        Instantaneous & The spell instantly causes some change which is thereafter nonmagical. It cannot be dispelled or dismissed. \\
        Lawful & \x \\
        Life & The spell only affects living creatures. \\
        Light & The spell's effect is blocked by effects that block sight, and can pass through barriers that do not block sight. \\
        Mind & The spell has no effect on creatures with an Intelligence of \minus8 or lower. \\
        Morale & \x \\
        Negative & \x \\
        Physical & The spell creates or modifies physical objects rather than using a direct magical effect. The spell does not allow spell resistance. \\
        Planar & \x \\
        Poison & \x \\
        Positive & \x \\
        Retributive & A creature may only benefit from one retributive effect at once. \\
        Scrying & The spell creates an invisible magical sensor that sends you information. Unless noted otherwise, the sensor has the same powers of sensory acuity that you possess. This level of acuity includes any spells or effects that target you, but not spells or effects that emanate from you. However, the sensor is treated as a separate, independent sensory organ of yours, and thus it functions normally even if you have been blinded, deafened, or otherwise suffered sensory impairment.
        Any creature trained in Spellcraft can notice the sensor by making a DC 20 Spellcraft check. The sensor can be dispelled as if it were an active spell. Lead sheeting or magical protection blocks a scrying spell, and you sense that the spell is so blocked. \\
        Shielding & \x \\
        Sizing & Multiple size increasing or size decreasing effects never stack. Opposing size modifications cancel each other out on one for one basis, and any remaining effect occurs normally. \\
        Sonic & \x \\
        Speech & The spell be cast using a specific language you know. The spell has no effect on targets that do not understand the language used. \\
        Subtle & \subtlespellnotes \\
        Summoning & The spell manifests a creature from another plane. The manifestation immediately disappears when it dies, its hit points reach 0, or the spell's duration ends. Damage to the manifestation does not affect the original creature. Summoned creatures cannot use summoning abilities. Summoned creatures must be summoned within the spell's range, but can travel beyond that range freely. \\
        Telekinesis & \x \\
        Teleportation & A teleported creature can bring along equipment and held objects as long as their weight doesn't exceed its maximum load. Any excess items are left behind, in order of their distance from the creature's body. \\
        Temporal & \x \\
        Trap & The spell has no immediately obvious effect, but it can be detected with the Perception skill. Unless otherwise noted, it is also possible to disable the spell with the Devices skill before it triggers. The DC to detect and disable the spell is equal to 25 \add spell level.
        \par No more than one trap spell can be placed on the same object or in the same area. Only the first trap placed has any effect. It must be dispelled or discharged before any new traps can be placed. \\
        \label{Unreal Spells}Unreal & The spell can be recognized as unreal with a Perception check against a DC equal to 10 \add your spellpower. Creatures get a \plus10 bonus on this Perception check when using senses which should be present in the figment, but which are missing. Alternately, if the figment has a visual manifestation, physical contact can reveal it as unreal.

        A creature faced with definitive proof that the figment is unreal can disbelieve it, treating it as if it were not there. Its effects can still be observed if desired, but they are mere shadows of the full effect: visual effects appear translucent outlines, sounds can be heard as ghostly echoes, and so on. \\
        Visual & Creatures that cannot see the spell's effect are immune to the spell. \\
        Water & \x \\
    \end{longtabu}
    \twocolumn
}

\section{Cantrips}\label{Cantrips}
Cantrips are special spell-like abilities that arcane casters can use at will. Unlike other spell-like abilities, they have verbal and somatic components and are subject to arcane spell failure. All cantrips take a standard action to cast unless specified otherwise in the description. Cantrips are considered to be 0th level for the purpose of spells and abilities which reference spell level. They are described at the end of Chapter 12.

\section{Rituals}\label{Rituals}
Rituals are ceremonies that create magical effects. Spellcasting characters, or characters with the Ritual Caster feat, can learn and perform rituals. You don't memorize a ritual as you would a normal spell; rituals are too complex for all but the mightiest wizards to commit to memory. To perform a ritual, you need to read from a book or a scroll containing it. Rituals are considered to be spells for many purposes, such as for spell resistance and for effects related to spells, but they are learned and cast in very different ways.
\subsection{Ritual Descriptions}
\par Like a spell, each ritual has a school, a level, and a magical effect. Rituals are described in Chapter 13. The description of each ritual follows the same format as the description of spells in Chapter 12, except for the description of ritual levels. Unlike spells, which have levels based on class, rituals have levels based on the source of magic: arcane, divine, or nature. A ritual always matches the magic source of the person performing the ritual. For example, \spell{scrying} is an arcane ritual when performed by a wizard, but a divine ritual when performed by a cleric.
\subsection{Ritual Requirements} In order to learn and perform a ritual, you must be able to cast at least one spell of the same level as the ritual.
\subsection{Ritual Books}
A ritual book contains one or more rituals that you can use as frequently as you like, as long as you can spare the time and the components to perform the ritual. Scribing a ritual in a ritual book costs an amount of precious inks.
\subsection{Ritual Components}
Every ritual has a material component cost. This cost can be paid with precious metals or gems, or with special materials designed to perform rituals. Some rituals have unique costs and may require specific material components.
\subsection{Ritual Costs}
The costs to scribe and perform rituals are described on \trefnp{Ritual Costs}.
\begin{dtable}
    \lcaption{Ritual Costs}
    \begin{dtabularx}{\columnwidth}{X l l l}
        \thead{Ritual Level} & \thead{Cost to Perform} & \thead{Cost to Scribe} & \thead{Item Level} \\
\hline
        1st-Level & 5 gp & 50 gp & 1st \\
        2nd-Level & 20 gp & 200 gp & 3rd \\
        3rd-Level & 50 gp & 500 gp & 4th \\
        4th-Level & 125 gp & 1250 gp & 7th \\
        5th-Level & 300 gp & 3000 gp & 9th \\
        6th-Level & 750 gp & 7500 gp & 11th \\
        7th-Level & 1500 gp & 15000 gp & 12th \\
        8th-Level & 3500 gp & 35000 gp & 14th \\
        9th-Level & 7500 gp & 75000 gp & 16th \\
    \end{dtabularx}
\end{dtable}

\subsection{Performing Rituals}
\par To perform a ritual, you must have a ritual book containing the ritual and the material components required for the ritual. If you are distracted during the ritual, you must make a Concentration test, just as if you were casting a spell of the ritual's level. If you fail, the ritual is ruined and you must start from the beginning. You can generally recover half the material components from an interrupted ritual.
\par Performing a ritual and casting a ritual mean the same thing.
\subsection{Magical Writings}
To record a spell in written form, a character uses complex notation that describes the magical forces involved in the spell. The notation constitutes a universal language that spellcasters have discovered, not invented. The writer uses the same system no matter what her native language or culture. However, each character uses the system in her own way. Another person's magical writing remains incomprehensible to even the most powerful spellcaster until she takes time to study and decipher it.

To decipher an magical writing (such as a single spell in written form on a scroll), a character must make a Spellcraft check (DC 20 \add the spell's level). If the skill check fails, the character cannot attempt to read that particular spell again until the next day. A \spell{read magic} spell automatically deciphers a magical writing without a skill check. If the person who created the magical writing is on hand to help the reader, success is also automatic.

Once a character deciphers a particular magical writing, she does not need to decipher it again. Deciphering a magical writing allows the reader to identify the spell and gives some idea of its effects (as explained in the spell description). If the magical writing was a scroll and the reader can cast spells of the appropriate type, she can attempt to use
the scroll.

\section{Types of Magical Abilities}

\parhead{Spell-Like Abilities} Usually, a spell-like ability works just like the spell of that name. A few spell-like abilities are unique; these are explained in the text where they are described.

A spell-like ability has no verbal, somatic, or material component, nor does it require a focus or have an XP cost. The user activates it mentally. Armor never affects a spell-like ability's use, even if the ability resembles an arcane spell with a somatic component.

A spell-like ability has the casting time of the spell it mimics unless noted otherwise in the ability description. In all other ways, a spell-like ability functions just like a spell.

Spell-like abilities are subject to spell resistance and to being dispelled by \spell{dispel magic}. They do not function in areas where magic is suppressed or negated. Spell-like abilities cannot be used to counterspell, nor can they be counterspelled.

Some creatures are actually sorcerers of a sort. They cast arcane spells as sorcerers do, using components when required. In fact, an individual creature could have some spell-like abilities and also cast other spells as a sorcerer.

\parhead{Supernatural Abilities} These abilities do not require concentration and cannot be disrupted in combat, as spells can. Supernatural abilities are not subject to spell resistance, counterspells, or to being dispelled by dispel magic, and do not function in areas where magic is suppressed or negated.

\parhead{Extraordinary Abilities} These abilities do not require concentration and cannot be disrupted in combat, as spells can. Effects or areas that negate or disrupt magic have no effect on extraordinary abilities. They are not subject to dispelling, and they function normally in an \spell{antimagic field}. Indeed, extraordinary abilities do not qualify as magical, though they may break the laws of physics.

\parhead{Natural Abilities} This category includes abilities a creature has because of its physical nature. Natural abilities are those not otherwise designated as extraordinary, supernatural, or spell-like.

