%\newcommand{\Variable}{0}
%\newcommand{\storevalue}[1]{\renewcommand{\Variable}{#1}}

\newcommand{\lcaption}[1]{\caption{#1}\label{cap:#1}}

\newcommand{\pref}[1]{page \pageref{#1}}
\newcommand{\tref}[1]{Table \ref{cap:#1}: #1 (\pref{cap:#1})}
\newcommand{\trefnp}[1]{Table \ref{cap:#1}: #1}

\newcommand{\minus}{\texttt{-{}}}
\newcommand{\plus}{\texttt{+}}    %Use a smaller positive marker for use directly adjacent to numbers
\newcommand{\add}{\texttt{+}\xspace}    %Use a smaller plus sign
\newcommand{\sub}{\texttt{-}\xspace}
\newcommand{\mult}{x}
\newcommand{\x}{x\xspace}
%NOTED BUG: Spell names with mixed capitalization (Dispel magic) will not work properly.
\newcommand{\spell}[1]{\emph{\mbox{\hyperlink{spell:#1}{#1}}}}    %Italicize spells
\newcommand{\spellindirect}[2]{\emph{\hyperlink{spell:#1}{#2}}}
\newcommand{\ritual}[1]{\emph{#1}}    %Italicize spells
\newcommand{\subcf}[1]{\parhead{#1}}    %Bold sub-class-feature headers
\newcommand{\subsk}[1]{\par \emph{#1}:}    %Italicize sub-skill headers

\newcommand{\subparhead}[1]{\par\emph{#1}:}
%\newcommand{\thead}[1]{\textbf{#1}}    %Bold the header of tables
\newcommand{\tb}[1]{\textbf{#1}}    %Bold the header of tables
\newcommand{\tdash}{\hskip 0.1em---\xspace}    %A null entry in a table
\newcommand{\ccol}{\centering\arraybackslash}    %Center columns
\newcommand{\lcol}{\raggedright\arraybackslash}
\newcommand{\rcol}{\raggedleft\arraybackslash}

\newcommand{\fn}[1]{\textsuperscript{#1}}
\newcommand{\subspell}[1]{\emph{#1}}    %For parts of spells
\newcommand{\magicitem}[1]{\emph{#1}}    %For the names of items
\newcommand{\mitem}[1]{\emph{#1}}

\newcommand{\tind}{\hspace{1em}}

\newcommand{\spelldesc}[1]{\par\noindent \textit{#1}}
\newcommand{\spelltwocol}[2]{\spelltabularcompressed #1 & #2\end{tabularx}}

\newcommand{\rngpers}{Personal}
\newcommand{\rngtouch}{Touch}
\newcommand{\rngclose}{Close \reminder{30 ft.}\xspace}
\newcommand{\rngmed}{Medium \reminder{100 ft.}\xspace}
\newcommand{\rnglong}{Long \reminder{300 ft.}\xspace}
\newcommand{\rngext}{Extreme \reminder{1,000 ft.}\xspace}
\newcommand{\areasmall}{Small \reminder{10 ft.}\xspace}
\newcommand{\areamed}{Medium \reminder{20 ft.}\xspace}
\newcommand{\arealarge}{Large \reminder{50 ft.}\xspace}
\newcommand{\areahuge}{Huge \reminder{100 ft.}\xspace}

\newcommand{\confusionexplanation}{A confused creature cannot take actions normally. If it is attacked, it automatically attacks a random attacker. Otherwise, at the beginning of each round, it randomly decides to take one of four actions that round: babble incoherently, flee from the caster as if panicked, attack the nearest creature, or act normally. A confused character who can't carry out the indicated action does nothing but babble incoherently.\xspace}

\newcommand{\featpre}{\parhead{Prerequisite} }
\newcommand{\featpres}{\parhead{Prerequisites} }
%featref no target
\newcommand{\featpref}[1]{\pageref{feat:#1}}
\newcommand{\mitempref}[1]{\pageref{item:#1}}
\newcommand{\featpcref}[1]{#1, page \pageref{feat:#1}}

\newcommand{\skill}[2]{\section{#1 (#2)}\label{#1}}

\newcommand{\reminder}[1]{\textcolor{darkgray}{\textit{(#1)}}}

%\newcommand{\vulnerableexplanation}{A vulnerable creature takes a \minus2 penalty to attacks, defenses, and checks.\xspace}
%\newcommand{\vulneffect}{\minus2 to attacks, defenses, and checks\xspace}

\newcommand{\blinded}{\glossterm{blinded}\xspace}
\newcommand{\bloodied}{\glossterm{bloodied}\xspace}
\newcommand{\charmed}{\glossterm{charmed}\xspace}
\newcommand{\confused}{\glossterm{confused}\xspace}
\newcommand{\defenseless}{\glossterm{defenseless}\xspace}
\newcommand{\disoriented}{\glossterm{disoriented}\xspace}
\newcommand{\dominated}{\glossterm{dominated}\xspace}
\newcommand{\exhausted}{\glossterm{exhausted}\xspace}
\newcommand{\fascinated}{\glossterm{fascinated}\xspace}
\newcommand{\fatigued}{\glossterm{fatigued}\xspace}
\newcommand{\frightened}{\glossterm{frightened}\xspace}
\newcommand{\dazed}{\glossterm{dazed}\xspace}
\newcommand{\dazzled}{\glossterm{dazzled}\xspace}
\newcommand{\deafened}{\glossterm{deafened}\xspace}
\newcommand{\grappled}{\glossterm{grappled}\xspace}
\newcommand{\goaded}{\glossterm{goaded}\xspace}
\newcommand{\helpless}{\glossterm{helpless}\xspace}
\newcommand{\ignited}{\glossterm{ignited}\xspace}
\newcommand{\immobilized}{\glossterm{immobilized}\xspace}
\newcommand{\nauseated}{\glossterm{nauseated}\xspace}
\newcommand{\panicked}{\glossterm{panicked}\xspace}
\newcommand{\paralyzed}{\glossterm{paralyzed}\xspace}
\newcommand{\petrified}{\glossterm{petrified}\xspace}
\newcommand{\prone}{\glossterm{prone}\xspace}
\newcommand{\shaken}{\glossterm{shaken}\xspace}
\newcommand{\sickened}{\glossterm{sickened}\xspace}
\newcommand{\slowed}{\glossterm{slowed}\xspace}
\newcommand{\staggered}{\glossterm{staggered}\xspace}
\newcommand{\stunned}{\glossterm{stunned}\xspace}
\newcommand{\taunted}{\glossterm{taunted}\xspace}
\newcommand{\wounded}{\glossterm{wounded}\xspace}
\newcommand{\unaware}{\glossterm{unaware}\xspace}

\newcommand{\concealment}{concealment\xspace}

\newcommand{\spelltablecolumns}{>{\lcol}X l}
\newcommand{\spelllevelschool}[2]{\spelllvl{#1} & #2 \\}
\newcommand{\spelllevelnew}[3]{\tb{\nth{#1} level} #2 & #3 \\}

\newcommand{\dismissable}{}

% http://tex.stackexchange.com/questions/135617/how-to-i-reduce-space-before-and-after-hrule
\newcommand{\HRule}[2][\medskipamount]{\par
  \vspace*{\dimexpr-\parskip-\baselineskip+#1}
  \noindent\rule{\linewidth}{#2}\par
  \vspace*{\dimexpr-\parskip-.5\baselineskip+#1}}

\newcommand{\spellline}{\par\HRule[0.5\medskipamount]{0.2pt}}
% weird spacing before twocol
\newcommand{\spelllinespaced}{\spellline\vspace{0.5\medskipamount}}

%\newcommand{\spelleffect}{\fieldhead{Effect}}
%\newcommand{\spellsuccess}{\fieldhead{Success}}
%\newcommand{\spellfailure}{\fieldhead{Failure}\xspace}
\newcommand{\spellnotes}{\par\noindent\textit{Notes}:\xspace}
\newcommand{\spellspecial}{\fieldhead{Special}}
\newcommand{\spellcheck}[1]{\fieldhead{Check} #1}

\ExplSyntaxOn

\DeclareDocumentCommand{\Create}{m}
{
    \newcounter{#1}
    \setcounter{#1}{0}
}

\DeclareDocumentCommand{\Set}{m m}
{
    \setcounter{#1}{#2}
}

\DeclareDocumentCommand{\Get}{m}
{
    \arabic{#1}
}

\DeclareDocumentCommand{\GetValue}{m}
{
    \value{#1}
}

\newcommand{\TempVar}{0}
\newcommand{\TempVarTwo}{0}

\DeclareDocumentCommand{\parhead}{s m o}
{
    \vspace{0.25em}
    \par \noindent
    \IfNoValueTF{#3}
    {\textbf{#2}:}
    {\textbf{#2}~[#3]:}\xspace
}

\DeclareDocumentCommand{\fieldhead}{m o}
{
    \par\noindent
    \IfNoValueTF{#2}
    {\textbf{#1}:}
    {\textbf{#1}~[#2]:}\xspace
}

\DeclareDocumentCommand{\spellfailure}{o}
{
    \IfNoValueTF{#1}
    {\fieldhead{Failure}}
    {\fieldhead{Failure}[#1]}
}
\DeclareDocumentCommand{\spellsuccess}{o}
{
    \IfNoValueTF{#1}
    {\fieldhead{Hit}}
    {\fieldhead{Hit}[#1]}
}
\DeclareDocumentCommand{\spellcritical}{o}
{
    \IfNoValueTF{#1}
    {\fieldhead{Critical~Hit}}
    {\fieldhead{Critical~Hit}[#1]}
}
\DeclareDocumentCommand{\spelleffect}{o}
{
    \IfNoValueTF{#1}
    {\fieldhead{Effect}}
    {\fieldhead{Effect}[#1]}
}
\DeclareDocumentCommand{\spelltgt}{m o}
{
    \IfNoValueTF{#2}
    {\fieldhead{Target} #1}
    {\fieldhead{#2~Target} #1}
}
\DeclareDocumentCommand{\spelltgts}{m o}
{
    \IfNoValueTF{#2}
    {\fieldhead{Targets} #1}
    {\fieldhead{#2~Targets} #1}
}

\DeclareDocumentCommand{\ritualcostparser}{m}
{
    \ifcase#1\relax
    \or 5~gp    %1
    \or 20~gp   %2
    \or 50~gp   %3
    \or 125~gp  %4
    \or 300~gp  %5
    \or 750~gp  %6
    \or 1,500~gp %7
    \or 3,000~gp %8
    \or 7,500~gp %9
    \else???
    \fi
}

\DeclareDocumentCommand{\ritualcost}{s m o}
{
    \IfBooleanF{#1}{\spellline}
    \spellmat{\ritualcostparser#2\xspace
        \IfValueTF{#3}
        {#3}
        {in~ritual~components.}
    }
}

% args:
%   magic item price or type of price
%   effective spell level
\DeclareDocumentCommand{\mitempricewithlevel}{m m}
{
    \IfSubStr{#1}{Passive}
    {
        %\renewcommand{\TempVarTwo}{\magicitempricepassive{#2}}
        \magicitempricepassive{#2}
    }
    {
        \IfSubStr{#1}{Active}
        {
            \magicitempriceactive{#2}
        }
        {
            \renewcommand{\TempVarTwo}{#1}
        }
    }

    \StrLen{\TempVarTwo}[\TempVar]
    \pgfmathparse{\TempVar > 4}
    \ifcase\pgfmathresult
        % handle normal magic items
        \fieldhead{Price~(Level)} \num{\TempVarTwo}~gp~(\cheapitemlevel{\TempVarTwo})
    \or
        % handle very expensive magic items
        % by passing the number with the right three digits removed
        \StrGobbleRight{\TempVarTwo}{3}[\TempVar]
        \fieldhead{Price~(Level)} \num{\TempVarTwo}~gp~(
            \magicitemexpensiveitemlevel{\TempVar}
        )
    \fi
}

\DeclareDocumentCommand{\mitempricetable}{m m}
{
    \IfSubStr{#1}{Passive}
    {
        %\renewcommand{\TempVarTwo}{\magicitempricepassive{#2}}
        \magicitempricepassive{#2}
    }
    {
        \IfSubStr{#1}{Active}
        {
            \magicitempriceactive{#2}
        }
        {
            \renewcommand{\TempVarTwo}{#1}
        }
    }
    \xdef\ItemCost{\num{\TempVarTwo}~gp}

    \StrLen{\TempVarTwo}[\TempVar]
    \pgfmathparse{\TempVar > 4}
    \ifcase\pgfmathresult
        % handle normal magic items
        \xdef\ItemLevel{\cheapitemlevel{\TempVarTwo}}
    \or
        % handle very expensive magic items
        % by passing the number with the right three digits removed
        \StrGobbleRight{\TempVarTwo}{3}[\TempVar]
        \xdef\ItemLevel{\magicitemexpensiveitemlevel{\TempVar}}
    \fi
    \ItemCost & \ItemLevel
}

% args:
%    spell level
\DeclareDocumentCommand{\magicitempriceactive}{m}
{
    \renewcommand{\TempVarTwo}{
        \ifcase#1~
        100    %0
        \or 200   %1
        \or 800   %2
        \or 2000  %3
        \or 5000  %4
        \or 12000  %5
        \or 30000 %6
        \or 60000 %7
        \or 140000 %8
        \or 300000 %9
        \else 0
        \fi
    }
}

% args:
%    spell level
\DeclareDocumentCommand{\magicitempricepassive}{m}
{
    \renewcommand{\TempVarTwo}{
        \ifcase#1~
        200   %0
        \or 800   %1
        \or 2000  %2
        \or 5000  %3
        \or 12000  %4
        \or 30000 %5
        \or 60000 %6
        \or 140000 %7
        \or 300000 %8
        \or 700000 %9
        \else 0
        \fi
    }
}

% If #1 = #2
%     print #3.
% Elseif #4 is provided
%     print #4
\DeclareDocumentCommand{\ifstr}{m m m o}
{
    \ifnum 0=\pdfstrcmp{#1}{#2}
        #3
    \else
        \IfValueT{#4}{#4}
    \fi
}

% args:
%   pre-spell header spell name, spell name modifier (Greater, etc.)
\DeclareDocumentCommand{\spellhead}{o m o}
{
    \IfValueTF{#1}
    {\item[#1]}
    {\item}
    \textbf{
        \IfValueTF{#3}
        {
            \hyperlink{spell:#3~#2}{#2,~#3}
            %\pageref{spell:#3~#2}
        }
        {
            \hyperlink{spell:#2}{#2}
            %\pageref{spell:#2}
        }
    }:
}

\DeclareDocumentCommand{\spellinfo}{m s m}
{
    \IfBooleanTF{#2}
    {
        \noindent#1
        \fieldhead{Lists} #3
    }
    {
        \spelltwocol{\fieldhead{Schools}~#1}{\fieldhead{Lists}~#3}
    }
}

\DeclareDocumentCommand{\spelltags}{m}
{
    \fieldhead{Tags} #1
}

\DeclareDocumentCommand{\imath}{m}
{
    %\pgfmathparse{int(floor(#1))}\pgfmathresult
    \pgfmathsetmacro{\TempVar}{int(floor(#1))}
    \TempVar
}

\DeclareDocumentCommand{\imathparse}{m}
{
    \pgfmathsetmacro{\TempVar}{int(floor(#1))}
}

\DeclareDocumentCommand{\imathresult}{}
{
    \TempVar
}

\DeclareDocumentCommand{\mathnth}{m}
{
    \pgfmathparse{int(floor(#1))}\nth{\pgfmathresult}
}

\DeclareDocumentCommand{\gooddefense}{m}
{
    \plus \imath{(#1*5/4)}
}

\DeclareDocumentCommand{\avgdefense}{m}
{
    \plus \imath{(#1*1)}
}

\DeclareDocumentCommand{\poordefense}{m}
{
    \plus \imath{(#1*3/4)}
}

\DeclareDocumentCommand{\babold}{m}
{
    \imathparse{int(floor((#1-1)/5+1))}

    \ifcase\imathresult \relax \or
        \plus #1 % case 0
    \or
        \plus #1 / \plus \imath{#1-5} % case 1
    \or
        \plus #1 / \plus \imath{#1-5} / \plus \imath{#1-10} % case 2
    \or
        \plus #1 / \plus \imath{#1-5} / \plus \imath{#1-10} / \plus \imath{#1-15} % case 3
    \or
        \plus #1 / \plus \imath{#1-5} / \plus \imath{#1-10} / \plus \imath{#1-15} / \plus \imath{#1-20} % case 3
    \fi
}

\DeclareDocumentCommand{\bab}{m o}
{
    \IfValueTF{#2}
    {
        \imathparse{int(floor((#1-1)/#2+1))}
    }
    {
        \imathparse{int(floor((#1-1)/5+1))}
    }

    \ifcase\imathresult
        {}#1 % case -1
    \or
        {}#1 % case 0
    \or
        {}#1~{}~(\mult2) % case 1
    \or
        {}#1~{}~(\mult3) % case 2
    \or
        {}#1~{}~(\mult4) % case 3
    \or
        {}#1~{}~(\mult5) % case 3
    \fi
}

\DeclareDocumentCommand{\goodbab}{m}
{
    \pgfmathparse{int(floor(#1+2))}
    \bab{\pgfmathresult}
}

\DeclareDocumentCommand{\avgbab}{m}
{
    \pgfmathparse{int(floor((#1*4/5+2)))}
    \bab{\pgfmathresult}
}

\DeclareDocumentCommand{\poorbab}{m}
{
    \pgfmathparse{int(floor((#1*2/3+1)))}
    \bab{\pgfmathresult}
}

\DeclareDocumentCommand{\alldefenseprogressionrow}{m}
{
    \nth{#1} & \gooddefense{#1} & \avgdefense{#1} & \poordefense{#1}
}

\DeclareDocumentCommand{\allbabprogressionrow}{m}
{
    \nth{#1} & \goodbab{#1} & \avgbab{#1} & \poorbab{#1}
}

\DeclareDocumentCommand{\barbarianprogressionrow}{m}
{
    \nth{#1} & \goodbab{#1}
}

\DeclareDocumentCommand{\clericprogressionrow}{m}
{
    \nth{#1} & \avgbab{#1}
}

\DeclareDocumentCommand{\druidprogressionrow}{m}
{
    \nth{#1} & \avgbab{#1}
}

\DeclareDocumentCommand{\halfdragonprogressionrow}{m}
{
    \nth{#1} & \goodbab{#1}
}

\DeclareDocumentCommand{\monkprogressionrow}{m}
{
    \nth{#1} & \goodbab{#1}
}

\DeclareDocumentCommand{\paladinprogressionrow}{m}
{
    \nth{#1} & \goodbab{#1}
}

\DeclareDocumentCommand{\rangerprogressionrow}{m}
{
    \nth{#1} & \goodbab{#1}
}

\DeclareDocumentCommand{\rogueprogressionrow}{m}
{
    \nth{#1} & \avgbab{#1}
}

\DeclareDocumentCommand{\spellwarpedprogressionrow}{m}
{
    \nth{#1} & \goodbab{#1}
}

\DeclareDocumentCommand{\mageprogressionrow}{m}
{
    \nth{#1} & \poorbab{#1}
}

\DeclareDocumentCommand{\fighterprogressionrow}{m}
{
    \nth{#1} & \goodbab{#1}
}

\DeclareDocumentCommand{\spelldamage}{m o}
{
    \IfValueTF{#2}{#2}{1d8}~
    \ifstr{#1}{#1~}{}
    damage~\plus1d~per~two~spellpower.
    % or alternately: standard damage based on your spellpower \add 2, etc.
}

\DeclareDocumentCommand{\spelldamageemp}{m o}
{
    \ifstr{#1}{}
    {
        \IfValueTF{#2}
        {1#2~damage~per~spellpower}
        {1d8~damage~per~spellpower}
    }
    {
        \IfValueTF{#2}
        {1#2~#1~damage~per~spellpower}
        {1d8~#1~damage~per~spellpower}
    }
}

\DeclareDocumentCommand{\spellquicktargeting}{s m m}
{
    \IfBooleanTF{#1}
    {
        \spelltwocol{\spelltgts{#2}}{\spellrng{#3}}
    }
    {
        \spelltwocol{\spelltgt{#2}}{\spellrng{#3}}
    }
}

\DeclareDocumentCommand{\spelltabular}{}
{
    \par\noindent
    \begin{tabularx}{\columnwidth}{>{\lcol}X l}
}

\DeclareDocumentCommand{\spelltabularcompressed}{}
{
    \renewcommand{\arraystretch}{0.5}
    \par\noindent
    \begin{tabularx}{\linewidth}{@{} >{\lcol}X l @{}}
}

\DeclareDocumentCommand{\spellmiscast}{m}
{
    \par\noindent\textit{#1~Miscast}:\xspace
}

% args:
%   class name
%   level, if any
%   ability name
%   ability type (Ex, Su, etc.), if any
\DeclareDocumentCommand{\cf}{s m o m o}
{
    \IfValueTF{#3}
    {
        \IfValueTF{#5}
        { \subsubsection{\nth{#3}~--~#4~(#5)} }
        { \subsubsection{\nth{#3}~--~#4} }
    }
    {
        \IfValueTF{#5}
        { \subsubsection{#4~(#5)} }
        { \subsubsection{#4} }
    }
    \IfBooleanF{#1}
    {
        \label{#2:#4}
    }
}

% feat level
% args:
%   level
%   name
%   ability type, if any
\DeclareDocumentCommand{\ff}{o m o}
{
    \IfValueTF{#1}
    {
        \IfValueTF{#3}
        { \nth{#1}~--~\textbf{#2}~(#3): }
        { \nth{#1}~--~\textbf{#2}: }
    }
    {
        \IfValueTF{#3}
        { \parhead{#2~(#3)} }
        { \parhead{#2} }
    }
}

\DeclareDocumentCommand{\itemhead}{s m}
{
    \item
    \IfValueTF{#1}
    {
        \textit{#2}
    }
    {
        \textbf{#2}
    }
}

\DeclareDocumentCommand{\altcf}{o m o}
{
    \cf*{}[#1]{#2}[#3]
}

\DeclareDocumentCommand{\featref}{s m}
{
    \hyperlink{feat:#2}{#2}
    \IfBooleanF{#1}
    {
        \hypertarget{ft:#2}{}
    }
}

\DeclareDocumentCommand{\mitemref}{s m}
{
    \IfBooleanF{#1}
    {
        \hypertarget{it:#2}{}
    }
    \hyperlink{item:#2}{#2}
}

\DeclareDocumentCommand{\term}{m}
{
    \textbf{#1}
}

% args:
%   #1: reference name of term, if different from display name
%   #2: display name of term
\DeclareDocumentCommand{\glossterm}{o m}
{
    \IfValueTF{#1}
    {
        % Enable this to test glossary links
        % \pref{#1}
        % TODO: remove \mbox before making major releases;
        % it fixes bizarre page break errors but makes text uglier
        \hyperlink{gloss:#1}{\textbf{\mbox{#2}}}
    }
    {
        % Enable this to test glossary links
        % \pref{#2}
        \hyperlink{gloss:#2}{\textbf{\mbox{#2}}}
    }
}

% args:
%   #1: name of term
%   #2: alternate name of term, only used for labels/targets
\DeclareDocumentCommand{\glossdef}{m o}
{
    \IfValueT{#2}
    {
        % Enable this to test glossary links
        % \label{#2}
        \hypertarget{gloss:#2}{}
    }
    % Enable this to test glossary links
    % \label{#1}
    \hypertarget{gloss:#1}{}
    \parhead{#1}
}

\DeclareDocumentCommand{\pcref}{o m}
{
    \IfValueTF{#1}
    {
        #2,~page~\pageref{#1}
    }
    {
        #2,~page~\pageref{#2}
    }
}

% args:
%   item price
\DeclareDocumentCommand{\cheapitemlevel}{m}
{
    % construct a statement that returns 0-20, incrementing by 1 for each
    % category of item level the item falls into
    \pgfmathparse{
        #1 <= 10 ? 0 : (
            #1 <= 50 ? 1 : (
                #1 <= 100 ? 2 : (
                    #1 <= 250 ? 3 : (
                        #1 <= 500 ? 4 : (
                            #1 <= 800 ? 5 : (
                                #1 <= 1200 ? 6 : (
                                    #1 <= 1800 ? 7 : (
                                        #1 <= 2750 ? 8 : (
                                            #1 <= 4000 ? 9 : (
                                                #1 <= 6500 ? 10 : (
                                                    #1 <= 10000 ? 11 : 12
                                                )
                                            )
                                        )
                                    )
                                )
                            )
                        )
                    )
                )
            )
        )
    }
    \ifstr{\pgfmathresult}{0}{1/2}[\nth{\pgfmathresult}]
}

\DeclareDocumentCommand{\magicitemexpensiveitemlevel}{m}
{
    \pgfmathparse{
        #1 <= 10 ? 11 : (
            #1 <= 16 ? 12 : (
                #1 <= 25 ? 13 : (
                    #1 <= 37 ? 14 : (
                        #1 <= 55 ? 15 : (
                            #1 <= 85 ? 16 : (
                                #1 <= 125 ? 17 : (
                                    #1 <= 190 ? 18 : (
                                        #1 <= 280 ? 19 : (
                                            #1 <= 400 ? 20 : 21
                                        )
                                    )
                                )
                            )
                        )
                    )
                )
            )
        )
    }
    \nth{\pgfmathresult}
}

% args:
%   effective spell level
\DeclareDocumentCommand{\magicitemaurastrength}{m}
{
    \pgfmathparse{
        #1 > 9 ? "Overwhelming" : (
            #1 > 6 ? "Strong" : (
                #1 > 3 ? "Moderate" : (
                    #1 >= 0 ? "Faint" : 0
                )
            )
        )
    }
    \pgfmathresult
}

% args:
%   star: if true, return spellpower with ``nth''
%   effective spell level
\DeclareDocumentCommand{\magicitemspellpower}{s m}
{
    \imathparse{max(#2*2,1)}
    \IfBooleanTF{#1}
    {
        \nth{\imathresult}
    }
    {
        \imathresult
    }
}

\DeclareDocumentCommand{\magicitemeffect}{o}
{
    \spelleffect[#1]
}

\DeclareDocumentCommand{\magicitemability}{m o}
{
    \fieldhead{#1~Ability}[#2]
}

\DeclareDocumentCommand{\spellaugment}{m m m o}
{
    \par\textbf{#1}~--~\textit{#2}:~#3
    \IfValueT{#4}{~This~is~a~#4~effect.}
}

\DeclareDocumentCommand{\augment}{m m}
{
    \vspace{0.5em}
    \par\noindent (\plus #1)~\textbf{#2}:\xspace
}

\DeclareDocumentCommand{\labeltext}{m}
{
    #1\label{#1}
}

\DeclareDocumentCommand{\hit}{}
{
    \subparhead{Hit}\xspace
}

\DeclareDocumentCommand{\crit}{}
{
    \subparhead{Critical~Hit}\xspace
}

\DeclareDocumentCommand{\miss}{}
{
    \subparhead{Miss}\xspace
}

\ExplSyntaxOff

\newcommand{\miscastexplode}{\spellmiscast{Explosive} The spell does not have its normal effect. Instead, the magical energy explodes, dealing 1d6 damage per two spellpower to you and all creatures adjacent to you.}
\newcommand{\miscastrandom}{\spellmiscast{Retargeting} The spell targets a random valid target within range (including yourself, if applicable).}
\newcommand{\miscastyou}{\spellmiscast{Localized} The spell targets you, or originates from your location, instead of its intended location. In addition, it only extends out 5 feet from your location, rather than its normal area.}

\newcommand{\classbasics}[1]{\par\noindent\textbf{#1}:}

\newcommand{\meh}[1]{\hypertarget{suspicious}{}#1}

\newcommand{\pari}{\par\noindent}

% for feat paths
\newcommand{\itempath}{\item\addtocounter{enumi}{1}\xspace}

\newcommand{\magical}{[\glossterm{Magical}]\xspace}
