%\newcommand{\Variable}{0}
%\newcommand{\storevalue}[1]{\renewcommand{\Variable}{#1}}

\newcommand{\lcaption}[1]{\caption{#1}\label{cap:#1}}
\newcommand{\lsec}[1]{\MakeUppercase{#1}\label{#1}}
\newcommand{\lssec}[1]{\MakeUppercase{#1}\label{#1}}
\newcommand{\lsssec}[1]{#1\label{#1}}

\newcommand{\ssecfake}[1]{\par \vspace{1em}{\Large\bfseries\raggedright #1} \vspace{1em} \par}
\newcommand{\sssecfake}[1]{\par \vspace{0.5em}{}{\large\raggedright #1} \vspace{0.5em} \par}
\newcommand{\sssecfakehref}[2]{\par \vspace{0.5em}{}{\large\raggedright \hyperlink{#1}{#2}} \vspace{0.5em} \par}
\newcommand{\pref}[1]{page \pageref{#1}}
\newcommand{\pdref}[1]{#1 (page \pageref{#1})}
\newcommand{\tref}[1]{Table \ref{cap:#1}: #1 (\pref{cap:#1})}
\newcommand{\trefnp}[1]{Table \ref{cap:#1}: #1}
\newcommand{\trefcp}[1]{Table \ref{cap:#1}: #1, \pref{cap:#1}}
\newcommand{\trefnum}[1]{Table \ref{cap:#1}}

\newcommand{\minus}{\texttt{-{}}}
\newcommand{\plus}{\texttt{+}}    %Use a smaller positive marker for use directly adjacent to numbers
\newcommand{\add}{\texttt{+}\xspace}    %Use a smaller plus sign
\newcommand{\sub}{\texttt{-}\xspace}
\newcommand{\mult}{x}
\newcommand{\mtimes}{x }
\newcommand{\x}{x\xspace}
\newcommand{\ki}{\emph{ki}\xspace}
\newcommand{\Ki}{\emph{Ki}\xspace}
%NOTED BUG: Spell names with mixed capitalization (Dispel magic) will not work properly.
\newcommand{\spell}[1]{\emph{\mbox{\hyperlink{spell:#1}{#1}}}}    %Italicize spells
\newcommand{\spellindirect}[2]{\emph{\hyperlink{spell:#1}{#2}}}
\newcommand{\ritual}[1]{\emph{#1}}    %Italicize spells
\newcommand{\cd}[1]{\par \textbf{#1}:}    %Bold class fluff description headers
\newcommand{\subcf}[1]{\parhead{#1}}    %Bold sub-class-feature headers
\newcommand{\subsk}[1]{\par \emph{#1}:}    %Italicize sub-skill headers

\newcommand{\subparhead}[1]{\par\emph{#1}:}
%\newcommand{\thead}[1]{\textbf{#1}}    %Bold the header of tables
\newcommand{\tb}[1]{\textbf{#1}}    %Bold the header of tables
\newcommand{\tdash}{\hskip 0.1em---\xspace}    %A null entry in a table
\newcommand{\ccol}{\centering\arraybackslash}    %Center columns
\newcommand{\lcol}{\raggedright\arraybackslash}
\newcommand{\rcol}{\raggedleft\arraybackslash}
\newcommand{\booktitle}[1]{\emph{#1}}
%\newcommand{\footnotetemp}[1]{\tiny{[#1]}}    %I don't actually know how to make footnotes yet, so placeholder
%\newcommand{\footnotetemp}[1]{\textsuperscript{#1}}
\newcommand{\fn}[1]{\textsuperscript{#1}}
\newcommand{\M}{\textsuperscript{M}}
\newcommand{\F}{\textsuperscript{F}}
\newcommand{\mfx}[1]{\emph{#1}}    % For material components, focuses, and XP costs for spells
\newcommand{\subspell}[1]{\emph{#1}}    %For parts of spells
\newcommand{\magicitem}[1]{\emph{#1}}    %For the names of items
\newcommand{\mitem}[1]{\emph{#1}}
\newcommand{\shead}[1]{\> \textbf{#1}}    %For the headers of spells in spell lists
%\newcommand{\spellhead}[2][]{\item[#1] \textbf{\hyperlink{spell:#2}{#2}:}}    %Spell in a spell list
%\newcommand{\spellheadc}[3][]{\item[#1] \textbf{\hyperlink{spell:#3 #2}{#2, #3}:}}
%\newcommand{\spellheadrestricted}[1]{\item \textit{\textbf{\hyperlink{spell:#1}{#1}:}}}  %Restricted spell in a spell list
%\newcommand{\spellheadrestrictedc}[2]{\item \textit{\textbf{\hyperlink{spell:#2 #1}{#1, #2}:}}}  %Restricted spell in a spell list
\newcommand{\spellsectionderp}[2]{\vspace{1em}\setlength\multicolsep{0pt}\begin{multicols}{2}
    \lowercase{\hypertarget{spell:#1}{}}\hypertarget{spell:#1}{\subsubsection{#1}}\columnbreak
    \begin{flushright}\large\textbf{\nth{#2} Level}\end{flushright}\end{multicols}
}
\newcommand{\spellsectioncomma}[3]{\vspace{1em}\setlength\multicolsep{0pt}\begin{multicols}{2}
    \lowercase{\hypertarget{spell:#2 #1}{}}\hypertarget{spell:#2 #1}{\subsubsection{#1, #2}}\columnbreak
    \begin{flushright}\large\textbf{\nth{#3} Level}\end{flushright}\end{multicols}
}
%\newcommand{\invocationsection}[1]{\lowercase{\hypertarget{spell:#1}{}}\hypertarget{spell:#1}{\subsubsection{#1}}}
%\newcommand{\invocationsectioncomma}[2]{\lowercase{\hypertarget{spell:#2 #1}{}}\hypertarget{spell:#2 #1}{\subsubsection{#1, #2}}}

\newcommand{\tind}{\hspace{1em}}

\newcommand{\appdescription}[7]{\subsubsection{#1} \parhead{Price (Level)} #2 gp (#3) \parhead{Body Location} #4 \parhead{Aura, Spellpower} #5, #6 \parhead{Activation} #7 \par}
\newcommand{\appdescriptiondc}[8]{\appdescription{#1}{#2}{#3}{#4}{#5}{#6}{#7} \parhead{Special Attack (Attack Bonus)} #8 \par}
\newcommand{\impdescription}[6]{\subsubsection{#1} \parhead{Price (Level)} #2 gp (#3) \parhead{Aura, Spellpower} #4, #5 \parhead{Activation} #6 \par}
\newcommand{\impdescriptionnoact}[5]{\subsubsection{#1} \parhead{Price (Level)} #2 gp (#3) \parhead{Aura, Spellpower} #4, #5 \par}
\newcommand{\impdescriptiondc}[7]{\impdescription{#1}{#2}{#3}{#4}{#5}{#6} \parhead{Special Attack (Attack Bonus)} #7 \par}
\newcommand{\impdescriptiondcnoact}[6]{\impdescription{#1}{#2}{#3}{#4}{#5} \parhead{Special Attack (Attack Bonus)} #6 \par}
%Spell creation requirements, without descriptors
%School, subschool, spell level, Spellpower, craft skill
\newcommand{\mitemreq}[5]{\par \textit{Creation Requirements:} #1 (#2) #3; spellpower #4 or #5}
\newcommand{\mitemreqdesc}[6]{\par \textit{Creation Requirements:} #1 (#2) [#3] #4; spellpower #5 or #6}
%Legacy
\newcommand{\tooldescription}[6]{\subsubsection{#1} \parhead{Price (Level)} #2 gp (#3) \parhead{Spellpower} #4 \parhead{Aura} #5 \parhead{Activation} #6 \par}
%one shot items should not have scaling DRs
\newcommand{\tooldescriptiondcos}[7]{\tooldescription{#1}{#2}{#3}{#4}{#5}{#6} \parhead{Special Attack (Attack Bonus)} #7\par}
\newcommand{\tooldescriptiondc}[7]{\tooldescription{#1}{#2}{#3}{#4}{#5}{#6} \parhead{Special Attack (Attack Bonus)} #7\par}

\newcommand{\prereq}[1]{\subparhead{Prerequisites:} #1}
\newcommand{\bpl}[3]{(\plus #1) #2 gp (#3)}
\newcommand{\bcl}[2]{(\plus #1) #2}

%Monster commands for format consistency
%Starting with \\ instead of \par yields an indent
\newcommand{\montypes}[5]{\begin{mstatblock}\par #1 #2 #3 \hfill \textbf{CR} #4 [#5]}
\newcommand{\montypessubtypes}[6]{\begin{mstatblock}\par #1 #2 #3 (#4) \hfill \textbf{CR} #5 [#6]}
\newcommand{\monsenses}[2]{\par \textbf{Init} #1; Perception #2}
\newcommand{\monsensesspellcraft}[3]{\par \textbf{Init} #1; Perception #2, Spellcraft #3}
\newcommand{\monsensesspecial}[3]{\par \textbf{Init} #1; Perception #2; \textbf{Senses} #3}
\newcommand{\monsensesfull}[4]{\par \textbf{Init} #1; Perception #2, Spellcraft #3; \textbf{Senses} #4}
\newcommand{\monaura}[1]{\par \textbf{Aura} #1}
\newcommand{\monlanguages}[1]{\par \textbf{Languages} #1}
%\newcommand{\monac}[5]{\par \textbf{AC} #1, touch #2, flat-footed #3; \textbf{CMD} #4 \\ (#5)}
%\newcommand{\monacspecial}[6]{\par \textbf{AC} #1, touch #2, flat-footed #3; \textbf{CMD} #4; #5 \\ (#6)}
\newcommand{\monhp}[2]{\par \textbf{HP} #1 (#2 HV)}
\newcommand{\mondr}[1]{\textbf{DR} #1}
\newcommand{\monhpdr}[3]{\mhp{#1}{#2}; \mdr{#3}}
\newcommand{\monimmune}[1]{\par \textbf{Immune} #1}
\newcommand{\monresist}[1]{\par \textbf{Resist} #1}
\newcommand{\monsr}[1]{\textbf{SR} #1}
\newcommand{\monresistsr}[2]{\mres{#1}; \msr{#2}}
\newcommand{\monsaves}[3]{\par \textbf{Fort} \plus #1, \textbf{Ref} \plus #2, \textbf{Will} \plus #3}
\newcommand{\monsavesnegative}[3]{\par \textbf{Fort} #1, \textbf{Ref} #2, \textbf{Will} #3}
\newcommand{\monspeed}[1]{\textbf{Speed} #1}
\newcommand{\monmelee}[1]{\par \textbf{Melee} #1}
\newcommand{\monrange}[1]{\par \textbf{Ranged} #1}
\newcommand{\monspace}[2]{\par \textbf{Space} #1 ft.; \textbf{Reach} #2 ft.}
\newcommand{\monspacenofeet}[2]{\par \textbf{Space} #1; \textbf{Reach} #2}
\newcommand{\moncmb}[2]{\par \textbf{BAB} \plus #1; \textbf{CMB} \plus #2}
\newcommand{\moncmbnegative}[2]{\par \textbf{BAB} \plus #1; \textbf{CMB} #2}
\newcommand{\monspecialact}[1]{\par \textbf{Special Actions} #1}
\newcommand{\monsla}[5]{\par \textbf{Spell-Like Abilities (Sp)} (CL #1, DR #2, #3/day, #4--based): \\ #5}
\newcommand{\monpsi}[5]{\par \textbf{Psionics (Sp)} (CL #1, DR #2, #3/day, #4--based): \\ #5}
\newcommand{\monslamore}[1]{\\ #1}
\newcommand{\monattributes}[6]{\par \textbf{Attributes} Str #1, Dex #2, Con #3, Int #4, Per #5, Wil #6}
\newcommand{\monsa}[1]{\par \textbf{SA} #1}
\newcommand{\monsq}[1]{\par \textbf{SQ} #1}
\newcommand{\monfeats}[1]{\par \textbf{Feats} #1}
\newcommand{\monskills}[1]{\par \textbf{Skills} #1}
\newcommand{\monitems}[1]{\par \textbf{Items} #1}
\newcommand{\monability}[2]{\par \textbf{#1} #2}
%http://tex.stackexchange.com/questions/128636/center-hrule-in-the-middle-of-the-page
\newcommand{\monlinerule}{\par\noindent\hfil\rule[0.5em]{0.9\columnwidth}{0.5pt}\hfil\par}
\newcommand{\mondescription}[1]{\par \end{mstatblock}\sssecfake{Description}\label{#1:Description}}
\newcommand{\mondescriptionnoblock}[1]{\par \sssecfake{Description}\label{#1:Description}}
\newcommand{\monbehavior}[1]{\sssecfake{Combat}\label{#1:Behavior}}

\newcommand{\wilditem}{\addtocounter{enumi}{1} \item}
\newcommand{\wilditemplus}{\addtocounter{enumi}{3} \item}

\newcommand{\spelldesc}[1]{\par\noindent \textit{#1}}
\newcommand{\spelldmg}[1]{\fieldhead{Damage} #1}
\newcommand{\spellheal}[1]{\fieldhead{Healing} #1}
\newcommand{\spellatk}[1]{\fieldhead{Attack} #1}
\newcommand{\spelltgr}[1]{\fieldhead{Trigger} #1}

%\newcommand{\spellfocus}[1]{\par\noindent \textit{Focus:} #1}
\newcommand{\spellmat}[1]{\par\noindent \textit{Material Components:} #1}

\newcommand{\spelltwocol}[2]{\spelltabularcompressed #1 & #2\end{tabularx}}

\newcommand{\spellschool}[1]{\noindent#1}
\newcommand{\spelllists}[1]{\fieldhead{Lists} #1}
\newcommand{\spellcmp}[1]{\fieldhead{Components} #1}
\newcommand{\spelltime}[1]{\fieldhead{Casting Time} #1}
\newcommand{\spellrng}[1]{\fieldhead{Range} #1}

\newcommand{\spellarea}[1]{\fieldhead{Area} #1}
\newcommand{\spellburst}[1]{\fieldhead{Burst} #1}
\newcommand{\spellemanation}[1]{\fieldhead{Emanation} #1}
\newcommand{\spelllimit}[1]{\fieldhead{Limit} #1}
\newcommand{\spellzone}[1]{\fieldhead{Zone} #1}
\newcommand{\spellzoneandburst}[1]{\fieldhead{Zone and Burst} #1}

%\newcommand{\spelltgt}[1]{\fieldhead{Target} #1}
%\newcommand{\spelltgts}[1]{\fieldhead{Targets} #1}
\newcommand{\spelltgtortgts}[1]{\fieldhead{Target or Targets} #1}
\newcommand{\spelltgtorarea}[1]{\fieldhead{Target or Area} #1}
\newcommand{\spelltgteffarea}[1]{\fieldhead{Target or Area} #1}

\newcommand{\spelldur}{\fieldhead{Duration}}
\newcommand{\spelltgrdur}[1]{\fieldhead{Trigger Duration} #1}
\newcommand{\spellfocus}[1]{\fieldhead{Focus} #1}
\newcommand{\spellsr}[1]{\fieldhead{Spell Resistance} #1}

\newcommand{\rngpers}{Personal}
\newcommand{\rngtouch}{Touch}
\newcommand{\rngclose}{Close \reminder{30 ft.}\xspace}
\newcommand{\rngmed}{Medium \reminder{100 ft.}\xspace}
\newcommand{\rnglong}{Long \reminder{300 ft.}\xspace}
\newcommand{\rngfar}{Long \reminder{300 ft.}\xspace}
\newcommand{\rngext}{Extreme \reminder{1,000 ft.}\xspace}
\newcommand{\areasmall}{Small \reminder{10 ft.}\xspace}
\newcommand{\areamed}{Medium \reminder{20 ft.}\xspace}
\newcommand{\arealarge}{Large \reminder{50 ft.}\xspace}
\newcommand{\areahuge}{Huge \reminder{100 ft.}\xspace}
\newcommand{\durbrief}{Brief \reminder{2 rounds}\xspace}
\newcommand{\durshort}{Short \reminder{Focus \add 5 rounds}\xspace}
\newcommand{\durmed}{Medium \reminder{5 minutes}\xspace}
\newcommand{\durlong}{Long \reminder{1 hour}\xspace}
\newcommand{\durext}{Extreme \reminder{12 hours}\xspace}
\newcommand{\durperm}{Permanent\xspace}
\newcommand{\durpersonallong}{\durshort}

\newcommand{\confusionexplanation}{A confused creature cannot take actions normally. If it is attacked, it automatically attacks a random attacker. Otherwise, at the beginning of each round, it randomly decides to take one of four actions that round: babble incoherently, flee from the caster as if panicked, attack the nearest creature, or act normally. A confused character who can't carry out the indicated action does nothing but babble incoherently.\xspace}

\newcommand{\bonusscalingdescription}{This bonus increases to \plus3 at 8th level, to \plus4 at 14th level, and finally to \plus5 at 20th level.\xspace}
\newcommand{\spellbonusscalingdescription}{This bonus increases to \plus3 at spellpower 8, to \plus4 at spellpower 14, and finally to \plus5 at spellpower 20.\xspace}
\newcommand{\babscalingdescription}{This bonus increases to \plus3 at combat prowess 8, to \plus4 at combat prowess 12, and finally to \plus5 at combat prowess 16.\xspace}

\newcommand{\feat}[2]{\hypertarget{feat:#1}{\sssecfakehref{ft:#1}{#1 [#2]}}\label{feat:#1}}
\newcommand{\featpre}{\parhead{Prerequisite} }
\newcommand{\featpres}{\parhead{Prerequisites} }
\newcommand{\featben}{\parhead{Benefit} }
\newcommand{\featspecial}{\parhead{Special} }
%featref no target
\newcommand{\featrefnt}[1]{\hyperlink{feat:#1}{#1}}
\newcommand{\featpref}[1]{\pageref{feat:#1}}
\newcommand{\mitempref}[1]{\pageref{item:#1}}
\newcommand{\featpcref}[1]{#1, page \pageref{feat:#1}}
\newcommand{\featpdref}[1]{#1 (page \pageref{feat:#1})}
%\newcommand{\featref}[1]{}
\newcommand{\stylereq}{\parhead{Style Requirement}}

\newcommand{\featbanenotes}{You can only apply the benefits of a single Bane feat against a particular foe.}

\newcommand{\skill}[2]{\section{#1 (#2)}\label{#1}}

\newcommand{\reminder}[1]{\textcolor{darkgray}{\textit{(#1)}}}

%\newcommand{\vulnerableexplanation}{A vulnerable creature takes a \minus2 penalty to attacks, defenses, and checks.\xspace}
%\newcommand{\vulneffect}{\minus2 to attacks, defenses, and checks\xspace}

\newcommand{\blinded}{\glossterm{blinded}\xspace}
\newcommand{\bloodied}{\glossterm{bloodied}\xspace}
\newcommand{\charmed}{\glossterm{charmed}\xspace}
\newcommand{\confused}{\glossterm{confused}\xspace}
\newcommand{\defenseless}{\glossterm{defenseless}\xspace}
\newcommand{\disoriented}{\glossterm{disoriented}\xspace}
\newcommand{\dominated}{\glossterm{dominated}\xspace}
\newcommand{\exhausted}{\glossterm{exhausted}\xspace}
\newcommand{\fascinated}{\glossterm{fascinated}\xspace}
\newcommand{\fatigued}{\glossterm{fatigued}\xspace}
\newcommand{\frightened}{\glossterm{frightened}\xspace}
\newcommand{\dazed}{\glossterm{dazed}\xspace}
\newcommand{\dazzled}{\glossterm{dazzled}\xspace}
\newcommand{\deafened}{\glossterm{deafened}\xspace}
\newcommand{\grappled}{\glossterm{grappled}\xspace}
\newcommand{\goaded}{\glossterm{goaded}\xspace}
\newcommand{\helpless}{\glossterm{helpless}\xspace}
\newcommand{\ignited}{\glossterm{ignited}\xspace}
\newcommand{\immobilized}{\glossterm{immobilized}\xspace}
\newcommand{\nauseated}{\glossterm{nauseated}\xspace}
\newcommand{\panicked}{\glossterm{panicked}\xspace}
\newcommand{\paralyzed}{\glossterm{paralyzed}\xspace}
\newcommand{\petrified}{\glossterm{petrified}\xspace}
\newcommand{\prone}{\glossterm{prone}\xspace}
\newcommand{\shaken}{\glossterm{shaken}\xspace}
\newcommand{\sickened}{\glossterm{sickened}\xspace}
\newcommand{\slowed}{\glossterm{slowed}\xspace}
\newcommand{\staggered}{\glossterm{staggered}\xspace}
\newcommand{\stunned}{\glossterm{stunned}\xspace}
\newcommand{\taunted}{\glossterm{taunted}\xspace}
\newcommand{\wounded}{\glossterm{wounded}\xspace}
\newcommand{\unaware}{\glossterm{unaware}\xspace}

\newcommand{\difficultterrain}{\glossterm{difficult terrain}\xspace}

\newcommand{\concealment}{concealment\xspace}

\newcommand{\destructivespellnotes}{If a destructive spell deals enough damage to an interposing barrier to shatter or breaks through it, its effects may continue beyond the barrier if the area permits; otherwise, it stops at the barrier just as any other spell effect does.\xspace}
\newcommand{\cursespellnotes}{Curses cannot be dispelled with \spell{dispel magic}, but can be removed with \spell{break enchantment} or \spell{remove curse}.\xspace}
\newcommand{\firespellnotes}{Fire spells shed light equivalent to a torch. They do not function underwater. They can ignite combustable materials, such as dry wood and cloth.\xspace}
\newcommand{\forcespellnotes}{Force spells also affect the Ethereal Plane.}
\newcommand{\fogspellnotes}{Fog spells do not function underwater and can be dispersed by wind. Localized wind, such as from a \spell{gust of wind} spell, only disperses the fog where it overlaps the fog. A fire spell burns away the fog in the area into which it deals damage.\xspace}
\newcommand{\fogwindspellnotes}{A moderate wind (11\add mph) disperses the fog in 5 rounds; a strong wind (21\add mph) disperses the fog in 1 round.\xspace}
\newcommand{\sensorspellnotes}{A magic sensor is a floating, invisible sphere approximately one inch in diameter. It has 1 hit point and an Armor defense of 8. Its special defenses are the same as yours.\xspace}
\newcommand{\subtlespellnotes}{Subtle spells have no obvious effects. A creature affected by a subtle spell is usually unaware that it is under magical influence. The DR to identify a subtle spell with the Spellcraft skill is 10 higher than normal (see \pcref{Spellcraft}).}
\newcommand{\mobilesensorspellnotes}[1]{\sensorspellnotes The sensor has a #1 foot fly speed with perfect maneuverability. It is unable to enter another plane of existence, even through a \spell{gate} or similar magical portal.}
\newcommand{\norepeatspellnotes}{You can only affect any individual creature with this spell once per 24 hours.\xspace}
\newcommand{\norepeatnotes}{You can only affect any individual creature with this ability once per 24 hours.\xspace}
\newcommand{\physicalspellnotes}{Physical spells cannot be dismissed or dispelled, and do not allow spell resistance.}
\newcommand{\sizingspellnotes}{Multiple magical effects which change size do not stack. Size increases offset size decreases on a one-for-one basis.}

\newcommand{\spelltablecolumns}{>{\lcol}X l}
\newcommand{\spelllevelschool}[2]{\spelllvl{#1} & #2 \\}
\newcommand{\spelllevelnew}[3]{\tb{\nth{#1} level} #2 & #3 \\}

\newcommand{\dismissable}{}
\newcommand{\shapeable}{(S)}
\newcommand{\rngunrestricted}{(Unrestricted)}
\newcommand{\undispellable}{(Undispellable)}

% http://tex.stackexchange.com/questions/135617/how-to-i-reduce-space-before-and-after-hrule
\newcommand{\HRule}[2][\medskipamount]{\par
  \vspace*{\dimexpr-\parskip-\baselineskip+#1}
  \noindent\rule{\linewidth}{#2}\par
  \vspace*{\dimexpr-\parskip-.5\baselineskip+#1}}

\newcommand{\spellline}{\par\HRule[0.5\medskipamount]{0.2pt}}
% weird spacing before twocol
\newcommand{\spelllinespaced}{\spellline\vspace{0.5\medskipamount}}

%\newcommand{\spelleffect}{\fieldhead{Effect}}
%\newcommand{\spellsuccess}{\fieldhead{Success}}
%\newcommand{\spellfailure}{\fieldhead{Failure}\xspace}
\newcommand{\spellnotes}{\par\noindent\textit{Notes}:\xspace}
\newcommand{\spellspecial}{\fieldhead{Special}}
\newcommand{\spellcheck}[1]{\fieldhead{Check} #1}

\ExplSyntaxOn

\DeclareDocumentCommand{\Create}{m}
{
    \newcounter{#1}
    \setcounter{#1}{0}
}

\DeclareDocumentCommand{\Set}{m m}
{
    \setcounter{#1}{#2}
}

\DeclareDocumentCommand{\Get}{m}
{
    \arabic{#1}
}

\DeclareDocumentCommand{\GetValue}{m}
{
    \value{#1}
}

\newcommand{\TempVar}{0}
\newcommand{\TempVarTwo}{0}

\DeclareDocumentCommand{\parhead}{s m o}
{
    \vspace{0.5em}    
    \par \IfBooleanT{#1}{\noindent}
    \IfNoValueTF{#3}
    {\textbf{#2}:}
    {\textbf{#2}~[#3]:}\xspace
}

\DeclareDocumentCommand{\fieldhead}{m o}
{
    \par\noindent
    \IfNoValueTF{#2}
    {\textbf{#1}:}
    {\textbf{#1}~[#2]:}\xspace
}

\DeclareDocumentCommand{\spellfailure}{o}
{
    \IfNoValueTF{#1}
    {\fieldhead{Failure}}
    {\fieldhead{Failure}[#1]}
}
\DeclareDocumentCommand{\spellsuccess}{o}
{
    \IfNoValueTF{#1}
    {\fieldhead{Success}}
    {\fieldhead{Success}[#1]}
}
\DeclareDocumentCommand{\spellcritical}{o}
{
    \IfNoValueTF{#1}
    {\fieldhead{Critical~Success}}
    {\fieldhead{Critical~Success}[#1]}
}
\DeclareDocumentCommand{\spelleffect}{o}
{
    \IfNoValueTF{#1}
    {\fieldhead{Effect}}
    {\fieldhead{Effect}[#1]}
}
\DeclareDocumentCommand{\spelltgt}{m o}
{
    \IfNoValueTF{#2}
    {\fieldhead{Target} #1}
    {\fieldhead{#2~Target} #1}
}
\DeclareDocumentCommand{\spelltgts}{m o}
{
    \IfNoValueTF{#2}
    {\fieldhead{Targets} #1}
    {\fieldhead{#2~Targets} #1}
}

\DeclareDocumentCommand{\ritualcostparser}{m}
{
    \ifcase#1\relax
    \or 5~gp    %1
    \or 20~gp   %2
    \or 50~gp   %3
    \or 125~gp  %4
    \or 300~gp  %5
    \or 750~gp  %6
    \or 1,500~gp %7
    \or 3,000~gp %8
    \or 7,500~gp %9
    \else???
    \fi
}

\DeclareDocumentCommand{\ritualcost}{s m o}
{
    \IfBooleanF{#1}{\spellline}
    \spellmat{\ritualcostparser#2\xspace
        \IfValueTF{#3}
        {#3}
        {in~ritual~components.}
    }
}

% args:
%   magic item price or type of price
%   effective spell level
\DeclareDocumentCommand{\mitempricewithlevel}{m m}
{
    \IfSubStr{#1}{Passive}
    {
        %\renewcommand{\TempVarTwo}{\magicitempricepassive{#2}}
        \magicitempricepassive{#2}
    }
    {
        \IfSubStr{#1}{Active}
        {
            \magicitempriceactive{#2}
        }
        {
            \renewcommand{\TempVarTwo}{#1}
        }
    }

    \StrLen{\TempVarTwo}[\TempVar]
    \pgfmathparse{\TempVar > 4}
    \ifcase\pgfmathresult
        % handle normal magic items
        \fieldhead{Price~(Level)} \num{\TempVarTwo}~gp~(\cheapitemlevel{\TempVarTwo})
    \or
        % handle very expensive magic items
        % by passing the number with the right three digits removed
        \StrGobbleRight{\TempVarTwo}{3}[\TempVar]
        \fieldhead{Price~(Level)} \num{\TempVarTwo}~gp~(
            \magicitemexpensiveitemlevel{\TempVar}
        )
    \fi
}

\DeclareDocumentCommand{\mitempricetable}{m m}
{
    \IfSubStr{#1}{Passive}
    {
        %\renewcommand{\TempVarTwo}{\magicitempricepassive{#2}}
        \magicitempricepassive{#2}
    }
    {
        \IfSubStr{#1}{Active}
        {
            \magicitempriceactive{#2}
        }
        {
            \renewcommand{\TempVarTwo}{#1}
        }
    }
    \xdef\ItemCost{\num{\TempVarTwo}~gp}

    \StrLen{\TempVarTwo}[\TempVar]
    \pgfmathparse{\TempVar > 4}
    \ifcase\pgfmathresult
        % handle normal magic items
        \xdef\ItemLevel{\cheapitemlevel{\TempVarTwo}}
    \or
        % handle very expensive magic items
        % by passing the number with the right three digits removed
        \StrGobbleRight{\TempVarTwo}{3}[\TempVar]
        \xdef\ItemLevel{\magicitemexpensiveitemlevel{\TempVar}}
    \fi
    \ItemCost & \ItemLevel
}

% args:
%    spell level
\DeclareDocumentCommand{\magicitempriceactive}{m}
{
    \renewcommand{\TempVarTwo}{
        \ifcase#1~
        100    %0
        \or 200   %1
        \or 800   %2
        \or 2000  %3
        \or 5000  %4
        \or 12000  %5
        \or 30000 %6
        \or 60000 %7
        \or 140000 %8
        \or 300000 %9
        \else 0
        \fi
    }
}

% args:
%    spell level
\DeclareDocumentCommand{\magicitempricepassive}{m}
{
    \renewcommand{\TempVarTwo}{
        \ifcase#1~
        200   %0
        \or 800   %1
        \or 2000  %2
        \or 5000  %3
        \or 12000  %4
        \or 30000 %5
        \or 60000 %6
        \or 140000 %7
        \or 300000 %8
        \or 700000 %9
        \else 0
        \fi
    }
}

% If #1 = #2
%     print #3.
% Elseif #4 is provided
%     print #4
\DeclareDocumentCommand{\ifstr}{m m m o}
{
    \ifnum 0=\pdfstrcmp{#1}{#2}
        #3
    \else
        \IfValueT{#4}{#4}
    \fi
}

% args:
%   pre-spell header spell name, spell name modifier (Greater, etc.)
\DeclareDocumentCommand{\spellhead}{o m o}
{
    \IfValueTF{#1}
    {\item[#1]}
    {\item}
    \textbf{
        \IfValueTF{#3}
        {
            \hyperlink{spell:#3~#2}{#2,~#3}
            %\pageref{spell:#3~#2}
        }
        {
            \hyperlink{spell:#2}{#2}
            %\pageref{spell:#2}
        }
    }:
}

\DeclareDocumentCommand{\spellinfo}{m s m}
{
    \IfBooleanTF{#2}
    {
        \noindent#1
        \fieldhead{Lists} #3
    }
    {
        \spelltwocol{\fieldhead{Schools}~#1}{\fieldhead{Lists}~#3}
    }
}

\DeclareDocumentCommand{\spelltags}{m}
{
    \fieldhead{Tags} #1
}

\DeclareDocumentCommand{\imath}{m}
{
    %\pgfmathparse{int(floor(#1))}\pgfmathresult
    \pgfmathsetmacro{\TempVar}{int(floor(#1))}
    \TempVar
}

\DeclareDocumentCommand{\imathparse}{m}
{
    \pgfmathsetmacro{\TempVar}{int(floor(#1))}
}

\DeclareDocumentCommand{\imathresult}{}
{
    \TempVar
}

\DeclareDocumentCommand{\mathnth}{m}
{
    \pgfmathparse{int(floor(#1))}\nth{\pgfmathresult}
}

\DeclareDocumentCommand{\gooddefense}{m}
{
    \plus \imath{(#1*5/4)}
}

\DeclareDocumentCommand{\avgdefense}{m}
{
    \plus \imath{(#1*1)}
}

\DeclareDocumentCommand{\poordefense}{m}
{
    \plus \imath{(#1*3/4)}
}

\DeclareDocumentCommand{\babold}{m}
{
    \imathparse{int(floor((#1-1)/5+1))}

    \ifcase\imathresult \relax \or
        \plus #1 % case 0
    \or
        \plus #1 / \plus \imath{#1-5} % case 1
    \or
        \plus #1 / \plus \imath{#1-5} / \plus \imath{#1-10} % case 2
    \or
        \plus #1 / \plus \imath{#1-5} / \plus \imath{#1-10} / \plus \imath{#1-15} % case 3
    \or
        \plus #1 / \plus \imath{#1-5} / \plus \imath{#1-10} / \plus \imath{#1-15} / \plus \imath{#1-20} % case 3
    \fi
}

\DeclareDocumentCommand{\bab}{m o}
{
    \IfValueTF{#2}
    {
        \imathparse{int(floor((#1-1)/#2+1))}
    }
    {
        \imathparse{int(floor((#1-1)/5+1))}
    }

    \ifcase\imathresult
        {}#1 % case -1
    \or
        {}#1 % case 0
    \or
        {}#1~{}~(\mult2) % case 1
    \or
        {}#1~{}~(\mult3) % case 2
    \or
        {}#1~{}~(\mult4) % case 3
    \or
        {}#1~{}~(\mult5) % case 3
    \fi
}

\DeclareDocumentCommand{\goodbab}{m}
{
    \pgfmathparse{int(floor(#1+2))}
    \bab{\pgfmathresult}
}

\DeclareDocumentCommand{\avgbab}{m}
{
    \pgfmathparse{int(floor((#1*4/5+2)))}
    \bab{\pgfmathresult}
}

\DeclareDocumentCommand{\poorbab}{m}
{
    \pgfmathparse{int(floor((#1*2/3+1)))}
    \bab{\pgfmathresult}
}

\DeclareDocumentCommand{\alldefenseprogressionrow}{m}
{
    \nth{#1} & \gooddefense{#1} & \avgdefense{#1} & \poordefense{#1}
}

\DeclareDocumentCommand{\allbabprogressionrow}{m}
{
    \nth{#1} & \goodbab{#1} & \avgbab{#1} & \poorbab{#1}
}

\DeclareDocumentCommand{\barbarianprogressionrow}{m}
{
    \nth{#1} & \goodbab{#1}
}

\DeclareDocumentCommand{\clericprogressionrow}{m}
{
    \nth{#1} & \avgbab{#1}
}

\DeclareDocumentCommand{\druidprogressionrow}{m}
{
    \nth{#1} & \avgbab{#1}
}

\DeclareDocumentCommand{\halfdragonprogressionrow}{m}
{
    \nth{#1} & \goodbab{#1}
}

\DeclareDocumentCommand{\monkprogressionrow}{m}
{
    \nth{#1} & \goodbab{#1}
}

\DeclareDocumentCommand{\paladinprogressionrow}{m}
{
    \nth{#1} & \goodbab{#1}
}

\DeclareDocumentCommand{\rangerprogressionrow}{m}
{
    \nth{#1} & \goodbab{#1}
}

\DeclareDocumentCommand{\rogueprogressionrow}{m}
{
    \nth{#1} & \avgbab{#1}
}

\DeclareDocumentCommand{\spellwarpedprogressionrow}{m}
{
    \nth{#1} & \goodbab{#1}
}

\DeclareDocumentCommand{\mageprogressionrow}{m}
{
    \nth{#1} & \poorbab{#1}
}

\DeclareDocumentCommand{\fighterprogressionrow}{m}
{
    \nth{#1} & \goodbab{#1}
}

\DeclareDocumentCommand{\spelldamage}{m o}
{
    \IfValueTF{#2}{#2}{1d8}~
    \ifstr{#1}{#1~}{}
    damage~\plus1d~per~two~spellpower.
    % or alternately: standard damage based on your spellpower \add 2, etc.
}

\DeclareDocumentCommand{\spelldamageemp}{m o}
{
    \ifstr{#1}{}
    {
        \IfValueTF{#2}
        {1#2~damage~per~spellpower}
        {1d8~damage~per~spellpower}
    }
    {
        \IfValueTF{#2}
        {1#2~#1~damage~per~spellpower}
        {1d8~#1~damage~per~spellpower}
    }
}

\DeclareDocumentCommand{\spellquicktargeting}{s m m}
{
    \IfBooleanTF{#1}
    {
        \spelltwocol{\spelltgts{#2}}{\spellrng{#3}}
    }
    {
        \spelltwocol{\spelltgt{#2}}{\spellrng{#3}}
    }
}

\DeclareDocumentCommand{\spelltabular}{}
{
    \par\noindent
    \begin{tabularx}{\columnwidth}{>{\lcol}X l}
}

\DeclareDocumentCommand{\spelltabularcompressed}{}
{
    \renewcommand{\arraystretch}{0.5}
    \par\noindent
    \begin{tabularx}{\linewidth}{@{} >{\lcol}X l @{}}
}

\DeclareDocumentCommand{\spellmiscast}{m}
{
    \par\noindent\textit{#1~Miscast}:\xspace
}

% args:
%   class name
%   level, if any
%   ability name
%   ability type (Ex, Su, etc.), if any
\DeclareDocumentCommand{\cf}{s m o m o}
{
    \IfValueTF{#3}
    {
        \IfValueTF{#5}
        { \subsubsection{\nth{#3}~--~#4~(#5)} }
        { \subsubsection{\nth{#3}~--~#4} }
    }
    {
        \IfValueTF{#5}
        { \subsubsection{#4~(#5)} }
        { \subsubsection{#4} }
    }
    \IfBooleanF{#1}
    {
        \label{#2:#4}
    }
}

% feat level
% args:
%   level
%   name
%   ability type, if any
\DeclareDocumentCommand{\ff}{o m o}
{
    \IfValueTF{#1}
    {
        \IfValueTF{#3}
        { \nth{#1}~--~\textbf{#2}~(#3): }
        { \nth{#1}~--~\textbf{#2}: }
    }
    {
        \IfValueTF{#3}
        { \parhead{#2~(#3)} }
        { \parhead{#2} }
    }
}

\DeclareDocumentCommand{\itemhead}{s m}
{
    \item
    \IfValueTF{#1}
    {
        \textit{#2}
    }
    {
        \textbf{#2}
    }
}

\DeclareDocumentCommand{\altcf}{o m o}
{
    \cf*{}[#1]{#2}[#3]
}

\DeclareDocumentCommand{\featref}{s m}
{
    \hyperlink{feat:#2}{#2}
    \IfBooleanF{#1}
    {
        \hypertarget{ft:#2}{}
    }
}

\DeclareDocumentCommand{\mitemref}{s m}
{
    \IfBooleanF{#1}
    {
        \hypertarget{it:#2}{}
    }
    \hyperlink{item:#2}{#2}
}

\DeclareDocumentCommand{\term}{m}
{
    \textbf{#1}
}

% args:
%   #1: reference name of term, if different from display name
%   #2: display name of term
\DeclareDocumentCommand{\glossterm}{o m}
{
    \IfValueTF{#1}
    {
        % Enable this to test glossary links
        % \pref{#1}
        % TODO: remove \mbox before making major releases;
        % it fixes bizarre page break errors but makes text uglier
        \hyperlink{gloss:#1}{\textbf{\mbox{#2}}}
    }
    {
        % Enable this to test glossary links
        % \pref{#2}
        \hyperlink{gloss:#2}{\textbf{\mbox{#2}}}
    }
}

% args:
%   #1: name of term
%   #2: alternate name of term, only used for labels/targets
\DeclareDocumentCommand{\glossdef}{m o}
{
    \IfValueT{#2}
    {
        % Enable this to test glossary links
        % \label{#2}
        \hypertarget{gloss:#2}{}
    }
    % Enable this to test glossary links
    % \label{#1}
    \hypertarget{gloss:#1}{}
    \parhead{#1}
}

\DeclareDocumentCommand{\pcref}{o m}
{
    \IfValueTF{#1}
    {
        #2,~page~\pageref{#1}
    }
    {
        #2,~page~\pageref{#2}
    }
}

% args:
%   item price
\DeclareDocumentCommand{\cheapitemlevel}{m}
{
    % construct a statement that returns 0-20, incrementing by 1 for each
    % category of item level the item falls into
    \pgfmathparse{
        #1 <= 10 ? 0 : (
            #1 <= 50 ? 1 : (
                #1 <= 100 ? 2 : (
                    #1 <= 250 ? 3 : (
                        #1 <= 500 ? 4 : (
                            #1 <= 800 ? 5 : (
                                #1 <= 1200 ? 6 : (
                                    #1 <= 1800 ? 7 : (
                                        #1 <= 2750 ? 8 : (
                                            #1 <= 4000 ? 9 : (
                                                #1 <= 6500 ? 10 : (
                                                    #1 <= 10000 ? 11 : 12
                                                )
                                            )
                                        )
                                    )
                                )
                            )
                        )
                    )
                )
            )
        )
    }
    \ifstr{\pgfmathresult}{0}{1/2}[\nth{\pgfmathresult}]
}

\DeclareDocumentCommand{\magicitemexpensiveitemlevel}{m}
{
    \pgfmathparse{
        #1 <= 10 ? 11 : (
            #1 <= 16 ? 12 : (
                #1 <= 25 ? 13 : (
                    #1 <= 37 ? 14 : (
                        #1 <= 55 ? 15 : (
                            #1 <= 85 ? 16 : (
                                #1 <= 125 ? 17 : (
                                    #1 <= 190 ? 18 : (
                                        #1 <= 280 ? 19 : (
                                            #1 <= 400 ? 20 : 21
                                        )
                                    )
                                )
                            )
                        )
                    )
                )
            )
        )
    }
    \nth{\pgfmathresult}
}

% args:
%   effective spell level
\DeclareDocumentCommand{\magicitemaurastrength}{m}
{
    \pgfmathparse{
        #1 > 9 ? "Overwhelming" : (
            #1 > 6 ? "Strong" : (
                #1 > 3 ? "Moderate" : (
                    #1 >= 0 ? "Faint" : 0
                )
            )
        )
    }
    \pgfmathresult
}

% args:
%   star: if true, return spellpower with ``nth''
%   effective spell level
\DeclareDocumentCommand{\magicitemspellpower}{s m}
{
    \imathparse{max(#2*2,1)}
    \IfBooleanTF{#1}
    {
        \nth{\imathresult}
    }
    {
        \imathresult
    }
}

\DeclareDocumentCommand{\magicitemeffect}{o}
{
    \spelleffect[#1]
}

\DeclareDocumentCommand{\magicitemability}{m o}
{
    \fieldhead{#1~Ability}[#2]
}

\DeclareDocumentCommand{\spellaugment}{m m m o}
{
    \par\textbf{#1}~--~\textit{#2}:~#3
    \IfValueT{#4}{~This~is~a~#4~effect.}
}

\DeclareDocumentCommand{\augment}{m m}
{
    \vspace{0.5em}
    \par \nth{#1}~--~\textbf{#2}:\xspace
}

\ExplSyntaxOff

\newcommand{\miscastexplode}{\spellmiscast{Explosive} The spell does not have its normal effect. Instead, the magical energy explodes, dealing 1d6 damage per two spellpower to you and all creatures adjacent to you.}
\newcommand{\miscastrandom}{\spellmiscast{Retargeting} The spell targets a random valid target within range (including yourself, if applicable).}
\newcommand{\miscastyou}{\spellmiscast{Localized} The spell targets you, or originates from your location, instead of its intended location. In addition, it only extends out 5 feet from your location, rather than its normal area.}

\newcommand{\classbasics}[1]{\par\noindent\textbf{#1}:}

\newcommand{\meh}[1]{\hypertarget{suspicious}{}#1}

\newcommand{\pari}{\par\noindent}

% for feat paths
\newcommand{\itempath}{\item\addtocounter{enumi}{1}\xspace}

\newcommand{\magical}{[\glossterm{Magical}]\xspace}
