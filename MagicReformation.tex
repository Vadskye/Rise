\chapter{Magic}

This is an abridged version of the original Magic chapter with a variety of elements relevant to the changes made in the Spell Reformation. There should be very little, if any, material here not also discussed in the general Spell Reformation thread.

\section{Combining Effects}
Spells or magical effects usually work as described, no matter how many other spells or magical effects happen to be operating in the same area or on the same recipient. Except in special cases, a spell does not affect the way another spell operates. Whenever a spell has a specific effect on other spells, the spell description explains that effect.

However, spells, feats, class features, and other effects that have very similar effects may not both help the subject. A character can only be increased so far beyond his or her normal limits; even layered with powerful magical effects, a commoner is no serious threat to a giant. The limitations on these effects are provided by the stacking rules described below.

\subsection{Stacking Effects}

\subsection{Bonuses, Penalties, and Modifiers}
A bonus is the most basic way that a roll or numerical statistic can be modified. All bonuses have one of four types: inherent, competence, enhancement, and circumstance. These types are described in Bonus Types, below. You use these types to determine whether different bonuses stack. Bonuses from the same source never stack.

A penalty is like a bonus, but it subtracts instead of adds. Penalties are never typed, and always stack, except that penalties from the same source don't stack. For example, a creature who is sickened by both a \spell{ghoul touch} spell and an \spell{unholy blight} spell takes a \minus4 penalty to everything that being sickened penalizes. However, a creature suffering from the \spell{crushing despair} spell does not take any additional penalties if another \spell{crushing despair} is cast on it -- even if the second spell is cast by a different caster.

A modifier is the sum of bonuses and penalties. Most of the time, a modifier is the number that you add to the d20 when you take an action. For example, when you attack, you add your attack modifier, which is determined by the sum of all of your attack bonuses and attack penalties. Armor Class is a special case, as discussed in Armor Class, below.

\subsubsection{Bonus Types}
There are four bonus types, described below. Everything which gives a numerical improvement to a character's abilities belongs to one of these bonus types.
\parhead{Inherent bonuses} These are bonuses which are an inseparable part of a character. They come from base attack bonus, skill ranks, ability modifiers, and other fundamental parts of a character. The bonus provided by wearing a suit of armor or wielding a shield is also considered an inherent bonus, since it is inherent to the item. Inherent bonuses stack with other inherent bonuses.
\parhead{Competence bonuses} These are bonuses which derive from a character's experience and ability. They usually come from feats, racial features, and class features. For example, Weapon Focus gives a competence bonus to attack rolls. Competence bonuses do not stack with other competence bonuses.
\parhead{Enhancement bonuses} These are bonuses which derive from magical improvements to a character's abilities. Almost all spells and magic items (except those relating to armor, shields, and weapons) give enhancement bonuses. Enhancement bonuses do not stack with other enhancement bonuses.
\parhead{Circumstance bonuses} These are bonuses which derive from specific circumstances. Feats, class features, and magic items which are dependent on circumstances or are extremely temporary in nature can give circumstance bonuses. For example, a rogue's sneak attack gives a circumstance bonus to damage. Circumstance bonuses stack with other circumstance bonuses.

\subsubsection{Armor Class}
Your AC is equal to 10 plus the sum of five modifiers, described below. Each of those modifiers is calculated separately before being added together to get your final AC. 
\parhead{Armor modifier} An armor modifier usually comes from armor that you wear. It does not apply against touch attacks.
\parhead{Shield modifier} A shield modifier usually comes from a shield that you wield. It does not apply when you are flat-footed.
\parhead{Natural armor modifier} A natural armor modifier represents the physical durability of your body. It does not apply against touch attacks. Armor and natural armor do not fully stack; see below.
\parhead{Dodge modifier} A dodge modifier represents your ability to avoid attacks. It does not apply when you are flat-footed.
\parhead{AC modifier} A small number of things can affect a character's Armor Class directly. AC modifiers apply to all types of Armor Class.

Armor and natural armor are very similar, so they do not fully stack. To calculate your character's total AC, add the higher modifier plus half the lower modifier to your AC. For example, if a warhorse (\plus4 natural armor modifier) wears chainmail barding (\plus6 armor modifier), it gets a total of a \plus8 bonus to AC: the chainmail provides \plus6, and its natural armor is halved to give a \plus2 bonus.

An example can help illustrate how armor class stacking works. Bob the Fighter is wearing \plus1 full plate, wielding a heavy shield, and wearing a Ring of Protection \plus1.
\begin{itemize*}
\item His AC modifier is \plus1 (enhancement, from the Ring of Protection).
\item His armor modifier is 8 (inherent) \add 1 (enhancement) = 9.
\item His shield modifier is 2 (inherent).
\item Therefore, his total armor class is 10 \add 1 \add 9 \add 2 = 22.
\end{itemize*}

If he were to receive a \spell{barkskin} spell, he would gain the full effect, since he does not currently have an enhancement bonus to AC.

\subsection{Special Stacking Rules}
Not every kind of effect uses the normal bonus stacking system. Exceptions to the normal stacking rules are described below. If a specific spell or effect says otherwise, it overrules these exceptions, as normal.
\begin{itemize*}
\item Magical effects that increase size do not stack.
\item Damage reduction does not stack. Only the best value applicable to the attack applies.
\item Spell resistance does not stack.
\item Effects that grant extra attacks (such as the \spell{haste} spell) do not stack.
\item Temporary hit points do not stack.
\item If a character has two separate abilities which let him add the same attribute to a given roll or numerical attribute, the attribute is still only added once.
\item Effects that reduce the effective spell level of a spell affected by metamagic can never reduce a spell below its original level.
\end{itemize*}

\section{Spell Descriptions}
The description of each spell is presented in a standard format. Each category of information is explained and defined below.

\subsection{Name}
The first line of every spell description gives the name by which the spell is generally known.

\subsection{Description}
Beneath the spell name is a brief description of the spell's effect. This description has no mechanical significance, and simply describes how the spell usually appears or is used.

\subsection{School/Schools (Subschool)}
The next line describes the schools and subschools of magic that the spell belongs to. Almost every spell belongs to at least one of eight schools of magic. A school of magic is a group of related spells that work in similar ways. They are described below.

Some spells belong to more than one school of magic. Treat these spells for all purposes as if they were a member of both schools simultaneously. If you are prohibited from casting spells from a certain school, you cannot cast a spell which belongs to that school, even if it also belongs to another school. Likewise, any benefits which apply to casting spells from a specific school apply normally. If you have abilities which apply when casting spells from both schools that make up a spell, the abilities do not stack.

A small number of spells (\spell{limited wish}, \spell{permanency}, \spell{prestidigitation}, and \spell{wish}) are universal, belonging to no school.

\subsubsection{Abjuration}
Abjuration spells manipulate the raw essence of magic to protect allies or ward off foes. There are four subschools of abjuration spells.
\parhead{Interdiction} An interdiction spell hedges out creatures or forces of an opposing alignment or type. \spellindirect{protection from evil}{Protection from evil} is an interdiction spell.
\parhead{Negating} A negating spell negates magical effects. \spell{Dispel magic} is a negating spell.
\parhead{Shielding} A shielding spell protects creatures or objects from harm. \spell{Shield} is a shielding spell.
\parhead{Warding} A warding spell protects an area from intrusion. If one warding spell is active within 10 feet of another for 24 hours or more, the magical fields interfere with each other and create barely visible energy fluctuations. The DC to find such spells with the Spot skill drops by 4. The DC drops by an additional 2 for each additional warding spell beyond the second. \spellindirect{glyph of warding}{Glyph of warding} is a warding spell.

\subsubsection{Conjuration}
Conjuration spells transport and create objects and creatures to aid you. A creature or object brought into being or transported to your location by a conjuration spell cannot appear inside another creature or object, nor can it appear floating in an empty space. It must arrive in an open location on a surface capable of supporting it. The creature or object must appear within the spell's range, but it does not have to remain within the range. There are three subschools of conjuration spells.
\parhead{Creation} A creation spell manipulates matter to create an object or creature in the place the spellcaster designates (subject to the limits noted above). If the spell has a duration other than instantaneous, magic holds the creation together, and when the spell ends, the conjured creature or object vanishes without a trace. If the spell has an instantaneous duration, the created object or creature is merely assembled through magic. It lasts indefinitely and does not depend on magic for its existence. \spell{Acid arrow} is a creation spell.
\parhead{Summoning} A summoning spell instantly brings a manifestation of a creature or object to a place you designate. When the spell ends or is dispelled, the manifestation disappears. A summoned creature also disappears if it is killed or if its hit points drop to 0 or lower. Because summoning spells do not physically transport the actual creature or object, even if the manifestation is injured or destroyed, the original is unharmed. However, it takes 24 hours for the manifestation to reform, during which time it can't be summoned again. Most summoning spells, including the \spell{summon monster} and \spell{summon nature's ally} spells, will automatically summon a different creature of the same type should this occur.
\par When the spell that summoned a creature ends and the creature disappears, all the spells it has cast expire. A summoned creature cannot use any innate summoning abilities it may have.
\par \spell{Summon monster I} is a summoning spell.
\parhead{Translocation} A translocation spell transports one or more creatures or objects a great distance. The most powerful of these spells can cross planar boundaries. Unlike summoning spells, the transportation is (unless otherwise noted) one-way and not dispellable. Many translocation effects involve teleportation (see Descriptors, below). \spell{Dimension door} is a translocation spell.

\subsubsection{Divination}
Divination spells enable you to predict the future, gain or share knowledge, find hidden things, and foil deceptive spells. There are four subschools of divination spells.

\parhead{Awareness} A awareness spell reveals objects, creatures, or effects within an area. Some awareness spells have cone-shaped areas. These move with you and extend in the direction you look. The cone defines the area that you can examine each round. If you study the same area for multiple rounds, you can often gain additional information, as noted in the descriptive text for the spell. \spellindirect{detect evil}{Detect evil} is an awareness spell.
\parhead{Communication} A communication spell magically enhances communication between creatures, often by transcending linguistic barriers or distance. \spellindirect{comprehend languages}{Comprehend Languages} is a communication spell.
\parhead{Knowledge} A knowledge spell grants the recipient information. Most knowledge spells give knowledge about the present, but some can reveal information about the future as well. \spellindirect{comprehend languages}{Comprehend languages} is a knowledge spell.
\parhead{Scrying} A scrying spell creates an invisible magical sensor that sends you information. Unless noted otherwise, the sensor has the same powers of sensory acuity that you possess. This level of acuity includes any spells or effects that target you, but not spells or effects that emanate from you. However, the sensor is treated as a separate, independent sensory organ of yours, and thus it functions normally even if you have been blinded, deafened, or otherwise suffered sensory impairment.
\par Any creature trained in Spellcraft can notice the sensor by making a DC 20 Spellcraft check. The sensor can be dispelled as if it were an active spell. Lead sheeting or magical protection blocks a scrying spell, and you sense that the spell is so blocked.
\par \spell{Scrying} is a scrying spell.

\subsubsection{Enchantment}
Enchantment spells affect the minds of others, influencing or controlling their behavior or mental capabilities. Almost all enchantment spells are mind-affecting spells. There are four subschools of enchantment spells.
\parhead{Beguilement} A beguilement spell influences the subject's opinions. Beguilement spells are the most subtle form of mental control, and a creature affected by such a spell usually does not realize that it is being manipulated until after the spell wears off -- if it does at all. \spellindirect{charm person}{Charm person} is a beguilement spell.
\parhead{Compulsion} A compulsion spell compels the subject to act in a particular way. Especially powerful compulsions can give you complete control over the subject. \spell{Sleep} is a compulsion spell.
\parhead{Emotion} An emotion spell influences the subject's emotions. \spell{Attraction} is an emotion spell.
\parhead{Inhibition} An inhibition spell prevents the subject's mind from working normally, typically preventing the target from acting. \spellindirect{hold person}{Hold person} is an inhibition spell.

\subsubsection{Evocation}
Evocation spells create and manipulate energy and forces or tap into divine or other powers to produce a desired end. In effect, they create energy or effects, but not physical objects, out of nothing. Many of these spells produce spectacular effects, and evocation spells can deal large amounts of damage. There are three subschools of evocation spells.
\parhead{Channeling} A channeling spell channels divine or other power. \spellindirect{holy smite}{Holy smite} is a channeling spell.
\parhead{Control} A control spell manipulates forces and moves inanimate objectss. Powerful control spells can manipulate forces on a large scale, even altering weather patterns. \spellindirect{gust of wind}{Gust of wind} is a control spell.
\parhead{Energy} An energy spell creates or manipulates energy, such as fire or electricity. \spell{Fireball} is an energy spell.

\subsubsection{Illusion}
Illusion spells deceive the senses of others. They conceal things that exist or cause people to perceive things that do not exist. There are three subschools of illusion spells.

\parhead{Figment} A figment spell creates a false sensation. Those who perceive the figment perceive the same thing, not their own slightly different versions of the figment. (It is not a personalized mental impression.) Figments cannot make something seem to be something else. A figment that includes audible effects cannot duplicate intelligible speech unless the spell description specifically says it can. If intelligible speech is possible, it must be in a language you can speak. If you try to duplicate a language you cannot speak, the image produces gibberish, unless you prescribe exactly which sounds to make. Likewise, you cannot make a visual copy of something unless you know what it looks like.
\par A figment's AC is equal to 10 \add its size modifier.
\par \spell{Silent image} is a figment spell.
\parhead{Glamer} A glamer spell changes a subject's sensory qualities, making it look, feel, taste, smell, or sound like something else, or even seem to disappear. \spell{Invisibility} is a glamer spell.
\parhead{Phantasm} A phantasm spell manipulates the subject's senses to create images or sensations that are not real. It creates personalized sensations, and no one else can observe the effect. \spellindirect{phantasmal killer}{Phantasmal killer} is a phantasm spell.
\parhead{Shadow} A shadow spell creates something that is partially real from extradimensional energy. Such illusions can have real effects. Damage dealt by a shadow illusion is real.

\parhead{Unreal Effects} Some figments and glamers are unreal (see Descriptors, below), which means that they can be disbelieved.

\subsubsection{Necromancy}
Necromancy spells manipulate the power of life and death, as well as souls. Spells involving positive and negative energy belong to this school. There are three subschools of necromancy spells.

\parhead{Flesh} A flesh spell affects the home of a creature's life energy: its body. Many flesh spells inflict or remove physical disabilities. \spell{ray of enfeeblement}{Ray of enfeeblement} is a flesh spell.
\parhead{Life} A life spell manipulates a creature's life force directly. \spellindirect{crush life}{Crush life} is a life spell.
\parhead{Soul} A soul spell manipulates the subject's soul, either restoring it to its proper place or fragmenting it for terrible purposes. \spellindirect{raise dead}{Raise dead} is a soul spell.
\parhead{Vitalism} An vitalism spell channels positive or negative energy. This can be used to enhance or destroy a subject's life energy, or to manipulate creatures powered by negative energy. \spellindirect{cure light wounds}{Cure light wounds} is a vitalism spell.

\subsubsection{Transmutation}
Transmutation spells change the properties of creatures and objects. There are three subschools of transmutation spells.

\parhead{Animation} An animation spell grants temporary ``life'' to an affected object. \spellindirect{animate objects}{Animate objects} is an animation spell.
\parhead{Alteration} An alteration spell changes the physical state of anything with a material form. \spellindirect{shape stone}{Shape stone} is an alteration spell.
\parhead{Augment} An augment spell enhances the existing physical or mental abilities of an object or creature. \spellindirect{totemic power}{Totemic power} is an augment spell.
\parhead{Imbuement} An imbuement spell infuses an object or creature with magic, granting it new abilities.. \spell{Fly} is an imbuement spell.
\parhead{Polymorph} A polymorph spell changes a creature's body into a new form. \spellindirect{enlarge person}{Enlarge person} is a polymorph spell.
\parhead{Temporal} A temporal spell manipulates time itself, speeding or slowing its passage for the subject. \spell{Haste} is a temporal spell.

\subsection{[Descriptor]}
Appearing on the same line as the school and subschool, when applicable, is a descriptor that further categorizes the spell in some way. Some spells have more than one descriptor.

The descriptors are acid, air, charm, chaotic, cold, curse, darkness, death, detection, disease, domination, earth, electricity, evil, fear, fire, fog, force, good, language-dependent, lawful, light, mind-affecting, morale, negative, planar, poison, positive, sight-dependent, size-affecting, sound-dependent, sonic, teleportation, trap, unreal, wall, water.

Many of these descriptors have no game effect by themselves, but they govern how the spell interacts with other spells, with special abilities, with unusual creatures, with alignment, and so on.

\begin{itemize*}
\item Air spells do not function in environments without air.
\item Barrier spells cannot be used offensively. If you force the barrier against a force or creature it prohibits, you feel a discernible pressure against the barrier. If you continue to apply pressure, you end the spell.
\item Curse spells cannot be dispelled by \spell{dispel magic} or similar effects. However, they can be removed with a \spell{break enchantment}, \spell{limited wish},\spell{miracle}, \spell{remove curse}, or \spell{wish} spell.
\item A detection spell can penetrate barriers, but is always blocked by special materials of some kind. Unless otherwise specified in the spell description, the spell is blocked by 1 foot of stone, 1 inch of common metal, a thin sheet of lead, or 3 feet of wood or dirt.
\item Fire spells do not function underwater.
\item Fog spells do not function underwater and can be dispersed by wind or fire. Unless the spell specifies otherwise, a moderate wind (11\add mph) disperses the fog in 5 rounds, and a strong wind (21\add mph) disperses the fog in 1 round. A fire spell or other powerful fire effect burns away the fog in the area into which it dealt damage.
\item Language-dependent spells use intelligible language as a medium for communication. If the target cannot understand or cannot hear what the caster of a language-dependant spell says, the spell fails.
\item Mind-affecting spells work only against creatures with an Intelligence score of 1 or higher.
\item Sight-dependent spells use sight as a fundamental component of the spell. If the target cannot see the spell, it has no effect.
\item Size-affecting spells alter a creature's size. Multiple size increasing or size decreasing effects never stack. If a creature is affected by both size-increasing and size-decreasing effects, they cancel out on a one for one basis, and any remaining effect occurs normally.
\item Sound-dependent spells use sound as a fundamental component of the spell. If the target cannot hear the spell, it has no effect.
\item Teleportation spells instantaneously move creatures by travelling through the Astral Plane. Anything that blocks planar travel also blocks teleportation.
\item Trap spells do not have obvious effects immediately. They can be detected with the Spot skill. The DC to detect a trap spell is 25 \add spell level. Most, but not all, traps can be disabled with the Disable Device skill. If it can be disabled, the DC is 25 \add spell level.
\par No more than one trap spell can be placed on the same object or in the same area. Only the first trap placed has any effect. It must be dispelled or discharged before any new traps can be placed.
\item Unreal spells do not have ``real'' effects and can be disbelieved. Unreal effects cannot cause damage to objects or creatures, support weight, provide nutrition, or provide protection from the elements. Consequently, these spells are useful for confounding or delaying foes, but useless for attacking them directly unless combined with a real effect.
\par Creatures encountering an unreal spell usually do not receive saving throws to recognize it as illusory until they study it carefully or interact with it in some fashion. A Spot or Listen check can be made to interact with an unreal effect if appropriate to the type of effect. Unless otherwise specified by the spell, the DC of such a check is equal to the saving throw DC of the spell.
\par Once a creature has interacted with an unreal effect, it can make a Will save. A successful saving throw reveals it to be false. Its effects can still be observed if desired, but they are mere shadows of the full effect: visual effects appear translucent outlines, sounds can be heard as ghostly echoes, and so on.
\par A failed saving throw indicates that a character fails to notice something is amiss. A character faced with definitive proof that an unreal effect isn't real needs no saving throw. If any viewer successfully disbelieves an unreal effect and communicates this fact to others, each such viewer gains a saving throw with a \plus4 bonus.
\end{itemize*}

\subsection{Level}
The next line of a spell description gives the spell's level, a number between 0 and 9 that defines the spell's relative power. This number is preceded by an abbreviation for the class whose members can cast the spell. The Level entry also indicates whether a spell is a domain spell and, if so, what its domain and its level as a domain spell are. A spell's level affects the DC for any save allowed against the effect.

Names of spellcasting classes are abbreviated as follows: bard Brd; cleric Clr; druid Drd; paladin Pal; ranger Rgr; sorcerer Sor; wizard Wiz.

The domains a spell can be associated with include Air, Animal, Chaos, Death, Destruction, Earth, Evil, Fire, Good, Healing, Knowledge, Law, Leadership, Magic, Plant, Protection, Strength, Travel, Trickery, War, and Water.

\subsection{Components}
A spell's components are what you must do or possess to cast it. All spells have verbal and somatic components unless the spell description says otherwise. The Components entry in a spell description includes abbreviations that tell you what type of components it has. Specifics for material components and focuses are given at the end of the descriptive text.

\parhead{Verbal (V)} A verbal component is a spoken incantation. To provide a verbal component, you must be able to speak in a strong voice.

A gag spoils the incantation (and thus the spell). A spellcaster who has been deafened has a 20\% chance to spoil any spell with a verbal component that he or she tries to cast. A \spell{silence} spell imposes a 50\% chance of failure.

\parhead{Somatic (S)} A somatic component is a measured and precise movement of the hand. You must have at least one hand free to provide a somatic component. Touch range spells often include the act of touching the spell recipient as part of the somatic component.

\parhead{Material (M)} A material component is one or more physical substances or objects that are annihilated by the spell energies in the casting process.

\parhead{Focus (F)} A focus component is a prop of some sort. Unlike a material component, a focus is not consumed when the spell is cast and can be reused.

\subsection{Casting Time}
All spells have a casting time of 1 standard action unless otherwise specified in the spell description. Some take 1 round or more, while a few require only a swift or immediate action.

A spell that takes 1 round to cast requires a full-round action. It comes into effect just before the beginning of your turn in the round after you began casting the spell. You then act normally after the spell is completed.

A spell that takes 1 minute to cast comes into effect just before your turn 1 minute later (and for each of those 10 rounds, you are casting a spell as a full-round action, just as noted above for 1 round casting times). These actions must be consecutive and uninterrupted, or the spell automatically fails.

When you begin a spell that takes 1 round or longer to cast, you must continue the concentration from the current round to just before your turn in the next round (at least). If you lose concentration before the casting is complete, you lose the spell.

A spell with a casting time of 1 swift action (such as a quickened spell) or 1 immediate action doesn't count against your normal limit of one spell per round. Casting a spell with those casting times doesn't provoke attacks of opportunity.

You make all pertinent decisions about a spell (range, target, area, effect, version, and so forth) when the spell comes into effect.

\subsection{Range}
A spell's range indicates how far from you it can reach, as defined in the Range entry of the spell description. It indicates the maximum distance at which you can designate the spell's point of origin. The effect of a spell can extend beyond that range if it affects an area. A spell without a range simply affects the area specified in the spell's description; if it becomes relevant, you are considered to be its point of origin. Standard ranges include the following.
\parhead{Personal} The spell affects only you.
\parhead{Touch} You must touch a creature or object to affect it. A touch spell that deals damage can score a critical hit just as a weapon can. A touch spell threatens a critical hit on a natural roll of 20 and deals double damage on a successful critical hit. Some touch spells allow you to touch multiple targets. You can touch as many willing targets as you can reach as part of the casting, but all targets of the spell must be touched in the same round that you finish casting the spell. Normally, you can touch no more than six targets per round of casting.

If you have the ability to make multiple touch attacks, such as from the \spell{chill touch} spell, and you can make multiple attacks in a round, you can make a touch attack on each of those attacks.

\parhead{Close} The spell reaches as far as 30 feet.
\parhead{Medium} The spell reaches as far as 100 feet.
\parhead{Far} The spell reaches as far as 300 feet.
\parhead{Unlimited} The spell reaches anywhere on the same plane of existence.
\parhead{Range Expressed in Feet} Some spells have no standard range category, just a range expressed in feet.

\subsection{Aiming a Spell}
You must make some choice about whom the spell is to affect or where the effect is to originate, depending on the type of spell. The next entry in a spell description defines the spell's target (or targets), its manifestation, or its area, as appropriate.

\parhead{Area}\label{Spell Area} Some spells affect an area. Sometimes a spell description specifies a specially defined area, but usually an area falls into one of the categories defined below.

When casting an area spell, you select the point where the spell originates. The point of origin of a spell is always a grid intersection. When determining whether a given creature is within the area of a spell, count out the distance from the point of origin in squares just as you do when moving a character or when determining the range for a ranged attack. The only difference is that instead of counting from the center of one square to the center of the next, you count from intersection to intersection.

You can freely decrease a spell's area, provided that you decrease it uniformly across all of the spell's dimensions. For example, you can cast a \spell{fireball} that affects a 5 foot radius if you choose to do so, but you can't cast a \spell{fireball} with any shape other than a sphere.

You can count diagonally across a square, but remember that every second diagonal counts as 2 squares of distance. If the far edge of a square is within the spell's area, anything within that square is within the spell's area. If the spell's area only touches the near edge of a square, however, anything within that square is unaffected by the spell.

\subparhead{Burst, Emanation, Spread} Many spells that affect an area function as a burst, an emanation, a limit, or a spread. In each case, you select the spell's point of origin and measure its effect from that point.

A burst spell affects whatever it catches in its area, even including creatures that you can't see. A burst's area defines how far from the point of origin the spell's effect extends. The effects of burst spells do not extend around corners.

An emanation spell functions like a burst spell, except that the effect continues to radiate from the point of origin for the duration of the spell. Many emanations are cones.

A spread spell spreads out like a burst but can turn corners. You select the point of origin, and the spell spreads out a given distance in all directions. Figure the area the spell effect fills by taking into account any turns the spell effect takes.

\subparhead{Limit} Some area spells specify a limit. A limiting area is like a range: the spell has effects within the area, but does not affect the entire area at once. The spell will specify the targets that it affects or the manifestations it creates. Limit spells, like bursts and emanations, do not go around corners.

\subparhead{Cone, Cylinder, Line, or Sphere} Most spells that affect an area have a particular shape, such as a cone, cylinder, line, or sphere. Unless otherwise specified, the shape of a burst, emanation, or spread is a sphere.

A cone-shaped spell shoots away from you in a quarter-circle in the direction you designate, extending out to a limit defined by the spell. It starts from any corner of your square and widens out as it goes. Most cones are either bursts or emanations (see above), and thus won't go around corners.

When casting a cylinder-shaped spell, you select the spell's point of origin. This point is the center of a horizontal circle, and the spell shoots down from the circle, filling a cylinder. A cylinder-shaped spell ignores any vertical obstructions within its area.

A line-shaped spell shoots away from you in a line in the direction you designate. It starts from any corner of your square and extends to the limit of its range or until it strikes a barrier that blocks line of effect. A line-shaped spell affects all creatures in squares that the line passes through. Unless otherwise specified, a line spell affects an area 10 feet wide. The affected squares are chosen such that they stay close to the chosen line as possible.

A sphere-shaped spell expands from its point of origin to fill a spherical area. Spheres are typically denoted by simply specifying the radius of the spell.

\subparhead{Area Sizes} The area affected by many spells falls into one of three sizes. Each size defines the extent to which the spell extends out from its origin, whether as a radius or as the length of a cone or line. Small spells extend 10 feet out. Medium spells extend 20 feet out. Large spells extend 50 feet out. Other spells affect a specific area defined in the spell's description.

\subparhead{Objects} A spell with this kind of area affects objects within an area you select (as Creatures, but affecting objects instead).

\subparhead{Other} A spell can have a unique area, as defined in its description.

%Does this still exist?
\subparhead{(S) Shapeable} If an Area or Effect entry ends with ``(S)," you can shape the spell. A shaped effect or area can have no dimension smaller than 10 feet. Many effects or areas are given as cubes to make it easy to model irregular shapes. Three-dimensional volumes are most often needed to define aerial or underwater effects and areas.

\parhead{Target or Targets} Some spells have a target or targets. You cast these spells on creatures or objects, as defined by the spell itself. You must be able to see or touch the target, and you must specifically choose that target. You do not have to select your target until you finish casting the spell.

If the target of a spell is yourself (the spell description has a line that reads Target: You), you do not receive a saving throw, and spell resistance does not apply. The Saving Throw and Spell Resistance lines are omitted from such spells.
Some spells restrict you to willing targets only. Declaring yourself as a willing target is something that can be done at any time (even if you're flat-footed or it isn't your turn). Unconscious creatures are automatically considered willing, but a character who is conscious but immobile or helpless (such as one who is bound, cowering, grappling, paralyzed, pinned, or stunned) is not automatically willing.

Some spells allow you to redirect the effect to new targets or areas after you cast the spell. Redirecting a spell is a swift action that does not provoke attacks of opportunity.

Most spells which have multiple targets also specify an area that the targets must reside in. If the spell says ``all creatures'', you do not have the ability to choose which creatures it affects; otherwise, you may pick and choose creatures within the area.

Many spells affect ``living creatures," which means all creatures other than constructs and undead. Creatures in the spell's area that are not of the appropriate type do not count against the creatures affected.

\parhead{Manifestation} Some spells create or summon things rather than affecting things that are already present. You must designate the location where these things are to appear, either by seeing it or defining it. Range determines how far away a manifestation can appear, but if the manifestation is mobile, it can move regardless of the spell's range.

\subparhead{Ray} Some effects are rays. You aim a ray as if using a ranged weapon, though typically you make a ranged touch attack rather than a normal ranged attack. As with a ranged weapon, you can fire into the dark or at an invisible creature and hope you hit something. You don't have to see the creature you're trying to hit, as you do with a targeted spell. Intervening creatures and obstacles, however, can block your line of sight or provide cover for the creature you're aiming at.

If a ray spell has a duration, it's the duration of the effect that the ray causes, not the length of time the ray itself persists.

If a ray spell deals damage, you can score a critical hit just as if it were a weapon. A ray spell threatens a critical hit on a natural roll of 20 and deals double damage on a successful critical hit.

\subparhead{Manifestations and Areas} Some effects, such clouds and fogs, create a manifestation within an area. Follow all of the normal rules for determining the area when determining the effects of such a spell.

\parhead{Line of Effect} A line of effect is a straight, unblocked path that indicates what a spell can affect. A line of effect is canceled by a solid barrier. It's like line of sight for ranged weapons, except that it's not blocked by fog, darkness, and other factors that limit normal sight.

You must have a clear line of effect to any target that you cast a spell on or to any space in which you wish to create an effect. You must have a clear line of effect to the point of origin of any spell you cast.

A burst, cone, cylinder, or emanation spell affects only an area, creatures, or objects to which it has line of effect from its origin (a spherical burst's center point, a cone-shaped burst's starting point, a cylinder's circle, or an emanation's point of origin).

An otherwise solid barrier with a hole of at least 1 square foot through it does not block a spell's line of effect. Such an opening means that the 5-foot length of wall containing the hole is no longer considered a barrier for purposes of a spell's line of effect.

\subsection{Duration}
A spell's Duration entry tells you how long the magical energy of the spell lasts.
\parhead{Concentration} The spell lasts as long as you concentrate on it. Concentrating to maintain a spell is a standard action that does not provoke attacks of opportunity. Anything that could break your concentration when casting a spell can also break your concentration while you're maintaining one, causing the spell to end. Concentrating on an active spell is easier than

You can't cast a spell while concentrating on another one. Sometimes a spell lasts for a short time after you cease concentrating.
\parhead{Timed Durations} Many durations are measured in rounds, minutes, hours, or some other increment. When the time is up, the magic goes away and the spell ends at the end of your turn. For example, a spell that lasts 1 round ends at the end of your next turn. If a spell's duration is variable, the duration is rolled secretly (the caster doesn't know how long the spell will last).
\subparhead{Short} The spell lasts for as long as you concentrate, plus 5 additional rounds.
\subparhead{Medium} The spell lasts for 5 minutes.
\subparhead{Long} The spell lasts for 1 hour.
\subparhead{Extreme} The spell lasts for 12 hours.
\subparhead{Other} Some spells, such as long-lasting rituals, last for a specific amount of time specified in the spell description.
\parhead{Instantaneous} The spell energy comes and goes the instant the spell is cast, though the consequences might be long-lasting.
\parhead{Permanent} The energy remains as long as the effect does. This means the spell is vulnerable to \spell{dispel magic}.

\parhead{Subjects, Effects, and Areas} If the spell affects creatures directly, the result travels with the subjects for the spell's duration. If the spell creates an manifestation, it lasts for the duration. The manifestation might move or remain still. Such an effect can sometimes be destroyed prior to when its duration ends. If the spell affects an area, then the spell stays with that area for its duration. Creatures become subject to the spell when they enter the area and are no longer subject to it when they leave.

\parhead{Touch Spells and Holding the Charge} In most cases, if you don't discharge a touch spell on the round you cast it, you can hold the charge (postpone the discharge of the spell) indefinitely. You can make touch attacks round after round. If you cast another spell, the touch spell dissipates.

Some touch spells allow you to touch multiple targets as part of the spell. You can't hold the charge of such a spell; you must touch all targets of the spell in the same round that you finish casting the spell.

\parhead{Discharge} Occasionally a spells lasts for a set duration or until triggered or discharged.

\parhead{(D) Dismissible} If the Duration line ends with ``(D)," you can dismiss the spell at will. You must be within range of the spell's effect and must speak words of dismissal, which are usually a shortened, modified form of the spell's verbal component. If the spell has no verbal component, you can dismiss the effect with a gesture. Dismissing a spell is a swift action that does not provoke attacks of opportunity. A spell that depends on concentration is dismissible by its very nature, and dismissing it does not take an action, since all you have to do to end the spell is to stop concentrating on your turn.

\subsection{Saving Throw}
Usually a harmful spell allows a target to make a saving throw to avoid some or all of the effect. The Saving Throw entry in a spell description defines which type of saving throw the spell allows and describes how saving throws against the spell work.
\parhead{Negates} The spell has no effect on a subject that makes a successful saving throw.
\parhead{Partial} The spell causes an effect on its subject. A successful saving throw means that some lesser effect occurs.
\parhead{Half} The spell deals damage, and a successful saving throw halves the damage taken (round down).
\parhead{None} No saving throw is allowed.
\parhead{Disbelief} A successful save lets the subject ignore the effect.
\parhead{(object)} The spell can be cast on objects, which receive saving throws only if they are magical or if they are attended (held, worn, grasped, or the like) by a creature resisting the spell, in which case the object uses the creature's saving throw bonus unless its own bonus is greater. (This notation does not mean that a spell can be cast only on objects. Some spells of this sort can be cast on creatures or objects.) A magic item's saving throw bonuses are each equal to 2 \add one-half the item's caster level.
\parhead{(harmless)} The spell is usually beneficial, not harmful, but a targeted creature can attempt a saving throw if it desires.

\subsubsection{Saving Throw Difficulty Class}
A saving throw against your spell has a DC of 10 \add half your caster level \add your casting attribute. If you have more than one caster level, use the caster level appropriate to the class that you are casting the spell from, including any modifiers specific to that spell (such as from the Spell Focus feat).

\parhead{Succeeding on a Saving Throw} A creature that successfully saves against a spell that has no obvious physical effects feels a hostile force or a tingle, but cannot deduce the exact nature of the attack without the use of the Spellcraft skill.

\parhead{Overwhelming Failures and Successes} A natural 1 (the d20 comes up 1) on a saving throw is treated as rolling a \minus10, and a natural 20 (the d20 comes up 20) is treated as rolling a 30. Under all but the most extreme circumstances, a natural 1 is an automatic failure, and a natural 20 is an automatic success.

\parhead{Voluntarily Giving up a Saving Throw} A creature can voluntarily forego a saving throw and willingly accept a spell's result. However, a character with a special resistance to magical effects cannot suppress that quality.

\parhead{Damaging Items} Unless the descriptive text for the spell specifies otherwise, all items carried or worn by a creature are assumed to survive a magical attack. If an item is not carried or worn and is not magical, it does not get a saving throw. It simply is dealt the appropriate damage.

\subsection{Spell Resistance}
Each spell that allows spell resistance specifies a saving throw type. A creature with spell resistance may always make a saving throw when a spell is cast on it. If it succeeds, the spell has no effect on it. The type of saving throw made is indicated by the spell. If the spell also allows a saving throw of the same type, only one roll is made.

Most creatures with spell resistance must voluntarily lower their resistance (a swift action) in order to be affected by any spell, even one noted as harmless. In such a case, you do not need to make the caster level check described above. A creature who lowers its spell resistance cannot benefit from it again until the beginning of its next turn, at which point its spell resistance automatically returns. It must spend another swift action to lower it again, if it wishes to do so. Some creatures lower their spell resistance differently; see the individual description for details.

\subsection{Effect}
This portion of a spell description details what the spell does and how it works. If one of the previous entries in the description included ``see text,'' this is where the explanation is found. There are several key parts of a spell which are also contained here.

\subsubsection{Damage}
This is the amount of damage the spell deals. Typically, the effect will specify who takes the damage. If no effect is specified, the spell damages all of its targets, or all creatures (but not objects) in the area.

\subsubsection{Healing}
This is the amount of damage the spell heals. Typically, the effect will specify who receives the healing. If no effect is specified, the spell heals all of its targets, or all creatures (but not objects) in the area.

\subsubsection{Healthy Effect}
This is the effect a spell has on a healthy subject (above half hit points remaining).

\subsubsection{Bloodied Effect}
This is the effect a spell has on a bloodied subject (at or below half hit points remaining). If the spell has a duration, and a healthy creature becomes bloodied during the duration of the spell, it immediately suffers the bloodied effect of the spell. If the spell does not have a duration, any damage the subject takes after being affected by the spell does not change which effect the spell has.

\section{Arcane Invocations}
Arcane invocations are special spell-like abilities that arcane casters can use at will. Unlike other spell-like abilities, they have verbal and somatic components and are subject to arcane spell failure. All arcane invocations take a standard action to cast unless specified otherwise in the description. Arcane invocations are considered to be 0th level for the purpose of spells and abilities which reference spell level. They are described at the end of Chapter 12.

\section{Rituals}
Rituals are ceremonies that create magical effects. You don't memorize a ritual as you would a normal spell; rituals are too complex for all but the mightiest wizards to commit to memory. To cast a ritual, you need to read from a book or a scroll containing it. Rituals are considered to be spells for many purposes, such as for spell resistance and for effects related to spells, but they are learned and cast in very different ways.
\subsection{Ritual Descriptions}
\par Like a spell, each ritual has a school, a level, and a magical effect. Rituals are described in Chapter 13. The description of each ritual follows the same format as the description of spells in Chapter 12, with two differences. First, unlike spells, which have levels based on class, rituals have levels based on the source of magic: arcane, divine, or both. A ritual always matches the magic source of the person performing the ritual. For example, \spell{scrying} is an arcane ritual when performed by a wizard, but a divine ritual when performed by a cleric or druid. Second, each ritual has one or more skills associated with it. In order to learn and perform a ritual, a character must be trained in one of the skills associated with the ritual.
\subsection{Ritual Books}
A ritual book contains one or more rituals that you can use as frequently as you like, as long as you can spare the time and the components to perform the ritual. Scribing a ritual in a ritual book costs an amount of precious inks equal to 100 \mtimes ritual level \mtimes ritual level.
\subsection{Ritual Components}
Every ritual has a material component cost. Unless otherwise specified in the ritual description, the material component cost for a ritual is equal to 20 \mtimes ritual level \mtimes ritual level. This cost can be paid with precious metals or gems.
\subsection{Casting Rituals}
\par To cast a ritual, you must have a ritual book containing the ritual and the material components required for the ritual. At the end of the ritual, make a skill check using a skill appropriate for the ritual. If you are trained in multiple skills associated with the ritual, you may choose which skill to use. The DC for the check is equal to 10 \add twice the ritual's level. If you succeed, the ritual is cast successfully and the material components are expended. If you fail by 9 or less, the ritual is cast successfully and the material components are expended, but channeling the magic drains you and you gain a negative level. This negative level persists for 24 hours before disappearing. If you fail by 10 or more, you botched the ritual. The material components are expended, but the ritual has no effect, and you gain a negative level that lasts for 24 hours.
\par You can take 10 on this skill check if you are not threatened, even if the skill is a Knowledge skill. You can also take 20 by spending twenty times the normal time to cast the ritual, and expending twice the normal amount of material components.
