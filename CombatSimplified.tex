\chapter{Combat}\label{Combat}

\section{How Combat Works}
Combat takes place in a series of ``rounds'', which represents about six seconds of action. Each round, every character gets a turn to act. The order in which creatures take turns is determined by their initiative (see \pcref{Initiative}). 

When it's your turn, you can take one standard action, one move action, and one swift action. If you want, you can take a full-round action instead of your standard and move actions. You can always ``downgrade'' an action to a lesser action: turning a standard action into a move action, or a move action into a swift action.

\section{Attacks}
As a standard action, you can make an attack roll with a weapon you are wielding, adding your attack modifier to the roll. If your result exceeds your target's Armor Class, your attack hits, and you deal damage.

\subsection{Attack Modifier}
Your attack modifier is equal to the following:

\begin{figure}[h]
\centering Base attack bonus \add attack attribute \add size modifier \add competence bonuses \add enhancement bonus
\end{figure}

Your base attack bonus is determined by your class levels. With medium or heavy weapons, your attack attribute is your Strength. With light weapons and projectile weapons, your attack attribute is your Dexterity. Your size modifier is described on \tref{Size Modifiers}.

\subsection{Reach}
When you wield a normal melee weapon, you can attack anyone within your reach. Typically, that means anyone adjacent to you. Reach for larger and smaller creatures is determined by size, as shown on \trefnp{Size in Combat}.

\subsection{Nonproficiency}
If you attack with a weapon you aren't proficient with, you provoke an attack of opportunity if you miss.

\subsection{Damage}
When your attack succeeds, you deal damage equal to your weapon's damage die \add half your Strength.

\section{Defenses}

\subsection{Armor Class}
Your Armor Class (AC) represents how hard it is for opponents to land a solid, damaging blow on you. It's the attack roll result that an opponent needs to achieve to hit you. Unlike every other combat statistic, your AC is made up of the sum of other modifiers, each of which can have competence and enhancement bonuses applied separately. Your AC is equal to the following:

{\centering \textbf{10 \add half base attack bonus \add armor modifier \add shield modifier \add Dexterity \add natural armor modifier \add deflection modifier \add dodge modifier \add size modifier}}

\parhead{Base Attack Bonus} Your experience and aptitude in combat affects your ability to defend yourself; experienced warriors know how to recognize and avoid or parry blows that would easily fell novices. As a result, you add half your base attack bonus to your dodge modifier to armor class. This is an inherent bonus, and stacks with all other bonuses to your dodge modifier.
\parhead{Natural Armor Modifier} Natural armor, such as from having unusually tough skin or thick hide, improves your AC. Armor and natural armor do not fully stack; add the higher modifier plus half the lower modifier to your AC. For example, if a warhorse (\plus4 natural armor modifier) wears chainmail barding (\plus6 armor modifier), it gets a total of a \plus8 bonus to AC: the chainmail provides \plus6, and its natural armor is halved to give a \plus2 bonus.
\parhead{Dodge Modifier} Your dodge modifier represents your ability to actively avoid blows. Any situation that denies you your Dexterity bonus also denies you your dodge modifier.
\parhead{Size Modifier} Your size modifier is described on \tref{Size Modifiers}.

\subsubsection{Flat-Footed Armor Class}
Sometimes you can't use your agility to avoid an attack. Your flat-footed armor class is equal to your armor class, ignoring your Dexterity bonus, dodge modifier, and shield modifier. If your Dexterity is negative, you always apply the penalty, even while flat-footed.

\subsubsection{Touch Armor Class}
Sometimes your armor doesn't help you avoid an attack. Your touch armor class is equal to your armor class, ignoring your armor modifier and natural armor modifier. Most touch attacks come from spells.

\subsection{Hit Points}
Your hit points is a defense that represents how much more punishment you can take. When you run out hit points, you can't act anymore, and you might die, as described in \pcref{Injury and Death}.

\subsection{Saving Throws}
Generally, when you are subject to an unusual or magical attack, you get a saving throw to avoid or reduce the effect. You make a saving throw by rolling a d20 and adding your saving throw modifier. If it is greater than the save DC of the attack, you resist the attack (though it may still have partial effects on you). Your saving throw modifier is calculated as follows:

\begin{figure}[h]
\centering Base save bonus \add primary attribute \add 1/2 secondary attribute \add competence bonuses \add enhancement bonus
\end{figure}

Your base save bonus is determined by your class levels.

\subsubsection{Saving Throw Types}
The three different kinds of saving throws are Fortitude, Reflex, and Will.
\parhead{Fortitude} These saves measure your ability to stand up to physical punishment or attacks against your vitality and health. Apply your Constitution and half your Strength to your Fortitude saving throws.
\parhead{Reflex} These saves test your ability to dodge area attacks. Apply your Dexterity and half your Wisdom to your Reflex saving throws.
\parhead{Will} These saves reflect your resistance to mental influence as well as many magical effects. Apply your Charisma and half your Intelligence to your Will saving throws.

\section{Attacks of Opportunity}
Sometimes a combatant in a melee lets her guard down. In this case, combatants near her can take advantage of her lapse in defense to attack her for free. These free attacks are called attacks of opportunity.

\parhead{Threatened Squares} When wielding a melee weapon you are proficient with, you threaten all squares into which you can make a melee attack, even when it is not your action. Generally, that means everything in all squares adjacent to your space (including diagonally). Creatures with reach weapons or very large creatures may threaten more squares. An enemy that takes certain actions while in a threatened square provokes an attack of opportunity from you.

\parhead{Provoking an Attack of Opportunity} All attacks of opportunity come from focusing on something other than the battle at hand. This can come in two forms: moving away from an opponent, and performing certain actions within a threatened area.
\subparhead{Leaving the Battle} Moving farther away from an opponent who threatens you provokes an attack of opportunity. The withdraw action can be used to avoid such an attack. In addition, if you move at no more than half your speed, you do not provoke for moving away from any opponent that you attacked during your turn.
\subparhead{Ignoring the Battle} Any action which requires you to concentrate on something other than defending yourself provokes an attack of opportunity. Casting a spell and attacking with a ranged weapon, for example, require concentrating on something other than avoiding a foe's blows. \tref{Actions in Combat} notes many of the actions that provoke attacks of opportunity.

\par Remember that even actions that normally provoke attacks of opportunity may have exceptions to this rule.

\parhead{Making an Attack of Opportunity} An attack of opportunity is a single melee attack. You can make a number of attacks of opportunity each round equal to 1 \add half your Dexterity, but never more than one per opportunity. You don't have to make an attack of opportunity if you don't want to.

An attack of opportunity ``interrupts'' the normal flow of actions in the round. If an attack of opportunity is provoked, immediately resolve the attack of opportunity, then continue with the next character's turn (or complete the current turn, if the attack of opportunity was provoked in the midst of a character's turn).

\section{Movement}

\section{Initiative}

At the start of a battle, each creature that is aware of the combat makes an initiative check. Your initiative check is calculated as follows:

\begin{figure}[h]
    \centering Dexterity \add half Wisdom \add competence bonuses \add enhancement bonus
\end{figure}

Creatures act in order of their initiative, highest to lowest. After the creature with the lowest initiative acts, a new round begins and the creature with the highest initiative acts again. You usually keep your initiative for the whole encounter, even if you can't act. However, some special actions can change your initiative count.

\subsubsection{Surprise Attacks}
Sometimes, some creatures are not aware of the combat when it starts. This most commonly happens with ambushes. Any creature that is not aware of the combat doesn't get to make an initiative check during the first round of combat. Until it takes its first action, it is flat-footed and can't take attacks of opportunity.

Sometimes, everyone is surprised, such as if a guard walks around a corner to unexpectedly find a group of muggers. In that case, initiative is rolled normally, but all creatures are flat-footed and can't take attacks of opportunity until they take their first action.

\section{Circumstances, Bonuses, and Penalties}

\subsection{Size in Combat}
Size affects your space and reach in combat. In addition, your attack modifier and armor class is affected by your size modifier. These effects are shown on \trefnp{Size in Combat}. 

\begin{dtable*}
    \lcaption{Size in Combat}
    \begin{tabularx}{\textwidth}{l l l l X}
        \thead{Size} & \thead{Size Modifier} & \thead{Space\fn{1}} & thead{Reach\fn{1}} & \thead{Example Creature} \\
        Fine & \plus8 & 1/2 ft. & 0 & Fly\\
        Diminutive & \plus4 & 1 ft. 0 & Toad \\
        Tiny & \plus2 & 2-1/2 ft. & 0 \\
        Small & \plus1 & 5 ft. & 5 ft. & Halfling \\
        Medium & \plus0 & 5 ft. & 5 ft. & Human \\
        Large (tall) & \minus1 & 10 ft. & 10 ft. & Ogre \\
        Large (tall) & \minus1 & 10 ft. & 5 ft. & Horse \\
        Huge (tall) & \minus2 & 15 ft. & 15 ft. & Cloud giant \\
        Huge (long) & \minus2 & 15 ft. & 10 ft. & Bulette \\
        Gargantuan (tall) & \minus4 & 20 ft. & 20 ft. & 50-ft. animated statue \\
        Gargantuan (long) & \minus4 & 20 ft. & 15 ft. & Kraken \\
        Colossal (tall) & \minus8 & 30 ft.\plus & 30 ft.\plus & Colossal animated object \\
        Colossal (long) & \minus8 & 30 ft.\plus & 20 ft.\plus & Great wyrm red dragon \\
    \end{tabularx}
    1 Creatures can vary in space and reach. These are simply typical values.
\end{dtable*}

\subsection{Range Increments}
When using a ranged weapon, you take a \minus2 penalty per range increment between you and your target. For example, when using a longbow with a range increment of 100 feet against a target 170 feet away, you take a \minus2 penalty to attack rolls.
