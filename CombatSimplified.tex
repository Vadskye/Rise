\chapter{Combat}\label{Combat}

\section{How Combat Works}
Combat takes place in a series of ``rounds'', which represents about six seconds of action. Each round, every character gets a turn to act. The order in which creatures take turns is determined by their initiative (see \pcref{Initiative}). 

When it's your turn, you can take one standard action, one move action, and one swift action. If you want, you can take a full-round action instead of your standard and move actions. You can always ``downgrade'' an action to a lesser action: turning a standard action into a move action, or a move action into a swift action.

\section{Attacks}

\subsection{Standard Attack}
As a standard action, you can try to strike a foe with a weapon you are wielding. To do so, make an attack roll with a weapon you are wielding, adding your attack modifier to the roll. If your result exceeds your foe's Armor Class, your attack hits, and your foe takes damage.

\subsubsection{Attack Modifier}
Your attack modifier is equal to the following:

\begin{figure}[h]
\centering Base attack bonus \add attack attribute \add size modifier \add competence bonuses \add enhancement bonus
\end{figure}

With medium or heavy weapons, your attack attribute is your Strength. With projectile weapons, your attack attribute is your Dexterity. With light weapons, you can choose between your Strength and Dexterity. 

Your size modifier is described in \tref{Size in Combat}.

\subsubsection{Damage}
If your attack succeeds, you deal damage equal to your weapon's damage die \add half your Strength.

\parhead{Dealing Nonlethal Damage} If you take a \minus4 penalty to your attack roll, you can deal nonlethal damage instead of lethal damage when you hit. See \pcref{Nonlethal Damage}.

\subsection{Combat Maneuvers}
As a standard action, you can try to hinder your foe by performing a combat maneuver, such as by disarming or tripping him. To do so, make an attack roll, adding your maneuver modifier to the roll instead of your attack modifier. If your result exceeds your foe's Maneuver Class, your attack hits, and your foe suffers the effects of the maneuver.

\subsubsection{Maneuver Modifier}
Your maneuver modifier is equal to the following:

\begin{figure}[h]
\centering Base attack bonus \add attack attribute \add special size modifier \add competence bonuses \add enhancement bonus
\end{figure}

Your attack attribute depends on the maneuver you are using, as described in \pcref{Combat Maneuvers}.

Your special size modifier is described in \tref{Size in Combat}.

\subsection{Reach}
Normally, you can attack anyone within five feet of you. That is called your ``reach''. The area that you can attack into is called your ``threatened area''. Reach for larger and smaller creatures is determined by size, as shown on \trefnp{Size in Combat}.


\section{Defenses}

\subsection{Armor Class}
Your Armor Class (AC) represents how hard it is for opponents to land a solid, damaging blow on you. It's the attack roll result that an opponent needs to achieve to hit you. Unlike most combat statistic, your AC is made up of the sum of other modifiers, each of which can have competence and enhancement bonuses applied separately. You can also apply competence and enhancement bonuses directly. Your AC is equal to the following:

\begin{figure}[h]
    \centering 10 \add half base attack bonus \add Dexterity \add armor modifier \add shield modifier \add natural armor modifier \add dodge modifier \add size modifier \add competence bonuses \add enhancement bonus
\end{figure}

\parhead{Base Attack Bonus} Your experience and aptitude in combat affects your ability to defend yourself; experienced warriors know how to recognize and avoid or parry blows that would easily fell novices. As a result, you add half your base attack bonus to your dodge modifier to armor class. This is an inherent bonus, and stacks with all other bonuses to your dodge modifier.
\parhead{Natural Armor Modifier} Natural armor, such as from having unusually tough skin or thick hide, improves your AC. Armor and natural armor do not fully stack; add the higher modifier plus half the lower modifier to your AC. For example, if a warhorse (\plus4 natural armor modifier) wears chainmail barding (\plus6 armor modifier), it gets a total of a \plus8 bonus to AC: the chainmail provides \plus6, and its natural armor is halved to give a \plus2 bonus.
\parhead{Dodge Modifier} Your dodge modifier represents your ability to actively avoid blows. Any situation that denies you your Dexterity bonus also denies you your dodge modifier.
\parhead{Size Modifier} Your size modifier is described on \tref{Size in Combat}.

\subsubsection{Flat-Footed Armor Class}
Sometimes you can't use your agility to avoid an attack. Your flat-footed armor class is equal to your armor class, ignoring your Dexterity, dodge modifier, and shield modifier. If your Dexterity is negative, you always apply the penalty, even while flat-footed.

\subsubsection{Touch Armor Class}
Sometimes your armor doesn't help you avoid an attack. Your touch armor class is equal to your armor class, ignoring your armor modifier and natural armor modifier. Most touch attacks come from spells.

\subsection{Maneuver Class}
Your Maneuver Class (MC) represents how good you are at defending against combat maneuvers. It's the attack roll result that an opponent needs to achieve to affect you with a maneuver. Like your Armor Class, your MC is made up of the sum of other modifiers, each of which can have competence and enhancement bonuses applied separately. You can also apply competence and enhancement bonuses directly. Your MC is equal to the following:

\begin{figure}[h]
    \centering 10 \add base attack bonus \add Strength \add Dexterity \add shield modifier \add dodge modifier \add special size modifier \add competence bonuses \add enhancement bonus
\end{figure}

Your special size modifier is described in \tref{Size in Combat}.

\subsection{Hit Points}
Your hit points is a defense that represents how much more punishment you can take. When you run out hit points, you can't act anymore, and you might die, as described in \pcref{Injury and Death}.

\subsection{Saving Throws}
Generally, when you are subject to an unusual or magical attack, you get a saving throw to avoid or reduce the effect. You make a saving throw by rolling a d20 and adding your saving throw modifier. If it is greater than the save DC of the attack, you resist the attack (though it may still have partial effects on you). Your saving throw modifier is calculated as follows:

\begin{figure}[h]
\centering Base save bonus \add primary attribute \add 1/2 secondary attribute \add competence bonuses \add enhancement bonus
\end{figure}

\subsubsection{Saving Throw Types}
The three different kinds of saving throws are Fortitude, Reflex, and Will.
\parhead{Fortitude} These saves measure your ability to stand up to physical punishment or attacks against your vitality and health. Apply your Constitution and half your Strength to your Fortitude saving throws.
\parhead{Reflex} These saves test your ability to dodge area attacks. Apply your Dexterity and half your Wisdom to your Reflex saving throws.
\parhead{Will} These saves reflect your resistance to mental influence as well as many magical effects. Apply your Charisma and half your Intelligence to your Will saving throws.

\section{Attacks of Opportunity}
Sometimes a combatant in a melee lets her guard down. In this case, combatants near her can take advantage of her lapse in defense to attack her for free. These free attacks are called attacks of opportunity.

\subsection{Provoking Attacks of Opportunity}
You can provoke attacks of opportunity in two ways.

\parhead{Leaving an Opening} To defend yourself, you must be actively wielding a weapon or shield you are proficient with. If at any point during your turn, you can't use a weapon or shield to defend yourself, you provoke an attack of opportunity. This can happen for several reasons. Some examples are given below.

\begin{itemize*}
    \item If you are unarmed, you provoke.
    \item If you can't use your weapon or shield (such as if you are trying to wield a heavy weapon in one hand), you provoke.
    \item If you are busy doing something else (such as casting a spell), you provoke.
    \item If you can do nothing at all (such as if you are helpless), you provoke.
\end{itemize*}

\parhead{Leaving the Battle} You provoke an attack of opportunity if you move farther away from an opponent who threatens you. This can be mitigated with the withdraw and backpedal movement actions.

\subsection{Taking Attacks of Opportunity}
If a creature within your threatened area provokes an attack of opportunity, you can immediately make a single melee attack against that creature. The attack of opportunity ``interrupts'' anything else the creature was been doing. The creature can continue its turn after your attack of opportunity is resolved.

You can make a number of attacks of opportunity each round equal to 1 \add half your Dexterity, but never more than one per opportunity. You don't have to make an attack of opportunity if you don't want to.

\section{Movement and Positioning}

\subsection{Taking up Space}
A typical human takes up a 5-ft. by 5-ft. space in combat. For convenience, this is often called a ``square''. Differently sized creatures can take up more or less space, as indicated on \tref{Size in Combat}. Normally, other creatures can't be in any squares you occupy.

Sometimes, movement and distance are represented in squares. A 30-ft. movement is the same thing as moving six squares.

\subsection{Ways to Move}

\parhead{Move} As a move action, you can move up to your speed.
\parhead{Stand Up} As a move action, you can stand up from being prone. For most creatures, this requires using a hand to help get up.
\parhead{Backpedal} As a move action, you can move up to half your speed. This movement does not provoke attacks of opportunity from any creature you attacked during your turn.
\parhead{Charge} As a full-round action, you can move up to twice your speed and make a single attack with a \plus2 circumstance bonus to hit at the end of your charge. While charging, and until the start of your next turn, you take a \minus2 penalty to armor class.
\par You must move at least 30 feet to gain the benefit of a charge, and all movement must be in a single straight line. If there are any obstacles in your path which hinder your movement, you cannot charge. A charge that fails or becomes invalid partway through becomes a hustle action.
\parhead{Hustle} As a full-round action, you can move up to twice your speed.
\parhead{Run} As a full-round action, you can run at full speed. This allows you to move up to quadruple your speed. You move at only triple speed if you are wearing medium or heavy armor. While running, and until the start of your next turn, you are flat-footed against all attacks. 
\parhead{Withdraw} As a full-round action, you can move up to your speed. Before you do so, you can designate one creature who threatens you. This movement does not provoke attacks of opportunity from that creature.
\parhead{Struggle} As a full-round action, you can move five feet, regardless of movement penalties. You can use this to move even if your speed is decreased below five feet by penalties. 

\subsection{Measuring Movement}
\parhead{Diagonals} When measuring distance, the first diagonal counts
as five feet of movement, and the second counds as ten feet of movement. The third costs five feet, the fourth costs ten feet, and so on.

You can't move diagonally past a corner. You can move diagonally past a creature, even an opponent. You can also move diagonally past other impassable obstacles, such as pits.
\parhead{Closest Creature} When it's important to determine the closest square or creature to a location, if two squares or creatures are equally close, pick one randomly.

\subsection{Movement Impediments}

\subsubsection{Moving Through Occupied Squares}

\parhead{Ally} You can move through a square occupied by an ally. When you move through a square occupied by an ally, that character doesn't provide you with cover.
\parhead{Opponent} You can't move through a square occupied by an opponent, unless the opponent is helpless. You can move through a square occupied by a helpless opponent without penalty.
\parhead{Ending Your Movement} You can't end your movement in the same square as another creature unless it is helpless.
%Need actual limits
%\parhead{Very Small Creatures} Creatures that take up less than a square of space, as shown on \tref{Size in Combat}, can occupy squares occupied by other creatures of the same size.  
\parhead{Large and Small Creatures} Any creature can move through a square occupied by a creature three size categories larger than it is.

\subsubsection{Difficult Terrain}

Difficult terrain hampers movement. Each square of difficult terrain counts as 2 squares of movement. (Each diagonal move into a difficult terrain square counts as 3 squares.) You can't run or charge across difficult terrain.

If you occupy squares with different kinds of terrain, you can move only as fast as the most difficult terrain you occupy will allow.

\subsubsection{Obstacles}

If an obstacle hampers movement but doesn't completely block it, treat it as taking up an extra five feet of movement to bypass. Some obstacles may also require a skill check to cross.

\subsubsection{Squeezing}

In some cases, you may have to squeeze into or through an area that isn't as wide as the space you take up. You can squeeze through or into a space that is at least half as wide as your normal space. Each move into or through a narrow space counts as if it were 2 squares, and while squeezed in a narrow space you take a \minus4 penalty on attack rolls and a \minus4 penalty to AC.

When a Large creature (which normally takes up four squares) squeezes into a space that's one square wide, the creature's miniature figure occupies two squares, centered on the line between the two squares. For a bigger creature, center the creature likewise in the area it squeezes into.

A creature can squeeze past an opponent while moving but it can't end its movement in an occupied square.  To squeeze through or into a space less than half your space's width, you must use the Escape Artist skill.

\parhead{Accidentally Squeezing} Sometimes a character ends its movement while moving through a space where it's not normally allowed to stop. When that happens, the character is squeezing in the space until it can move. If squeezing is impossible, it immediately moves to the closest available space where it can be. Try not to do this.

\parhead{Double Movement Cost} When your movement is hampered in some way, your movement usually costs double. For example, each square of movement through difficult terrain counts as 2 squares, and each diagonal move through such terrain counts as 3 squares (just as two diagonal moves normally do).

If movement cost is doubled twice, then each square counts as 4 squares (or as 6 squares if moving diagonally). If movement cost is doubled three times, then each square counts as 8 squares (12 if diagonal), and so on. This is an exception to the general rule that two doublings are equivalent to a tripling.

\section{Initiative}

At the start of a battle, each creature that is aware of the combat makes an initiative check. Your initiative check is calculated as follows:

\begin{figure}[h]
    \centering Dexterity \add half Wisdom \add competence bonuses \add enhancement bonus
\end{figure}

Creatures act in order of their initiative, highest to lowest. After the creature with the lowest initiative acts, a new round begins and the creature with the highest initiative acts again. You usually keep your initiative for the whole encounter, even if you can't act. However, some special actions can change your initiative count.

\subsubsection{Surprise Attacks}
Sometimes, some creatures are not aware of the combat when it starts. This most commonly happens with ambushes. Any creature that is not aware of the combat doesn't get to make an initiative check during the first round of combat. Until it takes its first action, it is flat-footed and can't take attacks of opportunity.

Sometimes, everyone is surprised, such as if a guard walks around a corner to unexpectedly find a group of muggers. In that case, initiative is rolled normally, but all creatures are flat-footed and can't take attacks of opportunity until they take their first action.

\section{Injury, Death, and Healing}
Your hit points measure how hard you are to kill. No matter how many hit points you lose, your character isn't hindered in any way until your hit points drop to 0 or lower.

\subsection{Losing Hit Points}
When you take lethal damage, you subtract that damage from your hit points.
\parhead{What Hit Points Represent} Hit points represent a combination of durability, luck, divine providence, and sheer determination, depending on the nature of your character. When you take 10 damage from an orc with a greataxe, the axe did not literally carve into your skin without affecting your ability to fight. Instead, you avoided the worst of the blow, but it bruised you through your armor, the effort to dodge the blow fatigued your character, or it barely nicked you through sheer luck -- and everyone's luck runs out eventually.

\subsection{Stages of Injury}

\subsubsection{Healthy} 
When you are above half hit points, you suffer no significant effects from losing hit points. If you take damage, you can become bloodied.

\subsubsection{Bloodied}
When you drop to half your hit points or below, you are bloodied. This makes you more vulnerable to certain spells and effects, but you suffer no direct penalties. If you take additional damage, you can become staggered. Some special attacks can cause you to immediately begin dying.

\subsubsection{Staggered}
When you take damage that would reduce your hit points to 0, you become staggered.  While staggered, you may take a single move action or standard action each round, but not both. You cannot take full-round actions, but you may take swift actions. In addition, you are vulnerable, causing you to take a \minus2 penalty on attack rolls, saving throws, checks, DCs, and AC.

Normally, any excess damage from the attack that brought you to 0 hit points is wasted. However, if you take additional damage, you begin dying.

\subsubsection{Dying}
When you take damage while you have no hit points remaining, that damage represents serious physical injury to your body. This is called critical damage. When you take critical damage, you immediately fall unconscious and begin dying. While you have critical damage, magical healing which would normally restore hit points cannot restore your hit points, though it can prevent you from dying.

While dying, you must make a DC 15 Fortitude save every round. The DC is equal to 15 \add the critical damage you have taken. If you fail three times, you die. If you succeed three times, or receive healing that would normally hit points, you become stable. Another character can give first aid to help you stabilize (see \pcref{Heal}).

If you take additional damage, you can die. 

\subsubsection{Dead}
If you critical damage exceeds 10 \add your Constitution, you die. You can also die from taking ability damage or suffering ability damage or drain that reduces your Constitution to 0.

\subsubsection{Stable}
If you have taken critical damage but managed to stave off death, you become stable. You remain unconscious until your hit points exceed your critical damage.

Unless your hit points go above 0, you remain unconscious. As long as you have critical damage, magical healing that restores hit points has no effect on your hit points.

\subsection{Healing}
After taking damage, you can recover hit points through natural healing or through magical healing. In any case, you can't regain hit points past your full normal hit point total.

\parhead{Natural Healing} With 8 hours of rest, you recover half your hit points. Any significant interruption (such as combat or the like) during your rest prevents you from healing. If you rest for an entire day (16 hours), you recover all your hit points.

\parhead{Magical Healing} Various abilities and spells can restore hit points. However, only certain spells can heal critical damage, as specified in the spell description. Unless a spell says it can cure critical damage, it cannot -- though it can still stabilize dying characters. Magical healing has no effect on the hit points of creatures with critical damage.

\parhead{Healing Ability Damage} Ability damage is temporary, just as hit point damage is. Ability damage returns at the rate of 1 point per 8 hours of rest for each affected attribute score.

\parhead{Healing Critical Damage} Critical damage takes much longer to heal than hit point damage. Resting for 1 week restores an amount of critical damage equal to 1 \add half the character's Constitution. A character can have both hit points and critical damage. As long as a character has critical damage, he is staggered, even if he is at full hit points.

\subsection{Nonlethal Damage}
Some attacks and environmental effects deal nonlethal damage. Nonlethal damage is not subtracted from your hit points. Instead, it is tracked separately. If your nonlethal damage exceeds your hit points, you become staggered, just as if you were at 0 hit points. If you take additional damage while staggered, you fall unconscious. However, you do not begin dying unless your hit points are actually below 0. 

\subsubsection{Healing Nonlethal Damage}
You heal half your hit points in nonlethal damage with 1 hour of rest. When a spell or a magical ability cures hit point damage, it also removes an equal amount of nonlethal damage.

\section{Circumstances, Bonuses, and Penalties}

\begin{dtable}
\lcaption{Attack Roll Bonuses and Penalties}
\begin{tabularx}{\columnwidth}{l X}
\thead{Attacker's Condition} & \thead{Effect} \\
Entangled & \minus2 \\
Invisible & \x\fn{1} \\
Prone & \minus4\fn{2} & \\
Squeezing through a space & \minus4 \\
Vulnerable & \minus2 \\
\end{tabularx}
1 The defender is flat-footed. \\
2 Most ranged weapons can't be used while the attacker is prone, but you can use a crossbow or shuriken while prone at no penalty.
\end{dtable}

\begin{dtable}
\lcaption{Armor Class Bonuses and Penalties}
\begin{tabularx}{\columnwidth}{>{\lcol}X >{\ccol}X >{\ccol}X}
\thead{Defender's Condition} & \thead{Effect} & \thead{Ranged} \\
Behind active cover & 20\% miss \\
Behind passive cover & \plus4 \\
Blinded & \x\fn{1} \\
Concealed & \plus4 \\
Cowering & \minus2\fn{1} \\
Crouching or kneeling & \minus2\fn{2} \\
Entangled & \minus2 \\
Grappling (but attacker is not) & \plus0\fn{1} \\
Helpless (such as paralyzed, sleeping, or bound) & \plus0\fn{3} \\
Invisible & see Invisibility \\
Overwhelmed & special\fn{4} \\
Pinned & \minus4\fn{3} \\
Prone & \minus4\fn{2} \\
Squeezing through a space & \minus4 \\
Stunned & \minus2\fn{1} \\
Vulnerable & \minus2 \\
\end{tabularx}
1 The defender is flat-footed. \\
2 Treat as a bonus against ranged attacks, instead of a penalty \\
2 The defender is flat-footed, and treat the defender's Dexterity as \minus10. \\
3 The creature suffers a penalty equal to the number of creatures threatening it.
\end{dtable}

\subsection{Cover}

Cover represents any obstacle that physically prevents you from striking your target, such as a tree or intervening creature. A creature with cover is more difficult to attack.

\parhead{Determining Cover} When making a melee attack against an adjacent target, your target has cover if any line from your square to the target's square goes through a wall (including a low wall) or other similar solid obstacle. If you occupy multiple squares, choose one square you occupy for this purpose.

When making an attack against a target that is not adjacent to you, choose a corner of any square you occupy. In addition, choose a square the target occupies. If any line from this corner to any corner of the target square passes through a square or border that blocks line of effect or provides cover, or through a square occupied by a creature, the target has cover.

There are two types of cover: active cover and passive cover.

\subsubsection{Active Cover}

If the obstacle is active and mobile, such as a creature or tree branches blowing in the wind, the defender has active cover. Any attacks against a creature with active cover relative to you have a 20\% miss chance. After rolling the attack, the attacker must make a miss chance percentile roll to see if the attack misses due to active cover. If an attack misses due to active cover, the attack is made against the intervening obstacle instead. If the attack is successful, the obstacle takes any damage from the attack normally. 

\subsubsection{Passive Cover}

If the obstacle is stationary, such as a tree trunk or wall, the defender has passive cover. A creature with passive cover relative to you has a \plus4 circumstance bonus to armor class.

\parhead{Reflex Saves} A creature with passive cover gains a \plus2 bonus on Reflex saves against attacks that originate or burst out from a point on the other side of the cover from it. Note that spread effects can extend around corners and thus negate this cover bonus.
\parhead{Low Obstacles} A low obstacle (such as a wall no higher than half your height) provides passive cover, but only to creatures within 30 feet (6 squares) of it. The attacker can ignore the cover if he's closer to the obstacle than his target. If two creatures are equally distant from the wall, it grants cover to both of them.
\parhead{Attacks of Opportunity} You can't execute an attack of opportunity against an opponent with passive cover relative to you.
\parhead{Hide Checks} You can use passive cover to make a Hide check, but not active cover. Without cover, you usually need concealment (see below) to make a Hide check.
\parhead{Total Cover} If you don't have line of effect to your target, he is considered to have total cover from you. You can't make an attack against a target that has total cover.

\subsubsection{Improved Cover}

A creature can benefit from both passive and active cover. However, cover of the same type generally doesn't stack; a creature behind two trees is not substantially more protected than a creature behind a single tree. In some cases, cover may stack. In that case, each additional obstacle increases the miss chance by 10\% or grants an additional \plus2 circumstance bonus to AC, as appropritae.

Exceptionally well covered opponents, such as a creature behind an arrow slit in a castle, may receive additional benefits. For example, it might gain improved evasion, and there may be limitations on what kind of attacks are possible.

\subsection{Concealment}
Concealment represents anything which makes it more difficult to see your target, such as dim lighting. A creature with concealment from you gains a \plus4 circumstance bonus to Armor Class. Concealment bonuses do not apply if you can't see your opponent (such as if you close your eyes).

\parhead{Determining Concealment} When making a melee attack against an adjacent target, your target has concealment if his space is entirely within an effect that grants concealment.

Deterining concealment for making an attack against a target that is not adjacent to you works exactly like determining cover for ranged attacks.

In addition, some magical effects provide concealment against all attacks, regardless of whether any intervening concealment exists.

\parhead{Concealment and Hide Checks} You can use concealment to make a Hide check. Without concealment, you usually need cover to make a Hide check.

\subsection{Invisibility}
If it is impossible to see your target, you can't attack him normally. However, you can attack a square that you think he occupies. An attack into a square occupied by an invisible enemy has a 50\% miss chance. If an adjacent invisible creature strikes you, you can automatically identify the square he occupied when he struck you.

You can't execute an attack of opportunity against an invisible opponent, even if you know what square or squares the opponent occupies.

\subsection{Overwhelm}
When a creature is being attacked by multiple foes at once, it is less able to defend itself. A creature is considered overwhelmed if it is being threatened by more than one creature. Multiple creatures occupying the same square count as a single creature when determining overwhelm penalties. If a creature is overwhelmed, it takes a penalty to armor class equal to the number of creatures threatening it.

\subsection{Helpless Defenders}
A helpless opponent is someone who is bound, sleeping, paralyzed, unconscious, or otherwise at your mercy.

\parhead{Regular Attack} A helpless character takes a \minus4 penalty to AC against melee attacks, but no penalty to AC against ranged attacks. A helpless defender is flat-footed. In fact, his Dexterity score is treated as if it were \minus10, giving him a \minus10 penalty to AC.

\parhead{Coup de Grace} As a full-round action, you can use a melee weapon to deliver a coup de grace to a helpless opponent. You can also use a bow or crossbow, provided you are adjacent to the target. You automatically hit and score a critical hit. If the defender survives the damage, he must make a Fortitude save (DC 10 \add damage dealt) or die. A rogue also gets her extra sneak attack damage against a helpless opponent when delivering a coup de grace.

Delivering a coup de grace provokes attacks of opportunity from
threatening opponents because it involves focused concentration
and methodical action on the part of the attacker. If any attacks of opportunity hit, you must make a Concentration check (DC 10 \add total damage dealt from all attacks of opportunity) or else the coup de grace fails.

You can't deliver a coup de grace against a creature that is immune to critical hits. You can deliver a coup de grace against a creature with total concealment, but doing this requires two consecutive full-round actions (one to ``find" the creature once you've determined what square it's in, and one to deliver the coup de grace).

\subsection{Range Increments}
When using a ranged weapon, you take a \minus2 penalty per range increment between you and your target. For example, when using a longbow with a range increment of 100 feet against a target 170 feet away, you take a \minus2 penalty to attack rolls.

\section{Special Rules}

\subsection{Critical Success and Failure}
A natural 1 (the d20 comes up 1) on an attack roll or saving throw is treated as rolling a \minus10. A natural 20 (the d20 comes up 20) is treated as rolling a 30. Under normal circumstances, a natural 1 automatically misses, and a natural 20 automatically hits.

\subsubsection{Critical Hits}
When you roll a natural 20 on an attack roll and hit, you have scored a critical threat. Roll another attack roll at the same attack bonus. If that attack also hits, you deal double damage.

Many weapons can also score critical threats on lower numbers, or deal additional damage on a critical hit.

\subsection{Doubling}
If you double any in-game value twice, it becomes three times as large. An additional doubling would make it four times as large, and so on. For example, if you make an attack that deals double damage and you get a critical hit, you deal triple damage.

\parhead{Real-World Values} Values that are ``real'', such as movement and distance, are an exception. If you double a real-world value twice, it becomes four times as large. 

\subsection{Size in Combat}
Size affects your space and reach in combat. In addition, your attack modifier and armor class is affected by your size modifier. These effects are shown on \trefnp{Size in Combat}. 

\begin{dtable*}
    \lcaption{Size in Combat}
    \begin{tabularx}{\textwidth}{l l l l l X}
        \thead{Size} & \thead{Size Modifier} & \thead{Space\fn{1}} & thead{Reach\fn{1}} & \thead{Size Modifier\fn{2}} & \thead{Special Size Modifier\fn{3}} & \thead{Example Creature} \\
        Fine & 1/2 ft. & 0 & \plus8 & \minus16 & Fly\\
        Diminutive & 1 ft. & 0 & \plus4 & \minus12 & Toad \\
        Tiny & 2-1/2 ft. & 0 & \plus2 & \minus8 & Cat \\
        Small & 5 ft. & 5 ft. & \plus1 & \minus4 & Halfling \\
        Medium & 5 ft. & 5 ft. & \plus0 & \plus0 & Human \\
        Large (tall) & 10 ft. & 10 ft. & \minus1 & \plus4 & Ogre \\
        Large (tall) & 10 ft. & 5 ft. & \minus1 & \plus4 & Horse \\
        Huge (tall) & 15 ft. & 15 ft. & \minus2 & \plus8 & Cloud giant \\
        Huge (long) & 15 ft. & 10 ft. & \minus2 & \plus8 & Bulette \\
        Gargantuan (tall) & 20 ft. & 20 ft. & \minus4 & \plus12 & 50-ft. animated statue \\
        Gargantuan (long) & 20 ft. & 15 ft. & \minus4 & \plus12 & Kraken \\
        Colossal (tall) & 30 ft.\plus & 30 ft.\plus & \minus8 & \plus16 & Colossal animated object \\
        Colossal (long) & 30 ft.\plus & 20 ft.\plus & \minus8 & \plus16 & Great wyrm red dragon \\
    \end{tabularx}
    1 Creatures can vary in space and reach. These are simply typical values. \\
    2 Added to your attack modifier and Armor Class \\
    3 Added to your maneuver modifier and Maneuver Class \\
\end{dtable*}

Unusually large or small creatures also have other special rules apply to them. 

\subsubsection{Very Small Creatures}
\parhead{Space} If a creature takes up less than a single square of space, you can fit multiple creatures in that square. You can fit four Tiny creatures in a square, twenty-five Diminuitive creatures, or 100 Fine creatures.

\parhead{Reach} Creatures that take up less than 1 square of space typically have a natural reach of 0 feet, meaning they can't reach into adjacent squares. They must enter an opponent's square to attack in melee. This provokes an attack of opportunity from the opponent. You can attack into your own square if you need to, so you can attack such creatures normally. Since they have no natural reach, they do not threaten the squares around them, allowing you to move past them without provoking attacks of opportunity.

If a creature without a natural reach uses a reach weapon, it gains no benefits or penalties.

\subsubsection{Very Large Creatures}
\parhead{Space} Very large creatures take up multiple squares. Anything which affects a single square the creature occupies affects the creature. 

\parhead{Reach} Creatures that take up more than 1 square typically have a natural reach of 10 feet or more, meaning that they can reach targets even if they aren't in adjacent squares. Creatures with a large natural reach can attack anyone within their reach, including adjacent foes.

Creatures with a large natural reach using reach weapons can strike at up to double their natural reach but can't strike at their natural reach or less, just like Medium sized creatures.
