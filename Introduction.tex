\chapter{Introduction}

\section{What Is Rise?} 
Rise is a role-playing game. In Rise, you play a character of your own design. Your character can try to do anything you can imagine in a world that the game master, or GM, creates. Of course, you won't always succeed. Your character's abilities are defined in the pages ahead; when you're done creating a character, it will have a personality of its own, along with strengths, weaknesses, and special abilities. Usually, your character will go on adventures with other characters, and together, you will create and experience a story with the game master.

\section{How To Do Things}
If you want your character to do something in Rise, just tell the game master what you want your character to do. Sometimes, you will have to wait until it's your turn. When your character tries to do something that has a chance of failure, you will roll a twenty-sided die, or ``d20''. You'll add some modifiers to that roll based on your character's abilities, and if the result is at least as high as the Difficulty Class (DC) of the action, your character will have succeeded at whatever he was trying to do.

The DC to perform an action depends on what how difficult your action is, as determined by the GM. This book contains guidelines on how difficult various things are, but they are only guidelines, and the GM always has the final say: they may know something you don't!

\subsection{Opposed Checks}
Sometimes, you're competing with another creature. In that case, you both roll and add your modifiers, and the creature with the higher result wins. This is called an opposed check. If you get the same result, the creature with the higher modifier wins. If you have the same modifier, roll again to break the tie!

\subsection{Actions}
Normally, you can just say that your character is going to do something, and the GM will take care of what happens and when. Sometimes, particularly during combat, it's important to keep track of exactly what order things are happening in. When that happens, your character will get a ``turn'', where she can take actions.

\subsubsection{Common Actions}
\parhead{Standard Action} A standard action is the most common type of action. You can use a standard action to attack with a weapon, cast a spell, drink a potion, and do most other things that take concentration and effort. You can take one standard action each turn.
\parhead{Move Action} A move action is usually used to move from one place to another. You can move a distance up to your base land speed with a move action. It can also be used for other activities that require some effort, such as drawing a weapon, opening a door, or standing up from a prone position. You can take one move action each turn.
\parhead{Full-Round Action} A full-round action requires your character's full attention. Most full-round actions involve a combination of movement and concentrated effort, such as charging to strike a distant foe or running at full speed. You can take a full-round action in place of both your standard and move actions for the turn.

\subsubsection{Uncommon Actions}
\parhead{Swift Action} A swift action represents a very brief moment of concentration on something. Most swift actions are related to magic in some way. You can take one swift action each turn.
\parhead{Immediate Action} An immediate action is special. Unlike other actions, you can take an immediate action when it's not your turn, even in the middle of another creature's action. The most common immediate action is to drop prone. If you take an immediate action, you don't get a swift action on your next turn. You can't take more than one immediate action between each of your turns.

\subsubsection{Actions On Your Turn}
When it's your turn, you can take one standard action, one move action, and one swift action. If you want, you can take a full-round action instead of your standard and move actions.

\parhead{Downgrading Actions} You can always ``downgrade'' an action to a lesser action: turning a standard action into a move action, or a move action into a swift action. If you really wanted to, you could take three swift actions on your turn, but that's usually not a good idea.

\subsection{Rounding}
In general, if you encounter a fractional number, you round it down.

\subsection{Multipliers}
Sometimes a rule makes you multiply a number or a die roll. As long as you�re applying a single multiplier, multiply the number normally. When two or more multipliers apply to any abstract value (such as a modifier or a die roll), however, combine them into a single multiple, with each extra multiple adding 1 less than its value to the first multiple. Thus, a double (\mult2) and a double (\mult2) applied to the same number results in a triple (\mult3, because 2 \add 1 = 3).

When applying multipliers to real-world values (such as weight or distance), normal rules of math apply instead. A creature whose size doubles (thus multiplying its weight by 8) and then is turned to stone (which would multiply its weight by a factor of roughly 3) now weighs about 24 times normal, not 10 times normal. Similarly, a blinded creature attempting to negotiate difficult terrain would need to spend 20 feet of movement to move 5 feet (doubling the cost twice, for a total multiplier of \mult4), rather than as 15 feet (adding 100\% twice).

\section{Principles of Rise}

\begin{itemize}
    \itemhead{The Game Master controls the world.} Everything about the game world is up to the GM: the people your character meets, the challenges she will overcome, and the very ground under her feet. Even the ``rules'' of the game are completely subject to the GM's whim.
    \itemhead{You control your character.} A GM should never tell you how your character feels or what he tries to do -- unless, of course, your character is being controlled by hostile magic or some other power.
    \itemhead{Respect and trust is critical.} The GM has a great deal of power in Rise, and the players have to trust that the GM knows what she is doing. Likewise, the GM needs to let the players do what they want -- even if it doesn't suit his idea of what ``should'' happen. Some of the most memorable events happen when players do things that are totally unexpected.
    \itemhead{Everything is flexible.} Rise contains hundreds of pages of rules. We think they're pretty good rules. But sometimes, you don't need them all -- or you think you've come up with something better. Do whatever works for you and your group.
    \itemhead{Do what makes sense.} This book doesn't contain rules for how to drink water, sleep, or blink. If something isn't described explicitly here, assume that it works the same way it does in reality.
    \itemhead{It's just a game, so have fun.}
\end{itemize}
