\section{Spell Descriptions}

\pdfbookmark[2]{A}{SpellDescriptionsA}
\begin{comment}
\subsubsection{A}
\end{comment}

\spellsection{Acid Fog}
\spelldesc{A billowing mass of acidic vapors fills the area, slowing creatures down and obscuring sight.}
\spellschool{Conjuration (Creation) [Acid]}
\spelllvl{Sor/Wiz 8, Destruction 8}
\spellrng{\rngmed}
\spellarea{\areamed radius spread}
\spelleff{Fog in the area}
\spelldur{\durshort}
\spellsave{Fortitude half}
\spellsr{No}
\spelldmg{3d6 acid damage per round}
\begin{spelleffect}
This spell functions like \spell{solid fog}, except that the spell's vapors are highly acidic, dealing damage to All creatures and objects within the area on each round at the start of your turn. The fog does not do damage in the round it is cast. A successful Fortitude save halves the damage.
\end{spelleffect}

\spellsection{Attraction}
\spelldesc{You cause the subject to feel attracted to something.}
\spellschool{Enchantment (Emotion) [Mind-Affecting]}
\spelllvl{Bard 1, Sor/Wiz 1}
\spellrng{\rngmed}
\spelltgt{One creature}
\spelldur{\durext}
\spellsave{Will negates}
\spellsr{Yes (Will)}
\begin{spelleffect}
An affected creature feels attracted to a particular person or object. The subject will take reasonable steps to meet, get close to, attend, or find the object of its implanted attraction. For the purpose of this spell, ``reasonable'' means that, while attracted, the subject doesn't suffer from blind obsession. He will act on this attraction only when not engaged in combat. The subject won't perform obviously suicidal actions. He can still recognize danger but will not flee unless the threat is immediate. If you make the subject feel an attraction to yourself, you can't command him indiscriminately, although he will be willing to listen to you (even if he disagrees).
\par This spell grants you a \plus4 circumstance bonus on any social interaction checks you make involving the subject (such as Bluff, Diplomacy, Intimidate, and Sense Motive).
\end{spelleffect}

\spellsection{Aversion}
\spelldesc{You make the subject want to avoid something.}
\spellschool{Enchantment (Emotion) [Mind-Affecting]}
\spelllvl{Bard 2, Sor/Wiz 2}
\spellrng{\rngmed}
\spelltgt{One creature}
\spelldur{\durext}
\spellsave{Will negates}
\spellsr{Yes (Will)}
\begin{spelleffect}
An affected creature feels an aversion to a particular person or object. If the object of the implanted aversion is an individual or a physical object, she will prefer not to approach within 30 feet of it. If it is a word, she will try not to utter it; if it is an action, she will not willingly attempt to perform it; and if it is an event, she will not willingly attend it. The subject will take reasonable steps to avoid the object of its aversion, but will not put herself in jeopardy by doing so.
\par If the subject is forced into taking an action she has an aversion to, she is bewildered as long as she is performing the action.
\end{spelleffect}

\pdfbookmark[2]{B}{SpellDescriptionsB}
\begin{comment}
\subsubsection{B}
\end{comment}

\spellsection{Backbiter}
\spelldesc{You subtly animate a weapon so that it strikes its wielder instead of its intended target.}
\spellschool{Transmutation (Animation)}
\spelllvl{Trans 1}
\spellrng{\rngmed}
\spelltgt{One weapon}
\spelldur{\durshort or until discharged}
\spellsave{Will negates (object)}
\spellsr{Yes (Will)}
\begin{spelleffect}
The next time the affected weapon is used to make a melee attack, it twists around so that the weapon automatically strikes the wielder instead. The wielder gets no warning or knowledge of the spell's effect on his weapon, and though he makes the attack, the self-dealt damage can't be consciously reduced (though damage reduction applies) or changed to nonlethal damage.
\par Once the weapon attacks its wielder (whether successfully or not), the spell is discharged.
\end{spelleffect}

\spellsection{Baleful Polymorph}
\spelldesc{You transmute your foe into a small, insignificant animal.}
\spellschool{Transmutation (Polymorph)}
\spelllvl{Drd 5, Trans 5}
\spellrng{Touch}
\spelltgt{Creature touched}
\spelldur{\durshort or Permanent (D); see text}
\spellsave{Fortitude negates, Will partial; see text}
\spellsr{Yes (Fortitude)}
\begin{spellhealthy}
The subject is sickened.
\end{spellhealthy}
\begin{spellblood}
You change the subject into a Small or smaller animal of no more than 1 HD (such as a dog, lizard, monkey, or toad). The subject takes on all the statistics and special abilities of an average member of the new form in place of its own except as follows:
\begin{itemize*} 
\item The target retains its own alignment, personality, and mental attribute scores.
\item If the target has the shapechanger subtype, it retains that subtype. 
\item The target retains its own hit points. 
\item The target is treated has having its normal Hit Values for purpose of adjudicating effects based on HV, though it uses the new form's base attack bonus, base save bonuses, and all other statistics derived from Hit Values. 
\item The target also retains the ability to understand (but not to speak) the languages it understood in its original form. It can write in the languages it understands, but only the form is capable of writing in some manner (such as drawing in the dirt with a paw). 
\end{itemize*}
With those exceptions, the target's normal game statistics are replaced by those of the new form. The target loses all the special abilities it has in its normal form, including its class features.

All items worn or carried by the subject fall to the ground at its feet, even if they could be worn or carried by the new form. 

If the new form would prove fatal to the creature (for example, if you polymorphed a landbound target into a fish, or an airborne target into a toad), the subject gets a \plus4 bonus on the save. 

If the spell succeeds, the subject must also make a Will save. If the second save fails, the transformation is permanent. Otherwise, the creature reverts to its true form after the \durshort duration.

If the subject remains in the new form for 24 consecutive hours, it must attempt another Will save. If this save fails, it loses its ability to understand language, as well as all other memories of its previous form, and its Hit Values, hit points, and mental attribute scores change to match an average creature of its new form. These abilities and statistics return to normal if the effect is later ended. The subject must repeat this save every 24 hours that it remains in its new form. 
\end{spellblood}
\begin{spellnotes}
Incorporeal or gaseous creatures are immune to \spell{baleful polymorph}, and a creature with the shapechanger subtype (such as a lycanthrope or a doppelganger) can revert to its natural form as a standard action (which ends the spell's effect). 
\end{spellnotes}

\spellsection{Bane}
\spelldesc{You fill your enemies with dismay, impairing their ability to fight.}
\spellschool{Enchantment (Emotion) [Mind-Affecting, Morale]}
\spelllvl{Clr 1, Evil 1}
\spellarea{\areamed radius emanation centered on you}
\spelldur{\durshort}
\spellsave{None}
\spellsr{Yes (Will)}
\begin{spelleffect}
All enemies within the area take a \minus2 penalty to attack rolls. This penalty lasts as long as they remain in the area.
\end{spelleffect}
\begin{spellnotes}
\spell{Bane} counters and dispels \spell{bless}.
\end{spellnotes}

\spellsection{Banishment}
\spelldesc{You force extraplanar creatures back to their home plane.}
\spellschool{Abjuration/Conjuration (Interdiction, Translocation) [Planar]}
\spelllvl{Clr 6, Sor/Wiz 6}
\spellcmp{V, S, F}
\spellrng{\rngmed}
\spelltgts{One extraplanar creature/round}
\spelldur{Concentration (up to 5 minutes)}
\begin{spelleffect}
This spell functions like \spell{dismissal}, except that you can banish one additional extraplanar creature each round that you concentrate on the spell. An individual creature can only be targeted once per casting of this spell.
\end{spelleffect}
\begin{spellnotes}
You can improve the spell's chance of success by presenting at least one object or substance that the target hates, fears, or otherwise opposes. For each such object or substance, you gain a \plus2 circumstance bonus on your caster level with the spell. For example, if this spell were cast on a demon that hated light and was vulnerable to holy water and cold iron weapons, you might use iron, holy water, and a torch in the spell. The three items would give you a \plus6 bonus on your caster level. 
\par Certain rare items might work twice as well as a normal item for the purpose of the bonuses (each providing a \plus4 circumstance bonus to your caster level).
\end{spellnotes}
\spellfocus{Any item that is distasteful to the subject (optional, see above)}

\spellsection{Bestow Curse}
\spelldesc{You place a curse on your foe, crippling its ability to act.}
\spellschool{Necromancy [Curse]}
\spelllvl{Clr 3, Evil 3, Sor/Wiz 3}
\spellrng{\rngtouch}
\spelltgt{Creature touched}
\spelldur{Permanent}
\spellsave{Will negates}
\spellsr{Yes (Will)}
\begin{spelleffect}
The subject suffers one of the following three effects, chosen by you:
\begin{itemize*}
\item \minus6 decrease to an attribute score (minimum 1).
\item \minus4 penalty on attack rolls, saves, checks, and armor class.
\item Each turn, the target has a 25\% chance to act normally; otherwise, it takes no action.
\end{itemize*}
\par You may also invent your own curse, but it should be no more powerful than those described above.
\end{spelleffect}
\begin{spellnotes}
Curses cannot be dispelled. \spell{Bestow curse} counters \spell{remove curse}.
\end{spellnotes}

\spellsection{Black Tentacles}
\spelldesc{You conjure a field of rubbery black tentacles, each 5 feet long. These waving members seem to spring forth from the earth, floor, or whatever surface is underfoot -- including water. They grasp and entwine around creatures that enter the area, holding them fast and crushing them with great strength.}
\spellschool{Conjuration/Transmutation (Animation, Creation)}
\spelllvl{Sor/Wiz 6}
\spellrng{\rngclose}
\spellarea{\areamed radius spread}
\spelleff{Black tentacles in the area}
\spelldur{\durshort (D)}
\spellsave{None}
\spellsr{No}
\begin{spelleffect}
At the start of your turn, every creature within the area of the spell is the target of a grapple attack. This attack is also made in the round that \spell{black tentacles} is cast. Treat the tentacles attacking a particular target as a Medium creature with a base attack bonus equal to your caster level and a Strength score of 18. Thus, its grapple attack modifier is equal to your caster level \plus8. Roll only once for the entire spell effect each round and apply the result to All creatures within the area of effect.
\par If the tentacles succeed in grappling a foe, that foe takes 1d8\plus4 damage. Each round that black tentacles succeeds on a grapple attack, it deals an additional 1d8\plus4 damage. The tentacles continue to crush the opponent until the spell ends or the opponent escapes.
\par The tentacles are immune to all types of damage. The entire area of effect is considered difficult terrain while the tentacles last, and any creature that enters the area of the spell is immediately attacked by the tentacles.
\end{spelleffect}

\spellsection{Blasphemy}
\spelldesc{You speak an unholy utterance of great power, afflicting all those nearby who do not share your allegiance to evil.}
\spellschool{Evocation (Channeling) [Evil]}
\spelllvl{Evil 7}
\spellcmp{V}
\spellarea{\arealarge radius spread centered on you}
\spelldur{Instantaneous}
\spellsave{None}
\spellsr{Yes (Will)}
\begin{spellhealthy}
\par Each nonevil creature in the area is sickened for 5 rounds.
\end{spellhealthy}
\begin{spellblood}
\par Each nonevil creature in the area suffers one or more of the following ill effects, depending on its Hit Values.
\begin{dtable}
\begin{tabularx}{\columnwidth}{l >{\lcol}X}
\par \thead{HV} & \thead{Effect} \\
\par Equal to caster level & Sickened \\
\par Up to caster level \minus5 & Nauseated, sickened \\
\par Up to caster level \minus10 & Paralyzed, nauseated, sickened \\
\par Up to caster level \minus15 & Killed\fn{1}
\end{tabularx}
1 Living creatures die. Nonliving creatures are destroyed.
\end{dtable}
\par \subspell{Sickened} The creature is sickened for 5 rounds.
\par \subspell{Nausated} The creature is nauseated for 1 round.
\par \subspell{Paralyzed} The creature is paralyzed and helpless for 5 rounds.
\par \subspell{Killed} Living creatures die. Nonliving creatures are destroyed.
\end{spellblood}
\begin{spellnotes}
Creatures whose Hit Values exceed your caster level are unaffected by \spell{blasphemy}.
\end{spellnotes}

\spellsection{Blindness/Deafness}
\spelldesc{You afflict one of the subject's senses.}
\spellschool{Necromancy (Flesh)}
\spelllvl{Clr 3, Sor/Wiz 3}
\spellcmp{V}
\spellrng{Touch}
\spelltgt{Living creature touched}
\spelldur{\durlong (D)}
\spellsave{Fortitude negates}
\spellsr{Yes (Fortitude)}
\begin{spellhealthy}
The subject is sickened.
\end{spellhealthy}
\begin{spellblood}
The subject is blinded, deafened, or sickened, as you choose.
\end{spellblood}
\begin{spellnotes}
The choice of bloodied conditions is made at the time the spell is cast.
\end{spellnotes}

\spellsection{Calm Emotions}
\spelldesc{You calm a group of creatures, preventing the situation from getting out of hand.}
\spellschool{Enchantment (Emotion) [Mind-Affecting]}
\spelllvl{Brd 2}
\spellrng{\rngmed}
\spellarea{\areamed radius spread}
\spelldur{Concentration}
\spellsave{Will negates}
\spellsr{Yes (Will)}
\begin{spelleffect}
Creatures in the area have their emotions calmed. Creatures so affected cannot take violent actions (although they can defend themselves) or do anything destructive.
\end{spelleffect}
\begin{spellnotes}
Any aggressive action against or damage dealt to a calmed creature immediately breaks the spell on all calmed creatures.

This spell automatically suppresses (but does not dispel) any effects of spells or abilities that affect or require emotions, including all other enchantment (emotion) spells.
\end{spellnotes}

\spellsection{Cause Fear}
\spelldesc{You fill your enemy with fear.}
\spellschool{Enchantment (Emotion) [Fear, Mind-Affecting]}
\spelllvl{Brd 1, Clr 1, Sor/Wiz 1}
\spellrng{\rngclose}
\spelltgt{One creature}
\spelldur{\durshort/1 round (D)}
\spellsave{Will negates}
\spellsr{Yes (Will)}
\begin{spellhealthy}
The subject is shaken.
\end{spellhealthy}
\begin{spellblood}
As the healthy effect, plus the subject is frightened for 1 round.
\end{spellblood}

\spellsection{Charm Monster}
\spelldesc{You manipulate a creature's mind so it thinks of you as a trusted friend and ally.}
\spellschool{Enchantment (Emotion) [Charm, Mind-Affecting]}
\spelllvl{Brd 4, Ench 5}
\spelldur{\durlong}
\spelltgt{One living creature}
\begin{spelleffect}
This spell functions like \spell{charm person}, except that the effect is not restricted by creature type and has a shorter duration.
\end{spelleffect}

\spellsection{Charm Monster}
\spelldesc{You manipulate a creature's mind so it thinks of you as a trusted friend and ally.}
\spellschool{Enchantment (Emotion) [Charm, Mind-Affecting]}
\spelllvl{Brd 4, Ench 5}
\spelldur{\durlong}
\spelltgt{One living creature}
\begin{spelleffect}
This spell functions like \spell{charm person}, except that the effect is not restricted by creature type and has a shorter duration.
\end{spelleffect}

\spellsection{Charm Person}
\spelldesc{You manipulate a person's mind so he thinks of you as a trusted friend and ally.}
\spellschool{Enchantment (Emotion) [Charm, Mind-Affecting]}
\spelllvl{Brd 2, Ench 2}
\spellrng{\rngmed}
\spelltgt{One humanoid creature}
\spelldur{\durlong}
\spellsave{Will negates}
\spellsr{Yes (Will)}
\begin{spelleffect}
This charm makes a humanoid creature regard you as its trusted friend and ally. If it is currently faced with any obvious threat from you or your allies, such as someone drawing a weapon, casting a spell, or aiming a ranged weapon at the creature, it receives a \plus5 circumstance bonus on its saving throw.
\par The spell does not enable you to control the subject as if it were an automaton, but it perceives your words and actions in the most favorable way. You can try to give the subject orders, but you must succeed at a Diplomacy check to convince it to do anything it wouldn't ordinarily do. (Retries are not allowed.) Treat the target as a friend (a \plus10 relationship modifier) for the purpose of the Diplomacy check. An affected creature never obeys suicidal or obviously harmful orders, but it might be convinced that something very dangerous is worth doing.
\end{spelleffect}
\begin{spellnotes}
Any act by you or your apparent allies that threatens the \spell{charmed} person breaks the spell. A creature that makes its saving throw against \spell{charm person} is immune to all further attempts by the same spellcaster for 24 hours.
\end{spellnotes}

\spellsectioncomma{Charm Person}{Mass}
\spelldesc{You manipulate the minds of many people so they think of you as a trusted friend and ally.}
\spellschool{Enchantment (Emotion) [Charm, Mind-Affecting]}
\spelllvl{Brd 6, Ench 7}
\spellrng{\rnglong}
\spellarea{\areamed radius}
\spelltgts{Five humanoid creatures within the area}
\begin{spelleffect}
This spell functions like \spell{charm person}, except that it affects multiple creatures at a longer range.
\end{spelleffect}

\spellsection{Circle of Death}
\spelldesc{You snuff out the life force of your weakened foes by flooding them with negative energy.}
\spellschool{Necromancy (Vitalism) [Death, Negative]}
\spelllvl{Death 6, Sor/Wiz 6}
\spellcmp{V, S, M}
\spellrng{\rngmed}
\spellarea{\areamed radius limit}
\spelltgts{Several living creatures within the area}
\spelldur{Instantaneous}
\spellsave{Fortitude negates}
\spellsr{Yes (Fortitude)}
\begin{spellblood}
The subjects immediately die.
\end{spellblood}
\begin{spellnotes}
This spell can affect 2 HV worth of living creatures per caster level. Creatures with the fewest HV are affected first; among creatures with equal HV, those who are closest to the burst's point of origin are affected first. No creature of more HV than half your caster level can be affected, and Hit Values that are not sufficient to affect a creature are wasted. Healthy creatures are not affected by the spell, and do not count against the spell's HV limit.
\end{spellnotes}
\spellmat{The powder of a crushed black pearl with a minimum value of 750 gp.}

\spellsection{Cloudkill}
\spelldesc{You conjure a yellowish green fog bank that obscures vision and slowly poisons creatures inside.}
\spellschool{Conjuration (Creation) [Fog, Poison]}
\spelllvl{Sor/Wiz 7}
\spellsave{None/Fortitude negates}
\begin{spelleffect}
This spell functions like \spell{fog cloud}, except that living creatures inside the fog take 1d4 Constitution damage on your turn each round. A successful Fortitude save negates the damage for that round.
\par Unlike a \spell{fog cloud}, the \spell{cloudkill} moves away from you at 10 feet per round, rolling along the surface of the ground. Figure out the cloud's new spread each round based on its new point of origin, which is 10 feet farther away from the point of origin where you cast the spell.
\end{spelleffect}
\begin{spellnotes}
Holding one's breath doesn't help against the poison, but creatures immune to poison are unaffected.
\par Because the vapors are heavier than air, they sink to the lowest level of the land, even pouring down den or sinkhole openings. It cannot penetrate liquids, nor can it be cast underwater.
\end{spellnotes}

\spellsection{Color Spray}
\spelldesc{You project a vivid cone of clashing colors from your outstretched hand, striking creatures in front of you.}
\spellschool{Illusion (Figment) [Light]}
\spelllvl{Sor/Wiz 1}
\spellarea{\areamed cone-shaped burst}
\spelldur{2d4 rounds/1 round}
\spellsave{Will partial}
\spellsr{Yes (Will)}
\begin{spelleffect}
Creatures in the area are dazzled. A successful Will save reduces the duration to 1 round.
\end{spelleffect}
\begin{spellnotes}
A dazzled creature treats everything it sees as if it had concealment (20\% miss chance), and takes a \minus4 penalty to Spot checks. It is unable to use darkvision.

Creatures who cannot see the light are not affected by this spell. Merely closing one's eyes is insufficient protection.
\end{spellnotes}

\spellsection{Command}
\spelldesc{You compel a foe to obey a single command of your choice.}
\spellschool{Enchantment (Compulsion) [Language-Dependent, Mind-Affecting]}
\spelllvl{Clr 1, Law 1}
\spellcmp{V}
\spellrng{\rngmed}
\spelltgt{One living creature}
\spelldur{1 round}
\spellsave{Will negates}
\spellsr{Yes (Will)}
\begin{spellhealthy}
The subject is bewildered.
\end{spellhealthy}
\begin{spellblood}
The subject must perform one of the following actions of your choice.
\par \subspell{Approach} On its turn, the subject moves toward you as quickly and directly as possible. The creature may do nothing but move during its turn, and it provokes attacks of opportunity for this movement as normal.
\par \subspell{Drop} As soon as possible, the subject drops whatever it is holding. It may act normally on its turn, except that it can't pick up any dropped items.
\par \subspell{Fall} As soon as possible, the subject falls to the ground. It may act normally on its turn, except that it can't get up from its prone position.
\par \subspell{Flee} On its turn, the subject moves away from you as quickly as possible. It may do nothing but move during its turn, and it provokes attacks of opportunity for this movement as normal.
\par \subspell{Halt} On its turn, the subject can take no actions, but it can defend itself normally.
\end{spellblood}
\begin{spellnotes}
If the subject can't understand or carry out your command, the spell automatically fails.
\end{spellnotes}

\spellsectioncomma{Command}{Mass}
\spelldesc{You compel many foes to obey your command.}
\spellschool{Enchantment (Compulsion) [Language-Dependent, Mind-Affecting]}
\spelllvl{Clr 5}
\spellarea{\areamed radius limit}
\spelltgts{Five creatures within the area}
\begin{spelleffect}
This spell functions like \spell{command}, except that it affects multiple creatures.
\end{spelleffect}

\spellsection{Confusion}
\spelldesc{You compel a creature to act randomly, sowing confusion in your foes' ranks.}
\spellschool{Enchantment (Compulsion) [Mind-Affecting]}
\spelllvl{Brd 3, Chaos 3, Sor/Wiz 3, Trickery 3}
\spellrng{\rngclose}
\spelltgt{One creature}
\spelldur{\durshort}
\spellsave{Will negates}
\spellsr{Yes (Will)}
\begin{spellhealthy}
The subject is bewildered.
\end{spellhealthy}
\begin{spellblood}
The subject is confused. \confusionexplanation
\end{spellblood}
\begin{spellnotes}
A bewildered creature takes a \minus2 penalty to attack rolls, saving throws, checks, DCs, and AC.
\par Attackers are not at any special advantage when attacking a \spell{confused} character. A \spell{confused} character will not make attacks of opportunity against any creature that it is not already devoted to attacking (either because of its most recent action or because it has just been attacked).
\end{spellnotes}

\spellsectioncomma{Confusion}{Mass}
\spelldesc{You compel a group of creatures to act randomly, sowing confusion in your foes' ranks.}
\spellschool{Enchantment (Compulsion) [Mind-Affecting]}
\spelllvl{Brd 6, Sor/Wiz 8, Trickery 8}
\spellrng{\rngmed}
\spellarea{\areamed radius limit}
\spelltgts{All creatures within the area, to a maximum of five creatures}
\begin{spelleffect}
This spell functions like \spell{confusion}, except that it affects multiple creatures. If there are more creatures in the area than you can affect, randomly determine which creatures are affected.
\end{spelleffect}

\spellsection{Contagion}
\spelldesc{You infect your foe with a contagious disease, knowing he will inadvertently spread it to his allies when he escapes.}
\spellschool{Necromancy (Flesh) [Disease]}
\spelllvl{Clr 3, Death 3, Destruction 3, Drd 3, Sor/Wiz 3}
\spellrng{\rngtouch}
\spelltgt{Living creature touched}
\spelldur{Instantaneous}
\spellsave{Fortitude negates}
\spellsr{Yes (Fortitude)}
\begin{spelleffect}
The subject contracts a disease selected from the table below, which strikes immediately (no incubation period). The DC for both the initial and subsequent saving throws is equal to this spell's save DC.
\begin{dtable}
\begin{tabularx}{\columnwidth}{l X}
\thead{Disease} & \thead{Damage} \\
Blinding sickness & 1d4 Str\footnotetemp{1} \\
Cackle fever & 1d6 Wis \\
Filth fever & 1d3 Dex and 1d3 Con \\
Mindfire & 1d4 Int \\
Red ache & 1d6 Str \\
Shakes & 1d8 Dex \\
Slimy doom & 1d4 Con
\end{tabularx}
1 Each time a victim takes 2 or more Strength damage from blinding sickness, he or she must make another Fortitude save or be permanently blinded.	 
\end{dtable}
\end{spelleffect}

\spellsection{Creeping Doom}
\spelldesc{You summon uncountable hordes of centipedes to overwhelm your foes.}
\spellschool{Conjuration (Summoning)}
\spelllvl{Drd 7}
\spelltime{Full-round action}
\spellrng{\rngclose; see text}
\spelleff{One swarm of centipedes per two levels}
\spelldur{\durmed}
\spellsave{None}
\spellsr{No}
\begin{spelleffect}
This spell creates one centipede swarm per two caster levels. They must all be adjacent at least one other swarm. You may summon the centipede swarms so that they share the area of other creatures. The swarms remain stationary, attacking any creatures in their area, unless you command the creeping doom to move (a standard action). As a standard action, you can command any number of the swarms to move toward any prey within \rngmed range of you. You cannot command any swarm to move more than \rngmed range away from you, and if you exceed that distance, the swarm remains stationary, attacking any creatures in its area (but it can be commanded again if you move within range).
\end{spelleffect}

\spellsection{Crushing Despair}
\spelldesc{You fill a number of creatures with sadness and gloom.}
\spellschool{Enchantment (Emotion) [Mind-Affecting]}
\spelllvl{Brd 2, Sor/Wiz 2}
\spellarea{\areamed cone-shaped burst}
\spelldur{\durmed}
\spellsave{None}
\spellsr{Yes (Will)}
\begin{spelleffect}
Each creature in the area is demoralized. Demoralized creatures take a \minus2 penalty on attack rolls, saving throws, checks, and armor class.
\end{spelleffect}
\begin{spellnotes}
\spellindirect{crushing despair}{Crushing despair} counters and dispels \spell{good hope}.
\end{spellnotes}

\spellsectioncomma{Cure Serious Wounds}{Mass}
\spelldesc{You stretch out your hand and channel positive energy into all of your allies, healing even serious injuries.}
\spellschool{Necromancy (Life) [Healing, Positive]}
\spelllvl{Clr 7, Drd 7, Life 7}
\spellheal{7d6 damage \add d6 per four caster levels above 14th}
\begin{spelleffect}
This spell functions like \spell{mass cure light wounds}, except that for every 15 points of healing granted by the spell, it can instead cure 1 point of critical damage.
\end{spelleffect}

\pdfbookmark[2]{D}{SpellDescriptionsD}
\begin{comment}
\subsubsection{D}
\end{comment}

\spellsection{Darkness}
\spellschool{Illusion (Glamer) [Darkness]}
\spelllvl{Brd 2, Sor/Wiz 2, Trickery 2}
\spellcmp{V}
\spellrng{\rngtouch}
\spelltgt{Object touched}
\spelldur{\durmed (D)}
\spellsave{None}
\spellsr{No}
\begin{spelleffect}
This spell causes an object to radiate shadowy illumination out to a \areamed radius. This causes the level of illumination to drop to shadowy illumination or the current prevailing condition, whichever is lower. Darkvision is ineffective in magical darkness, and confers no advantage over normal vision.
\end{spelleffect}
\begin{spellnotes}
\par If \spell{darkness} is cast on a small object that is then placed inside or under a lightproof covering, the spell's effect is blocked until the covering is removed.

Normal lights (torches, candles, lanterns, and so forth) are incapable of brightening the area or shining through it, as are light spells of lower level. Such effects are also suppressed if they originate from within the area of the darkness, preventing them from shining light elsewhere. Higher level light spells are not affected by darkness.

\par \spell{Darkness} counters or dispels any light spell of equal or lower spell level.
\end{spellnotes}

\spellsection{Daze}
\spelldesc{You cloud the mind of your foe, preventing it from taking any actions.}
\spellschool{Enchantment (Inhibition) [Mind-Affecting]}
\spelllvl{Brd 3, Sor/Wiz 3}
\spellrng{\rngclose}
\spelltgt{One creature}
\spelldur{\durshort}
\spellsave{None/Will negates}
\spellsr{Yes (Will)}
\begin{spellhealthy}
The subject is bewildered.
\end{spellhealthy}
\begin{spellblood}
As the healthy effect, and the subject is also dazed for 1 round if it fails a Will save. A dazed creature can take no actions, though it can defend itself normally.
\end{spellblood}
\begin{spellnotes}
A bewildered creature takes a \minus2 penalty to attack rolls, saving throws, checks, DCs, and AC.
\end{spellnotes}

\spellsection{Daze}{Mass}
\spelldesc{You cloud the mind of your foes, preventing them from taking any actions.}
\spellschool{Enchantment (Inhibition) [Mind-Affecting]}
\spelllvl{Sor/Wiz 9}
\spellrng{\rngmed}
\spellarea{\areamed limit}
\spelltgts{Five creatures within the area}
\begin{spelleffect}
This spell functions like \spell{daze}, except that it affects multiple creatures.
\end{spelleffect}

\spellsection{Destruction}
\spellschool{Necromancy (Flesh) [Death]}
\spelllvl{Clr 7, Destruction 7}
\spellcmp{V, S, F}
\spellrng{\rngclose}
\spelltgt{One creature}
\spelldur{Instantaneous}
\spellsave{Fortitude negates}
\spellsr{Yes (Fortitude)}
\begin{spellhealthy}
The target is staggered for 5 rounds.
\end{spellhealthy}
\begin{spellblood}
The target is instantly slain.
\end{spellblood}
\begin{spellnotes}
The reamins of a creature killed by this spell are consumed utterly (but not its equipment or possessions). The only way to restore life such a creature is to use \spell{true resurrection}, a carefully worded \spell{wish} spell followed by \spell{resurrection}, or \spell{miracle}.
\end{spellnotes}
\spellfocus{A special holy (or unholy) symbol of silver marked with verses of anathema (cost 500 gp).}

\spellsection{Dimensional Anchor}
\spellschool{Abjuration (Negation)}
\spelllvl{Clr 3, Sor/Wiz 3}
\spellrng{\rngmed}
\spelleff{Ray}
\spelldur{\durmed}
\spellsave{None}
\spellsr{Yes (Will)}
\begin{spelleffect}
A green ray springs from your outstretched hand. You must make a ranged touch attack to hit the target. Any creature or object struck by the ray is covered with a shimmering emerald field that completely blocks extradimensional travel. Forms of movement barred by a dimensional anchor include \spell{astral projection}, \spell{blink}, \spell{dimension door}, \spell{dimension slide}, \spell{ethereal jaunt}, \spell{etherealness}, \spell{gate}, \spell{maze}, \spell{plane shift}, \spell{shadow walk}, \spell{teleport}, and similar spell-like or psionic abilities. The spell also prevents the use of a \spell{gate} or \spell{teleportation circle} for the duration of the spell.
\end{spelleffect}
\begin{spellnotes}
A dimensional anchor does not interfere with the movement of creatures already in ethereal or astral form when the spell is cast, nor does it block extradimensional perception or attack forms. Also, dimensional anchor does not prevent summoned creatures from disappearing at the end of a summoning spell.
\end{spellnotes}

\spellsection{Discern Vulnerability}
\spellschool{Divination (Knowledge)}
\spelllvl{Div 4}
\spelltime{1 swift action}
\spellrng{\rngmed}
\spelltgt{One creature}
\spelldur{Instantaneous}
\spellsave{None}
\spellsr{No}
\begin{spelleffect}
You instantly recognize all of the target's vulnerabilities. This grants you a \plus2 circumstance bonus to attack rolls, weapon damage rolls, save DCs, and spell resistance checks against that creature. In addition, you learn any significant weaknesses the creature has. This includes, but is not limited to, the following information:
\begin{itemize*}
\item Which of the target's saving throws is lowest
\item If the target has any vulnerabilities to specific damage types
\item How to overcome the target's damage reduction, regeneration, or other similar abilities
\end{itemize*}
\end{spelleffect}
\begin{spellnotes}
This spell gives no information about a creature's strengths or abilities -- only its weaknesses.
\end{spellnotes}

\spellsection{Dismissal}
\spellschool{Abjuration/Conjuration (Interdiction, Translocation) [Planar]}
\spelllvl{Clr 4, Sor/Wiz 4}
\spellrng{\rngclose}
\spelltgt{One extraplanar creature}
\spelldur{Instantaneous}
\spellsave{Will negates; see text}
\spellsr{Yes (Will)}
\begin{spelleffect}
This spell forces an extraplanar creature, including any summoned creature, back to its proper plane. If the spell is successful, the creature is instantly whisked away, but there is a 20\% chance of actually sending the subject to a plane other than its own.
\end{spelleffect}

\spellsection{Disrupting Weapon}
\spellschool{Necromancy/Transmutation (Imbuement, Positive)}
\spelllvl{Clr 5}
\spellrng{\rngtouch}
\spelltgt{One melee weapon}
\spelldur{\durshort}
\spellsave{Will negates (object); see text}
\spellsr{Yes (Will)}
\begin{spelleffect}
This spell infuses a melee weapon with positive energy, making it deadly to undead. Once per round, the first bloodied undead creature struck in combat with this weapon must succeed on a Will save or be destroyed utterly. Healthy undead creatures suffer no ill effect. Spell resistance does not apply against the destruction effect.
\end{spelleffect}

\spellsection{Dominate Monster}
\spellschool{Enchantment (Compulsion) [Domination, Mind-Affecting]}
\spelllvl{Ench 8}
\spelltgt{One creature}
\begin{spelleffect}
This spell functions like \spell{dominate person}, except that the spell is not restricted by creature type.
\end{spelleffect}

\spellsection{Dominate Person}
\spellschool{Enchantment (Compulsion) [Domination, Mind-Affecting]}
\spelllvl{Brd 5, Ench 6}
\spelltime{Full-round action}
\spellrng{\rngclose}
\spelltgt{One humanoid}
\spelldur{One day}
\spellsave{Will negates}
\spellsr{Yes (Will)}
\begin{spelleffect}
You can control the actions of any humanoid creature through a telepathic link that you establish with the subject's mind.
\par If you and the subject have a common language, you can generally force the subject to perform as you desire, within the limits of its abilities. If no common language exists, you can communicate only basic commands, such as ``Come here," ``Go there," ``Fight," and ``Stand still." If you concentrate on the spell, you know what the subject is experiencing, but you do not receive direct sensory input from it, nor can it communicate with you telepathically.
\par Once you have given a dominated creature a command, it continues to attempt to carry out that command to the exclusion of all other activities except those necessary for day-to-day survival (such as sleeping, eating, and so forth). Because of this limited range of activity, a Sense Motive check against DC 15 (rather than DC 25) can determine that the subject's behavior is being influenced by an enchantment effect (see the Sense Motive skill description).
It takes time for the link to be established. For the first hour after the spell is cast, you must concentrate on the spell (a standard action) to control the subject's actions. While you are not concentrating on the spell, the creature acts as if confused, as the \spell{confusion} spell, except that it never attacks you. If the subject would randomly attack you, it instead is forced to follow your commands. At the end of the hour, the creature makes a second saving throw against the spell effect. If you concentrate on the spell during this time, it takes a \minus4 penalty on the saving throw. If it succeeds, it ignores the spell effect; otherwise, you dominate it fully for the remainder of the spell duration.
\par After the first hour, changing your instructions or giving a dominated creature a new command is the equivalent of redirecting a spell, so it is a move action.
\par By concentrating fully on the spell (a standard action), you can receive full sensory input as interpreted by the mind of the subject, though it still can't communicate with you. You can't actually see through the subject's eyes, so it's not as good as being there yourself, but you still get a good idea of what's going on.
\par Subjects resist this control, and any subject forced to take actions against its nature receives a new saving throw. This does not apply when a subject is merely ordered to perform an action it disagrees with -- the action must be directly opposed to the subject's beliefs. Ordering a paladin to murder an innocent would grant the paladin a saving throw, but ordering him to build a bridge that would allow an evil army to cross a river would not grant him a saving throw. Obviously self-destructive orders are never carried out. Once control is established, the range at which it can be exercised is unlimited, as long as you and the subject are on the same plane. You need not see the subject to control it.
\par If you recast this spell on a subject you have dominated before it escapes your control, you can extend the duration of the spell indefinitely. The subject does not get a new saving throw when you renew your control in this fashion.
\end{spelleffect}
\begin{spellnotes}

\spell{Protection from evil} or a similar spell can prevent you from exercising control or using the telepathic link while the subject is so shielded, but such an effect neither prevents the establishment of domination nor dispels it.
\end{spellnotes}

\pdfbookmark[2]{E}{SpellDescriptionsE}
\begin{comment}
\subsubsection{E}
\end{comment}

\spellsection{Energy Drain}
\spellschool{Necromancy (Vitalism) [Negative]}
\spelllvl{Clr 8, Death 8, Evil 8, Sor/Wiz 8}
\spelldur{\durext}
\spellsave{Fortitude half; see text}
\begin{spelleffect}
This spell functions like \spell{enervation}, except that the creature struck gains four negative levels, and the negative levels last longer.
\par An undead creature struck by the ray gains 40 temporary hit points \plus1 per caster level instead.
\end{spelleffect}

\spellsection{Enervation}
\spellschool{Necromancy (Vitalism) [Negative]}
\spelllvl{Death 4, Evil 4, Sor/Wiz 4}
\spellrng{\rngclose}
\spelleff{Ray of negative energy}
\spelldur{\durlong}
\spellsave{Fortitude half}
\spellsr{Yes (Fortitude)}
\begin{spelleffect}
You point your finger and utter the incantation, releasing a black ray of crackling negative energy that suppresses the life force of any living creature it strikes. You must make a ranged touch attack to hit. If the attack succeeds, the subject gains two negative levels. A successful Fortitude save halves the number of negative levels.
\par If the subject has at least as many negative levels as HV, it dies. Each negative level gives a creature a \minus1 penalty on attack rolls, saving throws, checks, and effective level (for determining the power, duration, DC, and other details of spells or special abilities). Additionally, a spellcaster loses one spell or spell slot from his or her highest available level.
\par An undead creature struck by the ray gains 20 temporary hit points \plus1 per caster level instead.
\end{spelleffect}
\begin{spellnotes}
This spell stacks with any effect that bestows negative levels, including itself.
\end{spellnotes}

\spellsection{Entangle}
\spellschool{Transmutation (Animation)}
\spelllvl{Drd 1, Nature 1}
\spellrng{\rngmed}
\spellarea{\areasmall radius spread}
\spelldur{\durshort (D)}
\spellsave{Reflex partial}
\spellsr{No}
\begin{spelleffect}
Grasses, weeds, bushes, and even trees wrap, twist, and entwine about creatures in the area or those that enter the area, holding them fast and causing them to become entangled. The creature can break free and move half its normal speed by using a standard action to make a combat maneuver check or an Escape Artist check against this spell's save DC. A creature that succeeds on a Reflex save is not entangled but can still move at only half speed through the area. Each round on your turn, the plants once again attempt to entangle all creatures that have avoided or escaped entanglement.
\end{spelleffect}
\begin{spellnotes}
The effects of the spell may be altered somewhat based on the nature of the entangling plants. If no plants exist in the area, the spell has no effect.
\end{spellnotes}

\spellsection{Entangling Growth}
\spellschool{Transmutation (Alteration, Animation)}
\spelllvl{Drd 4, Nature 4}
\spellarea{\areamed radius spread}
\begin{spelleffect}
This spell functions like \spell{entangle}, except that it affects a wider area and also grows new plants in the area. These plants grow from any terrain, even if it would not normally support plant life, and entangle creatures in the area for the duration of the spell. When the magic fades, the plants with and recede into the ground, leaving no trace that they were ever there.
\end{spelleffect}

\spellsection{Faerie Fire}
\spellschool{Illusion (Figment) [Light, Unreal]}
\spelllvl{Drd 2}
\spellrng{\rnglong}
\spellarea{\areasmall radius limit}
\spelleff{Dim lights in the area}
\spelldur{\durshort (D)}
\spellsave{None}
\spellsr{Yes (Will)}
\begin{spelleffect}
A pale glow surrounds and outlines all creatures and objects in the area. Outlined subjects shed light as candles. Outlined creatures do not benefit from the concealment normally provided by darkness (though a 3rd-level or higher magical darkness effect functions normally), \spell{blur}, \spell{displacement}, \spell{invisibility}, or similar effects. The light is too dim to have any special effect on undead or dark-dwelling creatures vulnerable to light. The faerie fire can be blue, green, or violet, according to your choice at the time of casting. This spell does not cause any harm to the objects or creatures thus outlined.
\end{spelleffect}

\spellsection{Fear}
\spelldesc{You project an invisible cone that drives creatures away from you in abject fear.}
\spellschool{Enchantment (Emotion) [Fear, Mind-Affecting]}
\spelllvl{Brd 4, Sor/Wiz 5}
\spellarea{\areamed cone-shaped burst}
\spelldur{\durshort (D)}
\spellsave{Will negates}
\spellsr{Yes (Will)}
\begin{spellhealthy}
Creatures in the area are shaken.
\end{spellhealthy}
\begin{spellblood}
Creatures in the area are frightened.
\end{spellblood}

\spellsection{Feeblemind}
\spellschool{Enchantment (Inhibition) [Mind-Affecting]}
\spelllvl{Evil 5, Sor/Wiz 5}
\spellrng{\rngtouch}
\spelltgt{Touched creature}
\spelldur{Instantaneous}
\spellsave{Will negates}
\spellsr{Yes (Will)}
\begin{spellhealthy}
The target is bewildered for 5 minutes.
\end{spellhealthy}
\begin{spellblood}
The target's Intelligence and Charisma scores each drop to 1, giving it roughly the intellect of a lizard. It is unable to use Intelligence- or Charisma-based skills, cast spells, understand language, or communicate coherently. Still, it knows who its friends are and can follow them and even protect them. The subject remains in this state until a \spell{heal}, \spell{limited wish}, \spell{miracle}, or \spell{wish} spell is used to cancel the effect of the \spell{feeblemind}. A creature that can cast arcane spells, such as a sorcerer or a wizard, takes a \minus4 penalty on its saving throw.
\end{spellblood}
\begin{spellnotes}
The target must be bloodied when the spell is cast to suffer the bloodied effect.
\end{spellnotes}

\spellsection{Finger of Death}
\spellschool{Necromancy (Life) [Death]}
\spelllvl{Death 7, Necro 7}
\spellrng{\rngclose}
\spelltgt{One living creature}
\spelldur{Instantaneous}
\spellsave{Fortitude negates}
\spellsr{Yes (Fortitude)}
\begin{spellhealthy}
The target is staggered for 5 rounds.
\end{spellhealthy}
\begin{spellblood}
The target is instantly slain.
\end{spellblood}

\spellsection{Fog Cloud}
\spelldesc{You conjure a bank of fog from a location you choose, concealing those inside.}
\spellschool{Conjuration (Creation) [Fog]}
\spelllvl{Drd 3, Sor/Wiz 3, Water 3}
\spellrng{\rngmed}
\begin{spelleffect}
This spell functions like \spell{obscuring mist}, except that you can create the cloud at any point within the spell's range.
\end{spelleffect}
\begin{spellnotes}
\par Fog spells do not function underwater and can be dispersed by wind. A moderate wind (11\add mph) disperses the fog in 5 rounds; a strong wind (21\add mph) disperses the fog in 1 round.
\end{spellnotes}

\spellsection{Forcecage}
\spellschool{Evocation (Control) [Force]}
\spelllvl{Evoc 7}
\spellcmp{V, S, M}
\spellrng{\rngclose}
\spelleff{Barred cage (20 ft. cube) or windowless cell (10 ft. cube)}
\spelldur{\durmed (D)}
\spellsave{Reflex negates; see text}
\spellsr{No}
\begin{spelleffect}
This powerful spell brings into being an immobile, invisible cubical prison composed of either bars of force or solid walls of force (your choice).
\par Creatures within the area are caught and contained unless they are too big to fit inside, in which case the spell automatically fails. Teleportation and other forms of astral travel provide a means of escape, but the force walls or bars extend into the Ethereal Plane, blocking ethereal travel.
\par A creature who makes a Reflex save chooses whether it wants to be inside or outside of the forcecage when it forms. The forcecage is formed regardless.
\par Like a \spell{wall of force} spell, a forcecage resists \spell{dispel magic}, but it is vulnerable to a \spell{disintegrate} spell, and it can be destroyed by a \magicitem{sphere of annihilation} or a \magicitem{rod of cancellation}.
\par \subspell{Barred Cage} This version of the spell produces a 20-foot cube made of bands of force (similar to a wall of force spell) for bars. The bands are a half-inch wide, with half-inch gaps between them. Any creature capable of passing through such a small space can escape; others are confined. You can't attack a creature in a barred cage with a weapon unless the weapon can fit between the gaps. Even against such weapons (including arrows and similar ranged attacks), a creature in the barred cage has cover. All spells and breath weapons can pass through the gaps in the bars.
\par \subspell{Windowless Cell} This version of the spell produces a 10-foot cube with no way in and no way out. Solid walls of force form its six sides.
\end{spelleffect}
\spellmat{Ruby dust worth 1000 gp, which is tossed into the air and disappears when you cast the spell.}

\spellsection{Ghoul Touch}
\spellschool{Necromancy (Flesh)}
\spelllvl{Sor/Wiz 2}
\spellrng{\rngtouch}
\spelltgt{Living humanoid touched}
\spelldur{\durshort}
\spellsave{None/Fortitude negates}
\spellsr{Yes (Fortitude)}
\begin{spellhealthy}
The subject is sickened.
\end{spellhealthy}
\begin{spellblood}
As the healthy effect, and the subject is paralyzed if it fails a Fortitude save. Each round that it is paralyzed, the subject can make a new saving throw. If it succeeds, it is no longer paralyzed by the spell, though it is still sickened. In addition, as long as it is paralyzed, the subject exudes a carrion stench that causes all living creatures (except you) in a \areasmall radius spread to become sickened (Fortitude negates) for 5 rounds.
\end{spellblood}

\spellsection{Glitterdust}
\spellschool{Conjuration (Creation)}
\spelllvl{Sor/Wiz 2}
\spellrng{\rngmed}
\spellarea{\areasmall radius spread}
\spelleff{Glittering particles in the area}
\spelldur{\durshort}
\spellsave{None}
\spellsr{No}
\begin{spelleffect}
A cloud of golden particles covers everyone and everything in the area, visibly outlining invisible things for the duration of the spell. It likewise negates the effects of blur and displacement, and reveals figments, mirror images, and projected images for what they are. All within the area are covered by the dust, which cannot be removed and continues to sparkle until it fades.
\par Any creature covered by the dust takes a \minus40 penalty on Hide checks.
\end{spelleffect}

\spellsectioncomma{Glitterdust}{Greater}
\spellschool{Conjuration (Creation)}
\spelllvl{Sor/Wiz 5}
\spellarea{\areamed radius spread}
\spellsave{None/Will negates}
\begin{spelleffect}
This spell functions like \spell{glitterdust}, except that creatures in the area must also make Will saves or be dazzled for the duration of the spell.
\end{spelleffect}

\spellsection{Grease}
\spelldesc{You conjure a layer of slippery grease on the ground, tripping up your foes.}
\spellschool{Conjuration (Creation)}
\spelllvl{Brd 1, Sor/Wiz 1}
\spellrng{\rngclose}
\spelltgtorarea{One object or a 10 ft. square}
\spelldur{\durshort (D)}
\spellsave{See text}
\spellsr{No}
\begin{spelleffect}
Any creature in the area when the spell is cast must make a successful Reflex save or fall. A creature can walk within or through the area of grease at half normal speed with a DC 10 Balance check. Failure means it can't move that round, while failure by 5 or more means it falls (see the Balance skill for details). A creature standing in a greased area loses its Dexterity and dodge modifiers to AC due to the slippery surface.
\par The spell can also be used to create a greasy coating on an item. Material objects not in use are always affected by this spell, while an object wielded or employed by a creature receives a Reflex saving throw to avoid the effect entirely. If the initial saving throw fails, the creature immediately drops the item. If the item is successfully greased, a saving throw must be made in each round that the creature attempts to pick up or use the greased item. A creature wearing greased armor or clothing gains a \plus10 bonus on Escape Artist checks and on grapple attacks made to resist or escape a grapple or to escape a pin.
\end{spelleffect}

\spellsection{Gust of Wind}
\spellschool{Evocation (Control) [Air]}
\spelllvl{Air 1, Drd 1, Sor/Wiz 1}
\spellarea{\arealarge line-shaped emanation from you}
\spelleff{Wind within the area}
\spelldur{1 round}
\spellsave{Fortitude partial; see text.}
\spellsr{No}
\begin{spelleffect}
This spell creates a severe blast of air (approximately 50 mph) that originates from you, affecting all creatures in its path. Creatures are affected according to their size category. A successful Fortitude save causes a creature to be affected as if it were one size category larger. Flying creatures are affected as if one size category smaller.
\begin{itemize*}
\item Tiny or smaller creatures are knocked prone and blown to the edge of the spell's range.
\item Small creatures are knocked prone by the force of the wind.
\item Medium creatures are unable to move forward against the force of the wind.
\item Large or larger creatures may move normally.
\end{itemize*}
\par Any creature, regardless of size, takes a \minus4 penalty on ranged attacks and Listen checks in the spell's area.
\par In addition to the effects noted, a gust of wind can do anything that a sudden blast of wind would be expected to do. It can extinguish open flames, create a stinging spray of sand or dust, fan a large fire, overturn delicate awnings or hangings, heel over a small boat, and blow gases or vapors to the edge of its range.
\end{spelleffect}
\begin{spellnotes}
\spell{Gust of wind} can be made permanent with a \spell{permanency} ritual.
\end{spellnotes}

\pdfbookmark[2]{H}{SpellDescriptionsH}
\begin{comment}
\subsubsection{H}
\end{comment}

\spellsection{Hideous Laughter}
\spelldesc{You force the subject to collapse into gales of manic laughter with an unnaturally amusing joke.}
\spellschool{Enchantment (Compulsion) [Mind-Affecting]}
\spelllvl{Brd 2, Chaos 2}
\spellrng{\rngclose}
\spelltgt{One creature}
\spelldur{\durshort}
\spellsave{Will negates}
\spellsr{Yes (Will)}
\begin{spellhealthy}
The subject is bewildered.
\end{spellhealthy}
\begin{spellblood}
The subject falls prone and can take no actions. It loses its Dexterity and dodge modifiers to AC. Each round on its turn, the affected creature can attempt a new saving throw. If it succeeds, the effect ends, though it can take no other actions that round.
\end{spellblood}
\begin{spellnotes}
A creature with an Intelligence score of 2 or lower is not affected. A creature whose type is different from the caster's receives a \plus4 circumstance bonus on its saving throw, because humor doesn't ``translate" well.
\end{spellnotes}
\begin{spellnotes}
A bewildered creature takes a \minus2 penalty to attack rolls, saving throws, checks, DCs, and AC.
\end{spellnotes}

\spellsection{Hold Monster}
\spellschool{Enchantment (Inhibition) [Mind-Affecting]}
\spelllvl{Brd 4, Law 4, Sor/Wiz 4}
\spellrng{\rngmed}
\spelltgt{One living creature}
\begin{spelleffect}
This spell functions like \spell{hold person}, except that it is not limited by creature type.
\end{spelleffect}

\spellsectioncomma{Hold Monster}{Mass}
\spellschool{Enchantment (Inhibition) [Mind-Affecting]}
\spelllvl{Sor/Wiz 9}
\spellrng{\rngmed}
\spellarea{\areamed radius limit}
\spelltgts{Five creatures within the area}
\begin{spelleffect}
This spell functions like \spell{hold monster}, except that it affects multiple creatures.
\end{spelleffect}

\spellsection{Hold Person}
\spellschool{Enchantment (Inhibition) [Mind-Affecting]}
\spelllvl{Brd 2, Clr 2, Sor/Wiz 2, War 2}
\spellrng{\rngclose}
\spelltgt{One humanoid creature}
\spelldur{\durshort (D); see text}
\spellsave{Will negates; see text}
\spellsr{Yes (Will)}
\begin{spellhealthy}
The subject is bewildered.
\end{spellhealthy}
\begin{spellblood}
The subject is paralyzed and unable to act. Each round on its turn, the subject may attempt a new saving throw to end the effect. If it succeeds, it is no longer paralyzed, though it can take no other actions that round.
\end{spellblood}
\begin{spellnotes}
A bewildered creature takes a \minus2 penalty to attack rolls, saving throws, checks, DCs, and AC.
\end{spellnotes}

\spellsectioncomma{Hold Person}{Mass}
\spellschool{Enchantment (Inhibition) [Mind-Affecting]}
\spelllvl{Brd 6, Clr 7, Sor/Wiz 7}
\spellarea{\areamed radius limit}
\spelltgts{Five creatures within the area}
\begin{spelleffect}
This spell functions like \spell{hold person}, except that it affects multiple creatures.
\end{spelleffect}

\spellsection{Holy Word}
\spellschool{Evocation (Channeling) [Good]}
\spelllvl{Good 7}
\spellcmp{V}
\spellarea{\arealarge radius spread centered on you}
\spelldur{Instantaneous}
\spellsave{None}
\spellsr{Yes (Will)}
\begin{spellhealthy}
\par Each nongood creature in the area is deafened for 5 rounds.
\end{spellhealthy}
\begin{spellblood}
\par Each nongood creature in the area suffers one or more of the following ill effects, depending on its Hit Values.
\begin{dtable}
\begin{tabularx}{\columnwidth}{l >{\lcol}X}
\par \thead{HV} & \thead{Effect} \\
\par Equal to caster level & Deafened \\
\par Up to caster level \minus5 & Blinded, deafened \\
\par Up to caster level \minus10 & Paralyzed, nauseated, deafened \\
\par Up to caster level \minus15 & Killed\fn{1}
\end{tabularx}
1 Living creatures die. Nonliving creatures are destroyed.
\end{dtable}
\par \subspell{Deafened} The creature is deafened for 5 rounds.
\par \subspell{Blinded} The creature is blinded for 2 rounds.
\par \subspell{Paralyzed} The creature is paralyzed and helpless for 5 rounds.
\par \subspell{Killed} Living creatures die. Nonliving creatures are destroyed.
\end{spellblood}
\begin{spellnotes}
Creatures whose Hit Values exceed your caster level are unaffected by \spell{holy word}.
\end{spellnotes}

\spellsection{Hypnotic Pattern}
\spelldesc{You create a twisting pattern of subtle, shifting colors that weaves through the air, fascinating creatures within it.}
\spellschool{Enchantment/Illusion (Compulsion, Figment) [Light, Mind-Affecting]}
\spelllvl{Brd 2, Sor/Wiz 2}
\spellcmp{V (Brd only), S; see text}
\spellrng{\rngmed}
\spellarea{\areasmall radius spread}
\spelleff{Colorful lights in the area}
\spelldur{Concentration \add 2 rounds}
\spellsave{Will negates}
\spellsr{Yes (Will)}
\begin{spelleffect}
Creatures within the spell's area are fascinated.
\end{spelleffect}
\begin{spellnotes}
Creatures who cannot see the lights are not affected by this spell. A wizard or sorcerer need not utter a sound to cast this spell, but a bard must perform a verbal component.
\end{spellnotes}

\pdfbookmark[2]{I}{SpellDescriptionsI}
\begin{comment}
\subsubsection{I}
\end{comment}

\spellsection{Implosion}
\spelldesc{You create a destructive responance in your foe's body that destroys it from the inside out.}
\spellschool{Evocation (Control)}
\spelllvl{Clr 9, Destruction 9, Evoc 9}
\spellrng{\rngclose}
\spelltgts{One corporeal creature/round}
\spelldur{Instantaneous and concentration (up to 5 rounds); see text}
\spellsave{None/Fortitude negates}
\spellsr{Yes (Fortitude)}
\begin{spellhealthy}
The target is staggered for 5 rounds.
\end{spellhealthy}
\begin{spellblood}
The target is instantly slain.
\end{spellblood}
\begin{spellnotes}
You can concentrate on one creature per round. You can target a particular creature only once with each casting of the spell.
\par \spell{Implosion} has no effect on creatures in \spell{gaseous form} or on incorporeal creatures.
\end{spellnotes}

\spellsection{Imprisonment}
\spellschool{Conjuration/Transmutation (Time, Translocation) [Teleportation]}
\spelllvl{Earth 9, Law 9, Sor/Wiz 9}
\spellrng{\rngmed}
\spelltgt{One creature}
\spelldur{Instantaneous and Permanent}
\spellsave{Will negates}
\spellsr{Yes (Will)}
\begin{spellhealthy}
The subject is slowed for 5 rounds.
\end{spellhealthy}
\begin{spellblood}
The subject is entombed in a state of suspended animation (as the \spell{temporal stasis} spell) in a small sphere far beneath the surface of the earth. It remains there unless an \spell{emancipation} spell is cast at the locale where the imprisonment took place.
\end{spellblood}
\begin{spellnotes}
A slowed creature can take only a single move action or standard action each turn, but not both. Additionally, it takes a -2 penalty to attack rolls, Strength and Dexterity-based skill checks, and armor class.

The subject must be bloodied at the time that the spell is cast to be imprisoned. Magical search by a crystal ball, a \spell{locate creature} spell, or some other similar divination does not reveal the fact that a creature is imprisoned, but \spell{discern location} does. A \spell{wish} or \spell{miracle} spell will not free the recipient, but will reveal where it is entombed.
\end{spellnotes}

\spellsection{Insanity}
\spellschool{Enchantment (Compulsion) [Mind-Affecting]}
\spelllvl{Chaos 6, Ench 6}
\spellrng{Touch}
\spelltgt{Living creature touched}
\spelldur{Permanent}
\spellsave{Will negates}
\spellsr{Yes (Will)}
\begin{spellhealthy}
The creature is bewildered.
\end{spellhealthy}
\begin{spellblood}
\par The affected creature is confused (see the \spell{confusion} spell).
\end{spellblood}
\begin{spellnotes}
A bewildered creature takes a \minus2 penalty to attack rolls, saving throws, checks, DCs, and AC. \spell{Remove curse} and \spell{dispel magic} do not remove \spell{insanity}. \spell{Greater restoration}, \spell{heal}, \spell{limited wish}, \spell{miracle}, or \spell{wish} can restore the creature.
\end{spellnotes}

\spellsection{Irresistible Dance}
\spelldesc{You fill your enemy with an overpowering urge to dance and caper in place. Against its will, it begins doing so, complete with foot shuffling and tapping.}
\spellschool{Enchantment (Compulsion) [Mind-Affecting]}
\spelllvl{Brd 6, Ench 9}
\spellrng{\rngclose}
\spelltgt{One living creature}
\spelldur{1d4\plus1 rounds}
\spellsave{None}
\spellsr{Yes (Will)}
\begin{spelleffect}
The subject is flat-footed and must spend a standard action each round to do nothing but dance, which provokes attacks of opportunity.
\end{spelleffect}

\pdfbookmark[2]{J-L}{SpellDescriptionsJ-L}
\begin{comment}
\subsubsection{J-L}
\end{comment}

\spellsection{Link Vitality}
\spellschool{Necromancy (Life)}
\spelllvl{Life 3, Necro 3}
\spellarea{\areamed radius limit centered on you}
\spelltgts{Any two living creatures within the area}
\spelldur{\durshort}
\spellsave{Will negates}
\spellsr{Yes (Will)}
\begin{spelleffect}
You temporarily link the fates of any two creatures, if both fail their saving throws. If either linked creature experiences pain, both feel it. When one loses hit points, the other loses the same amount. If one takes nonlethal damage, so does the other. Likewise, when one regains hit points, the other heals the same amount. Excess healing is simply lost. If one creature is subjected to an effect to which it is immune (such as a type of energy damage), the linked creature is not subjected to it.
\end{spelleffect}
\begin{spellnotes}
No other effects are transferred by \spell{link vitality}.
\end{spellnotes}

\spellsectioncomma{Link Vitality}{Mass}
\spellschool{Necromancy (Life)}
\spelllvl{Sor/Wiz 7}
\spelltgts{Five living creatures within the area}
\begin{spelleffect}
This spell functions as \spell{link vitality}, except that it affects many creatures. The spell links all creatures who fail their saving throws. If any of the linked creatures lose or gain hit points, all linked creatures lose or gain the same amount, and so on.
\end{spelleffect}

\spellsection{Mage's Disjunction}
\spellschool{Abjuration (Negation) [Magic]}
\spelllvl{Magic 9, Abjur 9}
\spellrng{\rngmed}
\spelltgtorarea{One magic item or \areamed radius burst}
\spelldur{Instantaneous}
\spellsave{Will negates (object)}
\spellsr{No}
\begin{spelleffect}
All magical effects within the radius of the spell, except for those on you, are disjoined. That is, spells and spell-like effects are separated into their individual components (ending the effect as a dispel magic spell does).
\par You also have a 2\% chance per caster level of destroying an \spell{antimagic field}.
\par You can also use this spell to target a single item. The item gets a Will save at a \minus5 penalty to avoid being permanently rendered nonmagical. Even artifacts are subject to this use of disjunction, though there is only a 1\% chance per caster level of actually affecting such powerful items. Additionally, if an artifact is destroyed, you permanently lose the ability to cast \spell{mage's disjunction}. (This ability cannot be recovered by mortal magic, not even \spell{miracle} or \spell{wish}.)
\par \subspell{Note} Destroying artifacts is a dangerous business, and it is 95\% likely to attract the attention of some powerful being who has an interest in or connection with the device.
\end{spelleffect}

\spellsection{Mind Fog}
\spelldesc{You conjure a fog bank that hampers the mental acuity of those caught in it.}
\spellschool{Conjuration/Enchantment (Creation, Inhibition) [Fog, Mind-Affecting]}
\spelllvl{Brd 4, Sor/Wiz 5, Trickery 5}
\spellrng{\rngclose}
\spelldur{\durlong and 5 rounds; see text}
\spellsave{None/Will negates}
\spellsr{None/Yes (Will)}
\begin{spelleffect}
This spell functions like \spell{fog cloud}, except each creature in the fog take a \minus5 penalty to Wisdom unless it makes a Will save. A creature that successfully saves against the fog is not affected, but if it remains in the fog, it must make a new save each minute to avoid being affected. Affected creatures take the penalty as long as they remain in the fog and for 5 rounds thereafter. The fog is stationary and lasts for 1 hour (or until dispersed by wind).
\end{spelleffect}
\begin{spellnotes}
A moderate wind (11\add mph) disperses the fog in 5 rounds; a strong wind (21\add mph) disperses the fog in 1 round.
\end{spellnotes}

\spellsection{Obscuring Mist}
\spelldesc{You conjure a bank of fog that arises around you, concealing you and your allies.}
\spellschool{Conjuration (Creation) [Fog]}
\spelllvl{Clr 1, Drd 1, Sor/Wiz 1, Water 1}
\spellarea{\areamed radius cylinder-shaped spread centered on you, 20 ft. high}
\spelleff{Fog in the area}
\spelldur{\durmed}
\spellsave{None}
\spellsr{No}
\begin{spelleffect}
Creatures within the spell's area have concealment (20\% miss chance). The cloud is stationary once created.
\end{spelleffect}
\begin{spellnotes}
A moderate wind (11\add mph), such as from a \spell{gust of wind} spell, disperses the fog in 5 rounds. A strong wind (21\add mph) disperses the fog in 1 round. A fire spell burns away the fog in the area into which it deals damage.
\par This spell does not function underwater.
\end{spellnotes}

\spellsection{Phantasmal Killer}
\spelldesc{You create a phantasmal image of the most fearsome creature imaginable to the subject simply by forming the fears of the subject's subconscious mind into something that its conscious mind can visualize: this most horrible beast.}
\spellschool{Enchantment/Illusion (Emotion, Phantasm) [Death, Fear, Mind-Affecting, Unreal]}
\spelllvl{Sor/Wiz 4}
\spellrng{\rngclose}
\spelltgt{One living creature}
\spelldur{Instantaneous}
\spellsave{Will disbelief and Fortitude negates; see text}
\spellsr{Yes (Will)}
\begin{spellhealthy}
If the subject fails its Will save, it is shaken for 5 rounds.
\end{spellhealthy}
\begin{spellblood}
As the healthy effect, and if it fails its Will save, it immediately dies unless it makes a successful Fortitude save.
\end{spellblood}

\spellsection{Phantom Maze}
\spelldesc{You manipulate the subject's perceptions, causing it to believe that it is trapped in a labyrinth.}
\spellschool{Illusion (Phantasm) [Unreal]}
\spelllvl{Illus 6}
\spellrng{\rngclose}
\spelltgt{One creature}
\spelldur{\durmed}
\spellsave{Will disbelief}
\spellsr{Yes (Will)}
\begin{spelleffect}
The subject is blinded and deafened, and acts as if it has been affected by a \spell{maze} spell. Typically, this means it moves in a random direction each round. If it encounters any physical resistance in its movements or takes any damage, it may immediately make a Will save to disbelieve the effect.
\end{spelleffect}

\spellsection{Poison}
\spellschool{Necromancy (Flesh) [Poison]}
\spelllvl{Clr 4, Drd 3}
\spellrng{Touch}
\spelltgt{Living creature touched}
\spelldur{Instantaneous; see text}
\spellsave{Fortitude negates; see text}
\spellsr{Yes (Fortitude)}
\begin{spelleffect}
Calling upon the venomous powers of natural predators, you infect the subject with a horrible poison that drains its life force by making a successful melee touch attack. The poison deals 1d6 points of Constitution damage immediately. A Fortitude save negates this damage. The spell continues dealing another 1d6 points of Constitution damage every 5 rounds until the subject makes two successful Fortitude saves to resist the poison.
\end{spelleffect}

\spellsection{Power Word Blind}
\spellschool{Necromancy (Flesh)}
\spelllvl{Sor/Wiz 7}
\spellcmp{V}
\spellrng{\rngclose}
\spelltgt{One creature}
\spelldur{Instantaneous}
\spellsave{None}
\spellsr{Yes (Fortitude)}
\begin{spellhealthy}
The subject is sickened for 5 rounds.
\end{spellhealthy}
\begin{spellblood}
The subject is blinded for 5 rounds.
\end{spellblood}
\begin{spellnotes}
The subject must be bloodied when the spell is cast to suffer the bloodied effect.
\end{spellnotes}

\spellsection{Power Word Kill}
\spelldesc{You utter a single word of power that instantly kills your foe, whether it can hear the word or not.}
\spellschool{Necromancy (Life) [Death]}
\spelllvl{Death 9, Sor/Wiz 9}
\spellcmp{V}
\spellrng{\rngclose}
\spelltgt{One living creature}
\spelldur{Instantaneous}
\spellsave{None}
\spellsr{Yes (Fortitude)}
\begin{spellhealthy}
The subject is sickened for 5 rounds.
\end{spellhealthy}
\begin{spellblood}
The subject is instantly slain.
\end{spellblood}
\begin{spellnotes}
The subject must be bloodied when the spell is cast to suffer the bloodied effect. Creatures whose HV exceed your caster level are sickened instead of killed.
\end{spellnotes}

\spellsection{Power Word Confuse}
\spellschool{Enchantment (Compulsion) [Mind-Affecting]}
\spelllvl{Sor/Wiz 6}
\spellcmp{V}
\spellrng{\rngclose}
\spelltgt{One creature}
\spelldur{Instantaneous}
\spellsave{None}
\spellsr{Yes (Will)}
\begin{spellhealthy}
The subject is bewildered for 5 rounds.
\end{spellhealthy}
\begin{spellblood}
The subject is confused for 5 rounds. \confusionexplanation
\end{spellblood}
\begin{spellnotes}
The subject must be bloodied when the spell is cast to suffer the bloodied effect.
\end{spellnotes}

\spellsection{Power Word Stun}
\spelldesc{You utter a single word of power that instantly causes your foe to become stunned, whether the creature can hear the word or not.}
\spellschool{Enchantment (Inhibition) [Mind-Affecting]}
\spelllvl{Sor/Wiz 8}
\spellcmp{V}
\spellrng{\rngclose}
\spelltgt{One creature}
\spelldur{Instantaneous}
\spellsave{None}
\spellsr{Yes (Will)}
\begin{spellhealthy}
The subject is bewildered for 5 rounds.
\end{spellhealthy}
\begin{spellblood}
The subject is stunned for 5 rounds.
\end{spellblood}
\begin{spellnotes}
The subject must be bloodied when the spell is cast to suffer the bloodied effect.
\end{spellnotes}

\spellsection{Rainbow Pattern}
\spelldesc{You create a glowing, rainbow-hued pattern of interweaving colors that fascinates those within it.}
\spellschool{Enchantment/Illusion (Compulsion, Figment) [Light, Mind-Affecting, Sight-Dependent]}
\spelllvl{Brd 4, Sor/Wiz 4}
\spellcmp{V (Brd only), S; see text}
\spellrng{\rngmed}
\spellarea{\areasmall radius spread}
\spelleff{Colorful lights in the area}
\spelldur{Concentration}
\spellsave{Will negates}
\spellsr{Yes (Will)}
\begin{spelleffect}
Creatures in the spell's area are fascinated. While concentrating on the spell, you can make the pattern move up to 30 feet per round (moving its effective point of origin). All fascinated creatures follow the moving rainbow of light, trying to get or remain within the effect. Fascinated creatures who are restrained and removed from the pattern still try to follow it. If the pattern leads its subjects into a dangerous area each fascinated creature gets a second save. If the view of the lights is completely blocked, creatures who can't see them are no longer affected.
\end{spelleffect}
\begin{spellnotes}
The spell does not affect sightless creatures.
\par Verbal Component: A wizard or sorcerer need not utter a sound to cast this spell, but a bard must sing, play music, or recite a rhyme as a verbal component.
\end{spellnotes}

\spellsection{Ray of Clumsiness}
\spelldesc{You fire a coruscating ray from your hand. When it strikes your foe, he becomes clumsier and less agile.}
\spellschool{Necromancy (Flesh)}
\spelllvl{Sor/Wiz 1}
\spellrng{\rngclose}
\spelleff{Ray}
\spelldur{\durshort}
\spellsave{Fortitude half}
\spellsr{Yes (Fortitude)}
\begin{spelleffect}
You must succeed on a ranged touch attack. The subject takes a \minus4 penalty to Dexterity.
\end{spelleffect}
\begin{spellnotes}
The subject's Dexterity score cannot drop below 1.
\end{spellnotes}

\spellsection{Ray of Enfeeblement}
\spelldesc{You fire a coruscating ray from your hand. When it strikes your foe, he becomes weaker.}
\spellschool{Necromancy (Unlife)}
\spelllvl{Sor/Wiz 1}
\spellrng{\rngclose}
\spelleff{Ray}
\spelldur{\durshort}
\spellsave{Fortitude half}
\spellsr{Yes (Fortitude)}
\begin{spelleffect}
You must succeed on a ranged touch attack. The subject takes a \minus4 penalty to Strength.
\end{spelleffect}
\begin{spellnotes}
The subject's Strength score cannot drop below 1.
\end{spellnotes}

\spellsection{Ray of Exhaustion}
\spelldesc{You fire a black ray at your foe, depleting his stamina.}
\spellschool{Necromancy (Flesh)}
\spelllvl{Sor/Wiz 3}
\spellrng{\rngclose}
\spelleff{Ray}
\spelldur{\durmed}
\spellsave{Fortitude partial; see text}
\spellsr{Yes (Fortitude)}
\begin{spelleffect}
If you succeed on a ranged touch attack with the ray, the subject is immediately exhausted. A successful Fortitude save means the creature is only fatigued.
\end{spelleffect}
\begin{spellnotes}
A creature that is already fatigued instead becomes exhausted. A creature that is already exhausted suffers no further penalties. Unlike normal exhaustion or fatigue, the effect ends as soon as the spell's duration expires.
\end{spellnotes}

\spellsection{Repulsion}
\spellschool{Abjuration (Shielding) [Barrier]}
\spelllvl{Abjur 6, Protection 6, Travel 6}
\spellarea{Up to a \arealarge radius emanation centered on you}
\spelldur{\durshort (D)}
\spellsave{Will negates}
\spellsr{Yes (Will)}
\begin{spelleffect}
An invisible, mobile field surrounds you and prevents creatures from approaching you. You decide how big the field is at the time of casting. Any creature within or entering the field must attempt a save. If it fails, it becomes unable to move toward you for the duration of the spell. Repelled creatures' actions are not otherwise restricted.
\par They can fight other creatures and can cast spells and attack you with ranged weapons. The creature is free to make melee attacks against you if you come within reach. If a repelled creature moves away from you and then tries to turn back toward you, it cannot move any closer if it is still within the spell's area.
\end{spelleffect}
\begin{spellnotes}
Unlike most barrier spells, this spell does not collapse if you move towards a creature held at bay by the barrier. The spell continues to prevent that creature from approaching you, but the creature suffers no other ill effect.
\end{spellnotes}

\spellsection{Resilient Sphere}
\spellschool{Evocation (Control) [Force]}
\spelllvl{Sor/Wiz 4}
\spellrng{\rngclose}
\spelleff{5 ft. radius sphere, centered around creatures or objects}
\spelldur{\durshort (D)}
\spellsave{Reflex negates}
\spellsr{Yes (Reflex)}
\begin{spelleffect}
This spell creates a globe of shimmering force centered around a creature or object. The sphere persists for the spell's duration, containing any creatures or objects held inside, provided they are small enough to fit within the diameter of the sphere. It is not subject to damage of any sort.
\par The subject may struggle, but the sphere cannot be physically moved either by people outside it or by the struggles of those within.
\end{spelleffect}
\begin{spellnotes}
The sphere can only be affected a \spell{disintegrate} spell, a targeted \spell{dispel magic} spell, or similar effects. These effects destroy the sphere without harm to the subject. Nothing can pass through the sphere, inside or out, though the subject can breathe normally.
 \end{spellnotes}

\spellsection{Revelation}
\spellschool{Divination (Knowledge)}
\spelllvl{Div 8, Sor/Wiz 9}
\spellrng{\rngmed}
\spelltgt{One creature}
\spelldur{\durshort/1 round}
\spellsave{None}
\spellsr{Yes (Will)}
\begin{spelleffect}
You grant the target a powerful revelatory vision of a possible future. This spell has different effects depending on the version chosen. Creatures without an Intelligence score are not affected by this spell. A successful Will save reduces the duration to 1 round.
\par \subspell{Revelation of Destruction} You inflict a vision of a terrible future upon the target. It takes a \minus4 penalty to attack rolls, checks, saving throws, and AC as it struggles to avoid the certainty of its own doom.
\par \subspell{Revelation of Prowess} You show the target a vision of its success in the combat to come. It gains the benefits of a \spell{greater precognition} spell.
\par \subspell{Revelation of Truth} You show the target the truth of the world around it. It gains the benefits of a \spell{true seeing} spell.
\end{spelleffect}

\spellsection{Reverse Gravity}
\spellschool{Transmutation}
\spelllvl{Air 8, Trickery 8, Sor/Wiz 8}
\spellrng{\rngclose}
\spellarea{Up to five 10 ft. cubes (S)}
\spelldur{Concentration (up to 5 rounds)}
\spellsave{None; see text}
\spellsr{No}
\begin{spelleffect}
This spell reverses gravity in an area, causing all unattached objects and creatures within that area to fall upward and reach the top of the area in 1 round. If some solid object (such as a ceiling) is encountered in this fall, falling objects and creatures strike it in the same manner as they would during a normal downward fall. If an object or creature reaches the top of the area without striking anything, it remains there, oscillating slightly, until the spell ends. At the end of the spell duration, affected objects and creatures fall downward.
\par A creature caught in the area can attempt a Reflex save to react to the shift in gravity. Common reactions include securing oneself if possible, or jumping to reach more stable ground.
\end{spelleffect}
\begin{spellnotes}
Creatures who can fly or levitate can keep themselves from falling, though the shift in gravity can be disorienting. A creature that reacts by jumping does not actually move until its turn, but it moves in the direction of its jump, rather than simply falling upwards.
\end{spellnotes}

\spellsection{Scintillating Pattern}
\spelldesc{You create a massive spread of colorful lights that spin and whirl in a complex pattern that bewilders your foes.}
\spellschool{Enchantment/Illusion (Compulsion, Figment) [Mind-Affecting, Sight-Dependent]}
\spelllvl{Sor/Wiz 8}
\spellarea{\arealarge radius spread centered on you}
\spelleff{Colorful lights in the area}
\spelldur{\durshort}
\spellsave{None}
\spellsr{Yes (Will)}
\begin{spelleffect}
All enemies within the spell's area are bewildered for as long as they can see the lights, and for 5 rounds thereafter. In addition, the area is illuminated in bright light out to a 100 ft. radius, and dim light extends an additional 100 ft. beyond that.
\end{spelleffect}
\begin{spellnotes}
A bewildered creature takes a \minus2 penalty to attack rolls, saving throws, checks, DCs, and AC. Your allies, and creatures unable to see the lights, are unaffected.
\end{spellnotes}

\spellsection{Sea of Fog}
\spellschool{Conjuration (Creation)}
\spelllvl{Drd 8, Sor/Wiz 8}
\spellarea{200 ft. radius spread centered on you, 50 ft. high}
\spelleff{Fog in the area}
\begin{spelleffect}
This spell functions like \spell{obscuring mist}, except that the effect is much larger.
\end{spelleffect}
\begin{spellnotes}
A severe wind disperses the fog within 1 minute, a windstorm disperses it within 5 rounds, and a hurricane disperses it within a round.
\end{spellnotes}

\spellsection{Shatter}
\spelldesc{You create a loud, ringing noise that sunders solid objects.}
\spellschool{Evocation (Energy) [Sonic]}
\spelllvl{Brd 2, Destruction 2, Sor/Wiz 2}
\spellrng{\rngclose}
\spelltgtorarea{One solid object or one crystalline creature; or \areasmall radius spread}
\spelldur{Instantaneous}
\spellsave{Will negates (object)/Will negates (object) or Fortitude half; see text}
\spellsr{Yes (Will)}
\spelldmg{4d6 sonic damage \add d6 per two levels after 4th}
\begin{spelleffect}
Used as an area attack, shatter destroys nonmagical objects of crystal, glass, ceramic, or porcelain. All such objects within a \areasmall radius of the point of origin are smashed into dozens of pieces by the spell. Objects weighing more than 1 pound per your level are not affected, but all other objects of the appropriate composition are shattered.
\par Alternatively, you can target a single solid object or crystalline creature. In the case of large objects, such as walls, you target a 5 ft. cube. The target takes damage, with a Fortitude save for half damage.
\par A creature holding vulnerable objects can attempt a Will save to negate any effect on to those objects.
\end{spelleffect}

\spellsection{Slay Living}
\spelldesc{Your hand seethes with an eerie dark fire as you reach out to touch your foe, instantly snuffing out his life.}
\spellschool{Necromancy (Life) [Death]}
\spelllvl{Clr 5, Death 5}
\spellrng{Touch}
\spelltgt{Living creature touched}
\spelldur{Instantaneous}
\spellsave{Fortitude negates}
\spellsr{Yes (Fortitude)}
\begin{spellhealthy}
The target is staggered for 5 rounds.
\end{spellhealthy}
\begin{spellblood}
The target is instantly slain.
\end{spellblood}

\spellsection{Sleep}
\spellschool{Enchantment (Compulsion) [Mind-Affecting, Sleep]}
\spelllvl{Brd 1, Sor/Wiz 1}
\spellrng{\rngmed}
\spelltgt{One living creature}
\spelldur{\durshort}
\spellsave{Will negates}
\spellsr{Yes (Will)}
\begin{spelleffect}
The subject is fatigued and attempts to go to sleep as soon as possible, though it will not stop fighting to do so. Awakening a creature put to sleep by this spell is difficult, and requires a standard action (an application of the aid another action).
\end{spelleffect}

\spellsectioncomma{Sleep}{Mass}
\spellschool{Enchantment (Compulsion) [Mind-Affecting, Sleep]}
\spelllvl{Brd 4, Sor/Wiz 4}
\spellarea{\areamed radius burst}
\spelltgts{Five creatures within the area}
\begin{spelleffect}
This spell functions like \spell{sleep}, except that it affects multiple creatures.
\end{spelleffect}

\spellsection{Slow}
\spelldesc{You decelerate your enemy's motions, causing her to move and act more slowly than normal.}
\spellschool{Transmutation (Temporal)}
\spelllvl{Brd 2, Sor/Wiz 2}
\spellrng{\rngclose}
\spelltgt{One creature}
\spelldur{\durshort}
\spellsave{Will negates}
\spellsr{Yes (Will)}
\begin{spelleffect}
The subject is slowed. This has several effects.
\par A slowed creature can take only a single move action or standard action each turn, but not both (nor may it take full-round actions).
\par A slowed creature takes a \minus2 penalty to attack rolls, Strength and Dexterity-based skill checks, and armor class.
\end{spelleffect}
\begin{spellnotes}
\spell{Slow} counters and dispels \spell{haste}.
\end{spellnotes}

\spellsectioncomma{Slow}{Mass}
\spelldesc{You decelerate your enemies' motions, causing them to move and act more slowly than normal.}
\spellschool{Transmutation (Temporal)}
\spelllvl{Sor/Wiz 7}
\spelltgts{Five creatures in an \areamed radius}
\begin{spelleffect}
This spell functions like \spell{slow}, except that it affects multiple creatures.
\end{spelleffect}

\spellsection{Solid Fog}
\spellschool{Conjuration (Creation)}
\spelllvl{Druid 6, Sor/Wiz 6, Water 6}
\spelldur{\durmed}
\spellsr{No}
\begin{spelleffect}
This spell functions like \spell{fog cloud}, but in addition to obscuring sight, the fog is so thick that any creature attempting to move through it progresses at a speed of 5 feet, regardless of its normal speed, and it takes a \minus2 penalty on all melee attack and melee damage rolls. The vapors prevent effective ranged weapon attacks (except for magic rays and the like). A creature or object that falls into solid fog is slowed, so that each 10 feet of vapor that it passes through reduces falling damage by 1d6.
\par A creature in the fog can take a full-round action to make a Strength check, moving 5 feet for every 5 by which the result exceeds DC 0. This movement is affected by any other effects which impede movement, as normal.
\end{spelleffect}
\begin{spellnotes}
A severe wind (31\add mph) disperses the fog in 5 rounds, and a hurricane force wind disperses the fog in 1 round.
\par Solid fog can be made permanent with a permanency spell. A permanent solid fog dispersed by wind reforms in 10 minutes.
\end{spellnotes}

\spellsection{Song of Discord}
\spellschool{Enchantment (Compulsion) [Auditory, Mind-Affecting]}
\spelllvl{Brd 6}
\spellrng{\rngmed}
\spellarea{\areamed radius spread}
\spelldur{\durshort}
\spellsave{Will negates}
\spellsr{Yes (Will)}
\begin{spelleffect}
This spell causes all creatures within the area to turn on each other rather than attack their foes. Each affected creature has a 50\% chance to attack the nearest target each round. (Roll to determine each creature's behavior every round at the beginning of its turn.) A creature that does not attack its nearest neighbor is free to act normally for that round. After each round that a subject is compelled to attack the nearest target, it may make a saving throw to throw off the effect.
\par Creatures forced by a \spell{song of discord} to attack their fellows employ all methods at their disposal, choosing their deadliest spells and most advantageous combat tactics. They do not, however, harm targets that have fallen unconscious.
\end{spelleffect}
\begin{spellnotes}
Creatures with HV in excess of your caster level are immune to this spell.
\end{spellnotes}

\spellsection{Spike Growth}
\spellschool{Transmutation (Alteration)}
\spelllvl{Drd 1}
\spellrng{\rngmed}
\spellarea{\areasmall radius}
\spelldur{\durshort (D)}
\spellsave{None/Reflex negates}
\spellsr{Yes (Reflex)}
\begin{spelleffect}
Any ground-covering vegetation in the spell's area becomes very hard and sharply pointed. In areas of bare earth, roots and rootlets act in the same way. Typically, spike growth can be cast in any outdoor setting except open water, ice, heavy snow, sandy desert, or bare stone. Any foe moving on foot into or through the spell's area takes 1d4 points of physical piercing damage for each 5 feet of movement through the spiked area. Allies suffer no ill effects.
\par Any creature that takes damage from this spell must also succeed on a Reflex save or suffer injuries to its feet and legs that slow its land speed by one-half. The Reflex save must be repeated each round that the creature moves through the area. This speed penalty lasts for 12 hours or until the injured creature receives magical healing. Another character can remove the penalty by taking 10 minutes to dress the injuries and succeeding on a Heal check against the spell's save DC.
\end{spelleffect}

\spellsection{Spike Stones}
\spellschool{Transmutation (Alteration) [Earth]}
\spelllvl{Drd 4, Earth 4}
\spellarea{\areamed radius}
\begin{spelleffect}
This spell functions like \spell{spike growth}, except that it deals d8 physical piercing damage to creatures moving through it and it can also be cast on rocky ground, stone floors, and similar surfaces.
\end{spelleffect}

\spellsection{Stinking Cloud}
\spellschool{Conjuration (Creation)}
\spelllvl{Sor/Wiz 5}
\spellsave{None/Fortitude negates}
\begin{spelleffect}
This spell functions like \spell{fog cloud}, except that creatures within the cloud are sickened. A successful Fortitude save negates the sickening. The condition lasts as long as the creature remains in the cloud and for 5 rounds after it leaves.
\end{spelleffect}
\begin{spellnotes}
Any creature that succeeds on its save but remains in the cloud must continue to save each round on your turn. \spellindirect{stinking cloud}{Stinking cloud} can be made permanent with a \spell{permanency} spell. A permanent \spell{stinking cloud} dispersed by wind reforms in 10 minutes.
\end{spellnotes}

\spellsection{Suggestion}
\spellschool{Enchantment (Compulsion) [Language-Dependent, Mind-Affecting]}
\spelllvl{Brd 4, Sor/Wiz 5}
\spellcmp{V, M}
\spellrng{\rngclose}
\spelltgt{One living creature}
\spelldur{\durext or until completed}
\spellsave{Will negates}
\spellsr{Yes (Will)}
\begin{spelleffect}
You influence the actions of the target creature by suggesting a course of activity (limited to a sentence or two). The suggestion must be worded in such a manner as to make the activity sound reasonable. Asking the creature to do some obviously harmful act automatically negates the effect of the spell. Additionally, any obvious threat, such as someone drawing a weapon, casting a spell, or aiming a ranged weapon at the fascinated creature, grants the creature a new saving throw with a \plus5 bonus.
\par The suggested course of activity can continue for the entire duration. If the suggested activity can be completed in a shorter time, the spell ends when the subject finishes what it was asked to do. You can instead specify conditions that will trigger a special activity during the duration. If the condition is not met before the spell duration expires, the activity is not performed.
\end{spelleffect}
\begin{spellnotes}
\par A very reasonable suggestion can cause the save to be made with a \minus2 or greater penalty. A creature that makes its saving throw against \spell{suggestion} is immune to all further attempts by the same spellcaster for 24 hours.
\end{spellnotes}

\spellsectioncomma{Suggestion}{Mass}
\spellschool{Enchantment (Compulsion) [Language-Dependent, Mind-Affecting]}
\spelllvl{Brd 6, Sor/Wiz 8}
\spelldur{\durshort}
\spellarea{\areamed radius limit}
\spelltgts{Five creatures within the area}
\begin{spelleffect}
This spell functions like \spell{suggestion}, except that it can affect multiple creatures and has a shorter duration. The same suggestion applies to all subjects.
\end{spelleffect}

\spellsection{Telekinesis}
\spelldesc{You move objects or creatures by concentrating on them.}
\spellschool{Evocation (Control)}
\spelllvl{Evoc 5}
\spellrng{\rngmed}
\spelltgtortgts{See text}
\spelldur{Concentration, up to \durmed/Instantaneous; see text}
\spellsave{Will negates (object)/None; see text}
\spellsr{Yes (Will)/None; see text}
\begin{spelleffect}
Depending on the version selected, the spell can provide a gentle, sustained force, perform a variety of combat maneuvers, or exert a single short, violent thrust.
\par \subspell{Sustained Force} As the \spell{telekinetic force} spell.

\par \subspell{Combat Maneuver} As the \spell{telekinetic maneuver} spell.

\par \subspell{Violent Thrust} Alternatively, the spell energy can be spent in a single round, as the \spell{telekinetic thrust} spell.
\end{spelleffect}

\spellsection{Telekinetic Force}
\spellschool{Evocation (Control)}
\spelllvl{Sor/Wiz 3}
\spellrng{\rngmed}
\spelltgt{One object at a time}
\spelldur{Concentration, up to \durmed}
\spellsave{Will negates (object); see text}
\spellsr{Yes (Will)}
\begin{spelleffect}
You move an object by concentrating your mind upon its current location and then the location you desire, creating a sustained force. You can move an object weighing no more than 25 pounds per caster level up to 20 feet per round. A creature can negate the effect on an object it possesses with a successful Will save or with spell resistance. The weight can be moved vertically, horizontally, or in both directions. An object cannot be moved beyond your range. The spell ends if the object is forced beyond the range. If you cease concentration for any reason, the object falls or stops. Each round, the subject can attempt a new saving throw to negate the effect.
\par An object can be telekinetically manipulated as if with one hand. For example, a lever or rope can be pulled, a key can be turned, an object rotated, and so on, if the force required is within the weight limitation. You might even be able to untie simple knots, though delicate activities such as these require Intelligence checks.
\par You can drop a weight and pick up another during the spell's duration, as long as you don't stop concentrating on maintaining the power. An object can be telekinetically manipulated as if you were moving it with one hand.
\par If you spend at least 5 rounds concentrating on an unattended object, you can attempt to break or burst it as if making a Strength check, except that you apply your casting attribute modifier to the check instead of your Strength modifier.
\end{spelleffect}

\spellsection{Telekinetic Maneuver}
\spellschool{Evocation (Control)}
\spelllvl{Sor/Wiz 3}
\spellrng{\rngmed}
\spelltgt{One creature}
\spelldur{Concentration, up to \durmed}
\spellsave{None}
\spellsr{Yes (Will)}
\begin{spelleffect}
You can affect a foe by concentrating your mind upon its current status and the status you desire, once per round. You can perform a bull rush, a disarm, a dirty trick, a grapple (including a pin), or a trip. Resolve these attempts as normal, except that they don't provoke attacks of opportunity, you use your caster level in place of your base attack bonus, and you use your casting attribute modifier in place of your Strength modifier. No save is allowed against these attempts, but spell resistance is applied normally.
\end{spelleffect}

\spellsection{Telekinetic Thrust}
\spellschool{Evocation (Control)}
\spelllvl{Evoc 5}
\spellrng{\rngmed}
\spelltgtortgts{Five objects or creatures in a \areamed radius \add one per four caster levels after 8th}
\spelldur{Instantaneous}
\spellsave{Will negates (object); see text}
\spellsr{Yes (Will)}
\begin{spelleffect}
You can throw the affected objects or creatures anywhere within the spell's range. All subjects of this spell must be thrown to the same place. You can hurl up to a total weight of 25 pounds per caster level.
\par You must succeed on ranged attack rolls (one per creature or object thrown) to hit the target of the hurled items with the items, applying your caster level and casting attribute modifier to the attack roll instead of your base attack bonus and Dexterity modifier. Hurled weapons deal their normal damage. Other objects deal damage ranging from 1 point per 25 pounds of weight (for less dangerous objects such as an empty barrel) to 1d6 points per 25 pounds of weight (for hard, dense objects such as a boulder).
\par Creatures are allowed Will saves (and spell resistance) to avoid being hurled or having their held possessions be targeted by this power.
\par If you use this power to hurl a creature against a solid surface, it takes damage as if it had fallen 50 feet (5d6 points).
\end{spelleffect}

\spellsection{Temporal Stasis}
\spellschool{Transmutation (Temporal)}
\spelllvl{Sor/Wiz 8}
\spellcmp{V, S, M}
\spellrng{Touch}
\spelltgt{Creature touched}
\spelldur{\durshort/Permanent}
\spellsave{None/Will negates}
\spellsr{Yes (Will)}
\begin{spelleffect}
If you succeed on a melee touch attack, the subject is slowed for a \durshort duration.
\end{spelleffect}
\begin{spellblood}
As the healthy effect, and the subject into a state of suspended animation unless it makes a successful Will save. For the creature, time ceases to flow and its condition becomes fixed. The creature does not grow older. Its body functions virtually cease, and no force or effect can harm it. This state persists until the magic is removed (such as by a successful \spell{dispel magic} spell or an \spell{emancipation} spell).
\end{spellblood}

\spellsection{Touch of Fatigue}
\spelldesc{You channel negative energy through your touch, fatiguing the target.}
\spellschool{Necromancy (Unlife) [Negative]}
\spelllvl{Sor/Wiz 0}
\spellrng{Touch}
\spelltgt{Creature touched}
\spelldur{\durshort}
\spellsave{Fortitude partial}
\spellsr{Yes (Fortitude)}
\begin{spelleffect}
The touched creature is fatigued. A successful save reduces the duration to 1 round.
\end{spelleffect}
\begin{spellnotes}
This spell has no effect on a creature that is already fatigued.
\end{spellnotes}

\spellsection{Touch of Idiocy}
\spellschool{Enchantment/Necromancy (Unlife) [Mind-Affecting]}
\spelllvl{Sor/Wiz 2}
\spellrng{Touch}
\spelltgt{Living creature touched}
\spelldur{\durlong}
\spellsave{Will half}
\spellsr{Yes (Will)}
\begin{spelleffect}
With a touch, you reduce the target's mental faculties. Your successful melee touch attack applies a \minus4 penalty to the target's Intelligence, Wisdom, and Charisma scores. This penalty can't reduce any of these scores below 1.
\end{spelleffect}
\begin{spellnotes}
This spell's effect may make it impossible for the target to cast some or all of its spells, if the requisite attribute score drops below the minimum required to cast spells of that level.
\end{spellnotes}

\spellsection{Transfer Suffering}
\spellschool{Necromancy (Life) [Healing]}
\spelllvl{Sor/Wiz 4}
\spellrng{Close}
\spelleff{Ray}
\spelldmg{8d8 life damage \add d8 per two caster levels above 8th}
\begin{spelleffect}
This spell functions like \spell{lesser transfer suffering}, except that it deals more damage and affects a target struck by a ray.
\end{spelleffect}
\begin{spellnotes}
If the ray misses, you are do not heal any damage.
\end{spellnotes}

\spellsectioncomma{Transfer Suffering}{Lesser}
\spelldmg{You transfer the wounds that you have taken to your foe, forcing it to suffer in your stead.}
\spellschool{Necromancy (Life) [Healing]}
\spelllvl{Sor/Wiz 2}
\spellrng{Touch}
\spelltgt{Living creature touched}
\spelldur{Instantaneous}
\spellsave{Will half}
\spellsr{Yes (Will)}
\spelldmg{4d6 life damage \add d6 per two caster levels above 4th}
\begin{spelleffect}
The touched creature takes damage, and you immediately regain hit points equal the amount of damage you transfer. You cannot transfer more damage than you have taken, and you cannot use this spell to gain hit points in excess of your full normal total.
\end{spelleffect}

\spellsection{Transmute Flesh and Stone}
\spellschool{Transmutation (Alteration)}
\spelllvl{Sor/Wiz 6}
\spellrng{\rngmed}
\spelltgt{One creature or a cylinder of stone from 1 ft. to 3 ft. in diameter and up to 10 ft. long}
\spelldur{\durshort/Instantaneous}
\spellsave{Fortitude negates (object); see text}
\spellsr{Yes (Fortitude)}
\begin{spelleffect}
This spell has different effects depending on the version chosen.
\par \subspell{Flesh to Stone} The subject is slowed for the duration of the spell, and takes 2d6 physical damage each round as its body gradually turns to stone. A Fortitude save negates this effect. If the subject reaches 0 hit points before the spell ends, it becomes a mindless, inert statue, along with all its carried gear. If the statue resulting from this effect is broken or damaged, the subject (if ever returned to its original state) has similar damage or deformities. The creature is not dead, but it is not considered alive either.
\par Only creatures made of flesh are affected by this effect.
\par \subspell{Stone to Flesh} This effect restores a petrified creature to its normal state, restoring life and goods. The creature must make a DC 15 Fortitude save to survive the process. Any petrified creature, regardless of size, can be restored. A restored creature has as many hit points as it had when it was petrified. Stone which was not originally a petrified creature is unaffected.
\end{spelleffect}

\spellsection{Undeath to Death}
\spellschool{Necromancy (Vitalism) [Positive]}
\spelllvl{Clr 6, Sor/Wiz 6}
\spellarea{\areamed radius limit}
\spelltgts{Several undead creatures within the area}
\spellsave{Will negates}
\begin{spelleffect}
This spell functions like \spell{circle of death}, except that it destroys undead creatures.
\end{spelleffect}
\spellmat{The powder of a crushed diamond worth at least 750 gp.}

\spellsection{Wail of the Banshee}
\spelldesc{You emit a terrible scream that kills anyone that hears it.}
\spellschool{Necromancy (Life) [Death, Sound-Dependent]}
\spelllvl{Death 9, Necro 9}
\spellcmp{V}
\spelltgts{Living creatures in a \arealarge spread centered on you, up to Five creatures}
\spelldur{Concentration, up to 2 rounds; see text}
\spellsave{Fortitude negates}
\spellsr{Yes (Fortitude)}
\begin{spellhealthy}
The subjects are sickened for 5 rounds. If you concentrate for a second round, subjects still in the area are nauseated for 1 round.
\end{spellhealthy}
\begin{spellblood}
The subjects are nauseated for 1 round. If you concentrate for a second round, subjects still in the area immediately die.
\end{spellblood}
\begin{spellnotes}
This spell affects a maximum number of creatures equal to your caster level. Creatures closest to you are affected first, so creatures farther away may be unaffected if there are enough intervening creatures. Each creature makes only one saving throw against the effect.
\end{spellnotes}

\spellsection{Warp Wood}
\spellschool{Transmutation (Alteration)}
\spelllvl{Destruction 2, Drd 2}
\spellrng{\rngclose}
\spellarea{\areamed radius limit}
\spelltgt{1 Small nonmagical wooden object/level within the area}
\spelldur{Instantaneous}
\spellsave{Will negates (object)}
\spellsr{Yes (Will)}
\begin{spelleffect}
You cause wood to bend and warp, permanently destroying its straightness, form, and strength. A warped door springs open (or becomes stuck, requiring a Strength check to open, at your option). A boat or ship springs a leak. Warped ranged weapons are useless. A warped melee weapon causes a \minus4 penalty on attack rolls.
\par You may warp one Small or smaller object or its equivalent per caster level. A Medium object counts as two Small objects, a Large object as four, a Huge object as eight, a Gargantuan object as sixteen, and a Colossal object as thirty-two.
\par Alternatively, you can unwarp wood (effectively warping it back to normal) with this spell, straightening wood that has been warped by this spell or by other means. \spellindirect{make whole}{Make whole}, on the other hand, does no good in repairing a warped item.
\end{spelleffect}
\begin{spellnotes}
You can combine multiple consecutive \spell{warp wood} spells to warp (or unwarp) an object that is too large for you to warp with a single spell. Until the object is completely warped, it suffers no ill effects.
\end{spellnotes}

\spellsection{Waves of Exhaustion}
\spellschool{Necromancy (Flesh)}
\spelllvl{Sor/Wiz 8, War 8}
\spellarea{\areamed cone-shaped burst}
\spelldur{\durshort}
\spellsave{No}
\spellsr{Yes (Fortitude)}
\begin{spelleffect}
Living creatures in the area are exhausted. This spell has no effect on a creature that is already exhausted.
\end{spelleffect}

\spellsection{Waves of Fatigue}
\spellschool{Necromancy (Flesh)}
\spelllvl{Sor/Wiz 5, War 5}
\spellarea{\arealarge cone-shaped burst}
\spelldur{\durshort}
\spellsave{No}
\spellsr{Yes (Fortitude)}
\begin{spelleffect}
Living creatures in the area are fatigued. This spell has no effect on a creature that is already fatigued.
\end{spelleffect}

\spellsection{Web}
\spelldesc{You create a many-layered mass of strong, sticky strands that entangle creatures caught within them. The strands are similar to spider webs, but larger and tougher.}
\spellschool{Conjuration (Creation)}
\spelllvl{Sor/Wiz 3}
\spellrng{\rngclose}
\spellarea{\areamed radius spread}
\spelleff{Webs in the area}
\spelldur{\durshort (D)}
\spellsave{Reflex negates; see text}
\spellsr{No}
\begin{spelleffect}
Each creature in the spell's area are entangled unless it makes a successful Reflex save. This save must be repeated each round that the creature moves or fights within the area. An entangled creature can spend a standard action to make a grapple attack or Escape Artist attempt against the spell's save DC to break the webs holding it, preventing it from being entangled. A creature entangled by the spell remains entangled until it breaks the webs holding it or escapes the spell's area.
\par If the strands can be anchored to two or more solid and diametrically opposed structures, such as walls, the strands are much more sturdy. A creature entangled within a sturdy web is unable to move from its square until it stops being entangled.
\end{spelleffect}
\begin{spellnotes}
An entangled creature moves at half speed, cannot run or charge, and takes a \minus2 penalty to attack rolls, Strength and Dexterity-based checks, and armor class. If it attempts to cast a spell must make a Concentration check (DC 10 \add double the spell's level) or lose the spell.
The strands are too widely spaced to significantly obscure sight, but are flammable. A magic flaming sword can slash them away as easily as a hand brushes away cobwebs. Any fire can set the webs alight and burn away 5 square feet in 1 round. All creatures within flaming webs take 2d4 points of fire damage from the flames.
\spell{Web} can be made permanent with a \spell{permanency} ritual. A permanent \spell{web} that is destroyed regrows in 10 minutes.
\end{spellnotes}

\spellsection{Weird}
\spellschool{Enchantment/Illusion (Emotion, Phantasm) [Death, Fear, Mind-Affecting, Unreal]}
\spelllvl{Sor/Wiz 9, Trickery 9}
\spellarea{\areamed radius limit}
\spelltgts{Five creatures within the area}
\begin{spelleffect}
This spell functions like \spell{phantasmal killer}, except that it affects multiple creatures.
\end{spelleffect}

\spellsection{Word of Chaos}
\spellschool{Evocation (Channeling) [Chaotic]}
\spelllvl{Chaos 7}
\spellcmp{V}
\spellarea{\arealarge radius spread centered on you}
\spelldur{Instantaneous}
\spellsave{None or Will negates; see text}
\spellsr{Yes (Will)}
\begin{spellhealthy}
Each nonchaotic creature in the area is bewildered for 5 rounds.
\end{spellhealthy}
\begin{spellblood}
Each nonchaotic creature in the area suffers one or more of the following ill effects, depending on its Hit Values.
\begin{dtable}
\begin{tabularx}{\columnwidth}{l >{\lcol}X}
\par \thead{HV} & \thead{Effect} \\
\par Equal to caster level & Bewildered \\
\par Up to caster level \minus5 & Confused, bewildered \\
\par Up to caster level \minus10 & Paralyzed, nauseated, sickened \\
\par Up to caster level \minus15 & Killed\fn{1}
\end{tabularx}
1 Living creatures die. Nonliving creatures are destroyed.
\end{dtable}
\par \subspell{Bewildered} The creature is bewildered for 5 rounds.
\par \subspell{Confused} The creature is confused for 2 rounds.
\par \subspell{Paralyzed} The creature is paralyzed and helpless for 5 rounds.
\par \subspell{Killed} Living creatures die. Nonliving creatures are destroyed.
\end{spellblood}
\begin{spellnotes}
A bewildered creature takes a \minus2 penalty to attack rolls, saving throws, checks, DCs, and AC. Creatures whose Hit Values exceed your caster level are unaffected by \spell{word of chaos}.
\end{spellnotes}

\spellsection{Zone of Truth}
\spellschool{Enchantment (Inhibition) [Mind-Affecting]}
\spelllvl{Clr 2, Law 2, Pal 2}
\spellrng{\rngclose}
\spellarea{\areamed radius emanation}
\spelldur{\durmed}
\spellsave{Will negates}
\spellsr{Yes (Will)}
\begin{spelleffect}
Creatures within the emanation area (or those who enter it) can't speak any deliberate and intentional lies. Each potentially affected creature is allowed a save to avoid the effects when the spell is cast or when the creature first enters the emanation area. Affected creatures are aware of this enchantment. Therefore, they may avoid answering questions to which they would normally respond with a lie, or they may be evasive as long as they remain within the boundaries of the truth. Creatures who leave the area are free to speak as they choose.
\end{spelleffect}