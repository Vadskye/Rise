\chapter{Magic Items}
Magic items are objects that have been imbued with magical energy. They can take almost any form, and their potential uses are only as limited as the magic that created them.

\section{Magic Item Types}
Magic items are divided into three broad categories:
\begin{itemize*}
  \item Apparel items provide access to their powers while worn. A \mitem{flaming burst full plate} and a \mitem{ring of protection} are apparel items.
  \item Implements provide access to their powers when wielded. A \mitem{flaming longsword} and a \mitem{wand of fire} are implements.
  \item Tools provide access to their powers when used in some way. A \mitem{bag of holding} is a tool.
\end{itemize*}

\parhead{Apparel} There are a wide variety of magic items, but there is a limit to how many apparel items a character can wear at any given time. For humanoid-shaped creatures, there are four core areas on the body that can hold magic items: the arms, the head, the torso, and the legs. A creature can wear only one item in the hands, arms, and legs, but it can use two torso items, provided that they are not identical (such as two belts). In addition, it can wear a suit of armor and two magic rings (one on each hand), for a total of eight item slots. Any additional items worn do not function.

Creatures with non-humanoid body structures may have different arrangement of items. For example, a wolf would be able to wear items on its head, its torso, and on two different sets of legs. Regardless of body shape or size, most creatures may not have more than eight item slots, and many have fewer. A list of common items and their placement on the body is given below.

\begin{itemize*}
  \item Arm items:
    \begin{itemize*}
      \item Bracers, bracelets, gauntlets, and gloves.
    \end{itemize*}
  \item Armor
  \item Head items:
    \begin{itemize*}
      \item Hats, headbands, and helmets
    \end{itemize*}
  \item Leg items:
    \begin{itemize*}
      \item Boots and shoes.
    \end{itemize*}
  \item Rings
  \item Torso items:
    \begin{itemize*}
      \item Amulets, belts, cloaks, mantles, necklaces, robes, shirts, and vests.
    \end{itemize*}
\end{itemize*}

Of course, a character may carry or possess as many items of the same type as he wishes. However, additional equipped items beyond those listed above have no effect.

A rare few apparel items can be ``worn'' without taking up space on a character's body. The description of an item indicates when it has this property.

\parhead{Implements} The most common implements are weapons and shields. Spellcasters also often use wands and staves to enhance their power.

\parhead{Tools} Tools can come in many varieties.

\section{Using Magic Items}

To use a magic item, it must be activated, although sometimes activation simply means wearing or holding the item.

\subsection{Activation Methods}
Magic items can be activated in one of four ways:

\begin{itemize*}
  \item Command word
  \item Specific action
  \item Spell completion
  \item Triggered
\end{itemize*}

These methods are described below.

\parhead{Command Word} A character must speak a special word defined by the item to activate it. Unless otherwise stated, activating a command word magic item is a standard action and does not provoke attacks of opportunity.

\parhead{Specific Action} A character must perform a specific action defined by the item to activate it. For example, a creature might need to drink the item, wrap its cloak around itself, or perform some other task. The activation time for such items can be varied. Unless otherwise stated, activating a specific action magic item is a standard action and does not provoke attacks of opportunity.

\parhead{Spell Completion} A character must complete a spell defined by the item to activate it. Unless otherwise stated, activating a spell completion magic item is a standard action and provokes attacks of opportunity as normal for casting. Both verbal and somatic components may be required, as appropriate to the spell to be completed.

In order to activate a spell completion item, a character must be able to cast spells of that level and he must have that spell on his spell list.

\parhead{Triggered} A creature must fulfill some triggering condition defined by the item to activate it. For example, a triggered magic item might activate when a character strikes a foe, is damaged, or is affected by a particular kind of magic. Unless otherwise stated, activating a triggered item is an immediate action that does not provoke attacks of opportunity. It is done completely mentally, requiring no physical action, so it can be done even while paralyzed. Some triggered items activate automatically, without requiring an action of any kind.

\section{Magic Item Effects}
Most magic items either provide numerical bonuses or emulate the effect of a spell in some way.

\subsection{Saving Throws}

If a magic item allows a saving throw against its effects, the DC is listed in the item's description. Typically, the DC is equal to 10 \add the character level of the item's wielder \add the minimum attribute required to cast a spell of the same level as the item's effect.

\begin{comment}
\subsection{Charges, Doses, and Multiple Uses}

Many items, particularly wands and staffs, are limited in power by the number of charges they hold. Normally, charged items have 50 charges at most. If such an item is found as a random part of a treasure, roll d\% and divide by 2 to determine the number of charges left (round down, minimum 1). If the item has a maximum number of charges other than 50, roll randomly to determine how many charges are left.

Prices listed are always for fully charged items. (When an item is created, it is fully charged.) An item with no charges left is worth half the price of a fully charged item. For an item that's worthless when its charges run out (which is the case for almost all charged items), the value of the partially used item is proportional to the number of charges left. For an item that has usefulness in addition to its charges, only part of the item's value is based on the number of charges left.
\end{comment}

\section{Magic Item Description Format}

Each general type of magic item gets an overall description, followed by descriptions of specific items.

General descriptions include notes on activation, random generation, and other material. The AC, hardness, hit points, and break DC are given for typical examples of some magic items. The AC assumes that the item is unattended and includes a \minus10 penalty for the item's effective Dexterity of \minus10. If a creature holds the item, use the creature's Dexterity in place of the \minus10 penalty.

Some individual items, notably those that simply store spells and nothing else, don't get full-blown descriptions. Reference the spell's description for details, modified by the form of the item (potion, scroll, wand, and so on). Assume that the spell is cast at the minimum level required to cast it.

Items with full descriptions have their powers detailed, and each of the following topics is covered in notational form at the end of the description.

\begin{itemize*}
\itemhead{Aura:} Most of the time, a Spellcraft check will reveal the school of magic associated with a magic item and the strength of the aura an item emits. This information (when applicable) is given at the beginning of the item's notational entry. See the Spellcraft skill for details.

\itemhead{Caster Level:} The next item in a notational entry gives the caster level of the item, indicating its relative power. The caster level determines the item's saving throw bonus, as well as other level-dependent aspects of the powers of the item (if variable). It also determines the level that must be contended with should the item come under the effect of a dispel magic spell or similar situation. This information is given in the form ``CL x,'' where ``CL'' is an abbreviation for caster level and ``x'' is a number representing the caster level itself.

For potions, scrolls, and wands, the creator can set the caster level of an item at any number high enough to cast the stored spell and not higher than her own caster level. For other magic items, the caster level is determined by the item itself. In this case, the creator's caster level must be as high as the item's caster level (and prerequisites may effectively put a higher minimum on the creator's level).

\itemhead{Requirements:} The qualifications that must be met to create the item, described in \pcref{Creating Magic Items}.

\itemhead{Market Price:} This gold piece value, given following the word ``Price,'' represents the price someone should expect to pay to buy the item. The market price for an item that can be constructed with an item creation feat is usually equal to the base price plus the price for any components.

\itemhead{Cost to Create:} The next part of a notational entry is the cost in gp to create the item, given following the word ``Cost.'' This information appears only for items with components which make their market prices higher than their base prices. The cost to create includes the costs derived from the base cost plus the costs of the components.

Items without components do not have a ``Cost'' entry. For them, the market price and the base price are the same. The cost in gp is 1/2 the market price.

\itemhead{Weight:} The notational entry for many wondrous items ends with a value for the item's weight. When a weight figure is not given, the item has no weight worth noting (for purposes of determining how much of a load a character can carry).
\end{itemize*}

\section{Apparel}

\begin{dtable}
    \lcaption{Armor and Shield Special Abilities}
    \begin{tabularx}{\columnwidth}{>{\lcol}X l l l}
        \thead{Special Ability} & \thead{Cost} & \thead{Item Level} & \thead{Location}\\
        Bashing, Lesser & 200 gp & 3rd & Shield \\
        Energy Resistance, Lesser & 200 gp & 3rd & Armor\\
        Flaming Burst & 800 gp & 5th & Armor, Shield \\
        Freezing Burst & 800 gp & 5th & Armor, Shield \\
        Glamered & 800 gp & 5th & Armor \\
        Bashing & 1,000 gp & 6th & Shield \\
        Ghost Touch & 1,000 gp & 7th & Armor, Shield \\
        Energy Resistance & 2,000 gp & 8th & Armor\\
        Shocking Burst & 2,000 gp & 8th & Armor, Shield \\
        Invulnerability & 2,000 gp & 12th & Armor \\
        Bashing, Greater & 5,000 gp & 10th & Shield \\
        Spell Resistance & 5,000 gp & 10th & Armor \\
        Invulnerability, Greater & 30,000 gp & 14th & Armor \\
        Spell Reflecting & 60,000 gp & 16th & Shield \\
    \end{tabularx}
\end{dtable}

\begin{dtable}
    \lcaption{Other Apparel Special Abilities}
    \begin{tabularx}{\columnwidth}{>{\lcol}X l l l}
        \thead{Special Ability} & \thead{Cost} & \thead{Item Level} & \thead{Location}\\ 
        Ring of Protection \plus1 & 100 gp & 2nd & Ring \\
        Ring of Energy Resistance, Lesser & 200 gp & 3rd & Ring \\
        Boots of Elvenkind & 300 gp & 5th & Legs \\
        Ring of Protection \plus2 & 500 gp & 4th & Ring \\
        Belt of Constitution & 2,000 gp & 8th & Torso \\
        Circlet of Wisdom & 2,000 gp & 8th & Head \\
        Cloak of Charisma & 2,000 gp & 8th & Torso \\
        Gauntlets of Ogre Strength & 2,000 gp & 8th & Arms \\
        Gloves of Dexterity & 2,000 gp & 8th & Arms \\
        Headband of Intellect & 2,000 gp & 8th & Head \\
        Boots of Mobility & 1,000 gp & 6th & Legs \\
        Ring of Sustenance & 1,000 gp & 6th & Ring \\
        Ring of Energy Resistance & 2,000 gp & 8th & Ring \\
        Ring of Protection \plus3 & 2,500 gp & 8th & Ring \\
        Boots of Speed & 5,000 gp & 10th & Legs \\
        Belt of Constitution, Greater & 12,000 gp & 12th & Torso \\
        Circlet of Wisdom, Greater & 12,000 gp & 12th & Head \\
        Cloak of Charisma, Greater & 12,000 gp & 12th & Torso \\
        Gauntlets of Ogre Strength, Greater & 12,000 gp & 12th & Arms \\
        Gloves of Dexterity, Greater & 12,000 gp & 12th & Arms \\
        Headband of Intellect, Greater & 12,000 gp & 12th & Head \\
        Ring of Protection \plus4 & 12,500 gp & 12th & Ring \\
        Ring of Protection \plus5 & 62,500 gp & 16th & Ring \\
    \end{tabularx}
\end{dtable}

\subsection{Armor and Shields}

Magic body armor and shields protect the wearer to a greater extent than their nonmagical equivalents. They always provide an enhancement bonus to a character's armor modifier or shield modifier to AC. In addition to an enhancement bonus, armor may have special abilities or be made of an unusual material.

Body armor is always created so that even if the type of armor comes with boots or gauntlets, these pieces can be switched for other magic boots or gauntlets.

\parhead{Armor Prices}
The prices of enhancement bonuses to armor are listed in \trefnp{Magic Armor and Shields}. If armor has a special ability, the price of the special ability is added to the price of the armor. The number of special abilities on the armor cannot exceed the enhancement bonus of the armor. Additionally, the price of all special abilities cannot exceed twice the price of the enhancement bonus on the armor.

\begin{dtable}
    \lcaption{Magic Armor and Shields}
    \begin{tabularx}{\columnwidth} {>{\ccol}X c c}
        \thead{Enhancement Bonus} & \thead{Base Price} & \thead{Item Level}\\
        \plus1 armor/shield & 100 gp & 2nd \\
        \plus2 armor/shield & 500 gp & 4th \\
        \plus3 armor/shield & 2,500 gp & 8th \\
        \plus4 armor/shield & 12,500 gp & 12th \\
        \plus5 armor/shield & 62,500 gp & 16th\\
    \end{tabularx}
\end{dtable}

\parhead{Caster Level for Armor and Shields} The caster level of a magic shield or magic armor with a special ability is given in the item description. For an item with only an enhancement bonus, the caster level is three times the enhancement bonus. If an item has both an enhancement bonus and a special ability, the higher of the two caster level requirements must be met.

\parhead{Shields} Shield enhancement bonuses do not act as attack or damage bonuses when the shield is used in a bash. However, a shield can be enhanced as a weapon.

\parhead{Hardness and Hit Points} Each \plus1 of enhancement bonus adds 2 to an armor or shield's hardness and \plus10 to its hit points.

\parhead{Activation} A character benefits from magic armor and shields in exactly the way a character benefits from nonmagical armor and shields - by wearing them. Special abilities on body armor are usually activated if the character is struck or damaged, while special abilities on shields are usually activated if the character avoids an attack.

\appdescription{Bashing, Lesser}{200}{3rd}{Shield}{Faint Transmutation (Augment)}{4th}{\x} When a shield with this ability is used to perform a shield bash, it deals damage as if it were one size category larger (so a Medium light shield deals 1d4 damage, and a Medium heavy shield deals 1d6 damage). This is considered an enhancement bonus, and does not stack with similar abilities.

\mitemreq{Transmutation}{Augment}{2}{4th}{Craft (as shield) 8}

\appdescription{Bashing}{1,000}{6th}{Shield}{Faint Transmutation (Augment)}{6th}{\x} When a shield with this ability is used to perform a shield bash, it deals damage as if it were two size categories larger (so a Medium light shield deals 1d6 damage, and a Medium heavy shield deals 1d8 damage). This is considered an enhancement bonus, and does not stack with similar abilities.

\mitemreq{Transmutation}{Augment}{3}{6th}{Craft (as shield) 10}

\appdescription{Bashing, Greater}{5,000}{10th}{Shield}{Faint Transmutation (Augment)}{10th}{\x} This shield functions like a \magicitem{bashing} shield. In addition, you gain a \plus2 bonus to your dodge defense modifier for 1 round against any creature you successfully shield bash with this shield.

\mitemreq{Transmutation}{Augment}{5}{10th}{Craft (as shield) 14}

\appdescription{Energy Resistance, Lesser}{200}{3rd}{Armor}{Faint Abjuration (Shielding)}{2nd}{Immediate (triggered)} When you take energy damage, you can activate this armor to reduce the damage by 5 \add half your level.

\mitemreq{Abjuration}{Shielding}{1}{2nd}{Craft (as armor) 6}

\appdescription{Energy Resistance}{2,000}{8th}{Armor}{Moderate Abjuration (Shielding)}{6th}{Immediate (triggered)} When you take energy damage, you can activate this armor to gain damage reduction equal to 10 \add half your level against that type of energy damage. In addition, the armor casts light as a torch. The color of the light depends on the energy damage resisted: green for acid, blue for cold, yellow for electricity, and red for fire. Both effects lasts for 5 rounds, during which time you cannot activate the item's abilities.

\mitemreq{Abjuration}{Shielding}{3}{6th}{Craft (as armor) 10}

\appdescription{Flaming Burst}{800}{5th}{Armor, Shield}{Faint Evocation (Energy) [Fire]}{4th}{Immediate (triggered)} When you are struck or missed by a melee attack, you can trigger a burst of flames which sear the foe that attacked you. Body armor triggers if the attack hits, and shields trigger if the attack misses.

If you activate the item, you deal 1d10 fire damage per two levels to your attacker (minimum 2d10). In addition, it is ignited for 5 rounds. An ignited creature is vulnerable, causing it to take a \minus2 penalty to attacks, defenses, and checks. In addition, at the end of each of its turns, it takes d6 damage from the fire. If the creature takes a move action, it can attempt a DC 10 Dexterity check to put out the flames. This action requires a free hand. Dropping prone as part of the action gives a \plus5 bonus on this check.

When you activate this ability, you are wreathed in flame, causing you to cast light as a torch. This effect lasts for 5 rounds, during which time you cannot activate the item's abilities.

\mitemreqdesc{Evocation}{Energy}{Fire}{2}{4th}{Craft (as armor) 8}

\appdescription{Freezing Burst}{800}{5th}{Armor, Shield}{Faint Evocation (Energy) [Cold]}{4th}{Immediate (triggered)} When you are struck or missed by a melee attack, you can trigger a frigid burst against the foe that attacked you. Body armor triggers if the attack hits, and shields trigger if the attack misses.

If you activate the item, you deal 1d10 cold damage per two levels to your attacker (minimum 2d10). In addition, it is fatigued for 5 rounds. A fatigued character can neither sprint nor charge and is vulnerable, giving it a \minus2 penalty to attacks, defenses, and checks.

When you activate this ability, you radiate frigid cold, causing you to snuff out torches and other small fires within a 5 foot radius of you. This effect lasts for 5 rounds, during which time you cannot activate the item's abilities.

\mitemreqdesc{Evocation}{Energy}{Cold}{2}{4th}{Craft (as armor) 8}

\appdescription{Ghost Touch}{1,000}{6th}{Armor, Shield}{Faint Conjuration (Translocation) [Planar]}{6th}{\x} This armor or shield seems almost translucent. You apply the full bonus granted by this armor or shield, including its enhancement bonus,against the attacks of incorporeal creatures. It can be picked up, moved, and worn by incorporeal creatures at any time. Incorporeal creatures gain the armor or shield's enhancement bonus against both corporeal and incorporeal attacks, and they can still pass freely through solid objects.

\mitemreqdesc{Conjuration}{Translocation}{Planar}{3}{6th}{Craft (as armor) 10}

\appdescription{Glamered}{800}{5th}{Armor}{Faint Illusion (Glamer)}{4th}{Standard (specific action)} If you trace the symbol of a mask on your chest (a standard action), this armor appears to change shape and form to assume the appearance of a normal set of clothing. You may choose the design of the clothing. The armor retains all its properties (including weight and sound) when glamered. Only a \spell{true seeing} spell or similar magic reveals the true nature of the armor when disguised. The armor remains disguised until you trace the symbol of the mask in the reverse direction, at which point it regains its normal appearance. 

\mitemreq{Illusion}{Glamer}{2}{4th}{Craft (as armor) 8}

\appdescription{Invulnerability}{2,000}{8th}{Armor}{Moderate Abjuration (Shielding)}{8th}{Standard (specific action)} If you strike your chest with a weapon or other hard object (a standard action), this armor grants you physical damage reduction equal to your level for 5 rounds. This allows you to ignore the first points of physical damage you take each round. If you are hit by an adamantine weapon, you cannot use your damage reduction for 1 round.

\mitemreq{Abjuration}{Shielding}{3}{6th}{Craft (as armor) 10}

\appdescription{Invulnerability, Greater}{30,000}{14th}{Armor}{Moderate Abjuration (Shielding)}{12th}{\x} This armor functions like \magicitem{invulnerability} armor, except that the damage reduction lasts for 12 hours.

\mitemreq{Abjuration}{Shielding}{6}{12th}{Craft (as armor) 16}

\appdescriptiondc{Shocking Burst}{2,000}{8th}{Armor, Shield}{Faint Evocation (Energy) [Electricity]}{6th}{Immediate (triggered)}{Fortitude (level \add 3)} When you are struck or missed by a melee attack, you can trigger a powerful jolt of electricity that zaps the foe that attacked you. Body armor triggers if the attack hits, and shields trigger if the attack misses.

If you activate the item, you deal 1d10 electricity damage per two levels to your attacker (minimum 3d10). In addition, if you succeed at a Fortitude attack, it is staggered for 5 rounds. A staggered character may take a single move action or standard action each round, but not both. She cannot take full-round actions, but she may take swift actions. In addition, she is vulnerable, causing her to take a \minus2 penalty on attacks, defenses, and checks.

When you activate this ability, you crackle with electrical energy, causing you to radiate flashing light as a torch. This effect lasts for 5 rounds, during which time you cannot activate the item's abilities.

\mitemreqdesc{Evocation}{Energy}{Electricity}{3}{6th}{Craft (as armor) 10}

\appdescription{Spell Reflecting}{60,000}{16th}{Shield}{Strong Abjuration (Shielding) [Magic]}{14th}{Immediate (triggered)} This shield's surface is completely reflective, allowing it to act as a mirror. When you are targeted by a spell or spell-like ability, you can activate the shield to reflect the spell back at its caster exactly like the \spell{spell turning} spell.

After you activate this ability, the shield's surface becomes dully metallic instead of reflective. This effect lasts for 5 rounds, during which time you cannot activate the item's abilities.

\mitemreqdesc{Abjuration}{Shielding}{Magic}{7}{14th}{Craft (as armor) 18}

\appdescription{Spell Resistance}{5,000}{10th}{Armor}{Moderate Abjuration (Shielding) [Magic]}{8th}{Standard (specific action)} If you crouch low and strike the ground with your fist (a standard action), this armor grants you spell resistance. The spell resistance lasts as long as you remain crouching, and for 5 rounds thereafter (maximum 5 minutes). You can move at half speed while crouching.

To affect a creature with spell resistance using a spell or spell-like ability, a special attack is always required. The defense used against the attack is indicated by the spell. If the attack fails, the spell has no effect.

\mitemreqdesc{Abjuration}{Shielding}{Magic}{4}{8th}{Craft (as armor) 12}

\subsection{Arms}

\appdescription{Gloves of Dexterity}{2,000}{8th}{Arms}{Faint Transmutation (Augment)}{4th}{Standard (specific action)} By rubbing your gloved palms together (a standard action), you command these gauntlets to grant you a \plus2 enhancement bonus to Dexterity for 5 minutes. Unless you can defend yourself without the use of your hands, you provoke attacks of opportunity when activating this item.

\mitemreq{Transmutation}{Augment}{2}{6th}{Craft (leather or textiles) 10}

\appdescription{Gloves of Dexterity, Greater}{12,000}{12th}{Arms}{Moderate Transmutation (Augment)}{12th}{Standard (specific action)} These gloves function like \magicitem{gloves of dexterity}, except that they grant a \plus4 enhancement bonus to Dexterity instead.

\mitemreq{Transmutation}{Augment}{5}{2th}{Craft (leather or textiles) 16}

\appdescription{Gauntlets of Ogre Strength}{2,000}{8th}{Arms}{Faint Transmutation (Augment)}{6th}{Standard (specific action)} By striking your gauntleted fists against each other in front of you (a standard action), you command these gauntlets to grant you a \plus2 enhancement bonus to Strength for 5 minutes. Unless you can defend yourself without the use of your hands, you provoke attacks of opportunity when activating this item.

\mitemreq{Transmutation}{Augment}{2}{6th}{Craft (metal) 10}

\appdescription{Gauntlets of Giant Strength}{12,000}{12th}{Arms}{Moderate Transmutation (Augment)}{8th}{Standard (specific action)} These gauntlets function like \magicitem{gauntlets of ogre strength}, except that they grant a \plus4 enhancement bonus to Strength instead.

\mitemreq{Transmutation}{Augment}{5}{2th}{Craft (metal) 16}

\subsection{Head}

\appdescription{Circlet of Wisdom}{2,000}{8th}{Head}{Faint Transmutation (Augment)}{4th}{Standard (specific action) 1/day} By bowing your head and closing your eyes in contemplation (a standard action), you command this circlet to grant you a \plus2 enhancement bonus to Wisdom for 5 minutes. Unless you can defend yourself while blind, you provoke attacks of opportunity when activating this item.

\mitemreq{Transmutation}{Augment}{2}{6th}{Craft (metal) 10}

\appdescription{Circlet of Wisdom, Greater}{12,000}{10th}{Head}{Moderate Transmutation (Augment)}{12th}{Standard (specific action) 1/day} This circlet functions like a \magicitem{circlet of wisdom}, except that it grants a \plus4 enhancement bonus to Wisdom instead.

\mitemreq{Transmutation}{Augment}{5}{2th}{Craft (metal) 16}

\appdescription{Headband of Intellect}{2,000}{8th}{Head}{Faint Transmutation (Augment)}{4th}{Standard (specific action)} By stroking your chin or beard thoughtfully with one hand (a standard action), you command this headband to grant you a \plus2 enhancement bonus to Intelligence for 5 minutes.

\mitemreq{Transmutation}{Augment}{2}{6th}{Craft (metal) 10}

\appdescription{Headband of Intellect, Greater}{12,000}{10th}{Head}{Moderate Transmutation (Augment)}{12th}{Standard (specific action)} This headband functions like a \magicitem{headband of intellect}, except that it grants a \plus4 enhancement bonus to Intelligence instead.

\mitemreq{Transmutation}{Augment}{5}{2th}{Craft (metal) 16}

\subsection{Legs}

\appdescription{Boots of Elvenkind}{300}{4th}{Legs}{Faint Transmutation (Augment)}{2nd}{Standard (specific action)} While wearing these boots, you gain a \plus2 enhancement bonus to Stealth checks. In addition, if you crouch down and touch the top of one of the boots (a standard action), you gain a \plus5 enhancement bonus to Stealth checks. The bonus lasts as long as you remain crouching, and for 5 rounds thereafter (maximum 5 minutes). You can move at half speed while crouching.

\mitemreq{Transmutation}{Augment}{1}{2nd}{Craft (leather) 6}

\appdescription{Boots of Mobility}{1,000}{6th}{Legs}{Faint Transmutation (Augment)}{4th}{\x} While wearing these boots, you gain a \plus4 enhancement bonus to Acrobatics and Athletics checks.

\mitemreq{Transmutation}{Augment}{1}{4th}{Craft (leather) 8}

\appdescription{Boots of Speed}{5,000}{10th}{Legs}{Moderate Transmutation (Temporal)}{8th}{Standard (specific action)} If you stomp your foot on the ground three times (a standard action), you gain the effects of the \spell{haste} spell. This benefit lasts as long as you move continuously without taking any other action, and for 5 rounds thereafter (maximum 5 minutes).

\mitemreq{Transmutation}{Temporal}{4}{8th}{Craft (leather) 12}

\subsection{Rings}

\parhead{Physical Description} Rings have no appreciable weight. Although exceptions exist that are crafted from glass or bone, the vast majority of rings are forged from metal - usually precious metals such as gold, silver, and platinum. A ring has AC 13, 2 hit points, hardness 10, and a break DC of 25.

\begin{comment}
\begin{dtable}
\lcaption{Rings}
\begin{tabularx}{\columnwidth}{>{\lcol}X l}
Ring & Market Price \\
Protection \plus1 & 2,000 gp \\
Feather falling & 2,200 gp \\
Climbing & 2,500 gp \\
Jumping & 2,500 gp \\
Sustenance & 2,500 gp \\
Swimming & 2,500 gp \\
Mind shielding & 8,000 gp \\
Protection \plus2 & 8,000 gp \\
Climbing, improved & 10,000 gp \\
Jumping, improved & 10,000 gp \\
Swimming, improved & 10,000 gp \\
Energy resistance, minor & 12,000 gp \\
Protection \plus3 & 18,000 gp \\
Energy resistance, major & 28,000 gp \\
Protection \plus4 & 32,000 gp \\
Energy resistance, greater & 44,000 gp \\
Protection \plus5 & 50,000 gp \\
\end{tabularx}
\end{dtable}
\end{comment}

\appdescription{Energy Resistance, Lesser}{200}{3rd}{Ring}{Faint Abjuration (Shielding)}{2nd}{Immediate (triggered)} When you take energy damage, you can activate this ring to reduce the damage by 5 \add half your level.

\mitemreq{Abjuration}{Shielding}{1}{2nd}{Craft (metal) 6}

\appdescription{Energy Resistance}{2,000}{8th}{Ring}{Moderate Abjuration (Shielding)}{6th}{Immediate (triggered)} When you take energy damage, you can activate this ring to gain damage reduction equal to 10 \add half your level against that type of energy damage for 5 rounds. During this time, the ring casts light as a torch. The color of the light depends on the energy damage resisted: green for acid, blue for cold, yellow for electricity, and red for fire.

\mitemreq{Abjuration}{Shielding}{3}{6th}{Craft (metal) 10}

\appdescription{Sustenance}{1,000}{6th}{Ring}{Faint Conjuration/Transmutation (Creation, Temporal)}{4th}{\x} This ring continually provides its wearer with life-sustaining nourishment. The ring also refreshes the body and mind, so that its wearer needs only sleep 2 hours per day to gain the benefit of 8 hours of sleep. This does not affect how much rest the wearer must get to regain spells. The ring must be worn for a full week before it begins to work. If it is removed, the owner must wear it for another week to reattune it to himself.

\mitemreq{Conjuration/Transmutation}{Creation, Temporal}{2}{4th}{Craft (metal) 8}

\appdescription{Protection}{Varies}{see text}{Ring}{Varied Abjuration (Shielding)}{}{\x}
A ring of protection grants an enhancement bonus to your saving throws while worn. The price of the ring depends on its enhancement bonus, as shown in the table below.

\begin{dtable}
    \caption{Ring of Protection}
    \begin{tabularx}{\columnwidth} {>{\ccol}X c c}
        \thead{Enhancement Bonus} & \thead{Base Price} & \thead{Item Level} \\
        \plus1 & 100 gp & 2nd \\
        \plus2 & 500 gp & 4th \\
        \plus3 & 2,500 gp & 8th \\
        \plus4 & 12,500 gp & 12th \\
        \plus5 & 62,500 gp & 16th \\
    \end{tabularx}
\end{dtable}

The caster level is equal to three times the item's enhancement bonus. To craft the item, you must have a number of ranks in Craft (jewelry) equal to the item's caster level \add 4.

\mitemreq{Abjuration}{Shielding}{1}{varies}{Craft (jewelry) varies}

\subsection{Torso}

\appdescription{Belt of Constitution}{2,000}{8th}{Torso}{Faint Transmutation (Augment)}{6th}{Standard (specific action)} By tightening the straps on this belt with both hands (a standard action), you command it to grant you a \plus2 enhancement bonus to Constitution for 5 minutes. Unless you can defend yourself without the use of your hands, you provoke attacks of opportunity when activating this item.

\mitemreq{Transmutation}{Augment}{2}{6th}{Craft (textiles) 10}

\appdescription{Belt of Constitution, Greater}{12,000}{12th}{Torso}{Moderate Transmutation (Augment)}{12th}{Standard (specific action) 1/day} This belt functions like a \magicitem{belt of constitution}, except that it grants a \plus4 enhancement bonus to Constitution instead.

\mitemreq{Transmutation}{Augment}{5}{2th}{Craft (textiles) 16}

\appdescription{Cloak of Charisma}{2,000}{8th}{Torso}{Faint Transmutation (Augment)}{6th}{Standard (specific action)} By flourishing this cloak with one hand (a standard action), you command it to grant you a \plus2 enhancement bonus to Charisma for 5 minutes.

\mitemreq{Transmutation}{Augment}{2}{6th}{Craft (textiles) 10}

\appdescription{Cloak of Charisma, Greater}{12,000}{12th}{Torso}{Moderate Transmutation (Augment)}{12th}{Standard (specific action)} This cloak functions like a \magicitem{cloak of charisma}, except that it grants a \plus4 enhancement bonus to Charisma instead.

\mitemreq{Transmutation}{Augment}{5}{2th}{Craft (textiles) 16}

\section{Implements}

\subsection{Weapons}

Magic weapons improve a character's combat abilities. They always provide an enhancement bonus to a character's attack and damage with attacks using the weapon. In addition to an enhancement bonus, weapons may have special abilities or be made of an unusual material.

\parhead{Weapon Prices} The prices of enhancement bonuses to weapons are listed in \trefnp{Magic Weapons}, and the prices of special abilities are listed on \trefnp{Weapon Special Abilities}. If a weapon has a special ability, the price of the special ability is added to the price of the weapon.

\begin{dtable}
\caption{Magic Weapons}
\begin{tabularx}{\columnwidth} {>{\ccol}X c c}
  \thead{Enhancement Bonus} & \thead{Base Price} & \thead{Item Level}\\
\plus1 weapon & 200 gp & 3rd \\
\plus2 weapon & 1,000 gp & 6th \\
\plus3 weapon & 5,000 gp & 10th \\
\plus4 weapon & 25,000 gp & 13th \\
\plus5 weapon & 125,000 gp & 17th \\
\end{tabularx}
\end{dtable}

\parhead{Special Ability Limitations} The number of special abilities on the weapon cannot exceed the enhancement bonus of the weapon. Additionally, the price of all special abilities cannot exceed twice the price of the enhancement bonus on the weapon.

\parhead{Caster Level for Weapons} The caster level of a magic weapon with a special ability is given in the item description. For an item with only an enhancement bonus, the caster level is three times the enhancement bonus. If an item has both an enhancement bonus and a special ability, the higher of the two caster level requirements must be met.

\parhead{Hardness and Hit Points} Each \plus1 of enhancement bonus adds 2 to a weapon's hardness and \plus10 to its hit points.

\parhead{Ranged Weapons and Ammunition} The enhancement bonus from a ranged weapon does not stack with the enhancement bonus from ammunition. Only the higher of the two enhancement bonuses applies. Special abilities are applied from both sources, as long as they are not identical. If conflicting special abilities exist, the special ability on the ammunition takes precedence.

Magic ammunition loses its magic after being fired, whether it hits or misses.

\parhead{Light Generation} Some magic weapons shed light equivalent to a light spell (bright light in a 20-foot radius, shadowy light in a 40-foot radius). These glowing weapons are quite obviously magical. Such a weapon can't be concealed when drawn, nor can its light be shut off. Some of the specific weapons detailed below always or never glow, as defined in their descriptions.

\parhead{Activation} Usually, a character benefits from a magic weapon in the same way a character benefits from a mundane weapon - by attacking with it. Special abilities on weapons are usually activated if the character strikes a foe with the weapon.

\parhead{Magic Weapons and Critical Hits} Some weapon qualities and some specific weapons have an extra effect on a critical hit. These special effects function against creatures not subject to critical hits. When fighting against such creatures, roll for critical hits as you would against any other creature subject to critical hits. On a successful critical roll, apply the special effect, but do not multiply the weapon's regular damage.

\begin{dtable*}
    \lcaption{Weapon Special Abilities}
    \begin{tabularx}{\textwidth}{l >{\lcol}X l l}
        \thead{Special Ability} & \thead{Description} & \thead{Cost} & \thead{Item Level} \\
        Bane & Add special ability that only functions against certain creatures & Special & Special \\
        Morphing & Weapon transforms into similar weapon & 200 gp & 3rd \\
        Entangling & Entangle struck foe & 400 gp & 4th \\
        Flaming & Ignite struck foe & 400 gp & 4th \\
        Forceful & Knock back struck foe &  400 gp & 4th \\
        Thundering & Deafen struck foe and those nearby & 400 gp & 4th \\
        Freezing & Fatigue struck foe & 400 gp & 4th \\
        Surestrike, Lesser & Roll critical confirmation twice & 400 gp & 4th \\
        Defending & Trade attack and damage for AC & 500 gp & 4th \\
        Returning & Weapon returns after being thrown &  1,000 gp & 6th \\
        Poisoning & Quickly coat weapon in duplicated poison & 1,600 gp & 7th \\
        Shocking & Stagger struck foe & 1,600 gp & 7th \\
        Vampiric & Lick weapon to regain hit points &  1,600 gp & 7th \\
        Surestrike & Reroll missed attacks &  4,000 gp & 9th \\
        Thieving & Absorb struck objects into weapon & 4,000 gp & 9th \\
        Returning, Greater & Weapon returns immediately after being thrown & 5,000 gp & 10th \\
        Heartseeking & Automatically score critical hit after striking target repeatedly &  12,000 gp & 12th \\
        Poisoning, Greater & Quickly coat weapon in potent duplicated poison & 24,000 gp & 13th \\
        Soulreaving & Weapon strikes the soul for delayed damage instead of normal damage &  60,000 gp & 16th \\
        Vorpal & 1/day sever foe's head in a single blow & 140,000 gp & 18th \\
    \end{tabularx}
\end{dtable*}

\subsubsection{Bane} \parhead{Price (Level)} Special \parhead{Aura, Caster Level} Special \par
A bane weapon excels at attacking a specific type of creature. Any weapon special ability can be designated as a ``bane'' ability, causing it to only function against a specific kind of creature. In exchange, the ability costs half the normal price in raw materials to add to the weapon. A list of possible foes is described on the following table.

\begin{dtable}
\begin{tabularx}{\columnwidth}{>{\lcol}X >{\lcol}X}
\thead{Designated Foe} & \thead{Designated Foe}\\
Aberrations & Animals \\
Constructs & Dragons \\
Elementals & Fey \\
Giants & Humanoids, civilized \\
Humanoids, savage & Magical beasts \\
Monstrous humanoids & Oozes \\
Outsiders, inner planes & Outsiders, outer planes \\
Plants & Undead \\
Vermin & \\
\end{tabularx}
\end{dtable}

\mitemreq{Transmutation}{Augment}{2}{4th}{Craft (as weapon) 8}

\impdescription{Defending}{500}{4th}{Faint Abjuration (Shielding)}{4th}{Move (specific action)} If you spin this weapon in your hands in a complete revolution clockwise, you do not apply its enhancement bonus to attack and damage. However, you add its enhancement bonus to your dodge defense modifier for 1 round whenever you make a standard attack or take the total defense action. If you spin the weapon counter-clockwise, the weapon instead applies its bonus to your attack and damage, as normal.

\mitemreq{Abjuration}{Shielding}{2}{4th}{Craft (as weapon) 8}

\impdescription{Entangling}{400}{4th}{Faint Conjuration (Creation)}{2nd}{Immediate (triggered)} When you strike a foe with this weapon, you can cause webbing to spring into existence, entangling the struck foe for 5 rounds. The foe can break out of the webbing with a grapple or Escape Artist check against a DC equal to 15 \add your level.

An entangled creature moves at half speed, cannot run or charge, and takes a \minus2 penalty to physical attacks and defenses, as well as Strength and Dexterity-based checks. An entangled character who attempts to cast a spell must make a Concentration check (DC 10 \add double the spell's level) or lose the spell.

\mitemreq{Conjuration}{Creation}{1}{2nd}{Craft (as weapon) 6}

\impdescription{Flaming}{400}{4th}{Faint Evocation (Energy) [Fire]}{2nd}{Immediate (triggered)} When you strike a foe with this weapon, you can engulf the struck creature in flames. If you do, it is ignited for 5 rounds. An ignited creature is vulnerable, causing it to take a \minus2 penalty to attacks, defenses, and checks. In addition, at the end of each of its turns, it takes d6 damage from the fire. If the creature takes a move action, it can attempt a DC 10 Dexterity check to put out the flames. This action requires a free hand. Dropping prone as part of the action gives a \plus5 bonus on this check.

When you activate this ability, the weapon is wreathed in flames, causing damage you deal with it to be treated as fire damage in addition to its other types. This effect lasts for 5 rounds, during which time you cannot activate the item's abilities.

\mitemreqdesc{Evocation}{Energy}{Fire}{1}{2nd}{Craft (as weapon) 6}

\impdescription{Freezing}{400}{4th}{Faint Evocation (Energy) [Cold]}{2nd}{Immediate (triggered)} When you strike a foe with this weapon, you can unleash an icy blast from the weapon. If you do, your foe is fatigued for 5 rounds. A fatigued creature can neither sprint nor charge and is vulnerable, giving it a \minus2 penalty to attacks, defenses, and checks.

When you activate this ability, the weapon radiates chilling cold, causing damage you deal with it to be treated as cold damage in addition to its other types. This effect lasts for 5 rounds, during which time you cannot activate the item's abilities.

\mitemreqdesc{Evocation}{Energy}{Cold}{1}{2nd}{Craft (as weapon) 6}

\impdescription{Forceful}{400}{4th}{Faint Evocation (Control)}{4th}{Immediate (triggered)} When you strike a foe with this weapon, you can activate the weapon to immediately make a bull rush attempt with a circumstance bonus equal to the damage you dealt with the attack. You do not provoke an attack of opportunity for the bull rush, even if you fail, and you do not have to move with your foe to knock it back the full distance.

After you activate this ability, the weapon feels heavier in your hands. This effect lasts for 5 rounds, during which time you cannot activate the item's abilities.

\mitemreq{Evocation}{Control}{1}{2nd}{Craft (as weapon) 6}

%As 5th level spell because it feels right. Based losely on Discern
%Vulnerability, although this feels like Awareness, so True Seeing?
\impdescription{Heartseeking}{12,000}{12th}{Moderate Divination (Awareness)}{10th}{Immediate (triggered)} When you strike the same foe with this weapon for multiple rounds in a row, you can suddenly perceive a critical weakness in your foe's defenses. You must strike the foe for a number of consecutive rounds equal to the critical multiplier of the weapon you are using. If you activate the item, the final hit automatically becomes a confirmed critical hit. This has no effect on creatures immune to critical hits.

When you activate this ability, you gain a \plus4 bonus to confirm critical hits. This effect lasts for 5 rounds, during which time you cannot activate the item's abilities.

\mitemreq{Divination}{Awareness}{5}{10th}{Craft (as weapon) 14}

\impdescription{Morphing}{200}{6th}{Faint Transmutation (Alteration)}{4th}{Standard (specific action)} A morphing weapon can transform into any other weapon from its weapon group. To transform a morphing weapon, you must grab it with both hands and strike it against your knee or other hard object, as if breaking it, while visualizing its new form (a standard action). It remains transformed until you transform it again.

\mitemreq{Transmutation}{Alteration}{1}{2nd}{Craft (as weapon) 6}

%price as 2nd level spell?
\impdescription{Poisoning}{1,600}{7th}{Faint Conjuration/Transmutation (Creation, Temporal)}{4th}{Swift (specific action) and standard (specific action)} A poisoning weapon can conjure poisons to cover the striking surface of the weapon. The poison must first be inserted into a small slot in the hilt of the weapon (a standard action). Once a poison is present in the slot, you can coat the weapon with a dose of the poison by pressing a small button on the hilt (a swift action). After a poison has been used, it takes 5 rounds for the weapon to create more poison, during which time the weapon cannot be activated. Only liquid poisons worth 100 gp per dose or less can be duplicated in this way.

The poison within the weapon is kept fresh magically, decaying at a rate of one minute per day. The weapon can be emptied by pressing a second button to open its slot and pouring the poison out (a standard action). You can freely insert and remove poison from the weapon, but coating the weapon in poison costs an activation.

\mitemreq{Conjuration, Transmutation}{2}{4th}{Craft (as weapon) 8}

%price as 5th level spell? similar to 20 uses of a one-shot item worth 1000 gp, so sure
\impdescription{Poisoning, Greater}{24,000}{13th}{Moderate Conjuration/Transmutation (Creation, Temporal)}{10th}{Swift (specific action) and standard (specific action)} This ability functions like the \magicitem{poisoning} weapon ability, except that it can duplicate liquid poisons of up to 1,000 gp per dose. In addition, up to five different poisons can be stored within the weapon. When you coat the weapon with poison, you may choose which poison to use.

\mitemreq{Conjuration, Transmutation}{5}{10th}{Craft (as weapon) 14}

\impdescription{Returning}{1,000}{6th}{Faint Conjuration (Translocation) [Teleportation]}{4th}{\x} After being thrown, a returning weapon teleports back to the creature that threw it. It returns to the thrower just before the creature's next turn (and is therefore ready to use again in that turn).

Catching a returning weapon when it comes back is a free action. If you can't catch it, the weapon drops to the ground in the square from which it was thrown.

\mitemreqdesc{Conjuration}{Translocation}{Teleportation}{2}{4th}{Craft (as weapon) 8}

\impdescription{Returning, Greater}{5,000}{10th}{Faint Conjuration (Translocation) [Teleportation]}{6th}{\x} This ability functions like the \magicitem{returning} ability, except that the weapon teleports back to the creature that threw it immediately after the attack is resolved, allowing the creature to make multiple attacks in the same round with the same thrown weapon. 

\mitemreqdesc{Conjuration}{Translocation}{Teleportation}{3}{6th}{Craft (as weapon) 10}

%As 7th level spell? I have no idea. There aren't actually any spells this can be based on.
\impdescription{Soulreaving}{60,000}{16th}{Strong Necromancy (Soul)}{16th}{\x and standard (specific action)} This ghostly, translucent weapon strikes directly at the target's soul. It ignores all damage reduction, but it does not deal hit point damage. In fact, a creature struck by the weapon only feels the weapon pass through it harmlessly. Damage that would be dealt by the weapon is delayed for up to 24 hours. While the damage is delayed, it cannot be cured.

In order to convert the delayed damage into real damage, the wielder must stab himself through the heart with the weapon as a standard action. This deals no damage to the wielder, but any creatures that have been dealt damage by the weapon immediately take lethal damage equal to the delayed damage the weapon has stored up for them. Any such damage dealt in excess of the creature's hit points is converted directly into critical damage.

A soulreaver weapon has no effect on objects or constructs. While wielded, it has physical form only for its wielder, making it impossible to sunder or disarm. While not in use, it can be picked up and touched normally.

\mitemreq{Necromancy}{Soul}{7}{14th}{Craft (as weapon) 18}

\impdescriptiondc{Shocking}{1,600}{7th}{Faint Evocation (Energy) [Electricity]}{4th}{Immediate (triggered)}{Fortitude (level \add 2)} When you strike a foe with this weapon, you can unleash an powerful electrical jolt from the weapon. If you do, make a Fortitude attack. If you succeed, your foe is staggered for 5 rounds. A staggered creature may take a single move action or standard action each round, but not both. She cannot take full-round actions, but she may take swift actions. In addition, she is vulnerable, causing her to take a \minus2 penalty on attacks, defenses, and checks.

When you activate this ability, the weapon crackles with electrical energy, causing damage you deal with it to be treated as electrical damage in addition to its other types. This effect lasts for 5 rounds, during which time you cannot activate the item's abilities.

\mitemreqdesc{Evocation}{Energy}{Electricity}{2}{4th}{Craft (as weapon) 8}

\impdescription{Surestrike, Lesser}{400}{4th}{Faint Divination (Knowledge)}{2nd}{Immediate (triggered)} When you threaten a critical hit with this weapon, you can activate it to receive a brief glimpse of the future, showing you how to wound your foe deeply. If you do, you may roll the threat confirmation twice and take whichever roll you prefer.

After you activate this ability, you see shadowy glimpses of alternate futures out of the corner of your eyes. This effect lasts for 5 rounds, during which time you cannot activate the item's abilities.

\mitemreq{Divination}{Knowledge}{1}{2nd}{Craft (as weapon) 6}

\impdescription{Surestrike}{4,000}{9th}{Faint Divination (Knowledge)}{8th}{Immediate (triggered)} When you miss an attack with this weapon, you can activate it to reroll the attack roll. You must take the second result.

When you activate this ability, you see shadowy glimpses of alternate futures superimposed over objects and creatures you see. This effect lasts for 5 rounds, during which time you cannot activate the item's abilities.

\mitemreq{Divination}{Knowledge}{3}{6th}{Craft (as weapon) 10}

\impdescription{Thieving}{4,000}{9th}{Faint Transmutation (Alteration)}{4th}{Immediate (trigger) and standard (specific action)} When you strike an object with this weapon, if the object is at least one size category smaller than the weapon, you may activate the weapon. If you do, the object is absorbed into the weapon, leaving no trace that it ever existed. Striking an attended object requires a successful disarm or sunder attempt.

You can retrieve objects from the weapon by running your hand along the length of the striking surface of the weapon (a standard action). If you do, the last item absorbed by the weapon appears in your hand. You may freely retrieve objects from within the weapon, but absorbing objects costs an activation.

The weapon can hold no more than three objects at once. If you attempt to absorb an object while the weapon is full, the attempt fails.

\mitemreq{Transmutation}{Alteration}{2}{4th}{Craft (as weapon) 8}

\impdescriptiondc{Thundering}{400}{4th}{Faint Evocation (Energy) [Sonic]}{2nd}{Immediate (triggered)}{Fortitude (level \add 1)} When you strike a foe with this weapon in melee, you can detonate a deafening roll of thunder. If you do, make a Fortitude attack against the struck foe and all other creatures within a \areasmall radius of you. A successful attack deafens a creature for 5 rounds. You are immune to the deafening effect.

After you activate this ability, the weapon emits non-damaging thunderous echoes whenever it strikes a solid object or creature. This effect lasts for 5 rounds, during which time you cannot activate the item's abilities.

\mitemreqdesc{Evocation}{Energy}{Sonic}{4}{8th}{Craft (as weapon) 8}

\impdescription{Vampiric}{1,600}{7th}{Faint Necromancy (Life)}{4th}{Move (specific action)} If you lick the striking part of this weapon (a move action), you regain hit points equal to the damage dealt by the weapon on its last successful attack. If the weapon has not dealt damage in the past round, you regain no hit points.

\mitemreq{Necromancy}{Life}{2}{4th}{Craft (as weapon) 8}

%8th level - 9th level for Power Word Kill, down to 8th because the trigger is
%so hard to pull off
\impdescription{Vorpal}{140,000}{18th}{Strong Transmutation (Augment)}{18th}{Immediate (triggered)} If you roll a 20 with this weapon and confirm the critical hit, you can instantly decapitate your foe. If you do, it dies immediately. This has no effect on creatures without a discernable head, creatures unaffected by the loss of a single head, or creatures whose head you cannot reach.

\mitemreq{Transmutation}{Augment}{9}{18th}{Craft (as weapon) 22}

\begin{comment}
\subsection{Rods}

Rods are scepterlike devices that have unique magical powers and do not usually have charges. Anyone can use a rod.

\parhead{Physical Description} Rods weigh approximately 5 pounds.

They range from 2 feet to 3 feet long and are usually made of iron or some other metal. (Many, as noted in their descriptions, can function as light maces or clubs due to their sturdy construction.)

These sturdy items have AC 9, 10 hit points, hardness 10, and a break DC of 27.

\parhead{Activation} Details relating to rod use vary from item to item. See the individual descriptions for specifics.
\end{comment}

\subsection{Staffs}

A staff is a long shaft, usually made of wood, that enhances a spellcaster's power. Staffs function exactly like wands (see below), except that they enhance all schools of magic at once.

\parhead{Staff Prices} Enhancement bonuses on staffs are three times as expensive as wands, but staffs otherwise use the same pricing rules as wands.

\begin{dtable}
    \caption{Staff Prices}
    \begin{tabularx}{\columnwidth} {>{\ccol}X c c}
        \thead{Enhancement Bonus} & \thead{Base Price} & \thead{Item Level}\\
        \plus1 staff & 150 gp & 3rd \\
        \plus2 staff & 750 gp & 5th \\
        \plus3 staff & 3,750 gp & 9th \\
        \plus4 staff & 18,750 gp & 13th \\
        \plus5 staff & 93,750 gp & 17th \\
    \end{tabularx}
\end{dtable}

\parhead{Physical Description} A typical staff is 4 feet to 7 feet long and 2 inches to 3 inches thick, weighing about 5 pounds. Most staffs are wood, but a rare few are bone, metal, or even glass. (These are extremely exotic.) Staffs often have a gem or some device at their tip or are shod in metal at one or both ends. Staffs are often decorated with carvings or runes. A typical staff is like a walking stick, quarterstaff, or cudgel. It has AC 7, 10 hit points, hardness 5, and a break DC of 24.

\parhead{Activation} Staffs use the same activation method as wands.

\subsection{Holy Symbols}
A holy symbol is a small object that enhances a divine spellcaster's power. Holy symbols function exactly like wands (see below), except that they enhance all schools of magic at once. 

\parhead{Holy Symbol Prices} Enhancement bonuses on holy symbols are three times as expensive as wands, but holy symbols otherwise use the same pricing rules as wands.

\begin{dtable}
    \caption{Holy Symbol Prices}
    \begin{tabularx}{\columnwidth} {>{\ccol}X c c}
        \thead{Enhancement Bonus} & \thead{Base Price} & \thead{Item Level}\\
        \plus1 holy symbol & 150 gp & 3rd \\
        \plus2 holy symbol & 750 gp & 5th \\
        \plus3 holy symbol & 3,750 gp & 9th \\
        \plus4 holy symbol & 18,750 gp & 13th \\
        \plus5 holy symbol & 93,750 gp & 17th \\
    \end{tabularx}
\end{dtable}

\parhead{Physical Description} A typical holy symbol is a no larger than 4 inches in each dimension and can be easily held in the palm of a hand. Most holy symbols are metal, but they can be made from wood, bone, or even more exotic materials, depending on the deity they symbolize.

Many holy symbols are designed to be worn as an amulet in addition to being held in the hand. When worn in this way, the holy symbol occupies a torso body slot.

\parhead{Activation} Holy symbols use the same activation method as wands.

\begin{dtable}
    \begin{tabularx}{\columnwidth}{X l l}
        \thead{Special Ability} & \thead{Cost} & \thead{Item Level} \\
        Channeling & 2,000 & 6th \\
        Channeling, Greater & 8,000 & 12th \\
    \end{tabularx}
\end{dtable}

\impdescription{Channeling}{1,500}{7th}{Faint Evocation (Channeling)}{6th}{Immediate (triggered)} When you channel energy, you can activate this holy symbol to inflict or heal an extra d6 points of damage. If you do not have the channel energy ability, this ability does not affect you.

\mitemreq{Evocation}{Channeling}{2}{4th}{Craft (as holy symbol) 8}

\impdescription{Channeling, Greater}{10,000}{11th}{Moderate Evocation (Channeling)}{12th}{Immediate (triggered)} This holy symbol functions like a \magicitem{channeling} holy symbol, except that it increases the damage by 2d6 instead.

\mitemreq{Evocation}{Channeling}{4}{8th}{Craft (as holy symbol) 12}

\subsection{Wands}

A wand is a thin baton that enhances a spellcaster's power. Wands always provide an enhancement bonus to caster level with a particular school of magic. In addition to an enhancement bonus, wands may have special abilities or be made of an unusual material.

\parhead{Wand Prices} The prices of enhancement bonuses to wands are listed in \trefnp{Magic wands}, and the prices of special abilities are listed on \trefnp{Wand Special Abilities}. If a wand has a special ability, the price of the special ability is added to the price of the wand.

\parhead{Special Ability Limitations} The number of special abilities on the wand cannot exceed the enhancement bonus of the wand. Additionally, the price of all special abilities cannot exceed twice the price of the enhancement bonus on the wand.

\subparhead{Multiple Schools} Some rare wands provide bonuses to two schools. The enhancement bonus on a wand costs twice as much if it provides bonuses to two schools. A wand cannot provide an enhancement bonus to more than two schools.

\begin{dtable}
\caption{Wand Prices}
\begin{tabularx}{\columnwidth} {>{\ccol}X c c}
  \thead{Enhancement Bonus} & \thead{Base Price} & \thead{Item Level}\\
\plus1 wand & 50 gp & 1st \\
\plus2 wand & 250 gp & 3rd \\
\plus3 wand & 1,250 gp & 7th \\
\plus4 wand & 6,250 gp & 10th \\
\plus5 wand & 31,250 gp & 14th \\
\end{tabularx}
\end{dtable}

\parhead{Physical Description} A typical wand is 6 inches to 12 inches long and about 1/4 inch thick, and often weighs no more than 1 ounce. Most wands are wood, but some are bone. A rare few are metal, glass, or even ceramic, but these are quite exotic. Occasionally, a wand has a gem or some device at its tip, and most are decorated with carvings or runes. A typical wand has AC 7, 5 hit points, hardness 5, and a break DC of 16.

\begin{dtable}
\lcaption{Wand Special Abilities}
\begin{tabularx}{\columnwidth}{>{\lcol}X l l}
  \thead{Special Ability} & \thead{Cost} & \thead{Item Level} \\
  Enlarging & 400 gp & 4th \\
  Flaming & 400 gp & 4th \\
  Freezing & 400 gp & 4th \\
  Shocking & 1,600 gp & 7th \\
\end{tabularx}
\end{dtable}
%Wand special abilities: priced as close range, immediate action trigger.
%Trigger needs to be immediate; otherwise, a caster casting a quickened spell
%could potentially unleash four separate effects simultaneously.
\impdescription{Enlarging}{400}{4th}{Faint Universal}{2nd}{Immediate (triggered)} When you cast a spell, you can activate this wand to double the range of the spell.

When you activate this ability, the wand doubles in length. This effect lasts for 5 rounds, during which time you cannot activate the item.

\mitemreq{No school}{}{1}{2nd}{Craft (as wand) 6}

\impdescription{Flaming}{400}{4th}{Faint Evocation (Energy) [Fire]}{2nd}{Immediate (triggered)} When you cast a spell, you can activate this wand to ignite a single creature affected by the spell for 5 rounds.

An ignited creature is vulnerable, causing it to take a \minus2 penalty to attacks, defenses, and checks. In addition, at the end of each of its turns, it takes d6 damage from the fire. If the creature takes a move action, it can attempt a DC 10 Dexterity check to put out the flames. This action requires a free hand. Dropping prone as part of the action gives a \plus5 bonus on this check.

When you activate this ability, the wand is wreathed in flame, causing it to cast light as a torch. This effect lasts for 5 rounds, during which time you cannot activate the item.

\mitemreqdesc{Evocation}{Energy}{Fire}{1}{2nd}{Craft (as wand) 6}

\impdescription{Freezing}{400}{4th}{Faint Evocation (Energy) [Cold]}{2nd}{Immediate (triggered)} As you cast a spell, you can activate this wand to fatigue a single creature affected by the spell for 5 rounds. A fatigued creature can neither sprint nor charge and is vulnerable, giving it a \minus2 penalty to attacks, defenses, and checks.

When you activate this ability, the wand radiate frigid cold, causing it to snuff out torches and other small fires within a 5 foot radius of it. This effect lasts for 5 rounds, during which time you cannot activate the item's abilities.

\mitemreqdesc{Evocation}{Energy}{Cold}{1}{2nd}{Craft (as wand) 6}

\impdescriptiondc{Shocking}{1,600}{7th}{Faint Evocation (Energy) [Electricity]}{8th}{Immediate (triggered) 1/day}{Fortitude (level \add 2)} As you cast a spell, you can activate this wand. If you do, make a Fortitude attack against a single creature affected by the spell. A successful attack causes the target to be staggered for 5 rounds.

A staggered creature may take a single move action or standard action each round, but not both. She cannot take full-round actions, but she may take swift actions. In addition, she is vulnerable, causing her to take a \minus2 penalty on attacks, defenses, and checks.

When you activate this ability, you crackle with electrical energy, causing you to radiate flashing light as a torch. This effect lasts for 5 rounds, during which time you cannot activate the item's abilities.

\mitemreqdesc{Evocation}{Energy}{Electricity}{2}{4th}{Craft (as wand) 8}

\section{Tools}

\subsection{Scrolls}
A scroll is a spell (or collection of spells) that has been stored in written form. A spell on a scroll can be used only once. The writing vanishes from the scroll when the spell is activated. Using a scroll is basically like casting a spell.

\parhead{Physical Description} A scroll is a heavy sheet of fine vellum or high-quality paper. An area about 8 1/2 inches wide and 11 inches long is sufficient to hold one spell. The sheet is reinforced at the top and bottom with strips of leather slightly longer than the sheet is wide. A scroll holding more than one spell has the same width (about 8 1/2 inches) but is an extra foot or so long for each extra spell. Scrolls that hold three or more spells are usually fitted with reinforcing rods at each end rather than simple strips of leather. A scroll has AC 9, 1 hit point, hardness 0, and a break DC of 8.

To protect it from wrinkling or tearing, a scroll is rolled up from both ends to form a double cylinder. (This also helps the user unroll the scroll quickly.) The scroll is placed in a tube of ivory, jade, leather, metal, or wood. Most scroll cases are inscribed with magic symbols which often identify the owner or the spells stored on the scrolls inside. The symbols often hide magic traps.

\parhead{Activation} To activate a scroll, a spellcaster must read the spell written on it. Doing so involves several steps and conditions.

\parhead{Decipher the Writing} The writing on a scroll must be deciphered before a character can use it or know exactly what spell it contains. This requires a read magic spell or a successful Spellcraft check (DC 20 \add spell level).

Deciphering a scroll to determine its contents does not activate its magic unless it is a specially prepared cursed scroll. A character can decipher the writing on a scroll in advance so that he or she can proceed directly to the next step when the time comes to use the scroll.

\parhead{Activate the Spell} Activating a scroll requires reading the spell from the scroll. The character must be able to see and read the writing on the scroll. Activating a scroll spell requires no material components or focus. (The creator of the scroll provided these when scribing the scroll.) Note that some spells are effective only when cast on an item or items. In such a case, the scroll user must provide the item when activating the spell. Activating a scroll spell is subject to disruption just as casting a normally prepared spell would be. Using a scroll is like casting a spell for purposes of arcane spell failure chance.

To have any chance of activating a scroll spell, the scroll user must meet the following requirements.
\begin{itemize*}
\item The spell must be of the correct type (arcane or divine). Arcane spellcasters (wizards and sorcerers) can only use scrolls containing arcane spells, and divine spellcasters (clerics, druids, and paladins) can only use scrolls containing divine spells. (The type of scroll a character creates is also determined by his or her class.)
\item The user must have the spell on his or her spell list.
\item The user must have the requisite attribute score.
\end{itemize*}

If the user meets all the requirements noted above, and her caster level is at least equal to the spell's caster level, she can automatically activate the spell without a check. If she meets all three requirements but her own caster level is lower than the scroll spell's caster level, then she has to make a caster level check (DC = scroll's caster level) to cast the spell successfully. If she fails, she must make a DC 5 Wisdom check to avoid a mishap (see Scroll Mishaps, below).

\parhead{Determine Effect} A spell successfully activated from a scroll works exactly like a spell cast the normal way. Assume the scroll spell's caster level is always the minimum level required to cast the spell for the character who scribed the scroll (usually twice the spell's level), unless the caster specifically desires otherwise.

The writing for an activated spell disappears from the scroll.

\parhead{Scroll Levels} Some spells are acquired by multiple classes at different levels. Use the entry on the table appropriate to the scribing of each individual scroll.

%needs item levels
\begin{dtable}
\lcaption{Scrolls and Potions}
\begin{tabularx}{\columnwidth}{l l l X}
    \thead{Common Spells\fn{1}} & \thead{Market Price} & \thead{Item Level} & \thead{Price per extra caster level} \\
1st-Level & 10 gp & 1st & 12.5 gp \\
2nd-Level & 40 gp & 1st & 25 gp \\
3rd-Level & 100 gp & 2nd & 37.5 gp \\
4th-Level & 250 gp & 3rd & 50 gp \\
5th-Level & 600 gp & 5th & 62.5 gp \\
6th-Level & 1500 gp & 7th & 75 gp \\
7th-Level & 3000 gp & 9th & 87.5 gp \\
8th-Level & 7000 gp & 11th & 100 gp \\
9th-Level & 15000 gp & 12th & 112.5 gp \\
\thead{Paladin Spells} & \thead{Market Price\fn{2}} \\
1st-Level & 40 gp & 1st & 25 gp \\
2nd-Level & 100 gp & 2nd & 50 gp \\
3rd-Level & 250 gp & 3rd & 75 gp \\
4th-Level & 600 gp & 5th & 100 gp \\
\end{tabularx}
1 Includes arcane, divine, and nature spells. \\
2 Scrolls and potions based on paladin spells cost as much as a spell of one level higher because of their rarity. The cost to create them is no different than normal, and players attempting to sell such items will find it difficult to find a buyer.
\end{dtable}

\subsection{Potions and Oils}

A potion is a magic liquid that produces its effect when imbibed. Magic oils are similar to potions, except that oils are applied externally rather than imbibed. A potion or oil can be used only once. It can duplicate the effect of a spell of up to 3rd level that has a casting time of a standard action or less.

Potions are like spells cast upon the imbiber. The character taking the potion doesn't get to make any decisions about the effect  - the caster who brewed the potion has already done so. The drinker of a potion is both the effective target and the effective caster of the effect.

The person applying an oil is the effective caster, but the object is the target.

\parhead{Physical Description} A typical potion or oil consists of 1 ounce of liquid held in a ceramic or glass vial fitted with a tight stopper. The stoppered container is usually no more than 1 inch wide and 2 inches high. The vial has AC 13, 1 hit point, hardness 1, and a break DC of 12. Vials hold 1 ounce of liquid.

\parhead{Identifying Potions} In addition to the standard methods of identification, PCs can sample from each container they find to attempt to determine the nature of the liquid inside. An experienced character learns to identify potions by memory -- for example, the last time she tasted a liquid that reminded her of almonds, it turned out to be a potion of cure moderate wounds.

\parhead{Activation} Drinking a potion or applying an oil requires no special skill. The user merely removes the stopper and swallows the potion or smears on the oil. The following rules govern potion and oil use.

Drinking a potion or using an oil on an item of gear is a standard action. The potion or oil takes effect immediately. Using a potion or oil provokes attacks of opportunity. A successful attack (including grapple attacks) against the character forces a Concentration check (as for casting a spell). If the character fails this check, she cannot drink the potion. An enemy may direct an attack of opportunity against the potion or oil container rather than against the character. A successful attack of this sort can destroy the container.

A creature must be able to swallow a potion or smear on an oil. Because of this, incorporeal creatures cannot use potions or oils.

Any corporeal creature can imbibe a potion. The potion must be swallowed, or in some other way ingested. Any corporeal creature can use an oil.

A character can carefully administer a potion to an unconscious creature as a full-round action, trickling the liquid down the creature's throat. Likewise, it takes a full-round action to apply an oil to an unconscious creature. Exceptionally large objects or creatures require a greater time expenditure.

\parhead{Potion Descriptions} The caster level for a standard potion is the minimum caster level needed to cast the spell (unless otherwise specified). Common potions refer to potions of spells on the cleric, druid, or unrestricted sorcerer/wizard spell lists. Any other spells, such as restricted sorcerer/wizard spells, are considered ``uncommon''.

\subsection{Wondrous Items}

Wondrous items are items which are inherently magical in some way. 

\tooldescription{Bag of Holding}{Varies}{see text}{Varied Conjuration (Translocation) [Planar]}{}{\x} This appears to be a common cloth sack about 2 feet by 4 feet in size. The bag of holding opens into a nondimensional space: Its inside is larger than its outside dimensions. Regardless of what is put into the bag, it weighs a fixed amount. This weight, and the limits in weight and volume of the bag's contents, depend on the bag's type, as shown on the table below.

\begin{dtable*}
\begin{tabularx}{\textwidth}{l l X X X X}
    \thead{Bag} & \thead{Bag Weight} & \thead{Weight Limit} & \thead{Space Limit} \thead{Base Price} & \thead{Item Level}\\
Type I & 15 lb. & 250 lb. & 5 ft. radius & 750 gp & 5th \\
Type II & 20 lb. & 500 lb. & 10 ft. radius & 1,500 gp & 7th \\
Type III & 25 lb. & 1,000 lb. & 15 ft. radius & 3,000 & 9th \\
Type IV & 30 lb. & 1,500 lb. & 20 ft. radius & 6,000 & 10th \\
\end{tabularx}
\end{dtable*}

If the bag is overloaded, or if sharp objects pierce it from the outside, the bag ruptures and is ruined. All contents are lost forever. If a bag of holding is turned inside out, its contents spill out, unharmed, but the bag must be put right before it can be used again. If living creatures are placed within the bag, they can survive for up to 10 minutes, after which time they suffocate. Retrieving a specific item from a bag of holding is a move action - unless the bag contains more than an ordinary backpack would hold, in which case retrieving a specific item is a full-round action.

If a bag of holding is placed within a portable hole a rift to the Astral Plane is torn in the space: Bag and hole alike are sucked into the void and forever lost. If a portable hole is placed within a bag of holding, it opens a gate to the Astral Plane: The hole, the bag, and any creatures within a 10-foot radius are drawn there, destroying the portable hole and bag of holding in the process.

\mitemreqdesc{Conjuration}{Translocation}{Planar}{2}{4th}{Craft (textiles) 8}

\section{Magic Item Rules}

\subsection{Magic Item Auras}

Magic items radiate magical auras which can be detected with the Spellcraft skill (see \pcref{Spellcraft}). Each item describes the auras that can be detected on it, including the strength, school, subschool, and descriptors, as appropriate.

\subsection{Damaging Magic Items}

A magic item doesn't need to make a saving throw unless it is unattended, it is specifically targeted by the effect, or its wielder rolls a natural 1 on his save. Magic items should always get a saving throw against spells that might deal damage to them - even against attacks from which a nonmagical item would normally get no chance to save. Magic items use the same saving throw bonus for all saves, no matter what the type (Fortitude, Reflex, or Will). A magic item's saving throw bonus equals 2 \add its caster level. The only exceptions to this are intelligent magic items, which make Will saves based on their own Charisma and Intelligence scores.

Magic items, unless otherwise noted, take damage as nonmagical items of the same sort. A damaged magic item continues to function, but if it is broken, its magic ceases to function until it is repaired. If it is destroyed, all its magical power is lost.

\subsection{Repairing Magic Items}

A magic item which is broken (but not destroyed) can be repaired for 10\% of the value of the item. 

\subsection{Intelligent Items}

Some magic items, particularly weapons, have an intelligence all their own. Only permanent magic items (as opposed to those with a single use or those with charges) can be intelligent. (This means that potions, scrolls, and wands, among other items, are never intelligent.)

In general, fewer than 1\% of magic items have intelligence.

\subsection{Cursed Items}

Some items are cursed - incorrectly made, or corrupted by outside forces. Cursed items might be particularly dangerous to the user, or they might be normal items with a minor flaw, an inconvenient requirement, or an unpredictable nature. Many cursed items are difficult to identify and remove, requiring the use of rituals such as \spell{remove curse}.
