\chapter{Magic Item Basics}

\section{Magic Item Types}
Magic items are divided into categories: armor, weapons, potions, rings, rods, scrolls, staffs, wands, and wondrous items. In addition, some magic items are cursed or intelligent. Finally, a few magic items are of such rarity and power that they are considered to belong to a category of their own: artifacts. Artifacts are classified in turn as minor (extremely rare but not one-of-a-kind items) or major (each one unique and extremely potent).

\parhead{Equipment}

\subparhead{Armor and Shields} Magic armor (including shields) offers improved, magical protection to the wearer. Some of these items confer abilities beyond a benefit to Armor Class.

\subparhead{Weapons} Magic weapons are created with a variety of combat powers and almost always improve the attack and damage rolls of the wielder as well.

\parhead{Implements} Implements are used by spellcasters to channel magical energy.

\subparhead{Scrolls} A scroll is a spell magically inscribed onto paper or parchment so that it can be used later.

\subparhead{Staffs} A staff has a number of different (but often related) spell effects. A newly created staff has 50 charges, and each use of the staff depletes one or more of those charges.

\subparhead{Wands} A wand is a short stick imbued with the power to cast a specific spell. A newly created wand has 50 charges, and each use of the wand depletes one of those charges.

\parhead{Wondrous Apparel} These objects include magic jewelry, clothing, and other worn items.

\subparhead{Rings} A ring is a circular metal band worn on the finger (no more than two rings per wearer) that has a spell-like power (often a constant effect that affects the wearer).

\parhead{Wondrous Tools} These objects include magical tools, books, horns, and other items.

\subparhead{Potions} A potion is an elixir concocted with a spell-like effect that affects only the drinker.

\subparhead{Rods} A rod is a scepter-like item with a special power unlike that of any known spell.

\subsection{Using Magic Items}

To use a magic item, it must be activated, although sometimes activation simply means putting a ring on your finger. Some items, once donned, function constantly. In most cases, using an item requires a standard action that does not provoke attacks of opportunity. By contrast, spell completion items are treated like spells in combat and do provoke attacks of opportunity.

Activating a magic item is a standard action unless the item description indicates otherwise. However, the casting time of a spell or ritual is the time required to activate the same power in an item, regardless of the type of magic item, unless the item description specifically states otherwise.

The four ways to activate magic items are described below.

\parhead{Spell Completion} This is the activation method for scrolls. A scroll is a spell that is mostly finished. The preparation is done for the caster, so no preparation time is needed beforehand as with normal spellcasting. All that's left to do is perform the finishing parts of the spellcasting (the final gestures, words, and so on). To use a spell completion item safely, a character must be of high enough level in the right class to cast the spell already. If he can't already cast the spell, there's a chance he'll make a mistake. Activating a spell completion item is a standard action and provokes attacks of opportunity exactly as casting a spell does.

\parhead{Spell Trigger} Spell trigger activation is similar to spell completion, but it's even simpler. No gestures or spell finishing is needed, just a special knowledge of spellcasting that an appropriate character would know, and a single word that must be spoken. Anyone with a spell on his or her spell list knows how to use a spell trigger item that stores that spell. (This is the case even for a character who can't actually cast spells, such as a 3rd-level paladin.) The user must still determine what spell is stored in the item before she can activate it. Activating a spell trigger item is a standard action and does not provoke attacks of opportunity.

\parhead{Command Word} If no activation method is suggested either in the magic item description or by the nature of the item, assume that a command word is needed to activate it. Command word activation means that a character speaks the word and the item activates. No other special knowledge is needed.

A command word can be a real word, but when this is the case, the holder of the item runs the risk of activating the item accidentally by speaking the word in normal conversation. More often, the command word is some seemingly nonsensical word, or a word or phrase from an ancient language no longer in common use. Activating a command word magic item is a standard action and does not provoke attacks of opportunity.

Sometimes the command word to activate an item is written right on the item. Occasionally, it might be hidden within a pattern or design engraved on, carved into, or built into the item, or the item might bear a clue to the command word.

The Knowledge (arcana) and Knowledge (local) skills might be useful in helping to identify command words or deciphering clues regarding them. A successful check against DC 30 is needed to come up with the word itself. If that check is failed, succeeding on a second check (DC 25) might provide some insight into a clue.

The \spell{identify} ritual and the \spell{analyze dweomer} spell both reveal command words.

\parhead{Use Activated} This type of item simply has to be used in order to activate it. A character has to drink a potion, swing a sword, interpose a shield to deflect a blow in combat, look through a lens, sprinkle dust, wear a ring, or don a hat. Use activation is generally straightforward and self-explanatory.

Many use-activated items are objects that a character wears. Continually functioning items are practically always items that one wears. A few must simply be in the character's possession (on his person). However, some items made for wearing must still be activated. Although this activation sometimes requires a command word (see above), usually it means mentally willing the activation to happen. The description of an item states whether a command word is needed in such a case.

Unless stated otherwise, activating a use-activated magic item is either a standard action or not an action at all and does not provoke attacks of opportunity, unless the use involves performing an action that provokes an attack of opportunity in itself. If the use of the item takes time before a magical effect occurs, then use activation is a standard action. If the item's activation is subsumed in its use and takes no extra time use activation is not an action at all.

Use activation doesn't mean that if you use an item, you automatically know what it can do. You must know (or at least guess) what the item can do and then use the item in order to activate it, unless the benefit of the item comes automatically, such from drinking a potion or swinging a sword.

\subsection{Magic Items on the Body}

Many magic items need to be donned by a character who wants to employ them or benefit from their abilities. There are a wide variety of magic items, but there is a limit to how many magical sources a character can sustain at any given time. For humanoid-shaped creatures, there are five areas on the body that can hold magic items: the arms, the head, the neck, the torso, and the legs. A creature can wear only one item in each area, plus two magic rings (one on each hand). Any additional items worn do not function.

Creatures with non-humanoid body structures may have different arrangement of items. For example, a wolf would be able to wear items on its head, its torso, and on two different sets of legs. Regardless of body shape or size, most creatures may not have more than five item areas, plus two ring slots (if applicable).

Arm items:
\begin{itemize*}
\item One pair of bracers or bracelets on the arms or wrists
\item One glove, pair of gloves, or pair of gauntlets on the hands
\end{itemize*}
Head items:
\begin{itemize*}
\item One headband, hat, helmet, or phylactery on the head
\item One pair of eye lenses or goggles on or over the eyes
\end{itemize*}
Neck items:
\begin{itemize*}
\item One amulet, brooch, medallion, necklace, periapt, or scarab around the neck
\end{itemize*}
Torso items:
\begin{itemize*}
\item One vest, vestment, or shirt on the torso
\item One robe or suit of armor on the body (over a vest, vestment, or shirt)
\item One belt around the waist (over a robe or suit of armor)
\item One cloak, cape, or mantle around the shoulders (over a robe or suit of armor)
\end{itemize*}
Leg items:
\begin{itemize*}
\item One pair of boots or shoes on the feet
\end{itemize*}

Of course, a character may carry or possess as many items of the same type as he wishes. However, additional equipped items beyond those listed above have no effect.

Some items can be worn or carried without taking up space on a character's body. The description of an item indicates when an item has this property.

\subsection{Saving Throws Against Magic Item Powers}

Many magic items produce spells or spell-like effects. Part of the DC to resist their effects is based on the level of the spell involved. However, magic items also draw on the power of their wielders to a degree. For a saving throw against a spell or spell-like effect from a magic item, the DC is 10 \add 1/2 the character level or HV of the item's wielder \add the attribute of the minimum attribute score needed to cast that level of spell.

Staffs are an exception to the rule. Treat the saving throw as if the wielder cast the spell, including caster level and all modifiers to save DC.

Most item descriptions give saving throw DCs for various effects, particularly when the effect has no exact spell equivalent (making its level otherwise difficult to determine quickly).

\subsection{Charges, Doses, and Multiple Uses}

Many items, particularly wands and staffs, are limited in power by the number of charges they hold. Normally, charged items have 50 charges at most. If such an item is found as a random part of a treasure, roll d\% and divide by 2 to determine the number of charges left (round down, minimum 1). If the item has a maximum number of charges other than 50, roll randomly to determine how many charges are left.

Prices listed are always for fully charged items. (When an item is created, it is fully charged.) An item with no charges left is worth half the price of a fully charged item. For an item that's worthless when its charges run out (which is the case for almost all charged items), the value of the partially used item is proportional to the number of charges left. For an item that has usefulness in addition to its charges, only part of the item's value is based on the number of charges left.

\subsection{Magic Item Descriptions}

Each general type of magic item gets an overall description, followed by descriptions of specific items.

General descriptions include notes on activation, random generation, and other material. The AC, hardness, hit points, and break DC are given for typical examples of some magic items. The AC assumes that the item is unattended and includes a \minus5 penalty for the item's effective Dexterity of 0. If a creature holds the item, use the creature's Dexterity in place of the \minus5 penalty.

Some individual items, notably those that simply store spells and nothing else, don't get full-blown descriptions. Reference the spell's description for details, modified by the form of the item (potion, scroll, wand, and so on). Assume that the spell is cast at the minimum level required to cast it

Items with full descriptions have their powers detailed, and each of the following topics is covered in notational form at the end of the description.

\begin{itemize*}
\itemhead{Aura:} Most of the time, a \spell{detect magic} spell will reveal the school of magic associated with a magic item and the strength of the aura an item emits. This information (when applicable) is given at the beginning of the item's notational entry. See the \spell{detect magic} spell description for details.

\itemhead{Caster Level:} The next item in a notational entry gives the caster level of the item, indicating its relative power. The caster level determines the item's saving throw bonus, as well as other level-dependent aspects of the powers of the item (if variable). It also determines the level that must be contended with should the item come under the effect of a dispel magic spell or similar situation. This information is given in the form ``CL x,'' where ``CL'' is an abbreviation for caster level and ``x'' is a number representing the caster level itself.

For potions, scrolls, and wands, the creator can set the caster level of an item at any number high enough to cast the stored spell and not higher than her own caster level. For other magic items, the caster level is determined by the item itself. In this case, the creator's caster level must be as high as the item's caster level (and prerequisites may effectively put a higher minimum on the creator's level).

\itemhead{Prerequisites:} Certain requirements must be met in order for a character to create a magic item. These include feats, spells, and miscellaneous requirements such as level, alignment, and race or kind. The prerequisites for creation of an item are given immediately following the item's caster level.

A spell prerequisite may be provided by a character who knows the spell, or through the use of a spell completion or spell trigger magic item or a spell-like ability that produces the desired spell effect. For each day that passes in the creation process, the creator must expend one spell completion item or one charge from a spell trigger item if either of those objects is used to supply a prerequisite.

It is possible for more than one character to cooperate in the creation of an item, with each participant providing one or more of the prerequisites. In some cases, cooperation may even be necessary.

If two or more characters cooperate to create an item, they must agree among themselves who will be considered the creator for the purpose of determinations where the creator's level must be known. Both characters suffer the negative levels required to make the item equally.

Typically, a list of prerequisites includes one feat and one or more spells (or some other requirement in addition to the feat).

When two spells at the end of a list are separated by ``or,'' one of those spells is required in addition to every other spell mentioned prior to the last two.

\itemhead{Market Price:} This gold piece value, given following the word ``Price,'' represents the price someone should expect to pay to buy the item. The market price for an item that can be constructed with an item creation feat is usually equal to the base price plus the price for any components.

\itemhead{Cost to Create:} The next part of a notational entry is the cost in gp to create the item, given following the word ``Cost.'' This information appears only for items with components which make their market prices higher than their base prices. The cost to create includes the costs derived from the base cost plus the costs of the components.

Items without components do not have a ``Cost'' entry. For them, the market price and the base price are the same. The cost in gp is 1/2 the market price.

\itemhead{Weight:} The notational entry for many wondrous items ends with a value for the item's weight. When a weight figure is not given, the item has no weight worth noting (for purposes of determining how much of a load a character can carry).
\end{itemize*}

\section{Armor}

In general, magic armor protects the wearer to a greater extent than nonmagical armor. Magic armor bonuses are enhancement bonuses, never rise above \plus5, and stack with regular armor bonuses (and with shield and magic shield enhancement bonuses). All magic armor is also masterwork armor, reducing armor check penalties by 1. The cost for making armor masterwork is subsumed into the cost of the magic on the armor, and does not need to be paid separately.

In addition to an enhancement bonus, armor may have special abilities. Special abilities usually  have bonus values, but do not improve AC. The special ability bonus value is used to determine the price of the special abilities. Like regular enhancement bonuses, the special ability bonus cannot exceed \plus5. A suit of armor with a special ability must have at least a \plus1 enhancement bonus.

A suit of armor or a shield may be made of an unusual material.

Armor is always created so that even if the type of armor comes with boots or gauntlets, these pieces can be switched for other magic boots or gauntlets.

\begin{dtable}
\caption{Armor and Shields}
\begin{tabularx}{\columnwidth} {>{\ccol}X c >{\ccol}X c}
\thead{Item} & \thead{Base Price} & \thead{Item} & \thead{Price Modifier} \\
\plus1 armor/shield & 1,000 gp & \plus1 special ability & \plus1,000 gp \\
\plus2 armor/shield & 4,000 gp & \plus2 special ability & \plus4,000 gp \\
\plus3 armor/shield & 9,000 gp & \plus3 special ability & \plus9,000 gp \\
\plus4 armor/shield & 16,000 gp & \plus4 special ability & \plus16,000 gp \\
\plus5 armor/shield & 25,000 gp & \plus5 special ability & \plus25,000 gp \\
\end{tabularx}
\end{dtable}

\parhead{Caster Level for Armor and Shields} The caster level of a magic shield or magic armor with a special ability is given in the item description. For an item with only an enhancement bonus, the caster level is three times the enhancement bonus. If an item has both an enhancement bonus and a special ability, the higher of the two caster level requirements must be met.

\parhead{Shields} Shield enhancement bonuses stack with armor enhancement bonuses. Shield enhancement bonuses do not act as attack or damage bonuses when the shield is used in a bash. The bashing special ability, however, does grant a \plus1 enhancement bonus on attack and damage rolls (see the special ability description).

A shield could be built that also acted as a magic weapon, but the cost of the enhancement bonus on attack rolls would need to be added into the cost of the shield.

As with armor, special abilities built into the shield add to the market value, although they do not improve AC. A shield cannot have an enhancement bonus or special ability bonus higher than \plus5. A shield with a special ability must have at least a \plus1 enhancement bonus.

\parhead{Hardness and Hit Points} Each \plus1 of enhancement bonus adds 2 to an armor or shield's hardness and \plus10 to its hit points.

\parhead{Activation} Usually a character benefits from magic armor and shields in exactly the way a character benefits from nonmagical armor and shields - by wearing them. If armor or a shield has a special ability that the user needs to activate then the user usually needs to utter the command word (a standard action).

\parhead{Armor for Unusual Creatures} The cost of armor for nonhumanoid creatures, as well as for creatures who are neither Small nor Medium, varies. The cost of any magical enhancement remains the same.

\subsection{Magic Armor and Shield Special Ability Descriptions}

Most magic armor and shields only have enhancement bonuses. Such items can also have one or more of the special abilities detailed below. Armor or a shield with a special ability must have at least a \plus1 enhancement bonus.

\begin{dtable}
\lcaption{Armor Special Abilities}
\begin{tabularx}{\columnwidth}{>{\lcol}X >{\lcol}X}
\thead{Special Ability} & \thead{Bonus} \\
Fortification, light & \plus1 bonus \\
Shadow & \plus1 bonus \\
Silent moves & \plus1 bonus \\
Slick & \plus1 bonus \\
Spell resistance (13) & \plus2 bonus \\
Energy resistance & \plus3 bonus \\
Fortification, moderate & \plus3 bonus \\
Shadow, improved & \plus3 bonus   \\
Silent moves, improved & \plus3 bonus \\
Slick, improved & \plus3 bonus \\
Spell resistance (15) & \plus3 bonus \\
Energy resistance, improved & \plus4 bonus \\
Spell resistance (17) & \plus4 bonus \\
Energy resistance, greater & \plus5 bonus \\
Fortification, heavy & \plus5 bonus \\
Shadow, greater & \plus5 bonus \\
Silent moves, greater & \plus5 bonus \\
Slick, greater & \plus5 bonus \\
Spell resistance (19) & \plus5 bonus \\
\end{tabularx}
\end{dtable}

\begin{dtable}
\lcaption{Shield Special Abilities}
\begin{tabularx}{\columnwidth}{>{\lcol}X >{\lcol}X}
\thead{Special Ability} & \thead{Bonus} \\
Fortification, light & \plus1 bonus \\
Spell resistance (13) & \plus2 bonus \\
Energy resistance & \plus3 bonus  \\
Fortification, moderate & \plus3 bonus \\
Spell resistance (15) & \plus3 bonus \\
Energy resistance, improved & \plus4 bonus \\
Spell resistance (17) & \plus4 bonus \\
Energy resistance, greater & \plus5 bonus  \\
Fortification, heavy & \plus5 bonus \\
Spell resistance (19) & \plus5 bonus \\
\end{tabularx}
\end{dtable}

\armdescription{Energy Resistance}{4}{12th}{Moderate abjuration}{\x{} and standard (command)} As electricity resistance, except it absorbs the first 20 points of energy damage per attack.

\prereq{Craft Magic Arms and Armor, resist energy}

\armdescription{Energy Resistance, Minor}{3}{8th}{Moderate abjuration}{\x{} and standard (command)} A suit of armor or a shield with this property is often brightly colored. Energy resistance armor absorbs one of the following kinds of energy damage: acid, cold, electricity, fire, or sonic damage. The armor absorbs the first 10 points of energy damage per attack that the wearer would normally take (similar to the resist energy spell). The wearer can change what kind of energy the armor resists by speaking a command word as a standard action. Doing so often changes the color of the armor to match the new energy type.

\prereq{Craft Magic Arms and Armor, resist energy}

\armdescription{Energy Resistance, Major}{5}{16th}{Strong abjuration}{\x{} and standard (command)} As electricity resistance, except it absorbs the first 30 points of energy damage per attack.

\prereq{Craft Magic Arms and Armor, resist energy}

\armdescription{Fortification}{1 (light); \plus3 (moderate); \plus5 (heavy)}{6th (light); 10th (moderate); 14th (heavy)}{Faint abjuration (light); moderate abjuration (moderate); strong abjuration (heavy)}{\x} This suit of armor or shield produces a magical force that protects vital areas of the wearer more effectively. When a critical hit or sneak attack is scored on the wearer, there is a chance that the critical hit or sneak attack is negated and damage is instead rolled normally.

\begin{dtable}
\begin{tabularx}{\columnwidth}{c >{\ccol}X c}
\thead{Fortification Type} & \thead{Chance for Normal Damage} & \thead{Bonus} \\
Light & 25\% & \plus1 bonus \\
Moderate & 75\% & \plus3 bonus \\
Heavy & 100\% & \plus5 bonus \\
\end{tabularx}
\end{dtable}

\prereq{Craft Magic Arms and Armor, ablative barrier or inertial barrier}

\armdescription{Shadow}{1; \plus3 (improved); \plus5 (greater)}{4th; 8th (improved); 12th (greater)}{Faint illusion; moderate illusion (improved and greater)}{\x} This armor is jet black and blurs the wearer whenever she tries to hide, granting a \plus5 enhancement bonus on Hide checks. (The armor's armor check penalty still applies normally.) Improved shadow armor grants a \plus10 enhancement bonus, and greater shadow armor grants a \plus15 enhancement bonus.

\prereq{Craft Magic Arms and Armor, blur}

\armdescription{Silent Moves}{1; \plus3 (improved); \plus5 (greater)}{4th; 8th (improved); 12th (greater)}{Faint illusion; moderate illusion (improved and greater)}{\x} This armor is well oiled and magically constructed so that it not only makes little sound, but it dampens sound around it. It provides a \plus5 enhancement bonus on its wearer's Move Silently checks. (The armor's armor check penalty still applies normally.)  Improved silent moves armor grants a \plus10 enhancement bonus, and greater silent moves armor grants a \plus15 enhancement bonus.

\prereq{Craft Magic Arms and Armor, silence}

\armdescription{Slick}{1; \plus3 (improved); \plus5 (greater)}{4th; 8th (improved); 12th (greater)}{Faint conjuration; moderate conjuration (improved and greater)}{\x} Slick armor seems coated at all times with a slightly greasy oil. It provides a \plus5 enhancement bonus on its wearer's Escape Artist checks. (The armor's armor check penalty still applies normally.) Improved slick armor grants a \plus10 enhancement bonus, and greater slick armor grants a \plus15 enhancement bonus.

\prereq{Craft Magic Arms and Armor, grease}

\armdescription{Spellpower}{1; \plus2; \plus3; \plus4; \plus5}{3rd; 6th; 9th; 12th; 15th}{}{\x} Spellpower armor grants its wearer an enhancement bonus to caster level when casting spells with a specific spell descriptor. Each spellpower armor grants a bonus to a different descriptor. This decision is made when the armor is forged and cannot be changed thereafter.

\begin{comment}
%Move to new SR system
\armdescription{Spell resistance}{2 (SR 13); \plus3 (SR 15); \plus4 (SR 17); \plus5 (SR 19)}{10th}{Moderate abjuration}{\x} This property grants the wearer spell resistance while the armor is worn. The spell resistance can be 13, 15, 17, or 19, depending on the armor.

\prereq{Craft Magic Arms and Armor, spell resistance}
\end{comment}

\section{Weapons}

Magic weapons have enhancement bonuses ranging from \plus1 to \plus5. They apply these bonuses to damage rolls when used in combat.

Weapons come in two basic categories: melee and ranged. Some of the weapons listed as melee weapons can also be used as ranged weapons. In this case, their enhancement bonus applies to either type of attack.

In addition to an enhancement bonus, weapons may have special abilities. Special abilities usually  have bonus values, but do not modify damage bonuses (except where specifically noted). The special ability bonus value is used to determine the price of the special abilities. Like regular enhancement bonuses, the special ability bonus cannot exceed \plus5. A weapon with a special ability must have at least a \plus1 enhancement bonus.

A weapon or a kind of ammunition may be made of an unusual material.

\parhead{Typed Damage} Some weapon effects change the type of the damage dealt by the weapon. When such a weapon strikes a foe, the foe can apply physical damage reduction or magical damage resistance, such as from the \spell{resist energy} spell, but not both. Use the better of the two values. For example, a creature with damage resistance 5/adamantine and fire resistance 10 would resist 10 damage from an attack by a flaming longsword, but only 5 damage from an attack by a frost longsword, and no damage at all from an adamantine frost longsword.

\parhead{Caster Level for Weapons} The caster level of a weapon with a special ability is given in the item description. For an item with only an enhancement bonus and no other abilities, the caster level is three times the enhancement bonus. If an item has both an enhancement bonus and a special ability, the higher of the two caster level requirements must be met.

\parhead{Ranged Weapons and Ammunition} The enhancement bonus from a ranged weapon does not stack with the enhancement bonus from ammunition. Only the higher of the two enhancement bonuses applies.

Ammunition fired from a projectile weapon with an enhancement bonus of \plus1 or higher is treated as a magic weapon for the purpose of overcoming damage reduction. Similarly, ammunition fired from a projectile weapon with an alignment gains the alignment of that projectile weapon (in addition to any alignment it may already have).

\parhead{Magic Ammunition and Breakage} A magic arrow, crossbow bolt, or sling bullet loses its magic after being fired, whether it hits or misses.

\parhead{Light Generation} Fully 30\% of magic weapons shed light equivalent to a light spell (bright light in a 20-foot radius, shadowy light in a 40-foot radius). These glowing weapons are quite obviously magical. Such a weapon can't be concealed when drawn, nor can its light be shut off. Some of the specific weapons detailed below always or never glow, as defined in their descriptions.

\parhead{Hardness and Hit Points} Each \plus1 of enhancement bonus adds 2 to the weapon's hardness and \plus10 to its hit points.

\parhead{Activation} Usually a character benefits from a magic weapon in the same way a character benefits from a mundane weapon - by attacking with it. If a weapon has a special ability that the user needs to activate then the user usually needs to utter a command word (a standard action).

\parhead{Magic Weapons and Critical Hits} Some weapon qualities and some specific weapons have an extra effect on a critical hit. This special effect functions against creatures not subject to critical hits. When fighting against such creatures, roll for critical hits as you would against any other creature subject to critical hits. On a successful critical roll, apply the special effect, but do not multiply the weapon's regular damage.

\parhead{Weapons for Unusually Sized Creatures} The cost of weapons for creatures who are neither Small nor Medium varies. The cost of any magical enhancement remains the same.

\begin{dtable}
\caption{Weapons}
\begin{tabularx}{\columnwidth} {>{\ccol}X c >{\ccol}X c}
\thead{Item} & \thead{Base Price} & \thead{Item} & \thead{Price Modifier} \\
\plus1 armor/shield & 1,000 gp & \plus1 special ability & \plus1,000 gp \\
\plus2 armor/shield & 4,000 gp & \plus2 special ability & \plus4,000 gp \\
\plus3 armor/shield & 9,000 gp & \plus3 special ability & \plus9,000 gp \\
\plus4 armor/shield & 16,000 gp & \plus4 special ability & \plus16,000 gp \\
\plus5 armor/shield & 25,000 gp & \plus5 special ability & \plus25,000 gp \\
\end{tabularx}
\end{dtable}

\section{Potions and Oils}

A potion is a magic liquid that produces its effect when imbibed. Magic oils are similar to potions, except that oils are applied externally rather than imbibed. A potion or oil can be used only once. It can duplicate the effect of a spell of up to 3rd level that has a casting time of less than 1 minute.

Potions are like spells cast upon the imbiber. The character taking the potion doesn't get to make any decisions about the effect  - the caster who brewed the potion has already done so. The drinker of a potion is both the effective target and the caster of the effect (though the potion indicates the caster level, the drinker still controls the effect).

The person applying an oil is the effective caster, but the object is the target.

\parhead{Physical Description} A typical potion or oil consists of 1 ounce of liquid held in a ceramic or glass vial fitted with a tight stopper. The stoppered container is usually no more than 1 inch wide and 2 inches high. The vial has AC 13, 1 hit point, hardness 1, and a break DC of 12. Vials hold 1 ounce of liquid.

\parhead{Identifying Potions} In addition to the standard methods of identification, PCs can sample from each container they find to attempt to determine the nature of the liquid inside. An experienced character learns to identify potions by memory - for example, the last time she tasted a liquid that reminded her of almonds, it turned out to be a potion of cure moderate wounds.

\parhead{Activation} Drinking a potion or applying an oil requires no special skill. The user merely removes the stopper and swallows the potion or smears on the oil. The following rules govern potion and oil use.

Drinking a potion or using an oil on an item of gear is a standard action. The potion or oil takes effect immediately. Using a potion or oil provokes attacks of opportunity. A successful attack (including grappling attacks) against the character forces a Concentration check (as for casting a spell). If the character fails this check, she cannot drink the potion. An enemy may direct an attack of opportunity against the potion or oil container rather than against the character. A successful attack of this sort can destroy the container.

A creature must be able to swallow a potion or smear on an oil. Because of this, incorporeal creatures cannot use potions or oils.

Any corporeal creature can imbibe a potion. The potion must be swallowed, or in some other way ingested. Any corporeal creature can use an oil.

A character can carefully administer a potion to an unconscious creature as a full-round action, trickling the liquid down the creature's throat. Likewise, it takes a full-round action to apply an oil to an unconscious creature. Exceptionally large objects or creatures require a greater time expenditure.

\subsubsection{Potion Descriptions}

The caster level for a standard potion is the minimum caster level needed to cast the spell (unless otherwise specified).

\begin{dtable}
\lcaption{Potions and Oils}
\begin{tabularx}{\columnwidth}{c c c}
Potion or Oil & Market Price & Extra Price Modifier\\
1st-level spell (common) & 50 gp  & \plus50 gp per caster level \\
1st-level spell (uncommon) & 75 gp & \plus75 gp per caster level \\
2nd-level spell (common) & 400 gp & \plus100 gp per caster level \\
2nd-level spell (uncommon) & 600 gp & \plus200 gp per caster level \\
3rd-level spell (common) & 900 gp & \plus150 gp per caster level \\
3rd-level spell (uncommon) & 1350 gp & \plus225 gp per caster level
\end{tabularx}
\end{dtable}

\subsection{Rings}

Rings bestow magical powers upon their wearers. Only a rare few have charges. Anyone can use a ring.

A character can only effectively wear two magic rings. A third magic ring doesn't work if the wearer is already wearing two magic rings.

\parhead{Physical Description} Rings have no appreciable weight. Although exceptions exist that are crafted from glass or bone, the vast majority of rings are forged from metal - usually precious metals such as gold, silver, and platinum. A ring has AC 13, 2 hit points, hardness 10, and a break DC of 25.

\parhead{Activation} Usually, a ring's ability is activated by a command word (a standard action that does not provoke attacks of opportunity) or it works continually. Some rings have exceptional activation methods, according to their descriptions.

\begin{comment}
\begin{dtable}
\lcaption{Rings}
\begin{tabularx}{\columnwidth}{>{\lcol}X l}
Ring & Market Price \\
Protection \plus1 & 2,000 gp \\
Feather falling & 2,200 gp \\
Climbing & 2,500 gp \\
Jumping & 2,500 gp \\
Sustenance & 2,500 gp \\
Swimming & 2,500 gp \\
Mind shielding & 8,000 gp \\
Protection \plus2 & 8,000 gp \\
Climbing, improved & 10,000 gp \\
Jumping, improved & 10,000 gp \\
Swimming, improved & 10,000 gp \\
Energy resistance, minor & 12,000 gp \\
Protection \plus3 & 18,000 gp \\
Energy resistance, major & 28,000 gp \\
Protection \plus4 & 32,000 gp \\
Energy resistance, greater & 44,000 gp \\
Protection \plus5 & 50,000 gp \\
\end{tabularx}
\end{dtable}
\end{comment}

\section{Rods}

Rods are scepterlike devices that have unique magical powers and do not usually have charges. Anyone can use a rod.

\parhead{Physical Description} Rods weigh approximately 5 pounds.

They range from 2 feet to 3 feet long and are usually made of iron or some other metal. (Many, as noted in their descriptions, can function as light maces or clubs due to their sturdy construction.)

These sturdy items have AC 9, 10 hit points, hardness 10, and a break DC of 27.

\parhead{Activation} Details relating to rod use vary from item to item. See the individual descriptions for specifics.

\section{Scrolls}
A scroll is a spell (or collection of spells) that has been stored in written form. A spell on a scroll can be used only once. The writing vanishes from the scroll when the spell is activated. Using a scroll is basically like casting a spell.

\parhead{Physical Description} A scroll is a heavy sheet of fine vellum or high-quality paper. An area about 8 1/2 inches wide and 11 inches long is sufficient to hold one spell. The sheet is reinforced at the top and bottom with strips of leather slightly longer than the sheet is wide. A scroll holding more than one spell has the same width (about 8 1/2 inches) but is an extra foot or so long for each extra spell. Scrolls that hold three or more spells are usually fitted with reinforcing rods at each end rather than simple strips of leather. A scroll has AC 9, 1 hit point, hardness 0, and a break DC of 8.

To protect it from wrinkling or tearing, a scroll is rolled up from both ends to form a double cylinder. (This also helps the user unroll the scroll quickly.) The scroll is placed in a tube of ivory, jade, leather, metal, or wood. Most scroll cases are inscribed with magic symbols which often identify the owner or the spells stored on the scrolls inside. The symbols often hide magic traps.

\parhead{Activation} To activate a scroll, a spellcaster must read the spell written on it. Doing so involves several steps and conditions.

\parhead{Decipher the Writing} The writing on a scroll must be deciphered before a character can use it or know exactly what spell it contains. This requires a read magic spell or a successful Spellcraft check (DC 20 \add spell level).

Deciphering a scroll to determine its contents does not activate its magic unless it is a specially prepared cursed scroll. A character can decipher the writing on a scroll in advance so that he or she can proceed directly to the next step when the time comes to use the scroll.

\parhead{Activate the Spell} Activating a scroll requires reading the spell from the scroll. The character must be able to see and read the writing on the scroll. Activating a scroll spell requires no material components or focus. (The creator of the scroll provided these when scribing the scroll.) Note that some spells are effective only when cast on an item or items. In such a case, the scroll user must provide the item when activating the spell. Activating a scroll spell is subject to disruption just as casting a normally prepared spell would be. Using a scroll is like casting a spell for purposes of arcane spell failure chance.

To have any chance of activating a scroll spell, the scroll user must meet the following requirements.
\begin{itemize*}
\item The spell must be of the correct type (arcane or divine). Arcane spellcasters (wizards, sorcerers, and bards) can only use scrolls containing arcane spells, and divine spellcasters (clerics, druids, paladins, and rangers) can only use scrolls containing divine spells. (The type of scroll a character creates is also determined by his or her class.)
\item The user must have the spell on his or her spell list.
\item The user must have the requisite attribute score.
\end{itemize*}

If the user meets all the requirements noted above, and her caster level is at least equal to the spell's caster level, she can automatically activate the spell without a check. If she meets all three requirements but her own caster level is lower than the scroll spell's caster level, then she has to make a caster level check (DC = scroll's caster level \add 1) to cast the spell successfully. If she fails, she must make a DC 5 Wisdom check to avoid a mishap (see Scroll Mishaps, below). A natural roll of 1 always fails, whatever the modifiers.

\parhead{Determine Effect} A spell successfully activated from a scroll works exactly like a spell cast the normal way. Assume the scroll spell's caster level is always the minimum level required to cast the spell for the character who scribed the scroll (usually twice the spell's level, minus 1), unless the caster specifically desires otherwise.

The writing for an activated spell disappears from the scroll.

\parhead{Scroll Levels} Some spells are acquired by multiple classes at different levels. Use the entry on the table appropriate to the scribing of each individual scroll.

\begin{dtable}
\lcaption{Spell Scrolls}
\begin{tabularx}{\columnwidth}{>{\lcol}X l}
\thead{Common Scrolls} & \thead{Market Price} \\
0-Level Spells  & 12 gp 5 sp \\
1st-Level Spells & 50 gp \\
2nd-Level Spells & 200 gp \\
3rd-Level Spells & 450 gp \\
4th-Level Spells & 800 gp \\
5th-Level Spells & 1250 gp \\
6th-Level Spells & 1800 gp \\
7th-Level Spells & 2450 gp \\
8th-Level Spells & 3200 gp \\
9th-Level Spells & 4050 gp \\
\thead{Bard Scrolls} & \thead{Market Price} \\
1st-Level Bard Spells & 50 gp \\
2nd-Level Bard Spells & 200 gp \\
3rd-Level Bard Spells & 525 gp \\
4th-Level Bard Spells & 1000 gp \\
5th-Level Bard Spells & 1625 gp \\
6th-Level Bard Spells & 2400 gp \\
\thead{Paladin/Ranger Scrolls} & \thead{Market Price\fn{2}} \\
1st-Level Paladin/Ranger Spells & 50 gp \\
2nd-Level Paladin/Ranger Spells & 500 gp \\
3rd-Level Paladin/Ranger Spells & 1200 gp \\
4th-Level Paladin/Ranger Spells & 2200 gp \\
\end{tabularx}
1 Includes cleric, druid, sorcerer, and wizard spells \\
2 Scrolls of paladin and ranger spells cost twice as much to buy because of their rarity. The cost to scribe them is no different than normal, and players attempting to sell such scrolls will find it difficult to find a buyer, so such items sell for a quarter of their market price.
\end{dtable}

\section{Staffs}

A staff is a long shaft of wood that stores several spells. Unlike wands, which can contain a wide variety of spells, each staff is of a certain kind and holds specific spells. A staff has 50 charges when created.

\parhead{Physical Description} A typical staff is 4 feet to 7 feet long and 2 inches to 3 inches thick, weighing about 5 pounds. Most staffs are wood, but a rare few are bone, metal, or even glass. (These are extremely exotic.) Staffs often have a gem or some device at their tip or are shod in metal at one or both ends. Staffs are often decorated with carvings or runes. A typical staff is like a walking stick, quarterstaff, or cudgel. It has AC 7, 10 hit points, hardness 5, and a break DC of 24.

\parhead{Activation} Staffs use the spell trigger activation method, so casting a spell from a staff is usually a standard action that doesn't provoke attacks of opportunity. (If the spell being cast, however, has a longer casting time than 1 standard action, it takes that long to cast the spell from a staff.) To activate a staff, a character must hold it forth in at least one hand (or whatever passes for a hand, for nonhumanoid creatures).

\section{Wands}

A wand is a thin baton that contains a single spell of 4th level or lower. Each wand has 25 charges when created, and each charge expended allows the user to use the wand's spell one time. A wand that runs out of charges is just a stick.

\parhead{Physical Description} A typical wand is 6 inches to 12 inches long and about 1/4 inch thick, and often weighs no more than 1 ounce. Most wands are wood, but some are bone. A rare few are metal, glass, or even ceramic, but these are quite exotic. Occasionally, a wand has a gem or some device at its tip, and most are decorated with carvings or runes. A typical wand has AC 7, 5 hit points, hardness 5, and a break DC of 16.

\parhead{Activation} Wands use the spell trigger activation method, so casting a spell from a wand is usually a standard action that doesn't provoke attacks of opportunity. (If the spell being cast, however, has a longer casting time than 1 action, it takes that long to cast the spell from a wand.) To activate a wand, a character must hold it in hand (or whatever passes for a hand, for nonhumanoid creatures) and point it in the general direction of the target or area. A wand may be used while grappling or while swallowed whole.

\begin{dtable}
\lcaption{Wands}
\begin{tabularx}{\columnwidth}{>{\lcol}X l}
\thead{Wand} & \thead{Market Price} \\
0-Level Spells  & 375 sp \\
1st-Level Spells\fn{1} & 500 gp \\
2nd-Level Spells & 2000 gp \\
3rd-Level Spells & 4500 gp \\
4th-Level Spells & 8000 gp \\
\thead{Bard Wands} & \thead{Market Price} \\
1st-Level Bard Spells & 500 gp \\
2nd-Level Bard Spells & 2000 gp \\
3rd-Level Bard Spells & 5250 gp \\
4th-Level Bard Spells & 10000 gp \\
\thead{Paladin/Ranger Wands} & \thead{Market Price\fn{2}} \\
1st-Level Paladin/Ranger Spells & 1000 gp \\
2nd-Level Paladin/Ranger Spells & 4000 gp \\
3rd-Level Paladin/Ranger Spells & 7500 gp \\
4th-Level Paladin/Ranger Spells & 14000 gp \\
\end{tabularx}
1 Includes cleric, druid, sorcerer, and wizard spells \\
2 Wands of paladin and ranger spells cost twice as much to buy because of their rarity. The cost to craft them is no different than normal, and players attempting to sell such wands will find it difficult to find a buyer, so such items sell for a quarter of their market price.
\end{dtable}

\subsection{Wand Descriptions}

All wands are simply storage devices for spells and thus have no special descriptions. Refer to the spell descriptions for all pertinent details.