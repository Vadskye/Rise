\documentclass{article}
\usepackage{wallpaper}
\begin{document}
\ThisCenterWallPaper{1}{leftPage.jpg}
   \begin{center}
      \begin{Large}     
                      Latex Sample File 1 \\
      \end{Large}
      \begin{large}
                 Basics:  Paragraphs and Environments \\
      \end{large}
                      Matchett, \today \\
                    La Crosse, WI  54601
   \end{center}
A minimal LaTeX file can begin with a  ``documentclass'' 
command followed immediately by a ``begin document'' command.
Type the ``end document'' command several lines below the
``begin document'' command, so you do not forget it.  The
material to be typed then goes between the ``begin document'' 
command and the ``end document'' command.  If the document you 
are typing has few mathematical symbols then the typing is easy.  
Use blank lines to separate paragraphs.  Avoid using blank lines 
for other purpose (except that one or more blank lines are ok 
right before the final ``end\{document\}'' command).  Note that 
      indenting in the source file is ignored by LaTeX, and it 
does not matter how long the lines  
are or how large       the       spaces are between        words.  
The LaTeX software will take care of all the formatting for you.  
This document has three paragraphs.

This file uses several special environments.  An environment is 
initiated by a ``begin'' command and ended by an ``end'' command.  
The words ``begin'' and ``end'' must each be preceeded with a 
backslash, and the name of the environment goes in curly braces 
after the words.  This document uses the environments, 
\emph{center}, \emph{Large}, \emph{large}, and \emph{verbatim}.  
In the \emph{center} environment, two backslashes indicate the 
end of a line, except in the case of the last line in the 
environment which never ends with two backslashes.

There are several special symbols that may not be used as is 
because they have special meaning for LaTeX.  They are the 
following: 
 \begin{center}      
    ``number symbol'', dollar sign, percent symbol, ampersand, \\ 
      tilde,   underline symbol,  exponent symbol,  backslash  \\
      curly left bracket \\
      curly right bracket
 \end{center}
Seven of these symbols can be included in a document by just 
preceeding the symbol with a backslash.  Here they are: 
 \begin{center}
       \# \$ \% \& \_ \{ \}
 \end{center}
 \begin{verbatim} 
The characters,  ~ ,  ^ ,  and  \  seldom need 
to appear in the final document.
 \end{verbatim} 
They can be produced in the ``verbatim'' environment and in math 
mode by special commands.  The percent character, when it is not 
preceeded by a backslash, causes all that 
follows it on a line to be omitted from the compiled document. 
% This gives a useful way to include comments that the writer does
% not want printed.  
\ClearShipoutPicture
\end{document}