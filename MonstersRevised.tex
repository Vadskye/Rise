\section{Monsters}

\subsection{Dire Animals}
\subsubsection{Dire Bear}
\parhead{Level} 12
\parhead{Archetypes} Brute 6, Disabler 1, Tank 5
\parhead{Attributes} 
\parhead{Traits} (4+12+6) Natural Weapon I x2, Natural Weapon II x2, Natural Armor I, Natural Armor II x2, Senses I x2, 

\subsubsection{Dire Lion}
\parhead{Level} 10
\parhead{Archetypes} Brute 7, Disabler 2, Tank 1
\parhead{Attributes} 5, 6, 4, \minus8, 1, 0
\parhead{Traits} (4+10+5) Natural Weapon I x2, Natural Weapon II x2, Natural Armor I, Senses I x2, Pounce, Area Debuff (shaken), Widen Area Debuff x2, Attribute (Dex) x3, HV Increase, Attribute (Str) x2, Attribute (Con), Size Increase

\subsubsection{Dire Wolf}
\parhead{Level} 8
\parhead{Archetypes} Brute 6, Disabler 1, Tank 1
\parhead{Attributes} 7, 2, 4, \minus8, 1, 0
\parhead{Traits} (4+8+4) Natural Weapon I, Natural Weapon II x2, Natural Armor I, Natural Armor II x2, Improved Trip, Size Increase, Senses I x2, Attribute (Strength) x4, Attribute (Con) x1, HV Increase

\subsection{Animals}

\subsubsection{Bear, Black}
\parhead{Level} 3
\parhead{Archetypes} Brute 1, Disabler 1, Tank 1
\parhead{Attributes} 4/0/5/\minus8/2/0
\parhead{Traits} (2+5) Natural Weapon I x2, Natural Armor I, Attr (Str), Attribute (Con x2), Improved Grab

\subsubsection{Bear, Brown}
\parhead{Level} 6
\parhead{Archetypes} Brute 2, Disabler 1, Tank 3
\parhead{Attributes} 6/0/6/\minus8/3/0
\parhead{Traits} (2+9) Natural Weapon I x2, Natural Armor I, NA II, Attr (Str), Attribute (Con x3), Attr (Wis), Improved Grab, Size Increase (Large)

\section{Monster Design}

Creating a monster involves following a series of steps. These steps need not be followed in the order given, as long as all of them are done.

\begin{enumerate*}
    \item Choose a level for the monster
    \item Choose a creature type (Animal, Dragon, etc.)
    \item Choose one or more archetypes (Warrior, Scout, etc.)
    \item Assign attributes
    \item Choose traits (special abilities unique to monsters)
    \item Choose skills and feats
    \item Choose equipment (if any)
\end{enumerate*}

\subsection{Monster Level}
A monster's level affects many things. Like a PC, a monster's level affects its core statistics, such as hit points, base attack bonus, and saving throws. In addition, monsters gain a number of special abilities based on their level, like class features for PCs.

\subsection{Creature Types}
\thv Hit points gained at each level.
\tbab Base attack bonus progression (Good, Average, or Poor).
\tsaves Saving throw progressions (Good, Average, or Poor).
\parhead{Attributes} The number of points the creature can spend on its attributes. If the creature has a special modifier to its attributes, such as the low Intelligence of animals, that is also listed here.
\parhead{Automatic Traits} Many creature types automatically grant certain traits. These are listed here. Individual creatures may not have these traits if appropriate.

\subsubsection{Aberration}
\thv 5
\tbab Average
\tsaves Average Fort, Will; poor Ref
\parhead{Attributes} 10 points
\parhead{Automatic Traits} Natural Weapon, Natural Armor

\subsubsection{Animal}
\thv 5
\tbab Average
\tsaves Average Fort, Ref; poor Will
\parhead{Attributes} 10 points, -8 Intelligence (max -5)
\parhead{Automatic Traits} Natural Weapon, Natural Armor, Senses (Low-light vision, Scent)

\subsection{Archetypes}
Archetypes are thematic representations of the powers a creature has and the role it plays in a typical combat. For example, a Warrior is more powerful in physical combat, while a Brute is more difficult to kill. A creature may have any number of archetypes.

Each archetype has two main effects. First, archetypes often change a creature's base progressions for attacks, saving throws, or hit points. Second, each archetype has an associated list of traits called a ``trait pool''. Creatures gain bonus traits from each of the archetypes they have, chosen from the traits in that archetype's trait pool.

For each archetype the creature has, it gains trait points equal to its level which can be spent on traits from that archetype's trait pool.

\subsubsection{Brute}
\tbab Increases by one step
\tsaves Fort increases by one step
\parhead{Trait Pool}
\begin{itemize*}
    \item Attribute (Str, Con)
    \item Combat Feat (Power)
    \item Natural Weapon I, II
    \item Poison I, II
    \item Size Increase
    \item Trample
\end{itemize*}

\subsubsection{Disabler}
\parhead{Trait Pool}
\begin{itemize*}
    \item Improved Grab
\end{itemize*}

\subsection{Monster Traits}
Traits are unique abilities that monsters gain
Some traits can be taken multiple times. Unless it says otherwise, no trait can be taken more times than half your HD (min 1).

\subsubsection{Attribute}
Choose an attribute.
\featben You gain a \plus1 inherent bonus to the appropriate attribute.
\parhead{Special} This feat can be taken multiple times. Its effects stack. You can choose a different attribute or the same attribute each time. You can apply the Attribute trait to a single attribute no more times than half your HD (min 1).

\subsubsection{Area Debuff}
\featben A number of times per day equal to 1 \add half your Constitution or Charisma, you can cause enemies in a \areasmall radius burst around you, or in a \areamed line or cone from you, to suffer a tier 3 condition unless they make a saving throw. The DC is equal to 10 \add your level \add your Constitution or Charisma.

\subsubsection{Area Debuff, Widen}
\featpre Area Debuff
\featben The area affected by your area debuff increases by one tier (\areasmall, to \areamed, to \arealarge, to 100 feet).

\subsubsection{Combat Feat}
Choose a combat feat.
\featben You gain that combat feat.
\parhead{Special} You can take this trait multiple times. Each time, you choose a different feat.

\subsubsection{HV Increase}
\featben You increase your hit value by 1, to a max of 7.
\subsubsection{Improved Grab}
\featben You gain the improved grab special ability. 

\subsubsection{Natural Armor I}
\featben You gain a \plus2 natural armor modifier.

\subsubsection{Natural Armor II}
\featpre Natural Armor I
\featben You increase your natural armor by 1.
\parhead{Special} You can take this trait multiple times. Its effects stack.

\subsubsection{Natural Weapon I}
\featben You gain a natural weapon that deals damage appropriate for your size from the following list. The damage listed is for Medium creatures.
\begin{itemize*}
    \item Bite (d8)
    \item Claws (d6/d6)
    \item Slam (d8)
\end{itemize*}

\subsubsection{Natural Weapon II}
\featpre Natural Weapon I
\featben You increase the damage die of one of your natural weapons by one category.
\parhead{Special} You can take this trait multiple times. Each time, you can apply it to a different natural weapon or to the same natural weapon. You cannot apply this trait to the same weapon more times than 1/2 your Constitution. 

\subsubsection{Poison I}
Choose one of your attacks.
\featben When you deal damage with the chosen attack, the struck creature must make a Fortitude save or be afflicted with poison. The DC for the save is equal to 10 \add 1/2 your HD \add Constitution.
\par The poison deals 1 damage each round to an attribute of your choice (other than Constitution). It continues until the subject makes two successful Fortitude saves against the poison.

\subsubsection{Poison II}
\featpre Poison I
\featben Your poison deals 1d4 damage per round to an attribute other than Constitution, or 1 damage per round to Constitution.

\subsubsection{Poison, Extended}
\featpre Poison I
\featben Your poison lasts until the subject makes three successful Fortitude saves.

\subsubsection{Resistance I}
\featben You gain damage reduction equal to your HV \add Constitution against any of the damage types listed below. The damage reduction must be overcome by another of these types.
\begin{itemize*}
    \item Aligned damage, single (chaotic, good, evil, lawful)
    \item Energy damage, single (acid, cold, electricity, or fire)
    \item Life damage
    \item Physical damage, single (bludgeoning, slashing, piercing, nonmagical)
    \item Solar damage
\end{itemize*}

\subsubsection{Resistance, Expanded}
\featpre Resistance I
\featben Your damage reduction changes to resist any of the damage types listed below.
\begin{itemize*}
    \item Energy damage, all
    \item Physical damage, all
    \item Spell damage
\end{itemize*}

\subsubsection{Senses I}
\featben You gain one of the following senses.
\begin{itemize*}
    \item Darkvision 60'
    \item Low-light vision
    \item Scent
\end{itemize*}

\subsubsection{Size Increase}
\featpre Medium size or smaller, Attribute (Strength) x2.
\featben Your size increases by one category.
\parhead{Special} You can take this trait multiple times. You must have Attribute (Strength) x2 to become Large, x4 to become Huge, x6 to become Gargantuan, and x8 to become Colossal.

\subsubsection{Trample I}
\featpre Large size or larger.
\featben You gain the trample special ability. It deals 1d10 damage \add Strength for a Large creature, and the save DC is equal to 10 \add 1/2 your HD \add Strength.

\subsection{Templates}
\subsubsection{Dire}
\featpre Animal type
\featben Base attack bonus progression improves by 1 step. HV improves by 1.
