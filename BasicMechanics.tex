\chapter{Basic Mechanics}

\section{Attributes}
Each character has six attributes: Strength (Str), Dexterity (Dex), Constitution (Con), Intelligence (Int), Perception (Per), and Willpower (Wil). Each attribute represents a character's raw talent in that area. A 0 in an attribute represents average human capacity. That doesn't mean that every commoner has a 0 in every attribute; not everyone is average, after all.

\subsection{Attribute Descriptions}

\subsubsection{Strength (Str)}
Strength measures muscle and physical power.
\begin{itemize}
    \item Strength determines how much a character can carry (see \tref{Weight Limits}.
    \item Strength can be used to attack with melee and thrown weapons.
    \item Strength can be used for Maneuver defense.
    \item Strength can be used for Climb, Jump, Sprint, and Swim checks.
    \item Half Strength is added to Fortitude defense, which can affect your hit points.
\end{itemize}

\subsubsection{Dexterity (Dex)}
Dexterity measures hand-eye coordination, agility, and reflexes.
\begin{itemize}
    \item Dexterity can be used to attack with melee and thrown weapons that are light.
    \item Dexterity can be used for all physical defenses (Armor, Maneuver, and Reflex).
    \item Dexterity can be used for Balance, Escape Artist, Ride, Sleight of Hand, Stealth, and Tumble checks.
\end{itemize}

\subsubsection{Constitution (Con)}
Constitution represents your character's health and stamina.
\begin{itemize}
    \item Constitution can be used for Armor defense.
    \item Constitution is added to Fortitude defense, which can affect your hit points.
\end{itemize}

\subsubsection{Intelligence (Int)}
Intelligence determines how well your character learns and reasons. It affects your character's knowledge in many areas.

\begin{itemize}
    \item Intelligence is added to Craft, Disguise, Heal, Knowledge, and Linguistics checks.
    \item Half Intelligence is added to the number of skill points a character gets.
    \item Half Intelligence is added to Mental defense, which can affect your hit points.
\end{itemize}

\par An animal has an Intelligence score of \minus6 or lower. A creature of humanlike intelligence has a score of at least a \minus5 Intelligence.

\subsubsection{Perception (Per)}
Perception describes a character's ability to observe and be aware of one's surroundings.
\begin{itemize}
    \item Perception can be used to attack with projectile weapons.
    %\item Perception can be used for Reflex defense.
    \item Perception can be used for Awareness, Creature Handling, Sense Motive, Spellcraft, and Survival checks.
    \item Half Perception is added to Reflex defense.
\end{itemize}

\subsubsection{Willpower (Wil)}
Willpower measures a character's ability to endure mental hardships.
\begin{itemize}
    \item Willpower can be used for Mental defense, which can affect your hit points.
\end{itemize}

\subsection{Using Attributes}

\subsubsection{Choosing Attributes to Use}
In many cases, multiple attributes can be used for the same thing. For example, both Strength and Dexterity can be used to attack with light weapons such as daggers. Whenever more than one attribute could be used, you must choose which one to use (usually, the higher attribute).

\subsubsection{Attributes and Spellcasting}
Using magic requires a strong mind. In order to cast spells, you must have a high Intelligence, Perception, or Willpower, depending on your class. Intelligence is required for wizards, Perception is required for druids, and Willpower is required for clerics and sorcerers. For more details, see \pcref{Casting Spells}.

\subsection{Determining Attributes}
There are several options for how to determine attribute scores.

\subsubsection{Predefined Attribute Scores}
This is the simplest method. Simply take the following set of attribute scores and distribute them as you choose among your character's abilities:

4, 3, 2, 1, 0, \minus1

This set of attribute scores is called the ``elite array''. For more extreme characters, you may use the ``savant array'':

5, 2, 1, 0, 0, \minus2.

Finally, for more well-balanced characters, you may use the ``balanced array'':

3, 3, 2, 1, 1, 0

\subsubsection{Point Buy}
With this method, you can fully control your character's attribute scores to match what you want your character to be. All your character's attribute scores start at 0. You get 10 points to distribute among your character's attribute scores. Attribute scores can be bought according to the costs on \trefnp{Attribute Score Point Costs}.

\begin{dtable}
    \lcaption{Attribute Score Point Costs}
    \begin{dtabularx}{\columnwidth}{X X X X}
        \thead{Attribute Score} & \thead{Point Cost} \\
\hline
        \hline
        \minus2 & \minus2\fn{1} \\
        \minus1 & \minus1\fn{1} \\
        0 & 0 \\
        1 & 1 \\
        2 & 2 \\
        3 & 3 \\
        4 & 5 \\
        5 & 8 \\
    \end{dtabularx}
    1 No more than two attribute scores can be reduced below 0 in this way.
\end{dtable}

\subsection{Changing Attributes}

Your attributes increase as you gain levels (see \pcref{Character Advancement}), and some special abilities can also increase your attributes, either permanently or for a brief period of time.
When an attribute changes, abilities and modifiers based on the attribute change at different times, as shown on \trefnp{Effects of Changing Attributes}.

\begin{dtable*}
    \lcaption{Effects of Changing Attributes}
    \begin{dtabularx}{\textwidth}{l >{\lcol}p{10em} X}
        \thead{Effect Type} & \thead{Timing of Change} & \thead{Example} \\
        \hline
        Numerical modifiers   & Immediately & A barbarian enters a rage. His attack bonus increases immediately. \\
        Ability prerequisites & Immediately & A paladin's Strength is drained by a ghost. She loses the benefits of her Power Attack feat immediately. \\
        Ability use limits    & When ability uses are regained & A fighter puts on a magic item that grants additional Willpower. He gains additional daily uses of his combat discipline ability after resting for the night. \\
        Hit points            & On level up\fn{1} & A druid casts \spell{totemic power} to increase his Constitution. His hit points do not change. \\
        Skill points          & On level up\fn{1} & A wizard reads an ancient magical tome that increases her Intelligence. Her skill points increase when she gains a level. \\
    \end{dtabularx}
    1. Hit points and skill points are never affected by spells, worn magic items, or other temporary effects.
\end{dtable*}

\section{Combat Overview}\label{Combat Overview}

Combat takes place in a series of ``rounds'', which represent about six seconds of action. In combat, creatures attack each other (see \pcref{Attacks}) and defend themselves (see \pcref{Defenses}), while moving around the battlefield (see \pcref{Movement and Positioning}). When your defenses fail, you can get hurt (see \pcref{Injury, Death, and Healing}). In unusual situations, you might become more or less likely to succeed at your actions (see \pcref{Circumstances, Bonuses, and Penalties}).

\subsection{Combat Phases}

Each round of a combat is divided into two phases: a movement phase and an action phase. During each phase, all characters declare their actions simultaneously, and then those actions are resolved simultaneously.

\subsubsection{The Movement Phase}\label{The Movement Phase}

Movement takes place first in the round. During the movement phase, all creatures can take move actions (see \pcref{Movement and Positioning}). You can take any number of move actions during the movement phase, as long as all of those actions can be performed simultaneously. For example, you can walk your speed and draw your sword in a single movement phase. However, you cannot draw a sword and equip a shield in the same phase. Equipping a shield takes two hands, leaving you with no free hand to draw your sword.

\parhead{Move Action} A move action is usually used to move from one place to another. You can move a distance up to your base land speed with a move action. It can also be used for other activities that require some effort, such as drawing a weapon, opening a door, or standing up from a prone position. You can take one move action each turn.

Once all creatures are done moving, the action phase begins.

\subsubsection{The Action Phase}\label{The Action Phase}

During the action phase, each creature can take a single standard action.

\parhead{Standard Action} A standard action is the most common type of action. You can use a standard action to attack with a weapon, cast a spell, drink a potion, and do most other things that take concentration and effort. You can take one standard action each turn.

Once all creatures have declared their actions, actions are resolved.

\subsection{Resolving Actions}\label{Resolving Actions}

The actions of all creatures are simultaneously resolved in the following order.

\begin{enumerate*}
    \item Determine affected targets.
    \item Check action success. Example: Making attack rolls.
    \item Determine action results. Example: Making damage rolls.
    \item Apply action results. Examples: Reducing hit points, moving character locations, and applying penalties.
\end{enumerate*}

In the vast majority of cases, there is no need to go through this order explicitly. Combats will run much faster if attack and damage rolls are generally made and announced at the same time as the actions are chosen, even before all characters have explicitly stated their actions. The order of resolution matters when taking actions that can significantly change the situation, such as grappling an enemy or casting complex spells.

\subsubsection{Conflicting Actions}\label{Conflicting Actions}

In some situations, actions that should take place at the same time directly conflict with each other. This most commonly happens with movement. In this case, each involved character rolls initiative. The creature with the highest initiative result succeeds. All other creatures come as close as possible to completing their intended action.

Your initiative check is calculated as follows:

\begin{figure}[h]
    \centering Dexterity or Perception \meh{\add enhancement bonus} \add other bonuses and penalties
\end{figure}

For example, if two creatures were racing to reach a door, they would both roll initiative. The winner would reach the door and stop in their intended square, and the loser would stop adjacent to their intended square. 

\subsection{Special Actions}

\parhead{Swift and Immediate Actions}\label{Swift and Immediate Actions} Each round, you can take a single swift or immediate action. Swift and immediate actions can be taken in either the movement or action phase. Swift actions must be declared along with any other actions you intend to take during that phase. Immediate actions do not need to be declared ahead of time. Instead, abilities that can be used as immediate actions specify triggering conditions that allow the action to be taken.

Swift and immediate actions are resolved immediately, before any other actions resolve. If multiple swift or immediate actions are taken simultaneously, they are resolved using the normal rules for resolving simultaneous actions.

\parhead{Full-Round Actions} A full-round action requires your character's full attention. Most full-round actions involve a combination of movement and concentrated effort, such as charging to strike a distant foe or running at full speed. Unless otherwise specified, you perform any movement required for the action during the movement phase, and the rest of the action during the action phase.

\parhead{Delaying}\label{Delaying}
In each phase, you can delay your action rather than acting immediately. If you delay, you do nothing until after the actions of all other creatures have been resolved. At that point, you declare your action and resolve it. If multiple creatures delay, all their actions are declared resolved simultaneously, after the actions of all the creatures that did not delay. You cannot delay more than once in a single phase.

\subsection{Attacks}\label{Attacks}
An \term{attack} is anything that affects another creatures in a potentially harmful way. There are two kinds of attacks: physical attacks, which are made with weapons or fists, and special attacks, which are made with magic or supernatural power. All physical attacks, and most special attacks, require making an attack roll against a defense. If the result of the attack roll meets or exceeds the defense, the attack succeeds.

\subsubsection{Full Attack}
As a standard action, you can try to strike a foe with a weapon you are wielding. To do so, make an attack roll with a weapon you are wielding, adding your attack modifier to the roll. If your result is at least equal to your foe's Armor defense, your attack hits, and your foe takes damage.

\subsubsection{Attack Modifier}
Your attack modifier is equal to the following:

\begin{figure}[h]
    \centering Base attack bonus or attack attribute \add proficiency bonus \meh{\add enhancement bonus} \add size modifier \add other bonuses and penalties
\end{figure}

\parhead{Attack Attribute} You can use Strength to attack with melee and thrown weapons, Dexterity to attack with melee and thrown weapons that are light, and Perception to attack with projectile weapons. 
\parhead{Proficiency Bonus} You gain a \plus4 attack bonus when using a weapon you are proficient with.
\parhead{Size Modifier} Your size modifier is described in \tref{Size in Combat}.

\subsubsection{Damage}
If your attack succeeds, you deal damage equal to your weapon's damage die \add half your base attack bonus or half your Strength.

\parhead{Dealing Nonlethal Damage} You can attempt to strike nonlethally with any weapon. If you hit, you deal half damage as nonlethal damage. See \pcref{Nonlethal Damage}.

\subsubsection{Reach}\label{Reach}
Normally, you can attack anyone within five feet of you. The range at which you can attack is called your ``reach'', and the area that you can attack into is called your ``threatened area''. Reach for larger and smaller creatures is determined by size, as shown on \trefnp{Size in Combat}.

\subsection{Defenses}\label{Defenses}
Usually, when you are attacked, the attacker has to make an attack roll against a specific defense. If the attack roll is at least as high as that defense, the attack succeeds. There are three physical defenses and two special defenses.
\begin{itemize}
    \item Armor defense (AD): Your Armor defense protects you from normal physical attacks, such as attempts to stab you with a sword. It is the most commonly used defense. Armor defense is a physical defense.
    \item Maneuver defense: Your Maneuver defense protects you from unusual physical attacks, such as attempts to trip or disarm you. Maneuver defense is a physical defense.
    \item Reflex defense: Your Reflex protects you from attacks you have to avoid, such as explosions or falling rocks. Reflex defense is a physical defense.
    \item Fortitude defense: Your Fortitude defense protects you from attacks you have to physically endure or resist, such as poisons and deadly spells. Fortitude defense is a special defense. 
    \item Mental defense: Your Mental defense protects you from attacks you have to mentally endure or resist, such as terrifying creatures and magical manipulation. Mental defense is a special defense.
\end{itemize}

\subsubsection{Defense Values}

Each of your defenses is calculated in the following way:

\begin{figure}[h]
    \centering 10 \add Base defense bonus or defense attribute(s) \meh{\add enhancement bonus} \add size modifier \add other bonuses and penalties
\end{figure}

The attributes and relevant bonuses which apply to each defense are described on \trefnp{Defense Calculations}.

\begin{dtable!*}
    \lcaption{Defense Calculations}
    \begin{dtabularx}{\textwidth}{l l l l l >{\lcol}X}
        \thead{Defense Name} & \thead{Defense Bonus} & \thead{Attributes} & \thead{Body Armor Modifier} & \thead{Shield Modifier} & \thead{Size Modifier} \\
\hline
        Armor defense     & Base attack bonus    & Dex or Con        & Yes & Yes & Yes     \\
        Maneuver defense  & Base attack bonus    & Str or Dex        & No  & Yes & Special \\
        Fortitude defense & Base Fortitude bonus & Con \add half Str & No  & No  & No      \\
        Reflex defense    & Base Reflex bonus    & Dex \add half Per & No  & Yes & Yes     \\
        Mental defense    & Base Mental bonus    & Wil \add half Int & No  & No  & No      \\
    \end{dtabularx}
\end{dtable!*}

\meh{\parhead{Base Attack Bonus} Your experience and aptitude in combat affects your ability to defend yourself; experienced warriors know how to recognize and avoid or parry blows that would easily fell novices. As a result, you add half your base attack bonus to your Armor defense. Your fighting experience is even more important when defending against combat maneuvers, so you add your full base attack bonus to your Maneuver defense.}
\meh{\parhead{Enhancement Bonus} You can have enhancement bonuses to all physical defenses, or to a specific physical defense. As normal, these enhancement bonuses do not stack with each other; only the highest applicable bonus is used.}
\parhead{Natural Armor} Creatures with unusually tough skin or thick hide, including most monsters, gain bonuses to their Armor defense. These bonuses stack with any armor such creatures might wear.
\parhead{Size Modifiers} Your size modifier and special size modifier are described on \tref{Size in Combat}.

\subsection{Movement and Positioning}\label{Movement and Positioning}

\subsubsection{Taking up Space}
A typical human takes up a 5-ft. by 5-ft. space in combat. For convenience, this is often called a ``square''. Differently sized creatures can take up more or less space, as indicated on \tref{Size in Combat}. Normally, other creatures can't be in any squares you occupy.

Sometimes, movement and distance are represented in squares. A 30-ft. movement is the same thing as moving six squares.

\subsubsection{Moving}

When you move, you can travel a number of feet up to your speed in any direction. For simplicity, all movement is measured in five-foot increments. While it is possible to be more precise than that, it's generally not worth the complexity.

\subsubsection{Measuring Movement}

\parhead{Diagonals} When measuring distance, the first diagonal counts as five feet of movement, and the second counds as ten feet of movement. The third costs five feet, the fourth costs ten feet, and so on.

You can't move diagonally past a corner. You can move diagonally past a creature, even an opponent. You can also move diagonally past other impassable obstacles, such as pits.
\parhead{Closest Creature} When it's important to determine the closest square or creature to a location, if multiple squares or creatures are equally close, pick one randomly.


\subsubsection{Combat Engagement}

At the start of each phase, if you are threatened by any enemies, you are engaged in combat until the end of the phase. While engaged in combat, you move at half speed.
\subsection{Injury, Death, and Healing}\label{Injury, Death, and Healing}
Your hit points measure how hard you are to kill. No matter how many hit points you lose, your character isn't significantly hindered until your hit points drop to 0.

\subsubsection{Hit Points}\label{Hit Points}
Your hit points represent how much punishment you can take. When you run out of hit points, your actions are limited and you might die.

Your hit points are equal to your \term{hit value} \mtimes your level. Your hit value is calculated as follows:

\begin{figure}[h]
    \centering Half Fortitude defense or half Mental defense \add other bonuses or penalties
\end{figure}

\parhead{Temporary Modifiers} Temporary effects which alter your Fortitude or Mental defenses, including changes to your Constitution or Willpower, do not alter your maximum or current hit points.
Your maximum number of hit points is determined when you gain a level, and generally does not change between levels.
Some effects specifically modify your maximum hit points, such as the \spell{curse of blood and bone} spell.

\subsubsection{Losing Hit Points}
When you take lethal damage, you subtract that damage from your hit points.
\parhead{What Hit Points Represent} Hit points represent a combination of durability, luck, divine providence, and sheer determination, depending on the nature of your character. When you take 10 damage from an orc with a greataxe, the axe did not literally carve into your skin without affecting your ability to fight. Instead, you avoided the worst of the blow, but it bruised you through your armor, the effort to dodge the blow fatigued your character, or it barely nicked you through sheer luck -- and everyone's luck runs out eventually.

\subsubsection{Critical Damage}\label{Critical Damage}
When you take damage while you have no hit points remaining, that damage represents serious physical injury to your body. This is called critical damage. You suffer a penalty to attacks, checks, and defenses equal to the amount of critical damage you have.

While you have critical damage, magical healing which would normally restore hit points cannot restore your hit points, though it can stabilize you, preventing you from dying. In addition, if you take damage that would reduce your hit points to 0 while you have any critical damage, any excess damage from the attack is dealt directly as critical damage.

\subsubsection{Overkill Damage}
Normally, when you take damage that would reduce your hit points to 0, any excess damage from the attack is wasted. However, some attacks deal such massive damage that you begin dying immediately, rather than just becoming staggered. If the damage dealt by an attack exceeds your maximum hit points (not current hit points), any damage past what would reduce your hit points to 0 is dealt as critical damage rather than being wasted.

\subsubsection{Stages of Injury}

\parhead{Healthy} When you are above half hit points, you suffer no significant effects from losing hit points. If you take damage, you can become bloodied (see Bloodied, below).

\parhead{Bloodied} When you drop to half your hit points or below, you are bloodied. While bloodied, you take a \minus5 penalty to Fortitude and Mental defense. If you take additional damage, you can become staggered.

\parhead{Staggered} When you take damage that would reduce your hit points to 0, you become staggered. While staggered, you can only act during the action phase. You are also still bloodied, and continue to take the \minus5 penalty to Fortitude and Mental defense.

If you take additional damage while at 0 hit points, you begin dying (see Dying, below).

\parhead{Dying}\label{Dying} While you are dying, you must make an attack against your own Fortitude defense at the end of every round. No bonuses or penalties apply to the attack roll, but critical damage can penalize your Fortitude defense. If this attack succeeds once, you fall unconscious. If it succeeds three times, you die. If this attack fails three times, you stabilize.

If you receive magical healing of any kind while dying, you become partially stabilized. While partially stabilized, you must make an attack against your Fortitude once per minute, instead of once per round.

An ally can make a Heal check to tend to you while you are dying. The Heal check result can be used in place of your Fortitude defense, although the critical damage you have taken applies as a penalty to the Heal check result as well.

\parhead{Stable}\label{Stable}
If you have taken critical damage but managed to stave off death, you become stable. As long as you have critical damage, magical healing has no effect on your hit points, though some magical effects can heal critical damage. If you became unconscious while dying, you regain consciousness as soon as you have hit points.

\subsubsection{Healing}
After taking damage, you can recover hit points through natural healing or through magical healing. In any case, you can't regain hit points past your full normal hit point total.

\parhead{Natural Healing} With 8 hours of rest, you recover half your hit points. Any significant interruption (such as combat or the like) during your rest prevents you from healing. If you rest for an entire day (16 hours), you recover all your hit points.

\parhead{Magical Healing} Various abilities and spells can restore hit points. However, only certain spells can heal critical damage, as specified in the spell description. Unless a spell says it can cure critical damage, it cannot -- though it can still stabilize dying characters. Magical healing has no effect on the hit points of creatures with critical damage.

\parhead{Healing Ability Damage} Ability damage is temporary, just as hit point damage is. Ability damage returns at the rate of 1 point per 8 hours of rest for each affected attribute score.

\parhead{Healing Critical Damage} Critical damage takes much longer to heal than hit point damage. Resting for 1 week restores an amount of critical damage equal to 1 \add half the character's Constitution (minimum 1). A character can have both hit points and critical damage. As long as a character has critical damage, he is staggered, even if he is at full hit points.

\subsubsection{Nonlethal Damage}\label{Nonlethal Damage}
Some attacks and environmental effects deal nonlethal damage. Nonlethal damage is not subtracted from your hit points. Instead, it is tracked separately. If your nonlethal damage exceeds your hit points, you become staggered, just as if you were at 0 hit points. If you take additional damage while staggered, you fall unconscious. However, you do not begin dying unless your hit points are actually below 0. 

\parhead{Healing Nonlethal Damage}
You heal half your hit points in nonlethal damage with 1 hour of rest. When a spell or a magical ability cures hit point damage, it also removes an equal amount of nonlethal damage.

\subsection{Temporary Hit Points}\label{Temporary Hit Points}
Certain effects give a character temporary hit points which act as a protection against damage. Whenever a character takes damage, if he has temporary hit points, the damage is applied to his temporary hit points first. Any excess damage is then applied to his hit points as normal. Temporary hit points are not ``real'' hit points, and cannot be healed. If a character has temporary hit points from multiple effects, only the highest value is used.

\subsection{Circumstances, Bonuses, and Penalties}

\subsubsection{Overwhelm}\label{Overwhelm}
When a creature is being attacked by multiple foes at once, it is less able to defend itself. A creature is considered overwhelmed if it is being threatened by more than one creature. Multiple creatures occupying the same square count as a single creature when determining overwhelm penalties. If a creature is overwhelmed, it takes a penalty to physical defenses equal to the number of creatures threatening it.

\subsubsection{Range Increments}
When using a ranged weapon, you take a \minus2 penalty per range increment between you and your target. For example, when using a longbow with a range increment of 100 feet against a target 170 feet away, you take a \minus2 penalty to attack rolls.

\subsubsection{Size in Combat}\label{Size in Combat}
Size affects your space and reach in combat. In addition, your physical attacks and defefenses are affected by your size modifier. These effects are shown on \trefnp{Size in Combat}. 

\begin{dtable*}
    \lcaption{Size in Combat}
    \begin{dtabularx}{\textwidth}{l l l l l X}
        \thead{Size} & \thead{Space\fn{1}} & \thead{Reach\fn{1}} & \thead{Size Modifier\fn{2}} & \thead{Special Size Modifier\fn{3}} & \thead{Example Creature} \\
\hline
        Fine              & 1/2 ft.    & 0          & \plus8  & \minus16 & Fly                      \\
        Diminutive        & 1 ft.      & 0          & \plus4  & \minus12 & Toad                     \\
        Tiny              & 2-1/2 ft.  & 0          & \plus2  & \minus8  & Cat                      \\
        Small             & 5 ft.      & 5 ft.      & \plus1  & \minus4  & Halfling                 \\
        Medium            & 5 ft.      & 5 ft.      & \plus0  & \plus0   & Human                    \\
        Large (tall)      & 10 ft.     & 10 ft.     & \minus1 & \plus4   & Ogre                     \\
        Large (long)      & 10 ft.     & 5 ft.      & \minus1 & \plus4   & Horse                    \\
        Huge (tall)       & 15 ft.     & 15 ft.     & \minus2 & \plus8   & Cloud giant              \\
        Huge (long)       & 15 ft.     & 10 ft.     & \minus2 & \plus8   & Bulette                  \\
        Gargantuan (tall) & 20 ft.     & 20 ft.     & \minus4 & \plus12  & 50-ft. animated statue   \\
        Gargantuan (long) & 20 ft.     & 15 ft.     & \minus4 & \plus12  & Kraken                   \\
        Colossal (tall)   & 30\add ft. & 30\add ft. & \minus8 & \plus16  & Colossal animated object \\
        Colossal (long)   & 30\add ft. & 20\add ft. & \minus8 & \plus16  & Great wyrm red dragon    \\
    \end{dtabularx}
    1 Creatures can vary in space and reach. These are simply typical values. \\
    2 Modifies physical attacks and defenses, except for maneuvers \\
    3 Modifies maneuver attack and defense. The opposite modifier applies to Stealth. \\
\end{dtable*}

Unusually large or small creatures also have other special rules apply to them, as described in \pcref{Special Size Rules}.

\subsubsection{Total Defense}\label{Total Defense}
As a standard action, you can focus entirely on defense, granting you a \plus4 bonus to your physical defenses for 1 round.

\subsection{Special Rules}\label{Special Rules}

\subsubsection{Critical Success and Failure}\label{Critical Success and Failure}
A natural 1 (the d20 comes up 1) on an attack roll is treated as rolling a \minus10. A natural 20 (the d20 comes up 20) is treated as rolling a 30. Under normal circumstances, a natural 1 automatically misses, and a natural 20 automatically hits.

\subsubsection{Critical Hits}\label{Critical Hits}
When you roll a natural 20 on an attack roll and hit, you have scored a critical threat. Roll another attack roll at the same attack bonus. If that attack also hits, you deal double damage.

%Keen weapons threaten a critical hit on a 19 or 20, while impact weapons deal triple damage when they score a critical hit.

\subsubsection{Unarmed Combat}\label{Unarmed Combat}
Every creature can attack with its body using an unarmed attack. You are not proficient with your unarmed attack, so you are usually \defenseless while unarmed. In addition, an unarmed attack always deals nonlethal damage. You may use any appropriate part of your body to make an unarmed strike - fists, feet, elbows, and so on. However, you only have one unarmed strike attack. You cannot dual-wield unarmed attacks as if you were fighting with two weapons at once (see \pcref{Two-Weapon Fighting}).

An unarmed attack is a type of natural weapon. Spells and abilities that affect natural weapons can affect your unarmed attack. Gauntlets can also be worn to increase the power of an unarmed attack (see \pcref{Unarmed Weapons}).

If you have the Improved Unarmed Combat feat, you become proficient with your unarmed attack, and can deal lethal damage with it (see Improved Unarmed Combat, \pref{feat:Improved Unarmed Combat}).
\section{Legend Points}\label{Legend Points}

As your character gains levels, she may gain legend points. Legend points allow you to change fate to ensure your character succeeds. Certain abilities can also grant offensive or defensive legend points.

\subsection{Using Legend Points}

Offensive legend points can be used for one of two things:
\begin{itemize}
    \item Reroll any attack or check your character made. You may choose to reroll after knowing whether the roll succeeded or failed.
    \item Treat a successful physical attack your character made as if it were a critical threat.
\end{itemize}

Defensive legend points can be used for one of two things:
\begin{itemize}
    \item Reroll any attack or check made against your character. You may choose to reroll after knowing whether the roll succeeded or failed.
    \item Treat a critical threat against your character as if it were an ordinary success.
\end{itemize}

Legend points which are not specifically offensive or defensive can be used for any of these four abilities. Using a legend point is not an action, and can be done at any time.

\subsection{Gaining Legend Points}

You gain legend points as you gain levels, as described in \pcref{Character Advancement}.
Magic weapons and armor can grant additional legend points, as well as certain spells.

\subsection{Restoring Legend Points}

At dawn each day, you regain all legend points you spent the previous day. This does not require rest or any specific action.

It is possible to regain legend points during the day by performing extraordinary actions worthy of legends.
