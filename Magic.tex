\chapter{Magic}\label{Magic}
Magic comes in many forms, but it is most commonly wielded with spells. A spell is a one-time magical effect. There are three types of spells: arcane (cast by sorcerers and wizards), divine (cast by clerics and experienced paladins), and nature (cast by druids). Cutting across these categories are the eight schools of magic. Each of the eight schools represents a different type of mastery over the world, based on fundamentally distinct principles. All spellcasters select their spells from a limited list of spells known, casting them spontaneously as the need arises.

\section{Casting Spells}\label{Casting Spells}
Whether a spell is arcane, divine, or natural, casting a spell works the same way.

\subsection{Choosing a Spell}
First, you must choose a spell that you know. If a spell has multiple versions, you choose which version to use when you cast it.

To cast a spell, you must be able to speak (if the spell has a verbal component), gesture with a free hand (if it has a somatic component), and manipulate the material components or focus (if any). Additionally, you must concentrate to cast a spell.

\subsection{Concentration}
To cast a spell, you must concentrate. While casting a spell, you can't focus on anything else: you become flat-footed, and you provoke attacks of opportunity, because you can't defend yourself. You can't take any other actions, even immediate actions.

If something interrupts your concentration while you're casting, you must make a Concentration check or lose the spell. The more distracting the interruption and the higher the level of the spell you are trying to cast, the higher the DC is. If you fail the check, the spell has no effect, but you still spend the spell slot.

\subsubsection{Making Concentration Checks}

To make a concentration check, roll d20 \add your character level \add your Constitution. Any condition which grants a bonus or penalty to checks affects Concentration checks. In addition, you apply your overwhelm penalty (if any) to your Concentration checks.

\begin{dtable}
\lcaption{Concentration Examples}
\begin{tabularx}{\columnwidth}{>{\lcol}p{6em} >{\lcol}X}
\thead{Concentration DC\footnotetemp{1}} & \thead{Distraction} \\
10 & Casting defensively \\
10 \add damage dealt & Damaged during the action.\footnotetemp{2} \\
10 \add half of continuous & Taking continuous damage during the
damage last dealt action.\footnotetemp{3} \\
5 & Vigorous motion (on a moving mount, taking a bouncy wagon ride, in a small boat
in rough water, belowdecks in a storm-tossed ship). \\
10 & Violent motion (on a galloping horse, taking a very rough wagon ride, in a small boat in
rapids, on the deck of a storm-tossed ship). \\
20 & Extraordinarily violent motion (earthquake). \\
10 & Entangled. \\
15 & Grappling.\fn{4} \\
5 & Weather is a high wind carrying blinding
rain or sleet. \\
10 & Weather is wind-driven hail, dust, or debris. \\
\end{tabularx}
1 If you are trying to cast, concentrate on, or direct a spell when the distraction occurs, add double the level of the spell to the indicated DC. \\
2 If you take damage from multiple sources at the same time, add up the damage taken and make a single Concentration check, rather than one check for each source. \\
3 Such as from \spell{acid arrow}. \\
4 You can cast only spells without somatic components for which you have any required material component in hand. \\
\end{dtable}

\subsection{Defending Yourself While Concentrating}
Concentrating does not prevent you from defending yourself normally. However, some actions that require concentration prevent you from defending yourself, such as casting spells.
You can make a Concentration check while casting a spell to multi-task, allowing you to defend yourself normally while casting the spell. The DC is equal to 10 \add double the level of the spell you're casting. Success means you can defend yourself normally. Failure means you are unable to maintain concentration on both things at once, and you must choose whether to lose the spell or lose the ability to defend yourself.

\subsubsection{Taking Damage}
If you take damage while casting a spell, you must make a Concentration check to maintain the spell. The DC is equal to 10 \add damage taken while casting the spell \add double the level of the spell you're casting. Success means you can continue casting the spell. Failure means you lose the spell without effect. If you take damage multiple times while casting a spell, roll once per source of damage, but add the total damage taken together to determine the DC of each successive check.

If you are taking continuous damage each round, half the damage is considered to take place while you are casting a spell. You must make a Concentration check (DC 10 \add 1/2 the damage that the continuous source last dealt \add double the level of the spell you're casting). If the last damage dealt was the last damage that the effect could deal, then the damage is over, and it does not distract you.

Repeated damage does not count as continuous damage.

\subsubsection{Grappling or Pinned}

The only spells you can cast while grappling or pinned are those without somatic components and whose material components (if any) you have in hand. Even so, you must make a Concentration check (DC 15 \add double the level of the spell you're casting) or lose the spell.

\subsubsection{Entangled}
If you want to cast a spell while entangled in a net or by a tanglefoot bag or while you're affected by a spell with similar effects, you must make a DC 10 \add double the level of the spell you're casting Concentration check to cast the spell. You lose the spell if you fail.

\subsubsection{Environmental Distractions}

\parhead{Vigorous Motion} If you are riding on a moving mount, taking a bouncy ride in a wagon, on a small boat in rough water, below-decks in a storm-tossed ship, or simply being jostled in a similar fashion, you must make a Concentration check (DC 5 \add double the level of the spell you're casting) or lose the spell.

\parhead{Violent Motion} If you are on a galloping horse, taking a very rough ride in a wagon, on a small boat in rapids or in a storm, on deck in a storm-tossed ship, or being tossed roughly about in a similar fashion, you must make a Concentration check (DC 10 \add double the level of the spell you're casting) or lose the spell.

\parhead{Violent Weather} You must make a Concentration check if you try to cast a spell in violent weather. If you are in a high wind carrying blinding rain or sleet, the DC is 5 \add double the level of the spell you're casting. If you are in wind-driven hail, dust, or debris, the DC is 10 \add double the level of the spell you're casting. In either case, you lose the spell if you fail the Concentration check. If the weather is caused by a spell, use the rules in the Spell subsection above.

\subsubsection{Extended Concentration}
Concentrating on a spell is mentally tiring. You can concentrate on a spell for up to 5 minutes without penalty. After 5 minutes, and every minute thereafter, you must make a Concentration check (DC 15 \add double the level of the spell). If you fail, you lose your concentration on the spell and become fatigued. The DC of the check increases by 2 for every additional minute.

\subsection{Counterspells}
It is possible to cast any spell as a counterspell. By doing so, you are using the spell's energy to disrupt the casting of the same spell by another character. Counterspelling works even if one spell is divine and the other arcane.

\parhead{How Counterspells Work} To use a counterspell, you must select an opponent as the target of the counterspell. You do this by choosing the ready action. In doing so, you elect to wait to complete your action until your opponent tries to cast a spell. (You may still move your speed, since ready is a standard action.)

If the target of your counterspell tries to cast a spell, make a Spellcraft check (DC 15 \add the spell's level). This check is a free action. If the check succeeds, you correctly identify the opponent's spell and can attempt to counter it. If the check fails, you can't do either of these things.

To complete the action, you must then cast the correct spell. As a general rule, a spell can only counter itself. If you are able to cast the same spell and you have it prepared (if you prepare spells), you cast it, altering it slightly to create a counterspell effect. If the target is within range, both spells automatically negate each other with no other results.

\parhead{Counterspelling Metamagic Spels} Metamagic feats are not taken into account when determining whether a spell can be countered.

\parhead{Specific Exceptions} Some spells specifically counter each other, especially when they have diametrically opposed effects.

\parhead{\spell{Dispel Magic} as a Counterspell} You can use \spell{dispel magic} to counterspell another spellcaster, and you don't need to identify the spell he or she is casting. However, \spell{dispel magic} doesn't always work as a counterspell (see the spell description).

\subsection{Caster Level}
A spell's power often depends on its caster level, which depends on the number of levels you have in your spellcasting class. Some magic items and feats also increase your caster level. Effects that increase caster level never increase a character's spells per day or spells known. Only a character's class levels affect those values.

You can cast a spell at a lower caster level than normal, but the caster level you choose must be high enough for you to cast the spell in question, and all level-dependent features must be based on the same caster level.

\subsection{Spell Failure}
If you ever try to cast a spell in conditions where the characteristics of the spell cannot be made to conform, the casting fails and the spell is wasted.

Spells also fail if your concentration is broken and might fail if you're wearing armor while casting a spell with somatic components.

\subsection{The Spell's Effect}
Once you know which creatures (or objects or areas) are affected, you can apply whatever results a spell entails.

\subsection{Special Spell Effects}
Many special spell effects are handled according to the school or subschool of the
spells in question. Certain other special spell features are found across spell schools.

\parhead{Attacks} Some spell descriptions refer to attacking. All offensive combat actions and special abilities, even those that don't damage opponents, are considered attacks. All spells that affect a creature or area and are not labeled ``harmless'' are considered attacks. However, if all targets of a targeted spell are willing, the spell is not considered an attack. Spells that summon monsters or other allies are not attacks because the spells themselves don't harm anyone.

\label{Bringing Back the Dead}
\parhead{Bringing Back the Dead} Several spells have the power to restore slain characters to life.

When a living creature dies, its soul departs its body, leaves the Material Plane, travels through the Astral Plane, and goes to abide on the plane where the creature's deity resides. If the creature did not worship a deity, its soul departs to the plane corresponding to its alignment. Bringing someone back from the dead means retrieving his or her soul and returning it to his or her body.

\subparhead{Negative Level} Any creature brought back to life gains a negative level. If the character was 1st level at the time of death, he or she takes 2 points of Constitution drain instead. This negative level or Constitution drain cannot be repaired by anything less than a \spell{wish} or \spell{miracle} spell, but it goes away naturally after a year has passed or after the character gains two new levels.

\subparhead{Preventing Revivification} Enemies can take steps to make it more difficult for a character to be returned from the dead. Keeping the body prevents others from using  \spell{raise dead} or \spell{resurrection} to restore the slain character to life. Casting \spell{trap the soul} or \spell{soul bind} prevents any sort of revivification unless the soul is first released.

\subparhead{Revivification against One's Will} A soul cannot be returned to life if it does not wish to be. A soul knows the name, alignment, and patron deity (if any) of the character attempting to revive it and may refuse to return on that basis.

\section{Combining Effects}
Spells or magical effects usually work as described, no matter how many other spells or magical effects happen to be operating in the same area or on the same recipient. Except in special cases, a spell does not affect the way another spell operates. Whenever a spell has a specific effect on other spells, the spell description explains that effect.

However, spells, feats, class features, and other effects that have very similar effects may not both help the subject. A character can only be increased so far beyond his or her normal limits; even layered with powerful magical effects, a commoner is no serious threat to a giant. The limitations on these effects are provided by the stacking rules described below.

\subsection{Stacking Effects}
Spells that provide bonuses or penalties usually do not stack with themselves. More generally, two enhancement bonuses don't stack even if they come from different spells; see \pcref{Stacking Rules}, for more details.

\parhead{Same Effect More than Once in Different Strengths} In cases when two or more identical spells are operating in the same area or on the same target, but at different strengths, only the best one applies. This is called overlapping.

\parhead{Same Effect with Differing Results} The same spell can sometimes produce varying effects if applied to the same recipient more than once. Usually, the last spell in the series trumps the others. None of the previous spells are actually removed or dispelled, but their effects become irrelevant while the final spell in the series lasts.

\parhead{One Effect Makes Another Irrelevant} Sometimes, one spell can render a later spell irrelevant. Both spells are still active, but one has rendered the other useless in some fashion.

\parhead{Multiple Mental Control Effects} Sometimes magical effects that establish mental control render each other irrelevant, such as a spell that removes the subject's ability to act. Mental controls that don't remove the recipient's ability to act usually do not interfere with each other. If a creature is under the mental control of two or more creatures, it tends to obey each to the best of its ability, and to the extent of the control each effect allows. If the controlled creature receives conflicting orders simultaneously, the competing controllers must make opposed Charisma checks to determine which one the creature obeys.

\parhead{Spells with Opposite Effects} Spells with opposite effects apply normally, with all bonuses, penalties, or changes accruing in the order that they apply. Some spells negate or counter each other. This is a special effect that is noted in a spell's description. If a spell negates another spell, it immediately dispels the other spell without a caster level check if cast on the same targets or in the same area, but does not itself take effect on those targets or in that area.

\parhead{Instantaneous Effects} Two or more spells with instantaneous durations work cumulatively when they affect the same target.

\section{Spell Descriptions}
The description of each spell is presented in a standard format. Each category of information is explained and defined below.

\subsection{Name}
The first line of every spell description gives the name by which the spell is generally known.

\subsection{Description}
Beneath the spell name is a brief description of the spell's effect. This description has no mechanical significance, and simply describes how the spell usually appears or is used.

\subsection{School/Schools (Subschool)}
The next line describes the schools and subschools of magic that the spell belongs to. Almost every spell belongs to at least one of eight schools of magic. A school of magic is a group of related spells that work in similar ways. They are described below.

Some spells belong to more than one school of magic. Treat these spells for all purposes as if they were a member of both schools simultaneously. If you are prohibited from casting spells from a certain school, you cannot cast a spell which belongs to that school, even if it also belongs to another school. Likewise, any benefits which apply to casting spells from a specific school apply normally. If you have abilities which apply when casting spells from both schools that make up a spell, the abilities do not stack.

A small number of spells (\spell{limited wish}, \spell{permanency}, \spell{prestidigitation}, and \spell{wish}) are universal, belonging to no school.

\subsubsection{Abjuration}
Abjuration spells manipulate the raw essence of magic to protect allies or ward off foes. There are four subschools of abjuration spells.
\parhead{Interdiction} An interdiction spell hedges out creatures or forces of an opposing alignment or type. \spellindirect{protection from evil}{Protection from evil} is an interdiction spell.
\parhead{Negating} A negating spell negates magical effects. \spell{Dispel magic} is a negating spell.
\parhead{Shielding} A shielding spell protects creatures or objects from harm. \spell{Shield} is a shielding spell.
\parhead{Warding} A warding spell protects an area from intrusion. If one warding spell is active within 10 feet of another for 24 hours or more, the magical fields interfere with each other and create barely visible energy fluctuations. The DC to find such spells with the Perception skill drops by 4. The DC drops by an additional 2 for each additional warding spell beyond the second. \spellindirect{glyph of warding}{Glyph of warding} is a warding spell.

\subsubsection{Conjuration}
Conjuration spells transport and create objects and creatures to aid you. A creature or object brought into being or transported to your location by a conjuration spell cannot appear inside another creature or object, nor can it appear floating in an empty space. It must arrive in an open location on a surface capable of supporting it. The creature or object must appear within the spell's range, but it does not have to remain within the range. There are three subschools of conjuration spells.
\parhead{Creation} A creation spell manipulates matter to create an object or creature in the place the spellcaster designates (subject to the limits noted above). If the spell has a duration other than instantaneous, magic holds the creation together, and when the spell ends, the conjured creature or object vanishes without a trace. If the spell has an instantaneous duration, the created object or creature is merely assembled through magic. It lasts indefinitely and does not depend on magic for its existence. \spell{Acid arrow} is a creation spell.
\parhead{Summoning} A summoning spell instantly brings a manifestation of a creature or object to a place you designate. When the spell ends or is dispelled, the manifestation disappears. A summoned creature also disappears if it is killed or if its hit points drop to 0 or lower. Because summoning spells do not physically transport the actual creature or object, even if the manifestation is injured or destroyed, the original is unharmed. However, it takes 24 hours for the manifestation to reform, during which time it can't be summoned again. Most summoning spells, including the \spell{summon monster} and \spell{summon nature's ally} spells, will automatically summon a different creature of the same type should this occur.
\par When the spell that summoned a creature ends and the creature disappears, all the spells it has cast expire. A summoned creature cannot use any innate summoning abilities it may have.
\par \spell{Summon monster I} is a summoning spell.
\parhead{Translocation} A translocation spell transports one or more creatures or objects a great distance. The most powerful of these spells can cross planar boundaries. Unlike summoning spells, the transportation is (unless otherwise noted) one-way and not dispellable. Many translocation effects involve teleportation (see Descriptors, below). \spell{Dimension door} is a translocation spell.

\subsubsection{Divination}
Divination spells enable you to predict the future, gain or share knowledge, find hidden things, and foil deceptive spells. There are four subschools of divination spells.

\parhead{Awareness} A awareness spell reveals objects, creatures, or effects within an area. Some awareness spells have cone-shaped areas. These move with you and extend in the direction you look. The cone defines the area that you can examine each round. If you study the same area for multiple rounds, you can often gain additional information, as noted in the descriptive text for the spell. \spellindirect{detect evil}{Detect evil} is an awareness spell.
\parhead{Communication} A communication spell magically enhances communication between creatures, often by transcending linguistic barriers or distance. \spellindirect{comprehend languages}{Comprehend languages} is a communication spell.
\parhead{Knowledge} A knowledge spell grants the recipient information. Most knowledge spells give knowledge about the present, but some can reveal information about the future as well. \spell{Augury} is a knowledge spell.
\parhead{Scrying} A scrying spell creates an invisible magical sensor that sends you information. Unless noted otherwise, the sensor has the same powers of sensory acuity that you possess. This level of acuity includes any spells or effects that target you, but not spells or effects that emanate from you. However, the sensor is treated as a separate, independent sensory organ of yours, and thus it functions normally even if you have been blinded, deafened, or otherwise suffered sensory impairment.
\par Any creature trained in Spellcraft can notice the sensor by making a DC 20 Spellcraft check. The sensor can be dispelled as if it were an active spell. Lead sheeting or magical protection blocks a scrying spell, and you sense that the spell is so blocked.
\par \spell{Scrying} is a scrying spell.

\subsubsection{Enchantment}
Enchantment spells affect the minds of others, influencing or controlling their behavior or mental capabilities. Almost all enchantment spells are mind-affecting spells. There are four subschools of enchantment spells.
\parhead{Beguilement} A beguilement spell influences the subject's opinions. Beguilement spells are the most subtle form of mental control, and a creature affected by such a spell usually does not realize that it is being manipulated until after the spell wears off -- if it does at all. \spellindirect{charm person}{Charm person} is a beguilement spell.
\parhead{Compulsion} A compulsion spell compels the subject to act in a particular way. Especially powerful compulsions can give you complete control over the subject. \spell{Sleep} is a compulsion spell.
\parhead{Emotion} An emotion spell influences the subject's emotions. \spell{Attraction} is an emotion spell.
\parhead{Inhibition} An inhibition spell prevents the subject's mind from working normally, typically preventing the target from acting. \spellindirect{hold person}{Hold person} is an inhibition spell.

\subsubsection{Evocation}
Evocation spells create and manipulate energy and forces or tap into divine or other powers to produce a desired end. In effect, they create energy or effects, but not physical objects, out of nothing. Many of these spells produce spectacular effects, and evocation spells can deal large amounts of damage. There are three subschools of evocation spells.
\parhead{Channeling} A channeling spell channels divine or other power. \spellindirect{holy smite}{Holy smite} is a channeling spell.
\parhead{Control} A control spell manipulates forces and moves inanimate objectss. Powerful control spells can manipulate forces on a large scale, even altering weather patterns. \spellindirect{gust of wind}{Gust of wind} is a control spell.
\parhead{Energy} An energy spell creates or manipulates energy, such as fire or electricity. \spell{Fireball} is an energy spell.

\subsubsection{Illusion}
Illusion spells deceive the senses of others. They conceal things that exist or cause people to perceive things that do not exist. There are three subschools of illusion spells.

\parhead{Figment} A figment spell creates a false sensation. Those who perceive the figment perceive the same thing, not their own slightly different versions of the figment. (It is not a personalized mental impression.) Figments cannot make something seem to be something else. A figment that includes audible effects cannot duplicate intelligible speech unless the spell description specifically says it can. If intelligible speech is possible, it must be in a language you can speak. If you try to duplicate a language you cannot speak, the image produces gibberish, unless you prescribe exactly which sounds to make. Likewise, you cannot make a visual copy of something unless you know what it looks like.
\par A figment's AC is equal to 10 \add its size modifier.
\par \spell{Silent image} is a figment spell.
\parhead{Glamer} A glamer spell changes a subject's sensory qualities, making it look, feel, taste, smell, or sound like something else, or even seem to disappear. \spell{Invisibility} is a glamer spell.
\parhead{Phantasm} A phantasm spell manipulates the subject's senses to create images or sensations that are not real. It creates personalized sensations, and no one else can observe the effect. \spellindirect{phantasmal killer}{Phantasmal killer} is a phantasm spell.
\parhead{Shadow} A shadow spell creates something that is partially real from extradimensional energy. Such illusions can have real effects. Damage dealt by a shadow illusion is real.

\parhead{Unreal Effects} Some figments and glamers are unreal (see Descriptors, below), which means that they can be disbelieved.

\subsubsection{Necromancy}
Necromancy spells manipulate the power of life and death, as well as souls. Spells involving positive and negative energy belong to this school. There are three subschools of necromancy spells.

\parhead{Flesh} A flesh spell affects the home of a creature's life energy: its body. Many flesh spells inflict or remove physical disabilities. \spell{Enfeeblement} is a flesh spell.
\parhead{Life} A life spell manipulates a creature's life force directly. \spellindirect{crush life}{Crush life} is a life spell.
\parhead{Soul} A soul spell manipulates the subject's soul, either restoring it to its proper place or fragmenting it for terrible purposes. \spellindirect{raise dead}{Raise dead} is a soul spell.
\parhead{Vitalism} A vitalism spell channels positive or negative energy. This can be used to enhance or destroy a subject's life energy, or to manipulate creatures powered by negative energy. \spellindirect{cure light wounds}{Cure light wounds} is a vitalism spell.

\subsubsection{Transmutation}
Transmutation spells change the properties of creatures and objects. There are three subschools of transmutation spells.

\parhead{Animation} An animation spell grants temporary ``life'' to an affected object. \spellindirect{animate objects}{Animate objects} is an animation spell.
\parhead{Alteration} An alteration spell changes the physical shape or state of anything with a material form. \spellindirect{shape stone}{Shape stone} is an alteration spell.
\parhead{Augment} An augment spell enhances the existing physical or mental abilities of an object or creature. \spellindirect{totemic power}{Totemic power} is an augment spell.
\parhead{Imbuement} An imbuement spell infuses an object or creature with magic, granting it new abilities. \spell{Fly} is an imbuement spell.
\parhead{Polymorph} A polymorph spell changes a creature's body into a new form. \spellindirect{reduce person}{Reduce person} is a polymorph spell.
\parhead{Temporal} A temporal spell manipulates time itself, speeding or slowing its passage for the subject. \spell{Haste} is a temporal spell.

\subsection{[Descriptor]}
Appearing on the same line as the school and subschool, when applicable, is a descriptor that further categorizes the spell in some way. Some spells have more than one descriptor.

The descriptors are acid, air, charm, chaotic, cold, curse, darkness, death, detection, disease, domination, earth, electricity, evil, fear, fire, fog, force, good, language-dependent, lawful, light, mind-affecting, morale, negative, planar, poison, positive, sight-dependent, size-affecting, sound-dependent, sonic, teleportation, trap, unreal, wall, water.

Many of these descriptors have no game effect by themselves, but they govern how the spell interacts with other spells, with special abilities, with unusual creatures, with alignment, and so on.

\begin{itemize*}
\item Air spells do not function in environments without air.
\item Barrier spells cannot be used offensively. If you force the barrier against a force or creature it prohibits, you feel a discernible pressure against the barrier. If you continue to apply pressure, the spell ends.
\item Curse spells cannot be dispelled by \spell{dispel magic} or similar effects. However, they can be removed with a \spell{break enchantment}, \spell{limited wish},\spell{miracle}, \spell{remove curse}, or \spell{wish} spell.
\item A detection spell can penetrate barriers, but is always blocked by special materials of some kind. Unless otherwise specified in the spell description, the spell is blocked by 1 foot of stone, 1 inch of common metal, a thin sheet of lead, or 3 feet of wood or dirt.
\item Fire spells do not function underwater. Unless otherwise noted, a fire spell provides light equivalent to a torch.
\item Fog spells do not function underwater and can be dispersed by wind or fire. Unless the spell specifies otherwise, a moderate wind (11\add mph) disperses the fog in 5 rounds, and a strong wind (21\add mph) disperses the fog in 1 round. A fire spell or other powerful fire effect burns away the fog in the area into which it dealt damage.
\item Language-dependent spells use intelligible language as a medium for communication. If the target cannot understand or cannot hear what the caster of a language-dependant spell says, the spell fails.
\item Mind-affecting spells work only against creatures with an Intelligence score of \minus8 or higher.
\item Sight-dependent spells use sight as a fundamental component of the spell. If the target cannot see the spell, it has no effect.
\item Size-affecting spells alter a creature's size. Multiple size increasing or size decreasing effects never stack. If a creature is affected by both size-increasing and size-decreasing effects, they cancel out on a one for one basis, and any remaining effect occurs normally.
\item Sound-dependent spells use sound as a fundamental component of the spell. If the target cannot hear the spell, it has no effect.
\item Teleportation spells instantaneously move creatures by travelling through the Astral Plane. Anything that blocks planar travel also blocks teleportation.
\item Trap spells do not have obvious effects immediately. They can be detected with the Perception skill. The DC to detect a trap spell is 25 \add spell level. Most, but not all, traps can be disabled with the Disable Device skill. If it can be disabled, the DC is 25 \add spell level.
\par No more than one trap spell can be placed on the same object or in the same area. Only the first trap placed has any effect. It must be dispelled or discharged before any new traps can be placed.
\label{Unreal Spells}\item Unreal spells do not have ``real'' effects and can be disbelieved. Unreal effects cannot cause damage to objects or creatures, support weight, provide nutrition, or provide protection from the elements. Consequently, these spells are useful for confounding or delaying foes, but useless for attacking them directly unless combined with a real effect.
\par When an unreal spell is cast, the caster makes a magic check, with the same bonus as his magic attack bonus. Creatures encountering an unreal spell usually cannot recognize it as illusory until they study it carefully or interact with it in some fashion. A Perception check can be made to interact with an unreal effect if appropriate to the type of effect. Unless otherwise specified by the spell, the DC of such a check is equal to the magic check result used when the effect was created.
\par If a creature interacts with an unreal effect, the magic check is opposed by the creature's Will defense. If the check fails, the creature recognizes the illusion as false. Its effects can still be observed if desired, but they are mere shadows of the full effect: visual effects appear translucent outlines, sounds can be heard as ghostly echoes, and so on.
\par If the check succeeds, the character fails to notice something is amiss. A character faced with definitive proof that an unreal effect isn't real automatically recognizes it as illusory. If any viewer successfully disbelieves an unreal effect and communicates this fact to others, each such viewer receives a \plus5 bonus to their Will defense against the effect.
\end{itemize*}

\subsection{Level}
The next line of a spell description gives the spell's level, a number between 1 and 9 that defines the spell's relative power. This number is preceded by an abbreviation for the class whose members can cast the spell. The Level entry also indicates whether a spell is a domain spell and, if so, what its domain and its level as a domain spell are.

Names of spellcasting classes are abbreviated as follows: cleric Clr; druid Drd; paladin Pal; ranger Rgr; sorcerer Sor; wizard Wiz.

The domains a spell can be associated with include Air, Chaos, Death, Destruction, Earth, Evil, Fire, Good, Knowledge, Law, Leadership, Magic, Nature, Protection, Strength, Travel, Trickery, Vitality, War, and Water.

\subsection{Components}
A spell's components are what you must do or possess to cast it. All spells have verbal and somatic components unless the spell description says otherwise. The Components entry in a spell description includes abbreviations that tell you what type of components it has. Specifics for material components and focuses are given at the end of the descriptive text.

\parhead{Verbal (V)} A verbal component is a spoken incantation. To provide a verbal component, you speak in a strong voice with a volume at least as loud as ordinary conversation.

A gag spoils the incantation (and thus the spell). A spellcaster who has been deafened has a 20\% chance to spoil any spell with a verbal component that he or she tries to cast. Likewise, a \spell{silence} spell imposes a 20\% chance of failure.

\parhead{Somatic (S)} A somatic component is a measured and precise movement of the hand. You must have at least one hand free to provide a somatic component. Touch range spells often include the act of touching the spell recipient as part of the somatic component.

\parhead{Material (M)} A material component is one or more physical substances or objects that are annihilated by the spell energies in the casting process.

\parhead{Focus (F)} A focus component is a prop of some sort. Unlike a material component, a focus is not consumed when the spell is cast and can be reused.

\subsection{Casting Time}
All spells have a casting time of 1 standard action unless otherwise specified in the spell description. Some take 1 round or more, while a few require only a swift or immediate action.

If a spell takes more than a full-round action to cast, you must spend each round of the casting time taking a full-round action to cast the spell. The spell takes effect at the end of your last round of casting. These actions must be consecutive and uninterrupted, or the spell automatically fails.

You make all pertinent decisions about a spell (range, target, area, effect, version, and so forth) when the spell comes into effect.

\subsection{Range}
A spell's range indicates how far from you it can reach, as defined in the Range entry of the spell description. It indicates the maximum distance at which you can designate the spell's point of origin. The effect of a spell can extend beyond that range if it affects an area. A spell without a range simply affects the area specified in the spell's description; if it becomes relevant, you are considered to be its point of origin. Standard ranges include the following.
\parhead{Personal} The spell affects only you.
\parhead{Touch} You must touch a creature or object to affect it. A touch spell that deals damage can score a critical hit just as a weapon can. A touch spell threatens a critical hit on a natural roll of 20 and deals double damage on a successful critical hit. Some touch spells allow you to touch multiple targets. You can touch as many willing targets as you can reach as part of the casting, but all targets of the spell must be touched in the same round that you finish casting the spell. Normally, you can touch no more than six targets per round of casting.

If you have the ability to make multiple touch attacks, such as from the \spell{chill touch} spell, and you can make multiple attacks in a round, you can make a touch attack on each of those attacks.

\parhead{Close} The spell reaches as far as 30 feet.
\parhead{Medium} The spell reaches as far as 100 feet.
\parhead{Far} The spell reaches as far as 300 feet.
\parhead{Unlimited} The spell reaches anywhere on the same plane of existence.
\parhead{Range Expressed in Feet} Some spells have no standard range category, just a range expressed in feet.

\subsection{Aiming a Spell}
You must make some choice about whom the spell is to affect or where the effect is to originate, depending on the type of spell. The next entry in a spell description defines the spell's target (or targets), its manifestation, or its area, as appropriate.

\parhead{Area}\label{Spell Area} Some spells affect an area. Sometimes a spell description specifies a specially defined area, but usually an area falls into one of the categories defined below.

When casting an area spell, you select the point where the spell originates. The point of origin of a spell is always a grid intersection. When determining whether a given creature is within the area of a spell, count out the distance from the point of origin in squares just as you do when moving a character or when determining the range for a ranged attack. The only difference is that instead of counting from the center of one square to the center of the next, you count from intersection to intersection.

You can freely decrease a spell's area, provided that you decrease it uniformly across all of the spell's dimensions. For example, you can cast a \spell{fireball} that affects a 5 foot radius if you choose to do so, but you can't cast a \spell{fireball} with any shape other than a sphere.

You can count diagonally across a square, but remember that every second diagonal counts as 2 squares of distance. If the far edge of a square is within the spell's area, anything within that square is within the spell's area. If the spell's area only touches the near edge of a square, however, anything within that square is unaffected by the spell.

\subparhead{Burst, Emanation, Spread} Many spells that affect an area function as a burst, an emanation, a limit, or a spread. In each case, you select the spell's point of origin and measure its effect from that point.

A burst spell affects whatever it catches in its area, even including creatures that you can't see. A burst's area defines how far from the point of origin the spell's effect extends. The effects of burst spells do not extend around corners.

An emanation spell functions like a burst spell, except that the effect continues to radiate from the point of origin for the duration of the spell. Many emanations are cones.

A spread spell spreads out like a burst but can turn corners. You select the point of origin, and the spell spreads out a given distance in all directions. Figure the area the spell effect fills by taking into account any turns the spell effect takes.

\subparhead{Limit} Some area spells specify a limit. A limiting area is like a range: the spell has effects within the area, but does not affect the entire area at once. The spell will specify the targets that it affects or the manifestations it creates. Limit spells, like bursts and emanations, do not go around corners.

\subparhead{Cone, Cylinder, Line, or Sphere} Most spells that affect an area have a particular shape, such as a cone, cylinder, line, or sphere. Unless otherwise specified, the shape of a burst, emanation, or spread is a sphere.

A cone-shaped spell shoots away from you in a quarter-circle in the direction you designate, extending out to a limit defined by the spell. It starts from any corner of your square and widens out as it goes. Most cones are either bursts or emanations (see above), and thus won't go around corners.

When casting a cylinder-shaped spell, you select the spell's point of origin. This point is the center of a horizontal circle, and the spell shoots down from the circle, filling a cylinder. A cylinder-shaped spell ignores any vertical obstructions within its area.

A line-shaped spell shoots away from you in a line in the direction you designate. It starts from any corner of your square and extends to the limit of its range or until it strikes a barrier that blocks line of effect. A line-shaped spell affects all creatures in squares that the line passes through. Unless otherwise specified, a line spell affects an area 10 feet wide. The affected squares are chosen such that they stay close to the chosen line as possible.

A sphere-shaped spell expands from its point of origin to fill a spherical area. Spheres are typically denoted by simply specifying the radius of the spell.

\subparhead{Area Sizes} The area affected by many spells falls into one of three sizes. Each size defines the extent to which the spell extends out from its origin, whether as a radius or as the length of a cone or line. Small spells extend 10 feet out. Medium spells extend 20 feet out. Large spells extend 50 feet out. Other spells affect a specific area defined in the spell's description.

\subparhead{Objects} A spell with this kind of area affects objects within an area you select (as Creatures, but affecting objects instead).

\subparhead{Other} A spell can have a unique area, as defined in its description.

%Does this still exist?
\subparhead{(S) Shapeable} If an Area or Effect entry ends with ``(S)," you can shape the spell. A shaped effect or area can have no dimension smaller than 10 feet. Many effects or areas are given as cubes to make it easy to model irregular shapes. Three-dimensional volumes are most often needed to define aerial or underwater effects and areas.

\parhead{Target or Targets} Some spells have a target or targets. You cast these spells on creatures or objects, as defined by the spell itself. You must be able to see or touch the target, and you must specifically choose that target. You do not have to select your target until you finish casting the spell.

\subparhead{Multiple Targets} Most spells which have multiple targets also specify an area that the targets must reside in. If the spell says ``all creatures'', you do not have the ability to choose which creatures it affects; otherwise, you may pick and choose creatures within the area.

\subparhead{Personal Spells} Some spells have a target of ``You'' and a range of ``Personal''. Such spells have no defense or spell resistance entries.

\subparhead{Redirecting a Spell} Some spells allow you to redirect the effect to new targets or areas after you cast the spell. Redirecting a spell is a swift action that does not provoke attacks of opportunity.

\subparhead{Targeting Restrictions} Many spells affect ``living creatures,'' which means all creatures other than constructs and undead. Creatures in the spell's area that are not of the appropriate type do not count against the creatures affected.

\subparhead{Willing Targets} Some spells restrict you to willing targets only. Declaring yourself as a willing target is something that can be done at any time (even if you're flat-footed or it isn't your turn). Unconscious creatures and objects are automatically considered willing, but a character who is conscious but immobile or helpless (such as one who is bound, cowering, grappling, paralyzed, pinned, or stunned) is not automatically willing.

\parhead{Manifestation} Some spells create or summon things rather than affecting things that are already present. You must designate the location where these things are to appear, either by seeing it or defining it. Range determines how far away a manifestation can appear, but if the manifestation is mobile, it can move regardless of the spell's range.

\subparhead{Ray} Some effects are rays. You aim a ray as if using a ranged weapon, though typically you make a ranged touch attack rather than a normal ranged attack. As with a ranged weapon, you can fire into the dark or at an invisible creature and hope you hit something. You don't have to see the creature you're trying to hit, as you do with a targeted spell. Intervening creatures and obstacles, however, can block your line of sight or provide cover for the creature you're aiming at.

If a ray spell has a duration, it's the duration of the effect that the ray causes, not the length of time the ray itself persists.

If a ray spell deals damage, you can score a critical hit just as if it were a weapon. A ray spell threatens a critical hit on a natural roll of 20 and deals double damage on a successful critical hit.

\subparhead{Manifestations and Areas} Some effects, such clouds and fogs, create a manifestation within an area. Follow all of the normal rules for determining the area when determining the effects of such a spell.

\parhead{Line of Effect} A line of effect is a straight, unblocked path that indicates what a spell can affect. A line of effect is canceled by a solid barrier. It's like line of sight for ranged weapons, except that it's not blocked by fog, darkness, and other factors that limit normal sight.

You must have a clear line of effect to any target that you cast a spell on or to any space in which you wish to create an effect. You must have a clear line of effect to the point of origin of any spell you cast.

A burst, cone, cylinder, or emanation spell affects only an area, creatures, or objects to which it has line of effect from its origin (a spherical burst's center point, a cone-shaped burst's starting point, a cylinder's circle, or an emanation's point of origin).

An otherwise solid barrier with a hole of at least 1 square foot through it does not block a spell's line of effect. Such an opening means that the 5-foot length of wall containing the hole is no longer considered a barrier for purposes of a spell's line of effect.

\subsection{Duration}
A spell's Duration entry tells you how long the magical energy of the spell lasts.
\parhead{Concentration} The spell lasts as long as you concentrate on it. Concentrating to maintain a spell is a standard action that does not provoke attacks of opportunity. Anything that could break your concentration when casting a spell can also break your concentration while you're maintaining one, causing the spell to end.

You can't cast a spell while concentrating on another one. Sometimes a spell lasts for a short time after you cease concentrating.
\parhead{Timed Durations} Many durations are measured in rounds, minutes, hours, or some other increment. When the time is up, the magic goes away and the spell ends at the end of your turn. For example, a spell that lasts 1 round ends at the end of your next turn. If a spell's duration is variable, the duration is rolled secretly (the caster doesn't know how long the spell will last).
\subparhead{Short} The spell lasts for as long as you concentrate, plus 5 additional rounds.
\subparhead{Medium} The spell lasts for 5 minutes.
\subparhead{Long} The spell lasts for 1 hour.
\subparhead{Extreme} The spell lasts for 12 hours.
\subparhead{Other} Some spells, such as long-lasting rituals, last for a specific amount of time specified in the spell description.
\parhead{Instantaneous} The spell energy comes and goes the instant the spell is cast, though the consequences might be long-lasting.
\parhead{Permanent} The energy remains as long as the effect does. This means the spell is vulnerable to \spell{dispel magic}.

\parhead{Subjects, Effects, and Areas} If the spell affects creatures directly, the result travels with the subjects for the spell's duration. If the spell creates an manifestation, it lasts for the duration. The manifestation might move or remain still. Such an effect can sometimes be destroyed prior to when its duration ends. If the spell affects an area, then the spell stays with that area for its duration. Creatures become subject to the spell when they enter the area and are no longer subject to it when they leave.

\parhead{Touch Spells and Holding the Charge} In most cases, if you don't discharge a touch spell on the round you cast it, you can hold the charge (postpone the discharge of the spell) indefinitely. You can make touch attacks round after round. If you cast another spell, the touch spell dissipates.

Some touch spells allow you to touch multiple targets as part of the spell. You can't hold the charge of such a spell; you must touch all targets of the spell in the same round that you finish casting the spell.

\parhead{Discharge} Occasionally a spells lasts for a set duration or until triggered or discharged.

\parhead{(D) Dismissible} If the Duration line ends with ``(D)," you can dismiss the spell at will. You must be within range of the spell's effect and must speak words of dismissal, which are usually a shortened, modified form of the spell's verbal component. If the spell has no verbal component, you can dismiss the effect with a gesture. Dismissing a spell is a swift action that does not provoke attacks of opportunity. A spell that depends on concentration is dismissible by its very nature, and dismissing it does not take an action, since all you have to do to end the spell is to stop concentrating on your turn.

\subsection{Defenses}\label{Magic Defenses}
Usually, a harmful spell requires a successful magic attack to have its full effect. The Defenses entry in a spell description defines which type of defenses the spell allows, and what effect the spell has if it is resisted.
\parhead{Negates} A failed attack means the spell has no effect.
\parhead{Partial} A failed attack means the spell still has a partial effect.
\parhead{Half} A failed attack means the spell deals half damage.
\parhead{None} No attack is required.
\parhead{Disbelief} The spell can be disbelieved ( \pcref{Unreal Spells}).
\parhead{(object)} The spell can be cast on objects, which have defenses only if they are magical or if they are attended (held, worn, grasped, or the like) by a creature resisting the spell, in which case the object uses the creature's defenses unless its own are greater. (This notation does not mean that a spell can be cast only on objects. Most spells of this sort can be cast on creatures or objects.) A magic item's special defenses are each equal to 12 \add one-half the item's caster level.
\parhead{(harmless)} The spell is usually beneficial, not harmful, but an attack may be required to affect an unwilling creature.

\subsubsection{Magic Attack Bonus}\label{Magic Attack Bonus}
To beat a creature's defenses, you make a magic attack. Your magic attack bonus is equal to half your caster level \add your casting attribute (Intelligence for a wizard, Wisdom for a druid, and Charisma for a sorcerer, paladin, and cleric). If you have more than one caster level, use the caster level appropriate to the class that you are casting the spell from, including any modifiers specific to that spell (such as from \pcref{Spell Focus}).

\subsubsection{Resisting a Spell} A creature that successfully resists a spell that has no obvious physical effects feels a hostile force or a tingle, but cannot deduce the exact nature of the attack without the use of the Spellcraft skill (see \pcref{Spellcraft}).

\subsubsection{Not Resisting a Spell} A creature can voluntarily forego its defenses and willingly accept a spell's result. However, a character with a special resistance to magical effects cannot suppress that quality.

\subsection{Spell Resistance}
Some creatures are unusually resistant to spells. To affect a creature with spell resistance using a spell or spell-like ability, a special attack is always required. The defense used against the attack is indicated by the spell. If the spell already requires an attack of that type, use the same attack to determine both spell resistance and the the effect of the spell. Otherwise, make a separate attack to determine spell resistance. If the attack fails, the spell has no effect on the creature.

Most creatures with spell resistance can willingly allow spells through their resistance if they desire. Some creatures cannot control their spell resistance, so an attack is necessary to affect them even with harmless spells. This is specified in the description of the creature's spell resistance.

\subsection{Effect}
This portion of a spell description details what the spell does and how it works. If one of the previous entries in the description included ``see text,'' this is where the explanation is found. There are several key parts of a spell which are also contained here.

\subsubsection{Damage}
This is the amount of damage the spell deals. Typically, the effect will specify who takes the damage. If no effect is specified, the spell damages all of its targets, or all creatures (but not objects) in the area. A spell with this entry is considered a damaging spell. A spell without this entry is not, even if it could be used to deal damage.

Spells can inflict many kinds of damage. Common damage types include acid, bludgeoning, cold, divine, electricity, fire, force, life, physical, piercing, slashing, solar, and sonic.

\parhead{Damaging Items} Unless the descriptive text for the spell specifies otherwise, all items carried or worn by a creature are assumed to survive a magical attack. If an item is not carried or worn and is not magical, it does not get any defenses. It simply is dealt the appropriate damage.

\subsubsection{Healing}
This is the amount of damage the spell heals. Typically, the effect will specify who receives the healing. If no effect is specified, the spell heals all of its targets, or all creatures (but not objects) in the area.

\subsubsection{Healthy Effect}
This is the effect a spell has on a healthy subject (above half hit points remaining).

\subsubsection{Bloodied Effect}
This is the effect a spell has on a bloodied subject (at or below half hit points remaining). If the spell has a duration, and a healthy creature becomes bloodied during the duration of the spell, it immediately suffers the bloodied effect of the spell. If the spell does not have a duration, any damage the subject takes after being affected by the spell does not change which effect the spell has.

\section{Arcane And Divine Spells}
Wizards and sorcerers cast arcane spells, which involve the direct manipulation of mystic energies. These manipulations require natural talent (in the case of sorcerers) or long study (in the case of wizards).

Clerics, druids, experienced paladins, and experienced rangers can cast divine spells. Unlike arcane spells, divine spells do not directly manipulate magical energy. Instead, they call upon divine powers to intercede on the caster's behalf, creating magical effects. Clerics gain spell power from deities or from divine forces. The divine force of nature powers druid and ranger spells. The divine forces of law and good power paladin spells.

\subsection{Casting and Regaining Spells}
An spellcaster's class level limits the number of spells he or she can cast. A spellcaster must have a casting attribute score at least equal to the spell's level to cast a spell.

\parhead{Spell Slots} To cast a spell of a given level, a spellcaster must spend a spell slot of the appropriate level. If the spellcaster has no spell slots of the appropriate level, she may use a higher-level spell slot instead. The spell is still treated as its actual level, not the level of the slot used to cast it. 

\subparhead{Rest for Arcane Casters} To regain his daily spells, an arcane caster must have a clear mind. To clear her mind she must first rest for 8 hours. The spellcaster does not have to slumber for every minute of the time, but she must refrain from movement, combat, spellcasting, skill use, conversation, or any other fairly demanding physical or mental task during the rest period. If her rest is interrupted, each interruption adds 1 hour to the total amount of time she has to rest in order to clear her mind, and she must have at least 1 hour of uninterrupted rest immediately prior to regaining her spells. If the character does not need to sleep for some reason, she still must have 8 hours of restful calm before preparing any spells.

Divine casters simply request their spells from a divine source, so they do not need rest to regain their spells.

\parhead{Daily Readying of Spells} Regardless of whether they need rest, all spellcasters must spend 15 minutes concentrating at the beginning of the day. During this period, the caster readies his mind to cast his daily allotment of spells. Without such a period to refresh himself, the character does not regain the spell slots he used up the day before.

\parhead{Recent Casting Limit/Rest Interruptions} If a spellcaster has cast spells recently, the drain on her resources reduces her capacity to regain spell slots. When she regains spells for the coming day, all the spell slots she has used within the last 8 hours count against her daily limit.

\parhead{Adding Spells Known} A spellcaster gains spells each time he attains a new level in his class. When your character gains a new level, consult the table for your character's class to learn how many spells from the appropriate spell list he now knows.

\parhead{Changing Spells Known} At each new level, a spellcaster can choose to learn a new spell in place of one he already knows. In effect, the sorcerer ``loses'' the old spell in exchange for the new one. The new spell's level must be the same as that of the spell being exchanged. A sorcerer may swap only a single spell at any given level, and must choose whether or not to swap the spell at the same time that he gains new spells known for the level.  

\subsection{Magical Writings}
To record a spell in written form, a character uses complex notation that describes the magical forces involved in the spell. The notation constitutes a universal language that spellcasters have discovered, not invented. The writer uses the same system no matter what her native language or culture. However, each character uses the system in her own way. Another person's magical writing remains incomprehensible to even the most powerful spellcaster until she takes time to study and decipher it.

To decipher an magical writing (such as a single spell in written form on a scroll), a character must make a Spellcraft check (DC 20 \add the spell's level). If the skill check fails, the character cannot attempt to read that particular spell again until the next day. A \spell{read magic} spell automatically deciphers a magical writing without a skill check. If the person who created the magical writing is on hand to help the reader, success is also automatic.

Once a character deciphers a particular magical writing, she does not need to decipher it again. Deciphering a magical writing allows the reader to identify the spell and gives some idea of its effects (as explained in the spell description). If the magical writing was a scroll and the reader can cast spells of the appropriate type, she can attempt to use
the scroll.

\section{Special Abilities }

\parhead{Spell-Like Abilities} Usually, a spell-like ability works just like the spell of that name. A few spell-like abilities are unique; these are explained in the text where they are described.

A spell-like ability has no verbal, somatic, or material component, nor does it require a focus or have an XP cost. The user activates it mentally. Armor never affects a spell-like ability's use, even if the ability resembles an arcane spell with a somatic component.

A spell-like ability has the casting time of the spell it mimics unless noted otherwise in the ability description. In all other ways, a spell-like ability functions just like a spell.

Spell-like abilities are subject to spell resistance and to being dispelled by \spell{dispel magic}. They do not function in areas where magic is suppressed or negated. Spell-like abilities cannot be used to counterspell, nor can they be counterspelled.

Some creatures are actually sorcerers of a sort. They cast arcane spells as sorcerers do, using components when required. In fact, an individual creature could have some spell-like abilities and also cast other spells as a sorcerer.

\parhead{Supernatural Abilities} These abilities cannot be disrupted in combat, as spells can, and they generally do not provoke attacks of opportunity. Supernatural abilities are not subject to spell resistance, counterspells, or to being dispelled by dispel magic, and do not function in areas where magic is suppressed or negated.

\parhead{Extraordinary Abilities} These abilities cannot be disrupted in combat, as spells can, and they generally do not provoke attacks of opportunity. Effects or areas that negate or disrupt magic have no effect on extraordinary abilities. They are not subject to dispelling, and they function normally in an \spell{antimagic field}. Indeed, extraordinary abilities do not qualify as magical, though they may break the laws of physics.

\parhead{Natural Abilities} This category includes abilities a creature has because of its physical nature. Natural abilities are those not otherwise designated as extraordinary, supernatural, or spell-like.

\section{Arcane Invocations}\label{Arcane Invocations}
Arcane invocations are special spell-like abilities that arcane casters can use at will. Unlike other spell-like abilities, they have verbal and somatic components and are subject to arcane spell failure. All arcane invocations take a standard action to cast unless specified otherwise in the description. Arcane invocations are considered to be 0th level for the purpose of spells and abilities which reference spell level. They are described at the end of Chapter 12.

\section{Rituals}\label{Rituals}
Rituals are ceremonies that create magical effects. Spellcasting characters, or characters with the Ritual Caster feat, can learn and perform rituals. You don't memorize a ritual as you would a normal spell; rituals are too complex for all but the mightiest wizards to commit to memory. To perform a ritual, you need to read from a book or a scroll containing it. Rituals are considered to be spells for many purposes, such as for spell resistance and for effects related to spells, but they are learned and cast in very different ways.
\subsection{Ritual Descriptions}
\par Like a spell, each ritual has a school, a level, and a magical effect. Rituals are described in Chapter 13. The description of each ritual follows the same format as the description of spells in Chapter 12, except for the description of ritual levels. Unlike spells, which have levels based on class, rituals have levels based on the source of magic: arcane, divine, or nature. A ritual always matches the magic source of the person performing the ritual. For example, \spell{scrying} is an arcane ritual when performed by a wizard, but a divine ritual when performed by a cleric.
\subsection{Ritual Requirements} In order to learn and perform a ritual, you must be able to cast at least one spell of the same level as the ritual.
\subsection{Ritual Books}
A ritual book contains one or more rituals that you can use as frequently as you like, as long as you can spare the time and the components to perform the ritual. Scribing a ritual in a ritual book costs an amount of precious inks.
\subsection{Ritual Components}
Every ritual has a material component cost. This cost can be paid with precious metals or gems, or with special materials designed to perform rituals. Some rituals have unique costs and may require specific material components.
\subsection{Ritual Costs}
The costs to scribe and perform rituals are described on \trefnp{Ritual Costs}.
\begin{dtable}
    \lcaption{Ritual Costs}
    \begin{tabularx}{\columnwidth}{X l l}
        \thead{Ritual Level} & \thead{Cost to Perform} & \thead{Cost to Scribe} & \thead{Item Level} & \\
        1st-Level & 5 gp & 50 gp & 1st \\
        2nd-Level & 20 gp & 200 gp & 3rd \\
        3rd-Level & 50 gp & 500 gp & 4th \\
        4th-Level & 125 gp & 1250 gp & 7th \\
        5th-Level & 300 gp & 3000 gp & 9th \\
        6th-Level & 750 gp & 7500 gp & 11th \\
        7th-Level & 1500 gp & 15000 gp & 12th \\
        8th-Level & 3500 gp & 35000 gp & 14th \\
        9th-Level & 7500 gp & 75000 gp & 16th \\
    \end{tabularx}
\end{dtable}

\subsection{Performing Rituals}
\par To perform a ritual, you must have a ritual book containing the ritual and the material components required for the ritual. If you are distracted during the ritual, you must make a Concentration check, just as if you were casting a spell of the ritual's level. If you fail, the ritual is ruined and you must start from the beginning. You can generally recover half the material components from an interrupted ritual.
\par Performing a ritual and casting a ritual mean the same thing.
