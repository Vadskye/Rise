\chapter{Feats}\label{Feats}

This chapter describes a set of optional rules that the GM can choose to use in a campaign.
If you use these rules, characters gain feats which allow them to further specialize in specific areas, making characters more mechanically distinct from each other.
Feats also make the system more mechanically complex, so they are not necessarily enjoyable for all groups.

\section{Gaining Feats}
  There are two main ways you can use feats in your game.
  You cannot gain the same feat twice.

  \parhead{Species Feat Only}
  One simple option is that characters gain a single feat at 1st level based on their species, and no other feats.
  This makes your choice of species more significant without dramatically increasing character complexity.

  \parhead{Feat Progression}
  If you want characters to be more complex and to have more powerful abilities, you can also use a feat progression system.
  For example, you could gain a feat from your species at 1st level, and an additional feat at 3rd, 6th, and 9th level.
  Alternately, you could gain feats based on the completion of major story events.
  In general, it is inadvisable to gain more than four feats total, or to gain feats after about 10th level.

  \subsection{Species Bonus Feats}\label{Species Bonus Feats}
    If you use this rule, each species grants a bonus feat at 1st level.
    Most species can only choose from a small group of feats.
    The specific feats for each species are listed below.
    A character must meet any prerequisites for these bonus feats, as normal.

    \parhead{Human} Any feat.

    \parhead{Dwarf} Any from the following list: \featref*{Battle Armory}, \featref*{Blindfighter}, \featref*{Craft Specialization} and \featref*{Endurance Specialization}, \featref*{Iron Will}, \featref*{Martial Training}, or \featref*{Toughness}.

    \parhead{Elf} Any Casting feat (see \pcref{Casting Feats}), or any from the following list: \featref*{Awareness Specialization} and \featref*{Balance Specialization}, \featref*{Sniper}, or \featref*{Rapid Reaction}.

    \parhead{Gnome} Any Casting feat (see \pcref{Casting Feats}), or any from the following list: \featref*{Blindfighter}, \featref*{Craft Specialization} and \featref*{Stealth Specialization}, \featref*{Telepath}, or \featref*{Toughness}.

    \parhead{Half-Elf} Any Skill feat (see \pcref{Skill Feats}).

    \parhead{Half-Orc} Any Combat feat (see \pcref{Combat Feats}), or any from the following list: \featref*{Endurance Specialization} and \featref*{Intimidate Specialization}, or \featref*{Toughness}.

    \parhead{Halfling} Any from the following list: \featref*{Balance Specialization}, \featref*{Stealth Specialization}, \featref*{Climb Specialization}, \featref*{Iron Will}, or \featref*{Rapid Reaction}.

    \subsubsection{Uncommon Species}
      If you are using uncommon species, the feat lists for each uncommon species are given below.
      Note that uncommon species are normally ineligible for any Ancestry feats.

      \parhead{Animal Hybrid} Any feat strongly associated with the chosen animal. For example, a hybrid shark might choose from Awareness Specialization, Survival Specialization, Swiftrunner, or Swim Specialization. A hybrid wolf might choose from Awareness Specialization, Rapid Reaction, Stealth Specialization, Survival Specialization, or Swiftrunner.

      \parhead{Awakened Animal} Any feat strongly associated with the chosen animal. For example, an awakened cat might choose from Awareness Specialization, Climb Specialization, Flexibility Specialization, Rapid Reaction, Stealth Specialization, or Swiftrunner.

      \parhead{Changeling} Chameleon, or any Skill feat.

      \parhead{Dragon} Iron Will, Toughness, or any Casting feat (see \pcref{Casting Feats}).

      \parhead{Drakkenfel} Draconic Ancestry. The type of dragon chosen for the drakkenfel's \textit{draconic ancestry} must match its \textit{draconic essence}.

      \parhead{Dryaidi} Herbalist, Mental Magic, Regenerator, Sphere Focus: Toxicology, Sphere Focus: Verdamancy, or Toughness.

      \parhead{Eladrin} Boongiver, Chameleon, Combat Style Versatility, Deception Specialization and Persuasion Specialization, or Spellwarped.

      \parhead{Kit} Balance Specialization and Stealth Specialization, Deception Specialization and Social Insight Specialization, or Swiftrunner.

      \parhead{Naiadi} Boongiver, Leadership, Mental Magic, Persuasion Specialization and Swim Specialization, or Sphere Focus: Aquamancy.

      \parhead{Orc} Any Combat feat (see \pcref{Combat Feats}), or any from the following list: \featref*{Endurance Specialization} and \featref*{Intimidate Specialization}, or \featref*{Toughness}.

      \parhead{Oozeborn} Blindfighter, Chameleon, Climb Specialization and Flexibility Specialization, Juggernaut, Regenerator, Sphere Focus: Toxicology, or Toughness.

      \parhead{Tiefling} \featref*{Deception Specialization} and \featref*{Intimidate Specialization}, \featref*{Executioner}, \featref*{Spellwarped}, or \featref*{Sphere Focus: Pyromancy}.

    \subsubsection{Changing Species}
      In extraordinary cases, a creature may change its species.
      For example, the \ritual{reincarnation} ritual returns a creature to life as a different species.
      Regardless of its new species, the creature keeps its original species bonus feat.

\section{Feat Mechanics}

  \subsection{Prerequisites}
    Some feats have prerequisites.
    Unless you meet all of the prerequisites, you cannot take the feat.
    Prerequisites can include a minimum attribute value, another feat or feats, a minimum level of training in a skill, or some other property of a character.
    A character can gain a feat at the same level at which they gain the prerequisite.

    A character who no longer meets all prerequisites for a feat loses all abilities from that feat.

  \subsection{Skill Feats}
    Skill feats are weaker and more narrow in their focus than other feats.
    Whenever a character would choose a single feat, they may instead choose two skill feats.

  \subsection{Feat Tags}
    All feats are organized into different groups by tags.

    \parhead{General} General feats can have a wide variety of effects.
    They often grant new abilities or improve your defenses.

    \parhead{Combat} Combat feats improve your combat capabilities.
    They can increase the damage you deal, grant you new combat abilities, or improve your defenses.

    \parhead{Casting} Casting feats improve your spellcasting abilities.
    Casting feats are useless to characters who cannot cast spells.

    \parhead{Skill} Skill feats improve your skills.
    They can make you more likely to succeed with skill checks and grant you new abilities based on your skills.

    \parhead{Ancestry Feats} Some characters have traces of monstrous blood running in their veins.
    Most of those will never understand the full potential of their unusual heritage.
    Ancestry feats allow characters to explore those posibilities by gaining abilities related to their ancestry.
    You can only have one Ancestry feat.

    \parhead{Magical Feats}
    All abilities granted by feats with the Magical type are \magical in nature.
    Many feats are not entirely magical, but have specific effects that are magical.

    % Feat names must follow ``I have'' or ``I am (a)''.
    \begin{longcolumn}
      \begin{longtablewrapper}
        \tablebookmark{Feats}{feats}
        \begin{longtable}{>{\lcol}p{13em} >{\lcol}p{10em} l >{\lcol}p{8em} >{\lcol}p{3em}}
          \lcaption{Feats}                                                                                                                                                                                  \\
          \tb{General Feats}\label{General Feats}         & \tb{Prerequisites}               & \tb{Benefits}                              & \tb{Feat Types}   & \tb{Page}                                   \\
          \featref{Ascetic}                               & Wil 1                            & Cannot use items, gain extra archetype     & \tdash            & \featpref{Ascetic}                          \\
          \magicalfeatref{Barbaric Ancestry}              & Str \add Dex \add Con is 3       & Gain aid from barbarian ancestors          & Ancestry          & \featpref{Barbaric Ancestry}                \\
          \magicalfeatref{Celestial Ancestry}             & Non-evil                         & Gain aspects of celestial beings           & Ancestry, Magical & \featpref{Celestial Ancestry}               \\
          \featref{Chameleon}                             & Trained Disguise, Int 2          & Adapt your archetypes and abilities        & \tdash            & \featpref{Chameleon}                        \\
          \featref{Draconic Ancestry}                     & \tdash                           & Gain aspects of draconic power             & Ancestry          & \featpref{Draconic Ancestry}                \\
          \magicalfeatref{Entropist}                      & Wil 2                            & Master chaos and entropy                   & Magical           & \featpref{Entropist}                        \\
          \magicalfeatref{Fateweaver}                     & Per 2                            & Ensure success for you and allies          & \tdash            & \featpref{Fateweaver}                       \\
          \featref{Herbalist}                             & Trained Knowledge (nature)       & Brew potions with natural ingredients      & \tdash            & \featpref{Herbalist}                        \\
          \featref{Iron Will}                             & Wil 2                            & Increase mental resilience                 & \tdash            & \featpref{Iron Will}                        \\
          \featref{Null}                                  & Wil 2                            & Become immune to magic                     & \tdash            & \featpref{Null}                             \\
          \featref{Precognition}                          & Int 3                            & React to future events                     & \tdash            & \featpref{Precognition}                     \\
          \featref{Regenerator}                           & Con 3                            & Heal wounds with inhuman speed             & \tdash            & \featpref{Regenerator}                      \\
          \featref{Rapid Reaction}                        & Dex 2                            & Increase reaction speed                    & \tdash            & \featpref{Rapid Reaction}                   \\
          \magicalfeatref{Spellwarped}                    & Wil 2                            & Gain limited spellcasting                  & Magical           & \featpref{Spellwarped}                      \\
          \featref{Swiftrunner}                           & Dex 3                            & Move more quickly                          & \tdash            & \featpref{Swiftrunner}                      \\
          \magicalfeatref{Telepath}                       & Int \add Wil is 3                & Communicate with creatures mentally        & Magical           & \featpref{Telepath}                         \\
          \featref{Toughness}                             & Con 2                            & Increase physical fortitude                & \tdash            & \featpref{Toughness}                        \\

          \tb{Skill Feats}\label{Skill Feats}             & \tb{Prerequisites}               & \tb{Benefits}                              & \tb{Feat Types}   & \tb{Page}                                   \\
          \featref{Awareness Specialization}              & Trained Awareness                & Improve use of chosen skill                & \tdash            & \featpref{Awareness Specialization}         \\
          \featref{Balance Specialization}                & Trained Balance                  & Improve use of chosen skill                & \tdash            & \featpref{Balance Specialization}           \\
          \featref{Climb Specialization}                  & Trained Climb                    & Improve use of chosen skill                & \tdash            & \featpref{Climb Specialization}             \\
          \featref{Craft Specialization}                  & Trained Craft                    & Improve use of chosen skill                & \tdash            & \featpref{Craft Specialization}             \\
          \featref{Creature Handling Specialization}      & Trained Creature Handling        & Improve use of chosen skill                & \tdash            & \featpref{Creature Handling Specialization} \\
          \featref{Deception Specialization}              & Trained Deception                & Improve use of chosen skill                & \tdash            & \featpref{Deception Specialization}         \\
          \featref{Devices Specialization}                & Trained Devices                  & Improve use of chosen skill                & \tdash            & \featpref{Devices Specialization}           \\
          \featref{Disguise Specialization}               & Trained Disguise                 & Improve use of chosen skill                & \tdash            & \featpref{Disguise Specialization}          \\
          \featref{Endurance Specialization}              & Trained Endurance                & Improve use of chosen skill                & \tdash            & \featpref{Endurance Specialization}         \\
          \featref{Flexibility Specialization}            & Trained Flexibility              & Improve use of chosen skill                & \tdash            & \featpref{Flexibility Specialization}       \\
          \featref{Intimidate Specialization}             & Trained Intimidate               & Improve use of chosen skill                & \tdash            & \featpref{Intimidate Specialization}        \\
          \featref{Jump Specialization}                   & Trained Jump                     & Improve use of chosen skill                & \tdash            & \featpref{Jump Specialization}              \\
          \featref{Knowledge Specialization}              & Trained Knowledge                & Improve use of chosen skill                & \tdash            & \featpref{Knowledge Specialization}         \\
          \featref{Medicine Specialization}               & Trained Medicine                 & Improve use of chosen skill                & \tdash            & \featpref{Medicine Specialization}          \\
          \featref{Perform Specialization}                & Trained Perform                  & Improve use of chosen skill                & \tdash            & \featpref{Perform Specialization}           \\
          \featref{Persuasion Specialization}             & Trained Persuasion               & Improve use of chosen skill                & \tdash            & \featpref{Persuasion Specialization}        \\
          \featref{Ride Specialization}                   & Trained Ride                     & Improve use of chosen skill                & \tdash            & \featpref{Ride Specialization}              \\
          \featref{Sleight of Hand Specialization}        & Trained Sleight of Hand          & Improve use of chosen skill                & \tdash            & \featpref{Sleight of Hand Specialization}   \\
          \featref{Social Insight Specialization}         & Trained Social Insight           & Improve use of chosen skill                & \tdash            & \featpref{Social Insight Specialization}    \\
          \featref{Stealth Specialization}                & Trained Stealth                  & Improve use of chosen skill                & \tdash            & \featpref{Stealth Specialization}           \\
          \featref{Survival Specialization}               & Trained Survival                 & Improve use of chosen skill                & \tdash            & \featpref{Survival Specialization}          \\
          \featref{Swim Specialization}                   & Trained Swim                     & Improve use of chosen skill                & \tdash            & \featpref{Swim Specialization}              \\

          \tb{Casting Feats}\label{Casting Feats}         & \tb{Prerequisites}               & \tb{Benefits}                              & \tb{Feat Types}   & \tb{Page}                                   \\
          \magicalfeatref{Boongiver}                      & Spellcasting                     & Improve ability to cast spells on allies   & Magical           & \featpref{Boongiver}                        \\
          \magicalfeatref{Blood Magic}                    & Spellcasting, Con 2              & Spend hit points to improve magic          & Magical           & \featpref{Blood Magic}                      \\
          \magicalfeatref{Mental Magic}                   & Spellcasting, Wil 3              & Cast spells without words or gestures      & Magical           & \featpref{Mental Magic}                     \\
          \magicalfeatref{Metacaster}                     & Spellcasting, Int 2              & Manipulate spell effects in creative ways  & Magical           & \featpref{Metacaster}                       \\
          \magicalfeatref{Mystic Archer}                  & Spellcasting                     & Imbue projectiles with magic               & Magical           & \featpref{Mystic Archer}                    \\
          \magicalfeatref{Prepared Spellcasting}          & Spellcasting, Int 3              & Prepare additional spells each day         & Magical           & \featpref{Prepared Spellcasting}            \\
          \magicalfeatref{Spellsword}                     & Spellcasting                     & Fight with sword and spell together        & Magical           & \featpref{Spellsword}                       \\
          \magicalfeatref{Sphere Focus: Aeromancy}        & \sphere{Aeromancy} sphere        & Improve casting with chosen sphere         & Magical           & \featpref{Sphere Focus: Aeromancy}          \\
          \magicalfeatref{Sphere Focus: Aquamancy}        & \sphere{Aquamancy} sphere        & Improve casting with chosen sphere         & Magical           & \featpref{Sphere Focus: Aquamancy}          \\
          \magicalfeatref{Sphere Focus: Astromancy}       & \sphere{Astromancy} sphere       & Improve casting with chosen sphere         & Magical           & \featpref{Sphere Focus: Astromancy}         \\
          \magicalfeatref{Sphere Focus: Channel Divinity} & \sphere{Channel Divinity} sphere & Improve casting with chosen sphere         & Magical           & \featpref{Sphere Focus: Channel Divinity}   \\
          \magicalfeatref{Sphere Focus: Chronomancy}      & \sphere{Chronomancy} sphere      & Improve casting with chosen sphere         & Magical           & \featpref{Sphere Focus: Chronomancy}        \\
          \magicalfeatref{Sphere Focus: Cryomancy}        & \sphere{Cryomancy} sphere        & Improve casting with chosen sphere         & Magical           & \featpref{Sphere Focus: Cryomancy}          \\
          \magicalfeatref{Sphere Focus: Electromancy}     & \sphere{Electromancy} sphere     & Improve casting with chosen sphere         & Magical           & \featpref{Sphere Focus: Electromancy}       \\
          \magicalfeatref{Sphere Focus: Enchantment}      & \sphere{Enchantment} sphere      & Improve casting with chosen sphere         & Magical           & \featpref{Sphere Focus: Enchantment}        \\
          \magicalfeatref{Sphere Focus: Fabrication}      & \sphere{Fabrication} sphere      & Improve casting with chosen sphere         & Magical           & \featpref{Sphere Focus: Fabrication}        \\
          \magicalfeatref{Sphere Focus: Photomancy}       & \sphere{Photomancy} sphere       & Improve casting with chosen sphere         & Magical           & \featpref{Sphere Focus: Photomancy}         \\
          \magicalfeatref{Sphere Focus: Polymorph}        & \sphere{Polymorph} sphere        & Improve casting with chosen sphere         & Magical           & \featpref{Sphere Focus: Polymorph}          \\
          \magicalfeatref{Sphere Focus: Prayer}           & \sphere{Prayer} sphere           & Improve casting with chosen sphere         & Magical           & \featpref{Sphere Focus: Prayer}             \\
          \magicalfeatref{Sphere Focus: Pyromancy}        & \sphere{Pyromancy} sphere        & Improve casting with chosen sphere         & Magical           & \featpref{Sphere Focus: Pyromancy}          \\
          \magicalfeatref{Sphere Focus: Revelation}       & \sphere{Revelation} sphere       & Improve casting with chosen sphere         & Magical           & \featpref{Sphere Focus: Revelation}         \\
          \magicalfeatref{Sphere Focus: Summoning}        & \sphere{Summoning} sphere        & Improve casting with chosen sphere         & Magical           & \featpref{Sphere Focus: Summoning}          \\
          \magicalfeatref{Sphere Focus: Telekinesis}      & \sphere{Telekinesis} sphere      & Improve casting with chosen sphere         & Magical           & \featpref{Sphere Focus: Telekinesis}        \\
          \magicalfeatref{Sphere Focus: Terramancy}       & \sphere{Terramancy} sphere       & Improve casting with chosen sphere         & Magical           & \featpref{Sphere Focus: Terramancy}         \\
          \magicalfeatref{Sphere Focus: Thaumaturgy}      & \sphere{Thaumaturgy} sphere      & Improve casting with chosen sphere         & Magical           & \featpref{Sphere Focus: Thaumaturgy}        \\
          \magicalfeatref{Sphere Focus: Toxicology}       & \sphere{Toxicology} sphere       & Improve casting with chosen sphere         & Magical           & \featpref{Sphere Focus: Toxicology}         \\
          \magicalfeatref{Sphere Focus: Umbramancy}       & \sphere{Umbramancy} sphere       & Improve casting with chosen sphere         & Magical           & \featpref{Sphere Focus: Umbramancy}         \\
          \magicalfeatref{Sphere Focus: Verdamancy}       & \sphere{Verdamancy} sphere       & Improve casting with chosen sphere         & Magical           & \featpref{Sphere Focus: Verdamancy}         \\
          \magicalfeatref{Sphere Focus: Vivimancy}        & \sphere{Vivimancy} sphere        & Improve casting with chosen sphere         & Magical           & \featpref{Sphere Focus: Vivimancy}          \\
          \magicalfeatref{Twinhand Spellcaster}           & Dex 2                            & Cast spells with two hands at once         & Magical           & \featpref{Twinhand Spellcaster}             \\
          \magicalfeatref{Wardweaver}                     & \abilitytag{Barrier} spell known & Create more powerful barriers              & Magical           & \featpref{Wardweaver}                       \\

          \tb{Combat Feats}\label{Combat Feats}           & \tb{Prerequisites}               & \tb{Benefits}                              & \tb{Feat Types}   & \tb{Page}                                   \\
          \featref{Arbalist}                              & Dex 2, Per 1                     & Use crossbows more quickly and easily      & \tdash            & \featpref{Arbalist}                         \\
          \featref{Battle Armory}                         & Str 1, Dex 2                     & Switch between different weapons easily    & \tdash            & \featpref{Battle Armory}                    \\
          \featref{Blindfighter}                          & Per 2                            & Fight unseen foes better                   & \tdash            & \featpref{Blindfighter}                     \\
          \featref{Brawler}                               & Str 1, Dex 1                     & Fight better unarmed and in close quarters & \tdash            & \featpref{Brawler}                          \\
          \featref{Combat Style Versatility}              & Int 1, combat style              & Use highly varied combat styles            & \tdash            & \featpref{Combat Style Versatility}         \\
          \featref{Duelist}                               & Dex 2, Int 1                     & Fight one-on-one better                    & \tdash            & \featpref{Duelist}                          \\
          \featref{Executioner}                           & Per 2                            & Kill weakened foes more easily             & \tdash            & \featpref{Executioner}                      \\
          \featref{Greatweapon Warrior}                   & Str 3                            & Fight better with two-handed weapons       & \tdash            & \featpref{Greatweapon Warrior}              \\
          \magicalfeatref{Ghostblade}                     & Dex 1, Wil 2                     & Tap into ghostly powers in combat          & Magical           & \featpref{Ghostblade}                       \\
          \featref{Juggernaut}                            & Str 1, Con 2                     & Become unstoppable and trample foes        & \tdash            & \featpref{Juggernaut}                       \\
          \featref{Leadership}                            & Int 2 or Wil 2                   & Inspire nearby allies                      & \tdash            & \featpref{Leadership}                       \\
          \featref{Maneuverist}                           & Int 2                            & Gain limited maneuver access               & \tdash            & \featpref{Maneuverist}                      \\
          \featref{Martial Training}                      & \tdash                           & Improve combat abilities                   & \tdash            & \featpref{Martial Training}                 \\
          \featref{Shieldbearer}                          & Str 2                            & Defend better with shields                 & \tdash            & \featpref{Shieldbearer}                     \\
          \featref{Sniper}                                & Per 2                            & Aim precisely at distant foes              & \tdash            & \featpref{Sniper}                           \\
          \featref{Trickshot}                             & Dex 2, Int 1                     & Use alchemical items with arrows           & \tdash            & \featpref{Trickshot}                        \\
          \featref{Twin-Weapon Fighting}                  & Dex 2                            & Fight better with two weapons at once      & \tdash            & \featpref{Twin-Weapon Fighting}             \\
          \featref{Weapon Focus}                          & \tdash                           & Fight better with a single type of weapon  & \tdash            & \featpref{Weapon Focus}                     \\
          \featref{Whirlwind Warrior}                     & Dex 2                            & Fight hordes with agile ease               & \tdash            & \featpref{Whirlwind Warrior}                \\
        \end{longtable}
      \end{longtablewrapper}
    \end{longcolumn}

\section{Feat Descriptions}
  Each feat has a set of benefits it provides to a character with the feat.
  Some feats also have specific requirements that a character must meet before taking the feat.
  These are listed under a \textbf{Prerequisites} heading.
  If a character loses the prerequisites for a feat, they lose all benefits of the feat until they meet the prerequisites again.

  \begin{feat}{Arbalist}{Combat}
    \featpre Dexterity 2, Perception 1.

    \ff[1]{Rapid Reload} Crossbows that would normally take a \glossterm{minor action} for you to reload instead take a \glossterm{free action}.
    This allows you to reload them any number of times per round.
    You must still have a free hand, if one would normally be required.

    \ff[6]{Multiload} Whenever you load a non-repeating crossbow, you can load it with two bolts instead of one.
    When you fire a multiloaded crossbow, you take a \minus1 accuracy penalty, but the strike has an additional \glossterm{secondary target} within 10 feet of the primary target.
    You cannot choose a secondary target with this ability that is also a primary target.

    \ff[12]{Rapid Reload+} You can reload crossbows that would normally require a standard action as a \glossterm{free action}.
    When you do, you increase your \glossterm{fatigue level} by one.

    \ff[18]{Multiload+} The number of secondary targets increases to two, and the maximum distance between the primary target and each secondary target increases to 20 feet.
  \end{feat}

  \begin{feat}{Ascetic}{General}
    \featpre Willpower 1.

    \ff[1]{Deny Wealth} You cannot knowingly own, consume, or use any item or collection of items that is worth more than 1gp.
    If you do, you lose all benefits of this feat, and you suffer a \minus2 penalty to all attacks, checks, and defenses.
    These penalties last until you spend a month following this restriction again.
    If you break this prohibition under duress, including magical compulsion, these penalties only last for three days.

    Any item you craft yourself is exempt from this limitation, as long as the raw materials used to create the item have a value less than 1gp.
    You are allowed to carry valuable items as long as you gain no direct benefit from them, so you can hold items for your allies or carry gold until you can find an appropriate place to donate it.
    You can also interact with expensive structures that you do not own, such as ornate doors.

    \ff[1]{Open Mind} You gain an additional \glossterm{insight point}.

    \magicalff[1]{Sanctified Garb} Any clothing you wear provides a \plus4 bonus to your Armor defense.
    This defense bonus does not stack with the benefits of body armor.
    However, you are not considered to be wearing body armor, so you can still gain damage reduction from effects like \ability{mage armor}.

    \magicalff[6]{Ascetic Archetype} Choose one archetype from your \glossterm{base class} that you do not already have any
    ranks in.
    You are rank 1 in that archetype, and gain the appropriate abilities.
    Your rank in that archetype increases by 1 at 9th level, and every 3 levels thereafter.

    \magicalff[6]{Renounce Legacy} You do not choose a legacy item, and you do not gain any legacy item upgrades (see \pcref{Legacy Items}).
    Instead, you gain a \plus1 bonus to all \glossterm{defenses}.
    This bonus increases to \plus2 at level 12 and to \plus3 at level 18.

    \ff[12]{Open Mind+} You gain an additional insight point.

    \magicalff[12]{Sanctified Garb+} The Armor defense bonus increases to \plus5.

    \ff[18]{Open Mind++} You gain two additional insight points.
  \end{feat}

  \begin{feat}{Awareness Specialization}{Skill}
    \featpre Awareness as a trained skill.

    \ff[1]{Specialization} You gain a \plus3 bonus to the Awareness skill.

    \ff[6]{Extraordinary Senses+} You gain one of the following senses: \trait{blindsense} (90 ft.), \trait{blindsight} (30 ft.), \trait{darkvision} (120 ft.), \trait{tremorsense} (60 ft.), or \trait{tremorsight} (30 ft.).
    As normal, if you already have the chosen sense from another source, you sum the ranges from both abilities to determine your total range with that sense.

    \ff[12]{Specialization+} The bonus from your \textit{specialization} ability increases to \plus6.

    \ff[18]{Extraordinary Senses+} You can choose an additional sense from the list given in your \textit{extraordinary senses} ability, except that the range is doubled.
    You cannot choose the same sense that you chose with your \textit{extraordinary senses} ability.
    However, you can also change the sense you chose with that ability.
  \end{feat}

  \begin{feat}{Balance Specialization}{Skill}
    \featpre Balance as a trained skill.

    \ff[1]{Specialization} You gain a \plus3 bonus to the Balance skill.

    \ff[6]{Instant Stand} You can use the \textit{rapid stand} ability as a \glossterm{free action} instead of a \glossterm{minor action} (see \pcref{Rapid Stand}).

    \ff[12]{Specialization+} The bonus from your \textit{specialization} ability increases to \plus6.

    % Modifier at 18: 3 dex + 6 spec + 9 level + 3 train = 21.
    \magicalff[18]{Air Dancer} You can attempt to move on surfaces that cannot support your weight, as described below.
    \begin{itemize}
      \item Surfaces that can support at least a quarter of your weight, such as thin tree branches and dense liquids, are \glossterm{difficulty value} 15.
      \item Surfaces that can support at least a tenth of your weight, such as water, are \glossterm{difficulty value} 20.
      \item Surfaces that can support at least a hundredth of your weight, such as tree leaves, are \glossterm{difficulty value} 25.
      \item Surfaces that cannot support your weight at all, such as air, are \glossterm{difficulty value} 30.
    \end{itemize}

    Success means you move along the surface.
    Failure means you fall through the surface, and you cannot use this ability again during the current phase.
    The \glossterm{difficulty value} increases by 5 for each consecutive round that you spend moving in this way.
  \end{feat}

  \begin{magicalfeat}{Barbaric Ancestry}{Ancestry, Magical}
    \featpre Strength, Dexterity, and Constitution sum to at least 3
    \parhead{Special} You can only have one Ancestry feat.

    % Better than Cleave in large group fights where you have time to prebuff, worse otherwise
    \magicalff[1]{Ancestral Battlecry}
    \begin{magicalactiveability}{Ancestral Battlecry}
      \abilityusagetime Standard action or \glossterm{minor action}.
      \abilitycost One \glossterm{fatigue level} if used as a minor action.
      \rankline
      Three ghostly spirits of your ancestors appear around you.
      They cannot be attacked or interacted with.
      Whenever you make a melee \glossterm{strike}, each spirit mimics you.
      For each spirit that mimics you, the strike can affect an additional secondary target of your choice within 15 foot \glossterm{range} of you, even if you are not adjacent to it.

      At the end of each subsequent round after you use this ability, one of the spirits disappears.
      When all of the spirits have disappeared, this effect ends.
    \end{magicalactiveability}

    \magicalff[6]{Ancestral Totem}
    \begin{magicalsustainability}{Ancestral Totem}{\abilitytag{Manifestation}, \abilitytag{Sustain} (free)}
      \abilityusagetime Standard action or \glossterm{minor action}.
      \abilitycost One \glossterm{fatigue level} if used as a minor action.
      \rankline
      Choose an unoccupied location on solid ground within \longrange of you.
      A Small totem representing your ancestors appears in that location.
      The totem is a creature that cannot take actions of any kind, including \glossterm{movement}.
      Its \glossterm{hit points} are equal to the standard value for your level and base class.
      Its \glossterm{damage resistance} is equal to half its hit points, ignoring any \glossterm{enhancement bonuses} to hit points.
      Each of its defenses is equal to 5 \add half your level.
      The totem does not block movement for any creatures.

      While you are within a \medarea radius \glossterm{emanation} from the totem, you gain a \plus2 accuracy bonus with melee strikes.
    \end{magicalsustainability}

    \magicalff[12]{Ancestral Protectors} While you have ghostly spirits active from your \ability{ancestral battlecry} ability, you gain a \plus1 bonus to your Armor defense.
    In addition, you create four spirits when you use that ability rather than three.

    \magicalff[18]{Ancestral Warchief} The range of secondary strikes from your \ability{ancestral battlecry} ability increases to \shortrange.
    In addition, while you are within the emanation of your \ability{ancestral totem}, it counts as an extra spirit for \ability{ancestral battlecry} that does not disappear.
  \end{magicalfeat}

  \begin{feat}{Battle Armory}{Combat}
    \featpre Strength 1, Dexterity 2.

    \ff[1]{Overburdened Quickdraw} You can draw or sheathe any weapon or shield other than a tower shield as a \glossterm{free action}.
    This still counts against your one free action object manipulation per round (see \pcref{Manipulating Objects}).
    In addition, you can store up to ten items for easy access on your body, rather than five (see \pcref{Storing Items}).

    \magicalff[6]{Legacy Armory} You do not choose an individual weapon as a \glossterm{legacy item} (see \pcref{Legacy Items}).
    Instead, if you choose weapons as your legacy item category, you choose magic weapon abilities that apply to all nonmagical weapons you wield.
    If you attune to a magical weapon, it keeps its own magical effects instead of your chosen legacy item properties.

    \ff[6]{Rapid Attunement} You can use the \ability{item attunement} ability to attune to weapons and armor as a \glossterm{minor action} (see \pcref{Item Attunement}).

    \ff[12]{Versatile Force} You gain a \plus1 bonus to your \glossterm{power} with all abilities.

    \ff[18]{Fast Hands} You get two free action object manipulations each round, rather than only one (see \pcref{Manipulating Objects}).
  \end{feat}

  \begin{feat}{Blindfighter}{Combat}
    \featpre Perception 2.

    \ff[1]{Unerring} You ignore all \glossterm{miss chances}, such as from \glossterm{concealment}.
    This does not prevent your attacks from failing for other reasons.

    \ff[6]{Blindsight} You gain \trait{blindsense} with a 90 foot range, allowing you to sense your surroundings without light (see \pcref{Blindsense}).
    If you already have blindsense, the range of your blindsense increases by 90 feet.
    In addition, you gain \trait{blindsight} with a 30 foot range, allowing you to see without light (see \pcref{Blindsight}).
    If you already have blindsight, the range of your blindsight increases by 30 feet.

    \ff[12]{Unseeing Precision} You gain a \plus1 bonus to \glossterm{accuracy}.

    \ff[18]{Blindsight+} The range of your blindsense increases by 90 feet.
    In addition, the range of your blindsight increases by 150 feet.
  \end{feat}

  \begin{magicalfeat}{Blood Magic}{Casting, Magical}
    \featpre Access to a \glossterm{mystic sphere}, Constitution 2.

    \magicalff[1]{Bloodspell} Whenever you cast a spell, you may use this ability.
    When you do, you lose \glossterm{hit points} equal to your maximum spell rank.
    In exchange, you gain a \plus2 power bonus with the spell, and the spell does not require \glossterm{verbal components} or \glossterm{somatic components}.

    \magicalff[6]{Bloodbind} Whenever you make a living creature lose \glossterm{hit points} using a spell, you can choose to bind the target's blood to yours.
    While the target is bound, you are always considered to have \glossterm{line of sight} to it.
    You can see it through all forms of \glossterm{concealment} and \glossterm{cover}, even if it is \trait{invisible} or fully behind a solid object.
    This binding lasts until you finish a \glossterm{long rest}.

    \magicalff[12]{Blood-Sworn Allegiance} You can swear a blood oath to your allies.
    You automatically fail any attack rolls you make against a creature that you have sworn this blood oath with.
    In addition, those creatures automatically fail any attack rolls they make against you.
    This allows you to safely use area spells that include your allies as targets, since they will suffer no ill effect from the attack.

    \magicalff[18]{Bloodbind+} You are always considered to have \glossterm{line of effect} with \magical abilities to a creature bound by your \textit{bloodbind} ability, regardless of intervening obstacles.
    This allows you to cast spells on them through solid walls.

    \magicalff[18]{Bloodspell+} The power bonus from your \textit{bloodspell} ability increases to \plus4.
  \end{magicalfeat}

  \begin{magicalfeat}{Boongiver}{Casting, Magical}
    \featpre Access to a \glossterm{mystic sphere}.

    \magicalff[1]{Share Boon}
    \begin{magicalactiveability}{Share Boon}
      \abilityusagetime Can be triggered when you cast a spell with the \abilitytag{Attune} tag.
      \rankline
      The spell's \abilitytag{Attune} tag changes to \abilitytag{Attune} (target).
      Choose one \glossterm{ally} within \rngmed range.
      That ally is the target of the spell, and the spell affects that creature as if it were you instead of affecting you.

      You can only use this ability to affect one spell at a time.
      If you use it again, the original ally's attunement to the old spell ends.
    \end{magicalactiveability}

    \magicalff[6]{Sustain Attunement} Whenever you cast an \abilitytag{Attune} spell that is not a \glossterm{deep attunement}, you can choose to replace its \abilitytag{Attune} tag with the \abilitytag{Sustain (minor)} tag.

    \magicalff[12]{Versatile Boon Lore} You learn additional \glossterm{spell}.
    The spell must have the \abilitytag{Attune} tag, but you do not need to have access to the \glossterm{mystic sphere} it is from.
    The mystic sphere it is from must still be part of your \glossterm{magic source}.
    As normal, you can exchange this spell for other spells as you gain access to new spell ranks, but the spell must always have the \abilitytag{Attune} tag.

    \magicalff[18]{Share Boon+} You can use your \textit{share boon} ability on up to three different spells at once.
    If you use the ability while it already affects another spell, you choose which spells are affected by the ability.
  \end{magicalfeat}

  \begin{feat}{Brawler}{Combat}
    \featpre Strength 1, Dexterity 1.

    \ff[1]{Unarmed Warrior} You gain a \plus2 accuracy bonus with the punch/kick \glossterm{natural weapon}, and you deal 1d4 damage with it (see \pcref{Natural Weapons}).
    This ability does not stack with the ability of the same name from the Perfected Form monk archetype (see \pcref{Perfected Form}).

    \ff[1]{Brawling Expertise} You gain a \plus2 bonus to \glossterm{accuracy} with all \abilitytag{Brawling} abilities (see \pcref{Special Combat Abilities}).

    \ff[6]{Large Grappler} You are considered one size category larger than normal for the purpose of the \textit{grapple} and \textit{maintain grapple} abilities.

    \ff[12]{Unarmed Warrior+} The damage from your \textit{unarmed warrior} ability increases to 1d6.

    \ff[12]{Brawling Expertise+} The bonus from your \textit{brawling expertise} ability increases to \plus4.

    \ff[18]{Grapple Supremacy} When you grapple a target with the \textit{grapple} ability, you automatically take control of the grapple (see \pcref{Controlling a Grapple}).
  \end{feat}

  \begin{magicalfeat}{Celestial Ancestry}{Ancestry, Magical}
    \featpre Non-evil alignment.
    \parhead{Special} You can only have one Ancestry feat.

    \magicalff[1]{Celestial Benevolence} Whenever an adjacent \glossterm{ally} is attacked by a \glossterm{targeted} ability, the attack is made against you instead.
    If that ability would already target you, it affects you twice, rolling separately for each attack.
    You can suppress or resume this ability as a \glossterm{free action} with the \abilitytag{Swift} ability tag.

    \magicalff[6]{Healing Radiance} When you use the \ability{recover} ability, you \glossterm{briefly} emit \glossterm{brilliant illumination} in a \largearea radius \glossterm{zone} from you.
    % TODO: unclear scaling
    At the end of each round, each \glossterm{ally} in the radius of brilliant illumination regains \glossterm{hit points} equal to half the hit points you regained with that \ability{recover} ability.

    \magicalff[12]{Angel Wings} You gain feathery wings that sprout from your back.
    These wings grant you a \glossterm{fly speed} equal to the \glossterm{base speed} for your size with a maximum height of 15 feet (see \pcref{Flight}).
    As a \glossterm{free action}, you can increase your \glossterm{fatigue level} by one to ignore this height limit until the end of the round.
    The wings themselves are \glossterm{mundane}, but the ability to fly and glide with them is \magical.

    \magicalff[12]{Celestial Benevolence+} This ability affects all allies within a \medarea radius \glossterm{emanation} from you.

    \magicalff[18]{Angel Wings+} The height limit increases to 30 feet.
  \end{magicalfeat}

  \begin{feat}{Chameleon}{General}
    \featpre Disguise as a trained skill, Intelligence 2.

    \ff[1]{Adaptive Archetype} Choose one archetype that you currently have, and two archetypes you do not have from among any of your classes.
    You cannot choose an archetype that you have which is a prerequisite for another archetype that you have.
    Whenever you finish a \glossterm{long rest}, you can choose which one of those three archetypes you actually have access to.
    You gain all benefits of your chosen archetype, and temporarily lose all benefits from the archetypes you did not choose in this way.

    You must track which choices you made for archetypes that you lose access to in this way, such as which spells and maneuvers you learned.
    When you regain access to that archetype, you must make the same choices.

    \ff[6]{Adaptive Specialty} Whenever you finish a \glossterm{short rest}, you may choose an effect from the list below.
    Each effect lasts until you finish a short rest.
    \begin{itemize}
      \item Martial: You become proficient with an additional \glossterm{weapon group} and an additional armor \glossterm{usage class} of your choice.
        You must be proficient with light armor to become proficient with medium armor, and you must be proficient with medium armor to become proficient with heavy armor.
      \item Mystic \sparkle: You gain the ability to use wands as if you were able to cast arcane spells.
        Your maximum spell rank for this purpose is 2.
        This spell rank increases by 1 at 9th level, and every 3 levels thereafter.
      \item Primal: You gain a \plus2 bonus to your \glossterm{fatigue tolerance}.
      \item Skilled: You gain a \plus1 bonus to all skills.
    \end{itemize}

    \ff[12]{Instant Adaptation} As a \glossterm{minor action}, you can change your choice of \textit{adaptive archetype} and \textit{adaptive specialty}.
    When you do, you increase your \glossterm{fatigue level} by two.

    \ff[18]{Adaptive Archetype+} Instead of choosing a single archetype to activate with your \textit{adaptive archetype}, you may choose a blend of two archetypes simultaneously.
    First, choose two archetypes to combine.
    For each rank you have access to, you choose one archetype and gain all abilities of that rank from that archetype and no abilities of that rank from the other archetype.
    You cannot choose abilities from an archetype that reference or improve abilities from that same archetype which you do not have.
    For example, you cannot choose the \textit{wildspell+} ability unless you also have the \textit{wildspell} ability.

    Whenever you finish a long rest, you may choose up to three archetype blends.
    You can use your \textit{instant adaptation} ability to switch between these archetype blends, but you cannot use it to create entirely new archetype blends.

    \ff[18]{Adaptive Specialty+} The effects of your \textit{adaptive specialty} ability improve, as described below.
    \begin{itemize}
      \item Martial: You gain a \plus1 bonus to your Armor defense.
      \item Mystic: You gain a \plus2 bonus to your Mental defense.
      \item Primal: The fatigue tolerance bonus increases to \plus4.
      \item Skilled: The skill bonus increases to \plus2.
    \end{itemize}
  \end{feat}

  \begin{feat}{Climb Specialization}{Skill}
    \featpre Climb as a trained skill.

    \ff[1]{Specialization} You gain a \plus3 bonus to the Climb skill.

    \ff[6]{Climb Speed} You gain a \glossterm{climb speed} 10 feet slower than the \glossterm{base speed} for your size.
    If you already have a climb speed, you gain a \plus10 foot bonus to your climb speed.
    A successful Climb check to move allows you to travel a distance equal to your climb speed.

    \ff[12]{Specialization+} The bonus from your \textit{specialization} ability increases to \plus6.

    \ff[18]{Rapid Climb} You gain a \plus10 foot bonus to your climb speed.
  \end{feat}

  \begin{feat}{Combat Style Versatility}{Combat}
    \featpre Intelligence 2, access to at least one \glossterm{combat style}.

    \ff[1]{Combat Styles} You gain access to all \glossterm{combat styles}.

    \ff[6]{Maneuvers} You learn two additional \glossterm{maneuvers}.
    When you gain access to new maneuver ranks, you can change which maneuvers you know.

    \ff[12]{Maneuver Fusion}
    \begin{activeability}{Maneuver Fusion}
      \abilityusagetime Standard action.
      \abilitycost You are unable to take any \glossterm{standard actions} during the \glossterm{action phase} of the following round.
      \rankline
      Choose two maneuvers that you know.
      You use both maneuvers simultaneously.
      Roll the attack roll and damage for each strike separately.
      You cannot use the \ability{desperate exertion} ability to affect either strike.
    \end{activeability}

    % \ff[18]{Combat Styles+} You can treat all weapons you wield as dealing slashing, piercing, and bludgeoning damage for the purpose of determining their eligibility for maneuvers.
    % This does not change the damage type that your attacks actually deal.

    \ff[18]{Maneuvers+} The number of additional maneuvers you learn increases to four.
  \end{feat}

  \begin{feat}{Craft Specialization}{Skill}
    \featpre Any Craft skill as a trained skill.

    \ff[1]{Specialization} You gain a \plus3 bonus to all Craft skills.

    \magicalff[6]{Craft Magic Item} You can imbue items with magic using your crafting skill.
    There are two ways to craft magic items: by sacrificing valuable raw materials or by salvaging other magic items.
    If you sacrifice valuable raw materials, you must destroy trade goods or gold pieces as if you were buying an item one rank lower than the item you are crafting  (see \pcref{Item Ranks}).
    If you salvage another magic item, you must either destroy a non-consumable magic item that is at least one rank higher than the item you are crafting, or destroy a non-consumable magic item with the exact same effect as the item you are crafting.
    As normal, you can treat five items of one rank as being equivalent to a single item of one rank higher for either of these crafting methods.

    Crafting a magic item in this way normally requires 24 hours of continuous work which may be split between any number of crafting sessions.
    You can make weaker items more quickly.
    The time required to craft magic items is halved for every rank by which your highest rank exceeds the item's rank, to a minimum of 15 minutes.

    You can also mend a broken magic item if it is one that you could make, or transfer a magical property from one item to a nonmagical item.
    If you transfer an item property in this way, the magic item ability must be valid for the new item.
    If you do so, you treat the item as if it were two ranks lower than its actual rank for the purpose of determining the cost and crafting time, to a minimum rank of 0.
    You cannot mend a \glossterm{destroyed} magic item.

    \ff[12]{Specialization+} The bonus from your \textit{specialization} ability increases to \plus6.

    \ff[18]{Craft Legacy Item} You can use your crafting ability to enhance your \glossterm{legacy item} (see \pcref{Legacy Item}).
    This allows you to add one additional magic item property to it.
    The maximum rank for this magic item property is 5, but the crafting costs and difficulty are calculated as if it were a rank 7 magic item property.
    You gain the benefits of this property without attunement, just like your other legacy item properties.
    You can recraft it to change this extra magic item property for the same cost as crafting a new item.
  \end{feat}

  \begin{feat}{Creature Handling Specialization}{Skill}
    \featpre Creature Handling as a trained skill.

    \ff[1]{Specialization} You gain a \plus3 bonus to the Creature Handling skill.

    \ff[6]{Binding Command} You can \glossterm{sustain} the effect of the \textit{command} ability from the Creature Handling skill with a \glossterm{minor action} instead of with a standard action.
    For details, see \pcref{Command}.

    \ff[6]{Efficient Training} You can teach a creature a trick with 12 hours of work, split as you choose, rather than in a week of 4 hour sessions (see \pcref{Training Creatures}).
    In addition, you can train creatures to learn two bonus tricks beyond their normal maximum (see \pcref{Bonus Tricks}).

    \ff[12]{Specialization+} The bonus from your \textit{specialization} ability increases to \plus6.

    \ff[18]{Battleforged Companion} You can teach a creature the Battleforged Companion trick.
    This does not work on creatures that are already significantly enhanced or altered from their natural state, such as a druid's \textit{natural servant} or a ranger's \textit{animal companion}.
    The \glossterm{difficulty value} to train the trick is 20.
    A creature with the trick gains a \plus2 bonus to \glossterm{accuracy} and all \glossterm{defenses}.
  \end{feat}

  \begin{feat}{Deception Specialization}{Skill}
    \featpre Deception as a trained skill.

    \ff[1]{Specialization} You gain a \plus3 bonus to the Deception skill.

    \magicalff[6]{Forked Tongue}
    \begin{sustainability}{Forked Tongue}{\abilitytag{Subtle}, \abilitytag{Sustain} (minor)}
      \abilityusagetime \glossterm{Minor action}.
      \rankline
      Whenever you speak, you can say the same words with two different vocal patterns, such as tone or accent, though you cannot significantly change the volume.
      You can freely choose which creatures hear which pattern, treating all creatures you are not aware of as a single group.

      You can freely choose different vocal patterns each round that you sustain this ability.

      \rankline
      \featlevel{12} You can speak entirely different words with your two voices.
      \featlevel{18} You can also speak with a third voice, using separate words and vocal patterns.
    \end{sustainability}

    \ff[6]{Undetectable Lies} Any \magical abilities which detect lies are unable to detect lies you speak.
    This does not protect you from magical effects which control your actions, such as by compelling you to tell the truth.

    \ff[12]{Specialization+} The bonus from your \textit{specialization} ability increases to \plus6.

    \magicalff[18]{Deceive Reality}
    \begin{magicalactiveability}{Deceive Reality}
      \abilityusagetime Standard action once per \glossterm{long rest}.
      \rankline
      You tell a lie that becomes true.
      The lie must not directly affect any creatures.
      The effects of the lie must be \glossterm{mundane}, so you cannot create artifacts or specific spell effects.
      You cannot affect anything beyond one mile from you, ignoring \glossterm{line of sight} and \glossterm{line of effect}.
      In addition, all effects of this ability end after ten minutes.
      However, within those limits, you can generally alter reality around you.

      Some examples of lies that would be valid to use with this ability are ``I have a bathtub full of diamonds in the other room'', ``Suddenly, all of the torches in the room were extinguished simultaneously'', and ``Don't worry, my basement cellar has always had an escape tunnel for emergencies''.
    \end{magicalactiveability}
  \end{feat}

  \begin{feat}{Devices Specialization}{Skill}
    \featpre Devices as a trained skill.

    \ff[1]{Specialization} You gain a \plus3 bonus to the Devices skill.

    \ff[6]{Rapid Improvisation} It takes you only a standard action to make a device of up to Tiny size with the \textit{improvise} ability (see \pcref{Improvise}).

    \ff[12]{Specialization+} The bonus from your \textit{specialization} ability increases to \plus6.

    \ff[18]{Rapid Improvisation+} It takes you only a standard action to make a device of up to Small size with the \textit{improvise} ability (see \pcref{Improvise}).
  \end{feat}

  \begin{feat}{Disguise Specialization}{Skill}
    \featpre Disguise as a trained skill.

    \ff[1]{Specialization} You gain a \plus3 bonus to the Disguise skill.

    \ff[6]{Versatile Disguise} Whenever you use the \textit{disguise creature} and \textit{emulate creature} abilities on yourself, you may simultaneously create two different disguises.
    This takes twice as long as creating a single disguise.
    You can change your appearance between the two chosen disguises as a \glossterm{minor action}.

    \ff[12]{Specialization+} The bonus from your \textit{specialization} ability increases to \plus6.

    \ff[18]{Quick Change} You can create disguises with the \textit{disguise creature}, \textit{emulate creature}, and \textit{versatile disguise} abilities as a single standard action, regardless of the complexity of the disguise.
  \end{feat}

  \begin{feat}{Draconic Ancestry}{Ancestry}
    \parhead{Special} You can only have one Ancestry feat.

    \ff[1]{Draconic Ancestry} Choose a type of dragon from among the dragons on \trefnp{Dragon Types}.
    You have the blood of that type of dragon in your veins.
    You are \glossterm{impervious} to that dragon's associated tag.

    \ff[1]{Scales} You gain a \plus1 bonus to your Armor defense.

    \ff[1]{Draconic Weapons} You gain a bite natural weapon and two claw natural weapons, one on each arm.
    For details, see \tref{Natural Weapons}.

    % The dragon breath stays at full damage scaling, but uses half area scaling
    % that would normally be appropriate for a d(R-1) spell.
    \ff[6]{Draconic Breath}
    \begin{activeability}{Draconic Breath}
      \abilityusagetime Standard action.
      \abilitycost You \glossterm{briefly} cannot use this ability again.
      \rankline
      This ability's tag matches your dragon's associated tag.
      Make an attack vs. Reflex against everything in the area defined by the type of dragon from your \textit{draconic ancestry} ability (see \trefnp{Dragon Types}).
      \hit \damagerankthree{}.
      \miss Half damage.

      \rankline
      % T2 area
      \featlevel{9} The area increases.
      A line breath weapon becomes a \arealarge, 5 ft.\ wide line.
      A cone breath weapon becomes a \areamed cone.
      \featlevel{12} The damage increases to \damagerankfive{}.
      % T4 area
      \featlevel{15} The area increases.
      A line breath weapon becomes a \areahuge, 10 ft.\ wide line.
      A cone breath weapon becomes a \arealarge cone.
      \featlevel{18} The damage increases to \damagerankseven{}.
      % \featlevel{21} The area increases.
      %     A line breath weapon becomes a \areagarg, 15 ft.\ wide line.
      %     A cone breath weapon becomes a \areahuge cone.
    \end{activeability}

    \ff[12]{Draconic Ancestry+} You become immune to damage of the type dealt by your dragon's breath weapon.

    % TODO: figure out a way to differentiate these from angel wings
    \ff[12]{Draconic Wings} You gain leathery wings that sprout from your back.
    These wings grant you a \glossterm{fly speed} equal to the \glossterm{base speed} for your size with a maximum height of 15 feet (see \pcref{Flight}).
    As a \glossterm{free action}, you can increase your \glossterm{fatigue level} by one to ignore this height limit until the end of the round.
    The wings themselves are \glossterm{mundane}, but the ability to fly and glide with them is \magical.

    \ff[18]{Scales+} The Armor defense bonus from your \textit{draconic scales} ability increases to \plus2.

    \ff[18]{Draconic Wings+} You gain a \plus10 foot bonus to the fly speed.
  \end{feat}

  \begin{dtable}
    \lcaption{Dragon Types}
    \begin{dtabularx}{\columnwidth}{l >{\lcol}X >{\lcol}X}
      \tb{Dragon} & \tb{Tag} & \tb{Breath Weapon} \tableheaderrule
      Black       & \atAcid             & \areamed, 5 ft. wide line \\
      Blue        & \atElectricity      & \areamed, 5 ft. wide line \\
      Brass       & \atFire             & \areamed, 5 ft. wide line \\
      Bronze      & \atElectricity      & \areamed, 5 ft. wide line \\
      Copper      & \atAcid             & \areamed, 5 ft. wide line \\
      Gold        & \atFire             & \areasmall cone           \\
      Green       & \atAcid             & \areasmall cone           \\
      Red         & \atFire             & \areasmall cone           \\
      Silver      & \atCold             & \areasmall cone           \\
      White       & \atCold             & \areasmall cone           \\
    \end{dtabularx}
  \end{dtable}

  \begin{feat}{Duelist}{Combat}
    \featpre Dexterity 2, Intelligence 1.

    \ff[1]{Duel Focus} At the start of each round, you may choose a creature you can see.
    During that round, you gain a \plus1 bonus to your defenses against that creature's attacks.

    \ff[6]{Duelist Strike}
    \begin{activeability}{Duelist Strike}
      \abilityusagetime Standard action.
      \rankline
      Make a melee \glossterm{strike} with a non-\weapontag{Heavy} weapon.
      This strikes only targets a single creature, even if your weapon would normally have the Sweeping tag.
      If you are the creature's only \glossterm{enemy} adjacent to it, you gain a \plus2 accuracy bonus with the strike.
      If that creature is not adjacent to any of its \glossterm{allies}, you gain an additional \plus2 accuracy bonus.

      \rankline
      \featlevel{9} If either accuracy bonus applies, the strike deals 1d4 \glossterm{extra damage}.
      \featlevel{12} The extra damage increases to 1d8.
      \featlevel{15} The extra damage increases to 2d8.
      \featlevel{18} The strike always deals double \glossterm{weapon damage}.
      % \featlevel{21} The extra damage increases to 4d8.
    \end{activeability}

    \ff[12]{Defensive Stance} You gain a \plus1 bonus to your Armor defense as long as you wield a non-\weapontag{Projectile} weapon.
    This bonus is doubled if you are not using a shield.

    \ff[18]{Duel Focus+} The bonuses from your \textit{duel focus} ability increase to \plus2.
  \end{feat}

  \begin{feat}{Endurance Specialization}{Skill}
    \featpre Endurance as a trained skill.

    \ff[1]{Specialization} You gain a \plus3 bonus to the Endurance skill.

    \ff[6]{Delay Condition} Whenever you gain a \glossterm{condition}, you can make an Endurance check.
    The \glossterm{difficulty value} starts at 10 and increases by 5 in each subsequent round.
    Success means that you do not suffer the effects of the condition.
    You must repeat this check at the end of each subsequent round to continue to delay the effects of the condition.
    Failure means that the condition has its normal effect on you.

    You can only delay one of your conditions in this way.
    If you gain a new condition, you can choose to either delay the new condition or continue delaying the old condition.
    If some other effect would delay the effects of the condition, such as a fighter's \ability{disciplined reaction} ability, you do not have to start Endurance checks with this ability until after that delay ends.

    \ff[12]{Specialization+} The bonus from your \textit{specialization} ability increases to \plus6.

    \ff[18]{Ignore Vital Wound} Whenever you gain a \glossterm{vital wound}, you can choose to ignore its effect after making the \glossterm{vital roll}.
    It still penalizes your future \glossterm{vital rolls}.
    You can only ignore one vital wound in this way at a time.
    If you ignore a new vital wound, you suffer the full effects of the old vital wound.
  \end{feat}

  \begin{magicalfeat}{Entropist}{General, Magical}
    \featpre Willpower 2.

    \magicalff[1]{Sudden Entropy}
    \begin{magicalactiveability}{Sudden Entropy}
      \abilityusagetime Standard action.
      \rankline
      Make an attack vs. Fortitude against one creature or object within \shortrange.
      If the target is \vulnerable to any ability tags, there is a 50\% chance that this ability has that tag.
      \hit \damageranktwo{}.
      \miss 50\% of the time, you still deal half damage.

      \rankline
      \featlevel{3} The base damage increases to 1d8.
      \featlevel{6} The damage bonus from your \glossterm{power} increases to 1d6 per 3 power.
      \featlevel{9} The damage bonus increases to 1d6 per 2 power.
      \featlevel{12} The damage bonus increases to 1d8 per 2 power.
      \featlevel{15} The damage bonus increases to 1d10 per 2 power.
      \featlevel{18} The damage bonus increases to 1d6 per power.
    \end{magicalactiveability}

    % TODO once people stop paying attention:
    % \magicalff[6]{Friend of Chaos} Whenever you roll to determine a random effect, such as a \glossterm{miss chance} or a sorcerer's \textit{wild magic} ability, you may reroll once.
    % If you do, you must keep the second result.
    \magicalff[6]{Friend of Chaos} Whenever you roll to determine a random effect, such as a \glossterm{miss chance} or a sorcerer's \textit{wild magic} ability, you may roll twice and keep whichever result you prefer.
    This includes random effects for that do not explicitly tell you to roll, such as your behavior while \confused.
    However, \glossterm{vital rolls} and your \ability{sudden entropy} ability are unaffected by this ability.

    \magicalff[12]{Entropic Defense} Whenever you are hit by a \glossterm{critical hit}, there is a 50\% chance that the attack is treated as a regular hit against you instead of a critical hit.
    This is not affected by your \textit{friend of chaos} ability, since the attacker rolls this chance instead of you.

    \magicalff[18]{Master of Chaos} Whenever you roll to determine a random effect, you can use this ability.
    When you do, you increase your \glossterm{fatigue level} by one, and you may freely choose the random result.
    You can use this ability after seeing what your result would have been.
    \glossterm{Vital rolls} and your \ability{sudden entropy} ability cannot be affected by this ability.
  \end{magicalfeat}

  \begin{feat}{Executioner}{Combat}
    \featpres Perception 2.

    \ff[1]{Marked for Execution} You consider living creatures that either have a \glossterm{vital wound} or have less than their maximum \glossterm{hit points} to be marked for execution.
    Several abilities from this feat affect creatures marked for execution.
    You can automatically identify whether creatures you see are marked for execution.

    \ff[1]{Finishing Blow} You gain a \plus1 accuracy bonus against creatures that are marked for execution.
    This bonus increases to \plus2 if you are adjacent to the creature.

    \ff[6]{Blood Sense} You gain \trait{lifesense} with a 60 foot range, allowing you to sense the location of living creatures without light (see \pcref{Lifesense}).
    Your lifesense functions like \trait{lifesight} against creatures who are marked for execution, allowing you to see them perfectly.
    In addition, you do not need \glossterm{line of sight} or \glossterm{line of effect} to see creatures that are marked for execution with lifesight.
    This allows you to see them through solid objects, but grants you no special ability to attack through solid objects, so your attacks are still affected normally by \glossterm{cover}.

    \ff[12]{Death is Inescapable} Whenever you deal damage to a creature that is marked for execution, it \glossterm{briefly} becomes \slowed.

    \ff[18]{Finishing Blow+} The accuracy bonus increases to \plus2, or to \plus4 if you are adjacent.
  \end{feat}

  \begin{feat}{Fateweaver}{General, Magical}
    \featpre Perception 2.

    \magicalff[1]{Inevitable Attack} Whenever you would make an attack roll, you can instead choose to treat your roll as a 5.
    You must use this ability before rolling.
    If this result does not cause you to hit the target, you roll the attack roll normally.

    % Average of +1.5 accuracy, so worse than a typical "-2 defenses" condition. A typical t1.5 debuff would be r3, requiring 7th level.
    % If this gets a one rank discount for being a feat ability, this is reasonably balanced.
    \magicalff[6]{Fateseal}
    \begin{magicalactiveability}{Fateseal}
      \abilityusagetime Standard action.
      \rankline
      Make an attack vs. Mental against one creature within \medrange of you.
      \hit The target is fatesealed as a \glossterm{condition}.
      Whenever you or one of your \glossterm{allies} make an attack roll against it, if the roll would be less than a 6, treat it as a 6 instead.
      This does not affect extra dice from exploding attacks.
      \crit This condition must be removed an additional time before the effect ends.

      \rankline
      You gain a \plus2 accuracy bonus for every 3 levels beyond 6.
    \end{magicalactiveability}

    \magicalff[12]{Inevitable Success} Whenever you would make a \glossterm{check}, you can instead choose to treat the result of your roll as a 5.
    You must use this ability before rolling.
    If this result does not cause you to succeed at the check, you roll the check normally.

    \magicalff[18]{Fated Victory} The fixed results from your \textit{inevitable attack} and \textit{inevitable success} abilities each increase to 7.
  \end{feat}

  % \begin{feat}{Fatesealed}{General}
  %     \featpre None.

  %     \ff[1]{Certain Doom} You are fated to die at a particular time.
  %     Until that time, you cannot die.
  %     Your minimum \glossterm{vital roll} result is 1.
  %     In addition, circumstances always intervene to make your survival seem plausible.
  %     Monsters may decide they don't like how you taste, intelligent foes may decide you are more useful as a hostage or messenger than dead, and so on.
  %     You may still suffer serious non-lethal consequences for defeat and failure.

  %     Each time you avoid death with this ability, your certain doom draws nearer.
  % \end{feat}

  \begin{feat}{Flexibility Specialization}{Skill}
    \featpre Flexibility as a trained skill.

    \ff[1]{Specialization} You gain a \plus3 bonus to the Flexibility skill.

    \ff[6]{Rapid Escape} You can escape bindings and use the \ability{escape grapple} ability as a \glossterm{movement}, rather than as a standard action.

    \ff[12]{Specialization+} The bonus from your \textit{specialization} ability increases to \plus6.

    \ff[18]{Unnatural Flexibility} You take no penalties while \squeezing.
  \end{feat}

  \begin{magicalfeat}{Ghostblade}{Combat, Magical}
    \featpre Dexterity 1, Willpower 2.

    \magicalff[1]{Ghost Step} When you use the \textit{sprint} ability, you can become \trait{invisible} during that phase (see \pcref{Invisible}, and \pcref{Sprint}).
    This usually makes it impossible for creatures to react to your movement, such as by using the \ability{block} or \ability{follow} abilities (see \pcref{Movement Abilities}).
    This ability has the \abilitytag{Swift} tag, so it affects attacks against you during the current phase.
    After using this ability, you \glossterm{briefly} cannot use it again.

    \magicalff[6]{Spectral Armament} The equipment you choose as your \glossterm{legacy item} becomes ghostly and translucent (see \pcref{Legacy Items}).
    If you chose a weapon, it is treated as cold iron, and attacks with it have the \atCold tag.
    If you chose body armor or a shield, you are \trait{impervious} to \atCold attacks and all attacks by \trait{incorporeal} creatures.

    % as rank 3 maneuver, except for hitting Reflex defense
    \magicalff[6]{Spectral Strike}
    \begin{magicalactiveability}{Spectral Strike}
      \abilityusagetime Standard action.
      \rankline
      Make a melee \glossterm{strike}.
      The attack is made against the target's Reflex defense instead of its Armor defense.
      You use the higher of your \glossterm{magical power} and your \glossterm{mundane power} to determine your damage with this ability (see \pcref{Power}).
      \hit If the target takes damage and your attack result beats its Fortitude defense, it is \slowed as a \glossterm{condition}.

      \rankline
      \featlevel{9} The strike deals 1d6 \glossterm{extra damage}.
      \featlevel{12} The extra damage increases to 1d6 \add 1 per 2 power.
      \featlevel{15} The extra damage increases to 1d6 \add 1 per power.
      \featlevel{18} The \glossterm{weapon damage} is doubled.
    \end{magicalactiveability}

    \magicalff[12]{Ghost Step+} When you use your \textit{ghost step} ability, you can also become \trait{incorporeal} for the duration of the movement.
    This ability has the \abilitytag{Swift} tag, so it grants you the defensive benefits of being incorporeal during the current phase.
    % TODO: wording
    If you choose not to become incorporeal, you do not have to wait \glossterm{briefly} before being able to use your \textit{ghost step} ability again.

    \magicalff[18]{Spectral Armament+} The effect of your \glossterm{legacy item} improves.
    If you chose a weapon, whenever you make a melee \glossterm{strike}, you can make the strike against each target's Reflex defense in place of its Armor defense.
    This has no effect on strikes that are not made against Armor defense.
    If you chose body armor or a shield, you may use your Armor defense in place of your Reflex defense against all attacks.
  \end{magicalfeat}

  \begin{feat}{Greatweapon Warrior}{Combat}
    \featpre Strength 3.

    \ff[1]{Cleave} Whenever you make a \glossterm{melee} \glossterm{strike} with a weapon you use with two hands, it gains the Sweeping (1) tag (see \pcref{Sweeping}).
    In addition, you can choose secondary targets within 15 feet of the primary target instead of the normal 10 feet.
    Each secondary target must still be adjacent to you unless you are using a \weapontag{Long} weapon (see \pcref{Weapon Tags}).
    If the weapon already has the Sweeping tag, you increase the tag's value by 1, allowing you to choose an additional secondary target.

    \ff[6]{Power Attack} Once per round, when you make a \glossterm{melee} strike with a weapon that you use with two hands, you may take a \minus2 accuracy penalty.
    If you do, the strike deals \glossterm{extra damage} equal to your power.
    After using this ability, you \glossterm{briefly} suffer a \minus1 penalty to your Armor and Reflex defenses.

    \ff[12]{Cleave+} You instead gain the Sweeping (2) tag, or you increase the existing tag's value by 2.

    \ff[18]{Power Attack+} The damage from your \textit{power attack} ability increases to 1d6 per 3 \glossterm{power}.
    In addition, the defense penalty is removed.
  \end{feat}

  \begin{feat}{Herbalist}{General}
    \featpre Knowledge (nature) as a trained skill.

    \ff[1]{Esoteric Concoction} You can use your Knowledge (nature) skill in place of Craft (alchemy) or Craft (poison) to create poisons and potions.
    This does not help you create other alchemical items, such as alchemist's fire.
    When you do, you must use esoteric natural ingredients in place of the normal ingredients.
    However, all poisons you create with this ability deteriorate and lose their effectiveness after 8 hours.

    The replacement ingredients must be difficult to acquire in large quantities and impossible to acquire in a normal city.
    For example, you can use the tail of a blind mouse or the dew from a four-leafed clover, but you could not use dirt or ordinary tree bark.
    Once you have determined a purpose for a particular replacement ingredient, you cannot use that ingredient as a replacement in any other poison or potion.

    In general, it requires an hour of work and a Knowledge (nature) check equal to 5 \add twice the rank of the item to find ingredients for an item in this way.
    Each time you find ingredients for an item this way, the time required to find ingredients again increases by an hour and the difficulty value increases by 5.
    Whenever you finish a \glossterm{long rest} or enter a different environment with different ingredients, these penalties reset.

    \ff[6]{Potent Poisons} You gain a \plus1 bonus to \glossterm{accuracy} with any poisons you create, including poisonous spells you cast.

    \ff[6]{Tempting Concoction}
    \begin{magicalattuneability}{Tempting Concoction}{\abilitytag{Attune}, \abilitytag{Emotion}, \abilitytag{Subtle}}
      \abilityusagetime Standard action.
      \rankline
      Choose one liquid poison or potion you created, or an object containing one of those liquids, within \rngshort range.
      Whenever an \glossterm{enemy} notices the chosen object, make a \glossterm{reactive attack} vs. Mental against it.
      You are not aware of this attack, so you cannot use abilities like \ability{desperate exertion} to affect it.
      If the poison or potion is not concealed inside a less suspicious object, such as a tankard of ale or an apple, you take a \minus4 penalty to \glossterm{accuracy}.
      You cannot make this attack more than once against any individual target during this ability's duration.
      \hit The target is filled with the desire to investigate and try to consume the liquid or the object containing the liquid.
      It will not generally interrupt combat or wander into obvious danger to fulfill its desire, but individual creatures may react more or less strongly.
      This effect lasts until the target consumes the object or until it finishes a \glossterm{short rest}.

      \rankline
      You gain a \plus2 accuracy bonus with the attack for every 3 levels beyond 6.
    \end{magicalattuneability}

    \ff[12]{Esoteric Concoction+} The time required to find ingredients with your \textit{esoteric concoction} ability is halved.
    In addition, the \glossterm{difficulty value} of the Knowledge check is reduced by 10.

    \ff[18]{Potent Poisons+} The bonus from your \textit{potent poisons} ability increases to \plus3.
  \end{feat}

  \begin{feat}{Intimidate Specialization}{Skill}
    \featpre Intimidate as a trained skill.

    \ff[1]{Specialization} You gain a \plus3 bonus to the Intimidate skill.

    % r3 debuff is allowed to inflict frightened as a condition
    \ff[6]{Greater Demoralize} When you use the \textit{demoralize} ability, the target is frightened by you even if it has damage resistance remaining.
    For details, see \pcref{Demoralize}.

    \ff[12]{Specialization+} The bonus from your \textit{specialization} ability increases to \plus6.

    \ff[18]{Threatening Presence} Creatures that are frightened or panicked by you suffer a penalty to their Armor, Fortitude, and Reflex defenses equal to the penalty they suffer to their Mental defense.
  \end{feat}

  \begin{feat}{Iron Will}{General}
    \featpre Willpower 2.

    \ff[1]{Mental Discipline} You gain a \plus2 bonus to your Mental defense, and you are \impervious to psychic damage.
    In addition, you automatically detect any \abilitytag{Compulsion} or \abilitytag{Emotion} effects on you which are \abilitytag{Subtle}.
    This does not make you immune to those effects, but it may help you control your reactions.

    \ff[6]{Unclouded Mind} You are immune to being \stunned and \confused.

    \ff[12]{Mental Discipline+} The defense bonus from your \textit{mental discipline} ability increases to \plus4.
    In addition, you become immune to psychic damage.

    \ff[18]{Unclouded Mind+} You are immune to all \abilitytag{Compulsion} and \abilitytag{Emotion} attacks.
  \end{feat}

  \begin{feat}{Juggernaut}{Combat}
    \featpre Strength 1, Constitution 2.

    \ff[1]{Unstoppable} You are immune to being \slowed, \immobilized, and \paralyzed.
    In addition, you are unaffected by \glossterm{difficult terrain}.

    \ff[6]{Brute Force} You gain a \plus1 bonus to your \glossterm{mundane power}.

    \ff[6]{Trample}
    \begin{activeability}{Trample}[\abilitytag{Brawling}, \abilitytag{Size-Based}]
      \abilityusagetime Standard action.
      \rankline
      This ability functions like the \textit{overrun} ability, except that creatures may not choose to avoid you.
      In addition, the attack deals \damageranktwo{}.
      % TODO: wording
      Creatures only take this damage if you move into their space during your movement with this ability.
      On a miss or glance, creatures still take half damage if you are somehow able to move into their space despite failing to overrun them.

      \rankline
      \featlevel{9} The damage bonus from your \glossterm{power} increases to 1d6 per 3 power.
      \featlevel{12} The damage bonus increases to 1d6 per 2 power.
      \featlevel{15} The damage bonus increases to 1d8 per 2 power.
      \featlevel{18} The damage bonus increases to 1d6 per power.
    \end{activeability}

    \ff[12]{Unblockable} Your attacks deal double damage to walls, doors, and similar vertical obstructions that block passage.
    This does not affect your damage to other objects.
    Whenever you touch a wall created by a \abilitytag{Barrier} ability, it is immediately destroyed unless its \glossterm{rank} exceeds your maximum rank.

    \ff[18]{Brute Force+} The mundane power bonus increases to \plus2.

    \ff[18]{Unstoppable+} You are immune to \glossterm{teleport}, \glossterm{knockback}, \glossterm{push} effects from attacks.
    This does not affect movement effects used by your \glossterm{allies}.
    You are also immune to being \grappled unless you initiate the grapple.
  \end{feat}

  \begin{feat}{Jump Specialization}{Skill}
    \featpre Jump as a trained skill.

    \ff[1]{Specialization} You gain a \plus3 bonus to the Jump skill.

    \ff[6]{Featherlight Leap} Your maximum jumping height is equal to your maximum horizontal jump distance, rather than half that distance (see \pcref{Jumping}).
    If you have the \ability{float like air} monk ability with the same effect, you instead gain a \plus5 foot bonus to your maximum jump height.
    This does not affect the forward distance you can reach with your jumps.

    \ff[12]{Specialization+} The bonus from your \textit{specialization} ability increases to \plus6.

    \ff[18]{Featherlight Leap+} Your maximum jumping height increases to twice your maximum horizontal jump distance.
    This does not affect the forward distance you can reach with your jumps.
    If you have the \ability{float like air+} monk ability with the same effect, you instead increase the bonus from your \textit{featherlight leap} ability to \plus15 feet.
  \end{feat}

  \begin{feat}{Knowledge Specialization}{Skill}
    \featpre Any Knowledge skill as a trained skill.

    \ff[1]{Specialization} You gain a \plus3 bonus to all Knowledge skills.

    \ff[6]{Studied Defense} You gain \plus2 bonus to your choice of your Fortitude, Reflex, or Mental defense.
    Whenever you finish a \glossterm{short rest}, you can change the defense this bonus applies to.

    \ff[12]{Specialization+} The bonus from your \textit{specialization} ability increases to \plus6.

    \ff[18]{Studied Defense+} The bonus increases to \plus4.
  \end{feat}

  \begin{feat}{Leadership}{Combat}
    \featpre Either Intelligence 2 or Willpower 2.

    \ff[1]{Battle Command}
    \begin{activeability}{Battle Command}
      \abilityusagetime Standard action.
      \rankline
      Choose an \glossterm{ally} within \rngmed range.
      During the current phase, the target gains a \plus2 bonus to \glossterm{accuracy} and rolls twice for any attacks it makes, keeping the better result.
    \end{activeability}

    \ff[6]{Encouraging Presence} As long as you are conscious, your \glossterm{allies} who can see or hear you are immune to being \frightened and \panicked.

    \ff[12]{Bolster}
    \begin{activeability}{Bolster}[\abilitytag{Emotion}, \abilitytag{Swift}]
      \abilityusagetime Standard action.
      \rankline
      One \glossterm{ally} who can see or hear you gains a \plus2 bonus to \glossterm{vital rolls} and all defenses this round.

      \rankline
      \featlevel{18} The bonus increases to \plus3.
    \end{activeability}

    \ff[18]{Unyielding Presence} As long as you are conscious, your \glossterm{allies} who can see or hear gain a \plus2 bonus to their \glossterm{vital rolls}.
  \end{feat}

  \begin{feat}{Maneuverist}{Combat}
    \featpre Intelligence 2.

    \ff[1]{Maneuver Access} You gain access to one \glossterm{combat style} that you did not already have access to (see \pcref{Combat Styles}).
    In addition, you learn one rank 1 \glossterm{maneuver} from that combat style.
    You may spend \glossterm{insight points} to learn one additional maneuver from that combat style per insight point.

    Maneuvers granted by this feat are not martial maneuvers.
    They cannot be improved with the \textit{enhanced maneuvers} ability from fighter or any other class.

    \ff[6]{Maneuver Expertise} You gain a \plus1 accuracy bonus with your rank 1 maneuvers from this feat.
    This bonus increases by 1 at 9th level, and every 3 levels thereafter.

    \ff[9]{Maneuver Rank} You become a rank 3 combat style user.
    This gives you access to maneuvers that require a minimum rank of 3.

    \ff[12]{Maneuver Expertise+} You gain a \plus1 accuracy bonus with your rank 3 maneuvers from this feat.
    This bonus increases by 1 at 15th level, and every 3 levels thereafter.

    \ff[12]{Maneuver Knowledge} You learn one maneuver.

    \ff[15]{Maneuver Rank} You become a rank 5 combat style user.
    This gives you access to maneuvers that require a minimum rank of 5.

    \ff[18]{Maneuver Expertise+} You gain a \plus1 accuracy bonus with your rank 5 maneuvers from this feat.
    This bonus increases by 1 at 21st level.

    \ff[18]{Maneuver Knowledge+} You learn one maneuver.

    \ff[21]{Maneuver Rank} You become a rank 7 combat style user.
    This gives you access to maneuvers that require a minimum rank of 7.
  \end{feat}

  \begin{feat}{Martial Training}{Combat}
    \ff[1]{Equipment Training} You choose one of the following benefits.
    Unless otherwise noted in its description, you can only choose each benefit once.
    \begin{itemize}
      \item You gain proficiency with a \glossterm{usage class} of \glossterm{armor} (light, medium, or heavy).
        You can choose this ability multiple times, choosing a different usage class each time.
        Gaining proficiency with medium armor requires proficiency with light armor, and gaining proficiency with heavy armor requires proficiency with medium armor.
      \item You gain proficiency with an additional \glossterm{weapon group} of your choice.
        You can choose this ability multiple times, choosing a different weapon group each time.
      \item You gain proficiency with \glossterm{exotic weapons} from a weapon group of your choice that you are already proficient with.
        You can choose this ability multiple times, choosing a different weapon group each time.
      \item You reduce the \glossterm{encumbrance} of \glossterm{body armor} you wear by 1.
        You can choose this ability multiple times, and its effects stack.
      \item You gain a \plus1 bonus to your \glossterm{accuracy} with \glossterm{strikes}.
      \item You gain a \plus1 bonus to your \glossterm{mundane power}.
    \end{itemize}

    \ff[6]{Trained Strike}
    \begin{activeability}{Trained Strike}
      \abilityusagetime Standard action.
      \rankline
      Make a \glossterm{strike} and choose either precision or power.
      If you choose precision, you gain a \plus2 accuracy bonus.
      If you choose power, the strike deals 1d4 \glossterm{extra damage}.

      \rankline
      \featlevel{9} \plus4 accuracy or 1d8 extra damage.
      \featlevel{12} \plus6 accuracy or 2d8 extra damage.
      \featlevel{15} \plus2 accuracy and double \glossterm{weapon damage} or 3d10 extra damage.
      \featlevel{18} \plus4 accuracy and double \glossterm{weapon damage} or 5d10 extra damage.
      % \featlevel{21} \plus6 accuracy and double \glossterm{weapon damage} or 7d10 extra damage.
    \end{activeability}

    \ff[12]{Equipment Training+} You gain two additional \textit{equipment training} abilities of your choice.

    \ff[18]{Equipment Training+} You gain two additional \textit{equipment training} abilities of your choice.
  \end{feat}

  \begin{feat}{Medicine Specialization}{Skill}
    \featpre Medicine as a trained skill.

    \ff[1]{Specialization} You gain a \plus3 bonus to the Medicine skill.

    % rank 3 equivalent, so dr5 healing
    \ff[6]{Healing Touch}
    \begin{activeability}{Healing Touch}[\abilitytag{Swift}]
      \abilityusagetime Standard action.
      \abilitycost One \glossterm{fatigue level} from the target.
      \rankline
      Choose yourself or a living \glossterm{ally} you \glossterm{touch}.
      The target regains 6d6 \glossterm{hit points}.
      In addition, make a Medicine check.
      For each poison and disease on the target, if your check result beats a \glossterm{difficulty value} equal to 10 \add twice its \glossterm{rank}, the effect is removed.

      \rankline
      \featlevel{9} The healing increases to 5d10.
      \featlevel{12} The healing increases to 7d10.
      \featlevel{15} The healing increases to 10d10.
      \featlevel{18} The healing increases to 14d10.
    \end{activeability}

    \ff[12]{Specialization+} The bonus from your \textit{specialization} ability increases to \plus6.

    \ff[18]{Vital Touch} When you use your \textit{healing touch} ability, the target can also remove one \glossterm{vital wound}.
    If it does, its increases its \glossterm{fatigue level} by three.
  \end{feat}

  \begin{magicalfeat}{Mental Magic}{Casting, Magical}
    \featpre Spellcasting ability, Willpower 3.

    \magicalff[1]{Mental Casting} You connect to the magical essence of the universe differently from other spellcasters, allowing you to cast spells with purely mental effort.
    None of your spells have \glossterm{somatic components} or \glossterm{verbal components}.
    However, you cannot use magic implements such as staffs to affect spells you cast.

    \magicalff[6]{Fractured Mind} Once per round, you can sustain an ability with the \abilitytag{Sustain} (minor) tag as a \glossterm{free action}.

    \magicalff[12]{Potent Mind} You gain a \plus1 bonus to your \glossterm{magical power}.

    \magicalff[18]{Fractured Mind+} You can also use your \textit{fractured mind} ability to sustain abilities with the \abilitytag{Sustain} (standard) tag.
    Whenever you sustain a Sustain (standard) ability in this way, you increase your \glossterm{fatigue level} by one.

    \magicalff[18]{Potent Mind+} The magical power bonus from your \textit{potent mind} ability increases to \plus2.
  \end{magicalfeat}

  \begin{magicalfeat}{Metacaster}{Casting, Magical}
    \featpre \ability{Metamagic} ability, Intelligence 2.

    \magicalff[1]{Sphere Access} You gain access to an additional \glossterm{mystic sphere}.
    Each \glossterm{mystic sphere} has a set of \glossterm{spells} associated with it.
    You automatically learn all \glossterm{cantrips} from any mystic sphere you have access to.
    % TODO: wording
    If you have multiple \glossterm{magic sources}, you can cast spells from that sphere with any magic source that the mystic sphere belongs to.
    w
    \magicalff[6]{Adaptive Metamagic} Choose one spell you know.
    Whenever you cast that spell, you may apply one \textit{metamagic} ability of your choice to that spell.
    You can choose any of the metamagic abilities offered by your class which are valid for that spell.
    This does not allow you to exceed the normal maximum of two metamagic abilities on a single spell.

    \magicalff[12]{Spell Fusion}
    \begin{magicalactiveability}{Spell Fusion}
      \abilityusagetime Standard action.
      \abilitycost You are unable to take any \glossterm{standard actions} during the \glossterm{action phase} of the following round.
      \rankline
      Choose two spells that you know which do not have the \abilitytag{Sustain} or \abilitytag{Attune} tags.
      You cast both spells simultaneously.
      Both spells that you fuse in this way must have the same area shape, such as a cone or sphere, and targeting restrictions, such as affecting only enemies or living creatures.
      If one spell affects a strictly larger area or a strictly larger number of targets than the other, you must use the smaller of the two areas or target counts.
      You must choose the same targets and area for both spells, if applicable.
      Roll the attack roll and damage for each spell separately.
      You cannot use the \ability{desperate exertion} ability to affect either spell.
    \end{magicalactiveability}

    \magicalff[18]{Adaptive Metamagic+} You can choose an additional spell you know with your \textit{adaptive metamagic} ability.
    In addition, you can apply two metamagic abilities to your chosen spells, rather than only one.
  \end{magicalfeat}

  \begin{magicalfeat}{Mystic Archer}{Casting, Magical}
    \featpre Access to a \glossterm{mystic sphere}.

    \magicalff[1]{Magical Projectiles} Whenever you make a ranged \glossterm{strike} with a \weapontag{Projectile} weapon, you can choose to treat that as a \magical ability.
    When you do, you use your \glossterm{magical power} to determine your damage instead of your \glossterm{mundane power} (see \pcref{Power}).

    \magicalff[1]{Imbue Projectile}
    \begin{magicalattuneability}{Imbue Projectile}{\abilitytag{Attune}}
      \abilityusagetime Standard action.
      \rankline
      Choose a spell you know that does not have the \abilitytag{Attune} or \abilitytag{Sustain} tags.
      The spell's power is imbued in a \glossterm{projectile} you hold.
      Unlike most \abilitytag{Attune} abilities, you can attune to this ability any number of times, choosing a different spell and projectile each time.
      An individual projectile can only be imbued with this ability once, even if multiple creatures use this ability on the same projectile.

      When you make a \glossterm{strike} with that projectile, the spell takes effect on one primary target of the strike.
      Area spells use the corner of the target's square that is closest to you as their \glossterm{point of origin}.
      If the spell has a direction, such as a cone or line, it faces directly away from you.

      You must make any attack rolls required for the spell separately from your attack roll with the strike.
      The spell takes effect even if the strike misses the target's defenses.
      However, the spell does not take effect if the strike misses or fails for some other reason, such as \glossterm{cover} or \glossterm{concealment}.

      After the spell takes effect, your attunement to this ability ends, and you \glossterm{briefly} cannot trigger any more spells you imbued this ability.
      If you fire multiple projectiles imbued with this ability at once, one of the spells takes effect and the others do not.
    \end{magicalattuneability}

    \magicalff[6]{Guided Projectiles} Your ranged \glossterm{strikes} with \weapontag{Projectile} weapons ignore all \glossterm{miss chances}, such as from \glossterm{concealment}.
    This does not prevent your attacks from failing for other reasons.

    \magicalff[12]{Imbue Projectile+} Each time you attune to your \textit{imbue projectile} ability, you can imbue two different projectiles with two different spells.
    Your attunement to the ability ends when either spell takes effect.

    \magicalff[18]{Phasing Projectiles} Your ranged \glossterm{strikes} with \weapontag{Projectile} weapons ignore \glossterm{cover}.
    In addition, you can ignore all physical obstacles in single one-foot span.
    This can allow you to fire projectiles through creatures or solid walls, though it does not grant you the ability to see through a wall.
  \end{magicalfeat}

  \begin{feat}{Null}{General}
    \featpre Willpower 2.

    \ff[1]{Mundane Resilience} You gain a bonus to your \glossterm{hit points} and \glossterm{damage resistance} equal to your level.

    \ff[1]{Nullify Magic} You gain a \plus4 bonus to your \glossterm{defenses} against \magical abilities.
    In exchange, you lose the benefits of all magical abilities you possess.
    You are unable to \glossterm{attune} to any magical abilities, such as magic items or spells cast by other creatures.
    You cannot use \glossterm{potions} or other similar magic items or abilities that affect you personally, even if they do not require attunement.
    In addition, you are never considered an \glossterm{ally} for a magical ability, even while \unconscious.

    \ff[6]{Sever Magic}
    \begin{activeability}{Sever Magic}
      \abilityusagetime Standard action.
      \rankline
      Make a \glossterm{strike}.
      If the target takes damage from the strike, it stops being \glossterm{attuned} to one effect of its choice that it is currently attuned to.
      If it has any magical abilities, but has no remaining attuned effects, it becomes \stunned as a \glossterm{condition} instead.
      On a \glossterm{critical hit}, the strike deals double damage and the target stops being attuned to two abilities of its choice that it is currently attuned to.
      In addition, as a \glossterm{condition}, it stops being able to attune to abilities.

      \rankline
      You gain a \plus2 accuracy bonus with the strike for every 3 levels beyond 6.
    \end{activeability}

    % TODO: does this really need to grant fatigue tolerance?
    \ff[6]{Personal Legacy} You do not choose a legacy item, and you do not gain any legacy item upgrades (see \pcref{Legacy Items}).
    Instead, you gain a \plus1 bonus to your \glossterm{accuracy}, your \glossterm{fatigue tolerance}, and all \glossterm{defenses}.
    These bonuses increase to \plus2 at level 12 and to \plus3 at level 18.

    \ff[12]{Disruptive Presence} Whenever an \glossterm{enemy} within an \medarea radius from you uses a \magical ability as a standard action, the ability has a 50\% chance to fail with no effect.
    This cannot cause passive or triggered abilities to fail.

    \ff[18]{Mundane Resilience+} The bonuses from your \textit{mundane resilience} ability increase to twice your level.

    \ff[18]{True Null} You are unaffected by all \magical abilities.
  \end{feat}

  \begin{feat}{Perform Specialization}{Skill}
    \featpre Any Perform skill as a trained skill.

    \ff[1]{Specialization} You gain a \plus3 bonus to all Perform skills.

    \ff[6]{Synergistic Performance} You can use any Perform skill you have as a \glossterm{trained skill} in place of other related skills.
    Each Perform skill has an associated skill that it can be used to replace, as listed below.
    When you replace a skill in this way, you add half your modifier with the Perform skill instead of your full modifier since the two skills do not exactly match.
    \begin{itemize}
      \item Acting: Deception
      \item Comedy: Deception
      \item Dance: Balance
      \item Keyboard instruments: Devices
      \item Oratory: Persuasion
      \item Percussion instruments: Creature Handling
      \item Singing: Persuasion
      \item String instruments: Devices
      \item Wind instruments: Creature Handling
    \end{itemize}

    \ff[12]{Specialization+} The bonus from your \textit{specialization} ability increases to \plus6.

    \magicalff[18]{Synergistic Performance+} You add your full Perform modifier instead of half your modifier.
  \end{feat}

  \begin{feat}{Persuasion Specialization}{Skill}
    \featpre Persuasion as a trained skill.

    \ff[1]{Specialization} You gain a \plus3 bonus to the Persuasion skill.

    \ff[6]{First Impressions} When you first meet creatures, you have an Ally relationship instead of a Just Met relationship (see \tref{Relationship Modifiers}.
    This does not affect your relationship with creatures who would not normally have a Just Met relationship with you.

    \ff[12]{Specialization+} The bonus from your \textit{specialization} ability increases to \plus6.

    \magicalff[18]{Suggestion}
    \begin{sustainability}{Suggestion}{\abilitytag{Emotion}, \abilitytag{Subtle}, \abilitytag{Sustain} (minor)}
      \abilityusagetime Standard action.
      \rankline
      Make an attack vs. Mental against a creature within \rngmed range.
      Your \glossterm{accuracy} is equal to your Persuasion skill.
      You must also make a verbal suggestion of a particular course of action to the target.
      If your suggestion does not seem reasonable, you take a \minus10 accuracy penalty on this attack.
      Exceptionally unreasonable suggestions can impose even greater penalties, and exceptionally reasonable suggestions can give accuracy bonuses.
      Whether you hit or miss, you cannot attack the target with this ability again until it finishes a \glossterm{short rest}.

      \hit As a \glossterm{condition}, the target thinks your suggestion is a good idea and will try to follow it to the best of its abilities.
      The effect ends once the target thinks it has completed the suggestion, or after ten minutes have passed.
      Any act by you or by creatures that appear to be your ally that damages the target or makes it feel that it is in danger breaks the effect.
      An observant target may interpret overt threats to its \glossterm{allies} as a threat to itself.
    \end{sustainability}
  \end{feat}

  \begin{feat}{Precognition}{General}
    \featpre Intelligence 3.

    \ff[1]{Foresight} You gain a \plus2 bonus to Awareness checks.
    In addition, you are never fully \unaware, though you can still be \partiallyunaware.

    \ff[1]{Precognitive Offense} You can add your Intelligence to your \glossterm{mundane power} instead of your Strength (see \pcref{Power}).

    \ff[6]{Combat Prediction}
    \begin{sustainability}{Combat Prediction}{\abilitytag{Subtle}, \abilitytag{Sustain} (minor)}
      \abilityusagetime Standard action.
      \rankline
      Choose a creature within \medrange of you.
      That creature's intentions become obvious to you as long as you sustain this ability.
      This gives you a \plus1 bonus to \glossterm{accuracy} and \glossterm{defenses} against that creature.

      In addition, when you choose your action during each phase, you can do so after learning the result of the actions the target took during that phase.
      This includes learning whether its attacks hit, and any obvious effects of those attacks.
      You do not gain any knowledge of actions that have no obvious signs, such as purely mental actions, or actions which you are not observant enough to notice.
      You cannot communicate your understanding of the creature's actions to your allies before they decide their actions.
    \end{sustainability}

    \ff[12]{Mobile Foresight} During the \glossterm{movement phase}, you choose your action after all other creatures have chosen their actions.
    When you choose your action, you have insight into the actions chosen by any creatures within \longrange of you that you can see.
    This insight gives you the same information as the insight from your \textit{combat prediction} ability, except that it only provides information about their actions during the movement phase.
    You choose your actions simultaneously with any other creatures who have a similar ability.

    Knowing another creature's action does not automatically allow you to interrupt that action.
    If you want to interrupt an action, such as by blocking a creature's intended movement, you must make an \glossterm{initiative} check as normal.

    \ff[18]{Combat Prediction+} The bonuses from your \textit{combat prediction} ability increase to \plus2.
  \end{feat}

  \begin{magicalfeat}{Prepared Spellcasting}{Magical, Spell}
    \featpre Access to a \glossterm{mystic sphere}, Intelligence 3.

    \magicalff[1]{Spellbook} Choose up to four spells you do not know from among \glossterm{mystic spheres} you have access to.
    The spells must be of a rank that you know how to cast.
    Whenever you gain access to a new spell rank, you may change the spells in your spellbook for any other spells you can cast.
    You inscribe the knowledge of those spells into a book you carry with you.
    This book is your spellbook.

    Whenever you finish a \glossterm{long rest}, you may choose one of the spells in your spellbook.
    You learn how to cast that spell until you choose a different spell with this ability.

    \magicalff[6]{Ritual Book} You gain the ability to perform rituals to create unique magical effects (see \pcref{Spells and Rituals}).
    In addition, you automatically learn one free ritual of each rank you have access to, including new ranks as you gain access to them.

    \magicalff[12]{Study of Magic} You gain a \plus1 bonus to your \glossterm{magical power}.

    \magicalff[18]{Spellbook+} You can choose up to six spells to be in your spellbook instead of only four.
    In addition, whenever you finish a \glossterm{long rest}, you may choose two spells in your spellbook with your \textit{spellbook} ability instead of one.
    You learn how to cast both spells until you choose a different pair of spells in this way.

    \magicalff[18]{Study of Magic+} The bonus from your \textit{study of magic} ability increases to \plus2.
  \end{magicalfeat}

  \begin{feat}{Rapid Reaction}{General}
    \featpre Dexterity 2.

    \ff[1]{Lightning Reflexes} You gain a \plus2 bonus to your Reflex defense.

    \ff[1]{Sidestep} If you have movement remaining with your \glossterm{land speed} after the \glossterm{movement phase}, you may use that movement during the \glossterm{action phase} as a \glossterm{free action}.
    You cannot carry over more than five feet of movement in this way.

    \ff[6]{Evasive Reaction} You take no damage from \glossterm{glancing blows} or misses caused by abilities that affect an area and attack your Armor or Reflex defense.
    This does not protect you from any non-damaging effects of those abilities, or from abilities that affect multiple specific targets without affecting an area.
    If you have the \textit{evasion} monk or rogue ability with the same effect as this ability, you also gain a \plus2 bonus to your Armor and Reflex defenses against area attacks.

    \ff[12]{Lightning Reflexes+} The bonus from your \textit{lightning reflexes} ability increases to \plus4.

    \ff[12]{Sidestep+} The movement you can carry over with your \textit{sidestep} ability increases to ten feet.

    \ff[18]{Evasive Reaction+} This ability also protects you from area attacks against your Fortitude and Mental defenses.
    If you have the \textit{evasion+} monk or rogue ability with the same effect as this ability, you also gain a \plus2 bonus to your Fortitude and Mental defenses against area attacks.
  \end{feat}

  \begin{feat}{Regenerator}{General}
    \featpre Constitution 3.

    \ff[1]{Regenerative Recovery}
    \begin{activeability}{Regenerative Recovery}[\abilitytag{Swift}]
      \abilityusagetime Standard action.
      \abilitycost One \glossterm{fatigue level}.
      \rankline
      You regain 3d6 \glossterm{hit points}.
      If you gained any \glossterm{vital wounds} during the previous round, your healing with this ability is doubled.

      \rankline
      \featlevel{3} The healing increases to 4d6.
      \featlevel{6} The healing increases to 6d6.
      \featlevel{9} The healing increases to 5d10.
      \featlevel{12} The healing increases to 7d10.
      \featlevel{15} The healing increases to 10d10.
      \featlevel{18} The healing increases to 14d10.
    \end{activeability}

    \ff[6]{Diehard} You gain a \plus3 bonus to \glossterm{vital rolls}.

    \ff[12]{Regenerative Rest} When you take a \glossterm{short rest}, you can remove any number of \glossterm{vital wounds} affecting you.
    If you do, you increase your \glossterm{fatigue level} by three per vital wound removed this way.
    You can use this ability to remove vital wounds even once your fatigue level would already make you unconscious.
    This allows you to recover any number of vital wounds if you go unconscious to do so, regardless of your maximum fatigue level.

    \ff[18]{Battlefield Regeneration} When you use the \textit{recover} action, you can also remove a single vital wound.
    If you do, you increase your fatigue level by one.

    \ff[18]{Regenerative Rest+} You only increase your fatigue by two per vital wound.
  \end{feat}

  \begin{feat}{Ride Specialization}{Skill}
    \featpre Ride as a trained skill.

    \ff[1]{Specialization} You gain a \plus3 bonus to the Ride skill.

    \ff[6]{Mounted Defense} Your mount gains a \plus3 bonus to all defenses, up to a maximum of your own corresponding defense.

    \ff[6]{Mounted Warrior} The penalty you take for making \glossterm{strikes} with \weapontag{Projectile} weapons while riding a moving mount is reduced by 2.
    In addition, while you are mounted, you gain a \plus1 \glossterm{accuracy} bonus with \weapontag{Mounted} weapons (see \pcref{Weapon Tags}).

    \ff[12]{Specialization+} The bonus from your \textit{specialization} ability increases to \plus6.

    \ff[18]{Mounted Defense+} The defense bonus from your \textit{mounted defense} ability increases to \plus6.

    \ff[18]{Mounted Warrior+} The penalty reduction from your \textit{mounted warrior} ability increases to 4.
    In addition, the accuracy bonus increases to \plus2.
  \end{feat}

  \begin{feat}{Shieldbearer}{Combat}
    \featpre Strength 2.

    \ff[1]{Shield Expertise} You gain a \plus1 bonus to Armor defense while you wield a shield.
    This bonus is doubled for the current round when you take the \ability{total defense} action.

    \ff[6]{Forceful Block} Whenever a creature misses or \glossterm{glances} you with a melee \glossterm{strike}, if you are wielding a shield, that creature \glossterm{briefly} takes a \minus2 penalty to Armor defense.
    As normal, this bonus does not stack with itself, even if the same creature misses you with multiple melee attacks.

    \ff[12]{Arrow Deflection} While you wield a shield, you and each \glossterm{ally} adjacent to you gain a \plus2 bonus to Armor defense against ranged \glossterm{strikes}.
    This bonus is doubled for the current round when you take the \ability{total defense} action.

    \ff[18]{Forceful Block+} The penalty from your \textit{forceful block} ability increases to \minus4.

    \ff[18]{Shield Expertise+} The bonus from your \textit{shield expertise} ability increases to \plus2.
  \end{feat}

  \begin{feat}{Sleight of Hand Specialization}{Skill}
    \featpre Sleight of Hand as a trained skill.

    \ff[1]{Specialization} You gain a \plus3 bonus to the Sleight of Hand skill.

    \ff[6]{Deep Pickpocket} You can use the \textit{pickpocket} ability to retrieve objects that are loose within larger containers, such as backpacks or sacks, even if they are not immediately accessible.
    You must be able to reach at least one of your fingers into the bag, such as through a narrow gap at the opening.
    This does not allow you to retrieve objects from locked containers with no openings.
    The container's size category cannot exceed your own size category.

    \ff[12]{Specialization+} The bonus from your \textit{specialization} ability increases to \plus6.

    \magicalff[18]{Extradimensional Pocket}
    \begin{magicalattuneability}{Extradimensional Pocket}{\abilitytag{Attune}}
      \abilityusagetime Can be triggered when you use the \ability{conceal object} ability (see \pcref{Sleight of Hand}).
      \rankline
      You conceal the object in a pocket dimension that cannot be accessed by nonmagical means.
      When your attunement to this ability ends, the object appears in a free hand.
      If you have no free hands, it drops to the ground.
    \end{magicalattuneability}
  \end{feat}

  \begin{feat}{Sniper}{Combat}
    \featpre Perception 2.

    \ff[1]{Aim}
    \begin{sustainability}{Aim}{\abilitytag{Subtle}, \abilitytag{Sustain} (minor)}
      \abilityusagetime Standard action.
      \rankline
      Choose a creature or object you can see.
      You gain a \plus2 accuracy bonus against the target.

      If you lose sight of the target for a full round, this effect ends.
    \end{sustainability}

    \ff[6]{Sniper's Precision} You gain a \plus1 bonus to \glossterm{accuracy} with ranged \glossterm{strikes}.

    \ff[12]{Steady Shot} If your location did not change since the start of the round, you reduce your \glossterm{longshot penalty} by 2.

    \ff[18]{Sniper's Precision+} The accuracy bonus from your \textit{sniper's precision} ability increases to \plus2.

    \ff[18]{Steady Shot+} The penalty reduction from your \textit{steady shot} ability increases to 4, allow you to ignore any \glossterm{longshot penalty}.
  \end{feat}

  \begin{feat}{Social Insight Specialization}{Skill}
    \featpre Social Insight as a trained skill.

    \ff[1]{Specialization} You gain a \plus3 bonus to the Social Insight skill.

    \magicalff[6]{Read Emotions}
    \begin{sustainability}{Read Emotions}{\abilitytag{Emotion}, \abilitytag{Sustain} (minor), \abilitytag{Subtle}}
      \abilityusagetime Standard action.
      \rankline
      Make an attack vs. Mental against a creature within \rngshort range.
      Your \glossterm{accuracy} is equal to your Social Insight skill.
      Whether you hit or miss, you cannot attack the target with this ability again until it finishes a \glossterm{short rest}.
      \hit You know the target's current emotions.
      In addition to the obvious effects, this grants you a \plus3 bonus to Deception, Persuasion, Intimidate, and Social Insight attacks and checks against the target.
      This bonus does not stack with other effects that allow you access to the target's mind, such as \ability{read mind}.

      \rankline
      \featlevel{12} The range increases to \medrange.
      \featlevel{18} The range increases to \longrange.
    \end{sustainability}

    \ff[12]{Specialization+} The bonus from your \textit{specialization} ability increases to \plus6.

    \ff[18]{Truthsense} Whenever a creature that you can hear and see speaks truth to the best of its knowledge with no attempt at evasion, concealment, or creative wording, you automatically recognize that.
    You do not recognize truth in this way if a creature is using the Deception skill in any way, even if it is speaking the truth.
  \end{feat}

  \begin{magicalfeat}{Spellsword}{Magical, Spell}
    \featpre Access to a \glossterm{mystic sphere}.

    % TODO: it would be nice if spellswords didn't dump Strength
    \magicalff[1]{Magical Strikes} Whenever you make a melee \glossterm{strike}, you can choose to treat it as a \magical ability.
    When you do, you use your \glossterm{magical power} to determine your damage instead of your \glossterm{mundane power} (see \pcref{Power}).

    \ff[1]{Martial Implement} You can cast spells using a non-\weapontag{Projectile} weapon as if it were an implement (see \pcref{Magic Implements}).

    \magicalff[6]{Spellstrike}
    \begin{magicalactiveability}{Spellstrike}
      \abilityusagetime Standard action.
      \abilitycost One \glossterm{fatigue level}, and you \glossterm{briefly} cannot use this ability again.
      \rankline
      You cast a spell and make a melee \glossterm{strike}, in either order.
      Because this is a \magical ability, you use your \glossterm{magical power} to determine your damage with the strike instead of your \glossterm{mundane power} (see \pcref{Power}).
      The spell and the strike must affect completely different targets, with no overlap between their targets or areas (if any).
      You cannot use the \ability{desperate exertion} ability to affect the strike or the spell.

      \rankline
      \featlevel{9} You gain a \plus1 accuracy bonus with the strike.
      \featlevel{12} The accuracy bonus increases to \plus2.
      \featlevel{15} The accuracy bonus increases to \plus4.
      \featlevel{18} The strike deals double \glossterm{weapon damage}.
    \end{magicalactiveability}

    \magicalff[12]{Spellsword Conduit} Whenever you deal damage with a melee \glossterm{strike}, you \glossterm{briefly} gain a \plus2 accuracy bonus with spells against each damaged creature.

    \magicalff[18]{Spellsword Conduit+} The accuracy bonus increases to \plus4.
  \end{magicalfeat}

  \begin{magicalfeat}{Spellwarped}{General, Magical}
    \featpre Willpower 2.

    \magicalff[1]{Mystic Sphere} You gain the ability to use arcane magic.
    You gain access to one arcane \glossterm{mystic sphere}, plus the \sphere{universal} mystic sphere (see \pcref{Arcane Mystic Spheres}).
    Each \glossterm{mystic sphere} has a set of \glossterm{spells} associated with it.
    You automatically learn all \glossterm{cantrips} from the mystic sphere you have access to.

    You require both \glossterm{verbal components} and \glossterm{somatic components} to cast spells from your chosen sphere.
    For details about mystic spheres and casting spells, see \pcref{Spell and Ritual Mechanics}.

    \magicalff[3]{Spell Rank} You become a rank 1 spellcaster in your chosen \glossterm{mystic sphere}.
    You learn one spell from that mystic sphere.
    In addition, you can spend \glossterm{insight points} to learn one additional arcane spell per \glossterm{insight point}.

    When you gain access to a spell rank,
    you can exchange any number of spells you know for other spells,
    including spells of the higher rank.

    \magicalff[6]{Spell Knowledge} You learn one spell from your \glossterm{mystic sphere}.

    \magicalff[6]{Spell Rank} You become a rank 2 spellcaster in your chosen \glossterm{mystic sphere}.
    This gives you access to spells that require a minimum rank of 2.

    \magicalff[9]{Spell Rank} You become a rank 3 spellcaster in your chosen \glossterm{mystic sphere}.
    This gives you access to spells that require a minimum rank of 3 and can improve the effectiveness of your existing spells.

    \magicalff[12]{Spell Knowledge} You learn one spell from your \glossterm{mystic sphere}.

    \magicalff[12]{Spell Rank} You become a rank 4 spellcaster in your chosen \glossterm{mystic sphere}.
    This gives you access to spells that require a minimum rank of 4 and can improve the effectiveness of your existing spells.

    \magicalff[15]{Spell Rank} You become a rank 5 spellcaster in your chosen \glossterm{mystic sphere}.
    This gives you access to spells that require a minimum rank of 5 and can improve the effectiveness of your existing spells.

    \magicalff[18]{Spell Knowledge} You learn one spell from your \glossterm{mystic sphere}.

    \magicalff[18]{Spell Rank} You become a rank 6 spellcaster in your chosen \glossterm{mystic sphere}.
    This gives you access to spells that require a minimum rank of 6 and can improve the effectiveness of your existing spells.

    \magicalff[21]{Spell Rank} You become a rank 7 spellcaster in your chosen \glossterm{mystic sphere}.
    This gives you access to spells that require a minimum rank of 7 and can improve the effectiveness of your existing spells.
  \end{magicalfeat}

  \begin{magicalfeat}{Sphere Focus: Aeromancy}{Casting, Magical}
    \includegraphics[width=\columnwidth]{feats/sphere focus aeromancy}
    \featpre Access to the \sphere{Aeromancy} \glossterm{mystic sphere}.

    \magicalff[1]{Distant Winds} You gain a \plus15 foot bonus to your range with ranged spells from the \sphere{Aeromancy} \glossterm{mystic sphere}.
    This does not affect spells that do not have a range listed in feet.

    \magicalff[6]{Windborne Spell} You learn an additional spell.
    The spell can be up to rank 3, even if you do not have access to rank 3 spells.
    When you cast that spell, choose a location within \shortrange of you.
    The spell takes effect as if you were in the chosen location.
    This affects your \glossterm{line of effect} for the ability, but not your \glossterm{line of sight} (since you still see from your normal location).
    Since an ability's range is measured from your location, this can allow you to affect targets outside your normal range.

    When you gain access to new spell ranks, you can change which spell you know with this ability, including spells with a higher rank.

    \magicalff[12]{Airborne} You gain a \glossterm{fly speed} equal to the \glossterm{base speed} for your size with a maximum height of 15 feet (see \pcref{Flight}).
    As a \glossterm{free action}, you can increase your \glossterm{fatigue level} by one to ignore this height limit until the end of the round.
    In addition, you gain a \plus15 foot bonus to the height limit of any fly speed you gain from \sphere{Aeromancy} spells.

    \magicalff[18]{Distant Winds+} The range bonus from your \textit{distant winds} ability increases to \plus60 feet.
  \end{magicalfeat}

  \begin{magicalfeat}{Sphere Focus: Aquamancy}{Casting, Magical}
    \featpre Access to the \sphere{Aquamancy} \glossterm{mystic sphere}.

    \magicalff[1]{Aquatic Propulsion} Once per round, when you make an attack with a spell from the \sphere{Aquamancy} \glossterm{mystic sphere}, you can \glossterm{push} yourself up to 10 feet.
    You can use this ability with \glossterm{strikes} using natural weapons from that sphere, such as from the \spell{aqueous tentacle} spell.
    The attack does not have to hit for you to use this ability.

    \magicalff[6]{Bubble Spell} You learn an additional spell.
    The spell can be up to rank 3, even if you do not have access to rank 3 spells.
    When you cast that spell, you can imbue it in a Small bubble of water that appears in your space.
    If you do, the spell does not take effect immediately, and you choose a location within \medrange of you.
    The bubble floats 15 feet through the air towards your chosen destination during each subsequent phase.
    It pops when it reaches its destination or hits an intervening obstacle.
    As normal for simultaneous actions, creatures cannot dodge the bubble by moving away in the same phase that it would reach them.

    When the bubble pops, the spell takes effect at the bubble's location as if you were there.
    A targeted spell only targets the creature or object that the bubble popped on.
    Area spells must have their point of origin where the bubble popped, but you can otherwise control their direction normally.
    This affects your \glossterm{line of effect} for the ability, but not your \glossterm{line of sight} (since you still see from your normal location).

    When you gain access to new spell ranks, you can change which spell you know with this ability, including spells with a higher rank.

    \magicalff[12]{Partially Liquid} You gain a \plus4 bonus to the Flexibility skill.
    In addition, you gain a \plus4 bonus to your defenses when determining whether a \glossterm{strike} gets a \glossterm{critical hit} against you instead of a normal hit.

    \magicalff[18]{Partially Liquid+} You are immune to \glossterm{critical hits} from \glossterm{strikes}.
  \end{magicalfeat}

  \begin{magicalfeat}{Sphere Focus: Astromancy}{Casting, Magical}
    \featpre Access to the \sphere{Astromancy} \glossterm{mystic sphere}.

    \magicalff[1]{Astral Spell Transit} You gain a \plus15 foot bonus to your range with abilities from the \sphere{Astromancy} \glossterm{mystic sphere}.

    \magicalff[6]{Transposing Spell} You learn an additional \glossterm{targeted} spell.
    The spell can be up to rank 3, even if you do not have access to rank 3 spells.
    When you cast the spell on a Large or smaller creature, you can switch places with the target.
    This requires an attack vs. Mental if the target is not an \glossterm{ally}.
    When you switch places, you \glossterm{teleport} into the target's location, and it teleports into your location.
    If the teleportation is invalid for either target, it fails for both targets.

    When you gain access to new spell ranks, you can change which spell you know with this ability, including spells with a higher rank.

    \magicalff[12]{Efficient Transit} You learn how to transport creatures and objects more smoothly between planes.
    The \glossterm{difficulty value} to hear noise caused by creatures and objects you \glossterm{teleport} increases by 10 (see \pcref{Teleportation Noise}).
    In addition, whenever you fail to teleport a creature or object due to specifying an invalid destination, you may immediately specify a different destination for that ability.
    If that second destination is also invalid, the ability fails normally.

    \magicalff[18]{Astral Spell Transit+} The range increase from your \textit{astral spell transit} ability increases to \plus30 feet.
    In addition, you gain a \plus15 foot bonus to your range with all \magical abilities that are not from the Astromancy mystic sphere.
  \end{magicalfeat}

  \begin{magicalfeat}{Sphere Focus: Channel Divinity}{Casting, Magical}
    \featpre Access to the \sphere{Channel Divinity} \glossterm{mystic sphere}.

    \magicalff[1]{Aspect of Divinity} You gain an \glossterm{enhancement bonus} equal to your level to your \glossterm{hit points} and \glossterm{damage resistance}.

    \magicalff[6]{Judging Spell} You learn an additional spell.
    The spell can be up to rank 3, even if you do not have access to rank 3 spells.
    When you cast the spell, each non-\glossterm{ally} affected by it \glossterm{briefly} takes a \minus1 bonus to all defenses.
    This penalty also affect any attacks made by the spell itself.
    If the spell did not affect any non-ally creatures, you \glossterm{briefly} gain a \plus1 bonus to all defenses.

    When you gain access to new spell ranks, you can change which spell you know with this ability, including spells with a higher rank.

    \magicalff[12]{Font of Divinity} Choose a spell with the \abilitytag{Attune} tag from the \sphere{channel divinity} mystic sphere that is not a \glossterm{deep attunement}.
    When you attune to that spell, you may also choose one \glossterm{ally} within \medrange.
    That ally can also choose to attune to the spell, and you both gain its benefits.
    When you stop attuning to that spell, your ally is also forced to stop attuning to the spell.

    Since you cannot attune to the same spell more than once, you cannot share the effects of the spell with more than one ally at a time in this way.
    You can change which spell you choose with this ability whenever you learn a new spell or gain access to a new spell rank.

    \magicalff[18]{Greater Aspect of Divinity} The bonus from your \textit{aspect of divinity} ability increases to twice your level.
  \end{magicalfeat}

  \begin{magicalfeat}{Sphere Focus: Chronomancy}{Casting, Magical}
    \featpre Access to the \sphere{Chronomancy} \glossterm{mystic sphere}.

    \magicalff[1]{Accelerated Mind} You can perform primarily mental tasks more quickly as normal.
    \glossterm{Mundane} mental actions that would normally take a \glossterm{standard action} instead take a \glossterm{minor action}.
    Long-term activities can be done twice as quickly as normal.
    This includes reading books, searching areas, and other similar activities.
    It does not affect spellcasting, performing rituals, or other similar magical abilities.

    \magicalff[6]{Quickened Chronomancy} You learn an additional spell from the \sphere{Chronomancy} mystic sphere.
    The spell can be up to rank 3, even if you do not have access to rank 3 spells.
    You can cast the spell as a \glossterm{minor action}.
    When you do, you \glossterm{briefly} become \slowed and cannot take standard actions.

    When you gain access to new spell ranks, you can change which spell you know with this ability, including spells with a higher rank.

    \magicalff[12]{Thief of Time} Whenever you cast a spell from the \sphere{Chronomancy} mystic sphere that deals damage to another creature or causes another creature to be \slowed, you \glossterm{briefly} gain a \plus20 foot bonus to your land speed.

    \magicalff[18]{Accelerated Mind+} Once per round, you can perform a purely mental action that would normally require a \glossterm{minor action} as a \glossterm{free action}.
    This can be used to sustain spells or perform other magical feats.
    However, you cannot use the same ability twice in the same round.
    In addition, the speed increase for long-term tasks from your \textit{accelerated mind} ability increases to five times normal speed.
  \end{magicalfeat}

  \begin{magicalfeat}{Sphere Focus: Cryomancy}{Casting, Magical}
    \featpre Access to the \sphere{Cryomancy} \glossterm{mystic sphere}.

    \magicalff[1]{Frozen Blood} You are \glossterm{impervious} to \atCold and \atPoison attacks.

    \magicalff[6]{Icy Carapace} You learn the \spell{icy shell} spell, or another spell from the \sphere{Cryomancy} mystic sphere if you already know that spell.
    In addition, the number of layers you can create with \spell{icy shell} increases by one.

    \magicalff[12]{Frozen Blood+} You are immune to \atCold and \atPoison attacks.

    \magicalff[18]{Icy Carapace+} The number of bonus layers you gain from your \textit{icy carapace} ability increases to two.
  \end{magicalfeat}

  \begin{magicalfeat}{Sphere Focus: Electromancy}{Casting, Magical}
    \featpre Access to the \sphere{Electromancy} \glossterm{mystic sphere}.

    \magicalff[1]{Electricity Tolerance} You are \trait{impervious} to \atElectricity attacks.

    \magicalff[1]{Extended Chain} The distance that your abilities can \glossterm{chain} is increased to 30 feet instead of 15 feet.

    \magicalff[6]{Chaining Spell} You learn an additional \glossterm{targeted} spell.
    The spell can be up to rank 3, even if you do not have access to rank 3 spells.
    It \glossterm{chains} one additional time, even if it did not originally chain.

    When you gain access to new spell ranks, you can change which spell you know with this ability, including spells with a higher rank.

    \magicalff[12]{Electricity Immunity} You are \trait{immune} to \atElectricity attacks.

    \magicalff[12]{Static Buildup} Whenever you use the \ability{sprint} ability, you gain a \plus2 accuracy bonus with your next spell from the \sphere{Electromancy} mystic sphere.
    This bonus disappears if not used before the end of the next round.

    \magicalff[18]{Extra Chain} All of your \glossterm{targeted} \sphere{Electromancy} spells \glossterm{chain} one additional time, even if they did not originally chain.
    If your \textit{chaining spell} ability affects a non-\sphere{Electromancy} spell, it also chains two additional times instead of only one.
  \end{magicalfeat}

  \begin{magicalfeat}{Sphere Focus: Enchantment}{Casting, Magical}
    \featpre Access to the \sphere{Enchantment} \glossterm{mystic sphere}.

    \magicalff[1]{Hidden Influence} You gain a \plus4 accuracy bonus with spells from the Enchantment mystic sphere against \unaware creatures.
    In addition, the \glossterm{difficulty value} to observe the effects of your \abilitytag{Emotion} abilities with the Awareness and Social Insight skills increases by 10 (see \pcref{Notice Subtle Effect}, and \pcref{Discern Enchantment}).

    \magicalff[6]{Mind Fragments} When you use a \magical ability that has the \abilitytag{Emotion} tag or deals psychic damage, you can affect creatures that are immune to that ability due to being \trait{mindless} or \trait{simple-minded}.
    You treat those creatures as being \impervious instead of immune.
    This does not allow you to affect creatures who are immune to those abilities for other reasons.

    \magicalff[12]{Sympathetic Enchantment} You learn an additional \glossterm{targeted} spell from the \sphere{Enchantment} mystic sphere.
    The spell can be up to rank 5, even if you do not have access to rank 5 spells.
    When you cast that spell, you can spend a \glossterm{minor action} to sympathetically link yourself to the spell's targets.
    If you do, you are target of that spell in addition to any other targets.
    In exchange, you roll all attacks for that spell twice, keeping the higher result.

    When you gain access to new spell ranks, you can change which spell you know with this ability, including spells with a higher rank.

    \magicalff[18]{Hidden Influence+} The accuracy bonus from your \textit{hidden influence} ability also applies against \partiallyunaware creatures.
    In addition, the \glossterm{difficulty value} increase from that ability increases to \plus20.

    \magicalff[18]{Mind Fragments+} You no longer treat those creatures as being impervious.
  \end{magicalfeat}

  \begin{magicalfeat}{Sphere Focus: Fabrication}{Casting, Magical}
    \featpre Access to the \sphere{Fabrication} \glossterm{mystic sphere}.

    \magicalff[1]{Fabricate Trinket+} The maximum size of the trinket you can create with your \textit{fabricate trinket} cantrip increases by one size category.
    You can also cast it with the \abilitytag{Sustain} (minor) tag.
    If you do, it lasts as long as you sustain it, and does not remove any previous trinkets you created with that cantrip.

    \magicalff[1]{Mighty Creator} You can use your \glossterm{power} in place of your Craft skill to create items with spells from the \sphere{Fabrication} mystic sphere.

    \magicalff[6]{Fabricated Armaments} You gain a \plus1 accuracy bonus with \glossterm{strikes} using weapons you created with spells from the Fabrication mystic sphere.
    In addition, you gain a \plus1 bonus to the Armor defense provided by body armor from the Fabrication mystic sphere.

    \magicalff[12]{Overbuilt Fabrication} You learn an additional spell from the \sphere{Fabrication} mystic sphere.
    The spell can be up to rank 5, even if you do not have access to rank 5 spells.
    When you cast the spell, you can spend a \glossterm{minor action} to expand its construction.
    If you do, the spell's area is doubled, and the hit points of any objects the spell creates are also doubled.

    When you gain access to new spell ranks, you can change which spell you know with this ability, including spells with a higher rank.

    \magicalff[18]{Fabricate Trinket+} The size increase from your \textit{fabricate trinket+} ability increases to two size categories.

    \magicalff[18]{Fabricated Armaments+} The accuracy bonus from your \textit{fabricated armaments} ability increases to \plus2.
  \end{magicalfeat}

  \begin{magicalfeat}{Sphere Focus: Photomancy}{Casting, Magical}
    \featpre Access to the \sphere{Photomancy} \glossterm{mystic sphere}.

    \magicalff[1]{Scour Vision} Whenever you cause a creature to become \dazzled with a spell from the \sphere{Photomancy} \glossterm{mystic sphere}, you \glossterm{briefly} become \trait{invisible} to that creature.
    This works even if the creature was already dazzled.
    The creature can still see everything other than you normally.

    \magicalff[6]{Hardlight Photomancy} You learn an additional spell from the \sphere{Photomancy} mystic sphere that does not have the \abilitytag{Sustain} or \abilitytag{Attune} tags.
    The spell can be up to rank 3, even if you do not have access to rank 3 spells.
    When you cast the spell, you can spend a \glossterm{minor action} to solidify the spell's light.

    If you do, the edges of any areas of \glossterm{brilliant illumination} created by the spell become physical barriers.
    These barriers provide \glossterm{cover} against attacks made across them, though they cannot be destroyed by damage.
    Moving through the barriers costs an additional ten feet of movement.
    The barriers are only at the edges of the brilliant illumination, so this does not affect movement or attacks entirely within or entirely outside of the illuminated area.

    When you gain access to new spell ranks, you can change which spell you know with this ability, including spells with a higher rank.

    \magicalff[12]{Solar Beacon} Whenever you make a creature lose \glossterm{hit points} with a spell from the \sphere{Photomancy} mystic sphere, that creature suffers consequences as if it had been struck by a beam of natural sunlight.
    This can be deadly for some creatures.

    \magicalff[12]{Unwavering Sight} You are immune to being \dazzled and \blinded.

    \magicalff[18]{Piercing Sight} You can see through solid objects and spell effects up to one inch thick.
    You can perceive the existence of obstacles thinner than that, but they do not inhibit your sight.
    This does not grant you \glossterm{line of effect} to anything you see in this way, since the obstacle still exists.
  \end{magicalfeat}

  \begin{magicalfeat}{Sphere Focus: Polymorph}{Casting, Magical}
    \featpre Access to the \sphere{Polymorph} \glossterm{mystic sphere}.

    \magicalff[1]{Augmented Body}
    \begin{magicalattuneability}{Augmented Body}{\abilitytag{Attune} (deep)}
      \abilityusagetime Can be triggered when you finish a \glossterm{short rest}.
      \rankline
      Choose a physical attribute: Strength, Dexterity, or Constitution.
      You gain a \plus1 \glossterm{enhancement bonus} to that attribute.
    \end{magicalattuneability}

    \magicalff[6]{Fleshbending Spell} You learn an additional spell.
    The spell can be up to rank 3, even if you do not have access to rank 3 spells.
    When you attack with the spell, if your attack beats a living creature's Fortitude defense, it takes a \minus4 penalty to its other defenses against that spell.

    When you gain access to new spell ranks, you can change which spell you know with this ability, including spells with a higher rank.

    \magicalff[12]{Augmented Body+} This loses the \abilitytag{Attune} (deep) tag and gains the \abilitytag{Attune} tag.

    \magicalff[12]{Malleable Flesh} You gain a \plus2 bonus to the Flexibility skill.
    In addition, you gain a \plus2 bonus to your defenses when determining whether a \glossterm{strike} gets a \glossterm{critical hit} against you instead of a normal hit.

    \magicalff[18]{Malleable Flesh+} The bonuses increase to \plus10.
  \end{magicalfeat}

  \begin{magicalfeat}{Sphere Focus: Prayer}{Casting, Magical}
    \featpre Access to the \sphere{Prayer} \glossterm{mystic sphere}.

    \magicalff[1]{Shared Boon} When you cast a targeted spell from the \sphere{Prayer} mystic sphere that does not have the \abilitytag{Sustain} or \abilitytag{Attune} tags on a single \glossterm{ally}, you can target an additional ally within range.

    \magicalff[6]{Sustained Blessing} Whenever you cast a spell from the \sphere{Prayer} mystic sphere with the \abilitytag{Attune} (target) tag that is not a \glossterm{deep attunement}, you can choose to replace that tag with the \abilitytag{Sustain} (minor) tag.
    When you do, you must cast the spell as a \glossterm{standard action}, even if it could normally be cast as a \glossterm{minor action}.
    You can only apply this ability to one spell at a time.

    \magicalff[12]{Sustained Blessing+} When you used your \textit{sustained blessing} ability, you can replace the tag with \abilitytag{Sustain} (free) instead of \abilitytag{Sustain} (minor).
    You can still only sustain one spell in this way at a time.

    \magicalff[18]{Shared Boon+} When you use your \textit{shared boon} ability, you can also target yourself or a third ally within range.
  \end{magicalfeat}

  \begin{magicalfeat}{Sphere Focus: Pyromancy}{Casting, Magical}
    \featpre Access to the \sphere{Pyromancy} \glossterm{mystic sphere}.

    \magicalff[1]{Spreading Flame} Whenever you cast a spell from the \sphere{Pyromancy} \glossterm{mystic sphere} with a standard area, you can increase its area to the next standard area category, to a maximum of a Gargantuan area.
    If you do, the spell deals no damage against targets that it misses, even if it would normally deal half damage.
    The standard area categories are: \smallarea, \medarea, \largearea, \hugearea, and \gargarea.

    \magicalff[1]{Fire Tolerance} You are \trait{impervious} to \atFire attacks.

    \magicalff[6]{Overheated Pyromancy} You learn an additional spell from the \sphere{Pyromancy} mystic sphere.
    The spell can be up to rank 3, even if you do not have access to rank 3 spells.
    When you cast the spell, you can spend a \glossterm{minor action} to \glossterm{briefly} overheat the spell.
    You roll all attacks for an overheated spell twice, keeping the higher result.
    After using this ability, you \glossterm{briefly} cannot use it again and are \vulnerable to \atCold attacks.

    When you gain access to new spell ranks, you can change which spell you know with this ability, including spells with a higher rank.

    \magicalff[12]{Friendly Fire} Whenever you deal damage to your \glossterm{allies} with a \atFire ability, you deal half damage.

    \magicalff[12]{Fire Immunity} You are \trait{immune} to \atFire attacks.

    \magicalff[18]{Spreading Flame+} When you use your \textit{spreading flame} ability, you can still get a glancing blow with the spell.
  \end{magicalfeat}

  \begin{magicalfeat}{Sphere Focus: Revelation}{Casting, Magical}
    \featpre Access to the \sphere{Revelation} \glossterm{mystic sphere}.

    \magicalff[1]{Oracle} Once per \glossterm{long rest}, you can gain the benefit of the \ritual{augury} ritual as a standard action.
    Using this ability does not increase your \glossterm{fatigue level}.

    \magicalff[6]{Prescient Spell} You learn an additional spell.
    The spell can be up to rank 3, even if you do not have access to rank 3 spells.
    As a \glossterm{minor action}, you can foresee the future of casting that spell.
    When you do, roll 1d10 and record the result.
    If you cast the spell, then the first time you attack with it this round, you must use this die result for the attack roll.
    This does not tell you whether your attack will hit, only high your roll will be.
    A 10 explodes as normal, but only when you actually cast the spell and make the attack, not when foreseeing the future.
    If you don't like the result, you can simply choose not to cast the spell this round, and the die result is ignored.
    Either way, you \glossterm{briefly} cannot use this ability again.

    When you gain access to new spell ranks, you can change which spell you know with this ability, including spells with a higher rank.

    \magicalff[12]{Blind Seer} You gain \trait{blindsense} with a 90 foot range (see \pcref{Blindsense}).
    If you already have blindsense, you increase its range by 90 feet.
    In addition, you gain \trait{blindsight} with a 30 foot range, allowing you to see without light (see \pcref{Blindsight}).
    If you already have blindsight, the range of your blindsight increases by 30 feet.

    \magicalff[18]{Oracle+} You can use your \textit{oracle} ability once per \glossterm{short rest} instead of once per long rest.
  \end{magicalfeat}

  \begin{magicalfeat}{Sphere Focus: Summoning}{Casting, Magical}
    \featpre Access to the \sphere{Summoning} \glossterm{mystic sphere}.

    \magicalff[1]{Empower Summon}
    \begin{magicalactiveability}{Empower Summon}[\abilitytag{Swift}]
      \abilityusagetime \glossterm{Minor action}.
      \rankline
      Choose one creature within \medrange that you created with a spell from the \sphere{Summoning} \glossterm{mystic sphere}.
      That creature gains a \plus2 bonus to its \glossterm{accuracy} and \glossterm{defenses} this round.
    \end{magicalactiveability}

    \magicalff[6]{Resummon}
    \begin{magicalactiveability}{Resummon}
      \abilityusagetime Standard action.
      \abilitycost One \glossterm{fatigue level}.
      \rankline
      Choose a spell from the \sphere{Summoning} mystic sphere that you are still \glossterm{attuned} to, even if all creatures created by that spell are dead.
      All effects of that spell, including any active creatures, immediately end.
      Then, you recreate all creatures from that spell as if you had just cast it.
      Those creatures do not act until the next round.
    \end{magicalactiveability}

    \magicalff[12]{Generosity} Whenever you \glossterm{attune} to an item or spell that is not a \glossterm{deep attunement}, you may use this ability.
    If you do, all creatures you create with the \sphere{Summoning} mystic sphere gain the benefit of all \glossterm{enhancement bonuses} from the chosen item or spell.
    It still affects you normally.
    You may only choose one attunement with this ability at a time.

    \magicalff[12]{Resummon+} You can use your \textit{resummon} ability as a \glossterm{minor action}.

    \magicalff[18]{Empower Summon+} The bonuses from your \textit{empower summon} ability increase to \plus4.
  \end{magicalfeat}

  \begin{magicalfeat}{Sphere Focus: Telekinesis}{Casting, Magical}
    \featpre Access to the \sphere{Telekinesis} \glossterm{mystic sphere}.

    \magicalff[1]{Efficient Hand} You can cast the \spell{distant hand} \glossterm{cantrip} as a \glossterm{minor action}, and you can \glossterm{sustain} it as a \glossterm{free action}.
    In addition, the distance you can move the object increases to 30 feet, as if you had that spell's rank 4 upgrade.

    \magicalff[6]{Telekinetic Strike}
    \begin{magicalactiveability}{Telekinetic Strike}
      \abilityusagetime Standard action.
      \rankline
      Choose a weapon you are proficient with and currently controlling using the \spell{distant hand} cantrip.
      Make a melee \glossterm{strike} with a \plus2 accuracy bonus using that weapon.
      The strike comes from the weapon's location, not your location.
      Because this is a \magical ability, you use your \glossterm{magical power} to determine your damage instead of your \glossterm{mundane power} (see \pcref{Power}).

      \rankline
      % TODO: awkward scaling
      \featlevel{9} The accuracy bonus increases to \plus4.
      \featlevel{12} The accuracy bonus increases to \plus6.
      \featlevel{15} The accuracy bonus is reduced to \plus2, but the strike deals double \glossterm{weapon damage}.
      \featlevel{18} The accuracy bonus increases to \plus4.
    \end{magicalactiveability}

    \magicalff[12]{Kinetic Spell} You learn an additional spell.
    The spell can be up to rank 5, even if you do not have access to rank 5 spells.
    Whenever you cast that spell, you can spend a \glossterm{minor action} to \glossterm{briefly} imbue it with kinetic force.
    When the spell deals damage to a Huge or smaller target, you \glossterm{knockback} that target 15 feet away from you.
    You can only knockback any individual creature or object once per round.
    After using this ability, you \glossterm{briefly} cannot use it again.

    When you gain access to new spell ranks, you can change which spell you know with this ability, including spells with a higher rank.

    \magicalff[18]{Multiple Hands} When you use the \spell{distant hand} cantrip, you can create two hands instead of one.
    This can allow you to hold multiple objects simultaneously, though you must still take separate actions to manipulate those objects in any way more complicated than simple movement.
    Alternately, you can hold a single object with two hands instead of one.
    This gives you a \plus1 bonus to your Willpower for the purpose of determining your \glossterm{weight limits}, and allows you to make two-handed attacks with \weapontag{Heavy} and \weapontag{Versatile Grip} weapons using your \ability{telekinetic strike} ability.
  \end{magicalfeat}

  \begin{magicalfeat}{Sphere Focus: Terramancy}{Casting, Magical}
    \featpre Access to the \sphere{Terramancy} \glossterm{mystic sphere}.

    \magicalff[1]{Earthen Alloys} You may treat iron, steel, and worked stone as if they were unworked stone for the purpose of spells from the \sphere{Terramancy} \glossterm{mystic sphere}.

    \magicalff[6]{Rocky Carapace} You learn the \spell{rocky shell} spell, or another spell from the \sphere{Terramancy} mystic sphere if you already know that spell.
    In addition, the number of layers you can create with that spell increases by one.

    \magicalff[12]{Earthen Alloys+} You may treat dirt, sand, glass, and all kinds of metal except for cold iron and adamantine as if they were unworked stone for the purpose of spells from the \sphere{Terramancy} \glossterm{mystic sphere}.

    \magicalff[18]{Rocky Carapace+} The number of bonus layers you gain from your \textit{rocky carapace} ability increases to two.
  \end{magicalfeat}

  \begin{magicalfeat}{Sphere Focus: Thaumaturgy}{Casting, Magical}
    \featpre Access to the \sphere{Thaumaturgy} \glossterm{mystic sphere}.

    \magicalff[1]{Mystic Power} You gain a \plus1 bonus to your \glossterm{magical power}.

    \magicalff[6]{Invented Spell} You learn an additional spell from any mystic sphere, even a mystic sphere you do not have access to.
    The spell can be up to rank 3, even if you do not have access to rank 3 spells.
    When you learn the spell, you can remove one of the following tags from that spell: \atAcid, \atAuditory, \atCold, \atCompulsion, \atEmotion, \atElectricity, \atFire, or \atVisual.
    You can also add one of those tags to the spell.
    In addition, you can replace one defense that the spell attacks with any one other defense not already referenced by the spell.

    When you gain access to new spell ranks, you can change which spell you know with this ability, including spells with a higher rank.

    \magicalff[12]{Countermagic}
    \begin{magicalactiveability}{Countermagic}[\abilitytag{Swift}]
      \abilityusagetime Standard action.
      \abilitycost See text.
      \rankline
      Choose a creature within \rngmed range of you.
      If the target uses a \magical ability as a standard action this round, that ability has no effect.
      When you negate an ability in this way, if that ability's rank was higher than your maximum rank, you increase your \glossterm{fatigue level} by one.
      After you use this ability, you \glossterm{briefly} cannot use it again, whether or not it actually negated a magical effect.

      If a creature is capable of using multiple abilities during a single phase, only the first ability it uses can be countered.
      This is common for \glossterm{elite} monsters.
    \end{magicalactiveability}

    \magicalff[18]{Mystic Power+} The bonus from your \textit{mystic power} ability increases to \plus2.
  \end{magicalfeat}

  \begin{magicalfeat}{Sphere Focus: Toxicology}{Casting, Magical}
    \featpre Access to the \sphere{Toxicology} \glossterm{mystic sphere}.

    \magicalff[1]{Poisonous Blood} Whenever you become poisoned by a contact or ingestion poison, such as by drinking poison or from an enemy's attack, your body naturally repurposes the poison.
    The poison has no effect on you, but your body gains a dose of natural poison.
    This has no effect on injury-based poisons, which affect you normally without being absorbed.

    Whenever a living creature makes you lose \glossterm{hit points} with a \glossterm{melee} strike using a non-Long weapon, you may have that creature become \glossterm{poisoned} by your choice of one of the poisons you store with this ability.
    This expends the dose of that poison.

    Poisons that you carry in your body with this ability automatically decay after 24 hours, regardless of the normal duration of the poison.
    You can store up to 3 doses in your body with this ability at a time.

    \magicalff[6]{Venomous Spell} You learn an additional spell.
    The spell can be up to rank 3, even if you do not have access to rank 3 spells.
    Whenever you damage a living creature with that spell, you can cause the creature to become poisoned with your choice of one of the poisons you store with your \textit{poisonous blood} ability.
    This expends the dose of that poison.
    You can poison multiple creatures simultaneously in this way, but it costs one poison dose per creature affected.

    When you gain access to new spell ranks, you can change which spell you know with this ability, including spells with a higher rank.

    \magicalff[12]{Poisonous Blood+} You can store up to 6 poison doses with your \textit{poisonous blood} ability.
    In addition, you become immune to all poisons that you do not absorb with that ability.

    \magicalff[18]{Bloodstream Corruption} Whenever you make a living creature lose hit points with a \sphere{Toxicology} spell or your \textit{venomous spell}, that creature becomes \vulnerable to all poisons as a \glossterm{condition}.
  \end{magicalfeat}

  \begin{magicalfeat}{Sphere Focus: Umbramancy}{Casting, Magical}
    \featpre Access to the \sphere{Umbramancy} \glossterm{mystic sphere}.

    \magicalff[1]{Lightbane} You can cast the \spell{suppress light} \glossterm{cantrip} from the Umbramancy mystic sphere as a \glossterm{minor action}, and you can \glossterm{sustain} it as a \glossterm{free action}.

    \magicalff[6]{Shadowspell} You learn an additional spell.
    The spell can be up to rank 3, even if you do not have access to rank 3 spells.
    You gain a \plus2 accuracy bonus with that spell against \glossterm{shadowed} targets.
    This stacks with any existing bonus the spell has against shadowed targets.

    When you gain access to new spell ranks, you can change which spell you know with this ability, including spells with a higher rank.

    \magicalff[12]{Shadow Jumper} While you are \glossterm{shadowed}, the distance you can teleport with spells from the \sphere{Umbramancy} mystic sphere is doubled.

    \magicalff[18]{Lightbane+} When you cast your \textit{suppress light} cantrip, you can choose to completely block all light in the area instead of dimming it to be \glossterm{shadowy illumination}.
    If you do, the maximum area is reduced to a \medarea radius, and you \glossterm{briefly} cannot cast it in this way again.
  \end{magicalfeat}

  \begin{magicalfeat}{Sphere Focus: Verdamancy}{Casting, Magical}
    \featpre Access to the \sphere{Verdamancy} \glossterm{mystic sphere}.

    \magicalff[1]{Verdant Allies} Your speed is not reduced when moving in heavy \glossterm{undergrowth}.
    In addition, you can ignore \glossterm{cover} and \glossterm{concealment} from plants whenever doing so would be beneficial to you, as the plants move out of the way to help you.
    This prevents you from suffering a miss chance on your attacks, and also prevents creatures from using cover or concealment from plants to hide from you.

    \magicalff[6]{Residual Undergrowth} Whenever you cast a spell from the \textit{verdamancy} sphere, you may create either \glossterm{light undergrowth} or \glossterm{heavy undergrowth} in the area of the spell.
    The undergrowth persists \glossterm{briefly}.
    It appears on the ground within the area for area spells, or on the ground in all spaces occupied by each target of the spell for targeted spells.

    \magicalff[12]{Lifeweb Spell} You learn an additional spell.
    The spell can be up to rank 5, even if you do not have access to rank 5 spells.
    Whenever you cast that spell, you can choose another Small or larger plant or living creature within \medrange of you.
    The spell takes effect as if you were in that plant or creature's space.
    This affects your \glossterm{line of effect} for the ability, but not your \glossterm{line of sight} (since you still see from your normal location).

    When you gain access to new spell ranks, you can change which spell you know with this ability, including spells with a higher rank.

    \magicalff[18]{Verdant Army} Your \glossterm{allies} within a \largearea radius \glossterm{emanation} from you also gain the benefit of your \textit{verdant allies} ability.
  \end{magicalfeat}

  \begin{magicalfeat}{Sphere Focus: Vivimancy}{Casting, Magical}
    \featpre Access to the \sphere{Vivimancy} \glossterm{mystic sphere}.

    % Should this be more specific, like "animates"?
    \magicalff[1]{Hidden Life} You can treat nonliving creatures other than undead as if they were living creatures for the purpose of your abilities from the \sphere{Vivimancy} \glossterm{mystic sphere}.

    \magicalff[1]{Personal Vitality} You gain a bonus equal to your level to your \glossterm{hit points}.

    \magicalff[6]{Soulscar Spell} You learn an additional spell that does not have the \abilitytag{Attune} or \abilitytag{Sustain} tags.
    The spell can be up to rank 3, even if you do not have access to rank 3 spells.
    Any hit point loss that would be inflicted on a living creature by the spell is doubled.
    However, whenever you cast the spell, you lose hit points equal to its rank.

    When you gain access to new spell ranks, you can change which spell you know with this ability, including spells with a higher rank.

    \magicalff[12]{Life Suppression} You are no longer considered a living creature for the purpose of attacks against you.
    This means that attacks which only affect living creatures have no effect against you.

    \magicalff[18]{Personal Vitality+} The hit point bonus increases to twice your level.
  \end{magicalfeat}

  \begin{feat}{Stealth Specialization}{Skill}
    \featpre Stealth as a trained skill.

    \ff[1]{Specialization} You gain a \plus3 bonus to the Stealth skill.

    \ff[6]{Mobile Stealth} Your penalties for moving while hiding are reduced by 5.
    This allows you to move at half speed without penalty.

    \ff[12]{Specialization+} The bonus from your \textit{specialization} ability increases to \plus6.

    \ff[18]{Ambush the Unwary} You gain a \plus2 accuracy bonus against \unaware and \partiallyunaware creatures.
  \end{feat}

  \begin{feat}{Survival Specialization}{Skill}
    \featpre Survival as a trained skill.

    \ff[1]{Specialization} You gain a \plus3 bonus to the Survival skill.

    \ff[6]{Terrain Tolerance} You ignore \glossterm{difficult terrain} and harmful natural terrain of any kind.
    If a skill check, such as Climb or Swim, would normally be required to move through the terrain, this ability does not help.

    \ff[12]{Specialization+} The bonus from your \textit{specialization} ability increases to \plus6.

    \magicalff[18]{Planar Survival} You are immune to damage and \glossterm{conditions} imposed by being on other planes.
    In addition, you gain a \plus5 bonus to checks and defenses related to planar effects, such as checks required to manipulate subjective gravity.
  \end{feat}

  \begin{feat}{Swiftrunner}{General}
    \featpre Dexterity 3.

    \ff[1]{Sprinter} When you use the \textit{sprint} ability, you can move up to triple your movement speed.

    \ff[6]{Rapid Movement} You gain a \plus10 foot bonus to your land speed.

    \ff[6]{Water Runner} During your movement with the \textit{sprint} ability, you can move on water and similar liquids as if they were solid ground.

    \ff[12]{Long-Distance Runner} You gain a \plus2 bonus to your \glossterm{fatigue tolerance}.

    \ff[18]{Cloud Runner} During your movement with the \textit{sprint} ability, you can move on clouds, fog, and similar gaseous substances as if they were solid ground.

    \ff[18]{Rapid Movement+} The speed bonus from your \textit{rapid movement} ability increases to \plus20 feet.
  \end{feat}

  \begin{feat}{Swim Specialization}{Skill}
    \featpre Swim as a trained skill.

    \ff[1]{Specialization} You gain a \plus3 bonus to the Swim skill.

    \ff[6]{Swim Speed} You gain a \glossterm{swim speed} 10 feet slower than the \glossterm{base speed} for your size.
    If you already have a swim speed, you gain a \plus10 foot bonus to your swim speed.
    A successful Swim check to move allows you to move a distance equal to your swim speed.

    \ff[12]{Specialization+} The bonus from your \textit{specialization} ability increases to \plus6.

    \ff[18]{Earth Swimmer} You can swim through loose earth and dirt as if it were water.
    Your swim speed in earth is 10 feet, regardless of any bonuses or penalties that would normally apply to your swim speed.
    In addition, you take a \minus4 penalty to \glossterm{accuracy} and your Armor and Reflex defenses while swimming in this way.
    The earth and dirt around you blocks line of sight and line of effect, so you usually cannot used ranged attacks of any kind.
  \end{feat}

  \begin{magicalfeat}{Telepath}{General, Magical}
    \featpre Intelligence and Willpower sum to at least 3

    \magicalff[1]{Telepathy} You gain \glossterm{telepathy} with a 120 foot range (see \pcref{Telepathy}).
    If you already have telepathy, the range of your telepathy increases by 120 feet.

    \magicalff[6]{Mental Assault}
    \begin{magicalactiveability}{Mental Assault}[\abilitytag{Compulsion}]
      \abilityusagetime Standard action.
      \rankline
      Make an attack vs. Mental against one creature within half the maximum range of your \glossterm{telepathy}.
      You gain an accuracy bonus with this attack equal to half the sum of your Intelligence and Willpower.
      \hit \damagerankone{psychic}.
      If the target loses \glossterm{hit points} from this damage, it becomes \stunned as a \glossterm{condition}.

      \rankline
      % increase to d2
      \featlevel{9} The damage bonus from your power increases to be equal to your power.
      % increase to d3ish
      \featlevel{12} The base damage increases to 1d10.
      % increase to d4ish
      \featlevel{15} The damage bonus increases to 1d6 per 3 power.
      % at r7, can upgrade condition
      \featlevel{18} The target becomes \confused instead of stunned.
    \end{magicalactiveability}

    \magicalff[12]{Read Mind}
    \begin{magicalsustainability}{Read Mind}{\abilitytag{Emotion}, \abilitytag{Subtle}, \abilitytag{Sustain} (standard)}
      \abilityusagetime Standard action.
      \rankline
      Make an attack vs. Mental against a creature within half the maximum range of your \glossterm{telepathy}.
      You use your full Intelligence or Willpower in place of half your Perception to determine your accuracy with this attack (see \pcref{Accuracy}).
      Whether you hit or miss, you cannot attack the target with this ability again until it finishes a \glossterm{short rest}.
      \hit You know the target's current thoughts and emotions.
      In addition to the obvious effects, this grants you a \plus5 bonus to Deception, Persuasion, Intimidate, and Social Insight attacks and checks against the target.
      This bonus does not stack with other effects that allow you access to the target's mind, such as \ability{read emotions}.
      You cannot directly search the target's mind for arbitrary thoughts or information.
      However, creatures often think about questions they are asked, and their thoughts may reveal much more than their words.
      \crit You can also delve through the target's mind to answer a specific question.
      You can pose a question to it mentally and search its mind to know the exact answer to that question.
      This takes five rounds of continuous concentration, and you can only get answers to one such question each time you use this ability.
      The process of searching a creature's mind in this way is no easier to notice than normal for a \abilitytag{Subtle} ability.

      \rankline
      You gain a \plus2 accuracy bonus for every 3 levels beyond 12.
    \end{magicalsustainability}

    \magicalff[12]{Telepathy+} You can maintain mental channels with up to 5 creatures at once with your telepathy.
    You can send separate thoughts to each creature.

    \magicalff[18]{Mindsight} You can perfectly see intelligent creatures within the maximum range of your telepathy.
    This functions like \trait{blindsight}, except that it only allows you to see creatures with an Intelligence of 0 or higher.

    \magicalff[18]{Telepathy++} The range of your telepathy increases by 120 feet.
  \end{magicalfeat}

  \begin{feat}{Toughness}{General}
    \featpre Constitution 2.

    \ff[1]{Fortified Body} You gain a \plus2 bonus to your Fortitude defense.
    In addition, you need half the normal amount of rest and sleep each day to function normally.
    For example, a human would only need four hours of sleep per night.
    This does not reduce the time required for you to take a \glossterm{long rest}.

    \ff[6]{Durable} You gain a bonus equal to your level to your \glossterm{hit points} and \glossterm{damage resistance}.

    \ff[12]{Fortified Body+} The defense bonus from your \textit{fortified body} ability increases to \plus4.
    In addition, you also only need half the normal amount of time to complete a \glossterm{long rest}.

    \ff[18]{Durable+} The bonuses increase to twice your level.
  \end{feat}

  \begin{feat}{Trickshot}{Combat}
    \featpre Dexterity 2, Intelligence 1.

    \ff[1]{Alchemical Arrow} As a standard action, you can apply a consumable alchemical item to an arrow.
    The item does not have its normal area or any other immediate effect.
    When you hit or \glossterm{glance} a target with a \glossterm{strike} using that arrow, the alchemical item has its normal effect on that target.
    The alchemical item makes its own attack roll separately from the attack roll for the strike.
    It only affects one target of the strike, even if the strike would normally affect multiple creatures or objects.
    If the alchemical item would normally affect an area, that area is centered on the target of the strike.
    Then, the alchemical item and the arrow are destroyed.

    At the end of the next round after you use this ability, the arrow and alchemical item are destroyed.

    \ff[6]{Combo Expertise} Whenever you use a consumable alchemical item, you \glossterm{briefly} gain a \plus1 accuracy bonus with strikes.
    Whenever you make a strike, you briefly gain a \plus1 accuracy bonus with consumable alchemical items.

    \ff[12]{Alchemical Arrow+} When you make a strike with the arrow, the alchemical item takes effect on the target even if the strike misses.
    You can also intentionally miss with the arrow while still delivering the alchemical item to its intended target, allowing you to apply healing potions and similar effects from a distance.

    \ff[18]{Combo Expertise+} Both accuracy bonuses increase to \plus2.
  \end{feat}

  \begin{magicalfeat}{Twinhand Spellcaster}{Casting, Magical}
    \featpre Dexterity 2.

    \magicalff[1]{Twinhand Precision} You can always choose to use \glossterm{somatic components} to cast your spells (see \pcref{Casting Components}).
    As long as you have two \glossterm{free hands}, you gain a \plus1 \glossterm{accuracy} bonus with spells that you cast using \glossterm{somatic components}.

    \magicalff[6]{Freehand Implement} You can gain the benefits of up to two magical implements, such as staves or wands, without having to hold them in your hands.
    You must still have them on your person, such as in a pocket or strapped to your back, and you must still be attuned to them to gain their benefits.
    In addition, if your legacy item is an apparel item, you may choose both apparel and implement magic item effects for it.

    \magicalff[12]{Twinspell}
    \begin{magicalactiveability}{Twinspell}
      \abilityusagetime Standard action.
      \abilitycost Two \glossterm{fatigue levels}.
      \rankline
      You can only use this ability if you have two \glossterm{free hands}.

      Choose two spells that you know.
      You cast both spells simultaneously, one with each hand.
      This gives the spells \glossterm{somatic components}, regardless of any other effects which would would normally prevent you from requiring somatic components.
      Both spells must affect completely different targets, with no overlap between their targets or areas (if any).
      You cannot use the \ability{desperate exertion} ability to affect either spell.
    \end{magicalactiveability}

    \magicalff[18]{Twinhand Precision+} The bonus from your \textit{twinhand precision} ability increases to \plus2.
  \end{magicalfeat}

  \begin{feat}{Twin-Weapon Fighting}{Combat}
    \featpre Dexterity 2.

    \ff[1]{Twin-Wielding} You are considered to be twin-wielding while two of your hands are each wielding an identical weapon that you are proficient with.
    Each hand must hold a different weapon, rather than simply holding a heavy weapon in two hands.
    You can use \glossterm{natural weapons} to meet this condition, such as claws, as long as your hand is exclusively dedicated to using the natural weapon and not holding or being used for anything else.
    Several abilities from this feat only function while you are twin-wielding.

    \ff[1]{Paired Precision} You gain a \plus1 bonus to \glossterm{accuracy} with \glossterm{strikes} while twin-wielding.

    \ff[6]{Twin-Weapon Stance} At the start of each round, you can enter one of the stances below or change which stance you are in.
    You can maintain that stance as long as you are twin-wielding.
    % TODO: do fancy balance math
    \begin{itemize}
      \item Balanced: You gain a \plus1 bonus to your Armor defense.
        This bonus is considered to come from a shield, and it does not stack with the benefits of using any other shield.
      \item Deflecting Offhand: You gain a \plus3 bonus to your Armor defense.
        However, you cannot make \glossterm{dual strikes}.
        This bonus is considered to come from a shield, and it does not stack with the benefits of using any other shield.
        % If glancing with dual strikes becomes possible, that should count here
      \item Offhand Precision: When you make a dual strike, if you miss with one weapon, you gain a \plus4 accuracy bonus against the same target with the second weapon.
      \item Steady Hands: You gain a \plus2 accuracy bonus with dual strikes.
        However, you cannot get \glossterm{critical hits} with dual strikes.
    \end{itemize}

    \magicalff[12]{Paired Imbuement} While you are twin-wielding, both of your weapons share all \magical properties from either weapon.

    \ff[18]{Paired Precision+} The accuracy bonus increases to \plus2.
  \end{feat}

  \begin{magicalfeat}{Wardweaver}{Casting, Magical}
    \featpre Knowledge of a spell with the \abilitytag{Barrier} tag.

    \magicalff[1]{Hardened Barriers} Objects you create using abilities with the \abilitytag{Barrier} \glossterm{ability tag} gain a bonus equal to your \glossterm{power} to their \glossterm{damage resistance}.
    In addition, they treat all damage as being \glossterm{environmental damage}, so their damage resistance is never reduced (see \pcref{Environmental Damage}).

    \magicalff[6]{Defensive Barriers} You gain a \plus1 bonus to your Armor defense.
    In addition, objects you create with \abilitytag{Barrier} abilities have minimum defenses equal to 5 \add half your level.
    If the object already has specific defenses listed, it gains a \plus2 bonus to those defenses.

    \magicalff[12]{Barrier Spell} You learn a spell with the \abilitytag{Barrier} ability tag from any \glossterm{mystic sphere} that your \glossterm{magic source} gives access to, even if you do not have access to that mystic sphere.
    When you gain access to new spell ranks, you can change which Barrier spell you know.

    \magicalff[18]{Defensive Barriers+} Your Armor defense bonus from your \textit{defensive barriers} ability increases to \plus2.
    In addition, objects you create with \abilitytag{Barrier} abilities gain an additional \plus2 bonus to their defenses.

    \magicalff[18]{Hardened Barriers+} The damage resistance bonus from your \textit{hardened barriers} ability increases to twice your \glossterm{power}.
  \end{magicalfeat}

  \begin{feat}{Weapon Focus}{Combat}
    \ff[1]{Focused Weapon} Choose one type of weapon, such as a broadsword.
    This is your focused weapon, and many abilities from this feat give you benefits with your focused weapon.

    \ff[1]{Precise Focus} You gain a \plus1 accuracy bonus with your focused weapon.

    \ff[6]{Enhanced Legacy} If you choose your focused weapon as your \glossterm{legacy item}, you can choose an additional magic item property with a maximum rank of 3.

    \ff[12]{Mighty Focus} You gain a \plus2 bonus to \glossterm{power} with attacks using your focused weapon.

    \ff[12]{Precise Focus+} The accuracy bonus increases to \plus2.

    \ff[18]{Enhanced Legacy+} You can change the item property, and the maximum rank increases to 7.
  \end{feat}

  \begin{feat}{Whirlwind Warrior}{Combat}
    \featpre Dexterity 2.

    \ff[1]{Unfettered Movement} You can move through spaces occupied by enemies as if they were unoccupied.

    \ff[6]{Cyclone}
    \begin{sustainability}{Cyclone}{\abilitytag{Sustain} (standard)}
      \abilityusagetime Standard action.
      \rankline
      When you use this ability, make a melee \glossterm{strike} with 1d4 \glossterm{extra damage}.
      The strike targets all \glossterm{enemies} adjacent to you.
      Whenever you sustain this ability, you can move up to half your \glossterm{land speed} and make a melee \glossterm{strike}.
      The strike targets all \glossterm{enemies} adjacent to you at any point during your movement.
      \miss Half damage.

      \rankline
      \featlevel{9} The extra damage increases to be equal to half your power.
      \featlevel{12} The extra damage increases to be equal to your power.
      \featlevel{15} The extra damage increases to 1d8 \add your power.
      \featlevel{18} The strike deals double \glossterm{weapon damage}.
    \end{sustainability}

    \ff[12]{Eye of the Storm} You take no penalties for \squeezing with other creatures.
    This does not reduce your penalties for squeezing in tight spaces.

    \ff[18]{Storm's Heart} You gain a \plus2 bonus to your Armor and Reflex defenses against creatures that you are sharing space with.
  \end{feat}

\section{Other Feat Rules}

  \subsection{Retraining Feats}
    At every level, you can choose to retrain an old feat in exchange for a new feat.
