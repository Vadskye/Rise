\chapter{Adventuring}

\section{Encumbrance}\label{Encumbrance}
Encumbrance rules determine how much a character's armor and equipment hinder his movement. There are several reasons a character can be encumbered. Encumbered characters may be unable to use certain class features and abilities which require free motion.

\subsection{Encumbrance by Armor}

A character's armor affects his or her Dexterity, armor check penalty, and speed. A character wearing medium or heavy armor is encumbered. Unless your character is weak or carrying a lot of gear, that's all you need to know. The extra gear your character carries won't slow him or her down any more than the armor already does.

If your character is weak or carrying a lot of gear, however, then you'll need to calculate encumbrance by weight. Doing so is most important when your character is trying to carry some heavy object.

\begin{dtable}
    \lcaption{Weight Limits}
    \setlength{\tabcolsep}{4pt}
    \begin{tabularx}{\columnwidth}{X l l l l}
        \thead{Strength} & \thead{Unencumbered} & \thead{Maximum} & \thead{Overloaded} & \thead{Push/Drag} \\
        \minus9 & 5 lb. & 10 lb. & 15 lb. & 50 lb. \\
        \minus8 & 10 lb. & 20 lb. & 30 & 100 \\
        \minus7 & 15 lb. & 30 lb. & 45 & 150 \\
        \minus6 & 20 lb. & 40 lb. & 60 & 200 \\
        \minus5 & 25 lb. & 50 lb. & 75 & 250 \\
        \minus4 & 30 lb. & 60 lb. & 90 & 300 \\
        \minus3 & 35 lb. & 70 lb. & 115 & 350 \\
        \minus2 & 40 lb. & 80 lb. & 120 & 400 \\
        \minus1 & 45 lb. & 90 lb. & 135 & 450 \\
        0 & 50 lb. & 100 lb. & 150 & 500 \\
        1 & 55 lb. & 110 lb. & 160 & 550 \\
        2 & 70 lb. & 140 lb. & 210 & 700 \\
        3 & 95 lb. & 190 lb. & 285 & 950 \\
        4 & 130 lb. & 260 lb. & 390 & 1,300 \\
        5 & 175 lb. & 350 lb. & 525 & 1,750 \\
        6 & 230 lb. & 460 lb. & 690 & 2,300\\
        7 & 295 lb. & 590 lb. & 885 & 2,950 \\
        8 & 370 lb. & 740 lb. & 1,110 & 3,700 \\
        9 & 455 lb. & 910 lb. & 1,365 & 4,550 \\
        10 & 550 lb. & 1,100 lb. & 1,650 & 5,500 \\
        11\plus\fn{1} & \x & \x & \x & \x \\
    \end{tabularx}
    1 A creature with extraordinary strength can carry a number of pounds equal to (Strength squared \mtimes 10 lb. \add 100 lb.).
\end{dtable}

\subsection{Encumbrance by Weight}
A creature's Strength determines how much it can carry, as shown on \trefnp{Weight Limits} A creature can carry up to its unencumbered weight limit without any penalty. If it carries more than that, but less than its maximum weight limit, it is encumbered. A creature encumbered by weight halves its Dexterity (if positive), takes a \minus4 armor check penalty, and moves at two-thirds speed (as if it were in heavy armor). This armor check penalty does not stack with the armor check penalty from any armor the creature is wearing.

\parhead{Lifting and Dragging} A character can lift as much as his or her maximum weight limit over his or her head.

A character can lift as much as 1-1/2 his or her maximum weight limit off the ground (the sum of his unencumbered and maximum weight limits). While overloaded in this way, the character is vulnerable (\minus2 to attacks, penalties, and checks) and can only move by spending a full-round action to move 5 feet.

A character can generally push or drag along the ground as much as five times his or her maximum load. Favorable conditions can double these numbers, and bad circumstances can reduce them to one-half or less.

\parhead{Bigger and Smaller Creatures} The figures on \trefnp{Weight Limits} are for Medium bipedal creatures. A larger bipedal creature can carry more weight depending on its size category, as follows: Large \mult2, Huge \mult4, Gargantuan \mult8, Colossal \mult16. A smaller creature can carry less weight depending on its size category, as follows: Small \mult3/4, Tiny \mult1/2, Diminutive \mult1/4, Fine \mult1/8.

Quadrupeds can carry heavier loads than characters can. Instead of the multipliers given above, multiply the value corresponding to the creature's Strength score from Table 9-1 by the appropriate modifier, as follows: Fine \mult1/4, Diminutive \mult1/2, Tiny \mult3/4, Small \mult1, Medium \mult1-1/2, Large \mult3, Huge \mult6, Gargantuan \mult12, Colossal \mult24.

\parhead{Tremendous Strength} For Strength scores not shown on \trefnp{Weight Limits}, find the Strength score between 20 and 29 that has the same number in the ``ones" digit as the creature's Strength score does. Multiply the figures by 4 for every ten points the creature's strength is above the score for that row.

\section{Movement}

\begin{dtable}
\lcaption{Movement and Distance}
\begin{tabularx}{\columnwidth}{>{\lcol}X c c c c}
 & \multicolumn{4}{c}{\x\x\x Speed \x\x\x} \\
 & 15 feet & 20 feet & 30 feet & 40 feet \\
One Round (Tactical)1 &  &  &  &  \\
Walk & 15 ft. & 20 ft. & 30 ft. & 40 ft. \\
Hustle & 30 ft. & 40 ft. & 60 ft. & 80 ft. \\
One Minute (Local) &  &  &  &  \\
Walk & 150 ft. & 200 ft. & 300 ft. & 400 ft. \\
Hustle & 300 ft. & 400 ft. & 600 ft. & 800 ft. \\
One Hour (Overland) &  &  &  &  \\
Walk & 1-1/2 miles & 2 miles & 3 miles & 4 miles \\
Hustle & 3 miles & 4 miles & 6 miles & 8 miles \\
One Day (Overland) &  &  &  &  \\
Walk & 15 miles & 20 miles & 30 miles & 40 miles \\
Hustle & \x & \x & \x & \x \\
\end{tabularx}
\end{dtable}

\begin{dtable}
\lcaption{Hampered Movement}
\begin{tabularx}{\columnwidth}{l >{\lcol}X >{\ccol}p{8em}}
Condition & Example Extra Movement Cost \\
Difficult terrain & Rubble, undergrowth, steep slope, ice, cracked and pitted surface, uneven floor & \mult2 \\
Obstacle\fn{1} & Low wall, deadfall, broken pillar & \mult2 \\
Poor visibility & Darkness or fog & \mult2 \\
Impassable & Floor-to-ceiling wall, closed door, blocked passage & \x \\
\end{tabularx}
1 May require a skill check
\end{dtable}

There are three movement scales in the game, as follows.
\begin{itemize}
\item Tactical, for combat, measured in feet (or squares) per round.
\item Local, for exploring an area, measured in feet per minute.
\item Overland, for getting from place to place, measured in miles per
hour or miles per day.
\end{itemize}

\parhead{Modes of Movement} While moving at the different movement scales, creatures generally walk, hustle.
\subparhead{Walk} A walk represents unhurried but purposeful movement at 3 miles per hour for an unencumbered human.
\subparhead{Hustle} A hustle is a jog at about 6 miles per hour for an unencumbered human. A character moving his or her
speed twice in a single round, or moving that speed in the same round that he or she performs a standard action or another move action is hustling when he or she moves.

\subsection{Tactical Movement}
Use tactical movement for combat. Characters generally don't walk during combat -- they hustle. A character who
moves his or her speed and takes some action, such as attacking or casting a spell, is hustling for about half the round and doing something else for the other half.

\parhead{Hampered Movement} Difficult terrain, obstacles, or poor visibility can hamper movement. When movement is
hampered, each square moved into usually counts as two squares, effectively reducing the distance that a character
can cover in a move.

If more than one condition applies, multiply together all additional costs that apply. (This is a specific exception to the normal rule for doubling).

In some situations, your movement may be so hampered that you don't have sufficient speed even to move 5 feet (1 square). In such a case, you may use a full-round action to move 5 feet (1 square) in any direction, even diagonally. (You can't take advantage of this rule to move through impassable terrain or to move when all movement is prohibited to you, such as while paralyzed.)

You can't sprint or charge through any square that would hamper your movement.

\subsection{Local Movement}
Characters exploring an area use local movement, measured in feet per minute.
\parhead{Walk} A character can walk without a problem on the local scale.
\parhead{Hustle} A character can hustle without a problem on the local scale. See \trefnp{Terrain and Overland Movement}, below, for movement measured in miles per hour.

\subsection{Overland Movement}

\begin{dtable}
\lcaption{Terrain and Overland Movement}
\begin{tabularx}{\columnwidth}{>{\lcol}X c c c}
\thead{Terrain}  & \thead{Highway} & \thead{Road or Trail} & \thead{Trackless} \\
Desert, sandy & \mult1 & \mult1/2 & \mult1/2 \\
Forest & \mult1 & \mult1 & \mult1/2 \\
Hills & \mult1 & \mult3/4 & \mult1/2 \\
Jungle & \mult1 & \mult3/4 & \mult1/4 \\
Moor & \mult1 & \mult1 & \mult3/4 \\
Mountains & \mult3/4 & \mult3/4 & \mult1/2 \\
Plains & \mult1 & \mult1 & \mult3/4 \\
Swamp & \mult1 & \mult3/4 & \mult1/2 \\
Tundra, frozen & \mult1 & \mult3/4 & \mult3/4
\end{tabularx}
\end{dtable}

\begin{dtable}
\lcaption{Mounts and Vehicles}
\begin{tabularx}{\columnwidth}{>{\lcol}X l l}
\thead{Mount/Vehicle} & \thead{Per Hour} & \thead{Per Day} \\
Mount (carrying load) &  &  \\
\tind Light horse or light warhorse & 6 miles & 60 miles \\
\tind Light horse (151-450 lb.)\fn{1} & 4 miles & 40 miles \\
\tind Light warhorse (231-690 lb.)\fn{1} & 4 miles & 40 miles \\
\tind Heavy horse or heavy warhorse & 5 miles & 50 miles \\
\tind Heavy horse (201-600 lb.)\fn{1} & 3-1/2 miles & 35 miles \\
\tind Heavy warhorse (301-900 lb.)\fn{1} & 3-1/2 miles & 35 miles \\
\tind Pony or warpony & 4 miles & 40 miles \\
\tind Pony (76-225 lb.)\fn{1} & 3 miles & 30 miles \\
\tind Warpony (101-300 lb.)\fn{1} & 3 miles & 30 miles \\
\tind Donkey or mule & 3 miles & 30 miles \\
\tind Donkey (51-150 lb.)\fn{1} & 2 miles & 20 miles \\
\tind Mule (231-690 lb.)\fn{1} & 2 miles & 20 miles \\
\tind Dog, riding & 4 miles & 40 miles \\
\tind Dog, riding (101-300 lb.)\fn{1} & 3 miles & 30 miles \\
\tind Cart or wagon & 2 miles & 20 miles \\
\thead{Ship} &  &  \\
\tind Raft or barge (poled or towed)\fn{2} & 1/2 mile & 5 miles \\
\tind Keelboat (rowed)\fn{2} & 1 mile & 10 miles \\
\tind Rowboat (rowed)\fn{2} & 1-1/2 miles & 15 miles \\
\tind Sailing ship (sailed) & 2 miles & 48 miles \\
\tind Warship (sailed and rowed) & 2-1/2 miles & 60 miles \\
\tind Longship (sailed and rowed) & 3 miles & 72 miles \\
\tind Galley (rowed and sailed) & 4 miles & 96 miles \\
\end{tabularx}
1 Quadrupeds, such as horses, can carry heavier loads than characters can. See \trefnp{Weight Limits}, above, for more information. \\
2 Rafts, barges, keelboats, and rowboats are used on lakes and rivers.
If going downstream, add the speed of the current (typically 3 miles per hour) to the speed of the vehicle. In addition to 10 hours of being rowed, the vehicle can also float an additional 14 hours, if someone can guide it, so add an additional 42 miles to the daily distance traveled. These vehicles can't be rowed against any significant current, but they can be pulled upstream by draft animals on the shores.
\end{dtable}

Characters covering long distances cross-country use overland movement. Overland movement is measured in miles per hour or miles per day. A day represents 10 hours of actual travel time. For rowed watercraft, a day represents 10 hours of rowing. For a sailing ship, it represents 24 hours.
\parhead{Walk} A character can walk 10 hours in a day of travel without a problem. Walking for longer than that can wear him or her out (see Forced March, below).
\parhead{Hustle} A character can hustle for 1 hour without a problem. Hustling for a second hour in between sleep cycles deals 1 point of nonlethal damage, and each additional hour deals twice the damage taken during the previous hour of hustling. A character who takes any nonlethal damage from hustling becomes fatigued.
\parhead{Terrain} The terrain through which a character travels affects how much distance he or she can cover in an hour or a day (see \trefnp{Terrain and Overland Movement}). A highway is a straight, major, paved road. A road is typically a dirt track. A trail is like a road, except that it allows only single-file travel and does not benefit a party traveling with vehicles. Trackless terrain is a wild area with no significant paths.
\parhead{Forced March} In a day of normal walking, a character walks for 10 hours. The rest of the daylight time is spent making and breaking camp, resting, and eating.

A character can walk for more than 10 hours in a day by making a forced march. For each hour of marching beyond 10 hours, a Constitution check (DC 10, \plus2 per extra hour) is required. If the check fails, the character takes 1d6 points of nonlethal damage. A character who takes any nonlethal damage from a forced march becomes fatigued. Eliminating the nonlethal damage also eliminates the fatigue. It's possible for a character to march into unconsciousness by pushing himself too hard.

\parhead{Mounted Movement} A mount bearing a rider can move at a hustle. The damage it takes when doing so, however, is lethal damage, not nonlethal damage. The creature can also be ridden in a forced march, but its Constitution checks automatically fail, and, again, the damage it takes is lethal damage. Mounts also become fatigued when they take any damage from hustling or forced marches.

See \trefnp{Mounts and Vehicles} for mounted speeds and speeds for vehicles pulled by draft animals.

\parhead{Waterborne Movement} See \trefnp{Mounts and Vehicles} for speeds for water vehicles.

\section{Exploration}
\subsection{Vision and Light}
Dwarves and half-orcs have darkvision, but everyone else needs light to see by.  In an area of bright light, all characters can see clearly. A creature can't hide in an area of bright light unless it is invisible or has cover.

In an area of shadowy illumination, a character can see dimly. Creatures within this area have concealment relative to that character. A creature in an area of shadowy illumination can make a Hide check to conceal itself.

In areas of darkness, creatures without darkvision are effectively blinded.  A blinded creature moves at half speed and is defenseless, causing it to provoke attacks of opportunity for its actions. All checks and activities that rely on vision (such as reading and visual Perception checks) automatically fail, and any checks related to vision (such as Climb and Sense Motive checks) take a \minus4 penalty. All opponents are considered to be invisible (50\% miss chance) relative to the blinded creature.

Characters with low-light vision (elves, gnomes, and half-elves) can see objects twice as far away as the given radius. Double the effective radius of bright light and of shadowy illumination for such characters.

Characters with darkvision (dwarves and half-orcs) can see lit areas normally as well as dark areas within a radius defined by the ability -- usually, 60 feet. A creature can't hide within that range of a character using darkvision unless it is invisible or has cover. Darkvision does not function if the character is in bright light or is dazzled, and does not resume functioning until 1 round the character leaves the area of bright light or stops being dazzled.

A character with darkvision can see dimly in totally dark areas beyond the range indicated by their ability. They can see in such areas as if the areas were in shadowy illumination.

\subsection{Breaking And Entering}
When attempting to break an object, you have two choices: smash it with a weapon or break it with sheer strength.

\subsubsection{Smashing an Object}
Smashing a weapon or shield with a slashing or bludgeoning weapon is accomplished by the sunder special attack. Smashing an object is a lot like sundering a weapon or shield, except that your attack roll is opposed by the object's AC. Generally, you can smash an object only with a bludgeoning or slashing weapon.

\parhead{Armor Class} Objects are easier to hit than creatures because they usually don't move, but many are tough enough to shrug off some damage from each blow. An object's Armor Class is equal to 10 \add its size modifier \add its Dexterity. An inanimate object has a Dexterity of 0 (\minus10 penalty to AC). Furthermore, if you take a move action to focus on the object, you automatically hit with melee weapons and get a \plus4 bonus with ranged weapons on any attacks you make during your turn against the object.

\parhead{Hardness} Each object has hardness -- a number that represents how well it resists damage. Whenever an object takes damage, subtract its hardness from the damage. Only damage in excess of its hardness is deducted from the object's hit points.
%None of these tables exist currently. Also object hit points are obnoxious.
%(see \trefnp{Common Armor, Weapon, and Shield Hardness and Hit Points}; \trefnp{Substance Hardness and Hit Points}; and \trefnp{Object Hardness and Hit Points}).
\parhead{Hit Points} An object's hit point total depends on what it is made of and how big it is. When an object's hit points reach 0, it's ruined.

Very large objects may have separate hit point totals for different sections.

\subparhead{Energy Attacks} Acid and sonic attacks deal damage to most objects just as they do to creatures; roll damage and apply it normally after a successful hit. Electricity and fire attacks deal half damage to most objects; divide the damage dealt by 2 before applying the hardness. Cold attacks deal one-quarter damage to most objects; divide the damage dealt by 4 before applying the hardness.

\subparhead{Ranged Weapon Damage} Objects take half damage from ranged weapons (unless the weapon is a siege engine or something similar). Divide the damage dealt by 2 before applying the object's hardness.

\subparhead{Ineffective Weapons} Certain weapons just can't effectively deal damage to certain objects.

\subparhead{Immunities} Objects are immune to nonlethal damage and to critical hits. Even animated objects, which are otherwise considered creatures, have these immunities because they are constructs.

\subparhead{Magic Armor, Shields, and Weapons} Each \plus1 of enhancement bonus adds 2 to the hardness of armor, a weapon, or a shield and \plus10 to the item's hit points.

\subparhead{Vulnerability to Certain Attacks} Certain attacks are especially successful against some objects. In such cases, attacks deal double their normal damage and may ignore the object's hardness.

\subparhead{Damaged Objects} A damaged object remains fully functional until the item's hit points are reduced to 0, at which point it is broken. Further damage applies as negative hit points. If an item has more negative hit points than its maximum hit points, it is destroyed.

Broken objects cannot be used for their intended purpose, but still retain enough of their original form to be repaired without too much work. For example, a broken wall lies in pieces on the ground and no longer blocks passage, but can be repaired with no more effort than damaged items. Magic items that are broken retain their magical properties once fixed. Damaged and broken (but not destroyed) objects can be repaired with the Craft skill.

\parhead{Saving Throws} Nonmagical, unattended items never make saving throws. They are considered to have failed their saving throws, so they always are affected by spells. An item attended by a character (being grasped, touched, or worn) makes saving throws as the character (that is, using the character's saving throw bonus).

\par Magic items always get saving throws. A magic item's Fortitude, Reflex, and Will save bonuses are equal to 2 \add one-half its caster level. An attended magic item either makes saving throws as its owner or uses its own saving throw bonus, whichever is better.

\subparhead{Animated Objects} Animated objects count as creatures for purposes of determining their Armor Class (do not treat them as inanimate objects).

\subsubsection{Breaking Items}
When a character tries to break something with sudden force rather than by dealing damage, use a Strength check (rather than an attack roll and damage roll, as with the sunder special attack) to see whether he or she succeeds. The DC depends more on the
construction of the item than on the material.

If an item has lost half or more of its hit points, the DC to break it drops by 2.

Larger and smaller creatures get size bonuses and size penalties on Strength checks to break open doors as follows: Fine \minus16, Diminutive \minus12, Tiny \minus8, Small \minus4, Large \plus4, Huge \plus8, Gargantuan \plus12, Colossal \plus16.

A crowbar or portable ram improves a character's chance of breaking open a door.
