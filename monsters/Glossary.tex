\chapter{Glossary}\label{Glossary}

\glossdef{adept} An adept monster has unique special abilities that it relies on in combat.
For details, see \pcref{Adept}.

\glossdef{adept attribute} The attribute that an \glossterm{adept} monster uses to determine the power of its abilities.
For details, see \pcref[Adp:Adept Attribute]{Adept Attribute}.

\glossdef{cover} Cover represents any obstacle that physically prevents you from striking your target, such as a tree or intervening creature. A creature behind cover is more difficult to attack.
For details, see the Core Rulebook.

\glossdef{darkvision} A creature with darkvision can see in the dark clearly up to a given range.
Beyond that, it can see dimly, treating areas of darkness as shadowy illumination.
Darkvision does not function if a creature is in a brightly lit area, and does not resume functioning until 1 round after the creature leaves the brightly lit area.

\glossdef{incorporeal} An incorporeal creature does not have a body.
It has no Strength or Constitution attributes.
It cannot take any action that requires having a body, and is immune to all such effects.
This includes suffering critical hits, moving objects, grappling, setting off pressure traps, and so on.

An incorporeal creature is immune to all nonmagical effects.
Even magical effects, including spells and attacks with magic weapons, have a 50\% chance to fail.

An incorporeal creature can enter or pass through solid objects, but it must remain adjacent to the object's exterior at all times.
If it is completely inside an object, it cannot see out or attack.
It can fight while partially inside an object, which grants it passive \glossterm{cover} and allows it to attack and see normallly.

\glossdef{physical defenses}[physical defense] Your physical defenses are your Armor and Reflex defenses.
For details, see the Core Rulebook.

\glossdef{power} All monsters have a power that affects the strength of many of their special abilities.
For details, see \pcref{Power}.

\glossdef{range increment} Physical ranged attacks often have a specific range increment.
A range increment is always measured in feet.
You take a \minus2 penalty to accuracy with the ranged attack for each full range increment between you and your target.

\glossdef{spell resistance} A creature with spell resistance can automatically resist spells and magical effects.
It functions like any other defense, except that it only works against magical effects.
To affect the creature with a spell, a caster must make an attack with an accuracy equal to its spellpower.
If the attack beats the creature's spell resistance, the spell works normally.
Otherwise, the spell has no effect on the creature.
