\chapter{Classes}\label{Classes}

Like player characters, monsters have levels in ``classes'' that represent their natural roles and abilities.
Unlike classes for player characters, monster classes do not affect a monster's base defenses.
A monster's base defenses are determined by its type (see \pcref{Types}).
There are four monster classes: adept, behemoth, and slayer.

\section{Class Overviews}

    \parhead{Adept} An adept monster has unique special abilities that it relies on in combat.
    Allips, medusas, and nymphs are adept monsters.
    \parhead{Behemoth} A behemoth monster is extraordinarily resilient or durable.
    Behemoth monsters are typically larger than other monsters.
    Giants, hydras, and trolls are behemoth monsters.
    \parhead{Slayer} A slayer monster has abilities that allow it to move and kill its foes quickly.
    Ankhegs, minotaurs, and winter wolves are slayer monsters.

\section{Adept}\label{Adept}

    \begin{dtable}
        \lcaption{Adept Progression}
        \begin{dtabularx}{\columnwidth}{>{\ccol}p{\levelcol} >{\ccol}p{\babcolgood} >{\lcol}X}
            \tb{Level} & \tb{Combat Prowess} & \tb{Special} \\
            \hline
            \adeptProgressionRow{1}  & \tdash \\
            \adeptProgressionRow{2}  & \tdash \\
            \adeptProgressionRow{3}  & \tdash \\
            \adeptProgressionRow{4}  & \tdash \\
            \adeptProgressionRow{5}  & \tdash \\
            \adeptProgressionRow{6}  & \tdash \\
            \adeptProgressionRow{7}  & \tdash \\
            \adeptProgressionRow{8}  & \tdash \\
            \adeptProgressionRow{9}  & \tdash \\
            \adeptProgressionRow{10} & \tdash \\
            \adeptProgressionRow{11} & \tdash \\
            \adeptProgressionRow{12} & \tdash \\
            \adeptProgressionRow{13} & \tdash \\
            \adeptProgressionRow{14} & \tdash \\
            \adeptProgressionRow{15} & \tdash \\
            \adeptProgressionRow{16} & \tdash \\
            \adeptProgressionRow{17} & \tdash \\
            \adeptProgressionRow{18} & \tdash \\
            \adeptProgressionRow{19} & \tdash \\
            \adeptProgressionRow{20} & \tdash \\
        \end{dtabularx}
    \end{dtable}

    \subsection{Base Class Abilities}
        A monster with adept as a base class gains the following abilities.

        \classbasics{Skill Points} 5.

        \classbasics{Combat Prowess} \plus2.

        \cf{Adp}{Weapon and Armor Proficiency}
        Adepts are proficient with any number of weapon groups.
        If a slayer is not described as using a weapon from a weapon group, it is not proficient with that weapon group.

    \subsection{Class Abilities}

        \cf{Adp}{Adept Attribute} An adept chooses one of its attributes to be its adept attribute.
        Its adept attribute affects the strength of its magical and supernatural abilities.

        \cf{Adp}{Adept Points} An adept has a number of adept points equal to its adept level or half its adept attribute, whichever is higher.
        Adept points can be spent to use various magical and supernatural abilities, depending on which traits the monster has.
        For details, see \pcref{Traits}.
        Spent adept points return after 1 hour.

\section{Behemoth}\label{Behemoth}

    \begin{dtable}
        \lcaption{Behemoth Progression}
        \begin{dtabularx}{\columnwidth}{>{\ccol}p{\levelcol} >{\ccol}p{\babcolgood} >{\lcol}X}
            \tb{Level} & \tb{Combat Prowess} & \tb{Special} \\
            \hline
            \behemothProgressionRow{1}  & \tdash \\
            \behemothProgressionRow{2}  & \tdash \\
            \behemothProgressionRow{3}  & \tdash \\
            \behemothProgressionRow{4}  & \tdash \\
            \behemothProgressionRow{5}  & \tdash \\
            \behemothProgressionRow{6}  & \tdash \\
            \behemothProgressionRow{7}  & \tdash \\
            \behemothProgressionRow{8}  & \tdash \\
            \behemothProgressionRow{9}  & \tdash \\
            \behemothProgressionRow{10} & \tdash \\
            \behemothProgressionRow{11} & \tdash \\
            \behemothProgressionRow{12} & \tdash \\
            \behemothProgressionRow{13} & \tdash \\
            \behemothProgressionRow{14} & \tdash \\
            \behemothProgressionRow{15} & \tdash \\
            \behemothProgressionRow{16} & \tdash \\
            \behemothProgressionRow{17} & \tdash \\
            \behemothProgressionRow{18} & \tdash \\
            \behemothProgressionRow{19} & \tdash \\
            \behemothProgressionRow{20} & \tdash \\
        \end{dtabularx}
    \end{dtable}

    \subsection{Base Class Abilities}
        A monster with behemoth as a base class gains the following abilities.

        \classbasics{Skill Points} 5.

        \classbasics{Combat Prowess} \plus2.

        \cf{Beh}{Weapon and Armor Proficiency}
        Behemoths are proficient with any number of weapon groups.
        If a slayer is not described as using a weapon from a weapon group, it is not proficient with that weapon group.

    \subsection{Class Abilities}
        Behemoths have no special class abilities.
        However, many traits that increase a monster's size and resilience require behemoth levels.
        See \pcref{Traits} for details.

\section{Slayer}\label{Slayer}

    \begin{dtable}
        \lcaption{Slayer Progression}
        \begin{dtabularx}{\columnwidth}{>{\ccol}p{\levelcol} >{\ccol}p{\babcolgood} >{\lcol}X}
            \tb{Level}                 & \tb{Combat Prowess} \\
            \hline
            \slayerProgressionRow{1}  & \tdash \\
            \slayerProgressionRow{2}  & \tdash \\
            \slayerProgressionRow{3}  & \tdash \\
            \slayerProgressionRow{4}  & \tdash \\
            \slayerProgressionRow{5}  & \tdash \\
            \slayerProgressionRow{6}  & \tdash \\
            \slayerProgressionRow{7}  & \tdash \\
            \slayerProgressionRow{8}  & \tdash \\
            \slayerProgressionRow{9}  & \tdash \\
            \slayerProgressionRow{10} & \tdash \\
            \slayerProgressionRow{11} & \tdash \\
            \slayerProgressionRow{12} & \tdash \\
            \slayerProgressionRow{13} & \tdash \\
            \slayerProgressionRow{14} & \tdash \\
            \slayerProgressionRow{15} & \tdash \\
            \slayerProgressionRow{16} & \tdash \\
            \slayerProgressionRow{17} & \tdash \\
            \slayerProgressionRow{18} & \tdash \\
            \slayerProgressionRow{19} & \tdash \\
            \slayerProgressionRow{20} & \tdash \\
        \end{dtabularx}
    \end{dtable}

    \subsection{Base Class Abilities}
        A monster with slayer as a base class gains the following abilities.

        \classbasics{Skill Points} 5.

        \classbasics{Combat Prowess} \plus2.

        \cf{Sly}{Weapon and Armor Proficiency}
        Slayers are proficient with any number of weapon groups.
        If a slayer is not described as using a weapon from a weapon group, it is not proficient with that weapon group.

    \subsection{Class Abilities}
        Slayers have no special class abilities.
        However, many traits that increase a monster's damage in combat require slayer levels.
        See \pcref{Traits} for details.
