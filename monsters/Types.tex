\chapter{Types}\label{Types}

Monsters are divided into types.
Monster types are like races for player characters, but they have a stronger effect on a monster's statistics.
There are eight monster types: aberrations, animals, constructs, humanoids, magical beasts, monsterous humanoids, outsiders, and undead.

\section{Type Overviews}

    \parhead{Aberration} An aberration is a monster with a bizarre anatomy, strange abilities, an alien mindset, or a similar separation from ``ordinary'' monsters.
    Aboleths and gibbering mouthers are aberrations.
    \parhead{Animal} An animal is a living, nonhumanoid creature with limited intelligence and no inherently magical abilities.
    Dogs and bears are animals.
    \parhead{Construct} A construct is an object or elemental force imbued with a semblance of life.
    Elementals, golems, and oozes are constructs.
    \parhead{Humanoid} A humanoid is a living creature with a human-shaped body: two arms, two legs, one head, and so on.
    Humanoids typically have only limited magical abilities, if any, and are always similar in size to humans.
    Humans, elves, and goblins are humanoids.
    \parhead{Magical Beast} A magical beast is a living, nonhumanoid creature with monstrous features, supernatural abilities, or extraordinary intelligence.
    Basilisks, griffons, and worgs are magical beasts.
    \parhead{Monstrous Humanoid} A monstrous humanoid is a living creature a generally human-shaped body as well as monstrous or animalistic features or significant magical abilities.
    Giants, medusas, and trolls are monstrous humanoids.
    \parhead{Outsider} An outsider is a creature composed of planar material from a plane other than the Material Plane.
    Angels, devils, and demons are outsiders.
    \parhead{Undead} An undead is a previously living creature reanimated by spiritual or supernatural powers.
    Ghosts, skeletons, and zombies are undead.

\section{Defense Progressions}
    The base defenses of monsters are determined by their type, not their class.
    Each monster type has one of three defense progressions: Good, Average, or Poor.
    The effects of these progressions on a monster's base defense progression are shown on \trefnp{Defense Progressions}.

    \parhead{Defense Bonuses}
    In addition to its base defense progression, a monster always gains a separate defense bonus based on its progression type.
    A Good progression grants a \plus4 bonus, an Average progression grants a \plus2 bonus, and a Poor progression grants no bonus.
    This is equivalent to the base class defense bonuses gained by characters.

    \begin{dtable}
        \lcaption{Defense Progressions}
        \begin{dtabularx}{\columnwidth}{>{\ccol}p{\levelcol} *{3}{>{\ccol}X}}
            \tb{Level} & \tb{Good} & \tb{Average} & \tb{Poor} \\
            \hline
            \demoProgressionRow{1} \\
            \demoProgressionRow{2} \\
            \demoProgressionRow{3} \\
            \demoProgressionRow{4} \\
            \demoProgressionRow{5} \\
            \demoProgressionRow{6} \\
            \demoProgressionRow{7} \\
            \demoProgressionRow{8} \\
            \demoProgressionRow{9} \\
            \demoProgressionRow{10} \\
            \demoProgressionRow{11} \\
            \demoProgressionRow{12} \\
            \demoProgressionRow{13} \\
            \demoProgressionRow{14} \\
            \demoProgressionRow{15} \\
            \demoProgressionRow{16} \\
            \demoProgressionRow{17} \\
            \demoProgressionRow{18} \\
            \demoProgressionRow{19} \\
            \demoProgressionRow{20} \\
        \end{dtabularx}
    \end{dtable}

\section{Type Descriptions}

    \subsection{Aberration}
        \parhead{Fortitude} Poor.
        \parhead{Reflex} Poor.
        \parhead{Mental} Good.
        \parhead{Darkvision} Aberrations have \glossterm{darkvision} with a range of 50 feet.

    \subsection{Animal}
        \parhead{Fortitude} Average.
        \parhead{Reflex} Average.
        \parhead{Mental} Poor.
        \parhead{Limited Intelligence} Animals have an Intelligence of \minus8 or lower.
        An animal whose Intelligence permanently increases to \minus7 or higher becomes a magical beast.

    \subsection{Construct}
        \parhead{Fortitude} Average.
        \parhead{Reflex} Poor.
        \parhead{Mental} Good.
        \parhead{Artificial Body} Constructs are immune to poison, disease, and all \glossterm{Vivimancy} effects.
        They do not recover hit points by resting, and effects that restore hit points have no effect on them unless they specifically affect constructs.
        All constructs can restore hit points in some manner, as stated in their monster description.
        \parhead{Mindless} Constructs are immune to Mind effects.
        \parhead{Nonliving} Constructs are not alive.
        They are immune to effects that affect living creatures.

    \subsection{Humanoid}
        \parhead{Fortitude} Poor.
        \parhead{Reflex} Poor.
        \parhead{Mental} Poor.

    \subsection{Magical Beast}
        \parhead{Fortitude} Average.
        \parhead{Reflex} Average.
        \parhead{Mental} Poor.
        \parhead{Low-Light Vision} Magical beasts have \glossterm{low-light vision}.

    \subsection{Outsider}
        \parhead{Fortitude} Average.
        \parhead{Reflex} Average.
        \parhead{Mental} Average.
        \parhead{Darkvision} Outsiders have \glossterm{darkvision} with a range of 50 feet.
        \parhead{Extraplanar Body} The body of an outsider is made of planar essence.
        Outsiders cannot be resurrected after dying.

    \subsection{Undead}
        \parhead{Fortitude} Poor.
        \parhead{Reflex} Poor.
        \parhead{Mental} Good.
        \parhead{Darkvision} Undead have \glossterm{darkvision} with a range of 50 feet.
        \parhead{Necromantic Body} Undead are immune to poison and disease.
        They do not recover hit points by resting, and effects that restore hit points have no effect on them unless they specifically affect undead.
        \parhead{Nonliving} Undead are not alive.
        They are immune to effects that affect living creatures.
