\chapter{Basic Mechanics}

\section{Starting Abilities}

    \subsection{Attributes}
    Each of a monster's starting attributes can range from \minus9 to 5, as appropriate for the type of monster.
    In general, a monster with higher starting attributes will be slightly stronger, but not all monsters need to start with the same total number of starting attributes.

    \subsection{Senses}
        In addition to the normal senses provided by a monster's type (see \pcref{Types}), it may have any number of special senses from the following list:
        \begin{itemize}
            \item Blindsense (50 ft.\ range)
            \item Darkvision (50 ft.\ range)
            \item Low-light vision
            \item Scent
            \item Tremorsense (50 ft.\ range)
        \end{itemize}

        These senses are too minor to be worth a trait.
        More powerful senses can be gained with the Superior Senses trait (see page \featpref{Superior Senses}).

    \subsection{Skills}
        A monster may acquire training in skills using the skill points from its base class.
        Many monsters do not use all of their available skill points.

\section{Level Advancement}\label{Level Advancement}

    As a monster becomes stronger, it gains levels.
    When a monster gains a level, it increases its level in a class of its choice, and gains the benefits described for that level in class.

    A monster that increases its level also gains the following benefits.
    \begin{itemize}
        \item Every odd level, including 1st level, it gains a trait (see \pcref{Traits}).
            At 1st level, it gains a bonus trait, for a total of two traits at 1st level.
        \item Every level after 1st level, it increases two diffent attributes of its choice by one.
        \item Every level after 1st level, it can change any or all of its current traits to new traits.
            It must currently meet the prerequisites for any new traits, but it does not have to have met those prerequisites when that trait was first acquired.
            This can allow it to change its 1st level trait into a new trait that it did not meet the prerequisites for at 1st level.
            However, any traits it gained with special restrictions, such as traits that must be associated with a certain class, can only be changed into other traits that fulfill those same restrictions.
    \end{itemize}
