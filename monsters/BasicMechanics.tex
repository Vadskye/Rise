\chapter{Basic Mechanics}

\section{Starting Abilities}\label{Starting Abilities}

    A monster can have many abilities that are not specifically defined by its traits.
    These abilities should be appropriate for the monster's nature and narrative role.
    While the rules do not prevent creating a monster with every ability described below, such a creature would be unnecessary complex and would be difficult to integrate into a realistic world.

    \subsection{Attributes}
        Each of a monster's starting attributes can range from \minus9 to 3, as appropriate for the type of monster.
        A monster's attributes scale with level in the same way as character attributes.
        A monster can also have up to two attributes starting at 4 or 5.
        In general, a monster with higher starting attributes will be slightly stronger, but not all monsters need to start with the same starting attribute total.

    \subsection{Combat Prowess}
        A monster's combat prowess is normally equal to four-fifths of its level \add 2.
        Some abilities can increase or decrease a monster's combat prowess.

    \subsection{Communication}
        A monster may know any number of languages.
        In addition, a monster may also have any number of communication abilities from the following list.
        \begin{itemize}
            \item Telepathy (100 ft. range, willing only)
        \end{itemize}

    \subsection{Damage Resistances}
        A monster may naturally have damage reduction against up to three specific damage types.
        It can resist bludgeoning, piercing, or slashing damage, but not all three.
        Its damage reduction is equal to its power.
        In addition, it may be completely immune to a single non-physical damage type.
        A monster may also be \glossterm{vulnerable} to any number of damage types.

    \subsection{Movement Speeds}
        A monster can have a land speed up to ten feet faster than the normal land speed for its size without spending a trait.
        Some monsters have slower land speeds, or no land speed at all.
        In addition to a land speed, a monster may have any number of movement speeds from the following list.
        \begin{itemize}
            \item Climb speed (up to a maximum of ten feet faster than a size-appropriate land speed)
            \item Swim speed (up to a maximum of ten feet faster than a size-appropriate land speed)
        \end{itemize}

        These movement speeds are too minor to be worth a trait.
        More powerful movement speeds can be gained with traits (see \pcref{Traits}).

    \subsection{Natural Armor}
        Most monsters have tough hide, armored skin, or similar abilities that make them more difficult to harm.
        A monster can naturally have up to a \plus4 bonus to Armor defense without using a trait.
        Higher bonuses to Armor defense require traits.

    \subsection{Natural Weapons}
        A monster can have any number of natural weapons.

    \subsection{Power}\label{Power}
        All monsters have a \glossterm{power}.
        A monster's power affects the strength of many of its special abilities.
        Its power is normally equal to its level \add its \glossterm{challenge rating}.
        Some abilities can increase a monster's power.

    \subsection{Senses}
        In addition to the normal senses provided by a monster's type (see \pcref{Types}), it may have any number of special senses from the following list.
        \begin{itemize}
            \item Blindsense (50 ft.\ range)
            \item Darkvision (50 ft.\ range)
            \item Low-light vision
            \item Scent
            \item Tremorsense (50 ft.\ range)
        \end{itemize}

        These senses are too minor to be worth a trait.
        More powerful senses can be gained with the Superior Senses trait (see page \featpref{Superior Senses}).

    \subsection{Skills}
        A monster has 10 skill points that it can spend in any skill.
        Most monsters do not use all of their available skill points.

    \subsection{Traits}
        A monster has a single trait (see \pcref{Traits}).
        It must meet all prerequisites for that trait.

\section{Level Advancement}\label{Level Advancement}

    As a monster becomes stronger, it gains levels.
    Monsters do not gain levels in specific classes, like characters.
    A monster simply has a single level that describes its overall strength.

    Whenever a monster gains a level, it can change any or all of its current traits to new traits.
    It must currently meet the prerequisites for any new traits, but it does not have to have met those prerequisites when that trait was first acquired.
    However, any traits it gained with special restrictions, such as traits that must be associated with a certain template, can only be changed into other traits that fulfill those same restrictions.

\section{Challenge Rating}
    All monsters have a \glossterm{challenge rating} (CR) that describes how difficult it is to defeat.
    A monster is an appropriate challenge for a number of player characters equal to its challenge rating, assuming both are the same level.
    Most monsters have a challenge rating of 1.
    Extraordinarily powerful monsters, such as dragons, can have a higher challenge rating.
    Monsters with a high CR automatically gain special abilities, as described below.
    A monster cannot have a CR higher than 4.

    \subsection{Low CR}
        Some monsters, such as horde monsters, have a CR of less than 1.

        \parhead{Weak} The monster has one quarter of its normal hit points.

        \parhead{Rapid Demise} Whenever the monster takes damage in excess of its hit points, it immediately dies.

    \subsection{CR 1}
        Most monsters have a CR of 1.

        \parhead{Rapid Defeat} Whenever the monster takes critical damage and has no hit points remaining, it falls unconscious.

    \subsection{CR 2}
        Powerful monsters, such as the leaders of warbands, can have a CR of 2.
        This grants the following abilities.

        \parhead{Legend Points} The monster gains a number of \glossterm{legend points} equal to one quarter of its level (minimum 1).

        \parhead{Durability} The monster gains bonus hit points equal half its normal hit points.

        \parhead{Alacrity} The monster can choose to take two standard actions during the action phase.
        The extra action is delayed.
        However, it can still only make a single \glossterm{standard attack} during each action phase.
        If it uses this ability, it cannot use it again during the next round.

    \subsection{CR 3}

        \parhead{Legend Points} The monster gains a number of \glossterm{legend points} equal to one third of its level (minimum 1).

        \parhead{Legendary Aura} Enemies within \rnglong range of the monster can only use one legend point per round.

        \parhead{Durability} The monster has twice as many hit points as it otherwise would.

        \parhead{Alacrity} The monster takes two standard actions during each action phase.
        The extra action is delayed.
        However, it can still only make a single \glossterm{standard attack} during each action phase.

    \subsection{CR 4}

        \parhead{Legend Points} The monster gains a number of \glossterm{legend points} equal to one third of its level (minimum 1).

        \parhead{Legendary Aura} Enemies within \rnglong range of the monster cannot use legend points.

        \parhead{Durability} The monster has three times as many hit points as it otherwise would.

        \parhead{Alacrity} The monster takes three standard actions during each action phase.
        However, it can still only make a single \glossterm{standard attack} during each action phase.
