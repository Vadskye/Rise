\chapter{Basic Mechanics}

\section{Starting Abilities}

    A monster can have many abilities that are not specifically defined by its traits.
    These abilities should be appropriate for the monster's nature and narrative role.
    While the rules do not prevent creating a monster with every ability described below, such a creature would be unnecessary complex and would be difficult to integrate into a realistic world.

    \subsection{Attributes}
        Each of a monster's starting attributes can range from \minus9 to 3, as appropriate for the type of monster.
        A monster's attributes scale with level in the same way as character attributes.
        A monster can also have up to two attributes starting at 4 or 5.
        In general, a monster with higher starting attributes will be slightly stronger, but not all monsters need to start with the same starting attribute total.

    \subsection{Communication}
        A monster may know any number of languages.
        In addition, a monster may also have any number of communication abilities from the following list.
        \begin{itemize}
            \item Telepathy (100 ft. range, willing only)
        \end{itemize}

    \subsection{Movement Speeds}
        A monster can have a land speed up to ten feet faster than the normal land speed for its size without spending a trait.
        Some monsters have slower land speeds, or no land speed at all.
        In addition to a land speed, a monster may have any number of movement speeds from the following list.
        \begin{itemize}
            \item Climb speed (up to a maximum of ten feet faster than a size-appropriate land speed)
            \item Swim speed (up to a maximum of ten feet faster than a size-appropriate land speed)
        \end{itemize}

        These movement speeds are too minor to be worth a trait.
        More powerful movement speeds can be gained with traits (see \pcref{Traits}).

    \subsection{Natural Armor}
        Most monsters have tough hide, armored skin, or similar abilities that make them more difficult to harm.
        A monster can naturally have up to a \plus4 bonus to Armor defense without using a trait.
        Higher bonuses to Armor defense require traits.

    \subsection{Natural Weapons}
        A monster can have any number of natural weapons.

    \subsection{Power}\label{Power}
        All monsters have a \glossterm{power}.
        A monster's power affects the strength of many of its special abilities.
        At 1st level, a monster's power is equal to its level \add 1.
        Its power increases with level, as described at \pcref{Level Advancement}.

    \subsection{Senses}
        In addition to the normal senses provided by a monster's type (see \pcref{Types}), it may have any number of special senses from the following list.
        \begin{itemize}
            \item Blindsense (50 ft.\ range)
            \item Darkvision (50 ft.\ range)
            \item Low-light vision
            \item Scent
            \item Tremorsense (50 ft.\ range)
        \end{itemize}

        These senses are too minor to be worth a trait.
        More powerful senses can be gained with the Superior Senses trait (see page \featpref{Superior Senses}).

    \subsection{Skills}
        A monster may acquire training in skills using the skill points from its base class.
        Many monsters do not use all of their available skill points.

\section{Level Advancement}\label{Level Advancement}

    As a monster becomes stronger, it gains levels.
    When a monster gains a level, it increases its level in a class of its choice, and gains the benefits described for that level in class.

    A monster that increases its level also gains the following benefits.
    \begin{itemize}
        \item Every odd level, including 1st level, it gains a trait (see \pcref{Traits}).
            At 1st level, it gains a bonus trait, for a total of two traits at 1st level.
        \item Every level after 1st level, it increases two diffent attributes of its choice by one.
        \item Every level after 1st level, it can change any or all of its current traits to new traits.
            It must currently meet the prerequisites for any new traits, but it does not have to have met those prerequisites when that trait was first acquired.
            This can allow it to change its 1st level trait into a new trait that it did not meet the prerequisites for at 1st level.
            However, any traits it gained with special restrictions, such as traits that must be associated with a certain class, can only be changed into other traits that fulfill those same restrictions.
        \item At 5th level, and every 5 levels thereafter, a monster's \glossterm{power} increases by 1, in addition to the normal increase from increasing its level.
            At 20th level, a monster's power is equal to its level \add 5.
    \end{itemize}
