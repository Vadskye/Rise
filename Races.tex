\chapter{Races}

Each character has a race.

\section{Racial Traits}

\subsection{Racial Bonus Feats}
Each race grants a bonus feat at 1st level. Most races can only choose from a small group of feats, listed in the description of the race. A character must meet any prerequisites for these bonus feats, as normal.

\subsection{Favored Weapons}
The names of some exotic weapons, such as the orcish double axe, include the name of a race. Members of the named race can treat those weapons as if they were martial weapons rather than exotic weapons.

\subsection{Race and Languages}
All characters know how to speak Common. A dwarf, elf, gnome, half-elf, half-orc, or halfling also speaks a racial language, as appropriate. A character who has an Intelligence bonus at 1st level speaks other languages as well, one extra language per point of Intelligence bonus as a starting character.

\parhead{Literacy} Any character except a barbarian can read and write all the languages he or she speaks.

\parhead{Class-Related Languages} Clerics, druids, and wizards can choose certain languages as bonus languages even if they're not on the lists found in the race descriptions. These class-related languages are as follows:
\subparhead{Cleric} Abyssal, Celestial, Infernal.
\subparhead{Druid} Sylvan.
\subparhead{Wizard} Draconic.

\subsection{Small Characters}
A Small character gets a \plus1 size bonus to Armor Class, a \plus1 size bonus on attack rolls, and a \plus2 size bonus on Stealth checks. A Small character's carrying capacity is three-quarters of that of a Medium character.

A Small character generally moves about two-thirds as fast as a Medium character.

A Small character must use smaller weapons than a Medium character.

\section{Humans}
\begin{itemize*}
\item Medium: As Medium creatures, humans have no special bonuses or penalties due to their size.
\item Human base land speed is 30 feet.
\item Humans can choose any feat for their racial bonus feat.
 \item 2 extra skill points at 1st level.
\item Automatic Language: Common. Bonus Languages: Any (other than secret languages, such as Druidic). See the Speak Language skill.
\end{itemize*}

\section{Dwarves}
\begin{itemize*}
\item \plus1 Constitution, \minus1 Dexterity.
\item Medium: As Medium creatures, dwarves have no special bonuses or penalties due to their size.
\item Dwarf base land speed is 20 feet.
\item Dwarves do not move slower in heavy armor, and can run normally in both medium and heavy armor.
\item Darkvision: Dwarves can see in the dark clearly up to 60 feet.   Beyond that, they can see dimly, treating areas of darkness as shadowy illumination. Darkvision does not function if a dwarf is in a brightly lit area or is dazzled, and does not resume functioning until 1 round after the dwarf leaves the brightly lit area or stops being dazzled.
\par Darkvision is black and white only, but it is otherwise like normal sight, and dwarves can function just fine with no light at all.
\item Stability: A dwarf gains a \plus2 competence bonus to Maneuver Class to resist being bull rushed, overrun, or tripped when standing on the ground (but not when climbing, flying, riding, or otherwise not standing firmly on the ground).
\item Dwarves can choose any of the following feats for their racial bonus feat: Armor Proficiency (any), Diehard, Dwarven Resilience, Endurance, Giantfighter, Great Fortitude, Perfect Health, Stonecunning, Toughness, Weapon Proficiency (axes)
\item Automatic Languages: Common and Dwarven. Bonus Languages: Giant, Gnome, Goblin, Orc, Terran, and Undercommon.
\end{itemize*}

\section{Elves}
\begin{itemize*}
\item \plus1 Dexterity, \minus1 Constitution.
\item Medium: As Medium creatures, elves have no special bonuses or penalties due to their size.
\item Elf base land speed is 30 feet.
\item Immunity to sleep effects.
\item Low-light Vision: An elf can see twice as far as a human in starlight, moonlight, torchlight, and similar conditions of poor illumination. She retains the ability to distinguish color and detail under these conditions.
 \item Trance: Elves that trance for 4 hours gain the same benefit as humans do from 8 hours of sleep. An elf in trance may make Listen checks at a \minus5 penalty.
\item Keen Senses: \plus2 competence bonus on Perception checks. An elf with 5 or more ranks in Perception who comes within 10 feet of a secret or concealed door may make a Perception check to notice it as if she were actively searching. 
\item Elves can choose any of the following feats for their racial bonus feat: Dilettante, Focused Mind, Improved Initiative, Lightning Reflexes, Swift, Weapon Proficiency (bows, heavy blades, or light blades)
\item Automatic Languages: Common and Elven. Bonus Languages: Draconic, Gnoll, Gnome, Goblin, Orc, and Sylvan.
\end{itemize*}

\section{Gnomes}
\begin{itemize*}
\item \plus1 Constitution, \minus1 Strength.
\item Small: As a Small creature, a gnome gains a \plus1 size bonus to Armor Class, a \plus1 size bonus on attack rolls, and a \plus2 size bonus on Stealth checks. However, he takes a \minus4 penalty to combat maneuver attack and defense, he uses smaller weapons than humans use, and his lifting and carrying limits are three-quarters of those of a Medium character.
\item Gnome base land speed is 20 feet.
\item Low-light Vision: A gnome can see twice as far as a human in starlight, moonlight, torchlight, and similar conditions of poor illumination. He retains the ability to distinguish color and detail under these conditions.
\item A gnome with a Charisma score of at least 0 gains spell-like abilities. These can be used a number of times per day equal to half the home's character level \add half Charisma:  \spell{create sound}, \spell{dancing lights}, and \spell{prestidigitation}. The gnome's caster level with these abilities is equal to the gnome's character level, and the save DC is equal to 10 \add half character level \add Charisma.

\item Gnomes can choose any magic feat or any gnomish racial feat for their racial bonus feat.
\item Automatic Languages: Common and Gnome. Bonus Languages: Draconic, Dwarven, Elven, Giant, Goblin, and Orc.
\end{itemize*}

\section{Half-Elves}
\begin{itemize*}
\item Medium: As Medium creatures, half-elves have no special bonuses or penalties due to their size.
\item Half-elf base land speed is 30 feet.
\item Immunity to sleep effects.
\item Low-light Vision: A half-elf can see twice as far as a human in starlight, moonlight, torchlight, and similar conditions of poor illumination. She retains the ability to distinguish color and detail under these conditions.
 \item Skill Affinity: Half-elves can master skills with particular ease. If a half-elf has a skill as a class skill from any class, it is treated as a class skill for all of his classes. For example, a half-elf rogue 1 / fighter 8 with 2 skill points in Stealth would have 12 ranks in Stealth.
\item Elven Blood: For all effects related to race, a half-elf is considered both a human and an elf.
\item Half-elves can choose any skill feat or any elven or human racial feat for their racial bonus feat.
\item Automatic Languages: Common and Elven. Bonus Languages: Any (other than secret languages, such as Druidic).
\end{itemize*}

\section{Half-Orcs}
\begin{itemize*}
\item \plus1 Strength, \minus1 Intelligence, \minus1 Wisdom.
\item Medium: As Medium creatures, half-orcs have no special bonuses or penalties due to their size.
\item Half-orc base land speed is 30 feet.
\item Darkvision: Half-orcs (and orcs) can see clearly in the dark up to 60 feet.  Beyond that, they can see dimly, treating areas of darkness as shadowy illumination. Darkvision does not function if an orc is in a brightly lit area or is dazzled, and does not resume functioning until 1 round after the orc leaves the brightly lit area or stops being dazzled.
\par Darkvision is black and white only, but it is otherwise like normal
sight, and half-orcs can function just fine with no light at all.
\item \plus2 competence bonus on Intimidate checks, but a \minus2 penalty on Persuasion checks.
\item Orc Blood: For all effects related to race, a half-orc is considered both a human and an orc.
\item Half-orcs can choose any combat feat or any orc or human racial feat for their racial bonus feat.
\item Automatic Languages: Common and Orc. Bonus Languages: Draconic, Giant, Gnoll, Goblin, and Abyssal.
\end{itemize*}

\section{Halflings}
\begin{itemize*}
\item \plus1 Dexterity, \minus1 Strength.
\item Small: As a Small creature, a halfling gains a \plus1 size bonus to Armor Class, a \plus1 size bonus on attack rolls, and a \plus2 size bonus on Stealth checks. However, she takes a \minus4 penalty to combat maneuver attack and defense, she uses smaller weapons than humans use, and her lifting and carrying limits are three-quarters of those of a Medium character.
\item Halfling base land speed is 20 feet.
\item \plus1 competence bonus on all saving throws.
\item Halflings can choose any of the following feats for their racial bonus feat: Athletic, Giantfighter, Weapon Proficiency (thrown)
\item Automatic Languages: Common and Halfling. Bonus Languages: Dwarven, Elven, Gnome, Goblin, and Orc.
\end{itemize*}
