\chapter{Races and Backgrounds}

Each character has a race.

\section{Racial Traits}

\subsection{Racial Bonus Feats}
Each race grants a bonus feat at 1st level. Most races can only choose from a small group of feats, listed in the description of the race. A character must meet any prerequisites for these bonus feats, as normal.

\subsection{Favored Weapons}
The names of some exotic weapons, such as the orcish double axe, include the name of a race. Members of the named race are treated as being proficient with exotic weapons for the purpose of wielding those weapons.

\subsection{Race and Languages}
All characters know how to speak Common. A dwarf, elf, gnome, half-elf, half-orc, or halfling also speaks a racial language, as appropriate. For each point of Intelligence a character has at 1st level, they also know one additional language.

\parhead{Literacy} Any character can read and write all the languages he or she speaks.

\parhead{Class-Related Languages} Clerics, druids, and wizards can choose certain languages as bonus languages even if they're not on the lists found in the race descriptions. These class-related languages are as follows:
\subparhead{Cleric} Abyssal, Celestial, Infernal.
\subparhead{Druid} Sylvan.
\subparhead{Wizard} Draconic.

\subsection{Small Characters}\label{Small Characters}
A Small character has the following effects based on their size.
  \begin{itemize} 
    \item \minus4 penalty to maneuver attack and defense.
    \item \plus1 bonus to other physical attacks and defenses.
    \item \plus4 bonus to Stealth checks.
    \item Carrying capacity is three-quarters that of a Medium character (see \pcref{Encumbrance}).
  \end{itemize}

In addition, a Small character generally moves about two-thirds as fast as a Medium character. A Small character must also use smaller weapons than a Medium character.

\section{Race Descriptions}

\subsection{Humans}
\begin{itemize}
\item Medium: As Medium creatures, humans have no special bonuses or penalties due to their size.
\item Human base land speed is 30 feet.
\item Humans can choose any feat for their racial bonus feat.
\item 2 extra skill points at 1st level.
\item Automatic Language: Common. Bonus Languages: Any (other than secret languages, such as Druidic). See the Speak Language skill.
\end{itemize}

\subsection{Dwarves}
\begin{itemize}
\item \plus1 Constitution, \minus1 Dexterity.
\item Medium: As Medium creatures, dwarves have no special bonuses or penalties due to their size.
\item Dwarf base land speed is 20 feet.
\item Dwarves do not move slower in heavy armor, and can run normally in both medium and heavy armor.
\item Darkvision: Dwarves can see in the dark clearly up to 50 feet.   Beyond that, they can see dimly, treating areas of darkness as shadowy illumination. Darkvision does not function if a dwarf is in a brightly lit area or is dazzled, and does not resume functioning until 1 round after the dwarf leaves the brightly lit area or stops being dazzled.
\par Darkvision is black and white only, but it is otherwise like normal sight, and dwarves can function just fine with no light at all.
\item Stability: A dwarf gains a \plus2 bonus to maneuver defense against overrun, shove, and trip attacks when standing on the ground (but not when climbing, flying, riding, or otherwise not standing firmly on the ground).
\item Dwarves can choose any of the following feats for their racial bonus feat: Armor Proficiency (any), Endurance, Diehard, Dwarven Resilience, Giantfighter, Great Fortitude, Perfect Health, Stonecunning, Toughness, Weapon Proficiency (axes)
\item Automatic Languages: Common and Dwarven. Bonus Languages: Giant, Gnome, Goblin, Orc, Terran, and Undercommon.
\end{itemize}

\subsection{Elves}
\begin{itemize}
\item \plus1 Dexterity, \minus1 Constitution.
\item Medium: As Medium creatures, elves have no special bonuses or penalties due to their size.
\item Elf base land speed is 30 feet.
\item Immunity to sleep effects.
\item Low-light Vision: An elf treats sources of light as if they had double their normal illumination range.
 \item Trance: An elf can trance for 4 hours instead of sleeping. An elf in trance may make Perception-based checks at a \minus5 penalty. Elves must still avoid strenuous activity for 8 hours to heal, avoid fatigue, and gain other benefits of resting.
\item Keen Senses: \plus2 bonus on Awareness checks.
\item Elves can choose any of the following feats for their racial bonus feat: Dilettante, Focused Mind, Lightning Reflexes, Swift, Weapon Proficiency (bows, heavy blades, or light blades)
\item Automatic Languages: Common and Elven. Bonus Languages: Draconic, Gnoll, Gnome, Goblin, Orc, and Sylvan.
\end{itemize}

\subsection{Gnomes}
\begin{itemize}
\item \plus1 Constitution, \minus1 Strength.
\item Small: As a Small creature, a gnome gains several benefits and penalties, as described at \pcref{Small Characters}.
\item Gnome base land speed is 20 feet.
\item Low-light Vision: A gnome can see twice as far as a human in starlight, moonlight, torchlight, and similar conditions of poor illumination. He retains the ability to distinguish color and detail under these conditions.

\item Gnomes can choose any magic feat, spellgift feat, or gnomish racial feat for their racial bonus feat.
\item Automatic Languages: Common and Gnome. Bonus Languages: Draconic, Dwarven, Elven, Giant, Goblin, and Orc.
\end{itemize}

\subsection{Half-Elves}
\begin{itemize}
\item Medium: As Medium creatures, half-elves have no special bonuses or penalties due to their size.
\item Half-elf base land speed is 30 feet.
\item Immunity to sleep effects.
\item Low-light Vision: A half-elf can see twice as far as a human in starlight, moonlight, torchlight, and similar conditions of poor illumination. She retains the ability to distinguish color and detail under these conditions.
 \item Skill Affinity: Half-elves can master skills with particular ease. If a half-elf has a skill as a class skill from any class, it is treated as a class skill for all of his classes.
\item Elven Blood: For all effects related to race, a half-elf is considered both a human and an elf.
\item Half-elves can choose any skill feat or any elven or human racial feat for their racial bonus feat.
\item Automatic Languages: Common and Elven. Bonus Languages: Any (other than secret languages, such as Druidic).
\end{itemize}

\subsection{Half-Orcs}
\begin{itemize}
\item \plus1 Strength, \minus1 Intelligence, \minus1 Perception.
\item Medium: As Medium creatures, half-orcs have no special bonuses or penalties due to their size.
\item Half-orc base land speed is 30 feet.
\item Darkvision: Half-orcs (and orcs) can see clearly in the dark up to 50 feet.  Beyond that, they can see dimly, treating areas of darkness as shadowy illumination. Darkvision does not function if an orc is in a brightly lit area or is dazzled, and does not resume functioning until 1 round after the orc leaves the brightly lit area or stops being dazzled.
\par Darkvision is black and white only, but it is otherwise like normal sight, and half-orcs can function just fine with no light at all.
\item \plus2 bonus on Intimidate checks, but a \minus2 penalty on Persuasion checks.
\item Orc Blood: For all effects related to race, a half-orc is considered both a human and an orc.
\item Half-orcs can choose any combat feat or any orc or human racial feat for their racial bonus feat.
\item Automatic Languages: Common and Orc. Bonus Languages: Draconic, Giant, Gnoll, Goblin, and Abyssal.
\end{itemize}

\subsection{Halflings}
\begin{itemize}
\item \plus1 Dexterity, \minus1 Strength.
\item Small: As a Small creature, a gnome gains several benefits and penalties, as described at \pcref{Small Characters}.
\item Halfling base land speed is 20 feet.
\item \plus1 bonus on all special defenses.
\item Halflings can choose any of the following feats for their racial bonus feat: Giantfighter, Great Fortitude, Lightning Reflexes, Iron Will, Swift, Weapon Proficiency (thrown).
\item Automatic Languages: Common and Halfling. Bonus Languages: Dwarven, Elven, Gnome, Goblin, and Orc.
\end{itemize}

\section{Backgrounds}
In addition to a race, each character also has at least one background. A background describes what a character has done before the start of the story. Suggested backgrounds are given below, but you can also create new backgrounds. You can choose anything your character might reasonably have done as a background. You can also choose to have multiple backgrounds if your character has done a variety of things.

Regardless of how you choose your background or backgrounds, choose any two skills related to what your character has done. You gain a \plus1 bonus to those skills.

\subsection{Civilized Backgrounds}

\subsubsection{Bodyguard}
\parhead{Skills} Perception, Sense Motive.

\subsubsection{Commoner}
\parhead{Skill} Profession (any).

\subsubsection{Linguist}
\parhead{Skills} Linguistics, Knowledge (local).

\subsubsection{Jester}
\parhead{Skills} Acrobatics, Perform (comedy).

\subsubsection{Mage's Apprentice}
\parhead{Skills} Knowledge (arcana), Spellcraft.

\subsubsection{Merchant}
\parhead{Skills} Persuasion, Knowledge (local).

\subsubsection{Nobility}
\parhead{Skills} Bluff, Knowledge (local).

\subsubsection{Priest}
\parhead{Skill} Heal, Knowledge (religion).

\subsubsection{Scholar}
\parhead{Skill} Knowledge (any).

\subsubsection{Scribe}
\parhead{Skill} Craft (manuscript), Linguistics.

\subsubsection{Smith}
\parhead{Skill} Craft (any).

\subsubsection{Spy}
\parhead{Skills} Bluff, Disguise.

\subsubsection{Watchman}
\parhead{Skills} Knowledge (local), Perception.

\subsection{Military Backgrounds}

\subsubsection{Border Guard}
\parhead{Skill} Knowledge (geography), Survival.

\subsubsection{Cavalry}
\parhead{Skill} Creature Handling, Ride.

\subsubsection{Combat Engineer}
\parhead{Skill} Craft (any), Knowledge (engineering).

\subsubsection{Diplomat}
\parhead{Skills} Persuasion, Sense Motive.

\subsubsection{Infiltrator}
\parhead{Skills} Disguise, Stealth.

\subsubsection{Officer}
\parhead{Skills} Intimidate, Persuasion.

\subsubsection{Saboteur}
\parhead{Skills} Devices, Stealth.

\subsubsection{Scout}
\parhead{Skills} Perception, Stealth.

\subsection{Uncivilized Backgrounds}

\subsubsection{Bandit}
\parhead{Skills} Intimidate, Stealth.

\subsubsection{Explorer}
\parhead{Skills} Knowledge (geography), Survival.

\subsubsection{Hermit}
\parhead{Skill} Knowledge (nature), Survival.

\subsubsection{Minstrel}
\parhead{Skill} Perform (any).

\subsubsection{Primitive}
\parhead{Skill} Survival.

\subsubsection{Thief}
\parhead{Skills} Sleight of Hand, Stealth.
