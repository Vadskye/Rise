\chapter{Equipment}

You begin with 100 gold pieces, and you decide how to spend them.
\section{Weapons}

Weapons are grouped into several interlocking sets of categories. These categories pertain to what weapon group the weapon belongs to (axes, bows, and so on), the weapon's usefulness either in close combat (melee) or at a distance (ranged, which includes both thrown and projectile weapons), and its relative encumbrance (light, medium, or heavy).

\subsection{Weapon Groups}

Weapons are organized into thematically related categories called weapon groups. They are described in \trefnp{Weapon Groups}. For example, all axes belong to the ``axes'' weapon group. Some weapons can be found in multiple weapon groups. For example, a dagger is a simple weapon, a light blade, and a thrown weapon.

\begin{dtable!*}
\lcaption{Weapon Groups}
\begin{tabularx}{\textwidth}{l >{\lcol}X >{\lcol}X}
\thead{Group} & \thead{Weapons} & \thead{Exotic Weapons} \\
Armor weapons & Heavy shield (and spiked), light shield (and spiked), spiked armor & \\
Axes & Battleaxe, greataxe, handaxe, throwing axe & Double axe, dwarven urgrosh,  dwarven waraxe \\
Blunt weapons & Club, greatclub, mace, morningstar, quarterstaff, sap & Gnome hooked hammer, maul \\
Blades, heavy & Falchion, greatsword, longsword, scimitar & Bastard sword, two-bladed sword \\
Blades, light & Dagger, punching dagger, rapier, short sword & Kukri \\
Bows & Longbow, shortbow & \\
Crossbows & Heavy crossbow, light crossbow & Hand crossbow, repeating crossbows \\
Flexible Weapons & Flail, heavy flail, nunchaku & Whip \\
Headed weapons & Heavy pick, light hammer, light pick, sickle, warhammer & \\
Monk weapons & Kama, nunchaku, quarterstaff, sai, shuriken, siangham & \\
Polearms & Glaive, guisarme, halberd, ranseur, scythe & \\
Simple weapons & Club, dagger, light crossbow, quarterstaff, unarmed strike & \\
Spears & Javelin, lance, longspear, shortspear, spear & \\
Thrown weapons & Dagger, dart, handaxe, javelin, light hammer, shuriken, sling & Bolas, net \\
\end{tabularx}
\end{dtable!*}

\subsubsection{Weapon Proficiency}
Each character is proficient with different weapon groups. These indicate the weapons that you can use effectively. You can wield weapons you are not proficient with, but you cannot use them to defend yourself, which can cause you to be \defenseless. In addition, you take a \minus4 penalty to attack unless the improvised weapon is a light melee weapon.

\subsection{Weapon Encumbrance}
This is a measure of how much effort it takes to wield a weapon in combat. A weapon's encumbrance indicates whether a melee weapon, when wielded by a character of the weapon's size category, is considered a light weapon, a medium weapon, or a heavy weapon.

\parhead{Light} A light weapon can be used with more finesse than a medium weapon. The wielder can add either his Dexterity or his Strength to attack rolls with light weapons, whichever he prefers. In addition, light weapons are easier to use in the off-hand or while grappling.

\parhead{Medium} A medium weapon can be used in one hand. It is difficult, but possible, to wield a medium weapon in your off-hand. You can also hold a medium weapon in two hands. Changing grips to hold it in one hand or two hands can be done as a move action.

\parhead{Heavy} Two hands are required to wield a heavy weapon. You can hold it in one hand, but while doing so you cannot attack or defend yourself with it. This usually causes you to be \defenseless. Changing grips to hold it in one hand or two hands is a move action.

\subsection{Using Weapons in Two Hands}
Whenever you use a weapon in two hands, you gain a \plus1 bonus to damage. This is included in the description of heavy weapons in \trefnp{Weapons}.

\subsection{Melee Weapons}
Melee weapons are used for making melee attacks against foes within your reach (see \pcref{Reach}). Most weapons are melee weapons.

\subsection{Ranged Weapons}
Ranged weapons are used to make attacks against distant foes within the range of your weapon. There are two kinds of ranged weapons: projectile weapons and thrown weapons.

\subsubsection{Range Increments} All ranged weapons have a ``range increment'', which indicates the distance at which you can effectively attack with the weapon. Any attack against a target within your range increment takes no penalty. For each range increment by which the target is distant from you, you take a cumulative \minus2 penalty to your attack roll. A thrown weapon has a maximum range of five range increments. A projectile weapon can shoot out to ten range increments.

\subsubsection{Projectile Weapons} Projectile weapons fire ammunition at a target to deal damage. The ammunition generally breaks when used.

\subsubsection{Thrown Weapons}\label{Thrown Weapons} Thrown weapons are thrown at a target to deal damage. They generally do not break when thrown.

\parhead{Thrown Weapons in Melee}\label{Thrown Weapons in Melee} You cannot defend yourself with a weapon you are throwing, which can cause you to be \defenseless. To avoid this, you can attempt to use the weapon as a melee weapon. Some thrown weapon are not designed for use in melee, such as shurikens. When using such a weapon, you are always treated as nonproficient with it, which means you cannot defend yourself with it.

\parhead{Throwing Other Weapons} It is possible to throw a weapon that is not designed to be thrown. The range increment is 10 feet. You are treated as being nonproficient with the weapon, causing you to take a \minus4 penalty on the attack roll. If it hits, the weapon deals its normal damage.

\parhead{Heavy Weapons} Heavy thrown weapons require a standard action to throw, rather than an attack action like normal. They are not always physically thrown with two hands, but they require the use of your entire body to propel the weapon, preventing you from using your offhand for anything else. This can cause you to be \defenseless.

\subsection{Weapon Size} Every weapon has a size category. This designation indicates the size of the creature for which the weapon was designed. Weapons for unusually large creatures deal more damage, while weapons for unusually small creatures deal less damage. These differences are shown on \trefnp{Weapon Damage and Size}.

\begin{dtable}
\lcaption{Weapon Damage and Size}
\begin{tabularx}{\columnwidth}{*{6}{l} >{\lcol}X}
\thead{Medium} & \thead{Tiny} & \thead{Small} & \thead{Large} & \thead{Huge} & \thead{Gargantuan} & \thead{Colossal} \\
1d2  & \x  & 1   & 1d3  & 1d4  & 1d8  & 2d6  \\
1d3  & 1   & 1d2 & 1d4  & 1d6  & 1d10 & 2d8  \\
1d4  & 1d2 & 1d3 & 1d6  & 1d8  & 2d6  & 2d10 \\
1d6  & 1d3 & 1d4 & 1d8  & 1d10 & 2d8  & 4d6  \\
1d8  & 1d4 & 1d6 & 1d10 & 2d6  & 2d10 & 4d8  \\
1d10 & 1d6 & 1d8 & 2d6  & 2d8  & 4d6  & 4d10 \\
\end{tabularx}
\end{dtable}

\parhead{Physical Size} Like all objects and creatures, weapons have a size category that represents how physically large they are. In general, a light weapon is an object two size categories smaller than the wielder, a medium weapon is an object one size category smaller than the wielder, and a heavy weapon is an object of the same size category as the wielder.

\parhead{Inappropriately Sized Weapons} A weapon's encumbrance is altered by one step for each size category of difference between the wielder's size and the size of the creature for which the weapon was designed. For example, a light weapon sized for a Large creature can be used by a Medium creature as if it had medium encumbrance. In addition, the wielder takes a \minus2 penalty to physical attacks per size difference. If a weapon's encumbrance would be changed to something other than light, medium, or heavy by this alteration, the creature can't wield the weapon at all.

\subsection{Improvised Weapons} Sometimes objects not crafted to be weapons nonetheless see use in combat. Because such objects are not designed for this use, any creature that uses one in combat is considered to be nonproficient with it.

To determine the size category and appropriate damage for an improvised weapon, compare its relative size and damage potential to the weapon list to find a reasonable match. An improvised weapon will generally deal damage as if it were one size category smaller than a similar manufactured weapon. An improvised thrown weapon has a range increment of 10 feet.

\subsection{Drawing and Sheathing Weapons}\label{Drawing and Sheathing Weapons}
Sheathing any weapon is a move action. The time it takes to draw a weapon depends on how encumbering the weapon is. Drawing a light weapon is a swift action, while drawing a medium or heavy weapon is a move action. Drawing a hidden weapon of any type is a standard action. 

\subsection{Natural Weapons}\label{Natural Weapons}
Every creature can attack with its body using an unarmed strike. Many monsters can attack more effectively with specific parts of their body, such as teeth and claws. These are called natural weapons. You are automatically proficient with any natural weapons you possess. Most humanoids possess no natural weapons other than an unarmed strike. Natural weapons are described on \tref{Natural Weapons}.

A creature with multiple natural weapons of the same type can designate one as a primary attack and one as a secondary attack, allowing it to fight with both at once (see \pcref{Two-Weapon Fighting}). You are only considered to have one unarmed strike, so you cannot two-weapon fight with only your unarmed strike (but see the unarmed warrior monk ability, \pref{Mnk:Unarmed Warrior}).

\subsection{Weapon Qualities}
Here is the format for weapon entries (given as column headings on \trefnp{Weapons}, below).

\parhead{Encumbrance} Describes whether the weapon is a light, medium, heavy, double, or ranged weapon.

\par This cost is the same for a Small or Medium version of the weapon. A Large version costs twice the listed price.
\parhead{Damage} The Damage column gives the damage dealt by the weapon on a successful hit. The damage given is for weapons used by Medium creatures. Weapons used by Small creatures have a damage die that is one size smaller, as shown on \trefnp{Weapon Damage and Size}.

If two damage ranges are given, the weapon is a double weapon. Use the second damage figure given for the double weapon's other side.

\parhead{Range Increment} The range increment of the weapon.

\parhead{Type} Weapons are classified according to the type of damage they deal: bludgeoning, piercing, or slashing. Some monsters may be resistant or immune to attacks from certain types of weapons.

Some weapons deal damage of multiple types. If a weapon is of two types, the damage it deals is not half one type and half another; all of it is both types. Therefore, a creature would have to be immune to both types of damage to ignore any of the damage from such a weapon.

In other cases, a weapon can deal either of two types of damage. In a situation when the damage type is significant, the wielder can choose which type of damage to deal with such a weapon.
\parhead{Cost} This value is the weapon's cost in gold pieces (gp) or silver pieces (sp). The cost includes miscellaneous gear that goes with the weapon.
\parhead{Weight} This column gives the weight of a Medium version of
the weapon. Halve this number for Small weapons, and double it for
Large weapons.

\parhead{Special} Some weapons have special features. See the weapon
descriptions for details.

\begin{dtable!*}
\lcaption{Weapons}
    \begin{tabularx}{\textwidth}{p{12em} c c >{\ccol}p{10em} c c >{\ccol}X}
    \thead{Weapons} & \thead{Encumbrance} & \thead{Dmg} & \thead{Damage Type\fn{1}} & \thead{Cost} & \thead{Weight\fn{2}} & \thead{Special} \\

    Armor weapons &&&&&& \\
    \tind Shield, heavy\fn{3} & Medium & 1d4 & Bludgeoning & special & special & Forceful \\
    \tind Shield, light\fn{3} & Light & 1d3 & Bludgeoning & special & special & Forceful \\
    \tind Spiked armor\fn{3} & Light & 1d6 & Piercing & special & special & Grappling \\
    \tind Spiked shield, heavy\fn{3} & Medium & 1d6 & Piercing & special & special & Forceful \\
    \tind Spiked shield, light\fn{3} & Light & 1d4 & Piercing & special & special & Forceful \\

    Axes &&&&&& \\
    \tind Axe, throwing & Light & 1d6 & Slashing & 8 gp & 2 lb. & Throwing (10 ft.) \\
    \tind Battleaxe & Medium & 1d8 & Slashing & 10 gp & 6 lb. & Impact \\
    \tind Greataxe & Heavy & 1d10\plus1 & Slashing & 20 gp & 12 lb. & Impact \\
    \tind Handaxe & Light & 1d6 & Slashing & 6 gp & 3 lb. & Impact \\
    \tind Waraxe, dwarven & Heavy & 1d10\plus1 & Slashing & 75 gp & 8 lb. & Exotic Grip \\

    Blades, heavy &&&&&& \\
    \tind Falchion & Heavy & 1d10\plus1 & Slashing & 50 gp & 8 lb. & Keen \\
    \tind Greatsword & Heavy & 1d10\plus1 & Slashing & 25 gp & 8 lb. & Keen \\
    \tind Longsword & Medium & 1d8 & Slashing & 15 gp & 4 lb. & Keen \\
    \tind Scimitar & Medium & 1d8 & Slashing & 15 gp & 4 lb. & Keen \\

    Blades, light &&&&&& \\
    \tind Dagger & Light & 1d4 & Piercing or slashing & 2 gp & 1 lb. & Keen, Small, Throwing (10 ft.) \\
    \tind Dagger, punching & Light & 1d4 & Piercing & 2 gp & 1 lb. & Impact, Small \\
    \tind Rapier\fn{3} & Light & 1d6 & Piercing & 20 gp & 2 lb. & Impact \\
    \tind Sword, short & Light & 1d6 & Piercing or slashing & 10 gp & 2 lb. & Keen \\

    Blunt weapons &&&&&& \\
    \tind Club & Medium & 1d6 & Bludgeoning & \x & 3 lb. & \x \\
    \tind Greatclub & Heavy & 1d10\plus1 & Bludgeoning & 5 gp & 8 lb. & \x \\
    \tind Mace & Light & 1d6 & Bludgeoning & 12 gp & 8 lb. & \x \\
    \tind Morningstar & Medium & 1d8 & Bludgeoning and piercing & 8 gp & 6 lb. & \x \\
    \tind Quarterstaff & Heavy & 1d6/1d6 & Bludgeoning & \x & 4 lb. & Double \\
    \tind Sap & Light & 1d6 & Bludgeoning & 1 gp & 2 lb. & Nonlethal \\

    Bows &&&&&& \\
    \tind Longbow\fn{3} & Heavy (Ranged) & 1d8 & Piercing & 40 gp & 3 lb. & Impact, Projectile (100 ft.) \\
    \tind Shortbow\fn{3} & Medium (Ranged) & 1d6 & Piercing & 30 gp & 2 lb. & Impact, Projectile (50 ft.) \\
    \tind Arrows (20) & \x & \x & \x & 1 gp & 3 lb. & Ammunition \\

    Crossbows &&&&&& \\
    \tind Crossbow, heavy\fn{3} & Heavy (Ranged) & 1d10 & Piercing & 50 gp & 8 lb. & Impact, Projectile (100 ft.) \\
    \tind Crossbow, light\fn{3} & Medium (Ranged) & 1d8 & Piercing & 40 gp & 4 lb. & Impact, Projectile (50 ft.) \\
    \tind Bolts, crossbow (10) & \x & \x & \x & 1 gp & 1 lb. & Ammunition \\

    Flexible weapons &&&&&& \\
    \tind Flail  & Medium & 1d8 & Bludgeoning & 8 gp & 5 lb. & Disarming, Tripping \\
    \tind Flail, heavy & Heavy & 1d10\plus1 & Bludgeoning & 15 gp & 10 lb. & Disarming, Tripping \\

\end{tabularx}
\end{dtable!*}

\begin{dtable!*}
    \begin{tabularx}{\textwidth}{p{12em} c c >{\ccol}p{10em} c c >{\ccol}X}
    \thead{Weapons} & \thead{Encumbrance} & \thead{Dmg} & \thead{Damage Type\fn{1}} & \thead{Cost} & \thead{Weight\fn{2}} & \thead{Special} \\
        Headed weapons &&&&&& \\
        \tind Hammer, light & Light & 1d4 & Bludgeoning & 1 gp & 2 lb. & Throwing (20 ft.) \\
        \tind Pick, heavy & Medium & 1d8 & Piercing & 8 gp & 6 lb. & Impact, Unbalanced \\
        \tind Pick, light & Light & 1d6 & Piercing & 4 gp & 3 lb. & Impact, Unbalanced \\
        \tind Sickle & Light & 1d6 & Slashing & 6 gp & 2 lb. & Tripping \\
        \tind Warhammer & Medium & 1d8 & Bludgeoning & 12 gp & 5 lb. & Impact \\

        Monk weapons &&&&&& \\
        \tind Kama & Light & 1d6 & Slashing & 2 gp & 2 lb. & Tripping \\
        \tind Nunchaku & Light & 1d6 & Bludgeoning & 2 gp & 2 lb. & Disarming, Unbalanced \\
        \tind Quarterstaff & Heavy & 1d6/1d6 & Bludgeoning & \x & 4 lb. & Double \\
        \tind Sai & Light & 1d4 & Piercing or bludgeoning & 1 gp & 1 lb. & Disarming, Impact \\
        \tind Shuriken (5) & Light (Ranged) & 1d4 & Piercing and slashing & 1 gp & 1/2 lb. & Ammunition, Keen, Thrown (10 ft.) \\
        \tind Siangham & Light & 1d6 & Piercing & 3 gp & 1 lb. & Parrying \\

        Polearms &&&&&& \\
        \tind Glaive & Heavy & 1d10\plus1 & Slashing & 8 gp & 10 lb. & Impact, Reach \\
        \tind Guisarme & Heavy & 1d10\plus1 & Slashing & 9 gp & 12 lb. & Reach, Tripping \\
        \tind Halberd & Heavy & 1d10\plus1 & Piercing or slashing & 10 gp & 12 lb. & Impact, Reach \\
        \tind Quarterstaff & Heavy & 1d6/1d6 & Bludgeoning & \x & 4 lb. & Double \\
        \tind Ranseur & Heavy & 1d10\plus1 & Piercing & 10 gp & 12 lb. & Disarming, Reach \\
        \tind Scythe & Heavy & 1d10\plus1 & Slashing & 18 gp & 10 lb. & Impact \\

        Simple weapons &&&&&& \\
        \tind Club & Medium & 1d6 & Bludgeoning & \x & 3 lb. & \x \\
        \tind Crossbow, light\fn{3} & Medium (Ranged) & 1d8 & Piercing & 35 gp & 4 lb. & Impact, Projectile (100 ft.) \\
        \tind Dagger & Light & 1d4 & Piercing or slashing & 2 gp & 1 lb. & Keen, Small, Thrown (10 ft.) \\
        \tind Quarterstaff & Heavy & 1d6/1d6 & Bludgeoning & \x & 4 lb. & Double \\
        \tind Unarmed strike & Light & 1d3 & Bludgeoning & \x & \x & Nonlethal, Unarmed \\

        Spears &&&&&& \\
        \tind Javelin & Medium (Ranged) & 1d6 & Piercing & 1 gp & 2 lb. & Thrown (30 ft.) \\
        \tind Lance & Heavy & 1d8\plus1 & Piercing & 10 gp & 10 lb. & Charging, Impact, Reach \\
        \tind Longspear & Heavy & 1d8\plus1 & Piercing & 5 gp & 9 lb. & Bracing, Impact, Reach \\
        \tind Shortspear & Light & 1d6 & Piercing & 1 gp & 3 lb. & Thrown (20 ft.) \\
        \tind Spear & Medium & 1d8 & Piercing & 2 gp & 6 lb. & Bracing, Thrown (20 ft.) \\

        Thrown weapons &&&&&& \\
        \tind Axe, throwing & Light & 1d6 & Slashing & 8 gp & 2 lb. & Thrown (10 ft.) \\
        \tind Dagger & Light & 1d4 & Piercing or slashing & 2 gp & 1 lb. & Keen, Small, Thrown (10 ft.) \\
        \tind Dart (5) & Light (Ranged) & 1d4 & Piercing & 1 gp & 1/2 lb. & Ammunition, Thrown (20 ft.) \\
        \tind Hammer, light & Light & 1d4 & Bludgeoning & 1 gp & 2 lb. & Thrown (20 ft.) \\
        \tind Javelin & Medium (Ranged) & 1d6 & Piercing & 1 gp & 2 lb. & Thrown (30 ft.) \\
        \tind Shuriken (5) & Light (Ranged) & 1d4 & Piercing and slashing & 1 gp & 1/2 lb. & Ammunition, Keen, Thrown (10 ft.) \\
        \tind Sling\fn{3} & Light (Ranged) & 1d4 & Bludgeoning & 2 gp & 0 lb. & Projectile (50 ft.) \\
        \tind Bullets, sling (20) & \x & \x & \x & 1 gp & 5 lb. & Ammunition \\

        Unarmed weapons &&&&&&\\
        \tind Claw Sheath\fn{3} & \x & \x & \x & 50 gp & 3 lb. & Special \\
        \tind Gauntlet & Light & 1d3 & Bludgeoning & 2 gp & 1 lb. & Unarmed \\
        \tind Gauntlet, spiked & Light & 1d4 & Piercing & 5 gp & 1 lb. & Unarmed \\
        \tind Unarmed strike & Light & 1d3 & Bludgeoning & \x & \x & Nonlethal, Unarmed \\
    \end{tabularx}
\end{dtable!*}

\begin{dtable!*}
    \begin{tabularx}{\textwidth}{p{12em} c c >{\ccol}p{10em} c c >{\ccol}X}
        \thead{Exotic Weapons} & \thead{Encumbrance} & \thead{Dmg} & \thead{Damage Type\fn{1}} & \thead{Cost} & \thead{Weight\fn{2}} & \thead{Special} \\
        Armor &&&&&& \\
        Axes &&&&&& \\
        \tind Axe, orc double & Heavy & 1d8/1d8 & Slashing & 60 gp & 15 lb. & Double, Impact \\
        \tind Urgrosh, dwarven\fn{3} & Heavy & 1d8/1d6 & Slashing or piercing & 50 gp & 12 lb. & Bracing, Double, Impact \\
        Blunt weapons &&&&&& \\
        Blades, heavy &&&&&& \\
        \tind Sword, bastard & Heavy & 1d10 & Slashing & 75 gp & 6 lb. & \x \\
        \tind Sword, two-bladed & Heavy & 1d8/1d8 & Slashing & 100 gp & 10 lb. & Double \\
        Blades, light &&&&&& \\
        \tind Kukri & Light & 1d4 & Slashing & 8 gp & 2 lb. & \x \\
        Bows &&&&&& \\
        Crossbows &&&&&& \\
        \tind Crossbow, hand\fn{3} & Light (Ranged) & 1d4 & Piercing & 100 gp & 2 lb. & Projectile (30 ft.) \\
        \tind Crossbow, repeating heavy\fn{3} & Heavy (Ranged) & 1d10 & Piercing & 400 gp & 12 lb. & Projectile (100 ft.) \\
        \tind Crossbow, repeating light\fn{3} & Medium (Ranged) & 1d8 & Piercing & 250 gp & 6 lb. & Projectile (50 ft.) \\
        \tind Bolts, hand (10) & \x & \x & \x & 1 gp & 1 lb. & Ammunition \\
        \tind Bolts, repeating (5) & \x & \x & \x & 1 gp & 1 lb. & Ammunition \\
        Flexible weapons &&&&&& \\
        \tind Flail, dire & Heavy & 1d8/1d8 & Bludgeoning & 90 gp & 10 lb. & Disarming, Double, Tripping \\
        \tind Whip\fn{3} & Light & 1d3 & Slashing & 1 gp & 2 lb. & Disarming, Nonlethal, Tripping \\
        Headed weapons &&&&&& \\
        \tind Hammer, gnome hooked\fn{3} & Heavy & 1d8/1d6 & Bludgeoning or piercing & 20 gp & 6 lb. & Double, Impact, Tripping \\
        Monk weapons &&&&&& \\
        Polearms &&&&&& \\
        Simple weapons &&&&&& \\
        Spear &&&&&& \\
        \tind Urgrosh, dwarven\fn{3} & Heavy & 1d8/1d6 & Slashing or piercing & 50 gp & 12 lb. & Bracing, Double, Impact \\
        Thrown weapons &&&&&& \\
        \tind Bolas & Light (Ranged) & 1d4\fn{3} & Bludgeoning & 5 gp & 2 lb. & Thrown (10 ft.), Tripping \\
        \tind Net\fn{3} & Medium (Ranged) & \x & \x & 20 gp & 6 lb. & Thrown (10 ft.) \\
        Unarmed weapons &&&&&&\\
    \end{tabularx}
    1 When two types are given, the weapon is both types if the entry specifies ``and," or either type (attacker's choice) if the entry specifies ``or." \\
    2 Weight figures are for Medium weapons. A Small weapon weighs half as much, and a Large weapon weighs twice as much. \\
    3 This weapon has special rules. \\
\end{dtable!*}

\begin{dtable!*}
    \lcaption{Natural Weapons}
    \begin{tabularx}{\textwidth}{p{12em} c c >{\ccol}p{15em} >{\ccol}X}
        \thead{Natural Weapons} & \thead{Encumbrance} & \thead{Dmg} & \thead{Damage Type\fn{2}} & \thead{Special} \\
        Bite & Medium & 1d8 & Piercing and bludgeoning  & \x \\
        Claw & Light & 1d6 & Slashing and piercing & \x \\
        Constrict\fn{2} & Heavy & 1d10 & Bludgeoning & \x \\
        Gore & Heavy & 1d8 & Piercing & \x \\
        Slam & Medium & 1d8 & Bludgeoning & \x \\
        Unarmed Strike & Light & 1d3\fn{3} & Bludgeoning & Unarmed \\
    \end{tabularx}
    1 When two types are given, the weapon is both types if the entry specifies ``and," or either type (attacker's choice) if the entry specifies ``or." \\
    2 This attack can only be used against a foe you are grappling with. \\
\end{dtable!*}

\subsection{Weapon Properties}
Some weapons found on \trefnp{Weapons} have special properties. The list of special properties is given below.
\parhead{Ammunition} This weapon is designed to thrown or fired by a projectile weapon in large quantities. It is cheaper to buy and craft, but ammunition that hits its target is destroyed, and ammunition that misses has a 50\% chance to be destroyed or lost.
\parhead{Bracing} As a move action, you can brace this weapon against a charge for 1 round. While you are bracing your weapon, you gain a \plus5 bonus to physical melee attacks against creatures that charge you that round.
\parhead{Charging} This weapon deals double damage when used from the back of a charging mount.
\parhead{Disarming} You can use this weapon to make disarm atacks. You gain a \plus2 bonus on such attacks, and can apply magical bonuses from the weapon to the attack.
\parhead{Double} This weapon has more than one striking surface. You can fight with both ends simultaneously, just like two-weapon fighting (see \pcref{Two-Weapon Fighting}). Alternately, you can attack with one end at a time. If you have the ability to use a double weapon in one hand, you can only fight with one end at a time, not both.
\parhead{Exotic Grip} If you have proficiency with exotic weapons, you can use this in one hand. While wielding it in one hand, you do not gain the \plus1 bonus to damage from wielding the weapon in two hands.
\parhead{Finesse} You apply your Dexterity instead of your Strength to physical attacks with this weapon, even if it isn't a light weapon for you.
\parhead{Forceful} You can use this weapon to make shove attacks to push people away from you. You gain a \plus2 bonus on such attacks.
\parhead{Grappling} You gain a \plus2 bonus to physical attacks with this weapon in a grapple.
\parhead{Impact} When this weapon scores a critical hit, all damage dealt in excess of the target's hit points is dealt as critical damage.
\parhead{Keen} You gain a \plus2 bonus to confirm critical threats with this weapon.
\parhead{Nonlethal} This weapon deals nonlethal damage rather than lethal damage. See \pcref{Nonlethal Damage}.
\parhead{Parrying} You can a \plus2 bonus to parry attempts with this weapon.
\parhead{Projectile} This weapon fires projectiles at range. Projectile weapons have a range increment listed in their description, which indicates the distance they can be easily fired. Projectile weapons must be reloaded. The time required to reload a projectile weapon is given in the weapon description.
\parhead{Reach} This weapon strikes at double your natural reach (so 10 feet for a typical Small or Medium creature). However, it cannot attack a creature within your natural reach.

Reach weapons can held using a different grip to strike nearby foes. This is called ``short hafting''. While short hafting a reach weapon, you ignore the weapon's reach property, but you take a \minus4 penalty to physical attacks with it.
\parhead{Small} This weapon is unusually small. It is one size category smaller than normal for a light weapon (that is, three size categories smaller than the creature it is intended for). This makes it easier to conceal (see \pcref{Sleight of Hand}).
\parhead{Throwing} This weapon is designed to be thrown. Throwing weapons have a range increment listed in their description, which indicates the distance they can be easily thrown. See \pcref{Thrown Weapons}.
\parhead{Tripping} You can use this weapon to make trip attacks. You gain a \plus2 bonus on such attacks.
\parhead{Unarmed} This weapon is used as part of an unarmed strike. It cannot be disarmed. Unless you have the Improved Unarmed Strike feat (see \featref{Improved Unarmed Strike}), you can't defend yourself with this weapon, which usually makes you \defenseless.
\parhead{Unbalanced} If you roll either a 1 or 2 when attacking this weapon, you critically fail the attack (see \pcref{Critical Success and Failure}.

\subsection{Weapon Special Abilities}
Some weapons in \trefnp{Weapons} have unique special abilities, which are described below.
\parhead{Claw Sheath} A claw sheath is not a weapon in itself, but a covering over a single claw that a creature can use to make claw attacks. Claw sheaths do not grant a creature without claws a claw attack. Normally, a sheath does not improve the claw attack, but magical claw sheaths can be made which grant bonuses to attacks made with the claw. It may be possible to find or craft unusual sheaths for natural weapons other than claws.
\parhead{Crossbow, Hand} You can draw a hand crossbow back by hand. Loading a hand crossbow is a move action.
\par You can fire a crossbow from a prone position without penalty.
\parhead{Crossbow, Heavy} You draw a heavy crossbow back by turning a small winch. Loading a heavy crossbow is a full-round action that requires both hands.
\par You can fire a crossbow from a prone position without penalty.
\parhead{Crossbow, Light} You draw a light crossbow back by pulling a lever. Loading a light crossbow is a move action that requires both hands.
\par You can fire a crossbow from a prone position without penalty.
\parhead{Crossbow, Repeating} The repeating crossbow (whether heavy or light) holds 10 crossbow bolts. As long as it holds bolts, you can reload it by pulling the reloading lever (a free action). Loading a new case of 10 bolts is a full-round action that requires both hands.
\par You can fire a crossbow from a prone position without penalty.
\parhead{Hammer, Gnome Hooked} This weapon has a hammer head which deals 1d8 points of damage, and a hook which deals 1d6 points of damage. The hook is a tripping weapon.
\parhead{Longbow} You need both hands to fire a bow, regardless of its size. One hand is free when not firing the bow. Loading a bow is a free action that requires both hands. A longbow is too unwieldy to use while you are mounted.
\par When attacking with a bow, you take a \minus4 penalty against creatures that threaten you.
\parhead{Net} A net is used to entangle enemies. When you throw a net, you make a ranged touch attack against your target. If you hit, the target is entangled. An entangled creature moves at half speed, cannot run or charge, and takes a \minus2 penalty to physical attacks and defenses, as well as Strength and Dexterity-based checks. If you control the trailing rope by succeeding on an opposed Strength check while holding it, the entangled creature can move only within the limits that the rope allows. If the entangled creature attempts to cast a spell, it must make a DC 15 \add double spell level Concentration check or be unable to cast the spell.
\par An entangled creature can escape with a DC 20 Escape Artist check (a full-round action). The net has 5 hit points and can be burst with a DC 20 Strength check (also a full-round action).
\par A net is useful only against creatures within one size category of you.
\par A net must be folded to be thrown effectively. The first time you throw your net in a fight, you make a normal ranged touch attack roll. After the net is unfolded, you take a \minus4 penalty on attack rolls with it. It takes 2 rounds for a proficient user to fold a net and twice that long for a nonproficient one to do so.
\parhead{Rapier} A rapier is difficult to use in the off-hand. It is treated as a medium weapon if it is used as a secondary weapon when fighting with two weapons at once.
\parhead{Shield, Heavy or Light} You can bash with a shield instead of using it for defense. See Armor for details.
\parhead{Shortbow} You need both hands to fire a bow, regardless of its size. One hand is free when not firing the bow. Loading a bow is a free action that requires both hands.
\par When attacking with a bow, you take a \minus4 penalty against creatures that threaten you.
\parhead{Sling} You can fire, but not load, a sling with one hand. Loading a sling is a free action that requires both hands.
\par You can hurl ordinary stones with a sling, but stones are not as dense or as round as bullets. You take a \minus1 penalty to attack and damage rolls when using ordinary stones.
\parhead{Spiked Armor} You can outfit your armor with spikes, which can deal damage in a grapple. See Armor for details.
\parhead{Spiked Shield, Heavy or Light} You can bash with a spiked shield instead of using it for defense. See Armor for details.

\parhead{Unarmed Strike} Anyone can attack unarmed, but it is more dangerous than attacking with a weapon. See \pcref{Unarmed Combat}.

\parhead{Urgrosh, Dwarven} This weapon has an axe head which deals 1d8 points of damage, and a spear which deals 1d6 points of damage. The spear is a bracing weapon. You must be proficient with axes to use the axe head, and proficient with spears to use the spear head.
\parhead{Whip} A whip is treated as a light melee weapon with 15 foot reach. However, you can't defend yourself with a whip, which can make you \defenseless, and you don't threaten the area into which you can make an attack. In addition, unlike most other weapons with reach, you can use it against foes anywhere within your reach (including adjacent foes).

\section{Armor}

Most characters use armor to protect themselves. There are two kinds of armor: body armor, such as chainmail or full plate, and shields.

\subsection{Armor Qualities}
\par Here is the format for armor entries (given as column headings on \trefnp{Armor and Shields}, below).

\parhead{Encumbrance} All armor restricts a character's movement to some degree. The heavier the armor, the more restrictive it is. Shields are divided into light shields (including bucklers), heavy shields, or tower shields, while body armor can be light, medium, or heavy. Many classes are not proficient with heavier kinds of armor.

\parhead{Cost} The cost of the armor for Small or Medium humanoid
creatures. See \trefnp{Armor for Unusual Creatures}, below, for armor prices for other creatures.
\parhead{Armor/Shield Bonus} Body armor improves your Armor defense, while shields improve all your physical defenses. Wearing multiple suits of armor or wielding multiple shields does not improve your defenses any further.

\parhead{Dexterity} Medium and heavy body armor limits mobility and agility, halving the character's Dexterity. This halving is applied in addition to the armor's armor check penalty (if any). A Dexterity penalty is not halved.

Your character's encumbrance (the amount of gear he or she carries) may also affect your character's Dexterity bonus.
\subparhead{Shields} Most shields do not affect a character's Dexterity. However, a tower shield halves a character's Dexterity. This is not cumulative with the halving from wearing medium or heavy armor.
\parhead{Armor Check Penalty} Any armor heavier than leather hurts a character's ability to use some skills. An armor check penalty number is the penalty that applies to Acrobatics, Climb, Escape Artist, Jump, Sleight of Hand, and Stealth checks by a character wearing a certain kind of armor. Double the normal armor check penalty is applied to Swim checks. A character's encumbrance (the amount of gear carried, including armor) may also apply an armor check penalty.
\subparhead{Shields} If a character is wearing armor and using a shield, both armor check penalties apply.
\subparhead{Nonproficient with Armor Worn} A character who wears armor and/or uses a shield with which he or she is not proficient takes the armor's (and/or shield's) armor check penalty on attack rolls and on all Strength-based and Dexterity-based ability and skill checks. The penalty for nonproficiency with armor stacks with the penalty for nonproficiency with shields.

\subparhead{Sleeping in Armor} A character who sleeps overnight in medium or heavy body armor is automatically fatigued the next day. He or she takes a \minus2 penalty on Strength and Dexterity and can't charge or run. Sleeping in light armor does not cause fatigue.
\parhead{Arcane Spell Failure} Armor interferes with the gestures that a spellcaster must make to cast an arcane spell that has a somatic component. Arcane spellcasters face the possibility of arcane spell failure if they're wearing armor.

\subparhead{Casting an Arcane Spell in Armor} A character who casts an arcane spell while wearing armor must usually make an arcane spell failure roll. The number in the Arcane Spell Failure Chance column on Table: Armor and Shields is the chance that the spell fails and is ruined. If the spell lacks a somatic component, however, it can be cast with no chance of arcane spell failure.

\subparhead{Shields} If a character is wearing armor and using a shield, add the two numbers together to get a single arcane spell failure chance.
\parhead{Speed} Medium or heavy armor slows the wearer down. The number on Table: Armor and Shields is the character's speed while wearing the armor. Humans, elves, half-elves, and half-orcs have an unencumbered speed of 30 feet.
\par They use the first column. Dwarves, gnomes, and halflings have an unencumbered speed of 20 feet. They use the second column. Remember, however, that a dwarf's land speed remains 20 feet even in medium or heavy armor or when carrying a medium or heavy load.
\subparhead{Shields} Shields do not affect a character's speed.
\parhead{Weight} This column gives the weight of the armor sized for a Medium wearer. Armor fitted for Small characters weighs half as much, and armor for Large characters weighs twice as much.

\subsubsection{Moving in Armor}
The heavier the armor, the more it slows your movement. Light armor has no effect on your movement speed, though it does impose armor check penalties. Running in medium or heavy armor is difficult, and you can run at only three times your movement speed instead of four times your movement speed. Heavy armor slows down your movement speed even further, causing you to move at 2/3 of your normal movement rate.

\subsection{Getting Into And Out Of Armor}
The time required to don armor depends on its type; see \trefnp{Donning Armor}. Donning armor of any kind takes both hands.
\parhead{Don} This column tells how long it takes a character to put the armor on. (One minute is 10 rounds.) Readying (strapping on) a shield is only a move action.
\parhead{Don Hastily} This column tells how long it takes to put the armor on in a hurry. The armor check penalty and armor bonus for hastily donned armor are each 1 point worse than normal.
\parhead{Remove} This column tells how long it takes to get the armor off. Loosing a shield (removing it from the arm and dropping it) is only a move action.

\begin{dtable}
\lcaption{Donning Armor}
\begin{tabularx}{\columnwidth}{>{\lcol}X c c c}
\thead{Armor Type} & \thead{Don} & \thead{Don Hastily} & \thead{Remove} \\
Shield (any) & 1 move action & n/a & 1 move action \\
Padded, leather, hide, studded leather, or chain shirt & 1 minute & 5 rounds & 1 minute\fn{1} \\
Breastplate, scale mail, chainmail, banded mail, or splint mail & 4 minutes\fn{1} & 1 minute & 1 minute\fn{1} \\
Half-plate or full plate & 4 minutes\fn{2} & 4 minutes\fn{1} & 1d4\plus1 minutes\fn{1} \\
\end{tabularx}
1 If the character has some help, cut this time in half. A single character doing nothing else can help one or two adjacent characters. Two characters can't help each other don armor at the same time. \\
2 The wearer must have help to don this armor. Without help, it can be donned only hastily.
\end{dtable}

\begin{dtable!*}
\lcaption{Armor and Shields}
\begin{tabularx}{\textwidth}{>{\lcol}X >{\ccol}p{6em} >{\ccol}p{5em} >{\ccol}p{6em} >{\ccol}p{7em} c c c c}
  &  &  &  &  &  \multicolumn{2}{c}{\x\x Speed \x\x} & & \\
\thead{Armor} & \thead{Armor/Shield Bonus} & \thead{Dex Modifier} & \thead{Armor Check Penalty} & \thead{Arcane Spell Failure Chance} & \thead{(30 ft.)} & \thead{(20 ft.)} & \thead{Cost} & \thead{Weight\fn{1}} \\
Light armor &  &  &  &  &  &  &  &  \\
\tind Padded 				& \plus1 		& 1\mult 	& 0       & 5\%  & 30 ft. & 20 ft. & 5 gp          & 10 lb.\\
\tind Leather 				& \plus2 		& 1\mult 	& \minus1 & 10\% & 30 ft. & 20 ft. & 10 gp        & 15 lb.\\
\tind Studded leather 		& \plus3 		& 1\mult 	& \minus2 & 15\% & 30 ft. & 20 ft. & 25 gp & 20 lb.\\
Medium armor 				&  &  &  &  &  &  &  & \\
\tind Chain shirt 			& \plus4 		& 1/2\mult 	& \minus2 & 20\% & 30 ft. & 20 ft. & 40 gp   & 25 lb.\\
\tind Hide 					& \plus4 		& 1/2\mult 	& \minus4 & 20\% & 30 ft. & 20 ft. & 15 gp           & 25 lb.\\
\tind Scale mail 			& \plus5 		& 1/2\mult 	& \minus4 & 25\% & 30 ft. & 20 ft. & 30 gp     & 30 lb.\\
\tind Chainmail 			& \plus6 		& 1/2\mult 	& \minus5 & 30\% & 30 ft. & 20 ft. & 50 gp     & 40 lb.\\
\tind Breastplate 			& \plus6 		& 1/2\mult 	& \minus4 & 25\% & 30 ft. & 20 ft. & 150 gp    & 30 lb.\\
Heavy armor 				&  &  &  &  &  &  &  & \\
\tind Splint mail 			& \plus6 		& 1/2\mult 	& \minus7 & 40\% & 20 ft. & 15 ft. & 75 gp   & 45 lb.\\
\tind Half-plate 			& \plus7 		& 1/2\mult 	& \minus6 & 40\% & 20 ft. & 15 ft. & 250 gp  & 50 lb.\\
\tind Full plate 			& \plus8 		& 1/2\mult 	& \minus6 & 35\% & 20 ft. & 15 ft. & 500 gp  & 50 lb.\\
Shields 					&  &  &  &  &  &  &  & \\
\tind Buckler 				& \plus1 		& \x 		& \minus1 & 5\% & \x & \x & 15 gp             & 5 lb.\\
\tind Shield, light wooden 	& \plus2 		& \x 		& \minus2 & 5\% & \x & \x & 3 gp & 5 lb.\\
\tind Shield, light steel 	& \plus2 		& \x 		& \minus2 & 5\% & \x & \x & 9 gp  & 6 lb.\\
\tind Shield, heavy wooden 	& \plus3 		& \x 		& \minus3 & 15\%\fn{2} & \x & \x & 7 gp & 10 lb.\\
\tind Shield, heavy steel 	& \plus3 		& \x 		& \minus3 & 15\%\fn{2} & \x & \x & 20 gp & 15 lb.\\
\tind Shield, tower 		& \plus4\fn{3} 	& 1/2\mult 	& \minus10 & 50\%\fn{2} & \x & \x & 30 gp       & 45 lb.\\
Extras 						&  &  &  &  &  &  &  & \\
\tind Armor spikes 			& \minus1\fn{4} & \x 		& \minus2 & \x & \x & \x & \plus50 gp & \plus10 lb.\\
\tind Gauntlet, locked 		& \x 			& \x 		& Special & \x\fn{2} & \x & \x & 8 gp & \plus5 lb.\\
\tind Shield spikes 		& \x 			& \x 		& \minus1 & \x & \x & \x & \plus10 gp & \plus5 lb.\\
\end{tabularx}
1 Weight figures are for armor sized to fit Medium characters. Armor fitted for Small characters weighs half as much, and armor fitted for Large characters weighs twice as much. \\
2 Hand not free to cast spells.
3 A tower shield can instead grant you cover. See the description. \\
4 Armor spikes reduce the defense bonus granted by the armor they are put on by 1. \\
\end{dtable!*}

\subsection{Armor Descriptions}
Any special benefits or accessories to the types of armor found on \trefnp{Armor and Shields} are described below.
\parhead{Armor Spikes} You can have spikes added to your armor, which allow you to deal extra piercing damage (see Table: Weapons) on a successful grapple or overrun attack. If you are not proficient with the spikes, you only deal the damage if you succeed at the attack by 10 or more.
\par An enhancement bonus to a suit of armor does not improve the spikes' effectiveness, but the spikes can be made into magic weapons in their own right.
\parhead{Banded Mail} The suit includes gauntlets.
\parhead{Breastplate} It comes with a helmet and greaves.
\parhead{Buckler} This small metal shield is worn strapped to your forearm. You can use a bow or crossbow without penalty while carrying it. You can also use your shield arm to wield a weapon (whether you are using an off-hand weapon or using your off hand to help wield a heavy weapon), but you take a \minus1 penalty on attack rolls while doing so. If you use a weapon in your off hand, you don't get the buckler's defense bonus for the rest of the round.
\par You can't bash someone with a buckler.
\parhead{Chain Shirt} A chain shirt comes with a steel cap.
\parhead{Chainmail} The suit includes gauntlets.
\parhead{Full Plate} The suit includes gauntlets, heavy leather boots, a visored helmet, and a thick layer of padding that is worn underneath the armor. Each suit of full plate must be individually fitted to its owner by a master armorsmith, although a captured suit can be resized to fit a new owner with a day of work and a DC 25 Craft (metalworking) check. The new owner must still be of the same size category as the size category that the suit was designed for.
\parhead{Gauntlet, Locked} This armored gauntlet has small chains and braces that allow the wearer to attach a weapon to the gauntlet so that it cannot be dropped easily. It provides a \plus10 bonus on any roll made to keep from being disarmed in combat. Removing a weapon from a locked gauntlet or attaching a weapon to a locked gauntlet is a full-round action.
\par The price given is for a single locked gauntlet. The weight given applies only if you're wearing a breastplate, light armor, or no armor. Otherwise, the locked gauntlet replaces a gauntlet you already have as part of the armor.
\par While the gauntlet is locked, you can't use the hand wearing it for casting spells or employing skills. (You can still cast spells with somatic components, provided that your other hand is free.)
Like a normal gauntlet, a locked gauntlet lets you deal lethal damage rather than nonlethal damage with an unarmed strike.
\parhead{Half-Plate} The suit includes gauntlets.
\parhead{Scale Mail} The suit includes gauntlets.
\parhead{Shield, Heavy, Wooden or Steel} You strap a shield to your forearm and grip it with your hand. A heavy shield is so heavy that you can't use your shield hand for anything else.
\subparhead{Wooden or Steel} Wooden and steel shields offer the same basic protection, though they respond differently to special attacks.
\subparhead{Shield Bash Attacks} You can bash an opponent with a heavy shield, using it as a medium bludgeoning weapon. See \trefnp{Weapons} for the damage dealt by a shield bash.  If you use your shield as a weapon, you lose its defense bonus until your next action (usually until the next round). An enhancement bonus on a shield does not improve the effectiveness of a shield bash made with it, but the shield can be made into a magic weapon in its own right.
\parhead{Shield, Light, Wooden or Steel} You strap a shield to your forearm and grip it with your hand. A light shield's weight lets you carry other items in that hand, although you cannot use weapons or cast spells with it.
\subparhead{Wooden or Steel} Wooden and steel shields offer the same basic protection, though they respond differently to special attacks.
\subparhead{Shield Bash Attacks} You can bash an opponent with a light shield, using it as a light bludgeoning weapon. See \trefnp{Weapons} for the damage dealt by a shield bash. If you use your shield as a weapon, you lose its defense bonus until your next action (usually until the next round). An enhancement bonus on a shield does not improve the effectiveness of a shield bash made with it, but the shield can be made into a magic weapon in its own right.
\parhead{Shield, Tower} This massive wooden shield is nearly as tall as you are. In most situations, it provides the indicated bonus to your physical defenses. However, when you take the total defense action with a tower shield, you can treat the shield as a wall along one edge of your square, providing you with total cover. The shield does not, however, provide cover against targeted spells; a spellcaster can cast a spell on you by targeting the shield you are holding. You cannot bash with a tower shield, nor can you use your shield hand for anything else.
\par When employing a tower shield in combat, you take a \minus2 penalty on attack rolls because of the shield's encumbrance.
\parhead{Shield Spikes} When added to your shield, these spikes turn it into a piercing weapon that increases the damage dealt by a shield bash as if the shield were designed for a creature one size category larger than you. You can't put spikes on a buckler or a tower shield. Otherwise, attacking with a spiked shield is like making a shield bash attack (see above).
\par An enhancement bonus on a spiked shield does not improve the effectiveness of a shield bash made with it, but a spiked shield can be made into a magic weapon in its own right.
\parhead{Splint Mail} The suit includes gauntlets.
\parhead{Studded Leather} Only the studs on studded leather are made of metal. If the studs are made from a special material, only the studs are affected, not the whole armor. For example, ironwood studded leather weighs 17.5 pounds, since the weight of the studs is halved.

\subsection{Armor for Unusual Creatures}\label{Armor for Unusual Creatures}
Armor and shields for unusually big creatures, unusually little creatures, and nonhumanoid creatures have different costs and weights from those given on \trefnp{Armor and Shields}. Refer to the appropriate line on the table below and apply the multipliers to cost and weight for the armor type in question.
\begin{dtable}
    \lcaption{Armor for Unusual Creatures}
\begin{tabularx}{\columnwidth}{>{\lcol}X c c c c}
  & \multicolumn{2}{c}{\thead{Humanoid}} & \multicolumn{2}{c}{\thead{Nonhumanoid}} \\
\thead{Size} & \thead{Cost} & \thead{Weight} & \thead{Cost} & \thead{Weight} \\
Tiny or smaller\fn{1} & \mult1/2 & \mult1/10 & \mult1 & \mult1/10 \\
Small & \mult1 & \mult1/2 & \mult2 & \mult1/2 \\
Medium & \mult1 & \mult1 & \mult2 & \mult1 \\
Large & \mult2 & \mult2 & \mult4 & \mult2 \\
Huge & \mult4 & \mult4 & \mult8 & \mult4 \\
Gargantuan & \mult8 & \mult8 & \mult16 & \mult8 \\
Colossal & \mult16 & \mult12 & \mult32 & \mult12 \\
\end{tabularx}
1 Divide armor bonus by 2.
\end{dtable}

\subsection{Special Materials}
Armor and shields can be made from special materials, which can alter the properties of the item. The most common special material is ironwood, which is made from wood magically treated using the \spell{ironwood} ritual to be as strong as steel. Ironwood armor functions exactly like  armor of the same type, except that it is made of wood. This causes it to weigh half the normal weight, and allows druids to wear it without penalty. However, it costs twice as much as armor of the same type would normally cost, to a minimum of an additional 200 gp.

