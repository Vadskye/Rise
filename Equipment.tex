\chapter{Equipment}

You begin with 100 gold pieces, and you decide how to spend them.
Assume a character owns at least one outfit of normal clothes. Pick any one of the following clothing outfits: artisan's outfit, entertainer's outfit, explorer's outfit, monk's outfit, peasant's outfit, scholar's outfit, or traveler's outfit.
\section{Wealth And Money}

\subsection{Coins}
The most common coin is the gold piece (gp). A gold piece is worth 10 silver pieces. Each silver piece is worth 10 copper pieces (cp). In addition to copper, silver, and gold coins, there are also platinum pieces (pp), which are each worth 10 gp.

The standard coin weighs about a third of an ounce (fifty to the pound).

\begin{dtable}
\lcaption{Coin Exchange Values}
\begin{tabularx}{\columnwidth}{l c *{4}{>{\ccol}X}}
& & \thead{CP} & \thead{SP} & \thead{GP} & \thead{PP} \\
Copper piece (cp) & = & 1 & 1/10 & 1/100 & 1/1,000 \\
Silver piece (sp) & = & 10 & 1 & 1/10 & 1/100 \\
Gold piece (gp) & = & 100 & 10 & 1 & 1/10 \\
Platinum piece (pp) & = & 1,000 & 100 & 10 & 1
\end{tabularx}
\end{dtable}

\subsection{Wealth Other Than Coins}
Merchants commonly exchange trade goods without using currency. As a means of comparison, some trade goods are detailed below.

\begin{dtable}
\lcaption{Trade Goods}
\begin{tabularx}{\columnwidth}{l >{\lcol}X}
\thead{Cost} & \thead{Item} \\
1 cp & One pound of wheat \\
2 cp & One pound of flour \\
1 sp & One pound of iron, or one chicken \\
5 sp & One pound of tobacco or copper \\
1 gp & One pound of cinnamon, or one goat \\
2 gp & One pound of ginger or pepper, or one sheep \\
3 gp & One pig \\
4 gp & One square yard of linen \\
5 gp & One pound of salt or silver \\
10 gp & One square yard of silk, or one cow \\
15 gp & One pound of saffron or cloves, or one ox \\
50 gp & One pound of gold \\
500 gp & One pound of platinum
\end{tabularx}
\end{dtable}

\subsection{Selling Loot}
In general, a character can sell something for half its listed price.

Trade goods are the exception to the half-price rule. A trade good, in this sense, is a valuable good that can be easily exchanged almost as if it were cash itself.

\section{Weapons}

Weapons are grouped into several interlocking sets of categories. These categories pertain to what weapon group the weapon belongs to (axes, bows, and so on), the weapon's usefulness either in close combat (melee) or at a distance (ranged, which includes both thrown and projectile weapons), its relative encumbrance (light, medium, or heavy), and its size (Small, Medium, or Large).

\subsection{Weapon Groups}

Weapons are organized into thematically related categories called weapon groups. They are described in \trefnp{Weapon Groups}. For example, all axes belong to the ``axes'' weapon group. Some weapons can be found in multiple weapon groups. For example, a dagger is a simple weapon, a light blade, and a thrown weapon.

\begin{dtable!*}
\lcaption{Weapon Groups}
\begin{tabularx}{\textwidth}{l >{\lcol}X >{\lcol}X}
\thead{Group} & \thead{Weapons} & \thead{Exotic Weapons} \\
Armor weapons & Heavy shield (and spiked), light shield (and spiked), spiked armor & \\
Axes & Battleaxe, greataxe, handaxe, throwing axe & Double axe, dwarven urgrosh,  dwarven waraxe \\
Blunt weapons & Club, greatclub, mace, morningstar, quarterstaff, sap & Gnome hooked hammer, maul \\
Blades, heavy & Falchion, greatsword, longsword, scimitar & Bastard sword, two-bladed sword \\
Blades, light & Dagger, punching dagger, rapier, short sword & Kukri \\
Bows & Longbow, shortbow & \\
Crossbows & Heavy crossbow, light crossbow & Hand crossbow, repeating crossbows \\
Flexible Weapons & Flail, heavy flail, nunchaku & Whip \\
Headed weapons & Heavy pick, light hammer, light pick, sickle, warhammer & \\
Monk weapons & Kama, nunchaku, quarterstaff, sai, shuriken, siangham & \\
Polearms & Glaive, guisarme, halberd, ranseur, scythe & \\
Simple weapons & Club, dagger, light crossbow, quarterstaff, unarmed strike & \\
Spears & Javelin, lance, longspear, shortspear, spear & \\
Thrown weapons & Dagger, dart, handaxe, javelin, light hammer, shuriken, sling & Bolas, net \\
\end{tabularx}
\end{dtable!*}

\subsubsection{Weapon Proficiency}
Each character is proficient with different weapon groups. These indicate the weapons that you can use effectively. You can wield weapons you are not proficient with, but you cannot use them to defend yourself, which can cause you to provoke attacks of opportunity for your actions (see \pcref{Attacks of Opportunity}). In addition, you take a \minus4 penalty to attack unless the improvised weapon is a light melee weapon.

\subsection{Weapon Encumbrance}
This is a measure of how much effort it takes to wield a weapon in combat. A weapon's encumbrance indicates whether a melee weapon, when wielded by a character of the weapon's size category, is considered a light weapon, a medium weapon, or a heavy weapon.

\parhead{Light} A light weapon can be used with more finesse than a medium weapon. The wielder can add either his Dexterity or his Strength to attack rolls with light weapons, whichever he prefers. In addition, light weapons are easier to use in the off-hand or while grappling.

\parhead{Medium} A medium weapon can be used in one hand. It is difficult, but possible, to wield a medium weapon in your off-hand. You can also hold a medium weapon in two hands. Changing grips to hold it in one hand or two hands is a swift action. 

\parhead{Heavy} Two hands are required to wield a heavy weapon. You can hold it in one hand, but while doing so you cannot attack or defend yourself with it. This usually causes you provoke an attack of opportunity (see \pcref{Attacks of Opportunity}). Changing grips to hold it in one hand or two hands is a swift action.

\subsection{Melee Weapons}
Melee weapons are used for making melee attacks against foes within your reach (see \pcref{Reach}). Most weapons are melee weapons.

\subsection{Ranged Weapons}
Ranged weapons are used to make attacks against distant foes within the range of your weapon. There are two kinds of ranged weapons: projectile weapons and thrown weapons.

\parhead{Range Increments} All ranged weapons have a ``range increment'', which indicates the distance at which you can effectively attack with the weapon. Any attack against a target within your range increment takes no penalty. For each range increment by which the target is distant from you, you take a cumulative \minus2 penalty to your attack roll. A thrown weapon has a maximum range of five range increments. A projectile weapon can shoot out to ten range increments.

\subsubsection{Thrown Weapons}\label{Thrown Weapons} Thrown weapons are thrown at a target to deal damage. They generally do not break when thrown.

\parhead{Thrown Weapons in Melee}\label{Thrown Weapons in Melee} You cannot defend yourself with a weapon you are throwing, which means that throwing weapons while threatened can cause you to provoke attacks of opportunity (see \pcref{Attacks of Opportunity}). To avoid this, you can attempt to use the weapon as a melee weapon. Some thrown weapon are not designed for use in melee, such as shurikens. When using such a weapon, you are always treated as nonproficient with it, which means you cannot defend yourself with it.

\parhead{Throwing Other Weapons} It is possible to throw a weapon that is not designed to be thrown. The range increment is 10 feet. You are treated as being nonproficient with the weapon, causing you to take a \minus4 penalty on the attack roll. If it hits, the weapon deals its normal damage, but it scores a critical threat only on a natural roll of 20 and deals double damage on a critical hit.

\parhead{Heavy Weapons} Heavy thrown weapons require a standard action to throw, rather than an attack action like normal. They are not always physically thrown with two hands, but they require the use of your entire body to propel the weapon, preventing you from using your offhand for anything else. This can cause you to provoke attacks of opportunity for being unable to defend yourself (see \pcref{Attacks of Opportunity}).

\subsection{Weapon Size} Every weapon has a size category. This designation indicates the size of the creature for which the weapon was designed. Weapons for unusually large creatures deal more damage, while weapons for unusually small creatures deal less damage. These differences are shown on \trefnp{Weapon Damage and Size}.

\begin{dtable}
\lcaption{Weapon Damage and Size}
\begin{tabularx}{\columnwidth}{*{6}{l} >{\lcol}X}
\thead{Medium} & \thead{Tiny} & \thead{Small} & \thead{Large} & \thead{Huge} & \thead{Gargantuan} & \thead{Colossal} \\
1d2  & \x  & 1   & 1d3  & 1d4  & 1d8  & 2d6  \\
1d3  & 1   & 1d2 & 1d4  & 1d6  & 1d10 & 2d8  \\
1d4  & 1d2 & 1d3 & 1d6  & 1d8  & 2d6  & 2d10 \\
1d6  & 1d3 & 1d4 & 1d8  & 1d10 & 2d8  & 4d6  \\
1d8  & 1d4 & 1d6 & 1d10 & 2d6  & 2d10 & 4d8  \\
1d10 & 1d6 & 1d8 & 2d6  & 2d8  & 4d6  & 4d10 \\
\end{tabularx}
\end{dtable}

\parhead{Physical Size} In general, a light weapon is an object two size categories smaller than the wielder, a medium weapon is an object one size category smaller than the wielder, and a heavy weapon is an object of the same size category as the wielder.

\parhead{Inappropriately Sized Weapons} A weapon's encumbrance is altered by one step for each size category of difference between the wielder's size and the size of the creature for which the weapon was designed. In addition, the wielder takes a \minus2 penalty to physical attacks per size difference. If a weapon's encumbrance would be changed to something other than light, medium, or heavy by this alteration, the creature can't wield the weapon at all.

\subsection{Improvised Weapons} Sometimes objects not crafted to be weapons nonetheless see use in combat. Because such objects are not designed for this use, any creature that uses one in combat is considered to be nonproficient with it. To determine the size category and appropriate damage for an improvised weapon, compare its relative size and damage potential to the weapon list to find a reasonable match. An improvised weapon will generally deal less damage than a similarly sized manufactured weapon. All improvised weapons scores a threat on a natural roll of 20 and deal double damage on a critical hit. An improvised thrown weapon has a range increment of 10 feet.

\subsection{Drawing and Sheathing Weapons}\label{Drawing and Sheathing Weapons}
Sheathing any weapon is a move action. The time it takes to draw a weapon depends on how encumbering the weapon is. Drawing a light weapon is a swift action, while drawing a medium or heavy weapon is a move action. Drawing a hidden weapon of any type is a standard action. 

\subsection{Natural Weapons}\label{Natural Weapons}
Every creature can attack with its body using an unarmed strike. Many monsters can attack more effectively with specific parts of their body, such as teeth and claws. These are called natural weapons. You are automatically proficient with any natural weapons you possess. Most humanoids possess no natural weapons other than an unarmed strike. Natural weapons are described on \tref{Natural Weapons}.

A creature with multiple natural weapons of the same type can designate one as a primary attack and one as a secondary attack, allowing it to fight with both at once (see \pcref{Two-Weapon Fighting}). You are only considered to have one unarmed strike, so you cannot two-weapon fight with only your unarmed strike (but see the unarmed warrior monk ability, \pref{Mnk:Unarmed Warrior}).

\subsection{Weapon Qualities}
Here is the format for weapon entries (given as column headings on \trefnp{Weapons}, below).

\parhead{Encumbrance} Describes whether the weapon is a light, medium, heavy, double, or ranged weapon.

\par This cost is the same for a Small or Medium version of the weapon. A Large version costs twice the listed price.
\parhead{Damage} The Damage column gives the damage dealt by the weapon on a successful hit. The damage given is for weapons used by Medium creatures. Weapons used by Small creatures have a damage die that is one size smaller, as shown on \trefnp{Weapon Damage and Size}.

If two damage ranges are given, the weapon is a double weapon. Use the second damage figure given for the double weapon's other side.

\parhead{Critical} The entry in this column notes how the weapon is used with the rules for critical hits. When your character scores a critical hit, roll the damage two, three, or four times, as indicated by its critical multiplier (using all applicable modifiers on each roll), and add all the results together.

\subparhead{Exception} Bonus damage dice over and above a weapon's normal damage dice are not multiplied when you score a critical hit.
\subparhead{\mult2} The weapon deals double damage on a critical hit.
\subparhead{\mult3} The weapon deals triple damage on a critical hit.
\subparhead{\mult3/\mult4} One head of this double weapon deals triple damage on a critical hit. The other head deals quadruple damage on a critical hit.
\subparhead{\mult4} The weapon deals quadruple damage on a critical hit.
\subparhead{19-20/\mult2} The weapon scores a threat on a natural roll of 19 or 20 (instead of just 20) and deals double damage on a critical hit. (The weapon has a threat range of 19-20.)
\subparhead{18-20/\mult2} The weapon scores a threat on a natural roll of 18, 19, or 20 (instead of just 20) and deals double damage on a critical hit. (The weapon has a threat range of 18-20.)

\parhead{Range Increment} The range increment of the weapon.

\parhead{Type} Weapons are classified according to the type of damage they deal: bludgeoning, piercing, or slashing. Some monsters may be resistant or immune to attacks from certain types of weapons.

Some weapons deal damage of multiple types. If a weapon is of two types, the damage it deals is not half one type and half another; all of it is both types. Therefore, a creature would have to be immune to both types of damage to ignore any of the damage from such a weapon.

In other cases, a weapon can deal either of two types of damage. In a situation when the damage type is significant, the wielder can choose which type of damage to deal with such a weapon.
\parhead{Cost} This value is the weapon's cost in gold pieces (gp) or silver pieces (sp). The cost includes miscellaneous gear that goes with the weapon.
\parhead{Weight} This column gives the weight of a Medium version of
the weapon. Halve this number for Small weapons, and double it for
Large weapons.

\parhead{Special} Some weapons have special features. See the weapon
descriptions for details.

\begin{dtable!*}
\lcaption{Weapons}
    \begin{tabularx}{\textwidth}{p{12em} c c >{\ccol}p{10em} c c >{\ccol}X}
    \thead{Weapons} & \thead{Encumbrance} & \thead{Dmg} & \thead{Damage Type\fn{1}} & \thead{Cost} & \thead{Weight\fn{2}} & \thead{Special} \\

    Armor weapons &&&&&& \\
    \tind Shield, heavy\fn{3} & Medium & 1d4 & Bludgeoning & special & special & Forceful \\
    \tind Shield, light\fn{3} & Light & 1d3 & Bludgeoning & special & special & Forceful \\
    \tind Spiked armor\fn{3} & Light & 1d6 & Piercing & special & special & Grappling \\
    \tind Spiked shield, heavy\fn{3} & Medium & 1d6 & Piercing & special & special & Forceful \\
    \tind Spiked shield, light\fn{3} & Light & 1d4 & Piercing & special & special & Forceful \\

    Axes &&&&&& \\
    \tind Axe, throwing & Light & 1d6 & Slashing & 8 gp & 2 lb. & Throwing (10 ft.) \\
    \tind Battleaxe & Medium & 1d8 & Slashing & 10 gp & 6 lb. & Impact \\
    \tind Greataxe & Heavy & 1d10 & Slashing & 20 gp & 12 lb. & Impact \\
    \tind Handaxe & Light & 1d6 & Slashing & 6 gp & 3 lb. & Impact \\
    \tind Waraxe, dwarven & Heavy & 1d10 & Slashing & 75 gp & 8 lb. & Exotic Grip \\

    Blades, heavy &&&&&& \\
    \tind Falchion & Heavy & 1d10 & Slashing & 50 gp & 8 lb. & Keen \\
    \tind Greatsword & Heavy & 1d10 & Slashing & 25 gp & 8 lb. & Keen \\
    \tind Longsword & Medium & 1d8 & Slashing & 15 gp & 4 lb. & Keen \\
    \tind Scimitar & Medium & 1d8 & Slashing & 15 gp & 4 lb. & Keen \\

    Blades, light &&&&&& \\
    \tind Dagger & Light & 1d4 & Piercing or slashing & 2 gp & 1 lb. & Keen, Small, Throwing (10 ft.) \\
    \tind Dagger, punching & Light & 1d4 & Piercing & 2 gp & 1 lb. & Impact, Small \\
    \tind Rapier & Medium & 1d6 & Piercing & 20 gp & 2 lb. & Finesse, Impact \\
    \tind Sword, short & Light & 1d6 & Piercing or slashing & 10 gp & 2 lb. & Keen \\

    Blunt weapons &&&&&& \\
    \tind Club & Medium & 1d6 & Bludgeoning & \x & 3 lb. & \x \\
    \tind Greatclub & Heavy & 1d10 & Bludgeoning & 5 gp & 8 lb. & \x \\
    \tind Mace & Light & 1d6 & Bludgeoning & 12 gp & 8 lb. & \x \\
    \tind Morningstar & Medium & 1d8 & Bludgeoning and piercing & 8 gp & 6 lb. & \x \\
    \tind Quarterstaff & Heavy & 1d6/1d6 & Bludgeoning & \x & 4 lb. & Double \\
    \tind Sap & Light & 1d6 & Bludgeoning & 1 gp & 2 lb. & Nonlethal \\

    Bows &&&&&& \\
    \tind Longbow\fn{3} & Heavy (Ranged) & 1d8 & Piercing & 40 gp & 3 lb. & Impact, Projectile (100 ft.) \\
    \tind Shortbow\fn{3} & Medium (Ranged) & 1d6 & Piercing & 30 gp & 2 lb. & Impact, Projectile (50 ft.) \\
    \tind Arrows (20) & \x & \x & \x & 1 gp & 3 lb. & Ammunition \\

    Crossbows &&&&&& \\
    \tind Crossbow, heavy\fn{3} & Heavy (Ranged) & 1d10 & Piercing & 50 gp & 8 lb. & Impact, Projectile (100 ft.) \\
    \tind Crossbow, light\fn{3} & Medium (Ranged) & 1d8 & Piercing & 40 gp & 4 lb. & Impact, Projectile (50 ft.) \\
    \tind Bolts, crossbow (10) & \x & \x & \x & 1 gp & 1 lb. & Ammunition \\

    Flexible weapons &&&&&& \\
    \tind Flail  & Medium & 1d8 & Bludgeoning & 8 gp & 5 lb. & Disarming, Tripping \\
    \tind Flail, heavy & Heavy & 1d10 & Bludgeoning & 15 gp & 10 lb. & Disarming, Tripping \\

\end{tabularx}
\end{dtable!*}

\begin{dtable!*}
    \begin{tabularx}{\textwidth}{p{12em} c c >{\ccol}p{10em} c c >{\ccol}X}
    \thead{Weapons} & \thead{Encumbrance} & \thead{Dmg} & \thead{Damage Type\fn{1}} & \thead{Cost} & \thead{Weight\fn{2}} & \thead{Special} \\
        Headed weapons &&&&&& \\
        \tind Hammer, light & Light & 1d4 & Bludgeoning & 1 gp & 2 lb. & Throwing (20 ft.) \\
        \tind Pick, heavy & Medium & 1d8 & Piercing & 8 gp & 6 lb. & Impact, Unbalanced \\
        \tind Pick, light & Light & 1d6 & Piercing & 4 gp & 3 lb. & Impact, Unbalanced \\
        \tind Sickle & Light & 1d6 & Slashing & 6 gp & 2 lb. & Tripping \\
        \tind Warhammer & Medium & 1d8 & Bludgeoning & 12 gp & 5 lb. & Impact \\

        Monk weapons &&&&&& \\
        \tind Kama & Light & 1d6 & Slashing & 2 gp & 2 lb. & Tripping \\
        \tind Nunchaku & Light & 1d6 & Bludgeoning & 2 gp & 2 lb. & Disarming, Unbalanced \\
        \tind Quarterstaff & Heavy & 1d6/1d6 & Bludgeoning & \x & 4 lb. & Double \\
        \tind Sai & Light & 1d4 & Piercing or bludgeoning & 1 gp & 1 lb. & Disarming, Impact \\
        \tind Shuriken (5) & Light (Ranged) & 1d4 & Piercing & 1 gp & 1/2 lb. & Ammunition, Thrown (10 ft.) \\
        \tind Siangham & Light & 1d6 & Piercing & 3 gp & 1 lb. & Parrying \\

        Polearms &&&&&& \\
        \tind Glaive & Heavy & 1d10 & Slashing & 8 gp & 10 lb. & Impact, Reach \\
        \tind Guisarme & Heavy & 1d10 & Slashing & 9 gp & 12 lb. & Reach, Tripping \\
        \tind Halberd & Heavy & 1d10 & Piercing or slashing & 10 gp & 12 lb. & Impact, Reach \\
        \tind Quarterstaff & Heavy & 1d6/1d6 & Bludgeoning & \x & 4 lb. & Double \\
        \tind Ranseur & Heavy & 1d10 & Piercing & 10 gp & 12 lb. & Disarming, Reach \\
        \tind Scythe & Heavy & 1d10 & Slashing & 18 gp & 10 lb. & Impact \\

        Simple weapons &&&&&& \\
        \tind Club & Medium & 1d6 & Bludgeoning & \x & 3 lb. & \x \\
        \tind Crossbow, light\fn{3} & Medium (Ranged) & 1d8 & Piercing & 35 gp & 4 lb. & Projectile (100 ft.) \\
        \tind Dagger & Light & 1d4 & Piercing or slashing & 2 gp & 1 lb. & Keen, Small, Thrown (10 ft.) \\
        \tind Quarterstaff & Heavy & 1d6/1d6 & Bludgeoning & \x & 4 lb. & Double \\
        \tind Unarmed strike & Light & 1d3 & Bludgeoning & \x & \x & Nonlethal, Unarmed \\

        Spears &&&&&& \\
        \tind Javelin & Medium (Ranged) & 1d6 & Piercing & 1 gp & 2 lb. & Thrown (30 ft.) \\
        \tind Lance & Heavy & 1d8 & Piercing & 10 gp & 10 lb. & Charging, Impact, Reach \\
        \tind Longspear & Heavy & 1d8 & Piercing & 5 gp & 9 lb. & Bracing, Imapct, Reach \\
        \tind Shortspear & Light & 1d6 & Piercing & 1 gp & 3 lb. & Thrown (20 ft.) \\
        \tind Spear & Medium & 1d8 & Piercing & 2 gp & 6 lb. & Bracing, Thrown (20 ft.) \\

        Thrown weapons &&&&&& \\
        \tind Axe, throwing & Light & 1d6 & Slashing & 8 gp & 2 lb. & Thrown (10 ft.) \\
        \tind Dagger & Light & 1d4 & Piercing or slashing & 2 gp & 1 lb. & Keen, Small, Thrown (10 ft.) \\
        \tind Dart (5) & Light (Ranged) & 1d4 & Piercing & 1 gp & 1/2 lb. & Ammunition, Thrown (20 ft.) \\
        \tind Hammer, light & Light & 1d4 & Bludgeoning & 1 gp & 2 lb. & Thrown (20 ft.) \\
        \tind Javelin & Medium (Ranged) & 1d6 & Piercing & 1 gp & 2 lb. & Thrown (30 ft.) \\
        \tind Shuriken (5) & Light (Ranged) & 1d2 & Piercing & 1 gp & 1/2 lb. & Ammunition, Thrown (10 ft.) \\
        \tind Sling\fn{3} & Light (Ranged) & 1d4 & Bludgeoning & 2 gp & 0 lb. & Projectile (50 ft.) \\
        \tind Bullets, sling (20) & \x & \x & \x & 1 gp & 5 lb. & Ammunition \\

        Unarmed weapons &&&&&&\\
        \tind Claw Sheath\fn{3} & \x & \x & \x & 50 gp & 3 lb. & Special \\
        \tind Gauntlet & Light & 1d3 & Bludgeoning & 2 gp & 1 lb. & Unarmed \\
        \tind Gauntlet, spiked & Light & 1d4 & Piercing & 5 gp & 1 lb. & Unarmed \\
        \tind Unarmed strike & Light & 1d3 & Bludgeoning & \x & \x & Nonlethal, Unarmed \\
    \end{tabularx}
\end{dtable!*}

\begin{dtable!*}
    \begin{tabularx}{\textwidth}{p{12em} c c >{\ccol}p{10em} c c >{\ccol}X}
        \thead{Exotic Weapons} & \thead{Encumbrance} & \thead{Dmg} & \thead{Damage Type\fn{1}} & \thead{Cost} & \thead{Weight\fn{2}} & \thead{Special} \\
        Armor &&&&&& \\
        Axes &&&&&& \\
        \tind Axe, orc double & Heavy & 1d8/1d8 & Slashing & 60 gp & 15 lb. & Double, Impact \\
        \tind Urgrosh, dwarven\fn{3} & Heavy & 1d8/1d6 & Slashing or piercing & 50 gp & 12 lb. & Bracing, Double, Impact \\
        Blunt weapons &&&&&& \\
        Blades, heavy &&&&&& \\
        \tind Sword, bastard & Heavy & 1d10 & Slashing & 75 gp & 6 lb. & \x \\
        \tind Sword, two-bladed & Heavy & 1d8/1d8 & Slashing & 100 gp & 10 lb. & Double \\
        Blades, light &&&&&& \\
        \tind Kukri & Light & 1d4 & Slashing & 8 gp & 2 lb. & \x \\
        Bows &&&&&& \\
        Crossbows &&&&&& \\
        \tind Crossbow, hand\fn{3} & Light (Ranged) & 1d4 & Piercing & 100 gp & 2 lb. & Projectile (30 ft.) \\
        \tind Crossbow, repeating heavy\fn{3} & Heavy (Ranged) & 1d10 & Piercing & 400 gp & 12 lb. & Projectile (100 ft.) \\
        \tind Crossbow, repeating light\fn{3} & Medium (Ranged) & 1d8 & Piercing & 250 gp & 6 lb. & Projectile (50 ft.) \\
        \tind Bolts, hand (10) & \x & \x & \x & 1 gp & 1 lb. & Ammunition \\
        \tind Bolts, repeating (5) & \x & \x & \x & 1 gp & 1 lb. & Ammunition \\
        Flexible weapons &&&&&& \\
        \tind Flail, dire & Heavy & 1d8/1d8 & Bludgeoning & 90 gp & 10 lb. & Disarming, Double, Tripping \\
        \tind Whip\fn{3} & Light & 1d3 & Slashing & 1 gp & 2 lb. & Disarming, Nonlethal, Tripping \\
        Headed weapons &&&&&& \\
        \tind Hammer, gnome hooked\fn{3} & Heavy & 1d8/1d6 & Bludgeoning or piercing & 20 gp & 6 lb. & Double, Impact, Tripping \\
        Monk weapons &&&&&& \\
        Polearms &&&&&& \\
        Simple weapons &&&&&& \\
        Spear &&&&&& \\
        \tind Urgrosh, dwarven\fn{3} & Heavy & 1d8/1d6 & Slashing or piercing & 50 gp & 12 lb. & Bracing, Double, Impact \\
        Thrown weapons &&&&&& \\
        \tind Bolas & Light (Ranged) & 1d4\fn{3} & Bludgeoning & 5 gp & 2 lb. & Thrown (10 ft.), Tripping \\
        \tind Net\fn{3} & Medium (Ranged) & \x & \x & 20 gp & 6 lb. & Thrown (10 ft.) \\
        Unarmed weapons &&&&&&\\
    \end{tabularx}
    1 When two types are given, the weapon is both types if the entry specifies ``and," or either type (attacker's choice) if the entry specifies ``or." \\
    2 Weight figures are for Medium weapons. A Small weapon weighs half as much, and a Large weapon weighs twice as much. \\
    3 This weapon has special rules. \\
\end{dtable!*}

\begin{dtable!*}
    \lcaption{Natural Weapons}
    \begin{tabularx}{\textwidth}{p{12em} c c >{\ccol}p{15em} >{\ccol}X}
        \thead{Natural Weapons} & \thead{Encumbrance} & \thead{Dmg} & \thead{Damage Type\footnotetemp{2}} & \thead{Special} \\
        Bite & Medium & 1d8 & Piercing and bludgeoning  & \x \\
        Claw & Light & 1d6 & Slashing and piercing & \x \\
        Constrict\fn{2} & Heavy & 1d10 & Bludgeoning & \x \\
        Gore & Heavy & 1d8 & Piercing & \x \\
        Slam & Medium & 1d8 & Bludgeoning & \x \\
        Unarmed Strike & Light & 1d3\fn{3} & Bludgeoning & Unarmed \\
    \end{tabularx}
    1 When two types are given, the weapon is both types if the entry specifies ``and," or either type (attacker's choice) if the entry specifies ``or." \\
    2 This attack can only be used against a foe you are grappling with. \\
\end{dtable!*}

\subsection{Weapon Properties}
Some weapons found on \trefnp{Weapons} have special properties. The list of special properties is given below.
\parhead{Ammunition} This weapon is designed to thrown or fired by a projectile weapon in large quantities. It is cheaper to buy and craft, but ammunition that hits its target is destroyed, and ammunition that misses has a 50\% chance to be destroyed or lost.
\parhead{Bracing} As a move action, you can brace this weapon against a charge for 1 round. While you are bracing your weapon, creatures that charge you provoke attacks of opportunity from you.
\parhead{Charging} This weapon deals double damage when used from the back of a charging mount.
\parhead{Disarming} You can use this weapon to make disarm atacks. You gain a \plus2 bonus on such attacks, and can apply magical bonuses from the weapon to the attack.
\parhead{Double} This weapon has more than one striking surface. You can fight with both ends simultaneously, just like two-weapon fighting (see \pcref{Two-Weapon Fighting}). Alternately, you can attack with one end at a time. If you have the ability to use a double weapon in one hand, you can only fight with one end at a time, not both.
\parhead{Exotic Grip} If you have have proficiency with exotic weapons, you can use this 
\parhead{Finesse} You apply your Dexterity instead of your Strength to physical attacks with this weapon, even if it isn't a light weapon for you.
\parhead{Forceful} You can use this weapon to make shove attacks to push people away from you. You gain a \plus2 bonus on such attacks, and can apply magical bonuses from the weapon to the attack.
\parhead{Grappling} You gain a \plus2 bonus to physical attacks with this weapon in a grapple.
\parhead{Impact} This weapon deals triple damage on a critical hit.
\parhead{Keen} This weapon scores a critical threat on a roll of a natural 19 or 20.
\parhead{Nonlethal} This weapon deals nonlethal damage rather than lethal damage. See \pcref{Nonlethal Damage}.
\parhead{Parrying} You can a \plus2 bonus to parry attempts with this weapon.
\parhead{Projectile} This weapon fires projectiles at range. Projectile weapons have a range increment listed in their description, which indicates the distance they can be easily fired. Projectile weapons must be reloaded. The time required to reload a projectile weapon is given in the weapon description.
\parhead{Reach} This weapon strikes at double your natural reach (so 10 feet for a typical Small or Medium creature). However, it cannot attack a creature within your natural reach.

Reach weapons can held using a different grip to strike nearby foes. This is called ``short hafting''. While short hafting a reach weapon, you ignore the weapon's reach property, but you take a \minus4 penalty to physical attacks with it.
\parhead{Small} This weapon is unusually small. It is one size category smaller than normal for a light weapon (that is, three size categories smaller than the creature it is intended for). This makes it easier to conceal (see \pcref{Sleight of Hand}).
\parhead{Throwing} This weapon is designed to be thrown. Throwing weapons have a range increment listed in their description, which indicates the distance they can be easily thrown. See \pcref{Thrown Weapons}.
\parhead{Tripping} You can use this weapon to make trip attacks. You gain a \plus2 bonus on such attacks, and can apply magical bonuses from the weapon to the attack.
\parhead{Unarmed} This weapon is used as part of an unarmed strike. It cannot be disarmed. Unless you have the Improved Unarmed Strike feat (see \featref{Improved Unarmed Strike}), you can't take attacks of opportunity or defend yourself with this weapon, which makes you defenseless. See \pcref{Attacks of Opportunity}.
\parhead{Unbalanced} If you roll either a 1 or 2 when attacking this weapon, you critically fail the attack (see \pcref{Critical Success and Failure}.

\subsection{Weapon Special Abilities}
Some weapons in \trefnp{Weapons} have unique special abilities, which are described below.
\parhead{Claw Sheath} A claw sheath is not a weapon in itself, but a covering over a single claw that a creature can use to make claw attacks. Claw sheaths do not grant a creature without claws a claw attack. Normally, a sheath does not improve the claw attack, but magical claw sheaths can be made which grant bonuses to attacks made with the claw. It may be possible to find or craft unusual sheaths for natural weapons other than claws.
\parhead{Crossbow, Hand} You can draw a hand crossbow back by hand. Loading a hand crossbow is a move action that provokes attacks of opportunity.
\par You can fire a crossbow from a prone position without penalty.
\parhead{Crossbow, Heavy} You draw a heavy crossbow back by turning a small winch. Loading a heavy crossbow is a full-round action that requires both hands.
\par You can fire a crossbow from a prone position without penalty.
\parhead{Crossbow, Light} You draw a light crossbow back by pulling a lever. Loading a light crossbow is a move action that requires both hands.
\par You can fire a crossbow from a prone position without penalty.
\parhead{Crossbow, Repeating} The repeating crossbow (whether heavy or light) holds 10 crossbow bolts. As long as it holds bolts, you can reload it by pulling the reloading lever (a free action). Loading a new case of 10 bolts is a full-round action that requires both hands.
\par You can fire a crossbow from a prone position without penalty.
\parhead{Hammer, Gnome Hooked} This weapon has a hammer head which deals 1d8 points of damage, and a hook which deals 1d6 points of damage. The hook is a tripping weapon.
\parhead{Longbow} You need both hands to fire a bow, regardless of its size. One hand is free when not firing the bow. Loading a bow is a free action that requires both hands. A longbow is too unwieldy to use while you are mounted.
\par When attacking with a bow, you take a \minus4 penalty against creatures that threaten you.
\parhead{Net} A net is used to entangle enemies. When you throw a net, you make a ranged touch attack against your target. If you hit, the target is entangled. An entangled creature moves at half speed, cannot run or charge, and takes a \minus2 penalty to physical attacks and defenses, as well as Strength and Dexterity-based checks. If you control the trailing rope by succeeding on an opposed Strength check while holding it, the entangled creature can move only within the limits that the rope allows. If the entangled creature attempts to cast a spell, it must make a DC 15 \add double spell level Concentration check or be unable to cast the spell.
\par An entangled creature can escape with a DC 20 Escape Artist check (a full-round action). The net has 5 hit points and can be burst with a DC 20 Strength check (also a full-round action).
\par A net is useful only against creatures within one size category of you.
\par A net must be folded to be thrown effectively. The first time you throw your net in a fight, you make a normal ranged touch attack roll. After the net is unfolded, you take a \minus4 penalty on attack rolls with it. It takes 2 rounds for a proficient user to fold a net and twice that long for a nonproficient one to do so.
\parhead{Shield, Heavy or Light} You can bash with a shield instead of using it for defense. See Armor for details.
\parhead{Shortbow} You need both hands to fire a bow, regardless of its size. One hand is free when not firing the bow. Loading a bow is a free action that requires both hands.
\par When attacking with a bow, you take a \minus4 penalty against creatures that threaten you.
\parhead{Sling} You can fire, but not load, a sling with one hand. Loading a sling is a free action that requires both hands.
\par You can hurl ordinary stones with a sling, but stones are not as dense or as round as bullets. You take a \minus1 penalty to attack and damage rolls when using ordinary stones.
\parhead{Spiked Armor} You can outfit your armor with spikes, which can deal damage in a grapple. See Armor for details.
\parhead{Spiked Shield, Heavy or Light} You can bash with a spiked shield instead of using it for defense. See Armor for details.

\parhead{Unarmed Strike} Anyone can attack unarmed, but it is more dangerous than attacking with a weapon. See \pcref{Unarmed Combat}.

\parhead{Urgrosh, Dwarven} This weapon has an axe head which deals 1d8 points of damage, and a spear which deals 1d6 points of damage. The spear is a bracing weapon. You must be proficient with axes to use the axe head, and proficient with spears to use the spear head.
\parhead{Whip} A whip is treated as a light melee weapon with 15 foot reach, though you don't threaten the area into which you can make an attack. In addition, unlike most other weapons with reach, you can use it against foes anywhere within your reach (including adjacent foes).
\par Using a whip provokes attacks of opportunity, just as if you had used a ranged weapon.

\section{Armor}

Most characters use armor to protect themselves. There are two kinds of armor: body armor, such as chainmail or full plate, and shields.

\subsection{Armor Qualities}
\par Here is the format for armor entries (given as column headings on \trefnp{Armor and Shields}, below).

\parhead{Encumbrance} All armor restricts a character's movement to some degree. The heavier the armor, the more restrictive it is. Shields are divided into light shields (including bucklers), heavy shields, or tower shields, while body armor can be light, medium, or heavy. Many classes are not proficient with heavier kinds of armor.

\parhead{Cost} The cost of the armor for Small or Medium humanoid
creatures. See \trefnp{Armor for Unusual Creatures}, below, for armor prices for other creatures.
\parhead{Armor/Shield Bonus} Body armor improves your armor defense modifier, while shields improve your shield defense modifier. Wearing multiple suits of armor or wielding multiple shields does not improve your defenses any further.

\parhead{Dexterity} Medium and heavy body armor limits mobility and agility, halving the character's Dexterity. This halving is applied in addition to the armor's armor check penalty (if any). A Dexterity penalty is not halved.

Your character's encumbrance (the amount of gear he or she carries) may also affect your character's Dexterity bonus.
\subparhead{Shields} Most shields do not affect a character's Dexterity. However, a tower shield halves a character's Dexterity. This is not cumulative with the halving from wearing medium or heavy armor.
\parhead{Armor Check Penalty} Any armor heavier than leather hurts a character's ability to use some skills. An armor check penalty number is the penalty that applies to Acrobatics, Climb, Escape Artist, Jump, Sleight of Hand, and Stealth checks by a character wearing a certain kind of armor. Double the normal armor check penalty is applied to Swim checks. A character's encumbrance (the amount of gear carried, including armor) may also apply an armor check penalty.
\subparhead{Shields} If a character is wearing armor and using a shield, both armor check penalties apply.
\subparhead{Nonproficient with Armor Worn} A character who wears armor and/or uses a shield with which he or she is not proficient takes the armor's (and/or shield's) armor check penalty on attack rolls and on all Strength-based and Dexterity-based ability and skill checks. The penalty for nonproficiency with armor stacks with the penalty for nonproficiency with shields.

\subparhead{Sleeping in Armor} A character who sleeps overnight in medium or heavy body armor is automatically fatigued the next day. He or she takes a \minus2 penalty on Strength and Dexterity and can't charge or run. Sleeping in light armor does not cause fatigue.
\parhead{Arcane Spell Failure} Armor interferes with the gestures that a spellcaster must make to cast an arcane spell that has a somatic component. Arcane spellcasters face the possibility of arcane spell failure if they're wearing armor.

\subparhead{Casting an Arcane Spell in Armor} A character who casts an arcane spell while wearing armor must usually make an arcane spell failure roll. The number in the Arcane Spell Failure Chance column on Table: Armor and Shields is the chance that the spell fails and is ruined. If the spell lacks a somatic component, however, it can be cast with no chance of arcane spell failure.

\subparhead{Shields} If a character is wearing armor and using a shield, add the two numbers together to get a single arcane spell failure chance.
\parhead{Speed} Medium or heavy armor slows the wearer down. The number on Table: Armor and Shields is the character's speed while wearing the armor. Humans, elves, half-elves, and half-orcs have an unencumbered speed of 30 feet.
\par They use the first column. Dwarves, gnomes, and halflings have an unencumbered speed of 20 feet. They use the second column. Remember, however, that a dwarf's land speed remains 20 feet even in medium or heavy armor or when carrying a medium or heavy load.
\subparhead{Shields} Shields do not affect a character's speed.
\subparhead{Flat-Footed} A character's shield defense modifier is lost when the character is flat-footed.
\parhead{Weight} This column gives the weight of the armor sized for a Medium wearer. Armor fitted for Small characters weighs half as much, and armor for Large characters weighs twice as much.

\subsubsection{Moving in Armor}
The heavier the armor, the more it slows your movement. Light armor has no effect on your movement speed, though it does impose armor check penalties. Running in medium or heavy armor is difficult, and you can run at only three times your movement speed instead of four times your movement speed. Heavy armor slows down your movement speed even further, causing you to move at 2/3 of your normal movement rate.

\subsection{Getting Into And Out Of Armor}
The time required to don armor depends on its type; see \trefnp{Donning Armor}. Donning armor of any kind takes both hands.
\parhead{Don} This column tells how long it takes a character to put the armor on. (One minute is 10 rounds.) Readying (strapping on) a shield is only a move action.
\parhead{Don Hastily} This column tells how long it takes to put the armor on in a hurry. The armor check penalty and armor bonus for hastily donned armor are each 1 point worse than normal.
\parhead{Remove} This column tells how long it takes to get the armor off. Loosing a shield (removing it from the arm and dropping it) is only a move action.

\begin{dtable}
\lcaption{Donning Armor}
\begin{tabularx}{\columnwidth}{>{\lcol}X c c c}
\thead{Armor Type} & \thead{Don} & \thead{Don Hastily} & \thead{Remove} \\
Shield (any) & 1 move action & n/a & 1 move action \\
Padded, leather, hide, studded leather, or chain shirt & 1 minute & 5 rounds & 1 minute\fn{1} \\
Breastplate, scale mail, chainmail, banded mail, or splint mail & 4 minutes\fn{1} & 1 minute & 1 minute\fn{1} \\
Half-plate or full plate & 4 minutes\fn{2} & 4 minutes\fn{1} & 1d4\plus1 minutes\fn{1} \\
\end{tabularx}
1 If the character has some help, cut this time in half. A single character doing nothing else can help one or two adjacent characters. Two characters can't help each other don armor at the same time. \\
2 The wearer must have help to don this armor. Without help, it can be donned only hastily.
\end{dtable}

\begin{dtable!*}
\lcaption{Armor and Shields}
\begin{tabularx}{\textwidth}{>{\lcol}X >{\ccol}p{6em} >{\ccol}p{5em} >{\ccol}p{6em} >{\ccol}p{7em} c c c c}
  &  &  &  &  &  \multicolumn{2}{c}{\x\x Speed \x\x} & & \\
\thead{Armor} & \thead{Armor/Shield Bonus} & \thead{Dex Modifier} & \thead{Armor Check Penalty} & \thead{Arcane Spell Failure Chance} & \thead{(30 ft.)} & \thead{(20 ft.)} & \thead{Cost} & \thead{Weight\fn{1}} \\
Light armor &  &  &  &  &  &  &  &  \\
\tind Padded 				& \plus1 		& 1\mult 	& 0       & 5\%  & 30 ft. & 20 ft. & 5 gp          & 10 lb.\\
\tind Leather 				& \plus2 		& 1\mult 	& \minus1 & 10\% & 30 ft. & 20 ft. & 10 gp        & 15 lb.\\
\tind Studded leather 		& \plus3 		& 1\mult 	& \minus2 & 15\% & 30 ft. & 20 ft. & 25 gp & 20 lb.\\
Medium armor 				&  &  &  &  &  &  &  & \\
\tind Chain shirt 			& \plus4 		& 1/2\mult 	& \minus2 & 20\% & 30 ft. & 20 ft. & 100 gp   & 25 lb.\\
\tind Hide 					& \plus4 		& 1/2\mult 	& \minus4 & 20\% & 30 ft. & 20 ft. & 15 gp           & 25 lb.\\
\tind Scale mail 			& \plus5 		& 1/2\mult 	& \minus4 & 25\% & 30 ft. & 20 ft. & 50 gp     & 30 lb.\\
\tind Chainmail 			& \plus6 		& 1/2\mult 	& \minus5 & 30\% & 30 ft. & 20 ft. & 100 gp     & 40 lb.\\
\tind Breastplate 			& \plus6 		& 1/2\mult 	& \minus4 & 25\% & 30 ft. & 20 ft. & 200 gp   & 30 lb.\\
Heavy armor 				&  &  &  &  &  &  &  & \\
\tind Splint mail 			& \plus7 		& 1/2\mult 	& \minus7 & 40\% & 20 ft. & 15 ft. & 200 gp   & 45 lb.\\
\tind Half-plate 			& \plus8 		& 1/2\mult 	& \minus8 & 40\% & 20 ft. & 15 ft. & 500 gp    & 50 lb.\\
\tind Full plate 			& \plus8 		& 1/2\mult 	& \minus6 & 35\% & 20 ft. & 15 ft. & 1,000 gp  & 50 lb.\\
Shields 					&  &  &  &  &  &  &  & \\
\tind Buckler 				& \plus1 		& \x 		& \minus1 & 5\% & \x & \x & 15 gp             & 5 lb.\\
\tind Shield, light wooden 	& \plus1 		& \x 		& \minus2 & 5\% & \x & \x & 3 gp & 5 lb.\\
\tind Shield, light steel 	& \plus1 		& \x 		& \minus2 & 5\% & \x & \x & 9 gp  & 6 lb.\\
\tind Shield, heavy wooden 	& \plus2 		& \x 		& \minus3 & 15\% & \x & \x & 7 gp & 10 lb.\\
\tind Shield, heavy steel 	& \plus2 		& \x 		& \minus3 & 15\% & \x & \x & 20 gp & 15 lb.\\
\tind Shield, tower 		& \plus4\fn{2} 	& 1/2\mult 	& \minus10 & 50\% & \x & \x & 30 gp       & 45 lb.\\
Extras 						&  &  &  &  &  &  &  & \\
\tind Armor spikes 			& \minus1\fn{3} & \x 		& \minus2 & \x & \x & \x & \plus50 gp & \plus10 lb.\\
\tind Gauntlet, locked 		& \x 			& \x 		& Special & \fn{4} & \x & \x & 8 gp & \plus5 lb.\\
\tind Shield spikes 		& \x 			& \x 		& \minus1 & \x & \x & \x & \plus10 gp & \plus5 lb.\\
\end{tabularx}
1 Weight figures are for armor sized to fit Medium characters. Armor fitted for Small characters weighs half as much, and armor fitted for Large characters weighs twice as much. \\
2 A tower shield can instead grant you cover. See the description. \\
3 Armor spikes reduce the defense bonus granted by the armor they are put on by 1. \\
4 Hand not free to cast spells.
\end{dtable!*}

\subsection{Armor Descriptions}
Any special benefits or accessories to the types of armor found on \trefnp{Armor and Shields} are described below.
\parhead{Armor Spikes} You can have spikes added to your armor, which allow you to deal extra piercing damage (see Table: Weapons) on a successful grapple or overrun attack. If you are not proficient with the spikes, you only deal the damage if you succeed at the attack by 10 or more.
\par An enhancement bonus to a suit of armor does not improve the spikes' effectiveness, but the spikes can be made into magic weapons in their own right.
\parhead{Banded Mail} The suit includes gauntlets.
\parhead{Breastplate} It comes with a helmet and greaves.
\parhead{Buckler} This small metal shield is worn strapped to your forearm. You can use a bow or crossbow without penalty while carrying it. You can also use your shield arm to wield a weapon (whether you are using an off-hand weapon or using your off hand to help wield a heavy weapon), but you take a \minus1 penalty on attack rolls while doing so. If you use a weapon in your off hand, you don't get the buckler's defense bonus for the rest of the round.
\par Unlike most shields, you do not apply your buckler's shield bonus to your touch defense.
\par You can't bash someone with a buckler.
\parhead{Chain Shirt} A chain shirt comes with a steel cap.
\parhead{Chainmail} The suit includes gauntlets.
\parhead{Full Plate} The suit includes gauntlets, heavy leather boots, a visored helmet, and a thick layer of padding that is worn underneath the armor. Each suit of full plate must be individually fitted to its owner by a master armorsmith, although a captured suit can be resized to fit a new owner with a day of work and a DC 25 Craft (metalworking) check. The new owner must still be of the same size category as the size category that the suit was designed for.
\parhead{Gauntlet, Locked} This armored gauntlet has small chains and braces that allow the wearer to attach a weapon to the gauntlet so that it cannot be dropped easily. It provides a \plus10 bonus on any roll made to keep from being disarmed in combat. Removing a weapon from a locked gauntlet or attaching a weapon to a locked gauntlet is a full-round action that provokes attacks of opportunity.
\par The price given is for a single locked gauntlet. The weight given applies only if you're wearing a breastplate, light armor, or no armor. Otherwise, the locked gauntlet replaces a gauntlet you already have as part of the armor.
\par While the gauntlet is locked, you can't use the hand wearing it for casting spells or employing skills. (You can still cast spells with somatic components, provided that your other hand is free.)
Like a normal gauntlet, a locked gauntlet lets you deal lethal damage rather than nonlethal damage with an unarmed strike.
\parhead{Half-Plate} The suit includes gauntlets.
\parhead{Scale Mail} The suit includes gauntlets.
\parhead{Shield, Heavy, Wooden or Steel} You strap a shield to your forearm and grip it with your hand. A heavy shield is so heavy that you can't use your shield hand for anything else.
\subparhead{Wooden or Steel} Wooden and steel shields offer the same basic protection, though they respond differently to special attacks.
\subparhead{Shield Bash Attacks} You can bash an opponent with a heavy shield, using it as a medium bludgeoning weapon. See \trefnp{Weapons} for the damage dealt by a shield bash.  If you use your shield as a weapon, you lose its defense bonus until your next action (usually until the next round). An enhancement bonus on a shield does not improve the effectiveness of a shield bash made with it, but the shield can be made into a magic weapon in its own right.
\parhead{Shield, Light, Wooden or Steel} You strap a shield to your forearm and grip it with your hand. A light shield's weight lets you carry other items in that hand, although you cannot use weapons or cast spells with it.
\subparhead{Wooden or Steel} Wooden and steel shields offer the same basic protection, though they respond differently to special attacks.
\subparhead{Shield Bash Attacks} You can bash an opponent with a light shield, using it as a light bludgeoning weapon. See \trefnp{Weapons} for the damage dealt by a shield bash. If you use your shield as a weapon, you lose its defense bonus until your next action (usually until the next round). An enhancement bonus on a shield does not improve the effectiveness of a shield bash made with it, but the shield can be made into a magic weapon in its own right.
\parhead{Shield, Tower} This massive wooden shield is nearly as tall as you are. In most situations, it provides the indicated shield defense bonus. However, when you take the total defense action with a tower shield, you can treat the shield as a wall along one edge of your square, providing you with total cover. The shield does not, however, provide cover against targeted spells; a spellcaster can cast a spell on you by targeting the shield you are holding. You cannot bash with a tower shield, nor can you use your shield hand for anything else.
\par When employing a tower shield in combat, you take a \minus2 penalty on attack rolls because of the shield's encumbrance.
\parhead{Shield Spikes} When added to your shield, these spikes turn it into a piercing weapon that increases the damage dealt by a shield bash as if the shield were designed for a creature one size category larger than you. You can't put spikes on a buckler or a tower shield. Otherwise, attacking with a spiked shield is like making a shield bash attack (see above).
\par An enhancement bonus on a spiked shield does not improve the effectiveness of a shield bash made with it, but a spiked shield can be made into a magic weapon in its own right.
\parhead{Splint Mail} The suit includes gauntlets.
\parhead{Studded Leather} Only the studs on studded leather are made of metal. If the studs are made from a special material, only the studs are affected, not the whole armor. For example, ironwood studded leather weighs 17.5 pounds, since the weight of the studs is halved.

\subsection{Armor for Unusual Creatures}\label{Armor for Unusual Creatures}
Armor and shields for unusually big creatures, unusually little creatures, and nonhumanoid creatures have different costs and weights from those given on \trefnp{Armor and Shields}. Refer to the appropriate line on the table below and apply the multipliers to cost and weight for the armor type in question.
\begin{dtable}
    \lcaption{Armor for Unusual Creatures}
\begin{tabularx}{\columnwidth}{>{\lcol}X c c c c}
  & \multicolumn{2}{c}{\thead{Humanoid}} & \multicolumn{2}{c}{\thead{Nonhumanoid}} \\
\thead{Size} & \thead{Cost} & \thead{Weight} & \thead{Cost} & \thead{Weight} \\
Tiny or smaller\fn{1} & \mult1/2 & \mult1/10 & \mult1 & \mult1/10 \\
Small & \mult1 & \mult1/2 & \mult2 & \mult1/2 \\
Medium & \mult1 & \mult1 & \mult2 & \mult1 \\
Large & \mult2 & \mult2 & \mult4 & \mult2 \\
Huge & \mult4 & \mult4 & \mult8 & \mult4 \\
Gargantuan & \mult8 & \mult8 & \mult16 & \mult8 \\
Colossal & \mult16 & \mult12 & \mult32 & \mult12 \\
\end{tabularx}
1 Divide armor bonus by 2.
\end{dtable}

\subsection{Special Materials}
Armor and shields can be made from special materials, which can alter the properties of the item. The most common special material is ironwood, which is made from wood magically treated using the \spell{ironwood} ritual to be as strong as steel. Ironwood armor functions exactly like  armor of the same type, except that it is made of wood. This causes it to weigh half the normal weight, and allows druids to wear it without penalty. However, it costs twice as much as armor of the same type would normally cost, to a minimum of an additional 200 gp.

\section{Goods And Services}

\begin{dtable!*}
\lcaption{Goods and Services}
\begin{tabularx}{\textwidth}{>{\lcol}X c c >{\lcol}X c c >{\lcol}X c c}
\thead{Item} & \thead{Cost} & \thead{Weight} & \thead{Item} & \thead{Cost} & \thead{Weight} & \thead{Item} & \thead{Cost} & \thead{Weight} \\
\multicolumn{3}{>{\columncolor{tbrown}}c}{\thead{Adventuring Gear}} & \tind Average & 40 gp & 1 lb. & \multicolumn{3}{>{\columncolor{tbrown}}c}{\thead{Special Substances and Items}} \\
Backpack (empty) & 2 gp & 2 lb.\fn{1} & \tind Good & 80 gp & 1 lb. & Acid (flask) & 5 gp & 1 lb. \\
Barrel (empty) & 2 gp & 30 lb. & \tind Amazing & 150 gp & 1 lb. & Alchemist's fire (flask) & 5 gp & 1 lb. \\
Basket (empty) & 4 sp & 1 lb. & Manacles & 15 gp & 2 lb. & Antitoxin (vial) & 10 gp & \x \\
Bedroll & 1 sp & 5 lb.\fn{1} & Manacles, masterwork & 50 gp & 2 lb. & Everburning torch & 100 gp & 1 lb. \\
Bell & 1 gp & \x & Mirror, small steel & 10 gp & 1/2 lb. & Holy water (flask) & 10 gp & 1 lb. \\
Blanket, winter & 5 sp & 3 lb.\fn{1} & Mug/Tankard, clay & 2 cp & 1 lb. & Smokestick & 5 gp & 1/2 lb. \\
Block and tackle & 5 gp & 5 lb. & Oil (1-pint flask) & 1 sp & 1 lb. & Sunrod & 2 gp & 1 lb. \\
Bottle, wine, glass & 2 gp & \x & Paper (sheet) & 4 sp & \x & Tanglefoot bag & 15 gp & 4 lb. \\
Bucket (empty) & 5 sp & 2 lb. & Parchment (sheet) & 2 sp & \x & Thunderstone & 10 gp & 1 lb. \\
Caltrops & 1 gp & 2 lb. & Pick, miner's & 3 gp & 10 lb. & Tindertwig & 5 sp & \x \\
Candle & 1 cp & \x & Pitcher, clay & 2 cp & 5 lb. & \multicolumn{3}{c}{\thead{Tools and Skill Kits}}  \\
Canvas (sq. yd.) & 1 sp & 1 lb. & Piton & 1 sp & 1/2 lb. & Item & Cost & Weight \\
Case, map or scroll & 1 gp & 1/2 lb. & Pole, 10 foot & 5 cp & 8 lb. & Alchemist's lab & 500 gp & 40 lb. \\
Chain (10 ft.) & 30 gp & 2 lb. & Pot, iron & 5 sp & 10 lb. & Artisan's tools & 5 gp & 5 lb. \\
Chalk, 1 piece & 1 cp & \x & Pouch, belt (empty) & 1 gp & 1/2 lb.\fn{1} & Artisan's tools, masterwork & 55 gp & 5 lb. \\
Chest (empty) & 2 gp & 25 lb. & Ram, portable & 10 gp & 20 lb. & Climber's kit & 80 gp & 5 lb.\fn{1} \\
Crowbar & 2 gp & 5 lb. & Rations, trail (per day) & 5 sp & 1 lb.\fn{1} & Disguise kit & 50 gp & 8 lb.\fn{1} \\
Firewood (per day) & 1 cp & 20 lb. & Rope, hempen (50 ft.) & 1 gp & 10 lb. & Healer's kit & 50 gp & 1 lb. \\
Fishhook & 1 sp & \x & Rope, silk (50 ft.) & 10 gp & 5 lb. & Holly and mistletoe & \x & \x \\
Fishing net, 25 sq. ft. & 4 gp & 5 lb. & Sack (empty) & 1 sp & 1/2 lb.\fn{1} & Holy symbol, wooden & 1 gp & \x \\
Flask (empty) & 3 cp & 1-1/2 lb. & Sealing wax & 1 gp & 1 lb. & Holy symbol, silver & 25 gp & 1 lb. \\
Flint and steel & 1 gp & \x & Sewing needle & 5 sp & \x & Hourglass & 25 gp & 1 lb. \\
Grappling hook & 1 gp & 4 lb. & Signal whistle & 8 sp & \x & Magnifying glass & 100 gp & \x \\
Hammer & 5 sp & 2 lb. & Signet ring & 5 gp & \x & Musical instrument, common & 5 gp & 3 lb.\fn{1} \\
Ink (1 oz. vial) & 8 gp & \x & Sledge & 1 gp & 10 lb. & Musical instrument, masterwork & 100 gp & 3 lb.\fn{1} \\
Inkpen & 1 sp & \x & Soap (per lb.) & 5 sp & 1 lb. & Scale, merchant's & 2 gp & 1 lb. \\
Jug, clay & 3 cp & 9 lb. & Spade or shovel & 2 gp & 8 lb. & Spell component pouch & 5 gp & 2 lb. \\
Ladder, 10 foot & 2 sp & 20 lb. & Spyglass & 1,000 gp & 1 lb. & Spellbook, wizard's (blank) & 15 gp & 3 lb. \\
Lamp, common & 1 sp & 1 lb. & Tent & 10 gp & 20 lb.\fn{1} & Thieves' tools & 30 gp & 1 lb. \\
Lantern, bullseye & 12 gp & 3 lb. & Torch & 1 cp & 1 lb. & Thieves' tools, masterwork & 100 gp & 2 lb. \\
Lantern, hooded & 7 gp & 2 lb. & Vial, ink or potion & 1 gp & 1/10 lb. & Tool, masterwork & 50 gp & 1 lb. \\
Lock &   & 1 lb. & Waterskin & 1 gp & 4 lb.\fn{1} & Water clock & 1,000 gp & 200 lb. \\
\tind Very simple & 20 gp & 1 lb. & Whetstone & 2 cp & 1 lb. &  &  &  \\
\end{tabularx}
1 These items weigh one-quarter this amount when made for Small characters. Containers for Small characters also carry one-quarter the normal amount. \\
\x No weight, or no weight worth noting.	
\end{dtable!*}

\begin{dtable!*}
\begin{tabularx}{\textwidth}{>{\lcol}X c c >{\lcol}X c c >{\lcol}X c c}
\thead{Clothing} &  &  & \thead{Mounts and Related Gear} &  &  & \thead{Transport} &  & \\
\thead{Item} & \thead{Cost} & \thead{Weight} & \thead{Item} & \thead{Cost} & \thead{Weight} & \thead{Item} & \thead{Cost} & \thead{Weight} \\
Artisan's outfit & 1 gp & 4 lb. & Barding &  &  & Carriage & 100 gp & 600 lb. \\
Cleric's vestments & 5 gp & 6 lb. & \tind Medium creature & \mult2\fn{2} & \mult1\fn{2}  & Cart & 15 gp & 200 lb. \\
Cold weather outfit & 8 gp & 7 lb. & \tind Large creature & \mult4\fn{2} & \mult2\fn{2}  & Galley & 30,000 gp & \x \\
Courtier's outfit & 30 gp & 6 lb. & Bit and bridle & 2 gp & 1 lb. & Keelboat & 3,000 gp & \x \\
Entertainer's outfit & 3 gp & 4 lb. & Dog, guard & 25 gp & \x & Longship & 10,000 gp & \x \\
Explorer's outfit & 10 gp & 8 lb. & Dog, riding & 150 gp & \x & Rowboat & 50 gp & 100 lb. \\
Monk's outfit & 5 gp & 2 lb. & Donkey or mule & 8 gp & \x & Oar & 2 gp & 10 lb. \\
Noble's outfit & 75 gp & 10 lb. & Feed (per day) & 5 cp & 10 lb. & Sailing ship & 10,000 gp & \x \\
Peasant's outfit & 1 sp & 2 lb. & Horse &  &  & Sled & 20 gp & 300 lb. \\
Royal outfit & 200 gp & 15 lb. & \tind Horse, heavy & 200 gp & \x & Wagon & 35 gp & 400 lb. \\
Scholar's outfit & 5 gp & 6 lb. & \tind Horse, light & 75 gp & \x & Warship & 25,000 gp & \x \\
Traveler's outfit & 1 gp & 5 lb. & \tind Pony & 30 gp & \x & \thead{Spellcasting and Services} &  &  \\
\thead{Food, Drink, and Lodging} &  &  & \tind Warhorse, heavy & 400 gp & \x & Service & \multicolumn{2}{l}{Cost} \\
Item & Cost & Weight & \tind Warhorse, light & 150 gp & \x & Coach cab & \multicolumn{2}{l}{3 cp per mile} \\
Ale &  &  & \tind Warpony & 100 gp & \x & Hireling, trained & \multicolumn{2}{l}{3 sp per day} \\
\tind Gallon & 2 sp & 8 lb. & Saddle &  &  & Hireling, untrained & \multicolumn{2}{l}{1 sp per day} \\
\tind Mug & 4 cp & 1 lb. & \tind Military & 20 gp & 30 lb. & Messenger & \multicolumn{2}{l}{2 cp per mile} \\
Banquet (per person) & 10 gp & \x & \tind Pack & 5 gp & 15 lb. & Road or gate toll & \multicolumn{2}{l}{1 cp} \\
Bread, per loaf & 2 cp & 1/2 lb. & \tind Riding & 10 gp & 25 lb. & Ship's passage & \multicolumn{2}{l}{1 sp per mile} \\
Cheese, hunk of & 1 sp & 1/2 lb. & \thead{Saddle, Exotic} &  &  &  & & \\
Inn stay (per day) &  &  & \tind Military & 60 gp & 40 lb. & Spell, 1st-level & \multicolumn{2}{l}{Caster level x 10 gp\fn{1}} \\
\tind Good & 2 gp & \x & \tind Pack & 15 gp & 20 lb. & Spell, 2nd-level & \multicolumn{2}{l}{Caster level x 20 gp\fn{1}} \\
\tind Common & 5 sp & \x & \tind Riding & 30 gp & 30 lb. & Spell, 3rd-level & \multicolumn{2}{l}{Caster level x 30 gp\fn{1}} \\
\tind Poor & 2 sp & \x & Saddlebags & 4 gp & 8 lb. & Spell, 4th-level & \multicolumn{2}{l}{Caster level x 40 gp\fn{1}} \\
Meals (per day) &  &  & Stabling (per day) & 5 sp & \x & Spell, 5th-level & \multicolumn{2}{l}{Caster level x 50 gp\fn{1}} \\
\tind Good & 5 sp & \x & &  &  & Spell, 6th-level & \multicolumn{2}{l}{Caster level x 60 gp\fn{1}} \\
\tind Common & 3 sp & \x & &  &  & Spell, 7th-level & \multicolumn{2}{l}{Caster level x 70 gp\fn{1}} \\
\tind Poor & 1 sp & \x & &  &  & Spell, 8th-level & \multicolumn{2}{l}{Caster level x 80 gp\fn{1}} \\
Meat, chunk of & 3 sp & 1/2 lb. & &  &  & Spell, 9th-level & \multicolumn{2}{l}{Caster level x 90 gp\fn{1}} \\
Wine &  &  & &  &  &  &  & \\
\tind Common (pitcher) & 2 sp & 6 lb. & &  &  &  &  & \\
\tind Fine (bottle) & 10 gp & 1-1/2 lb. & & & & & & \\
\end{tabularx}
1 See spell description for additional costs. If the additional costs put the spell's total cost above 2,000 gp, that spell is not generally available. \\
2 Relative to normal armor of the same type
\end{dtable!*}

\subsection{Adventuring Gear}
A few of the pieces of adventuring gear found on Table: Goods and Services are described below, along with any special benefits they confer on the user ("you").
\parhead{Caltrops} A caltrop is a four-pronged iron spike crafted so that one prong faces up no matter how the caltrop comes to rest. You scatter caltrops on the ground in the hope that your enemies step on them or are at least forced to slow down to avoid them. One 2- pound bag of caltrops covers an area 5 feet square.
\par Each time a creature moves into an area covered by caltrops (or spends a round fighting while standing in such an area), it might step on one. The caltrops make an attack roll (base attack bonus \plus0) against the creature. For this attack, the creature's shield, armor, and deflection bonuses do not count. If the creature is wearing shoes or other footwear, it gets a \plus2 armor defense bonus. If the caltrops succeed on the attack, the creature has stepped on one. The caltrop deals 1 point of damage, and the creature's speed is reduced by one-half because its foot is wounded. This movement penalty lasts for 24 hours, or until the creature is successfully treated with a DC 15 Heal check, or until it receives at least 1 point of magical curing. A charging or running creature must immediately stop if it steps on a caltrop. Any creature moving at half speed or slower can pick its way through a bed of caltrops with no trouble.
\par Caltrops may not be effective against unusual opponents.
\parhead{Candle} A candle dimly illuminates a 5 foot radius and burns for 1 hour.
\parhead{Chain} Chain has hardness 10 and 5 hit points. It can be burst with a DC 26 Strength check.
\parhead{Crowbar} A crowbar it grants a \plus2 bonus on Strength checks made for such purposes. If used in combat, treat a crowbar as a improvised weapon equivalent to a club.
\parhead{Flint and Steel} Lighting a torch with flint and steel is a full-round action, and lighting any other fire with them takes at least that long.
\parhead{Grappling Hook} Throwing a grappling hook successfully requires a ranged attack (DC 10, \plus2 per 10 feet of distance thrown).
\parhead{Hammer} If a hammer is used in combat, treat it as a medium improvised weapon that deals bludgeoning damage equal to that of a spiked gauntlet of its size.
\parhead{Ink} This is black ink. You can buy ink in other colors, but it costs twice as much.
\parhead{Jug, Clay} This basic ceramic jug is fitted with a stopper and holds 1 gallon of liquid.
\parhead{Lamp, Common} A lamp clearly illuminates a 15 foot radius, provides shadowy illumination out to a 30 foot radius, and burns for 6 hours on a pint of oil. You can carry a lamp in one hand.
\parhead{Lantern, Bullseye} A bullseye lantern provides clear illumination in a 60 foot cone and shadowy illumination in a 120 foot cone. It burns for 6 hours on a pint of oil. You can carry a bullseye lantern in one hand.
\parhead{Lantern, Hooded} A hooded lantern clearly illuminates a 30 foot radius and provides shadowy illumination in a 60 foot radius. It burns for 6 hours on a pint of oil. You can carry a hooded lantern in one hand.
\parhead{Lock} The DC to open a lock with the Open Lock skill depends on the lock's quality: simple (DC 20), average (DC 25), good (DC 30), or superior (DC 40).
\parhead{Manacles and Manacles, Masterwork} Manacles can bind a Medium creature. A manacled creature can use the Escape Artist skill to slip free (DC 30, or DC 35 for masterwork manacles). Breaking the manacles requires a Strength check (DC 26, or DC 28 for masterwork manacles). Manacles have hardness 10 and 10 hit points.
\par Most manacles have locks; add the cost of the lock you want to the cost of the manacles.
\par For the same cost, you can buy manacles for a Small creature.
\par For a Large creature, manacles cost ten times the indicated amount, and for a Huge creature, one hundred times this amount. Gargantuan, Colossal, Tiny, Diminutive, and Fine creatures can be held only by specially made manacles.
\parhead{Oil} A pint of oil burns for 6 hours in a lantern. You can use a flask of oil as a splash weapon. Use the rules for alchemist's fire, except that it takes a full round action to prepare a flask with a fuse. Once it is thrown, there is a 50\% chance of the flask igniting successfully.
\par You can pour a pint of oil on the ground to cover an area 5 feet square, provided that the surface is smooth. If lit, the oil burns for 2 rounds and deals 1d3 points of fire damage to each creature in the area.
\parhead{Ram, Portable} This iron-shod wooden beam gives you a \plus2 bonus on Strength checks made to break open a door and it allows a second person to help you without having to roll, increasing your bonus by 2.
\parhead{Rope, Hempen} This rope has 2 hit points and can be burst with a DC 23 Strength check.
\parhead{Rope, Silk} This rope has 4 hit points and can be burst with a DC 24 Strength check.
\parhead{Spyglass} Objects viewed through a spyglass are magnified to twice their size.
\parhead{Torch} A torch burns for 1 hour, clearly illuminating a 20 foot radius and providing shadowy illumination out to a 40 foot radius. If a torch is used in combat, treat it as a light improvised weapon that deals bludgeoning damage equal to that of a gauntlet of its size, plus 1 point of fire damage. Wielding a torch gives a \plus2 bonus to dirty trick attempts made to dazzle an opponent. Lighting a torch with an open flame or tindertwig is a standard action, while using flint and steel requires a full-round action.
\parhead{Vial} A vial holds 1 ounce of liquid. The stoppered container usually is no more than 1 inch wide and 3 inches high.

\subsection{Special Substances And Items}
Any of these substances except for the everburning torch and holy water can be made by a character with the Craft (alchemy) skill.
\parhead{Acid} You can throw a flask of acid as a splash weapon. Treat this attack as a ranged touch attack with a range increment of 10 feet. A direct hit deals 1d6 points of acid damage. Every creature within 5 feet of the point where the acid hits takes 1 point of acid damage from the splash.
\parhead{Alchemist's Fire} You can throw a flask of alchemist's fire as a splash weapon. Treat this attack as a ranged touch attack with a range increment of 10 feet.
\par A direct hit deals 1d6 points of fire damage and ignites the creature for 1 round. Every creature within 5 feet of the point where the flask hits takes 1 point of fire damage from the splash. An ignited creature is vulnerable, causing it to take a \minus2 penalty to attacks, defenses, and checks. In addition, at the end of its turn, it takes d6 damage from the fire. If the creature takes a move action, it can attempt a DC 10 Dexterity check to put out the flames. This action requires a free hand. Dropping prone as part of the action gives a \plus5 bonus on this check.
\parhead{Antitoxin} If you drink antitoxin, you get a \plus5 bonus to Fortitude defenses against poison for 1 hour.
\parhead{Everburning Torch} This otherwise normal torch has a continual flame spell cast upon it. An everburning torch clearly illuminates a 20 foot radius and provides shadowy illumination out to a 40 foot radius.
\parhead{Holy Water} Holy water damages undead creatures and evil outsiders almost as if it were acid. A flask of holy water can be thrown as a splash weapon.
\par Treat this attack as a ranged touch attack with a range increment of 10 feet. A flask breaks if thrown against the body of a corporeal creature, but to use it against an incorporeal creature, you must open the flask and pour the holy water out onto the target. Thus, you can douse an incorporeal creature with holy water only if you are adjacent to it. Doing so is a ranged touch attack that does not provoke attacks of opportunity.
\par A direct hit by a flask of holy water deals 2d4 points of damage to an undead creature or an evil outsider. Each such creature within 5 feet of the point where the flask hits takes 1 point of damage from the splash.
\par Temples to good deities sell holy water at cost (making no profit).
\parhead{Smokestick} This alchemically treated wooden stick instantly creates thick, opaque smoke when ignited. The smoke fills a 10 foot cube (treat the effect as a fog cloud spell, except that a moderate or stronger wind dissipates the smoke in 1 round). The stick is consumed after 1 round, and the smoke dissipates naturally.
\parhead{Sunrod} This 1 foot long, gold-tipped, iron rod glows brightly when struck. It clearly illuminates a 30 foot radius and provides shadowy illumination in a 60 foot radius. It glows for 6 hours, after which the gold tip is burned out and worthless.
\parhead{Tanglefoot Bag} When you throw a tanglefoot bag at a creature (as a ranged touch attack with a range increment of 10 feet), the bag comes apart and the goo bursts out, entangling the target and then becoming tough and resilient upon exposure to air. An entangled creature moves at half speed, cannot run or charge, and takes a \minus2 penalty to physical attacks and defenses, as well as Strength and Dexterity-based checks. An entangled creature who attempts to cast a spell must make a Concentration check (DC 10 \add double the spell's level) or lose the spell. A tanglefoot bag does not function underwater.
\par A creature that is glued to the floor (or unable to fly) can break free by making a DC 17 Strength check or by dealing 15 points of damage to the goo with a slashing weapon. A creature trying to scrape goo off itself, or another creature assisting, does not need to make an attack roll; hitting the goo is automatic, after which the creature that hit makes a damage roll to see how much of the goo was scraped off. Once free, the creature can move (including flying) at half speed. A character capable of spellcasting who is bound by the goo must make a DC 15 Concentration check to cast a spell. The goo becomes brittle and fragile after 2d4 rounds, cracking apart and losing its effectiveness. An application of universal solvent to a stuck creature dissolves the alchemical goo immediately.
\parhead{Thunderstone} You can throw this stone as a ranged attack with a range increment of 20 feet. When it strikes a hard surface (or is struck hard), it creates a deafening bang that is treated as a sonic attack. You make a Fortitude attack against each creature within a \areasmall radius spread to deafen them for 5 minutes. Your special attack bonus on this attack is \plus5. A deafened creature automatically fails Listen checks, takes a \minus2 penalty to any checks which involve hearing, and has a 20\% chance of spell failure when casting spells with verbal components.
\par Since you don't need to hit a specific target, you can simply aim at a particular 5 foot square. Treat the target square as if its physical defense was 5.
\parhead{Tindertwig} The alchemical substance on the end of this small, wooden stick ignites when struck against a rough surface. Creating a flame with a tindertwig is much faster than creating a flame with flint and steel (or a magnifying glass) and tinder. Lighting a torch with a tindertwig is a standard action (rather than a full-round action), and lighting any other fire with one is at least a standard action.

\subsection{Tools and Skill Kits}
\parhead{Alchemist's Lab} An alchemist's lab always has the perfect tool for making alchemical items, so it provides a \plus2 bonus on Craft (alchemy) checks. It has no bearing on the costs related to the Craft (alchemy) skill. Without this lab, a character with the Craft (alchemy) skill is assumed to have enough tools to use the skill but not enough to get the \plus2 bonus that the lab provides.
\parhead{Artisan's Tools} These special tools include the items needed to pursue any craft. Without them, you have to use improvised tools (\minus2 penalty on Craft checks), if you can do the job at all (see \pcref{Craft}).
\parhead{Artisan's Tools, Masterwork} These tools serve the same purpose as artisan's tools (above), but masterwork artisan's tools are the perfect tools for the job, so you get a \plus2 bonus on Craft checks made with them.
\parhead{Climber's Kit} This is the perfect tool for climbing and gives you a \plus2 bonus on Climb checks.
\parhead{Disguise Kit} The kit is the perfect tool for disguise and provides a \plus2 bonus on Disguise checks. A disguise kit is exhausted after ten uses.
\parhead{Healer's Kit} It is the perfect tool for healing and provides a \plus2 bonus on Heal checks. A healer's kit is exhausted after ten uses.
\parhead{Holy Symbol, Silver or Wooden} A holy symbol focuses positive energy. A cleric or paladin uses it as the focus for his spells and as a tool for turning undead. Each religion has its own holy symbol.
\parhead{Unholy Symbols} An unholy symbol is like a holy symbol except that it focuses negative energy and is used by evil clerics (or by neutral clerics who want to cast evil spells or command undead).
\parhead{Magnifying Glass} This simple lens allows a closer look at small objects. It is also useful as a substitute for flint and steel when starting fires. Lighting a fire with a magnifying glass requires light as bright as sunlight to focus, tinder to ignite, and at least a full-round action. A magnifying glass grants a \plus2 bonus on Appraise checks
involving any item that is small or highly detailed.
\parhead{Musical Instrument, Common or Masterwork} A masterwork instrument grants a \plus2 bonus on Perform checks involving its use.
\parhead{Pitons} You can make your own handholds and footholds by pounding pitons into a wall. Doing so takes 1 minute per piton, and one piton is needed per 3 feet of distance. As with any surface that offers handholds and footholds, a wall with pitons in it has a DC of 15.
\parhead{Spell Component Pouch} A spellcaster with a spell component pouch is assumed to have all the material components and focuses needed for spellcasting, except for those components that have a specific cost, divine focuses, and focuses that wouldn't fit in a pouch.
\parhead{Spellbook, Wizard's (Blank)} A spellbook has 100 pages of parchment, and each spell takes up one page per spell level (one page each for 0-level spells).
\parhead{Thieves' Tools} This kit contains the tools you need to use the Disable Device and Open Lock skills. Without these tools, you must improvise tools, and you take a \minus2 circumstance penalty on Disable Device and Open Locks checks.
\parhead{Thieves' Tools, Masterwork} This kit contains extra tools and tools of better make, which grant a \plus2 bonus on Disable Device and Open Lock checks.
\parhead{Tool, Masterwork} This well-made item is the perfect tool for the job. It grants a \plus2 bonus on a related skill check (if any). Bonuses provided by multiple masterwork items used toward the same skill check do not stack.
\parhead{Water Clock} This large, bulky contrivance gives the time accurate to within half an hour per day since it was last set. It requires a source of water, and it must be kept still because it marks time by the regulated flow of droplets of water.

\subsection{Clothing}
\parhead{Artisan's Outfit} This outfit includes a shirt with buttons, a skirt or pants with a drawstring, shoes, and perhaps a cap or hat. It may also include a belt or a leather or cloth apron for carrying tools.
\parhead{Cleric's Vestments} These ecclesiastical clothes are for performing priestly functions, not for adventuring.
\parhead{Cold Weather Outfit} A cold weather outfit includes a wool coat, linen shirt, wool cap, heavy cloak, thick pants or skirt, and
boots. This outfit grants a \plus5 bonus to Fortitude defenses against exposure to cold weather.
\parhead{Courtier's Outfit} This outfit includes fancy, tailored clothes in whatever fashion happens to be the current style in the courts of the nobles. Anyone trying to influence nobles or courtiers while wearing street dress will have a hard time of it (\minus2 penalty on Charisma-based skill checks to influence such individuals). If you wear this outfit without jewelry (costing an additional 50 gp), you look like an out-of-place commoner.
\parhead{Entertainer's Outfit} This set of flashy, perhaps even gaudy, clothes is for entertaining. While the outfit looks whimsical, its practical design lets you tumble, dance, walk a tightrope, or just run (if the audience turns ugly).
\parhead{Explorer's Outfit} This is a full set of clothes for someone who never knows what to expect. It includes sturdy boots, leather breeches or a skirt, a belt, a shirt (perhaps with a vest or jacket), gloves, and a cloak. Rather than a leather skirt, a leather overtunic may be worn over a cloth skirt. The clothes have plenty of pockets (especially the cloak). The outfit also includes any extra items you might need, such as a scarf or a wide-brimmed hat.
\parhead{Monk's Outfit} This simple outfit includes sandals, loose breeches, and a loose shirt, and is all bound together with sashes. The outfit is designed to give you maximum mobility, and it's made of high-quality fabric. You can hide small weapons in pockets hidden in the folds, and the sashes are strong enough to serve as short ropes.
\parhead{Noble's Outfit} This set of clothes is designed specifically to be expensive and to show it. Precious metals and gems are worked into the clothing. To fit into the noble crowd, every would-be noble also needs a signet ring (see Adventuring Gear, above) and jewelry (worth at least 100 gp).
\parhead{Peasant's Outfit} This set of clothes consists of a loose shirt and baggy breeches, or a loose shirt and skirt or overdress. Cloth wrappings are used for shoes.
\parhead{Royal Outfit} This is just the clothing, not the royal scepter, crown, ring, and other accoutrements. Royal clothes are ostentatious, with gems, gold, silk, and fur in abundance.
\parhead{Scholar's Outfit} Perfect for a scholar, this outfit includes a robe, a belt, a cap, soft shoes, and possibly a cloak.
\parhead{Traveler's Outfit} This set of clothes consists of boots, a wool skirt or breeches, a sturdy belt, a shirt (perhaps with a vest or jacket), and an ample cloak with a hood.

\subsection{Food, Drink, and Lodging}
\parhead{Inn} Poor accommodations at an inn amount to a place on the floor near the hearth. Common accommodations consist of a place on a raised, heated floor, the use of a blanket and a pillow. Good accommodations consist of a small, private room with one bed, some amenities, and a covered chamber pot in the corner.
\parhead{Meals} Poor meals might be composed of bread, baked turnips, onions, and water. Common meals might consist of bread, chicken stew, carrots, and watered-down ale or wine. Good meals might be composed of bread and pastries, beef, peas, and ale or wine.

\subsection{Mounts and Related Gear}
\parhead{Barding, Medium Creature and Large Creature} Barding is a type of armor that covers the head, neck, chest, body, and possibly legs of a horse or other mount. Barding made of medium or heavy armor provides better protection than light barding, but at the expense of speed. Barding can be made of any of the armor types found on Table: Armor and Shields.
\par Armor for a horse (a Large nonhumanoid creature) costs four times as much as armor for a human (a Medium humanoid creature) and also weighs twice as much as the armor found on Table: Armor and Shields (see \pref{Armor for Unusual Creatures}). If the barding is for a pony or other Medium mount, the cost is only double, and the weight is the same as for Medium armor worn by a humanoid. Medium or heavy barding slows a mount that wears it, as shown on the table below.

\begin{dtable}
\begin{tabularx}{\columnwidth}{>{\lcol}X *{3}{>{\ccol}X}}
 & \multicolumn{3}{c}{\x\x\x Base Speed\x\x\x} \\
Barding & (40 ft.) & (50 ft.) & (60 ft.) \\
Medium & 30 ft. & 35 ft. & 40 ft. \\
Heavy & 30 ft.\fn{1} & 35 ft.\fn{1} & 40 ft.\fn{1} \\
\end{tabularx}
1 A mount wearing heavy armor moves at only triple its normal speed when running instead of quadruple.	
\end{dtable}

\par Flying mounts can't fly in medium or heavy barding.
\par Removing and fitting barding takes five times as long as the figures given on Table: Donning Armor. A barded animal cannot be used to carry any load other than the rider and normal saddlebags.
\parhead{Dog, Riding} This Medium dog is specially trained to carry a Small humanoid rider. It is brave in combat like a warhorse. You take no damage when you fall from a riding dog.
\parhead{Donkey or Mule} Donkeys and mules are stolid in the face of danger, hardy, surefooted, and capable of carrying heavy loads over vast distances. Unlike a horse, a donkey or a mule is willing (though not eager) to enter dungeons and other strange or threatening places.
\parhead{Feed} Horses, donkeys, mules, and ponies can graze to sustain themselves, but providing feed for them is much better. If you have a riding dog, you have to feed it at least some meat.
\parhead{Horse} A horse (other than a pony) is suitable as a mount for a human, dwarf, elf, half-elf, or half-orc. A pony is smaller than a horse and is a suitable mount for a gnome or halfling.
\par Warhorses and warponies can be ridden easily into combat. Light horses, ponies, and heavy horses are hard to control in combat.
\parhead{Saddle, Exotic} An exotic saddle is like a normal saddle of the same sort except that it is designed for an unusual mount. Exotic saddles come in military, pack, and riding styles.
\parhead{Saddle, Military} A military saddle braces the rider, providing a \plus2 bonus on Ride checks related to staying in the saddle. If you're knocked unconscious while in a military saddle, you have a 75\% chance to stay in the saddle (compared to 50\% for a riding saddle).
\parhead{Saddle, Pack} A pack saddle holds gear and supplies, but not a rider. It holds as much gear as the mount can carry.
\parhead{Saddle, Riding} The standard riding saddle supports a rider.

\subsection{Transport}
\parhead{Carriage} This four-wheeled vehicle can transport as many as four people within an enclosed cab, plus two drivers. In general, two horses (or other beasts of burden) draw it. A carriage comes with the harness needed to pull it.
\parhead{Cart} This two-wheeled vehicle can be drawn by a single horse (or other beast of burden). It comes with a harness.
\parhead{Galley} This three-masted ship has seventy oars on either side and requires a total crew of 200. A galley is 130 feet long and 20 feet wide, and it can carry 150 tons of cargo or 250 soldiers. For 8,000 gp more, it can be fitted with a ram and castles with firing platforms fore, aft, and amidships. This ship cannot make sea voyages and sticks to the coast. It moves about 4 miles per hour when being rowed or under sail.
\parhead{Keelboat} This 50- to 75 foot long ship is 15 to 20 feet wide and has a few oars to supplement its single mast with a square sail. It has a crew of eight to fifteen and can carry 40 to 50 tons of cargo or 100 soldiers. It can make sea voyages, as well as sail down rivers (thanks to its flat bottom). It moves about 1 mile per hour.
\parhead{Longship} This 75 foot long ship with forty oars requires a total crew of 50. It has a single mast and a square sail, and it can carry 50 tons of cargo or 120 soldiers. A longship can make sea voyages. It moves about 3 miles per hour when being rowed or under sail.
\parhead{Rowboat} This 8- to 12 foot long boat holds two or three Medium passengers. It moves about 1-1/2 miles per hour.
\parhead{Sailing Ship} This larger, seaworthy ship is 75 to 90 feet long and 20 feet wide and has a crew of 20. It can carry 150 tons of cargo. It has square sails on its two masts and can make sea voyages. It moves about 2 miles per hour.
\parhead{Sled} This is a wagon on runners for moving through snow and over ice. In general, two horses (or other beasts of burden) draw it. A sled comes with the harness needed to pull it.
\parhead{Wagon} This is a four-wheeled, open vehicle for transporting heavy loads. In general, two horses (or other beasts of burden) draw it. A wagon comes with the harness needed to pull it.
\parhead{Warship} This 100 foot long ship has a single mast, although oars can also propel it. It has a crew of 60 to 80 rowers. This ship can carry 160 soldiers, but not for long distances, since there isn't room for supplies to support that many people. The warship cannot make sea voyages and sticks to the coast. It is not used for cargo. It moves about 2-1/2 miles per hour when being rowed or under sail.

\subsection{Spellcasting And Services}
Sometimes the best solution for a problem is to hire someone else to take care of it.
\parhead{Coach Cab} The price given is for a ride in a coach that transports people (and light cargo) between towns. For a ride in a cab that transports passengers within a city, 1 copper piece usually takes you anywhere you need to go.
\parhead{Hireling, Trained} The amount given is the typical daily wage for mercenary warriors, masons, craftsmen, scribes, teamsters, and other trained hirelings. This value represents a minimum wage; many such hirelings require significantly higher pay.
\parhead{Hireling, Untrained} The amount shown is the typical daily wage for laborers, porters, cooks, maids, and other menial workers.
\parhead{Messenger} This entry includes horse-riding messengers and runners. Those willing to carry a message to a place they were going anyway may ask for only half the indicated amount.
\parhead{Road or Gate Toll} A toll is sometimes charged to cross a well-trodden, well-kept, and well-guarded road to pay for patrols on it and for its upkeep. Occasionally, a large walled city charges a toll to enter or exit (or sometimes just to enter).
\parhead{Ship's Passage} Most ships do not specialize in passengers, but many have the capability to take a few along when transporting cargo. Double the given cost for creatures larger than Medium or creatures that are otherwise difficult to bring aboard a ship.
\parhead{Spell} The indicated amount is how much it costs to get a spellcaster to cast a spell for you. This cost assumes that you can go to the spellcaster and have the spell cast at his or her convenience. If you want to bring the spellcaster somewhere to cast a spell you need to negotiate with him or her, and the default answer is no.
\par The cost given is for a spell with no cost for a material component or focus component. If the spell includes a material component, add the cost of that component to the cost of the spell.
\par If the spell has a focus component (other than a divine focus), add 1/10 the cost of that focus to the cost of the spell.
\par Furthermore, if a spell has dangerous consequences, the spellcaster will certainly require proof that you can and will pay for dealing with any such consequences (that is, assuming that the spellcaster even agrees to cast such a spell, which isn't certain). In the case of spells that transport the caster and characters over a distance, you will likely have to pay for two castings of the spell, even if you aren't returning with the caster.
\par In addition, not every town or village has a spellcaster of sufficient level to cast any spell. In general, you must travel to a small town (or larger settlement) to be reasonably assured of finding a spellcaster capable of casting 1st-level spells, a large town for 2nd-level spells, a small city for 3rd- or 4th-level spells, a large city for 5th- or 6th-level spells, and a metropolis for 7th- or 8th-level spells. Even a metropolis isn't guaranteed to have a local spellcaster able to cast 9th-level spells. Additionally, while you may find spellcasters able to cast some spells, there is no guarantee that they are able to cast the spell you desire. In general, the more common the spell, the more likely you are to find a spellcaster able to cast it.
