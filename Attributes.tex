\chapter{Attributes}
Each character has six attributes: Strength (Str), Dexterity (Dex), Constitution (Con), Intelligence (Int), Wisdom (Wis), and Charisma (Cha). These attributes represent a character's raw talent in that area. A 0 in an attribute represents average human capacity. That doesn't mean that every commoner has a 0 in every attribute; not everyone is average, after all.

\section{Attribute Descriptions}

\subsection{Strength (Str)}
Strength measures muscle and physical power.
\begin{itemize}
    \item Strength is added to physical attacks with medium and heavy weapons.
    \item Half Strength is added to physical damage with all weapons.
    \item Strength determines how much a character can carry (see \tref{Weight Limits}.
    \item Strength is added to Athletics, Climb, and Swim checks.
    \item Half Strength is added to the number of skill points a character gets in Strength-based skills.
    \item Half Strength is added to Fortitude defense.
\end{itemize}

\subsection{Dexterity (Dex)}
Dexterity measures hand-eye coordination, agility, and reflexes.
\begin{itemize}
    \item Dexterity is added to physical attacks with light and projectile weapons.
    \item Dexterity is added to all physicial defenses.
    \item Dexterity is added to Acrobatics, Escape Artist, Ride, Sleight of Hand, and Stealth checks.
    \item Half Dexterity is added to the number of skill points a character gets in Dexterity-based skills.
\end{itemize}

\subsection{Constitution (Con)}
Constitution represents your character's health and stamina.
\begin{itemize}
    \item Half Constitution is added to hit points at each level.
    \item Constitution is added to Fortitude defense.
\end{itemize}

\subsection{Intelligence (Int)}
Intelligence determines how well your character learns and reasons. It affects your character's knowledge in many areas.

\begin{itemize}
    \item Intelligence is added to the number of skill points a character gets in all skills.
    \item Intelligence is added to Craft, Disguise, Knowledge, and Linguistics checks.
    \item Half Intelligence is added to Will defense.
\end{itemize}

\par An animal has an Intelligence score of \minus5 or lower. A creature of humanlike intelligence has a score of at least a \minus4 Intelligence.

\subsection{Wisdom (Wis)}
Wisdom describes a character's common sense, perception, and intuition. While Intelligence represents one's ability to analyze information, Wisdom represents being in tune with and aware of one's surroundings.
\begin{itemize}
    \item Wisdom is added to Heal, Perception, Sense Motive, Spellcraft, and Survival checks.
    \item Half Wisdom is added to the number of skill points a character gets in Wisdom-based skills.
    \item Half Wisdom is added to Reflex defense.
\end{itemize}

\subsection{Charisma (Cha)}
Charisma measures a character's force of personality, willpower, and personal magnetism. It affects your character's actual strength of personality, not merely how your character is perceived by others in a social setting.
\begin{itemize}
    \item Charisma is added to Bluff, Creature Handling, Perform, and Persuasion checks.
    \item Half Charisma is added to the number of skill points a character gets in Charisma-based skills.
    \item Charisma is added to Will defense.
\end{itemize}

\section{Attributes and Spellcasters}
Using magic requires a strong mind. The attribute that governs spellcasting depends on what type of spellcaster your character is: Intelligence for wizards; Wisdom for druids; or Charisma for paladins, sorcerers, and clerics. That attribute is called your casting attribute. The power of many spells is affected by your casting attribute. In addition, you must have a minimum score in your casting attribute to cast spells. Divine casters (clerics, druids, and paladins) must have a casting attribute of at least half the spell's level. Arcane casters (sorcerers and wizards) must have a casting attribute at least equal to the spell's level. In addition to having a high attribute, a spellcaster must be of high enough class level to be able to cast spells of a given spell level. (See the class descriptions for details.)

\section{Changing Attributes}

Your attributes increase as you gain levels (see \pcref{Character Advancement}), and some special abilities can also increase your attributes. Usually, when you change an attribute, you also alter your skill points, hit points, and all other effects of the attribute appropriately. However, temporary bonuses and penalties to attributes, such as from spells and magic items, do not affect your skill points.

\section{Determining Attributes}
There are several options for how to determine attribute scores.

\subsubsection{Predefined Attribute Scores}
This is the simplest method. Simply take the following set of attribute scores and distribute them as you choose among your character's abilities:

4, 3, 2, 1, 0, \minus1

This set of attribute scores is called the ``elite array''. For more extreme characters, you may use the ``savant array'':

5, 2, 1, 0, 0, \minus2.

Finally, for more well-balanced characters, you may use the ``balanced array'':

3, 3, 2, 1, 1, 0

\subsection{Point Buy}
With this method, you can fully control your character's attribute scores to match what you want your character to be. All your character's attribute scores start at 0. You get 10 points to distribute among your character's attribute scores. Attribute scores can be bought according to the costs on \trefnp{Attribute Score Point Costs}.

\begin{dtable}
\lcaption{Attribute Score Point Costs}
\begin{tabularx}{\columnwidth}{X X X X}
\thead{Attribute Score} & \thead{Point Cost} & \thead{Attribute Score} & \thead{Point Cost} \\
\minus2 & \minus2\fn{1} & 2 & 2 \\
\minus1 & \minus1\fn{1} & 3 & 3 \\
0 & 0 & 4 & 5 \\
1 & 1 & 5 & 8 \\
\end{tabularx}
1 No more than two attribute scores can be reduced below 0 in this way.
\end{dtable}
