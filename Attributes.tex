\chapter{Attributes}
Each character has six attributes: Strength (Str), Dexterity (Dex), Constitution (Con), Intelligence (Int), Perception (Per), and Willpower (Wil). Each attribute represents a character's raw talent in that area. A 0 in an attribute represents average human capacity. That doesn't mean that every commoner has a 0 in every attribute; not everyone is average, after all.

\section{Attribute Descriptions}

\subsection{Strength (Str)}
Strength measures muscle and physical power.
\begin{itemize}
    \item Strength determines how much a character can carry (see \tref{Weight Limits}.
    \item Strength can be used to attack with melee and thrown weapons.
    \item Strength can be used for Maneuver defense.
    \item Strength can be used for Climb, Jump, Sprint, and Swim checks.
    %\item Half Strength is added to Fortitude defense.
\end{itemize}

\subsection{Dexterity (Dex)}
Dexterity measures hand-eye coordination, agility, and reflexes.
\begin{itemize}
    \item Dexterity can be used to attack with melee and thrown weapons that are light.
    \item Dexterity can be used for all physical defenses (Armor, Maneuver, and Reflex).
    \item Dexterity can be used for Balance, Escape Artist, Ride, Sleight of Hand, Stealth, and Tumble checks.
\end{itemize}

\subsection{Constitution (Con)}
Constitution represents your character's health and stamina.
\begin{itemize}
    \item Constitution can be used for Armor and Fortitude defense.
    \item Constitution can increase the hit points gained at each level.
\end{itemize}

\subsection{Intelligence (Int)}
Intelligence determines how well your character learns and reasons. It affects your character's knowledge in many areas.

\begin{itemize}
    \item Intelligence is added to Craft, Disguise, Heal, Knowledge, and Linguistics checks.
    \item Half Intelligence is added to the number of skill points a character gets.
    %\item Half Intelligence is added to Will defense.
\end{itemize}

\par An animal has an Intelligence score of \minus6 or lower. A creature of humanlike intelligence has a score of at least a \minus5 Intelligence.

\subsection{Perception (Per)}
Perception describes a character's ability to observe and be aware of one's surroundings.
\begin{itemize}
    \item Perception can be used to attack with projectile weapons.
    %\item Perception can be used for Reflex defense.
    \item Perception can be used for Awareness, Creature Handling, Sense Motive, Spellcraft, and Survival checks.
    %\item Half Wisdom is added to Reflex defense.
\end{itemize}

\subsection{Willpower (Wil)}
Willpower measures a character's ability to endure mental hardships.
\begin{itemize}
    \item Willpower can be used for Will defense.
    \item Willpower can be used to increase the hit points gained at each level.
    \item 
\end{itemize}

\section{Using Attributes}

\subsection{Choosing Attributes to Use}
In many cases, multiple attributes can be used for the same thing. For example, both Strength and Dexterity can be used to attack with light weapons such as daggers. Whenever more than one attribute could be used, you must choose which one to use (usually, the higher attribute).

\subsection{Attributes and Spellcasting}
Using magic requires a strong mind. In order to cast spells, you must have a high Intelligence, Perception, or Willpower, depending on your class. Intelligence is required for wizards, Perception is required for druids, and Willpower is required for clerics and sorcerers. For more details, see \pcref{Spellcasting}.

\section{Determining Attributes}
There are several options for how to determine attribute scores.

\subsubsection{Predefined Attribute Scores}
This is the simplest method. Simply take the following set of attribute scores and distribute them as you choose among your character's abilities:

4, 3, 2, 1, 0, \minus1

This set of attribute scores is called the ``elite array''. For more extreme characters, you may use the ``savant array'':

5, 2, 1, 0, 0, \minus2.

Finally, for more well-balanced characters, you may use the ``balanced array'':

3, 3, 2, 1, 1, 0

\subsection{Point Buy}
With this method, you can fully control your character's attribute scores to match what you want your character to be. All your character's attribute scores start at 0. You get 10 points to distribute among your character's attribute scores. Attribute scores can be bought according to the costs on \trefnp{Attribute Score Point Costs}.

\begin{dtable}
\lcaption{Attribute Score Point Costs}
\begin{tabularx}{\columnwidth}{X X X X}
\thead{Attribute Score} & \thead{Point Cost} \\
\hline
\minus2 & \minus2\fn{1} \\
\minus1 & \minus1\fn{1} \\
0 & 0 \\
1 & 1 \\
2 & 2 \\
3 & 3 \\
4 & 5 \\
5 & 8 \\
\end{tabularx}
1 No more than two attribute scores can be reduced below 0 in this way.
\end{dtable}
