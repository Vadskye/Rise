\chapter{Attributes}
Each character has six attributes: Strength (Str), Dexterity (Dex), Constitution (Con), Intelligence (Int), Wisdom (Wis), and Charisma (Cha). These attributes represent a character's raw talent in that area. A 0 in an attribute represents average human capacity. That doesn't mean that every commoner has a 0 in every attribute; not everyone is average, after all.

\section{Attribute Descriptions}

\subsection{Strength (Str)}
Strength measures muscle and physical power. It affects your character's accuracy and power with weapons, as well as the amount of equipment your character can carry.

\subsection{Dexterity (Dex)}
Dexterity measures hand-eye coordination, agility, and reflexes. It affects your character's accuracy with some weapons and ability to avoid or react to attacks.

\subsection{Constitution (Con)}
Constitution represents your character's health and stamina. It affects how much punishment your character can take.

\subsection{Intelligence (Int)}
Intelligence determines how well your character learns and reasons. It affects your character's knowledge in many areas.

\par An animal has an Intelligence score of \minus5 or lower. A creature of humanlike intelligence has a score of at least a \minus4 Intelligence.

\subsection{Wisdom (Wis)}
Wisdom describes a character's common sense, perception, and intuition. While Intelligence represents one's ability to analyze information, Wisdom represents being in tune with and aware of one's surroundings. It affects your character's perceptiveness.

\subsection{Charisma (Cha)}
Charisma measures a character's force of personality, willpower, and personal magnetism. It affects your character's actual strength of personality, not merely how your character is perceived by others in a social setting.

When an attribute score changes, almost all abilities associated with that score change accordingly.

\section{Using Attributes}
When you do something related to an attribute, you usually add the attribute score to your roll. Attributes are also used for other things, such as determining how much you can carry or how difficult you are to kill. 

\subsection{Attributes and Spellcasters}
Using magic requires a strong mind. The attribute that governs spellcasting depends on what type of spellcaster your character is: Intelligence for wizards; Wisdom for druids; or Charisma for paladins, sorcerers, and most clerics. That attribute is called your casting attribute. The power of many spells is affected by your casting attribute. In addition, you must have a minimum score in your casting attribute to cast spells. Divine casters (clerics, druids, and paladins) must have a casting attribute of at least half the spell's level. Arcane casters (sorcerers and wizards) must have a casting attribute at least equal to the spell's level. In addition to having a high attribute, a spellcaster must be of high enough class level to be able to cast spells of a given spell level. (See the class descriptions for details.)

\subsection{Attribute Limits}
Your attributes can never exceed 10. Magical and extraordinary creatures can exceed this limitation, but ordinary mortals are limited by their physical form.

\subsection{Increasing Attributes}
Your attributes increase as you gain levels, and some special abilities can also increase your attributes. Usually, when you increase an attribute, you also increase your skill points, hit points, and all other effects of the attribute appropriately. However, enhancement and circumstance bonuses to attributes do not increase your skill points.

\section{Determining Attribute Scores}
There are several options for how to determine attribute scores.

\subsubsection{Predefined Attribute Scores}
This is the simplest method. Simply take the following set of attribute scores and distribute them as you choose among your character's abilities:

4, 3, 2, 1, 0, \minus1

This set of attribute scores is called the ``elite array''. For more extreme characters, you may use the ``savant array'':

5, 2, 1, 0, 0, \minus2.

Finally, for more well-balanced characters, you may use the ``balanced array'':

3, 3, 2, 1, 1, 0

\subsection{Point Buy}
With this method, you can fully control your character's attribute scores to match what you want your character to be. All your character's attribute scores start at 0. You get 10 points to distribute among your character's attribute scores. Attribute scores can be bought according to the costs on \trefnp{Attribute Score Point Costs}.

\begin{dtable}
\lcaption{Attribute Score Point Costs}
\begin{tabularx}{\columnwidth}{X X X X}
\thead{Attribute Score} & \thead{Point Cost} & \thead{Attribute Score} & \thead{Point Cost} \\
\minus2 & \minus2\fn{1} & 2 & 2 \\
\minus1 & \minus1\fn{1} & 3 & 3 \\
0 & 0 & 4 & 5 \\
1 & 1 & 5 & 8 \\
\end{tabularx}
1 No more than two attribute scores can be reduced below 0 in this way.
\end{dtable}
