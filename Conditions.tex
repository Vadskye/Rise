\appendix
\chapter{Conditions}

\parhead{Ability Damaged} The character has temporarily lost 1 or more attribute score points. Lost points return at a rate of 1 per day unless noted otherwise by the condition dealing the damage. A character with Strength \minus10 falls to the ground and is helpless. A character with Dexterity \minus10 is paralyzed. A character with Constitution \minus10 is dead. A character with Intelligence, Wisdom, or Charisma \minus10 is unconscious. Ability damage is different from penalties to attribute scores, which go away when the conditions causing them go away.

\parhead{Ability Drained} The character has permanently lost 1 or more attribute score points. The character can regain these points only through magical means. A character with Strength \minus10 falls to the ground and is helpless. A character with Dexterity \minus10 is paralyzed. A character with Constitution \minus10 is dead. A character with Intelligence, Wisdom, or Charisma \minus10 is unconscious.

\parhead{Bewildered} A bewildered creature is mentally affected in a way that detracts from his ability to act, causing him to be vulnerable. A vulnerable creature takes a \minus2 penalty to attacks, defenses, and checks.

\parhead{Blinded} A blinded character cannot see. She is flat-footed and moves at half speed. In addition, she is defenseless, causing her to provoke attacks of opportunity for all her actions. All checks and activities that rely on vision (such as reading and Spot checks) automatically fail, and any checks related to vision (such as Climb and Sense Motive checks) take a \minus4 penalty. All opponents are considered to be invisible (50\% miss chance) relative to the blinded character.

\parhead{Blown Away} Depending on its size, a creature can be blown away by winds of high velocity. A creature on the ground that is blown away is knocked down and rolls 1d4 \mtimes 10 feet, taking 1d4 points of nonlethal damage per 10 feet. A flying creature that is blown away is blown back 2d6 \mtimes 10 feet and takes 2d6 points of nonlethal damage due to battering and buffering.

\parhead{Bloodied} At or below half hit points. Bloodied creatures are more vulnerable to many spells and effects.

\parhead{Checked} Prevented from achieving forward motion by an applied force, such as wind. Checked creatures on the ground merely stop. Checked flying creatures move back a distance specified in the description of the effect.

\parhead{Confused}\label{Confused} A confused character is unable to independently control its actions. \confusionexplanation A confused character does not make attacks of opportunity against any creature that it is not already devoted to attacking (either because of its most recent action or because it has just been attacked).

\parhead{Cowering} The character is frozen in fear and can take no actions. A cowering character is vulnerable and flat-footed, causing it to take a \minus2 penalty on attacks, defenses, and checks.

\parhead{Dazed} The creature is unable to act normally. A dazed creature can take no actions, but has no penalty to AC. A dazed condition typically lasts 1 round.

\parhead{Dazzled} The creature is unable to see well because of overstimulation of the eyes. A dazzled creature has a 20\% miss chance on all attack rolls and takes a \minus4 penalty to Spot checks. He is also unable to see with darkvision.

\parhead{Dead} The character's critical damage exceeds his Constitution score, his Constitution drops to 0, or he is killed outright by a spell or effect. The character's soul leaves his body. Dead characters cannot benefit from normal or magical healing, but they can be restored to life via magic. A dead body decays normally unless magically preserved, but magic that restores a dead character to life also restores the body either to full health or to its condition at the time of death (depending on the spell or device). Either way, resurrected characters need not worry about rigor mortis, decomposition, and other conditions that affect dead bodies.

\parhead{Deafened} A deafened character cannot hear. She automatically fails Listen checks, takes a \minus2 penalty to any checks which involve hearing, and has a 20\% chance of spell failure when casting spells with verbal components.

\parhead{Dying} A dying character is unconscious and near death. See \pcref{Dying}.

\parhead{Encumbered} An encumbered creature has its motion restricted by armor or weight. It may be unable to use certain class features and abilities which require free motion. See \pcref{Encumbrance} for details.

\parhead{Entangled} The character is ensnared. Being entangled impedes movement, but does not entirely prevent it unless the bonds are anchored to an immobile object or tethered by an opposing force. An entangled creature moves at half speed, cannot run or charge, and takes a \minus2 penalty to attack rolls, Strength and Dexterity-based checks, and armor class. An entangled character who attempts to cast a spell must make a Concentration check (DC 10 \add double the spell's level) or lose the spell.

\parhead{Exhausted} An exhausted character cannot sprint or charge, moves at half speed, and takes a \minus4 penalty to attacks, defenses, and checks. After 1 hour of complete rest, an exhausted character becomes fatigued. A fatigued character becomes exhausted by doing something else that would normally cause fatigue.

\parhead{Fascinated} A fascinated creature is entranced by a supernatural or spell effect. The creature stands or sits quietly, taking no actions other than to pay attention to the fascinating effect for as long as the effect lasts. It takes a \minus4 penalty on skill checks made as reactions, such as Listen and Spot checks. Any potential threat, such as a hostile creature approaching, allows the fascinated creature a new saving throw against the fascinating effect. Any obvious threat, such as noticing someone draw a weapon, cast a spell, or aim a ranged weapon at the fascinated creature automatically breaks the effect. A fascinated creature's ally may shake it free of the spell as a standard action.

\parhead{Fatigued} A fatigued character can neither sprint nor charge and is vulnerable, giving it a \minus2 penalty to attacks, defenses, and checks. Doing anything that would normally cause fatigue causes the fatigued character to become exhausted. After 8 hours of complete rest, fatigued characters are no longer fatigued.

\parhead{Flat-Footed} A flat-footed character is unable to react normally to attacks. It does not apply its Dexterity (if positive), dodge modifier, or shield modifier to its AC. It still suffers any penalties from having a negative Dexterity.

\parhead{Frightened} A frightened creature flees from the source of its fear as best it can. If unable to flee, it may fight. A frightened creature is vulnerable, causing it to take a \minus2 penalty on all attacks, defenses, and checks. A frightened creature can use special abilities, including spells, to flee; indeed, the creature must use such means if they are the only way to escape.

A character shaken by multiple sources becomes frightened. A character frightened by multiple sources becomes panicked.

\parhead{Grappled} A grappled character is in wrestling or some other form of hand-to-hand struggle with one or more attackers. While grappled, you suffer certain penalties and restrictions, as described below.
\begin{itemize*}
\item You must use one of your hands (or equivalent limbs) to grapple, preventing you from taking any actions which would require having two free hands. For example, you cannot attack with a two-handed weapon while grappling.
\item You take a \minus2 penalty to all attack rolls except those made to grapple.
\item You lose your Dexterity and dodge modifiers to AC to all opponents except the one you are grappling.
\item Anyone making a ranged attack at you has a 50\% chance to strike the other participant in the grapple instead.
\item You do not threaten any opponents except for the creature you are grappling with.
\item You take an additional \minus4 penalty to attack rolls made with one-handed weapons, since they are too large and cumbersome to be used effectively in a grapple.
\item You cannot cast spells with somatic components.
\item Casting a spell without somatic components requires a DC 20 \add double spell level Concentration check.
\item You cannot move normally.
\end{itemize*}
Other than the restrictions listed above, you can act normally. You can also try to move the grapple, escape the grapple, or pin your opponent. See \pref{Grapple} for more information.

\parhead{Helpless} A helpless character is paralyzed, held, bound, sleeping, unconscious, or otherwise completely at an opponent's mercy. It is flat-footed and is treated as having a Dexterity of \minus10. Helpless creatures can be killed instantly by a coup de grace (see \pcref{Coup de Grace}).

\parhead{Incorporeal} Having no physical body. Incorporeal creatures are immune to all nonmagical attack forms. They can be harmed only by other incorporeal creatures, \plus1 or better magic weapons, spells, spell-like effects, or supernatural effects.

\parhead{Ignited} An ignited creature has been set on fire. It is vulnerable, causing it to take a \minus2 penalty to attacks, defenses, and checks. In addition, at the end of its turn, it takes d6 damage per round from the fire. If the creature takes a move action, it can attempt a DC 15 Reflex save to put out the flames. This action requires a free hand. Dropping prone as part of the action gives a \plus4 circumstance bonus on this save.

\parhead{Immobilized} An immobilized creature can't move out of the space it was in when it became immobilized. Immobilized flying creatures that have the ability to hover can maintain their initial altitude. All other flying creatures subjected to this condition descend at a rate of 20 feet per round until they reach the ground, taking no falling damage.

\parhead{Invisible} Visually undetectable. An invisible creature is almost always hidden, causing its foes to be flat-footed against its attacks. It is difficult to detect by visual means, gaining a \plus20 circumstance bonus to Hide checks while moving and a \plus40 bonus while immobile. Attackers suffer a 50\% miss chance even if they know the location of the invisible creature, and knowing the location of an invisible creature does not prevent you from being flat-footed against its attacks. See \pcref{Perception} and \pcref{Stealth}, for how to identify invisible creatures.

\parhead{Knocked Down} Depending on their size, creatures can be knocked down by winds of high velocity. Creatures on the ground are knocked prone by the force of the wind. Flying creatures are instead blown back 1d6 \mtimes 10 feet.

\parhead{Nauseated} Experiencing stomach distress. Nauseated creatures are unable to attack, cast spells, concentrate on spells, or do anything else requiring attention. The only action such a character can take is a single move action per turn, plus free actions. A nauseated creature is also vulnerable, causing it to take a \minus2 penalty to attacks, defenses, and checks.

\parhead{Overwhelmed} An overwhelmed creature is surrounded by enemies. Any creature suffering overwhelm penalties is considered to be overwhelmed. If a creature cannot be overwhelmed, it is immune to overwhelm penalties.

\parhead{Negative Levels} A negative level gives a creature a \minus1 penalty on attack rolls, saving throws, checks, and DCs. Additionally, it lowers the creature's maximum hit points by 5, and a spellcaster loses one spell slot from his or her highest available level. If the subject has at least as many negative levels as it has levels, it dies.

\parhead{Panicked} A panicked creature must drop anything it holds and flee at top speed from the source of its fear, as well as any other dangers it encounters, along a random path. It can't take any other actions. In addition, the creature is vulnerable, causing it to take a \minus2 penalty on attacks, defenses, and checks. If cornered, a panicked creature cowers and does not attack, typically using the total defense action in combat. A panicked creature can use special abilities, including spells, to flee; indeed, the creature must use such means if they are the only way to escape.

Panicked is a more extreme state of fear than shaken or frightened.

\parhead{Paralyzed} A paralyzed character is frozen in place and unable to move or act. A paralyzed character has effective Dexterity and Strength scores of \minus10 and is helpless, but can take purely mental actions. A winged creature flying in the air at the time that it becomes paralyzed cannot flap its wings and falls. A paralyzed swimmer can't swim and may drown. A creature can move through a space occupied by a paralyzed creature -- ally or otherwise. Each square occupied by a paralyzed creature, however, counts as 2 squares.

\parhead{Petrified} A petrified character has been turned to stone and is considered unconscious. If a petrified character cracks or breaks, but the broken pieces are joined with the body as he returns to flesh, he is unharmed. If the character's petrified body is incomplete when it returns to flesh, the body is similarly incomplete and there is some amount of permanent hit point loss and/or debilitation.

\parhead{Pinned} Held immobile (but not helpless) in a grapple.

\parhead{Prone} The character is on the ground. An attacker who is prone has a \minus4 penalty on melee attack rolls and cannot use a ranged weapon (except for a crossbow). A defender who is prone gains a \plus4 bonus to Armor Class against ranged attacks, but takes a \minus4 penalty to AC against melee attacks.

Standing up is a move action that does not provoke attacks of opportunity.

\parhead{Shaken} A shaken character afraid, making it vulnerable. A vulnerable creature takes a \minus2 penalty on attacks, defenses, and checks.

A character shaken by multiple sources becomes frightened. A character frightened by multiple sources becomes panicked.

\parhead{Sickened} A sickened character feels physically ill, making him vulnerable. A vulnerable creature takes a \minus2 penalty on attacks, defenses, and checks.

\parhead{Slowed} A slowed creature can take only a single move action or standard action each turn, but not both. It cannot take full-round actions, but it may take swift actions. Additionally, it takes a \minus2 penalty to attack rolls, Strength and Dexterity-based checks, and armor class.

\parhead{Stable} A character who was dying but who has stopped losing hit points and still has critical damage is stable. The character is no longer dying, but is still unconscious. See \pcref{Stable}.

\parhead{Staggered} A staggered character may take a single move action or standard action each round, but not both. She cannot take full-round actions, but she may take swift actions. In addition, she is vulnerable, causing her to take a \minus2 penalty on attacks, defenses, and checks. A character with 0 hit points is staggered.

\parhead{Stunned} A stunned creature drops everything held, can't take actions, and is flat-footed and vulnerable, causing her to take a \minus2 penalty on attacks, defenses, and checks.

\parhead{Unaware} An unaware creature has not had time to react. It is flat-footed and cannot take attacks of opportunity or immediate actions. All creatures are unaware until they take their first action in an encounter. Note that an encounter can start before initiative is rolled.

\parhead{Unconscious} Knocked out and helpless. Unconsciousness usually results from taking significant damage. 

\parhead{Vulnerable} Weakened and susceptable to attack. A creature can be vulnerable for many reasons. A vulnerable creature takes a \minus2 penalty to attacks, defenses, and checks.
