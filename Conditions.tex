\appendix
\chapter{Conditions}\label{Conditions}

\condition{Ability Damaged} The creature has temporarily lost 1 or more attribute score points. Lost points return at a rate of 1 per day unless noted otherwise by the condition dealing the damage. A creature with Strength \minus10 falls to the ground and is helpless. A creature with Dexterity \minus10 is paralyzed. A creature with Constitution \minus10 is dead. A creature with Intelligence, Wisdom, or Charisma \minus10 is unconscious. Ability damage is different from penalties to attribute scores, which go away when the conditions causing them go away.

\condition{Ability Drained} The creature has permanently lost 1 or more attribute score points. The creature can regain these points only through magical means. A creature with Strength \minus10 falls to the ground and is helpless. A creature with Dexterity \minus10 is paralyzed. A creature with Constitution \minus10 is dead. A creature with Intelligence, Wisdom, or Charisma \minus10 is unconscious.

\condition{Bewildered} A bewildered creature is mentally affected in a way that detracts from his ability to act, causing him to be vulnerable. A vulnerable creature takes a \minus2 penalty to attacks, defenses, and checks.

\condition{Blinded} A blinded creature cannot see. It moves at half speed and is defenseless, causing it to provoke attacks of opportunity for its actions. All checks and activities that rely on vision (such as reading and visual Perception checks) automatically fail, and any checks related to vision (such as Climb and Sense Motive checks) take a \minus4 penalty. All opponents are considered to be invisible (50\% miss chance) relative to the blinded creature.

\condition{Blown Away} Depending on its size, a creature can be blown away by winds of high velocity. A creature on the ground that is blown away is knocked down and rolls 1d4 \mtimes 10 feet, taking 1d4 points of nonlethal damage per 10 feet. A flying creature that is blown away is blown back 2d6 \mtimes 10 feet and takes 2d6 points of nonlethal damage due to battering and buffering.

\condition{Bloodied} At or below half hit points. Bloodied creatures are more vulnerable to many spells and effects.

\condition{Checked} Prevented from achieving forward motion by an applied force, such as wind. Checked creatures on the ground merely stop. Checked flying creatures move back a distance specified in the description of the effect.

\condition{Confused} A confused creature is unable to independently control its actions. \confusionexplanation A confused creature does not make attacks of opportunity against any creature that it is not already devoted to attacking (either because of its most recent action or because it has just been attacked).

\condition{Cowering} The creature is frozen in fear and can take no actions. A cowering creature is vulnerable (\minus2 penalty to attacks, saves, and checks).

\condition{Dazed} The creature is unable to act normally. A dazed creature can take no actions, but can defend itself normally. A dazed condition typically lasts 1 round.

\condition{Dazzled} The creature is unable to see well because of overstimulation of the eyes. A dazzled creature has a 20\% miss chance on all attack rolls and takes a \minus4 penalty to visual Perception checks. He is also unable to see with darkvision.

\condition{Dead} The creature's critical damage exceeds its Constitution score, its Constitution drops to 0, or it is killed outright by a spell or effect. The creature's soul leaves its body. Dead creatures cannot benefit from normal or magical healing, but they can be restored to life via magic. A dead body decays normally unless magically preserved, but magic that restores a dead creature to life also restores the body either to full health or to its condition at the time of death (depending on the spell or device). Either way, resurrected creatures need not worry about rigor mortis, decomposition, and other conditions that affect dead bodies.

\condition{Deafened} A deafened creature cannot hear. All checks and activities that rely on hearing fail, and any checks related to hearing (such as Sense Motive checks) take a \minus4 penalty. In addition, the creature has a 20\% chance of spell failure when casting spells with verbal components.

\condition{Defenseless} A defenseless creature is unable to defend itself in melee combat. Any creature not capable of using a weapon or shield to defend itself, such as most unarmed creatures, is defenseless. A defenseless creature provokes an attack of opportunity each time it takes a standard, move, or full-round action.

\parhead{Dying} A dying creature is unconscious and near death. See \pcref{Dying}.

\condition{Encumbered} An encumbered creature has its motion restricted by armor or weight. It may be unable to use certain class features and abilities which require free motion. See \pcref{Encumbrance} for details.

\condition{Entangled} The creature is ensnared in a net or other physical restraint. An entangled creature moves at half speed, cannot sprint or charge, and takes a \minus2 penalty to physical attacks, defenses, and checks.

\condition{Exhausted} An exhausted creature cannot sprint or charge, moves at half speed, and takes a \minus4 penalty to attacks, defenses, and checks. After 1 hour of complete rest, an exhausted creature becomes fatigued. A fatigued creature becomes exhausted by doing something else that would normally cause fatigue.

\condition{Fascinated} A fascinated creature is entranced by a supernatural or spell effect. The creature stands or sits quietly, taking no actions other than to pay attention to the fascinating effect for as long as the effect lasts. It takes a \minus4 penalty on skill checks made as reactions, such as Perception checks. If the creature notices any obvious threat, such as someone aiming a ranged weapon at it, drawing a weapon, or casting a spell, it is no longer fascinated. A fascinated creature's ally may shake it free of the effect as a standard action.

\condition{Fatigued} A fatigued creature can neither sprint nor charge and is vulnerable, giving it a \minus2 penalty to attacks, defenses, and checks. If the fatigue does not have a set duration, doing anything that would normally cause fatigue causes the fatigued creature to become exhausted. After 8 hours of complete rest, fatigued creatures are no longer fatigued.

\condition{Frightened} A frightened creature flees from the source of its fear as best it can. If unable to flee, it may fight. A frightened creature is vulnerable, causing it to take a \minus2 penalty on all attacks, defenses, and checks. A frightened creature can use special abilities, including spells, to flee; indeed, the creature must use such means if they are the only way to escape.

A creature shaken by multiple sources becomes frightened. A creature frightened by multiple sources becomes panicked.

\condition{Grappled} A grappled creature is in wrestling or some other form of hand-to-hand struggle with one or more attackers. While grappled, you suffer certain penalties and restrictions, as described below.

\begin{itemize}
    \item You must use a free hand (or equivalent limbs) to grapple, preventing you from taking any actions which would require having two free hands. If you cannot free a hand, you suffer a \minus10 penalty to all physical attacks, including grapple attacks, until you have a free hand.
    \item You take a \minus4 penalty to physical defenses against creatures you are not grappling with.
    \item You take a \minus4 penalty to attack rolls made with weapons that are not light, since they are too large and cumbersome to be used effectively in a grapple.
    \item Spellcasting is extremely difficult. You cannot cast spells with somatic components. Using a spell-like ability or casting a spell without somatic components requires a Concentration check with a DC equal to 20 \add double spell level.
    \item You cannot move normally (but see Move the Grapple, below).
\end{itemize}

Other than the restrictions listed above, you can act normally. You can also try to move the grapple, escape the grapple, or pin your opponent. See \pref{Grapple} for more information.

\condition{Helpless} A helpless creature is completely at an opponent's mercy. Its physical defenses are equal to 10 \add its size modifier. In addition, it cannot take attacks of opportunity. Paralyzed, bound, and unconscious creatures are helpless, as well as creatures completely unaware of an attack. Helpless creatures can be killed instantly by a coup de grace (see \pcref{Coup de Grace}).

\condition{Incorporeal} Having no physical body. Incorporeal creatures are immune to all nonmagical attack forms. They can be harmed only by other incorporeal creatures, \plus1 or better magic weapons, spells, spell-like effects, or supernatural effects.

\condition{Ignited} An ignited creature has been set on fire. It is vulnerable, causing it to take a \minus2 penalty to attacks, defenses, and checks. In addition, at the end of each of its turns, it takes d6 damage from the fire. If the creature takes a move action, it can attempt a DC 10 Dexterity check to put out the flames. This action requires a free hand. Dropping prone as part of the action gives a \plus5 bonus on this check.

\condition{Immobilized} An immobilized creature can't move out of the space it was in when it became immobilized. Immobilized flying creatures that have the ability to hover can maintain their initial altitude. All other flying creatures subjected to this condition descend at a rate of 20 feet per round until they reach the ground, taking no falling damage.

\condition{Invisible} An invisible creature or object cannot be seen. Other creatures are defenseless against an invisible creature, causing them to provoke attacks of opportunity for their actions. Attackers suffer a 50\% miss chance even if they know the location of the invisible creature. See \pcref{Perception} and \pcref{Stealth}, for how to identify invisible creatures.

\condition{Knocked Down} Depending on their size, creatures can be knocked down by winds of high velocity. Creatures on the ground are knocked prone by the force of the wind. Flying creatures are instead blown back 1d6 \mtimes 10 feet.

\condition{Nauseated} Experiencing incapacitating stomach distress. Nauseated creatures are unable act during the action phase. A nauseated creature is also vulnerable, causing it to take a \minus2 penalty to attacks, defenses, and checks.

\condition{Overwhelmed} An overwhelmed creature is surrounded by enemies. Any creature suffering overwhelm penalties is considered to be overwhelmed. If a creature cannot be overwhelmed, it is immune to overwhelm penalties.

\condition{Negative Levels} A negative level gives a creature a \minus1 penalty on attacks, special defenses, and checks. Additionally, it lowers the creature's maximum hit points by 5, and a spellcaster loses one spell slot from his or her highest available level. If the subject has at least as many negative levels as it has levels, it dies.

\condition{Panicked} A panicked creature must drop anything it holds and flee at top speed from the source of its fear, as well as any other dangers it encounters, along a random path. It can't take any other actions. In addition, the creature is vulnerable, causing it to take a \minus2 penalty on attacks, defenses, and checks. If cornered, a panicked creature cowers and does not attack, typically using the total defense action in combat. A panicked creature can use special abilities, including spells, to flee; indeed, the creature must use such means if they are the only way to escape.

Panicked is a more extreme state of fear than shaken or frightened.

\condition{Paralyzed} A paralyzed creature is frozen in place and unable to move or act. A paralyzed creature has effective Dexterity and Strength scores of \minus10 and is helpless, but can take purely mental actions. A winged creature flying in the air at the time that it becomes paralyzed cannot flap its wings and falls. A paralyzed swimmer can't swim and may drown. A creature can move through a space occupied by a paralyzed creature -- ally or otherwise. Each square occupied by a paralyzed creature, however, counts as 2 squares.

\condition{Petrified} A petrified creature has been turned to stone and is considered unconscious. If a petrified creature cracks or breaks, but the broken pieces are joined with the body as he returns to flesh, he is unharmed. If the creature's petrified body is incomplete when it returns to flesh, the body is similarly incomplete and there is some amount of permanent hit point loss and/or debilitation.

\condition{Pinned} A pinned creature is held completely immobile in a grapple. Like a helpless creature, its physical defenses are equal to 10 \add its size modifier. Unlike a helpless creature, a pinned creature cannot be killed instantly by a coup de grace.

\condition{Prone} The creature is on the ground. An attacker who is prone has a \minus4 penalty on melee attack rolls and cannot use a ranged weapon (except for a crossbow). A defender who is prone gains a \plus4 bonus to physical defenses against ranged attacks, but takes a \minus4 penalty to physical defenses against melee attacks.

Standing up is a move action that does not provoke attacks of opportunity.

\condition{Shaken} A shaken creature is afraid, making it vulnerable. A vulnerable creature takes a \minus2 penalty on attacks, defenses, and checks.

A creature shaken by multiple sources becomes frightened. A creature frightened by multiple sources becomes panicked.

\condition{Sickened} A sickened creature feels physically ill, making him vulnerable. A vulnerable creature takes a \minus2 penalty on attacks, defenses, and checks.

\condition{Slowed} A slowed creature cannot act during the movement phase. It cannot take full-round actions, but it may take swift actions. Additionally, it takes a \minus2 penalty to physical attacks, defenses, and checks.

\parhead{Stable} A creature who was dying but who has stopped losing hit points and still has critical damage is stable. The creature is no longer dying, but is still unconscious. See \pcref{Stable}.

\condition{Staggered} A staggered creature cannot act during the movement phase. In addition, it is \vulnerable. A creature with 0 hit points is staggered.

\condition{Stunned} A stunned creature can't take actions and is vulnerable, causing it to take a \minus2 penalty on attacks, defenses, and checks.

\condition{Unaware} A creature unaware that it is being attacked is helpless. Once it has been attacked, the creature is not helpless, even if cannot see or identify its attacker.

\condition{Unconscious} Knocked out and helpless. Unconsciousness usually results from taking significant damage. 

\condition{Vulnerable} Weakened and susceptable to attack. A creature can be vulnerable for many reasons. A vulnerable creature takes a \minus2 penalty to attacks, defenses, and checks.
