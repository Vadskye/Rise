\chapter{Alternate Class Features}

\section*{BARBARIAN}
\parhead{Frenzy:} You frenzy instead of raging. Frenzy is identical to rage, with the following exceptions:
\begin{itemize*}
\item You gain a bonus to Dexterity instead of to Strength.
\item You don't gain temporary hit points from frenzying.
\item Your armor class doesn't decrease.
\end{itemize*}

\cf{Bbn}{Spellsundering Strike (Su)} At 7th level, a barbarian can put such force into an attack that it can break magical effects on the target. The fact that most barbarians (and even many wizards) have no idea how this works is no impediment to its effectiveness.
\par A barbarian may make a spellsundering strike a number of times per day equal to his Constitution as part of an attack action. Anything hit by a spellsundering strike becomes the target of a targeted \spell{dispel magic} effect, with a bonus equal to the barbarian's class level (not capped at \plus10). However, the barbarian cannot dispel any effect with a caster level higher than the damage he dealt with the spellsundering strike (though the damage need not be lethal).