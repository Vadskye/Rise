\chapter{Species}\label{Species}
Each character has a species.
There are eight common species described below.
At the GM's discretion, you may be able to play a character with a more unusual species (see \pcref{Uncommon Species}).

\sectiongraphic*{Humans}{width=\columnwidth}{characters/human}

  Humans are the most common and least rigidly defined of all Rise species.
  They are not the smartest, the strongest, or the most durable of the civilized races.
  They have no supernatural senses or impossible talents; anything a human can do, a member of another species could do at least as well.
  Despite their limitations, humans are practically universal, and their civilizations are the most powerful and numerous of all.

  The success of humanity comes from one core strength: their adaptability, both individually and as a whole.
  Individual humans can learn new skills with surprising ease compared to other species, and they often have a breadth of talent that few can rival.
  The relatively short human lifespan prevents their society from stagnating under the guidance of elders whose wisdom is now hundreds of years out of date.
  When radical changes sweep the world, humans can adapt where other species would founder.

  \parhead{Size} Medium.
  \parhead{Attributes} See the Versatile ability.
  \parhead{Special Abilities}
  \begin{raggeditemize}
    \itemhead{Skilled} Humans gain an additional \glossterm{trained skill} (see \pcref{Skills}).
    \itemhead{Versatile} Humans gain a \plus1 bonus to any attribute.
      This cannot increase that attribute above 3 when attributes are initially determined (see \pcref{Attributes}).
      That attribute be increased above 3 by other means, such as by your \glossterm{base class} or by the attribute bonus at level 3 (see \pcref{Character Advancement and Gaining Levels}).
  \end{raggeditemize}
  \parhead{Automatic Language} Common, any one \glossterm{common language} (see \tref{Common Languages}).

\sectiongraphic*{Dwarves}{width=\columnwidth}{characters/dwarf}

  Dwarves are short, stout, and sturdy.
  It has been said that the first dwarf was carved from stone, and the similarities have been noted by many.
  All dwarves naturally have beards, and the vast majority keep them long and elegantly maintained.

  Most dwarves live underground in mining communities.
  These communities can grow to massive size, and dwarven kings can rule vast underground cities.
  The dwarven fascination with strong drink is legendary, though somewhat misleading.
  Their natural resilience means they need stronger drinks to even notice the effects, so other species tend to gain an exaggerated impression of dwarven drunkenness when they try to drink dwarven ale.

  \parhead{Size} Medium.
  \parhead{Attributes} \plus1 Constitution, \plus1 Willpower, \minus1 Dexterity.
  \parhead{Special Abilities}
  \begin{raggeditemize}
    \itemhead{Armored From Birth} Dwarves are proficient with all \glossterm{usage classes} of metal body armor (see \pcref{Armor}).
      This does not grant proficiency with other kinds of armor, such as shields.
    \itemhead{Darkvision} Dwarves have \trait{darkvision} with a 60 foot range, allowing them to see in complete darkness (see \pcref{Darkvision}).
    \itemhead{Depth Sense} Dwarves can intuitively sense their approximate depth underground as naturally as a human can sense which way is up.
    \itemhead{Earthen Crafting} Dwarves gain a \plus2 bonus to the Craft (metal) and Craft (stone) skills.
    \itemhead{Slow and Steady} Dwarves have a \minus10 foot penalty to their speed with all \glossterm{movement modes}.
      However, wearing heavy \glossterm{body armor} does not reduce a dwarf's speed (see \pcref{Armor Usage Classes}).
      In addition, a dwarf's land speed cannot be more than 10 feet slower than their \glossterm{base speed}, even while \slowed or under similar effects.
      This does not affect movement through \glossterm{difficult terrain} and similar effects, since those affect movement cost rather than your character's intrinsic movement speed.
  \end{raggeditemize}
  \parhead{Automatic Languages} Common, Dwarven, any one \glossterm{common language} (see \tref{Common Languages}).

\sectiongraphic*{Elves}{width=\columnwidth}{characters/elf}

  Elves are tall, lithe, and graceful.
  They tend to have an air of confidence at all times, and even their mistakes seem intentional.
  Elves have the longest lifespan of any civilized species, and even comparatively young elves carry a weight of experience that can be daunting for non-elves.

  For millenia, elves were the most powerful civilization above ground, while dwarves claimed the underground.
  More recently, humans have usurped elves as the most powerful civilization above ground, while dwarves have kept their claim.
  This history, combined with their natural differences, has created an ancient rivalry between elves and dwarves.

  \parhead{Size} Medium.
  \parhead{Attributes} \minus1 Constitution, either \plus1 Dexterity or \plus1 Perception.
  \parhead{Special Abilities}
  \begin{raggeditemize}
    \itemhead{Elven Perfection} Elves only gain one \glossterm{fatigue level} when using the \ability{desperate exertion} ability to affect checks (see \pcref{Desperate Exertion}).
    \itemhead{Keen Senses} Elves gain a \plus2 bonus to the Awareness skill (see \pcref{Awareness}).
    \itemhead{Low-light Vision} Elves have \trait{low-light vision}, allowing them to see clearly in \glossterm{shadowy illumination} (see \pcref{Low-light Vision}).
    \itemhead{Sure-Footed} Elves gain a \plus2 bonus to the Balance and Stealth skills (see \pcref{Balance}, and \pcref{Stealth}).
    \itemhead{Trance} Elves do not sleep, and are immune to \magical effects that would cause them to sleep.
      Instead of sleeping, elves can trance for 4 hours.
      An elf in trance may make Perception-based checks at a \minus5 penalty.
      Elves must still avoid strenuous activity for 8 hours to heal and gain other benefits of taking a \glossterm{long rest}.
  \end{raggeditemize}
  \parhead{Automatic Languages} Common, Elven, any one \glossterm{common language} (see \tref{Common Languages}).

\sectiongraphic*{Gnomes}{width=\columnwidth}{characters/gnome}

  Gnomes are the smallest, most magical, and most short-lived of the civilized species.
  Their large eyes and heads give even adult gnomes almost child-like proportions.
  Fae blood runs in the blood of all gnomes, and gnome societies have many traditions and rituals that seem superstitious to outsiders.
  However, these rituals have a purpose, and gnomes understand that failing to appease the hidden powers in the world can have dangerous consequences.

  Most gnomes live in forests, but they can be found in remote areas all over the world.
  Gnomish settlements are almost always overseen by minor fae, such as dryads, who protect the settlement.
  In many cases, the settlements were originally built around a site of mystic power, though some settlements have outlived their original protectors.

  \parhead{Size} Medium.
  \parhead{Attributes} \minus1 Strength, either \plus1 Constitution or \plus1 Intelligence.
  \parhead{Special Abilities}
  \begin{raggeditemize}
    \magicalitemhead{Fae Light}
    \begin{activeability}{Fae Light}
      \abilityusagetime Standard action.
      \rankline
      A Tiny glowing orb appears at a location within \rngmed range.
      It sheds pale, \glossterm{bright illumination} in a \areasmall radius, and \glossterm{shadowy illumination} in a \areamed radius.
      The orb is intangible, and cannot be moved once placed.

      This ability lasts until you use it again or until you \glossterm{dismiss} it.
    \end{activeability}
    \itemhead{Magic Affinity} Gnomes gain an additional \glossterm{insight point}.
      They can only spend this insight point to learn \magical abilities, such as spells.
    \itemhead{Short Stature} Gnomes gain a \plus3 bonus to the Stealth skill.
    \itemhead{Tinkerer} Gnomes gain a \plus2 bonus to one Craft skill of their choice (see \pcref{Craft}).
  \end{raggeditemize}
  \parhead{Automatic Languages} Common, Gnome, either Sylvan or any one \glossterm{common language} (see \tref{Common Languages}).

\sectiongraphic*{Half-Elves}{width=\columnwidth}{characters/halfelf}

  Half-elves carry both human and elven heritage.
  They are caught between two worlds, with neither the unconscious grace of elves nor the limitless adaptability of humans.
  However, they have their own unique forms of versatility based on their understanding of both worlds.

  \parhead{Size} Medium.
  \parhead{Attributes} \minus1 Constitution, \plus1 to any other attribute.
  \parhead{Special Abilities}
  \begin{raggeditemize}
    \itemhead{Cross-Cultural Experience} Half-elves only need to spend one \glossterm{insight point} to gain access to an additional class (see \pcref{Multiclass Characters}).
    \itemhead{Diplomatic} Half-elves gain a \plus2 bonus to the Persuasion skill.
    \itemhead{Keen Senses} Half-elves gain a \plus2 bonus to the Awareness skill (see \pcref{Awareness}).
    \itemhead{Low-light Vision} Half-elves have \trait{low-light vision}, allowing them to see clearly in \glossterm{shadowy illumination} (see \pcref{Low-light Vision}).
  \end{raggeditemize}
  \parhead{Automatic Language} Common, Elven, any two \glossterm{common languages} or one \glossterm{rare language} (see \pcref{Communication and Languages}).

\sectiongraphic*{Half-orc}{width=\columnwidth}{characters/halforc}

  Half-orcs carry both human and orcish heritage.
  They have much of the brute strength of orcs, but tempered by human adaptability.
  While half-elves are often welcome in both human and elvish societies, half-orcs tend to face more challenges navigating both human and orc societies.

  \parhead{Size} Medium.
  \parhead{Attributes} \plus1 Strength, either \minus1 Intelligence or \minus1 Perception.
  \parhead{Special Abilities}
  \begin{raggeditemize}
    \itemhead{Cross-Cultural Experience} Half-orcs only need to spend one \glossterm{insight point} to gain access to an additional class (see \pcref{Multiclass Characters}).
    \itemhead{Intimidating} Half-orcs gain a \plus2 bonus to the Intimidate skill (see \pcref{Intimidate}).
    \itemhead{Low-light Vision} Half-orcs have \trait{low-light vision}, allowing them to see clearly in \glossterm{shadowy illumination} (see \pcref{Low-light Vision}).
    \itemhead{Physical Instincts} Half-orcs gain an additional \glossterm{insight point}.
      They can only spend this insight point to learn \glossterm{mundane} abilities, such as combat styles and maneuvers.
  \end{raggeditemize}
  \parhead{Automatic Languages} Common, Orcish, any one \glossterm{common language}.

\sectiongraphic*{Halflings}{width=\columnwidth}{characters/halfling}

  Halflings stand at about half the height of a human, but have generally human-like proportions.
  They tend to be plucky, adventurous, and outgoing.
  Of all species, halflings have the fewest halfling-only communities.
  Instead, halfling groups tend to live in the gaps between the ``big people'', especially in large cities.

  \parhead{Size} Medium.
  \parhead{Attributes} \minus1 Strength, either \plus1 Dexterity or \plus1 Willpower.
  \parhead{Special Abilities}
  \begin{raggeditemize}
    \itemhead{Nimble Combatant} Halflings gain a \plus1 bonus to Reflex defense.
    \itemhead{Short Stature} Halflings gain a \plus3 bonus to the Stealth skill.
    \itemhead{Stout-Hearted} Halflings gain a \plus1 bonus to Mental defense.
    \itemhead{Sure-Footed} Halflings gain a \plus2 bonus to the Balance skill (see \pcref{Balance}).
  \end{raggeditemize}
  \parhead{Automatic Languages} Common, Halfling, any one \glossterm{common language} (see \tref{Common Languages}).

\section{Orcs}

  Orcs are tall, physically imposing green-skinned humanoids with a reputation for brutishness.
  The orcs that interact with human civilization must content with prejudice, especially in rural areas.
  This is partially inspired by their less civilized cousins who live in more violent environments, and partially a simple reaction to their raw physical presence.

  \parhead{Size} Medium.
  \parhead{Attributes} \plus1 Strength, \minus1 Intelligence, \minus1 Perception, either \plus1 Dexterity or \plus1 Constitution.
  \parhead{Special Abilities}
  \begin{raggeditemize}
    \itemhead{Intimidating} Orcs gain a \plus3 bonus to the Intimidate skill (see \pcref{Intimidate}).
    \itemhead{Low-light Vision} Orcs have \trait{low-light vision}, allowing them to see clearly in \glossterm{shadowy illumination} (see \pcref{Low-light Vision}).
    \itemhead{Martial Upbringing} Orcs are proficient with all weapons from one \glossterm{weapon group} of their choice, including exotic weapons.
    \itemhead{Raw Strength} Orcs gain a \plus1 bonus to their Strength for the purpose of determining their \glossterm{weight limits} (see \pcref{Weight Limits}).
  \end{raggeditemize}
  \parhead{Automatic Languages} Common, Orcish.
