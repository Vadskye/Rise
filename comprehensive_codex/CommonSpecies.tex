\chapter{Species}\label{Species}
Each character has a species.
There are eight common species described below.
At the GM's discretion, you may be able to play a character with a more unusual species (see \pcref{Uncommon Species}).

\sectiongraphic*{Humans}{width=\columnwidth}{common species/human}

  Humans are the most common and least rigidly defined of all Rise species.
  They are not the smartest, the strongest, or the most durable of the civilized races.
  They have no supernatural senses or impossible talents; anything a human can do, a member of another species could do at least as well.
  Despite their limitations, humans are practically universal, and their civilizations are the most powerful and numerous of all.

  The success of humanity comes from one core strength: their adaptability, both individually and as a whole.
  Individual humans can learn new skills with surprising ease compared to other species, and they often have a breadth of talent that few can rival.
  The relatively short human lifespan prevents their society from stagnating under the guidance of elders whose wisdom is now hundreds of years out of date.
  When radical changes sweep the world, humans can adapt where other species would founder.

  \parhead{Size} Medium.
  \parhead{Attributes} See the Versatile ability.
  \parhead{Special Abilities}
  \begin{raggeditemize}
    \itemhead{Versatile} Humans gain a \plus1 bonus to any two attributes.
      This cannot increase those attributes above 3 when attributes are initially determined (see \pcref{Attributes}).
      They can be increased above 3 by other means, such as by the attribute bonus at level 3 (see \pcref{Character Advancement and Gaining Levels}).
  \end{raggeditemize}
  \parhead{Automatic Language} Common, any one \glossterm{common language} (see \tref{Common Languages}).

\sectiongraphic*{Dwarves}{width=\columnwidth}{common species/dwarf}

  Dwarves are short, stout, and sturdy.
  It has been said that the first dwarf was carved from stone, and the similarities have been noted by many.
  All dwarves naturally have beards, and the vast majority keep them long and elegantly maintained.

  Most dwarves live underground in mining communities.
  These communities can grow to massive size, and dwarven kings can rule vast underground cities.
  The dwarven fascination with strong drink is legendary, though somewhat misleading.
  Their natural resilience means they need stronger drinks to even notice the effects, so other species tend to gain an exaggerated impression of dwarven drunkenness when they try to drink dwarven ale.

  \parhead{Size} Medium.
  \parhead{Attributes} \plus1 Constitution, \plus1 Willpower, \minus1 Dexterity.
  \parhead{Special Abilities}
  \begin{raggeditemize}
    \itemhead{Armored From Birth} Dwarves are proficient with all \glossterm{usage classes} of metal body armor (see \pcref{Armor}).
      This does not grant proficiency with other kinds of armor, such as shields.
    \itemhead{Darkvision} Dwarves have \trait{darkvision} with a 60 foot range, allowing them to see in complete darkness (see \pcref{Darkvision}).
    \itemhead{Depth Sense} Dwarves can intuitively sense their approximate depth underground as naturally as a human can sense which way is up.
    \itemhead{Earthen Crafting} Dwarves gain a \plus2 bonus to the Craft (metal) and Craft (stone) skills.
    \itemhead{Slow and Steady} Dwarves have a \minus10 foot penalty to their \glossterm{movement speed}.
      However, wearing heavy \glossterm{body armor} does not reduce a dwarf's speed (see \pcref{Armor Usage Classes}).
      In addition, a dwarf's movement speed cannot be more than 10 feet slower than their \glossterm{base speed}, even while \slowed or under similar effects.
      This does not affect movement through \glossterm{difficult terrain} and similar effects, since those affect movement cost rather than your character's intrinsic movement speed.
  \end{raggeditemize}
  \parhead{Automatic Languages} Common, Dwarven, any one \glossterm{common language} (see \tref{Common Languages}).

\sectiongraphic*{Elves}{width=\columnwidth}{common species/elf}

  Elves are tall, lithe, and graceful.
  They tend to have an air of confidence at all times, and even their mistakes seem intentional.
  Elves have the longest lifespan of any civilized species, and even comparatively young elves carry a weight of experience that can be daunting for non-elves.

  For millenia, elves were the most powerful civilization above ground, while dwarves claimed the underground.
  More recently, humans have usurped elves as the most powerful civilization above ground, while dwarves have kept their claim.
  This history, combined with their natural differences, has created an ancient rivalry between elves and dwarves.

  \parhead{Size} Medium.
  \parhead{Attributes} \minus1 Constitution, either \plus1 Dexterity or \plus1 Perception.
  \parhead{Special Abilities}
  \begin{raggeditemize}
    \itemhead{Elven Perfection} Elves only gain one \glossterm{fatigue level} when using the \ability{desperate exertion} ability to affect checks (see \pcref{Desperate Exertion}).
    \itemhead{Keen Senses} Elves gain a \plus2 bonus to the Awareness skill (see \pcref{Awareness}).
    \itemhead{Low-light Vision} Elves have \trait{low-light vision}, allowing them to see clearly in \glossterm{dim illumination} (see \pcref{Low-light Vision}).
    \itemhead{Sure-Footed} Elves gain a \plus1 bonus to the Balance and Stealth skills (see \pcref{Balance}, and \pcref{Stealth}).
    \itemhead{Trance} Elves do not sleep, and are immune to \magical effects that would cause them to sleep.
      Instead of sleeping, elves can trance for 4 hours.
      An elf in trance may make Perception-based checks at a \minus5 penalty.
      Elves must still avoid strenuous activity for 8 hours to heal and gain other benefits of taking a \glossterm{long rest}.
  \end{raggeditemize}
  \parhead{Automatic Languages} Common, Elven, any one \glossterm{common language} (see \tref{Common Languages}).

\sectiongraphic*{Halflings}{width=\columnwidth}{common species/halfling}

  Halflings stand at a little over half the height of a human, but have generally human-like proportions.
  They tend to be friendly and welcoming hosts.
  Few halflings seek out adventure, though they are brave and capable when danger comes to them.

  Most halflings live stable and comfortable lives surrounded by friends and family.
  They make their own communities wherever they are, whether as a farming village or a large communal house in the city.

  \parhead{Size} Medium.
  \parhead{Attributes} \minus1 Strength, either \plus1 Dexterity or \plus1 Willpower.
  \parhead{Special Abilities}
  \begin{raggeditemize}
    \itemhead{Nimble Combatant} Halflings gain a \plus1 bonus to their Reflex defense.
    \itemhead{Short Stature} Halflings gain a \plus2 bonus to the Stealth skill.
    \itemhead{Stout-Hearted} Halflings gain a \plus1 bonus to their Mental defense.
  \end{raggeditemize}
  \parhead{Automatic Languages} Common, Halfling, any one \glossterm{common language} (see \tref{Common Languages}).

\sectiongraphic*{Kobolds}{width=\columnwidth}{common species/kobold}

  Kobolds have reptilian features stemming from their draconic ancestry.
  They have a short tail and scales that can be brightly colored.
  Mottled patterns are more common than consistent coloration across the whole body.
  They are even shorter than halflings, with a typical height under three feet.

  Kobolds are naturally cooperative, though they are rarely leaders.
  For some kobolds, spending some time in service to a dragon is a ritual akin to a pilgrimage.
  Serving a dragon marks kobolds, shifting their natural coloration to match the dragon they served.
  While in a dragon's service, they generally use their stealth and crafting talents to make traps that hinder any who would seek out the dragon's treasure while they are away.

  Kobolds that do not seek out true dragons may still find worthy leaders to follow.
  For example, a kobold who aspired to be a baker would usually try to find a skilled baker to serve as an apprentice.
  Such service is always conditional on the quality of their leader.
  An unfit leader can be quickly abandoned, though usually not betrayed.

  \parhead{Size} Medium.
  \parhead{Attributes} \minus1 Strength, either \plus1 Dexterity or \plus1 Intelligence.
  \parhead{Special Abilities}
  \begin{raggeditemize}
    \itemhead{Crafty} Kobolds gain a \plus2 bonus to a Craft skill of their choice.
    \itemhead{Draconic Power} Kobolds gain a \plus1 bonus to their \glossterm{mundane power} and \glossterm{magical power}.
    \itemhead{Resilient Scales} Kobolds gain a \plus1 bonus to their \glossterm{durability}.
    \itemhead{Short Stature} Kobolds gain a \plus2 bonus to the Stealth skill.
  \end{raggeditemize}
  \parhead{Automatic Languages} Common, Draconic.

\sectiongraphic*{Mixed Species}{width=\columnwidth}{common species/mixed}

  Mixed species characters come in many forms.
  They may have ancestry from two different species, such as a half-elf with a human parent and an elven parent.
  They may be biologically from one species, but raised in the culture of another.
  For example, an an orc raised in an elven commune might not have the same intimidating presence and martial upbringing of an orc raised in an orcish society.
  You can use this species to represent whatever combination you want.

  \parhead{Size} Medium.
  \parhead{Attributes} \plus1 to any attribute, \minus1 to any other attribute
  \parhead{Special Abilities}
  \begin{raggeditemize}
    \itemhead{Cross-Cultural Experience} Mixed species characters do not need to spend an \glossterm{insight point} to gain access to their first additional class (see \pcref{Multiclass Characters}).
    \itemhead{Skilled} Mixed species characters gain a \plus2 bonus to any one skill.
  \end{raggeditemize}
  \parhead{Automatic Language} Common, any two \glossterm{common languages} or one \glossterm{rare language} (see \pcref{Communication and Languages}).

\sectiongraphic*{Orcs}{width=\columnwidth}{common species/orc}

  Orcs are tall, physically imposing, and have a reputation for brutishness.
  Their skin is green, with hues ranging from the dull brownish green of a forest floor to the vibrant green of new grass.
  The orcs that interact with human civilization must content with prejudice, especially in rural areas.
  This is partially inspired by their less civilized cousins who live in more violent environments, and partially a simple reaction to their raw physical presence.

  \parhead{Size} Medium.
  \parhead{Attributes} \plus1 Strength, either \minus1 Dexterity or \minus1 Intelligence.
  \parhead{Special Abilities}
  \begin{raggeditemize}
    \itemhead{Intimidating} Orcs gain a \plus2 bonus to the Intimidate skill (see \pcref{Intimidate}).
    \itemhead{Low-light Vision} Orcs have \trait{low-light vision}, allowing them to see clearly in \glossterm{dim illumination} (see \pcref{Low-light Vision}).
    \itemhead{Martial Upbringing} Orcs are proficient with all weapons from one \glossterm{weapon group} of their choice, including exotic weapons.
    \itemhead{Raw Strength} Orcs gain a \plus1 bonus to their Strength for the purpose of determining their \glossterm{weight limits} (see \pcref{Weight Limits}).
  \end{raggeditemize}
  \parhead{Automatic Languages} Common, Orcish.
