\chapter{Optional Rules}

This chapter describes a variety of optional rules that the GM can choose to use in their campaign.
These rule changes change the tone of the game, making it more gritty or more tactical.
They can also provide more detailed character customization options, increasing character uniqueness but potentially increasing the complexity of character creation.

\section{Attributes}

  \subsection{Other Methods of Attribute Generation}
    Point buy offers the fairest and most customizable system for determining attribute scores, ensuring that players can be almost any character they want to be. However, some groups may wish to determine attribute scores differently. Other options are provided below.

    \subsubsection{Simple Random Point Buy}
      With this method, you have only a small degree of control over your attribute scores, but all characters generated in this way are equally powerful.
      As with the point buy method, all your attribute scores start at 0, and you get 15 points to distribute among your attribute scores.
      However, you do not have full control over how to distribute those points.

      For each attribute, starting with the attributes you care about most, roll 1d8.
      You spend that many points on that attribute, ignoring any extra points that can't be spent
      For example, if you roll a 4, you spend 3 points on the attribute, causing you to start with a 2.
      If you do not have enough points remaining to spend the amount indicated by the die roll, spend as many as you can and move on to the next attribute.

      If you have points remaining after rolling all of your attribute scores, you may distribute the points freely among your abilities, using the normal point buy rules.
      You cannot increase the starting value of any individual attribute by more than 1 during this stage.
      If any of your attributes start as a 0, you may choose to lower them to gain the normal benefits from having low attributes (see \pcref{Attribute Penalties}).

      To further limit your character creation options, you may choose to randomize the order in which you roll your attributes instead of rolling them in an order of your choice.

    \subsubsection{Smoothed Random Point Buy}
      This method functions like the Simple Random Point Buy method, except that the resulting attribute values have a smoother distribution, and you can randomly end up with attribute penalties.

      For each attribute, starting with the attributes you care about most, roll 4d6.
      Then, remove any one of the rolls after seeing the results.
      Sum the results of the remaining three dice and spend the appropriate number of attribute points as indicated in \trefnp{Smoothed Random Point Buy Results}.
      If you do not have enough points remaining to spend the amount indicated by the die roll, spend as many as you can and move on to the next ability.

      If you have points remaining after rolling all of your attribute scores, you may distribute the points freely among your abilities, using the normal point buy rules.
      You cannot increase the starting value of any individual attribute by more than 1 during this stage.

      To further limit your character creation options, you may choose to randomize the order in which you roll your attributes instead of rolling them in an order of your choice.

      \begin{dtable}
        \lcaption{Smoothed Random Point Buy Results}
        \begin{dtabularx}{\columnwidth}{X X X}
          \tb{Roll} & \tb{Attribute} & \tb{Point Cost} \tableheaderrule
          3-4       & \minus2        & 0\fn{1} \\
          5-6       & \minus1        & 0\fn{2} \\
          7-8       & 0              & 0       \\
          9-10      & 1              & 1       \\
          11-12     & 2              & 2       \\
          13-15     & 3              & 3       \\
          16-18     & 4              & 5       \\
        \end{dtabularx}
        1 You gain one \glossterm{insight point}. \\
        2 You gain an additional \glossterm{trained skill}. \\
      \end{dtable}

    \subsubsection{Classic Hardcore}

      This method is completely random and can generate very overpowered or underpowered characters.
      It represents the unfairness of the world, where some people are just better or worse than others.
      For each attribute, roll 2d6, take the average (rounded down), and subtract 2.
      If you roll a 1 on both dice, treat the average as a 0.
      The result is your base value for that attribute.

\section{Epic Fates}
  After 21st level, characters no longer gain levels normally.
  However, they can still increase their personal power as they make progress towards their ultimate fate.

  When you reach 21st level, you may choose an epic fate that you qualify for, or you may delay choosing until you meet the prerequisites for your desired fate.
  You do not start with any ranks in you chosen epic fate.
  Each epic fate specifies ways that you can make progress towards that epic fate.
  Whenever you make dramatic progress towards your epic fate, your rank in that epic fate may increase, at the discretion of the Game Master.

  None of the epic fate abilities have a tag to indicate that they are \magical abilities.
  Many of them are not fundamentally \glossterm{mundane} in nature, but they are beyond normal magic, and effects like an \spell{antimagic field} cannot interact with or suppress them.

  \subsection{Artificial Immortality}
    You have sought out strange magical power in search of a way to artificially prolong your life.
    As your power grows, you become increasingly able to resist death and return from it.
    Eventually, you will transcend death entirely.

    \parhead{Prerequisites} You must perform a series of rituals to prepare yourself for immortality, at least one of which must be rank 7 or higher. There are many kinds of immortality that you can pursue with this epic fate, and the exact nature of the rituals will change depending on the type of immortality you pursue.

    \parhead{Progression} You must discover powerful new magic rituals that support your particular form of immortality. This generally requires exploring sites of ancient magic, gaining favor with powerful creatures who have relevant knowledge or abilities, and independent experimentation based on your findings.

    \subsubsection{Artifical Immortality Ranks}

      \parhead{Rank 1 -- Life After Death} If you die from any cause other than old age, you resurrect according to nature of your chosen path to immortality.
      For example, you can have a phylactery regenerate a new body for you like a lich, or you can create clones of yourself or golems that you inhabit if your first body dies.
      You must always return in a new body of some sort.

      Your specific form of immortality determines where you return, such as at the site of your death or at your personal sanctum.
      However, it cannot cannot be based on the location or state of your old corpse, since that corpse is no longer ``you''.
      The timing of your resurrection may also differ based on your immortality, but you cannot complete your resurrection sooner than one day after the time of your death. After you resurrect in this way, this ability does not function for one week, allowing you to be killed normally.

      This immortality may change your base species, such as if you become a lich or move your body into a flesh golem. If it does, you retain all benefits and modifiers from your original species other than size, and you gain the effects of the new species in addition.

      \parhead{Rank 2 -- Death Familiarity} You become so familiar with the trauma of injury and death that your mind and body adapt to it.
      You gain a \plus10 bonus to vital rolls.
      In addition, the time of your vulnerability to true death after resurrection is reduced to 48 hours.

      \parhead{Rank 3 -- Artificial Life} Whenever you resurrect with your \textit{life after death} ability, your new body gains a \plus2 bonus to two random attributes. The attributes are randomized differently for each new body. In addition, that resurrection functions even if the cause of your death was old age, and you can control the physical age of your new body.

      \parhead{Rank 4 -- Deathcaller} You are deeply familiar with death, and know how to most effectively inflict it on others.
      Whenever you cause a living creature to lose at least half its hit points in a single round, you may kill that creature outright.
      In addition, your \textit{artificial life} ability grants a bonus to three random attributes instead of two random attributes.

      \parhead{Rank 5 -- True Immortality} You become fully immortal. There is no time limit after the resurrection from your \textit{life after death} ability where you become vulnerable to a true death. In addition, the resurrection can complete as quickly as one minute after your death. If a physical component limits your immortality, such as a phylactery, it can no longer be damaged or destroyed without the direct intervention of a rank 5 Slayer.

  \subsection{Ascendant}
    You have begun to see through the weave of the world and glimpse the higher truths beyond.
    As your insight into the true nature of reality grows, you begin to transcend the physical realm.
    Eventually, you become a being of pure energy.

    \parhead{Prerequisites} You must have spent at least a week living in each of the following planes: Air, Astral, Earth, Fire, Material, and Water.
    In addition, you must have an Intelligence or Perception of at least 2.

    \parhead{Progression} You must discover and spend time in exotic environments with unusual properties, especially with energy-related phenomena, to discern the underlying structure of the universe revealed in extremes.
    This involves a mix of meditation, observation, and potentially dangerous personal experience.
    Discovering potentially valuable locations may require extensive research.
    In order to reach the highest ranks, you must journey into forbidden realms of powerful magic, like the inner sanctums of major deities or the horrific depths of the Eternal Void.

    \subsubsection{Ascendant Ranks}

      \parhead{Rank 1 -- Energetic Soul} Whenever you use an ability, you can give that ability the \atCold, \atElectricity, and \atFire tags.
      In addition, whenever you take or deal damage with any of those tags while not \trait{incorporeal}, you and your equipment \glossterm{briefly} become incorporeal.
      While you are incorporeal in this way, you gain an average \glossterm{fly speed} and lose all other movement modes (see \pcref{Flight}).
      This flight has a \glossterm{height limit} of 60 feet.
      In this form you are considered native to the air, so flying in this way does not penalize your Armor or Reflex defenses.

      \parhead{Rank 2 -- See Through the Weave} You can see everything within 120 feet of you perfectly, regardless of obstacles of any kind or light levels.
      This is similar to \trait{blindsight}, except that it also ignores solid obstacles of any kind, allowing you to have \glossterm{line of sight} through walls.
      You can perceive the presence of obstacles just as well as you can see what lies behind them.

      \parhead{Rank 3 -- Reach Through the Weave} When you use any of your abilities, you can treat yourself as being up to 60 feet away from your true location.
      You do not need \glossterm{line of effect} to your chosen location.
      For example, this allows you to make melee attacks against creatures up to 60 feet away.
      This changes your \glossterm{line of effect}, but does not change your \glossterm{line of sight}.

      % Does this need to have an explicit Recover reminder? Seems like that just works as intended
      \parhead{Rank 4 -- Become Energy} You become permanently \trait{incorporeal}, along with any equipment you carry.
      The fly speed from your \textit{energetic soul} ability also becomes permanent.
      In addition, you no longer age and no longer have hit points.
      Instead, you gain a bonus to your maximum \glossterm{damage resistance} equal to the number of hit points you would normally have from your level, base class, and Constitution.
      Other effects that would increase or decrease your maximum hit points have no effect on you.
      You gain vital wounds based on taking damage in excess of your damage resistance rather than in excess of your hit points, including the extra vital wounds for taking massive damage.

      \parhead{Rank 5 -- Ascension} You cannot be killed, only dissipated.
      When you die, you automatically reform at a random location within a mile of your death after 10 minutes.
      Reforming in this way returns you to full damage resistance and removes all conditions and vital wounds, but your \glossterm{fatigue levels} and other effects remain the same.

  \subsection{Deity}
    People have begun to worship you, putting you on the path to become a deity.
    As your followers grow, you become capable of ever greater miraculous acts, and you can grant your followers some of your power.
    Eventually, you ascend into the pantheon of gods.

    \parhead{Prerequisites} You must have at least a hundred worshippers with souls to choose this epic fate.
    In addition, you must not have any cleric archetypes.

    \parhead{Progression} To progress towards this epic fate, you must gain a significant number of additional worshippers.
    In general, you must at least double your worshippers to progress towards each new rank of this fate, though this can vary widely.
    Having worshippers among many different places is more valuable than converting an isolated group to worship you, though both are helpful.

    \subsubsection{Deity Ranks}
      \parhead{Rank 1 -- Domain Influence} Choose a cleric domain.
      You gain all abilities from that domain except for its mastery ability.
      In addition, your worshippers become eligible to gain cleric archetypes, though they cannot exceed a maximum rank in those archetypes of twice your rank in this epic fate (to a maximum of 8).
      This does not grant additional archetypes to worshippers who have already chosen their three archetypes, and is usually only relevant to NPC worshippers.

      \parhead{Rank 2 -- Prayers} You hear all prayers directed to you.
      Once per week, you can teleport yourself and up to ten \glossterm{allies} any distance within the same plane as a \glossterm{standard action}.
      Your destination must either be a worshipper actively praying to you or a holy place dedicated to you.
      In addition, choose a second cleric domain.
      You gain all abilities from that domain except for its mastery ability.

      \parhead{Rank 3 -- Domain Mastery} Choose a third cleric domain.
      You gain all abilities from that domain.
      In addition, you gain the mastery ability from the domains you chose with your \textit{domain influence} and \textit{prayers} abilities.

      \parhead{Rank 4 -- Demigod} You become a demigod.
      You no longer age normally, and you cannot die from old age.
      You become a planeforged native to an Aligned Plane matching your alignment.
      For details about the aligned planes, see the Grimoire of Guidance.
      While you are on that plane, you can teleport to any plane with your \textit{prayers} ability from this epic fate.
      In addition, you can use that teleportation ability once per hour instead of once per week.

      \parhead{Rank 5 -- Deification} You become a deity.
      You are transported to an Aligned Plane matching your alignment, and you gain divine dominion over an amount of territory in that plane.
      While you are in your territory, you can can freely reshape your territory with a thought to match your desires, and you are immune to all damage and \glossterm{conditions}.

      Regardless of which plane you are on, you can teleport to anywhere within your home plane as a \glossterm{standard action}.
      In addition, there is no limit on the number of times you can teleport with your \textit{prayers} ability from this epic fate.

  \subsection{Hero of Legend}
    You are widely known as a hero, rescuing those in need.
    As your deeds of heroism spread, you gain abilities to help you protect others.
    Although you will eventually die, your legend will live on, inspiring others to save people as you did.

    \parhead{Prerequisites} You must be publicly known to be involved with saving at least one major country or similarly large group of people from some sort of disaster to choose this epic fate.

    \parhead{Progression} To progress towards this epic fate, you must publicly contribute to saving large numbers of people from death or other major disasters in a way that builds your reputation.
    Reaching the higher ranks typically requires saving a significant fraction of a major plane from some sort of catastrophe.

    \subsubsection{Hero of Legend Ranks}
      \parhead{Rank 1 -- Worthy Hero} You and all \glossterm{allies} who can see or hear you are immune to being frightened and panicked.
      In addition, you gain a \plus100 bonus to your maximum \glossterm{hit points}.

      \parhead{Rank 2 -- Heroic Intervention} As a \glossterm{free action}, you may choose any number of \glossterm{allies} within \shortrange of you.
      Whenever a chosen creature would be attacked that round, that attack is made against you instead.
      If the attack would have targeted both you and that ally, the attack only targets you once, not twice.
      This ability has the \abilitytag{Swift} tag.

      \parhead{Rank 3 -- Invincible Hero} You gain a \plus4 bonus to all defenses.
      In addition, you cannot be \vulnerable for any reason.

      \parhead{Rank 4 -- Answer the Call} You gain an intuitive sense for when people need your aid.
      Whenever someone on the same plane as you is in danger, you are aware of the existence of that danger.
      You can sense the general category of danger (fire, combat, drowning, etc.) and a very approximate direction and distance.
      This generally allows you to sense if a large number of people are in danger from the same thing.
      As a \glossterm{standard action}, you can teleport any distance within that plane to reach a person in danger.

      \parhead{Rank 5 -- Heroic Legacy} If you die, your legend lives on.
      You may choose a worthy successor, either before your death or after your death from your afterlife plane.
      When you die, or as soon as you choose a successor while dead, your successor immediately gains the Rank 1 benefit from this archetype.
      As long as your successor lives and remains worthy, you cannot choose to be resurrected from your afterlife.
      As a player, you may choose to play as your successor instead of your original character if the GM allows it.

  \subsection{Mutant}
    Your body has been altered by battle scars and strange experiments.
    As your mutations grow ever more extreme, you become more powerful - and more monstrous.
    Eventually, you can regenerate from death itself, though some scars never fade.

    \parhead{Prerequisites} Dangerous experiments to mutate your body in extreme ways must have been performed on you.
    You can choose the nature of these experiments, such as alchemical or magical.
    Some mutants do this to themselves, while others find willing collaborators.
    In addition, you must have a Constitution of at least 2.

    \parhead{Progression} To progress towards this epic fate, you must continue ever more radical forms of experimentation.
    As you become inured to ordinary alterations to your body, you must travel and research to find unique substances of immense power to fuel the experiments.
    To reach the higher ranks, you must undergo experiments that kill you far more often than they succeed, so you will need to be resurrected multiple times to continue down this path.

    \subsubsection{Mutant Ranks}
      \parhead{Rank 1 -- Unnatural Arsenal} You grow an extra functioning arm and hand.
      In addition, you gain a wide variety of natural weapons (see \tref{Natural Weapons}).
      You gain a bite, horn, ram, stinger, and tentacle.
      In addition, two of your hands become claws.
      You gain these weapons in addition to any natural weapons you already have, no matter how biologically implausible that may be.
      In addition, each of your natural weapons has the \weapontag{Sweeping} (3) weapon tag in addition to its other weapon tags.

      \parhead{Rank 2 -- Regeneration} At the end of each round, you regain hit points equal to a quarter of your maximum hit points.
      In addition, you may increase the result of one of your \glossterm{vital wounds} by 1, to a maximum of 10.
      If you are unconscious, this automatically applies to your most severe vital wounds first.

      \parhead{Rank 3 -- Monstrous Form} Your size increases by one size category.
      In addition, you gain a wide variety of movement modes.

      You gain a average \glossterm{climb speed}, average \glossterm{swim speed}, and a slow \glossterm{burrow speed}.
      Wings also grow from your back, granting you a average \glossterm{fly speed} with a 60 foot \glossterm{height limit} (see \pcref{Flight}).

      \parhead{Rank 4 -- Two Heads Are Better Than One} You grow a second head.
      Whenever you gain a \glossterm{condition}, you choose which head gains the condition, with the restriction that the chosen head must not already have more conditions than the other head.
      At the start of each round, you can choose which heads are active during that round.
      You are only subject to the effects of the conditions affecting active heads.
      If you choose for both heads to be active, you can use an extra \glossterm{minor action} during the \glossterm{action phase}.

      \parhead{Rank 5 -- Regenerative Immortality} You can regenerate from any wounds, even lethal ones.
      When you die, your \textit{unnatural regeneration} ability continues functioning.
      Once that ability improves your vital wounds so they are all above 0, you return to life.
      However, each time you die, you gain a new scar that your regeneration always recreates.

      If your corpse is mutilated, burned, immersed in acid, or fully destroyed, this process can take much longer to complete, but it cannot be fully stopped.
      Some drop of blood, flake of skin, or other remnant of your corpse will always persist and regenerate eventually.
      If your body is separated into pieces while you are dead, each piece will attempt to regenerate individually.
      Your soul will automatically return to the first piece that regenerates completely, at which point the remaining fragments will wither and die.

  \subsection{Paradox}
    You exist partly outside of the ordinary flow of time.
    Your very existence wreaks havoc on prophecies and the orderly sequence of events.
    As your alterations to the natural timeline of the universe grow in scope, your ability to bend time to your whims grows in turn.
    Eventually, you become a fixed point across all of time.

    \parhead{Prerequisites} You must have been directly involved with an action that resulted in a significant change to at least one major country or similarly significant entity.
    Any type of change is acceptable, as long as it would be historically important would not have happened without your intervention.

    \parhead{Progression} To progress towards this epic fate, you must be alter the course of other major events that will be remembered to history.
    Reaching the higher ranks typically requires changing the fate of major planes, or creatures of similar importance.

    \subsubsection{Paradox Ranks}
      \parhead{Rank 1 -- Temporal Aberration} Your actions, and events involving you, cannot be observed in any effect that sees or predicts the future.
      This applies against both magical abilities and abilities that rely on direct observation.
      Prophecies are only able to describe how events would happen without your intervention, and are blind to any changes you might cause.

      In addition, whenever you would make a \glossterm{movement}, you can make two different movements and then decide which one was the movement you actually made.
      You can make this decision after observing how other creatures react to your movement, but before taking any other actions.

      The other movement never happened, and had no effect.
      Only the movement itself is reverted in this way.
      Any other abilities you used during the resolution of that movement, such as the \ability{sprint} ability, still happened, so you would still gain fatigue and resolve other effects.

      \parhead{Rank 2 -- A Fork In Time's Road} Whenever you would take a standard action, you can take two different standard actions and then decide which one was the action you actually took.
      You can make this decision after seeing all die results and observing all effects of both actions, but before taking any other actions.

      The other action never happened, and has no effect.
      Only the standard action itself is reverted in this way.
      Any other abilities you used during the resolution of that action, such as the \ability{desperate exertion} ability, still happened, so you would still gain fatigue and resolve other effects.

      \parhead{Rank 3 -- Choose Fate} Once per \glossterm{short rest}, whenever you or any other creature you are aware of rolls an attack or check, you can choose the result of that die.
      You can choose to use this ability after learning the result of the action using that die roll, including whether it succeeded or failed and the result of any damage dice based on the attack.
      However, you must use it before any other actions resolve.
      If you use this to make an attack \glossterm{explode}, subsequent dice after the die you modify in this way are rolled normally.
      Using this to affect an enemy's action may change the actions taken by other enemies in that enemy's \glossterm{allied group}, which the GM should resolve.

      \parhead{Rank 4 -- Paradoxical Defense} Whenever a creature attacks you, it must \glossterm{reroll} the attack roll once and keep the lower result.
      This does not protect any other targets of the attack.

      \parhead{Rank 5 -- Fixed Point} You become an immutable fact across all of time.
      You no longer age.
      Whenever you die, history is rewritten so you retroactively never died instead.
      The changes are as subtle and believable as possible, but even extraordinary coincidences can occur to save you from death.
      You typically still end up unconscious from vital wounds, and are always removed from combat or otherwise unable to usefully act for at least ten minutes, but you survive.

  \subsection{Slayer}
    You are a killer of legendary skill.
    As your body count increases, you gain abilities to help you track down and kill increasingly powerful foes.
    Eventually, your powers threaten the gods themselves, allowing you a unique ability to transcend death.

    \parhead{Prerequisites} You must be directly involved with slaying at least one \glossterm{elite} creature with a level of at least 21.

    \parhead{Progression} To progress towards this epic fate, you must publicly contribute to slaying increasingly dangerous and fearsome foes.
    To reach the higher ranks, you must kill creatures of singular power whose influence is felt across multiple planes.
    This might include demon princes, supreme dragons beyond even the power of wyrms, or the nightmarish precursor aberrations in the Eternal Void.

    \subsubsection{Slayer Ranks}
      \parhead{Rank 1 -- Lethality} You gain a \plus4 bonus to your accuracy for the purpose of determining whether you get a \glossterm{critical hit}.
      This bonus stacks with other abilities with the same effect, such as \weapontag{Keen} weapons.

      \parhead{Rank 2 -- Precision Killer} You gain a \plus4 bonus to your \glossterm{accuracy}.
      In addition, you can inflict \glossterm{critical hits} on any creature regardless of its body structure, magic items, or other abilities.

      \parhead{Rank 3 -- Mark of the Slayer} As a \glossterm{minor action}, you can choose to mark any creature you can unambiguously identify.
      This includes any creature you can see, as well as any creature you know the name of and can differentiate from other similar creatures.
      You can only mark one creature at a time, and applying a new mark replaces any previous mark.
      You cannot use this ability to replace a mark that is less than a week old if the recipient of the previous mark still lives.

      This mark is visible on the creature's body with a design that is recognizably yours.
      It appears on top of any clothing or other attempt to conceal it, even if the creature is invisible.
      Anyone can recognize the significance of the mark with a \glossterm{difficulty value} 15 Knowledge (arcana or local) check, and creatures that understand the significance of the mark may refuse to give your target aid of any kind to avoid risking your wrath.

      You know the exact distance and direction to any creature you have marked with this ability that is on the same plane as you.
      As a \glossterm{standard action}, you can create a \glossterm{scrying sensor} adjacent to them that you can see and hear through.
      The sensor lasts as long as you \glossterm{sustain} it as a \glossterm{free action}.
      It moves to stay adjacent to the target, regardless of its speed.

      \parhead{Rank 4 -- Slayer's Journey} As a \glossterm{standard action}, you can \glossterm{teleport} yourself and up to ten \glossterm{allies} any distance within the same plane to the location of a creature affected by your \textit{mark of the slayer} ability from this epic fate.
      You cannot precisely choose the destination of this ability, and it does not leave you immediately adjacent to the marked creature.
      Generally, it leaves you just outside any sort of fortress or defenses the marked creature has constructed.
      After you use this ability, you cannot use it to travel to the same creature for a day.
      This does not limit your ability to travel to a different creature if you mark a different creature.

      \parhead{Rank 5 -- Godslayer}
      Your damaging attacks ignore all forms of invulnerability and immunity.
      You can punch ghosts, set fire to fire elementals, and so on.
      In addition, you can destroy artifacts and even inflict damage on deities in their divine dominion.
      As a result, even deities fear to interfere with you directly.
      If you ever die, you can generally threaten or fight your way past any planar guardians to leave your afterlife whenever you want.
      After you do this once, you become a planeforged native to your afterlife plane, since your new body is formed from the raw material of that plane.

\section{Classes}
  \subsection{Anointed}
    An anointed is a votive who made their pact with a deity.
    This is an unusual arrangement, as deities would normally influence their clerics to achieve their aims.
    However, in special circumstances, a deity may want to empower a non-worshipper to influence mortal affairs.
    The anointed class functions like the votive class, with the following exceptions:
    \begin{itemize}
      \item The magic source for the anointed class is divine magic instead of pact magic.
        This changes the \glossterm{mystic spheres} an anointed has access to and all other effects based on their source of magic.
        However, they still require both \glossterm{verbal components} and \glossterm{somatic components} to cast spells from the anointed class (see \pcref{Ability Usage Components}).
      \item They must choose either \sphere{channel divinity} or \sphere{prayer} as one of their mystic spheres, replacing the normal list from the Soulkeeper Spheres ability.
      \item An anointed cannot choose the \textit{blessings of the abyss} archetype. However, the \textit{domain influence} cleric archetype is considered to be part of their class, and they may choose that archetype without spending insight points to multiclass.
    \end{itemize}

  \subsection{Bard}
    A bard is a rogue with the ability to perform magical feats through music.
    It is unclear whether bards actually draw power from music in the same way that druids draw power from nature, or whether they simply channel their innate magical talent through music.
    The bard class functions like the rogue class, with the following exceptions:
    \begin{itemize}
      \item A bard cannot choose the \textit{assassin} archetype. However, the \textit{arcane magic} sorcerer archetype is considered to be part of their class, and they may choose that archetype without spending insight points to multiclass.
      \item A bard casts spells without \glossterm{somatic components}.
      \item A bard can only cast spells while sustaining a performance with the Perform skill. This performance can be either a mundane performance or a \textit{bardic performance} ability.
    \end{itemize}

  \subsection{Blighter}
    Blighter practice a strange inversion of druidic traditions.
    While druids venerate nature in all its forms, blighters dedicate their lives to the destruction of nature for its own sake.
    They rip power directly from the death of natural beings, using it to fuel their own warped version of nature magic.
    The blighter class functions like the druid class, with the following exceptions:
    \begin{itemize}
      \item Whenever a blighter rests, they automatically destroy nature and kill anything living around them.
        Plants wither and die, insects fall dead in the air, and so on.
        A ten minute rest destroys life in a radius equal to five feet times the blighter's highest rank in the blighter class (minimum 5 feet total).
        In general, Diminutive or larger creatures and Medium or larger plants suffer no ill effects, though creatures may feel subtle pains.
        An eight hour rest destroys life in ten times that radius, and kills life one size category larger.
        Resting beyond that point does not increase the radius or severity of the effect.
        This destruction spreads out gradually throughout the resting period, and even a partially completed rest destroys some natural life.
      \item A blighter cannot choose the \textit{wildspeaker} archetype.
        However, the \textit{domain influence} cleric archetype is considered to be part of their class, and they may choose that archetype without spending insight points to multiclass.
        A blighter can only choose the Death, Destruction, and Evil domains.
      \item A blighter cannot gain access to the \sphere{verdamancy} mystic sphere by any means.
    \end{itemize}

  \subsection{Shaman}
    A shaman, like a cleric, is a divine worshipper.
    However, while clerics worship powerful, well-established deities, shamans worship more primitive deities of lesser power.
    As a result, their divine powers are more limited and take different forms.
    Shamans are common among less civilized humanoid societies like bugbears.
    The shaman class functions like the cleric class, with the following exceptions:
    \begin{itemize}
      \item The magic source for the shaman class is nature magic instead of divine magic.
        This changes the \glossterm{mystic spheres} a shaman has access to and all other effects based on their source of magic.
      \item A shaman cannot choose the \textit{divine spell mastery} archetype. However, the \textit{elementalist} druid archetype is considered to be part of their class, and they may choose that archetype without spending insight points to multiclass.
      \item A shaman cannot gain access to more than two \textit{mystic spheres} from the magic source granted by the shaman class by any means.
      \item Shamans add Knowledge (nature) to their class skill list and remove Knowledge (planes).
    \end{itemize}

\section{Alternate Play Styles}

  \subsection{Being Surrounded}\label{Being Surrounded}
    Normally, exact positioning doesn't matter that much in combat.
    This makes it easier to play without a grid, or to just spend less time worrying about the details of everyone's positions on a grid.
    With this optional rule, you can make positioning more important in combat, increasing tactical depth for melee characters.
    This generally has the downside of making movement more complicated, however, as combatants try to surround others and avoid being surrounded themselves.

    If you play with this alternate rule, when you are being attacked by multiple foes at once, you are less able to defend yourself.
    If every space adjacent to you either contains an \glossterm{enemy} or is adjacent to an \glossterm{enemy}, you are surrounded.
    A creature that is surrounded takes a \minus2 penalty to its Armor and Reflex defenses.
    When determining whether you are surrounded, ignore any enemies that are sharing space with you, and ignore any enemies that are at least two size categories smaller than you.

    Any effect that makes a creature immune to being \partiallyunaware, such as the \spell{foresight} spell, also makes that creature immune to being surrounded.

  \subsection{Complex Cover}
    Normally, cover is a binary effect.
    You either have cover or you don't.
    If you want to make cover and tactical positioning more important, you can use more subtle variations of cover.
    This variant uses four cover variants:
    \begin{itemize}
      \item Quarter cover: You have approximately a quarter of your body behind an obstacle.
        Quarter cover grants no defense bonuses.
        However, if you would be affected by a \glossterm{glancing blow} or an attack that deal half damage on a miss, the obstacle takes damage instead of you.
      \item Half cover: You have approximately half of your body behind an obstacle.
        Half cover grants a \plus2 bonus to Armor and Reflex defenses.
        It also grants the same protection as quarter cover against glancing blows and misses.
      \item Three-quarters cover: You have approximately three quarters of your body behind an obstacle.
        Three-quarters cover grants a \plus4 bonus to Armor and Reflex defenses.
        It also grants the same protection as quarter cover against glancing blows and misses.
      \item Total cover: You have your entire body behind an obstacle.
        Total cover blocks \glossterm{line of effect}, which makes most attacks impossible (see \pcref{Line of Effect}).
    \end{itemize}

    Normally, asymmetric cover means that one target has cover while the other doesn't.
    With this variant, asymmetric cover will often mean that one target simply has better cover than the other, but cover is relevant for both targets.
    For example, hiding behind a tree might grant you half cover from a target, but they might also gain quarter cover from you.

    The downside of this optional rule is that it requires more ad-hoc rulings from the GM about subtle differences in the environment.
    To keep the pace of the game moving, players have to resist the urge to argue with potentially arbitrary rulings.
    Complex cover is generally only meaningful if you are already using battle maps rich with detail or improvising substantial elements of the environment on the fly.

  \subsection{Critical Failure}
    Normally, there is no explicit penalty for catastrophic failure built into the rules.
    Even if you fail at a check by a large amount, it doesn't leave you worse off than when you started.
    Sometimes, it may be narratively appropriate to punish significant failure more severely, at the GM's discretion.
    For example, attempting a difficult Persuasion check and completely botching the execution might leave the target feeling more hostile than if no Persuasion had been attempted at all.
    A good threshold for critical failure would generally be failing a check by 8 or more.
    For specific tasks, it may make more sense to have punishments for failure at lower thresholds as well.

    This is considered an optional rule because it generally makes trying silly ideas or extremely difficult tasks more dangerous, which isn't appropriate for every game.
    It also depends heavily on GM discretion.

  \subsection{Easy Magic Item Reforging}
    The Craft Specialization feat allows characters to transfer magic item properties between different items.
    For example, if the players find a magic meteor hammer that none of them could use, they could reforge that item as a magic battleaxe so they could use its property.
    With this optional rule, skilled item crafters capable of this action are assumed to be common in major cities or towns.
    The typical price to reforge an item in this way is two ranks lower than the item's rank, to a minimum rank of 1.

    The advantage of using this optional rule is that it makes magic items more likely to be useful to the party.
    Without this rule, you may be forced to have the party ``randomly'' only find magic items that they are coincidentally proficient with, or the party may frequently find magic items that they can't use.
    On the other hand, this rule assumes a more magical and highly developed civilization.
    It also may require the party to frequently return to town to reforge useless items into items that are useful for them.
    Either of those requirements may not match the intended tone of your campaign.

  \subsection{Expanded Insight Points}
    Normally, \glossterm{insight points} can only be used to learn new special abilities from your class, or from a small number of feats.
    This alternate rule allows you to spend insight points to gain a wide variety of other proficiencies and benefits.
    This makes character creation more complicated, but it also allows you to personalize your character much more precisely.

    If you play with this alternate rule, you can spend insight points in any of the following ways.
    \begin{raggeditemize}
      \item You can spend an \glossterm{insight point} to gain an additional \glossterm{trained skill}.
      \item You can spend an \glossterm{insight point} to gain proficiency in an additional \glossterm{usage class} of armor (light, medium, or heavy).
        You must be proficient with light armor to become proficient with medium armor, and you must be proficient with medium armor to become proficient with heavy armor.
      \item You can spend an \glossterm{insight point} to gain proficiency in all non-exotic weapons from one \glossterm{weapon group} (see \pcref{Weapon Groups}).
      \item You can spend an \glossterm{insight point} to learn two \glossterm{common languages} or one \glossterm{rare language} (see \pcref{Communication and Languages}).
    \end{raggeditemize}

  \subsection{Exploding Checks}
    Normally, \glossterm{checks} do not \glossterm{explode}.
    There are practical issues with allowing explosions on retryable actions.
    In theory, a character could simply try an otherwise impossible check a thousand times to guarantee a sufficiently high result on the exploding die.
    With this optional rule, checks can explode, but only in tense or time-limited situations where the check is not indefinitely retryable.
    Narratively, this could represent adrenaline helping people reach superhuman feats in times of stress.
    This requires GM interaction to identify situations where checks can be reasonably retried.
    For example, even a check that is normally indefinitely retryable, such as unlocking a door, could explode if unlocking the door quickly was important in a combat situation.

  \subsection{Longer Rests}\label{Longer Rests}
    Normally, characters can take a short rest in ten minutes and a long rest in eight hours.
    With this optional rule, a short rest instead requires eight hours of rest, and long rest requires a week.

    This dramatically slows down the narrative pacing of the world, and makes the world feel much more brutal and unforgiving.
    Characters will often be forced to start a combat while missing damage resistance or even hit points, and taking vital wounds can be crippling.

  \subsection{Obscure Magic Items}\label{Obscure Magic Items}
    The base rules of Rise make it fairly easy to identify magic items.
    This keeps the pace of the game up when players find magic items frequently.
    However, you may choose to treat magic items as being more rare and mysterious.
    If you do, make the following changes:
    \begin{itemize}
      \item The \ability{identify item} ability from the Craft and Knowledge skills provides no information about how to use a magic item's properties or what they might be.
        It can still be used to identify whether or not an item is magical.
      \item The Knowledge (items) Knowledge skill is removed entirely.
      \item Magic items are more rare, and therefore more valuable.
        Calculate the prices for all magic items as if they were one rank higher than they actually are.
        Rank 7 magic items cannot be bought for any price - they are simply too rare.
      \item All spells with the \abilitytag{Attune} tag require an additional \glossterm{attunement point} to attune to.
        If magic items are hard to find and use, spellcasters gain a powerful benefit, since their personal attunement spells are still reliably available.
        This change ensures that spellcasters still gain a benefit from their personal access to magic, but they are not drastically more powerful than characters who depend on finding useful magic items.
    \end{itemize}

    You may also want to add complex or unintuitive activation conditions to magic items.
    For example, \mitem{boots of speed} may only function while hopping on one foot, or while you are not wearing socks.
    This can encourage players to experiment more with magic items to figure out how to use them.

  \subsection{Rage Accuracy}
    Normally, a barbarian's \ability{rage} ability provides a +2 accuracy bonus.
    With this variant, raging barbarians instead gain no accuracy bonus, but roll 1d12 instead of 1d10 for their attack rolls while raging.
    They also \glossterm{explode} on a 9 or higher on the first roll, though subsequent rolls must roll a 12 to continue exploding.
    A barbarian's overall accuracy and damage output with this rule essentially equivalent to the normal rule, but they are more likely to get critical hits or completely whiff on important attacks.

    This variant can be more fun for people who like big hits and big misses, and for RPG veterans who naturally associate a barbarian with a d12.
    It is considered a variant rule because not everyone owns a 12-sided die, and you shouldn't need to buy one just to play a barbarian.

  \subsection{Restricted Archetype Order}
    Normally, when a character in Rise levels up, they can freely choose which of their class archetypes they want to rank up (as long as they don't exceed their maximum rank).
    However, this means that most levels require making a choice that may be confusing for newer players.
    The process of leveling up can be simplified if each player chooses an order for their archetypes.

    With this variant, each character has a primary archetype, a secondary archetype, and a tertiary archetype.
    This choice is made at character creation.
    Whenever they increase their maximum rank, they increase their rank in their primary archetype.
    In their next level up, they increase their rank in their secondary archetype, and then finally their tertiary archetype.

  \subsection{Sleeping While Encumbered}
    Normally, characters can sleep in their armor without any penalty.
    This is unrealistic, but it can be time-consuming to make everyone track how their sleeping statistics differ from their waking statistics.
    Being ambushed while sleeping is very rare in most games, so it's generally not worth the hassle.
    However, if you want a more realistic game with more punishing night ambushes, you can use this alternate rule.

    If you play with this alternate rule, resting in armor is difficult.
    If you take a \glossterm{long rest} while you have \glossterm{encumbrance}, you finish your rest with a \glossterm{fatigue level} equal to the value of your encumbrance.
    In addition, only half the time you spend sleeping while you have encumbrance counts as sleep for the purpose of determining your fatigue (see \pcref{Sleep and Fatigue}).

  \subsection{Tap Out}
    With this optional rule, whenever you gain a vital wound, you can ``tap out'' to guarantee that you survive while taking yourself out of the fight.
    If you tap out, you treat the result of the vital roll for that vital wound as a 10, regardless of any bonuses or penalties you would normally have to the vital roll.
    However, you fall unconscious immediately, and you cannot regain consciousness by any means until you finish a \glossterm{short rest}.

    This optional rule significantly reduces the likelihood of character death, and makes fights less likely to impose long-term consequences on characters.
    However, it also makes vital wounds more likely to entirely knock characters out of a fight, which can increase the risk that the entire party is defeated.
