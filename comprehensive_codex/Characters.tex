\chapter{Characters}\label{Characters}

This chapter describes how individual characters in Rise are defined, including their statistics.
For context about how characters act in combat, see \pcref{Combat}.
For context about how characters act more generally, see \pcref{Adventuring}.

Some of the information in this chapter won't fully make sense until you've read future chapters.
You can either skim past terms you don't yet understand or look them up as you go along.

\section{Attributes}\label{Attributes}

  Attributes represent raw talent in a particular area.
  Each character has six \glossterm{attributes}: Strength (Str), Dexterity (Dex), Constitution (Con), Intelligence (Int), Perception (Per), and Willpower (Wil).
  These attributes have a wide range of effects on a character's core statistics.

  A 0 in an attribute represents average human capability.
  That doesn't mean that every commoner has a 0 in every attribute; not everyone is average, after all.
  Player characters have an average attribute higher than 0 because they are exceptional individuals.
  For details about determining your attributes, see Step 4 of Character creation, \pref{Step 4: Attributes}.

  Your attributes can increase after character creation, such as by reaching level 3 (see \pcref{Character Advancement and Gaining Levels}).
  Increasing your attributes generally changes all of their effects appropriately.

  \subsection{Strength (Str)}\label{Strength}
    {
      Strength measures your muscle and physical power.
      Characters with a high Strength tend to have strong offensive capabilities with nonmagical abilities, and prefer wearing heavier armor.
      It has the following effects:
      \begin{raggeditemize}
        \item Strength determines how much you can carry (see \pcref{Weight Limits}).
          You generally need a Strength of at least 2 to wear heavy body armor.
        \item You add your Strength to your Brawn defense.
        \item You add your Strength to your \glossterm{mundane power} (see \pcref{Power}).
        \item You add your Strength to your level to determine your \glossterm{brawling accuracy} (see \pcref{Brawling Accuracy}).
        \item Your Strength improves Strength-based \glossterm{skills}: Climb, Jump, and Swim (see \pcref{Skills}).
        % \item You need at least 3 Strength to use \weapontag{Heavy} weapons in one hand (see \pcref{Weapon Tags}).
      \end{raggeditemize}
    }

  \subsection{Dexterity (Dex)}\label{Dexterity}
    {
      Dexterity measures your hand-eye coordination, agility, and reflexes.
      Characters with a high Dexterity tend to have strong defensive capabilities, and prefer wearing lighter armor.
      It has the following effects:
      \begin{raggeditemize}
        \item You add your Dexterity to your Armor defense.
          This bonus is halved if you use medium or heavy body armor or shields (see \tref{Armor and Shields}).
        \item You add your Dexterity to your Reflex defense.
        \item You add your Dexterity to Dexterity-based \glossterm{skills}: Balance, Flexibility, Ride, Sleight of Hand, and Stealth (see \pcref{Skills}).
        \item You need at least 2 Dexterity to make \glossterm{dual strikes} (see \pcref{Dual Strikes}).
      \end{raggeditemize}
    }

  \subsection{Constitution (Con)}\label{Constitution}
    {
      Constitution represents your health and stamina.
      Characters with a high Constitution tend to have strong defensive capabilities.
      It has the following effects:
      \begin{raggeditemize}
        \item You add your Constitution to your \glossterm{durability}, which increases your \glossterm{hit points} (see \pcref{Hit Points}).
        \item You add your Constitution to your \glossterm{fatigue tolerance} (see \pcref{Fatigue}).
        \item You add your Constitution to your Fortitude defense.
        \item You add your Constitution to the Constitution-based \glossterm{skill}: Endurance (see \pcref{Skills}).
      \end{raggeditemize}
    }

  \subsection{Intelligence (Int)}\label{Intelligence}
    {
      Intelligence represents how well you learn and reason.
      Characters with a high Intelligence tend to have more options and special abilities.
      It has the following effects:

      \begin{raggeditemize}
        \item If your Intelligence is positive, you gain a number of \glossterm{trained skills} equal to your Intelligence (see \pcref{Trained Skills}).
        \item You add your Intelligence to the number of \glossterm{insight points} you gain (see \pcref{Insight Points}).
        \item You add your Intelligence to Intelligence-based \glossterm{skills}: Craft, Deduction, Disguise, Knowledge, and Medicine (see \pcref{Skills}).
      \end{raggeditemize}

      \par Creatures incapable of complex cognition and speech, like animals, have an Intelligence of \minus6 or lower.
      Creatures capable of speech have an Intelligence of at least \minus5.
    }

    \subsubsection{Changing Intelligence}\label{Changing Intelligence}
      When your Intelligence permanently increases, you also permanently gain an additional insight point and trained skill.
      Temporary Intelligence modifiers, such as from magic items, do not affect your insight points or trained skills.

  \subsection{Perception (Per)}\label{Perception}
    {
      Perception describes your ability to observe and be aware of your surroundings.
      Characters with a high Perception tend to have strong offensive capabilities.
      It has the following effects:
      \begin{raggeditemize}
        \item You add your Perception to your level to determine your \glossterm{accuracy} with almost all attacks (see \pcref{Accuracy}).
        \item You add your Perception to Perception-based \glossterm{skills}: Awareness, Creature Handling, Social Insight, and Survival (see \pcref{Skills}).
      \end{raggeditemize}
    }

  \subsection{Willpower (Wil)}\label{Willpower}
    {
      Willpower represents your ability to endure mental hardships.
      Characters with a high Willpower tend to be better at attacking with and defending against magical abilities.
      It has the following effects:
      \begin{raggeditemize}
        \item You add your Willpower to your \glossterm{magical power} (see \pcref{Power}).
        \item You add your Willpower to your Mental defense.
      \end{raggeditemize}
    }

\section{Combat Statistics}\label{Combat Statistics}

  \subsection{Accuracy}\label{Accuracy}
    Your accuracy with an \glossterm{attack} is the number that you add to the \glossterm{attack roll} (see \pcref{Attack Rolls}).
    Your accuracy with most abilities is normally equal to half the sum of your level and Perception.
    Many abilities can also modify your accuracy.

    \subsubsection{Brawling Accuracy}\label{Brawling Accuracy}
      Some abilities cause you to make a \glossterm{brawling attack}, which uses your \glossterm{brawling accuracy} in place of your regular accuracy.
      Your brawling accuracy is normally equal to half the sum of your level and Strength.
      Any ability that modifies your overall \glossterm{accuracy} also modifies your brawling accuracy.
      However, some abilities only affect your brawling accuracy.

      Abilities that make brawling attacks generally have the \atBrawling tag.
      Commonly used Brawling abilities are listed in \pcref{Universal Combat Abilities}.

  \subsection{Defenses}\label{Defenses}
    When you are attacked, your defenses determine the value that the attacker needs to get on the attack roll in order to hit you (see \pcref{Attack Rolls}).
    The five defenses are described below.
    \begin{raggeditemize}
      \item Armor defense (AD): Your Armor defense protects you from normal physical attacks, such as attempts to hit you with a sword.
        It is the most commonly used defense.
      \item Brawn defense: Your Brawn protects you from attacks that physically restrain, relocate, or otherwise control you, such as grappling and shoving.
      \item Reflex defense: Your Reflex protects you from attacks that you can dodge or evade, such as many area attacks.
      \item Fortitude defense: Your Fortitude defense protects you from attacks against your body or life, such as poisons and life-draining magic.
      \item Mental defense: Your Mental defense protects you from attacks against your mind, such as terrifying creatures and magical mind manipulation.
    \end{raggeditemize}

    Your defenses are calculated in the following way:
    \begin{raggeditemize}
      \itemhead{Armor} Half level \add Dexterity (modified depending on equipped armor) \add base class defense bonus (if any) \add defense bonuses from equipped body armor and shield
      \itemhead{Brawn} Half level \add Strength \add base class defense bonus
      \itemhead{Fortitude} Half level \add Constitution \add base class defense bonus
      \itemhead{Reflex} Half level \add Dexterity \add base class defense bonus \add defense bonuses from equipped shield
      \itemhead{Mental} Half level \add Willpower \add base class defense bonus
    \end{raggeditemize}
    Each defense may also have various bonuses or penalties applied by special abilities.
    Your defenses can never go below 0, no matter how many penalties you have.

  \subsection{Durability}\label{Durability}
    Your \glossterm{durability} is used to calculate your \glossterm{hit points}, as described below in Hit Points.
    It is equal to your Constitution \add your body armor durability bonus (see \pcref{Armor}).
    As your level increases, you gain a bonus to your durability (see \tref{Character Advancement and Gaining Levels}).

  \subsection{Encumbrance}\label{Encumbrance}
    Your encumbrance is a value that represents how much you are burdened by your armor (see \pcref{Armor}).
    You apply your encumbrance as a penalty to all Dexterity-based checks and Strength-based skill checks you make.
    It does not affect direct Strength checks.

  \subsection{Hit Points}\label{Hit Points}
    Your \glossterm{hit points} measure how much damage you can take before you die.
    They are often abbreviated as HP.
    When you take damage, it reduces your hit points (see \pcref{Taking Damage}).
    When your hit points drop below 0, you gain \glossterm{vital wounds}, which can eventually kill you (see \pcref{Vital Wounds}).

    Your hit points depend on your \glossterm{character rank}, as shown below.
    \begin{columntable}
      \begin{dtabularx}{\columnwidth}{l l X}
        \tb{Rank} & \tb{Flat HP} & \tb{Bonus HP} \\
        1 & 10 & 1 \x \glossterm{durability} \\
        2 & 10 & 2 \x \glossterm{durability} \\
        3 & 10 & 3 \x \glossterm{durability} \\
        4 & 10 & 4 \x \glossterm{durability} \\
        5 & 10 & 6 \x \glossterm{durability} \\
        6 & 10 & 8 \x \glossterm{durability} \\
        7 & 10 & 10 \x \glossterm{durability} \\
      \end{dtabularx}
    \end{columntable}

    Your maximum hit points can never be less than 1.
    If your maximum hit points would be reduced to 0 or less, you immediately die.

    \parhead{What Hit Points Represent} Hit points represent a combination of resilience, luck, divine providence, and sheer determination, depending on the nature of the creature being damaged.
    When lose hit points from an orc with a greataxe, the axe did not literally carve into your skin without affecting your ability to fight.
    Instead, you avoided the worst of the blow, but it bruised you through your armor, the effort to dodge the blow fatigued you, or it barely nicked you through sheer luck -- and everyone's luck runs out eventually.

  \subsection{Injury Point}\label{Injury Point}
    While your \glossterm{hit points} are at or below your \glossterm{injury point}, you are \glossterm{injured}.
    Being injured does not directly penalize you, but some attacks are much more effective if you are injured.
    Your injury point can be abbreviated as IP.

    Your injury point depends on your \glossterm{character rank}, as shown below.
    \begin{columntable}
      \begin{dtabularx}{\columnwidth}{l l X}
        \tb{Rank} & \tb{Flat IP} & \tb{Bonus IP}                                                        \\
        1         & 10           & Half the sum of level \add Constitution                              \\
        2         & 10           & Level \add Constitution                                              \\
        3         & 10           & Level \add Constitution \add half the sum of level \add Constitution \\
        4         & 10           & 2 \x (level \add Constitution)                                       \\
        5         & 10           & 3 \x (level \add Constitution)                                       \\
        6         & 10           & 4 \x (level \add Constitution)                                       \\
        7         & 10           & 5 \x (level \add Constitution)                                       \\
      \end{dtabularx}
    \end{columntable}

    Some special abilities can also modify your injury point.

  \subsection{Power}\label{Power}
    Your \glossterm{power} affects how much damage you deal.
    Some abilities also use it for other effects, such as healing.
    You have two types of power: your \glossterm{magical power}, which affects your \magical abilities, and your \glossterm{mundane power}, which affects your \glossterm{mundane} abilities (see \pcref{Magical and Mundane Abilities}).
    If you gain a bonus to your \glossterm{power}, and it does not specify which type of power it affects, it affects both your \glossterm{magical power} and your \glossterm{mundane power}.

    Your mundane power is equal to your Strength \add half your level.
    Similarly, your magical power is equal to your Willpower \add half your level.

    Many abilities have stronger effects depending on your \glossterm{power}, especially damaging abilities.
    % TODO: weapon damage wording
    For example, you gain a bonus to your \glossterm{weapon damage} equal to half your \glossterm{power} (see \pcref{Weapon Damage}).

\section{Resources}\label{Resources}

  \subsection{Attunement Points}\label{Attunement Points}
    Many special abilities and magic items only function as long as a creature attunes to them.
    Attuning to an ability requires investing least one \glossterm{attunement point}.
    For details, see \pcref{Attuned Abilities}.

    Your attunement points are defined by your \glossterm{base class}.
    Some classes and abilities can grant additional attunement points.

  \subsection{Fatigue}\label{Fatigue}
    Thoughout the day, you can become fatigued by your exertions both in and out of combat.
    While \glossterm{hit points} are easy to restore, reducing your \glossterm{fatigue level} generally requires a \glossterm{long rest}.
    Fatigue is still easier to recover from than \glossterm{vital wounds}.

    \subsubsection{Fatigue Level}\label{Fatigue Level}
      Your \glossterm{fatigue level} measures how tired you are.
      Some abilities increase your fatigue level when used, such as the \ability{desperate exertion} ability (see \pcref{Universal Abilities}).

    \subsubsection{Fatigue Tolerance}\label{Fatigue Tolerance}
      Becoming slightly fatigued is not immediately detrimental.
      Your fatigue level can be as high as your \glossterm{fatigue tolerance} before you suffer penalties.
      Your fatigue tolerance is equal to a bonus from your \glossterm{base class} \add your Constitution.
      Some abilities can increase your fatigue tolerance.

    \subsubsection{Fatigue Penalty}\label{Fatigue Penalty}
      You take a penalty to \glossterm{accuracy} and \glossterm{checks} equal your \glossterm{fatigue level} \sub your \glossterm{fatigue tolerance} (minimum 0).
      This penalty is called your \glossterm{fatigue penalty}.

    \subsubsection{Exhaustion}\label{Exhaustion}
      When your \glossterm{fatigue penalty} reaches \minus5, you fall \unconscious until your fatigue penalty is reduced below \minus5.
      Generally, this means that you are unconscious for 8 hours.

    \subsubsection{Recovering From Fatigue}
      When you finish a \glossterm{long rest}, your \glossterm{fatigue level} is restored to 0 (see \pcref{Resting}).
      There are no other ways to reduce your fatigue level.

    \subsubsection{Paying Fatigue Costs}
      Some abilities indicate that they cost a certain number of fatigue levels.
      That means that you increase your fatigue level by the given amount after the ability resolves.
      This means you do not suffer a \glossterm{fatigue penalty} from that extra fatigue while using the ability.
      You can even use abilities that cause you to drop unconscious from fatigue.

  \subsection{Insight Points}\label{Insight Points}
    You can spend \glossterm{insight points} to gain new special abilities.
    % Clarify that negative Int offsets the insight points you gain from level?
    Your insight points are defined by your \glossterm{base class}.
    Some classes and abilities can also grant insight points.
    If your base insight point total is negative, it offsets any insight points you would gain from other sources.

    Every class has at least one way to spend \glossterm{insight points} to learn additional abilities.
    These options are listed below.
    \begin{raggeditemize}
      \item Barbarian: Combat styles, exotic weapons, and maneuvers
      \item Cleric: Mystic spheres and spells
      \item Druid: Mystic spheres, spells, and wild aspects
      \item Fighter: Battle tactics, combat styles, exotic weapons, and maneuvers
      \item Monk: Combat styles, exotic weapons, ki manifestations, and maneuvers
      \item Paladin: Mystic spheres and spells
      \item Ranger: Combat styles, exotic weapons, hunting styles, know your enemy, and maneuvers
      \item Rogue: Bardic performances, combat styles, exotic weapons, maneuvers, and trained skills
      \item Sorcerer: Mystic spheres and spells
      \item Votive: Combat styles, spells, maneuvers
      \item Wizard: Alchemical discoveries, metamagic, mystic spheres, portable workshop, scholastic insights, and spells
    \end{raggeditemize}

    In addition, you can spend one \glossterm{insight point} to become a \glossterm{multiclass} character (see \pcref{Multiclass Characters}).

  \subsection{Trained Skills}\label{Trained Skills}
    You are trained in certain skills, which increases your bonus with those skills (see \pcref{Skills}).
    Your \glossterm{base class} grants you a certain number of \glossterm{trained skills} from among the \glossterm{class skills} for that class.
    Your Intelligence grants you additional trained skills that do not have to be class skills for you.
    When you gain trained skills by any other means, you can choose any skill, not just your class skills.
    Some abilities can also grant additional trained skills.

\section{Size and Weight}\label{Size and Weight}

  \subsection{Size Categories}\label{Size Categories}
    Your size affects your \glossterm{space} in combat, your \glossterm{base speed}, your attributes, and how noticeable you are (see \pcref{Stealth}).
    These effects are shown on \trefnp{Size Categories}.
    Size categories are also relevant for determining weight (see \pcref{Weight Limits}).

    \begin{dtable*}
      \lcaption{Size Categories}
      \begin{dtabularx}{\textwidth}{l l l l l l l l X}
        \tb{Size}  & \tb{Space}\fn{1} & \tb{Base Speed} & \tb{Weight Limits}\fn{2} & \tb{Brawn} & \tb{Reflex} & \tb{Stealth} & \tb{Weapons}             & \tb{Example Creature} \tableheaderrule
        Fine       & 1/4 ft.          & 10 ft.          & \minus4 Str              & \minus4    & \plus4      & \plus16      & \tdash                   & Fly                   \\
        Diminutive & 1/2 ft.          & 10 ft.          & \minus3 Str              & \minus3    & \plus3      & \plus12      & \tdash                   & Mouse                 \\
        Tiny       & 1 ft.            & 20 ft.          & \minus2 Str              & \minus2    & \plus2      & \plus8      & \tdash                   & Rat                   \\
        Small      & 2-1/2 ft.        & 20 ft.          & \minus1 Str              & \minus1    & \plus1      & \plus4       & \tdash                   & Cat                   \\
        Medium     & 5 ft.            & 30 ft.          & \tdash                   & \tdash     & \tdash      & \tdash       & \tdash                   & Human                 \\
        Large      & 10 ft.           & 40 ft.          & \plus1 Str               & \plus1     & \minus1     & \minus4      & \tdash                   & Ogre                  \\
        Huge       & 20 ft.           & 50 ft.          & \plus2 Str               & \plus2     & \minus2     & \minus8     & \weapontag{Sweeping} (1) & Hill giant            \\
        Gargantuan & 40 ft.           & 60 ft.          & \plus3 Str               & \plus3     & \minus3     & \minus12     & \weapontag{Sweeping} (2) & Roc                   \\
        Colossal   & 80\add ft.       & 80 ft.          & \plus4 Str               & \plus4     & \minus4     & \minus16     & \weapontag{Sweeping} (4) & Great wyrm red dragon \\
      \end{dtabularx}
      1 Creatures can vary in space. These are simply typical values. \\
      2 This modifies Strength only for the purpose of determining a creature's \glossterm{weight limits} (see \pcref{Weight Limits}). \\
    \end{dtable*}

    \subsubsection{Space}\label{Space}
      A creature's \glossterm{space} is the area it occupies in combat.
      Humanoids typically occupy a 5-ft. by 5-ft. \glossterm{square}.
      A creature's space is generally larger than its physical body, especially for large creatures.

    \subsubsection{Base Speed}\label{Base Speed}
      Each size category has a \glossterm{base speed} that indicates how far creatures of that size category can generally move.
      For details about movement in combat, see \pcref{Movement and Positioning}.

    \subsubsection{Other Effects}
      A creature's size affects some additional skills and abilities.
      For example, larger creatures are immune to \atSizeBased abilities used by creatures two or more size categories smaller than them.
      Some special cases are mentioned below.

    \subsubsection{Very Small Creatures}
      \parhead{Space} If a creature takes up less than a single square of space, you can fit multiple creatures in that square.
      Ignoring flight, you can fit four Small creatures in a square, twenty-five Tiny creatures, 100 Diminutive creatures, or 400 Fine creatures.
      If the creatures can fly, the number of creatures that can fit into a space increases drastically.

      \parhead{Movement} Creatures two size categories smaller than you are not considered obstacles and do not hinder your movement.

    \subsubsection{Very Large Creatures}\label{Very Large Creatures}
      \parhead{Space} Very large creatures take up multiple squares. Anything which affects a single square the creature occupies affects the creature.

      \parhead{Movement} Creatures two size categories larger than you are not considered obstacles and do not hinder your movement.

      \parhead{Sweeping Weapons} Weapons used by creatures that are Huge or larger are automatically \weapontag{Sweeping}, as shown in \trefnp{Size Categories}.
      If the creature's weapons would already be Sweeping, add that value to the normal Sweeping value for that weapon.
      For example, a greatsword used by a Gargantuan creature would have Sweeping (3).

  \subsection{Weight Limits}\label{Weight Limits}

    \begin{dtable}
      \lcaption{Weight Limits by Strength}
      \setlength{\tabcolsep}{4pt}
      \begin{dtabularx}{\columnwidth}{X X X}
        \tb{Strength} & \tb{Carrying Capacity} & \tb{Push/Drag} \tableheaderrule
        -9            & Fine x4                & Diminutive x4 \\
        -8            & Fine x8                & Diminutive x8 \\
        -7            & Diminutive x2          & Tiny x2       \\
        -6            & Diminutive x4          & Tiny x4       \\
        -5            & Diminutive x8          & Tiny x8       \\
        -4            & Tiny x2                & Small x2      \\
        -3            & Tiny x4                & Small x4      \\
        -2            & Tiny x8                & Small x8      \\
        -1            & Small x2               & Medium x2     \\
        0             & Small x4               & Medium x4     \\
        1             & Small x8               & Medium x8     \\
        2             & Medium x2              & Large x2      \\
        3             & Medium x4              & Large x4      \\
        4             & Medium x8              & Large x8      \\
        5             & Large x2               & Huge x2       \\
        6             & Large x4               & Huge x4       \\
        7             & Large x8               & Huge x8       \\
        8             & Huge x2                & Gargantuan x2 \\
        9             & Huge x4                & Gargantuan x4 \\
        10            & Huge x8                & Gargantuan x8 \\
        11            & Gargantuan x2          & Colossal x2   \\
        12            & Gargantuan x4          & Colossal x4   \\
        13            & Gargantuan x8          & Colossal x8   \\
        14            & Colossal x2            & Colossal x16  \\
        15\plus\fn{1} & \tdash                 & \tdash        \\
      \end{dtabularx}
      1 To calculate the weight limits for a creature with epic Strength, double the number of objects it can carry and drag for every point of Strength beyond 14.
    \end{dtable}

    Your Strength determines how much you can carry or push, as shown in \trefnp{Weight Limits by Strength}.
    Your weight limits are measured in terms of how many objects or creatures of a given \glossterm{weight category} that you can carry or push at once (see \pcref{Weight Categories}).
    The limit of how much you can hold in your hands or on your body without suffering any penalties is called your \glossterm{carrying capacity}.
    If you need to move more weight than that, you can push or drag objects or creatures up your pushing and dragging limit as a standard action (see \pcref{Shove}).

    In general, it is not meaningful to consider the weight of any objects with a weight category lighter than your maximum weight category.
    If it matters, you can treat eight objects of one weight category as having an equivalent weight to a single object that is one weight category heavier.

    \parhead{Large Creatures} Unusually large or small creatures gain a bonus to their Strength for the purpose of determining their weight limits.
    For details, see \tref{Size Categories}.

    \parhead{Multi-Legged Creatures} The figures on \trefnp{Weight Limits by Strength} are for bipedal creatures.
    A creature with four or more legs can carry, push, or drag twice as many objects as a bipedal creature of the same Strength.

    \subsubsection{Weight Categories}\label{Weight Categories}
      Weight is generally measured in \glossterm{weight categories} rather than pounds or kilograms.
      Weight categories use the same terms as \glossterm{size categories}, as shown in \tref{Weight Categories}.
      In general, a creature's weight category is the same as its size category.

      Objects and creatures can also be either \glossterm{lightweight} or \glossterm{heavyweight}.
      Lightweight objects and creatures have a weight category that is one category lighter than their size category.
      Heavyweight objects and creatures have a weight category that is one category heavier than their size category.

      Objects that occupy only a small percentage of the space appropriate for their size category, such as swords, are usually lightweight.
      Objects that fully occupy the space appropriate for their size category, like boulders, are usually heavyweight.

      \begin{dtable}
        \lcaption{Weight Categories}
        \begin{dtabularx}{\textwidth}{l l X}
          \tb{Weight Category} & \tb{Falling Damage Die} & \tb{Average Weight} \tableheaderrule
          Fine                 & \tdash\fn{1}            & 1/2 oz.       \\
          Diminutive           & \tdash                  & 1/4 lb.     \\
          Tiny                 & 1d2                     & 2 lb.       \\
          Small                & 1d6                     & 15 lb.      \\
          Medium               & 1d8                     & 120 lb.     \\
          Large                & 1d10                    & 1,000 lb.   \\
          Huge                 & 2d6                     & 8,000 lb.   \\
          Gargantuan           & 2d8                     & 64,000 lb.  \\
          Colossal             & 2d10                    & 512,000 lb. \\
        \end{dtabularx}
        1. Creatures and objects with a Fine or Diminutive weight category have no falling damage dice and cannot cause falling damage.
      \end{dtable}

      % TODO: clarify that being immune to falling damage doesn't make you immune to damage from being fallen on.
  \subsection{Falling Damage}\label{Falling Damage}
    A falling creature or object descends by 300 feet at the end of each phase.
    If it hits an obstacle while falling, the falling creature or object takes falling damage.
    The obstacle it lands on takes half that damage, to a maximum amount of damage equal to the falling creature or object's hit points.

    This falling damage is based on the weight category of the falling object or creature.
    Each weight category has an associated falling damage die listed in \tref{Weight Categories}.
    For every 10 feet of the fall, the falling damage die is rolled once, to a maximum fall distance of 300 feet.
    If the total fall distance was less than 10 feet, no falling damage is dealt.
    For example, a Medium creature falling 50 feet would take 5d8 falling damage.

    If you fall intentionally, you treat your fall as being 10 feet shorter.
    If you fall after intentionally jumping, you measure falling damage from 10 feet lower than your height at the start of your jump, ignoring the maximum height you reached during the jump.

\section{Calculation Guidelines}
  This explains some general guidelines to follow when calculating character statistics.

  \subsection{Stacking Rules}\label{Stacking Rules}
    Usually, modifiers stack with each other, meaning that you add or subtract all of the modifiers to get the final result.
    However, some modifiers do not stack with each other, as described below.
    When bonuses don't stack with each other, you only apply the largest bonus.
    Likewise, when penalties don't stack with each other, you only apply the largest penalty.

    \begin{raggeditemize}
      \item Effects from abilities with the same name do not stack.
      \item Enhancement bonuses do not stack with each other.
      \item If a creature gains the same condition multiple times, the effects do not stack, but each instance is tracked separately.
        The creature must remove all instances of the condition before the effects are removed.
      \item Multiple \magical effects that change a creature's \glossterm{size category} do not stack.
        If multiple magical effects both increase and decrease size, size increases offset size decreases on a one-for-one basis to determine the creature's final size.
      \item If you have two separate abilities which grant you a special sense with a particular range, such as \trait{darkvision} or \trait{blindsight}, you sum the range from both abilities to find your total range with that sense.
    \end{raggeditemize}

  \subsection{Minimum and Maximum Modifiers}\label{Minimum and Maximum Modifiers}
    Some bonuses specify that they cannot increase the value beyond a given point.
    These bonuses must always be applied last, and cannot be combined with other bonuses to exceed the maximum value.
    If multiple bonuses specify different maximum values, use the lower maximum value.
    If a bonus with a maximum value is applied to a value that already exceeds the maximum value the bonus can provide, simply ignore the bonus and its maximum value.

    Similarly, some abilities have a minimum total modifier.
    The minimum is applied after all other modifiers.

  \subsection{Doubling and Halving}\label{Doubling and Halving}
    Normally, doubling twice results in four times the original value, and halving twice results in one quarter of the original value.
    However, if a single effect doubles something twice, it only adds an additional increment of the original value, making it three times as large.

    To clarify, here are some examples, assuming you're making using a weapon that normally deals 10 damage:
    \begin{raggeditemize}
      \item Normal hit: 10 damage.
      \item Critical hit: 20 damage.
      \item Double damage maneuver: 20 damage.
      \item Double critical hit: 30 damage (since it's two doublings from the same effect)
      \item Critical hit with a double damage maneuver: 40 damage.
    \end{raggeditemize}

  \subsection{Changing Statistics}

    Your modifiers and defenses can change for many reasons.
    In general, all changes take effect immediately.

    It is not normally possible for a character to lose access to resources that require them to make choices, such as insight points or trained skills.
    If a character does somehow lose the prerequisites for choices they have made, such as if their Intelligence is permanently reduced, they immediately lose relevant abilities until they are within their new limits.

  \subsection{Rounding}
    In general, if you encounter a fractional number, you round it down.
    For negative numbers, this means rounding it away from 0, not towards 0.

\section{Character Creation}\label{Character Creation}

  Creating a character involves a mixture of thematic and mechanical decisions that will work together to create a fun character that is rewarding to play.
  As mentioned earlier in this chapter, there are four core systems for customizing your character's mechanics: class, attributes, skills, and species.
  In addition, there are five core thematic considerations when creating a character: concept, personality, motivation, background, and appearance.

  These decisions are described below in a order that makes sense for many characters, and full details for each decision are given after this initial list.
  It is essentially a sandwich, with narrative decisions wrapped around a central core of your character's mechanical components.
  However, you can make several of these decisions in any order, and you may find it easier to create a character in a different way.
  The only real limitation is that your skills should generally be the last mechanical choice you make, since they are strongly affected by your class and attributes.

  \begin{enumerate}
    \item Character concept: Describe your character with a short, simple phrase that captures their essence.
    \item Motivation and goal: Describe what your character wants.

    \item Species: Define your character's species.
    \item Attributes: Define your character's fundamental physical and mental potential.
    \item Class archetypes: Define your character's source of power.
    \item Insight points: Learn new abilities.
    \item Skills: Define your character's areas of non-combat expertise.

    \item Personality: Describe how your character acts and reacts to the world.
    \item Background: Describe what made your character become who they are now.
    \item Appearance: Describe what your character looks like.
    \item Alignment: Describe your character's moral compass.
    \item Name: Choose a name.
  \end{enumerate}

  \subsection{Step 1: Character Concept}

    Fundamentally, who is your character?
    You should think of a short phrase that describes the core concept behind the character you will create.
    It's best to think in broad strokes when creating a character concept.
    Your concept should be more than just a factual description of your species or what you do.
    It should be something that makes you memorable.
    Some sample character concepts are given below for inspiration.

    \begin{raggeditemize}
      \item Pragmatic wanderer
      \item Artistic pixie
      \item Mushroom-obsessed hermit
      \item Bumbling do-gooder
      \item Dim-witted bodyguard
      \item Cowardly storyteller
      \item Bear-barian
      \item Parsimonious law enforcer
      \item Peaceful naturalist
      \item Trigger-happy pyromaniac
      \item Heroic, simple-minded warrior
      \item Friendly necromancer
      \item Chaotic speed demon
      \item Pompous ex-noble
      \item Sarcastic mercenary
      \item Battle-scarred priest
      \item Ambitious arcane prodigy
      \item Charismatic musician
      \item Aloof scholar
      \item Blunt-spoken warrior
      \item Crazed prophet
      \item Polite warrior
      \item World-weary pirate
      \item Devout cultist
      \item Con artist with a heart of gold
    \end{raggeditemize}

  \subsection{Step 2: Motivation and Goal}
    Why does your character put in all of the effort that adventuring requires?
    They probably have a goal that they are trying to achieve, or an ideal that they are trying to embody.
    Writing down a specific goal or ideal can be helpful as an anchor point when defining the character.

  \subsection{Step 3: Species}
    It's often convenient to make your species your first mechanically relevant choice.
    Your species can have a strong effect on your personality and narrative, but it has a relatively small effect on your character's play style.
    It's also easier to know your species before you choose your attributes, since your species can slightly modify your attributes.

    Choose one of the eight common species options, or talk with your GM about choosing an uncommon species (see \pcref{Uncommon Species}).
    Record any specific abilities the species gives you on your character sheet, but if this is your first mechanical choice, you won't be able to finalize any of your statistics yet.
    You should also choose the languages that you can speak, since that is influenced by your species (see \pcref{Communication and Languages}).

  \subsection{Step 4: Attributes}\label{Step 4: Attributes}
    Your attributes are a good option for your second mechanically relevant choice.
    They have a large impact on your character's strengths and weaknesses, so it's useful to know them as soon as possible.
    They're also much easier to understand and finalize than your class archetypes.

    You have 8 points to distribute among your attributes.
    Increasing an attribute by 1 costs 1 point, and you can increase each attribute up to a maximum of 3.
    Instead of allocating points yourself, you can use one of the following three common attribute arrays:
    \begin{raggeditemize}
      % 3 + 2 + 2 + 1 = 8
      \item Standard: 3, 2, 2, 1, 0, 0
        % 3 + 3 + 2 = 8
      \item Specialized: 3, 3, 2, 0, 0, 0
        % 2 + 2 + 2 + 1 + 1 = 8
      \item Balanced: 2, 2, 2, 1, 1, 0
    \end{raggeditemize}

    Once you have chosen your attributes, add your species modifier to your attributes (if any).
    Then, record in your character sheet the various effects that your attributes have on your statistics.

    \subsubsection{Attribute Penalties}\label{Attribute Penalties}
      You can voluntarily take penalties to your attributes.
      If you reduce an attribute by a total of \minus1, you gain an additional \glossterm{trained skill} (see \pcref{Trained Skills}).
      If you reduce an attribute by a total of \minus2, you instead gain an additional \glossterm{insight point} (see \pcref{Insight Points}).
      You cannot gain these benefits from reducing more than two attributes below 0 in this way.
      In addition, you can never reduce an attribute below \minus2 in this way.

  \subsection{Step 5: Class and Class Archetypes}
    This is the most complicated choice you have to make for your character.
    It requires reading at least some of the Classes chapter to understand which classes are interesting to you.
    Class details can be found in \pcref{Classes}.

    You should choose one of the eleven classes, and apply all effects of choosing that as your \glossterm{base class}.
    Then, choose one of the five archetypes within that class.
    You gain the rank 1 ability from that archetype (see \pcref{Archetypes}).

    When you reach levels 2 and 3, you'll choose new archetypes from the same class, becoming rank 1 in each of those archetypes as well.
    After that, you won't gain any more new archetypes when you gain levels.
    Instead, you'll just increase your rank in the three archetypes you already have.

    If you are particularly adventurous, this is also when you should choose if you want to be a multiclass character.
    Multiclass characters can gain archetypes from multiple classes.
    This does not increase the number of archetypes you know, so it does not directly increase your power.
    However, multiclass characters can be more specialized or more versatile than single-class characters, and can represent unusual character concepts.
    For details, see \pcref{Multiclass Characters}.

  \subsection{Step 6: Insight Points}
    Once you have chosen your class archetypes, attributes, and species, you know how many insight points you have, and can choose how to spend them.
    Don't forget to record on your character sheet how you spent each insight point.
    Otherwise, you might get confused later about why you have more spells known than you normally would.

    In some circumstances, you might want to delay spending your insight points until you are higher level.
    For example, a fighter/sorcerer multiclass character who wants to have both spells and maneuvers can't have access to both spells and maneuvers at level 1, so they wouldn't be able to spend insight points on both spells and maneuvers.
    You aren't forced to spend all of your insight points, so you can save them up for later.
    You can also talk to your GM about spending them at level 1 and then retraining those insight points once you are higher level.

  \subsection{Step 7: Skills}
    You should choose which skills you have as \glossterm{trained skills} (see \pcref{Skills}).
    Your \glossterm{class} gives you a certain number of trained skills from among its \glossterm{class skills}.
    The class skills for each class are summarized in \tref{Class Skills}.

    There are other ways to become trained in skills that are not part of your class.
    If your Intelligence is positive, you gain additional trained skills equal to your Intelligence.
    You can also spend \glossterm{insight points} to gain one trained skill per insight point (see \pcref{Insight Points}).
    Some abilities can grant additional trained skills.

    If you are untrained in a skill, your bonus with that skill is equal to half of its associated attribute (if any).
    If you are trained in a skill, your bonus with that skill is equal to 3 \add the higher of its associated attribute (if any) and half your level.
    Many abilities can increase or decrease your bonus with particular skills.

    The number of skills you can have trained, and which skills those are, depend on every preceding step, so it's a good place to finish.

    Sometimes, you might have more trained skills than you know what to do with, especially if you are still figuring out the details of your character concept.
    You aren't forced to decide all of your trained skills at level 1, so you can save them up and choose more trained skills when you level up.
    You can also talk to your GM about letting you decide your trained skills on the fly during the first game session or two based on what actions you take during the session.
    This can be a fun way to figure out what your character's personality is through the process of playing them.

  \subsection{Step 8: Starting Equipment}
    When you create a character, they can start with some basic items.
    Items have \glossterm{item ranks} that indicate the approximate rank that characters can reasonably get access to them.
    Typically, you can start with a single rank 1 item, up to three rank 0 items, and a standard adventuring kit.
    Individual campaigns or character backstories may significantly change what starting equipment is available, so check with your GM.

  \subsection{Step 9: Personality}

    How does your character behave?
    You should decide, in broad terms, what your character's personality is.
    This will change over time, especially as you start playing the character in the game, so you don't need to define everything perfectly.
    However, having a general sense of how your character behaves is helpful.

    For most games, it's important to have a personality that can tolerate working with others in a group.
    A character that is excessively aloof, moody, or obnoxious can make the game more difficult to enjoy for everyone.
    Likewise, a character who tries to speak for everyone or who repeatedly steals the spotlight from others can be frustrating to work with.
    You should figure out the right balance with your fellow players and your GM.\@

  \subsection{Step 10: Background}
    What happened in their character's past to make them the way that they are?
    What were their parents like, and where are they now?
    You don't have to have all of the answers when you first create a character, but it's good to have some idea.
    The richer your backstory, the more the GM can weave that into the narrative of the current story.
    Sometimes, it's fun to take a break from saving the world to go visit someone's grandma.
    For details about suggested backgrounds that have a strong effect on the game world, see \pcref{Backgrounds}.

  \subsection{Step 11: Appearance}
    What does your character look like?
    What would someone's first impression of them be?
    This can be helpful for understanding how other characters in the game world - or even monsters - would react to you.

  \subsection{Step 12: Alignment}
    Your character's alignment reflects their moral character: are they more inclined to good or evil, and to chaos or order?
    Alignments are described in more detail at \pcref{Alignment}.

  \subsection{Step 13: Name}
    What is your character's name?
    This choice can influence the tone your character will set in the game.
    If your name is Sir Patty Cakes or Shanky, the game is likely to be lighter and sillier in tone.
    Fancy fantasy-appropriate names like Ayala or Theodolus tend to push the game in a slightly more serious direction, especially if you make the daring choice to include a canonical last name.
    As always, stay in tune with what the GM and the other players are expecting.

\section{Alignment}\label{Alignment}
  A creature's general moral and personal attitudes can be represented by its alignment: lawful good, neutral good, chaotic good, lawful neutral, neutral, chaotic neutral, lawful evil, neutral evil, or chaotic evil.

  Alignment is a tool for developing your character's perspective.
  Like a character's class, it is intended to provide a canvas to inspire creativity, not a narrow window to constrain identity.
  Each alignment represents a broad range of personality types or personal philosophies, so two characters of the same alignment can still be quite different from each other.
  In addition, few people are completely consistent.

  \subsection{Aligned Characters}
    Alignment is a spectrum, not a binary.
    Some characters are defined as being ``good'' or ``evil'', but this is a broad definition with a great deal of variation.
    Only angels and demons are ``pure'' good or evil.

    Approximately half of the general population of the world is neutral between good and evil, with a quarter being good and another quarter being evil.
    This does not mean that a quarter of people are all saintly do-gooders and another quarter are all psychotic murders.
    It would be more accurate to say ``good'' people are simply the most altruistic quarter of the population.
    There are a rare few saints, but all good characters have some amount of selfishness in particular contexts.
    Likewise, evil characters may act altruistically in some situations still having a fundamentally selfish nature.

    A similar ratio exists for law and chaos.
    This means that the overall population follows the ratios given in \trefnp{Aligned Population Ratios}.
    Of course, these populations are not distributed equally in the world.
    In general, humanoid civilization tends to have more good and lawful characters, and monsters tend to be more chaotic and evil.
    The GM can give more context about how alignment is used in their specific world.
    Of course, the GM can also redefine good and evil itself, so talk with them if alignment is important to you or your character.

    Monsters have their typical alignments listed in their description.
    Even monsters who are listed as ``always good'' or ``always evil'' are not \textit{maximally} good or evil -- just good or evil enough to fall into that quartile.
    Many evil monsters can be perfectly reasonable if they are capable of speech, and will uphold bargains that serve their interests.
    Likewise, good monsters have their own objectives, and will not simply do whatever an adventuring party asks of them.

    \begin{dtable}
      \lcaption{Aligned Population Ratios}
      \begin{dtabularx}{\textwidth}{X X X X}
        \tb{Alignment} & \tb{Good} & \tb{Neutral} & \tb{Evil} \tableheaderrule
        \tb{Lawful}    & 6.25\%    & 12.5\%       & 6.25\% \\
        \tb{Neutral}   & 12.5\%    & 25\%         & 12.5\% \\
        \tb{Chaotic}   & 6.25\%    & 12.5\%       & 6.25\% \\
      \end{dtabularx}
    \end{dtable}

  \subsection{Good vs. Evil}
    Distinguishing good from evil is a deeply complex task.
    In a universe where angels, demons, and deities exist and interact with the affairs of mortals, being able to clearly define good and evil is important.
    For the purposes of Rise, good and evil are strictly defined according to the intentions of one's actions, not their eventual outcomes.
    Good intentions are altruistic, and evil intentions are selfish.

    This model for good and evil has limited value as a moral system for the real world, and it neglects several dimensions of morality that people might consider important.
    It is intentionally vague about what consistutes ``other beings'', and reasonable characters might disagree about how to consider the needs of non-humanoid living things like plants and animals.
    However, is a useful way to define alignment as a character trait and roleplaying tool.
    Good characters have recognizably different behaviors from evil characters, but they are not so intrinsically opposed that they can't coexist in the same group of adventurers.
    Evil characters may cause you problems if you get in their way, but no civilized area would allow simply killing evil on sight.

    \parhead{Good} Good intentions are altruistic.
    They are based on respect and empathetic consideration for other beings.

    A good character will try to learn what other beings want or need so they can help most effectively.
    They might try to keep everyone's spirits up with cameraderie and good humor, donate money whenever possible to help those in need, or dedicate their life to punishing criminals and protecting the innocent.
    Good characters may have significant disagreements about what actions are best, and not all care about some lofty ``greater good''.
    Some believe that self-sacrifice is noble, while many would say that it does more harm than good to neglect your own needs.
    There are many interpretations of altruism.

    Even an action with good intentions may have disastrous consequences.
    Unintentionally causing harm does not make a character evil, but good characters pay attention to the effects their actions have in practice.
    If a good character caused harm, intentionally or otherwise, they would try to rectify their mistake.

    \parhead{Evil} Evil intentions are selfish.
    They come from prioritizing one's own desires when that conflicts with the knowable needs and desires of other beings.

    An evil character will generally not care what other beings want or need unless they personally benefit from that knowledge.
    They might betray allies, break laws or abuse legal loopholes to gain an advantage, or bully other creatures into doing their bidding.
    Evil characters may take actions that help others and can even work effectively as a team, but their ultimate motivation is to help themselves or make themselves feel better, not to help others.

    \parhead{Neutral} Characters that are neutral between good and evil are neither consistently altruistic nor consistently selfish.
    Most neutral characters behave altruistically in some ways and selfishly in other ways -- either at different times, or about different aspects of life.
    They often have strong bonds to particular individuals who they care about selflessly, but are not altruistic in a general sense.

    Intentions that do not involve other beings are neither good nor evil.
    Similarly, actions taken to meet one's own mandatory needs are neither good nor evil.
    Wild animals may act primarily out of self-interest, but since they generally lack the capacity to recognize the needs and desires of other beings, they are considered neutral rather than good or evil.

  \subsection{Law vs. Chaos}
    \parhead{Law} Lawful characters value consistency.
    They obey rules that guide their actions.
    Some lawful characters draw their rules from external forces, such as serving a particular master or following the legal laws of the land.
    Other lawful characters follow rules they make for themselves.

    \parhead{Chaos} Chaotic characters value flexibility and freedom.
    They make decisions based on what they think or feel at the time, even if it is inconsistent with their previous statements or actions.

    \parhead{Neutral} Characters that are neutral between law and chaos are neither exceptionally consistent nor exceptionally inconsistent.
    They tend to be generally consistent but may change their minds under the right circumstances.
    Non-sapient beings such as animals are neutral rather than lawful or chaotic.

\section{Personal Appearance}\label{Personal Appearance}
  \begin{dtable!*}
    \lcaption{Species Age}
    \begin{dtabularx}{\textwidth}{l *{5}{>{\ccol}X}}
      \tb{Species} & \tb{Adulthood} & \tb{Middle Age} & \tb{Old}  & \tb{Venerable} & \tb{Maximum Age} \tableheaderrule
      Human        & 18 years       & 40 years        & 55 years  & 70 years       & \plus3d10 years \\
      % 50% longer than humans
      Dwarf        & 30 years       & 60 years        & 85 years  & 110 years      & \plus5d10 years \\
      % 4x longer than humans, scaling up to 5x/6x for old/venerable
      Elf          & 70 years       & 160 years       & 275 years & 420 years      & \plus3d\% years \\
      % 25% longer than humans
      Halfling     & 22 years       & 50 years        & 70 years  & 90 years       & \plus4d10 years \\
      % 0 -> 20 -> 40 -> 60% longer than humans
      Kobold       & 18 years       & 50 years        & 80 years  & 110 years      & \plus5d10 years \\
      % 25% longer than humans
      Mixed        & 22 years       & 50 years        & 70 years  & 90 years       & \plus4d10 years \\
      % Same as humans, scaling down for middle+
      Orc          & 18 years       & 35 years        & 45 years  & 55 years       & \plus3d10 years \\
    \end{dtabularx}
  \end{dtable!*}

  \begin{columntable}
    \columncaption{Typical Height and Weight}
    \begin{dtabularx}{\columnwidth}{l *{2}{>{\lcol}X}}
      \tb{Species} & \tb{Average Height} & \tb{Average Weight} \tableheaderrule
      Human        & 5' 5''              & 140 lb. \\
      Dwarf        & 4' 2''              & 120 lb. \\
      Elf          & 5' 9''              & 120 lb. \\
      Halfling     & 3' 4''              & 50 lb.  \\
      Kobold       & 2' 10''             & 40 lb.  \\
      Mixed        & 5' 4''              & 130 lb. \\
      Orc          & 6' 0''              & 190 lb. \\
    \end{dtabularx}
  \end{columntable}

  \subsection{Age}
    The typical age for each species is listed in \trefnp{Typical Ages}.
    The Adulthood column indicates the minimum age for adulthood.
    Most adventurers are somewhere between adulthood and middle age.

    If you are old, you take a \minus2 penalty to \glossterm{checks} based on Strength, Dexterity, Constitution, and Perception.
    However, you gain a \plus2 bonus to \glossterm{checks} based on Intelligence and Willpower.
    If you are venerable, these modifiers change to \minus4 and \plus4 respectively.
    In general, player characters should not start as old or venerable age, but the GM can always allow it for specific campaigns if they want.

    When you reach venerable age, the GM secretly rolls your maximum age, which is the number from the Venerable column on \trefnp{Typical Ages} plus the result of the dice roll indicated on the Maximum Age column on that table.
    They record the result.
    If you reach your maximum age, you die of age-related illnesses or frailty at some time during that year.

  \subsection{Height and Weight}
    The typical height and weight for each species is listed in \tref{Typical Height and Weight}.
    The average man from each species is slightly taller and heavier than the average woman, but this is not a restriction for player characters.

\section{Backgrounds}\label{Backgrounds}
  Each character has a history from before the current campaign.
  This can include their childhood, family, previous occupations, and more.
  Backgrounds generally do not have direct effects on a character's statistics.
  Narratively, a background explains a character's statistics, not defines them.
  However, a character's background can still have a significant influence on how they interact with the world.

  Backgrounds can be summarized as a set of benefits and flaws.
  If you choose a background benefit, you must also choose a flaw.
  Some backgrounds internally provide both benefits and flaws.
  These are called mixed backgrounds, and taking them does not change how many benefit or flaw backgrounds you can choose.
  You generally shouldn't have more than one benefit and one flaw, or a single mixed background.

  Not every character should have a specific background with effects listed here.
  These backgrounds can significantly change how a character relates to the world.
  It's fine to have a simpler background to put more focus on other aspects of your character's narrative journey.

  At the GM's discretion, background benefits can also be acquired during a campaign.
  For example, successfully performing a heroic and extremely public feat might give the whole party the benefits of the Folk Hero background.
  Conversely, a party who is defeated in battle may find themselves with the Indebted background, as the victors decide to spare them - with a price.

  Many backgrounds, especially background benefits, are only relevant in some area where your reputation could plausibly be known.
  If you're deep in the Astral Plane, nobody will care that you were a folk hero back home.
  The GM should ensure that backgrounds are usually relevant, and your reputation may become more broadly known through the course of a campaign.

  \subsection{Background Benefits}
    \parhead{Criminal Connections} You have a reputation in criminal circles as a trustworthy partner in crime.
    This reputation is not generally known outside of that domain, so law enforcement will not generally cause you problems.
    You know how to reliably identify and contact other criminals, at least in a significant local area.
    They will generally be helpful, though they may still charge a fair price for any direct assistance.
    \parhead{Folk Hero} You are known generally by most common folk as a heroic and benevolent figure, at least in a significant local area.
    This reputation could be well-earned or built on deception.
    In either case, it is widely believed.
    Common people will generally try to be helpful whenever you need it, as long as you act appropriately to match the tales.
    \parhead{Guild Member} You are a member of a major trade guild.
    This could be based on your abilities with a particular Craft skill, your class, or some other abilities you have that are common enough in the world to merit a trade guild.
    The GM can provide more guidance about what guilds exist.
    In general, the more advanced and civilized the world is, the more niche guilds exist.
    Your guild will have a presence in any major city and some minor ones.
    People will generally be more willing to believe that you can perform tasks relevant to your guild membership.
    In addition, fellow guild members will be more helpful when you need it.
    \parhead{Mysterious Heirloom} You have a family heirloom that you must keep safe.
    In the right hands, it would probably have great value or importance, but most people would dismiss it as a simple trinket or oddity.
    \parhead{Landed} You have a family estate that you can live in comfortably, including people who work the land and maintain the house.
    It is well equipped for most types of training, research, or other long-term work you and your allies might need to do.
    You do not have sole ownership of the estate, so selling it or otherwise attempting to directly gain value from it would be complicated and risky.
    However, it is a safe refuge when you need it.
    \parhead{Noble} Your family is at least minor nobility.
    This can change people's reactions to you and allow you to more easily access events and important people in high society.

  \subsection{Background Flaws}
    \parhead{Escapee} You have escaped from some private individual or institution who is very interested in your return.
    This could be because you were a slave, because of something unique about your heritage, or any other reason.
    The people who want your return will invest time and resources into pursuing you wherever it is feasible, including hiring bounty hunters to track you down and bring you back.
    % \parhead{Forsworn} You are generally known to be a liar who has broken an important oath.
    % Anyone aware of this fact will not trust you or your apparent allies with anything of importance.
    \parhead{Indebted} Thanks to your own past mistakes or your family's actions, you owe a significant debt.
    Directly paying off the debt would require at least a rank 3 wealth item (1,000 gp).
    The creditor may require you to make smaller recurring payments, or may compel you to perform tasks for them to pay down your debt over time.
    They can generally be negotiated with to a limited degree, but if the relationship turns sour, this flaw may switch to the Escapee flaw.
    \parhead{Nemesis} Someone specifically wants you to suffer due to your shared history.
    They and their allies are initially stronger than you in a direct conflict, and directly attacking them is almost certainly illegal.
    You will have to avoid directly confronting them, at least at first.
    Unlike the Escapee and Wanted flaws, your nemesis will not generally try to kill you or physically harm you.
    Instead, they will act to subvert your goals, either directly or through agents.
    Anything you seem to want, they will do their best to thwart.
    \parhead{Repulsive} You are personally noxious, odorous, grotesque, or otherwise unpleasant to see and spend time with.
    The cause could be injury, disease, a powerful curse, or some other reason.
    This negatively affects almost everyone's reactions to you, which makes social interactions more difficult.
    In addition, any living creature who can see, hear, or smell you is unable to benefit from a \glossterm{short rest} or \glossterm{long rest}.
    Even if you can find loyal travelling companions, they still have their limits, forcing you to camp alone.
    \parhead{Wanted} You are wanted for serious crimes in some major area.
    This flaw does not require that there are posters with your name in every city, but the area where you are wanted must be important to the campaign.
    You will have to avoid identification by both law enforcement and even common people while in that area, and in other areas that may be aware of those details.
    In addition, bounty hunters may pursue you wherever you go.
    You may be innocent of the charges, but if so, it would be hard to prove your innocence.

  \subsection{Mixed Backgrounds}
    \parhead{Scion} You are in the line of inheritance for an important throne or noble house.
    You are not the designated heir, but some small distance removed, such as a third child.
    If the obstacles to your inheritance were cleared, you could become powerful and wealthy.
    However, you may also be a target for people trying to ensure their own inheritance, or simply using you to manipulate your family.
    As an important figure, your family or related people may try to place restrictions on your actions to ensure your safety.


    % \section{Sample Characters}

    %     This section lists sample characters for each class archetype.
    %     You can simply pick up one of these characters and use it as your character.
    %     Alternately, you can use a sample character as a starting point and adjust it to match your own character concept.
    %     The sample characters are ordered by class first, and by archetype within each class second.

    %     \subsection{Barbarian}

    %         \subsubsection{Battleforged Resilience}
    %             \parhead{Species} Dwarf.
    %             % 2 1 3 0 2 2 base
    %             \parhead{Attributes} 2 Str, -1 Dex, 4 Con, 0 Int, 2 Per, 2 Wil (after species modifiers).
    %             \parhead{Class} Barbarian.
    %             \parhead{Archetypes} Battleforged Resilience first, Primal Warrior second, Totemist (bear totem) third.
    %             \parhead{Insight Points} 1.
    %             \parhead{Skills} Awareness, Climb, Endurance
    %             \parhead{Languages} Common, Dwarven, Giantish.
    %             \parhead{Equipment} Battleaxe, standard shield, scale mail. As you gain levels, use the best medium armor you can afford. If you gain proficiency with heavy armor, use that instead.
    %             \parhead{Legacy Item} Shield.
    %                 At level 6, choose \mitem{covering shield}.
    %                 At level 12, upgrade to \mitem{shield of arrow catching++}.
    %                 At level 18, choose a suitable Rank 7 shield property (e.g., \mitem{supreme shield of deflection}).
    %             \parhead{Combat Styles} Herald of War, Unbreakable Defense.
    %             \parhead{Suggested Feats} Shieldbearer, Martial Training, Regenerator, Toughness.
    %             \parhead{Combat Tactics} You are extremely difficult to kill.
    %             Take advantage of that by wading into the front lines of combat and drawing attention away from your more vulnerable allies.
    %             If you find yourself in danger, use defensive maneuvers like \maneuver{defensive strike} and \maneuver{flamboyant parry} to keep yourself safe.
    %             On the other hand, if your foes try to ignore you after realizing how durable you are, force them to engage with you using maneuvers like \maneuver{challenging strike} and \maneuver{guard the pass}.

    %         \subsubsection{Battlerager}
    %             \parhead{Species} Half-orc.
    %             % 3 2 2 0 2 1 base
    %             \parhead{Attributes} 4 Str, 2 Dex, 2 Con, -1 Int, 1 Per, 0 Wil (after species modifiers).
    %             \parhead{Class} Barbarian.
    %             \parhead{Archetypes} Battlerager first, Primal Warrior second, Totemist (lion totem) third.
    %             \parhead{Insight Points} 0.
    %             \parhead{Skills} Awareness, Endurance, Intimidate.
    %             \parhead{Equipment} Greatmace, scale mail. As you gain levels, buy a heavy crossbow and use the best medium armor you can afford.
    %             \parhead{Legacy Item} Weapon.
    %                 At level 6, choose \mitem{bloodfuel}.
    %                 At level 12, upgrade to \mitem{bloodfuel+}.
    %                 At level 18, upgrade to \mitem{bloodfuel++}.
    %             \parhead{Combat Styles} Herald of War.
    %             \parhead{Suggested Feats} Greatweapon Warrior, Rapid Reaction, Swiftrunner.
    %             \parhead{Combat Tactics} You are a furious frenzy of devastating damage and lethal critical hits.
    %             When you roll a 10 on an attack roll, whatever you attacked will probably die.
    %             Staying close to your allies is generally a good plan, since you don't have the durability to run into the middle of a horde of enemies safely.
    %             Your maneuvers help you deal with high-Armor enemies and enemy swarms, and give you the ability to sacrifice most of your statistics other than damage in exchange for more damage.

    %         \subsubsection{Outland Savage}
    %             \parhead{Species} Elf.
    %             % 2 3 2 1 2 0 base
    %             \parhead{Attributes} 2 Str, 4 Dex, 1 Con, 1 Int, 1 Per, -1 Wil (after species modifiers).
    %             \parhead{Class} Barbarian.
    %             \parhead{Archetypes} Outland Savage first, Primal Warrior second, Totemist (wolf totem) third.
    %             \parhead{Insight Points} 2 points (1 for proficiency with exotic armor weapons, 1 for Mobile Hunter combat style).
    %             \parhead{Skills} Awareness, Climb, Endurance, Stealth, Survival.
    %             \parhead{Languages} Common, Orcish.
    %             \parhead{Equipment} Flail, standard shield, scale mail. As you gain levels, use the best light armor you can afford. When you can, get spikes and a spiked knee crafted onto your armor.
    %             \parhead{Legacy Item} Apparel.
    %                 At level 6, choose \mitem{phasestep boots}.
    %                 At level 12, upgrade to \mitem{phasestep boots+}.
    %                 At level 18, choose a suitable Rank 7 apparel property (e.g., \mitem{supreme boots of agility}).
    %             \parhead{Combat Styles} Dirty Fighting, Mobile Hunter
    %             \parhead{Suggested Feats} Savage, Brawler, Swiftrunner.
    %             \parhead{Combat Tactics} You can move around the battlefield very quickly, and you are incredibly accurate with special combat actions like shoving and grappling enemies.
    %             Make the most of that by repositioning enemies, tripping them, or holding them in grapples so your allies can hit them.
    %             While you aren't in a grapple, use your flail in two hands to maximize your damage.
    %             When you enter a grapple, use your spiked knee to attack, since your flail is much less effective while grappling.
    %             If you don't have any allies who like being on the front lines, you won't be as effective at helping them deal damage to enemies, but you're still very skilled at preventing enemies from reaching your allies.
    %             In that case, consider choosing bear totem or shark totem instead of wolf totem.

    %         \subsubsection{Primal Warrior}
    %             \parhead{Species} Human.
    %             \parhead{Attributes} 3 Str, 2 Dex, 2 Con, 1 Int, 2 Per, 0 Wil.
    %             \parhead{Class} Barbarian.
    %             \parhead{Archetypes} Primal Warrior first, Battleforged Resilience second, Outland Savage third.
    %             \parhead{Insight Points} 1 point for an additional combat style, 1 point for an additional maneuver.
    %             \parhead{Skills} Awareness, Climb, Endurance, Intimidate.
    %             \parhead{Languages} Common, Dwarven, Orcish.
    %             \parhead{Equipment} Greataxe, scale mail. As you gain levels, buy a heavy crossbow and use the best medium armor you can afford.
    %             \parhead{Legacy Item} Weapon.
    %                 At level 6, choose \mitem{seeking}.
    %                 At level 12, upgrade to \mitem{honed}.
    %                 At level 18, upgrade to \mitem{honed+}.
    %             \parhead{Combat Styles} Dirty Fighting, Herald of War, Unbreakable Defense.
    %             \parhead{Suggested Feats} Greatweapon Warrior, Weapon Focus, Swiftrunner.
    %             \parhead{Combat Tactics} You have a great breadth of options available to you thanks to the number of maneuvers you know.
    %             You have the survivability to stand in close combat, especially if you use maneuvers from Unreakable Defense, but you can also shout at mobile enemies from range with maneuvers from Herald of War.
    %             Both Dirty Fighting and Herald of War give you maneuvers that work well against enemies with a high Armor defense, so you can adapt to whatever battle you find yourself in.
    %             You can make the most of your versatility by learning maneuvers like \maneuver{knockback shove} that are sometimes useless, but which can be devastatingly effective in the right context.

    %         \subsubsection{Totemist}
    %             Characters from this archetype can be very different based on their chosen totem.
    %             A bear totem character might resemble the typical character for the Battleforged Resilience archetype.
    %             A lion totem or shark totem character might resemble the typical character for the Battlerager archetype.
    %             A wolf totem character might resemble the typical character for the Outland Savage archetype.

    %             If you want to quickly create a character based on the eagle totem from this archetype, make the following choices:
    %             \parhead{Species} Human.
    %             \parhead{Attributes} 2 Str, 2 Dex, 3 Con, 0 Int, 4 Per, -1 Wil.
    %             \parhead{Class} Barbarian.
    %             \parhead{Archetypes} Totemist (eagle totem) first, Primal Warrior second, Outland Savage third.
    %             \parhead{Insight Points} 1 point for Perfect Precision combat style.
    %             \parhead{Skills} Awareness, Balance, Climb, Creature Handling, Survival.
    %             \parhead{Languages} Common, Elven, Giantish.
    %             \parhead{Equipment} Longbow, leather body armor. As you gain levels, use the best light armor you can afford.
    %             \parhead{Legacy Item} Weapon.
    %                 At level 6, choose \mitem{longshot}.
    %                 At level 12, upgrade to \mitem{longshot+}.
    %                 At level 18, choose a suitable Rank 7 weapon property (e.g., \mitem{supreme longshot}).
    %             \parhead{Combat Styles} Perfect Precision.
    %             \parhead{Suggested Feats} Sniper, Blindfighter, Swiftrunner.
    %             \parhead{Combat Tactics} You have incredible accuracy from very long range.
    %             Your defenses are relatively low, but as long as you stay far enough away from your foes, they can't take advantage of that weakness.
    %             You have the ability to prioritize any target on the battlefield, so make the most of your maneuvers that impose conditions or deal additional damage on weakened foes.

    %     \subsection{Cleric}

    %         \subsubsection{Divine Magic}
    %             \parhead{Species} Gnome.
    %             % 0 0 2 1 2 3 base
    %             \parhead{Attributes} -1 Str, 0 Dex, 3 Con, 1 Int, 2 Per, 4 Wil (after species modifiers).
    %             \parhead{Class} Cleric.
    %             \parhead{Archetypes} Divine Magic first, Divine Spell Mastery second, Domain Influence third.
    %             \parhead{Insight Points} 3 points (2 points for an additional mystic sphere (Photomancy), 1 point for an additional rank 1 spell).
    %             \parhead{Skills} Knowledge (local, religion), Medicine, Persuasion, Social Insight
    %             \parhead{Languages} Common, Gnome, Halfling.
    %             \parhead{Equipment} Mace, standard shield, scale mail. As you gain levels, use the best medium armor you can afford.
    %             \parhead{Legacy Item} 1-handed implement.
    %                 At level 6, choose \mitem{splitting staff}.
    %                 At level 12, upgrade to \mitem{educated staff+}.
    %                 At level 18, upgrade to \mitem{splitting staff+}.
    %             \parhead{Domains} Good, Magic
    %             \parhead{Mystic Spheres} Channel Divinity and Photomancy
    %             \parhead{Suggested Feats} Celestial Ancestry, Sphere Focus: Photomancy, Sphere Focus: Prayer
    %             \parhead{Combat Tactics} You can protect yourself and your allies and invoke divine wrath on your foes.
    %             Your attacks can hit a variety of defenses, so use the best spells for the situation.
    %             If you are facing a foe that not particularly vulnerable to your attacks, you can focus on helping your allies with ``boon'' spells to make their actions more effective and keep them safe.

    %         \subsubsection{Divine Spell Mastery}
    %             Use the typical character for the Divine Magic cleric archetype.
    %             Even if you focus on spells through this archetype, you should generally still rank up your spells before improving your rank in this archetype.

    %         \subsubsection{Domain Influence}
    %             Characters from this archetype can be very different based on their chosen domains.
    %             A character with spellcasting-focused domains might resemble the typical character for the Divine Magic cleric archetype.
    %             If you want to quickly create a more martial character based on the Strength and War domains from this archetype, make the following choices:

    %             \parhead{Species} Dwarf.
    %             % 2 0 3 0 2 1 base
    %             \parhead{Attributes} 2 Str, -1 Dex, 4 Con, 0 Int, 2 Per, 2 Wil (after species modifiers).
    %             \parhead{Class} Cleric.
    %             \parhead{Archetypes} Domain Influence first, Divine Magic second, Preacher third.
    %             \parhead{Insight Points} 2 points (unspent).
    %             \parhead{Skills} Awareness, Knowledge (local, religion), Medicine
    %             \parhead{Languages} Common, Draconic, Dwarven.
    %             \parhead{Equipment} Morning star, standard shield, scale mail. As you gain levels, use the best heavy armor you can afford.
    %             \parhead{Legacy Item} Armor.
    %                 At level 6, choose \mitem{armor of transfusion}.
    %                 At level 12, upgrade to \mitem{armor of transfusion+}.
    %                 At level 18, upgrade to \mitem{armor of transfusion++}.
    %             \parhead{Domains} Destruction, War
    %             \parhead{Mystic Sphere} Channel Divinity
    %             \parhead{Suggested Feats} Weapon Focus, Sphere Focus: Channel Divinity, Shieldbearer
    %             \parhead{Combat Tactics} You are a frontline fighter first and foremost.
    %             Your magically enhanced resistance and high defenses make you durable in combat, though you lack mobility. 
    %             When you need to distract foes or face down hordes, you can use your abilities from the Preacher archetype.

    %         \subsubsection{Healer}
    %             \parhead{Species} Human.
    %             % 0 1 2 1 0 2 base
    %             \parhead{Attributes} 0 Str, 2 Dex, 2 Con, 1 Int, 0 Per, 3 Wil.
    %             \parhead{Class} Cleric.
    %             \parhead{Archetypes} Healer first, Divine Magic second, Domain Influence third.
    %             \parhead{Insight Points} 3 points (2 points for an additional mystic sphere (Vivimancy), 1 point for an additional rank 1 spell).
    %             \parhead{Skills} Awareness, Deduction, Knowledge (local, religion), Medicine
    %             \parhead{Languages} Common, Draconic, Halfling.
    %             \parhead{Equipment} Morning star, standard shield, scale mail. As you gain levels, use the best medium armor you can afford.
    %             \parhead{Legacy Item} 1-handed implement.
    %                 At level 6, choose \mitem{staff of silence+}.
    %                 At level 12, upgrade to \mitem{selective staff}.
    %                 At level 18, upgrade to \mitem{spell wand, 7th}.
    %             \parhead{Domains} Life, Protection
    %             \parhead{Mystic Spheres} Prayer, Vivimancy
    %             \parhead{Suggested Feats} Sphere Focus: Vivimancy, Boongiver, Sphere Focus: Prayer
    %             \parhead{Combat Tactics} You have an unmatched mastery of healing and protection.
    %             You have reasonably high defenses, so you can take to the front lines as necessary to make the most of your \ability{divine aid} and \ability{divine protection} abilities.
    %             Although your \ability{healer's grace} ability is powerful, you shouldn't feel bad about attacking enemies.
    %             That's especially important early in a fight when your allies don't need healing yet and your enemies haven't realized that it's pointless to attack your allies while you are still standing.

    %         \subsubsection{Preacher}
    %             \parhead{Species} Human.
    %             % 0 0 2 1 3 0 base
    %             \parhead{Attributes} 0 Str, 0 Dex, 2 Con, 2 Int, 4 Per, 0 Wil.
    %             \parhead{Class} Cleric.
    %             \parhead{Archetypes} Preacher first, Divine Magic second, Divine Spell Mastery third.
    %             \parhead{Insight Points} 4 points (2 points for an additional mystic sphere (Revelation), 2 points unspent).
    %             \parhead{Skills} Awareness, Knowledge (local, religion), Medicine, Persuasion, Social Insight
    %             \parhead{Languages} Common, Dwarven, Elven.
    %             \parhead{Equipment} Club, standard shield, scale mail. As you gain levels, use the best medium armor you can afford.
    %             \parhead{Legacy Item} Apparel.
    %                 At level 6, choose \mitem{ruler's circlet}.
    %                 At level 12, upgrade to \mitem{quickcleanse ring}.
    %                 At level 18, upgrade to \mitem{amulet of revivification}.
    %             \parhead{Mystic Spheres} Enchantment, Revelation
    %             \parhead{Suggested Feats} Persuasion Specialization, Sphere Focus: Enchantment, Sphere Focus: Revelation
    %             \parhead{Combat Tactics} Your social skills are virtually unmatched, and you have a wide variety of spells that give you narrative power in social situations.
    %             In combat, your \ability{denounce the heathens} ability is essentially guaranteed to hit, so you should stay close enough to the front lines to make good use of it.
    %             You can take advantage of the lowered defenses of your denounced foes to succeed with powerful mind-affecting spells.

    %     \subsection{Druid}

    %         \subsubsection{Elementalist}
    %             \parhead{Species} Human.
    %             % 0 0 2 1 3 2 base
    %             \parhead{Attributes} 0 Str, 0 Dex, 2 Con, 2 Int, 4 Per, 2 Wil.
    %             \parhead{Class} Druid.
    %             \parhead{Archetypes} Nature Magic first, Elementalist second, Nature Spell Mastery third.
    %             \parhead{Insight Points} 3 points (2 points for an additional mystic sphere, 1 point unspent).
    %             \parhead{Skills} Awareness, Balance, Endurance, Knowledge (dungeoneering, nature), Survival, Swim
    %             \parhead{Languages} Common, Sylvan
    %             \parhead{Equipment} Sickle, buckler, hide armor. As you gain levels, keep using hide armor.
    %             You may want to keep leather armor around in case you need to do a lot of jumping or swimming - or at high levels, flying.
    %             \parhead{Legacy Item} Implement.
    %                 At level 6, choose \mitem{perceptive staff}.
    %                 At level 12, upgrade to \mitem{selective staff}.
    %                 At level 18, upgrade to \mitem{echoing staff}.
    %             \parhead{Mystic Spheres} Any two of the four elemental mystic spheres.
    %             Your \textit{elemental spell} ability gives you access to spells from the fourth mystic sphere.
    %             That means that the specific three mystic spheres you choose mostly just affect which wands you can use and which feats you can take.
    %             \parhead{Suggested Feats} Sphere Focus: Aeromancy, Aquamancy, Pyromancy, or Terramancy
    %             \parhead{Combat Tactics} You are a master of all four elements, so you have an immense variety of options available to you - if you choose the right spells.
    %             You have a very high accuracy thanks to your Perception and a reasonably high power, so your primary role in combat will usually be to deploy the perfect damaging spell or debuff for the situation.
    %             Your skills and Elementalist abilities give you a lot of narrative power, so stay alert for opportunities to overcome challenges without needing to fight at all.

    %         \subsubsection{Nature Magic}
    %             \parhead{Species} Elf.
    %             % 0 3 0 0 3 2 base
    %             \parhead{Attributes} 0 Str, 3 Dex, -1 Con, 0 Int, 4 Per, 2 Wil (after species modifiers).
    %             \parhead{Class} Druid.
    %             \parhead{Archetypes} Nature Magic first, Nature Spell Mastery second, Elementalist third.
    %             \parhead{Insight Points} 1 point (unspent).
    %             \parhead{Skills} Awareness, Creature Handling, Knowledge (nature), Survival
    %             \parhead{Languages} Common, Elven, Halfling.
    %             \parhead{Equipment} Sickle, buckler, leather armor. As you gain levels, use the best light armor you can afford.
    %             \parhead{Legacy Item} Implement.
    %                 At level 6, choose \mitem{extending staff}.
    %                 At level 12, upgrade to \mitem{selective staff}.
    %                 At level 18, upgrade to \mitem{extending staff+}.
    %             \parhead{Mystic Spheres} Toxicology.
    %             \parhead{Suggested Feats} Sphere Focus: Verdamancy, Sphere Focus: Aquamancy, Herbalist
    %             \parhead{Combat Tactics} You are a master of plants and natural magic.
    %             Your spells excel at constraining and debilitating your foes, especially with poisons.
    %             Larger foes tend to be resistant to both poisons and movement impediments, so it's a good idea to have Reflex attacks like \spell{ensnaring grasp} and \spell{fire seeds} to deal with them.

    %         \subsubsection{Nature Spell Mastery}
    %             Use the typical character for the Nature Magic druid archetype.
    %             Even if you focus on spells through this archetype, you should generally still rank up your spells before improving your rank in this archetype.

    %         \subsubsection{Shifter}
    %             \parhead{Species} Human.
    %             % 3 2 2 1 0 0 base
    %             \parhead{Attributes} 3 Str, 3 Dex, 3 Con, 1 Int, 0 Per, 0 Wil.
    %             \parhead{Class} Druid.
    %             \parhead{Archetypes} Shifter first, Nature Magic second, Wildspeaker third.
    %             \parhead{Insight Points} 2 points.
    %             \parhead{Skills} Awareness, Balance, Climb, Stealth, Survival
    %             \parhead{Languages} Common, Sylvan
    %             \parhead{Equipment} Natural weapon, buckler, chain shirt. As you gain levels, use the best light armor you can afford.
    %             \parhead{Legacy Item} Armor.
    %                 At level 6, choose \mitem{belt of constitution}.
    %                 At level 12, upgrade to \mitem{sprinting boots}.
    %                 At level 18, upgrade to \mitem{boots of freedom+}.
    %             \parhead{Mystic Sphere} Polymorph
    %             \parhead{Suggested Wild Aspects} Your choice of wild aspects has a significant effect on your capabilities, so choose wild aspects that match your goals.
    %             The Bear, Viper, and Wolf forms excel at dealing damage in combat.
    %             The Bull and Constrictor forms improve your ability to take unusual combat actions.
    %             Other forms can be useful in specific circumstances and out of combat.
    %             \parhead{Suggested Feats} Sphere Focus: Polymorph, Regenerator, Brawler, Savage
    %             \parhead{Combat Tactics} You are a lethal blend of claws and teeth.
    %             You can shift your form to gain the perfect abilities for your current circumstances, and your high physical attributes make you hard to kill and hard to ignore.
    %             Your flexibility between natural weapons, spells, and high physical skills give you a lot of options in and out of combat.
    %             In general, you do the most damage in close quarters where you can attack with your natural weapons, but you can use your spells to soften up strong enemies and finish off weakened enemies.

    %         \subsubsection{Wildspeaker}
    %             \parhead{Species} Gnome.
    %             \parhead{Attributes} -1 Str, 0 Dex, 4 Con, 0 Int, 3 Per, 3 Wil.
    %             \parhead{Class} Druid.
    %             \parhead{Archetypes} Wildspeaker first, Nature Magic second, Nature Spell Mastery third.
    %             \parhead{Insight Points} 1 point.
    %             \parhead{Skills} Awareness, Creature Handling, Knowledge (nature), Survival
    %             \parhead{Languages} Common, Gnome, Sylvan
    %             \parhead{Equipment} Sickle, buckler, scale mail. As you gain levels, use the best medium armor you can afford.
    %             \parhead{Legacy Item} Apparel.
    %                 At level 6, choose \mitem{amulet of sturdy companionship}.
    %                 At level 12, upgrade to \mitem{amulet of sturdy companionship+}.
    %                 At level 18, upgrade to \mitem{amulet of sturdy companionship++}.
    %             \parhead{Mystic Sphere} Electromancy
    %             \parhead{Suggested Feats} Sphere Focus: Electromancy, Ride Specialization, Creature Handling Specialization, Toughness
    %             \parhead{Combat Tactics} You lead your faithful natural servant in battle.
    %             It distracts your enemies while you blast them with lightning from afar.
    %             Once you get a \textit{shrinking belt} or some other way to shrink yourself, you can ride your \textit{natural servant} into battle, which compensates for your short gnomish legs.
    %             If you are both lucky and persuasive, you be able to use your \textit{speak with animals} ability to convince an animal to aid you on your journey, at least for a short time, in addition to your \textit{natural servant}.

    %     \subsection{Fighter}

    %         \subsubsection{Combat Discipline}
    %             \parhead{Species} Dwarf.
    %             % 3 0 3 1 0 1 base
    %             \parhead{Attributes} 3 Str, -1 Dex, 4 Con, 1 Int, 0 Per, 2 Wil (after species modifiers).
    %             \parhead{Class} Fighter.
    %             \parhead{Archetypes} Combat Discipline first, Martial Mastery second, Sentinel third.
    %             \parhead{Insight Points} 2 points.
    %             \parhead{Skills} Awareness, Climb, Endurance, Swim
    %             \parhead{Languages} Common, Dwarven, Orcish.
    %             \parhead{Equipment} Battleaxe, standard shield, scale mail. As you gain levels, buy a heavy crossbow and use the best heavy armor you can afford.
    %             You can switch between a shepherd's axe for hard to hit enemies, a battleaxe for multi-enemy fights or fights where you need the extra damage from holding it in two hands, and throwing axes when you need a ranged weapon.
    %             \parhead{Legacy Item} Shield.
    %                 At level 6, choose \mitem{shield of arrow catching}.
    %                 At level 12, upgrade to \mitem{shield of arrow catching++}.
    %                 At level 18, choose a suitable Rank 7 shield property (e.g., \mitem{supreme shield of deflection}).
    %             \parhead{Combat Styles} Rip and Tear, Unbreakable Defense
    %             \parhead{Suggested Feats} Shieldbearer, Toughness, Regenerator
    %             \parhead{Combat Tactics} You are extremely difficult to kill, and your ability to ignore and remove conditions makes it hard for your foes to whittle you down over time.
    %             You can charge confidently into the middle of battle, cutting down enemy ranged attackers regardless of their surrounding allies.
    %             Alternately, you can hold the line to protect your own allies.

    %         \subsubsection{Equipment Training}
    %             \parhead{Species} Halfling.
    %             % 0 2 2 1 2 1 base
    %             \parhead{Attributes} -1 Str, 3 Dex, 2 Con, 1 Int, 2 Per, 1 Wil.
    %             \parhead{Class} Fighter.
    %             \parhead{Archetypes} Equipment Training first, Martial Mastery second, Combat Discipline third.
    %             \parhead{Insight Points} 2 points.
    %             \parhead{Skills} Awareness, Balance, Flexibility, Stealth
    %             \parhead{Languages} Common, Gnomish, Halfling
    %             \parhead{Equipment} Kukri, standard shield, scale mail. As you gain levels, buy a longbow and use the best armor you can afford that allows you to apply your full Dexterity bonus to your Armor defense.
    %                 Keep an extra kukri with you so you can dual wield in fights where you don't need to use a shield.
    %             \parhead{Legacy Item} Weapon.
    %                 At level 6, choose \mitem{seeking}.
    %                 At level 12, upgrade to \mitem{seeking+}.
    %                 At level 18, choose a suitable Rank 7 weapon property (e.g., \mitem{honed+}).
    %             \parhead{Combat Styles} Flurry of Blows, Rip and Tear
    %             \parhead{Suggested Feats} Swiftrunner, Rapid Reaction, Executioner, Two-Weapon Fighting
    %             \parhead{Combat Tactics} You have an exceptionally high Armor defense, and your strikes are very accurate.
    %             However, your low Strength and small weapons mean that you don't deal a lot of damage.
    %             Focus on debilitating maneuvers like \maneuver{brow gash} or maneuvers that increase your damage like \maneuver{tear exposed flesh} to stay relevant in combat.
    %             When your Armor defense isn't as important, such as when fighting spellcasters, you can dual wield to increase your damage.
    %             Unlike most fighters, you are very agile and stealthy, so you can accompany rogues on scouting missions.

    %         \subsubsection{Martial Mastery}
    %             \parhead{Species} Human.
    %             % 3 0 2 1 2 0 base
    %             \parhead{Attributes} 3 Str, 0 Dex, 3 Con, 1 Int, 3 Per, 0 Wil.
    %             \parhead{Class} Fighter.
    %             \parhead{Archetypes} Martial Mastery first, Combat Discipline second, Tactician third.
    %             \parhead{Insight Points} 2 points.
    %             \parhead{Skills} Awareness, Climb, Endurance, Swim
    %             \parhead{Languages} Common, Giantish, Orcish.
    %             \parhead{Equipment} Broadsword, standard shield, scale mail. As you gain levels, buy throwing axes and use the best heavy armor you can afford.
    %             \parhead{Legacy Item} Armor.
    %                 At level 6, choose \mitem{lifeweave armor}.
    %                 At level 12, upgrade to \mitem{lifeweave armor+}.
    %                 At level 18, upgrade to \mitem{lifeweave armor++}.
    %             \parhead{Combat Styles} Ebb and Flow, Herald of War, Rip and Tear
    %             \parhead{Suggested Feats} Executioner, Blindfighter, Toughness
    %             \parhead{Combat Tactics} You have great versatility in combat.
    %             You have a large number of manuvers, and by mixing in thrown weapons and shouts, you can attack at multiple ranges and hit multiple defenses.
    %             When your shield is unnecessary, you can hold your broadsword in two hands to improve your already respectable damage.
    %             Many of your maneuvers work at any range, so you aren't forced to fight in melee against highly mobile or excessively lethal foes.
    %             In addition to using maneuvers, you can coordinate your allies with battle tactics.

    %         \subsubsection{Sentinel}
    %             \parhead{Species} Dwarf.
    %             % 3 0 2 0 3 0 base
    %             \parhead{Attributes} 3 Str, -1 Dex, 3 Con, 0 Int, 3 Per, 1 Wil.
    %             \parhead{Class} Fighter.
    %             \parhead{Archetypes} Sentinel first, Martial Mastery second, Equipment Training third.
    %             \parhead{Insight Points} 1 point.
    %             \parhead{Skills} Awareness, Endurance, Intimidate
    %             \parhead{Languages} Common, Dwarven, Orcish
    %             \parhead{Equipment} Shepherd's axe, standard shield, scale mail. As you gain levels, buy a throwing axes, a heavy crossbow, and a greataxe and use the best heavy armor you can afford.
    %             \parhead{Legacy Item} Apparel.
    %                 At level 6, choose \mitem{distant protector's amulet}.
    %                 At level 12, upgrade to \mitem{boots of speed}.
    %                 At level 18, upgrade to \mitem{distant protector's amulet+}.
    %             \parhead{Combat Styles} Unbreakable Defense
    %             \parhead{Suggested Feats} Shieldbearer, Toughness, Regenerator
    %             \parhead{Combat Tactics} You hold the line in the middle of the fray, protecting your allies all over the battlefield.
    %             You have an unmatched ability to constrain your foes' movement and force them to pay attention to you, limiting their ability to harm your allies.
    %             Your damage is reasonable, but your main focus should be on defense so you can protect your more vulnerable and reckless allies.
    %             A shepherd's axe is convenient because it allows you to attack foes who try to keep their distance from you.
    %             However, when you need to deal damage, consider switching to a greataxe or dual-wielding a second shepherd's axe instead of holding a shield.

    %         \subsubsection{Tactician}
    %             \parhead{Species} Human.
    %             % 2 2 1 1 2 0 base
    %             \parhead{Attributes} 2 Str, 3 Dex, 1 Con, 2 Int, 2 Per, 0 Wil.
    %             \parhead{Class} Fighter.
    %             \parhead{Archetypes} Tactician first, Martial Mastery second, Equipment Training third.
    %             \parhead{Insight Points} 2 points (1 for Mobile Hunter combat style, 1 unspent).
    %             % 3 fighter + 2 int + 1 human = 6 trained
    %             \parhead{Skills} Awareness, Deduction, Endurance, Knowledge (dungeoneering, local), Medicine
    %             \parhead{Languages} Common, Draconic, Elven
    %             \parhead{Equipment} Longhammer, standard shield, chain shirt. As you gain levels, buy a longbow and use the best armor you can afford that allows you to apply your full Dexterity bonus to your Armor defense.
    %             \parhead{Legacy Item} Armor.
    %                 At level 6, choose \mitem{featherlight armor+}.
    %                 At level 12, upgrade to \mitem{swiftstep armor}.
    %                 At level 18, upgrade to \mitem{featherlight armor++}.
    %             \parhead{Combat Styles} Brute Force, Mobile Hunter
    %             \parhead{Suggested Feats} Precognition, Leadership, Medicine Specialization
    %             \parhead{Combat Tactics} You have a wealth of options in combat.
    %             You can buff your allies' attacks, defend your allies, deal damage, or debuff your foes.
    %             Use whichever battle tactics are most relevant to the current situation.
    %             Your longhammer and high mobility allows you to keep your distance in combat while knocking your foes into more tactically advantageous positions.
    %             Since you don't use a shield, you don't have the defensive power of most fighters, so you can't just charge heedlessly into the fray.

    %     \subsection{Monk}

    %         \subsubsection{Airdancer}
    %             \parhead{Species} Elf.
    %             % 2 2 0 0 2 2 base
    %             \parhead{Attributes} 2 Str, 3 Dex, -1 Con, 0 Int, 2 Per, 2 Wil.
    %             \parhead{Class} Monk.
    %             \parhead{Archetypes} Airdancer first, Esoteric Warrior second, Ki third.
    %             \parhead{Insight Points} 1 point.
    %             \parhead{Skills} Balance, Climb, Flexibility, Stealth, Swim
    %             \parhead{Languages} Common, Elven, Halfling
    %             \parhead{Equipment} Two jitte. Use your \textit{ki barrier} for your body armor.
    %             \parhead{Legacy Item} Apparel.
    %                 At level 6, choose \mitem{boots of reliable motion}.
    %                 At level 12, upgrade to \mitem{boots of reliable motion+}.
    %                 At level 18, upgrade to \mitem{gloves of dexterity+}.
    %             \parhead{Combat Styles} Perfect Precision
    %             \parhead{Suggested Feats} Balance Specialization, Swiftrunner, Two-Weapon Fighting
    %             \parhead{Combat Tactics} You are highly acrobatic in combat, leaping around your opponents with ease.
    %             Once you can jump over enemies, you can start ignoring attempts to block your movement.
    %             The Leap of the Heavens \textit{ki manifestation} can help you reach that point quickly.
    %             You are highly accurate, and your high Dexterity helps both your defenses and your skills.
    %             If you are in physical danger, you can sheathe one of your kamas and use your \textit{ki barrier} as a shield to further increase your Armor defense.
    %             However, your middling Constitution means you should pick your fights carefully.
    %             Use your mobility to pick your battles and avoid being surrounded unnecessarily.

    %         \subsubsection{Esoteric Warrior}
    %             \parhead{Species} Human
    %             % 2 2 2 1 1 0 base
    %             \parhead{Attributes} 2 Str, 3 Dex, 2 Con, 1 Int, 2 Per, 0 Wil.
    %             \parhead{Class} Monk.
    %             \parhead{Archetypes} Esoteric Warrior first, Perfected Form second, Transcendent Sage third.
    %             \parhead{Insight Points} 2 points (1 for Flurry of Blows combat style, 1 point unspent).
    %             \parhead{Skills} Awareness, Balance, Climb, Flexibility, Endurance, Stealth
    %             \parhead{Languages} Common, Draconic, Elven
    %             \parhead{Equipment} Two kunai, chain shirt. As you gain levels, buy spare kunai, and use the best light armor you can afford.
    %             \parhead{Legacy Item} Apparel.
    %                 At level 6, choose \mitem{gloves of dexterity}.
    %                 At level 12, upgrade to \mitem{boots of speed}.
    %                 At level 18, upgrade to \mitem{gloves of dexterity+}.
    %             \parhead{Combat Styles} Dirty Fighting, Flurry of Blows
    %             \parhead{Suggested Feats} Brawler, Juggernaut, Swiftrunner, Two-Weapon Fighting
    %             \parhead{Combat Tactics} You can beat your opponents to death with nothing more than your bare hands.
    %             Your primary combat strategy is generally to grapple, trip, or otherwise debuff your opponents with your free hands before you pummel them into submission.
    %             You have a high movement speed, and you can take advantage of that by rushing down enemies who would prefer to keep their distance.
    %             When that is combined with your immunity to many common debuffs, you are exceptionally effective against enemy spellcasters.
    %             If you find yourself fighting more martially skilled foes, you may need to keep your distance with kunai or a bow, at least until they are weakened.

    %         \subsubsection{Ki}
    %             \parhead{Species} Halfling
    %             % 0 2 3 1 0 2 base
    %             \parhead{Attributes} -1 Str, 2 Dex, 3 Con, 1 Int, 0 Per, 3 Wil.
    %             \parhead{Class} Monk.
    %             \parhead{Archetypes} Ki first, Esoteric Warrior second, Transcendent Sage third.
    %             \parhead{Insight Points} 2 points.
    %             \parhead{Skills} Balance, Flexibility, Endurance, Knowledge (arcana), Survival, Stealth
    %             \parhead{Languages} Common, Draconic, Elven
    %             \parhead{Equipment} Two kama. Use your \textit{ki barrier} for your body armor. As you gain levels, buy spare kunai.
    %             \parhead{Legacy Item} Weapon.
    %                 At level 6, choose \mitem{bloodfuel}.
    %                 At level 12, upgrade to \mitem{honed}.
    %                 At level 18, upgrade to \mitem{honed+}.
    %             \parhead{Combat Styles} Flurry of Blows, Mobile Hunter
    %             \parhead{Suggested Feats} Two-Weapon Fighting, Ghostblade, Iron Will, Spellwarped
    %             \parhead{Combat Tactics} Although you appear small and physically weak, your attacks hit hard thanks to your \textit{ki energy} ability.
    %             You can use a variety of ki manifestations to have surprising effects in combat.
    %             Look for tricky combinations, like tripping your foes at a distance with \ability{extend the flow of ki} or using \ability{burst of blinding speed} to increase the power of movement-based maneuvers.
    %             You can also use your ki manifestations to augment your skills in non-combat situations.
    %             Your defenses are high and well-rounded, and you have immunities to a variety of common debuffs, so you can fight aggressively in combat.
    %             Generally, you should dual-wield kama, but you can drop to a single kama if you need more Armor defense, or you can switch to throwing kunai to hit distant foes.

    %         \subsubsection{Perfected Form}
    %             \parhead{Species} Human
    %             % 3 2 2 1 0 0 base
    %             \parhead{Attributes} 3 Str, 3 Dex, 3 Con, 1 Int, 0 Per, 0 Wil.
    %             \parhead{Class} Monk.
    %             \parhead{Archetypes} Perfected Form first, Esoteric Warrior second, Airdancer third.
    %             \parhead{Insight Points} 2 points (1 for Mobile Hunter combat style, 1 point unspent).
    %             \parhead{Skills} Balance, Climb, Flexibility, Endurance, Stealth, Swim
    %             \parhead{Languages} Common, Giantish, Orcish
    %             \parhead{Equipment} Two kunai, chain shirt. As you gain levels, buy spare kunai, and use the best light armor you can afford.
    %             \parhead{Legacy Item} Apparel.
    %                 At level 6, choose \mitem{gloves of dexterity}.
    %                 At level 12, upgrade to \mitem{boots of speed}.
    %                 At level 18, upgrade to \mitem{gloves of dexterity+}.
    %             \parhead{Combat Styles} Dirty Fighting, Mobile Hunter
    %             \parhead{Suggested Feats} Brawler, Juggernaut, Swiftrunner, Two-Weapon Fighting
    %             \parhead{Combat Tactics} Your general fighting style is the same as the Esoteric Warrior sample character.
    %             Your main differentiating factor is that you have even greater martial aptitude and durability.
    %             However, you are less able to avoid to debilitating conditions, so be careful when fighting magical foes.

    %         \subsubsection{Transcendent Sage}
    %             The transcendent sage archetype by itself does not strongly influence your character's fighting style or abilities.
    %             It provides a variety of passive abilities and immunities that require other archetypes to make a compelling character concept.
    %             A typical character focusing on this archetype would be similar to the Ki character.
    %             If you want to be more martially inclined, follow the Esoteric Warrior character.

    %     \subsection{Paladin}

    %         \subsubsection{Devoted Paragon}
    %             The devoted paragon archetype can have different play styles based on your devoted alignment.
    %             The character below is reasonable for any alignment other than evil.
    %             An evil devoted paragon might focus more on inflicting debilitating conditions with spells from the \sphere{enchantment} or \sphere{vivimancy} \glossterm{mystic spheres}.

    %             \parhead{Species} Dwarf
    %             % 2 0 3 0 2 1 base
    %             \parhead{Attributes} 2 Str, -1 Dex, 4 Con, 0 Int, 2 Per, 2 Wil.
    %             \parhead{Class} Paladin.
    %             \parhead{Archetypes} Devoted Paragon first, Divine Magic second, Stalwart Guardian third.
    %             \parhead{Insight Points} 1 point.
    %             \parhead{Skills} Endurance, Persuasion, Social Insight
    %             \parhead{Languages} Common, Dwarven, Giantish
    %             \parhead{Equipment} Morning star, standard shield, scale mail. As you gain levels, buy a heavy crossbow and use the best heavy armor you can afford.
    %             \parhead{Legacy Item} Shield.
    %                 At level 6, choose \mitem{covering shield}.
    %                 At level 12, upgrade to \mitem{soulguard shield}.
    %                 At level 18, upgrade to \mitem{shield of mystic reflection}.
    %             \parhead{Mystic Sphere} Prayer.
    %             \parhead{Suggested Feats} Leadership, Weapon Focus, Shieldbearer, Sphere Focus: Prayer
    %             \parhead{Combat Tactics} You are a beacon to guide your allies in combat.
    %             You should stay close to the front lines to ensure that your allies gain the benefit of your \textit{aligned aura} ability and take advantage of your reasonable melee damage.
    %             You are very durable, and you can blanket your allies with powerful buffs from the \sphere{prayer} mystic sphere.
    %             At high levels, make sure you have a special strike ability like \spell{exalted strike} or a \mitem{powerstrike} weapon.

    %         \subsubsection{Divine Magic}

    %             \parhead{Species} Human
    %             % 1 0 2 2 1 2 base
    %             \parhead{Attributes} 1 Str, 0 Dex, 2 Con, 3 Int, 1 Per, 3 Wil.
    %             \parhead{Class} Paladin.
    %             \parhead{Archetypes} Divine Magic first, Divine Spell Expertise second, Devoted Paragon third.
    %             \parhead{Insight Points} 3 points (2 points for an additional mystic sphere (Vivimancy), 1 point for an additional rank 1 spell).
    %             \parhead{Skills} Awareness, Deduction, Endurance, Knowledge (local, religion), Medicine, Persuasion
    %             \parhead{Languages} Common, Dwarven, Giantish
    %             \parhead{Equipment} Battleaxe, standard shield, scale mail. As you gain levels, use the best heavy armor you can afford and switch to using an implement instead of a weapon.
    %             \parhead{Legacy Item} Implement.
    %                 At level 6, choose \mitem{educated staff}.
    %                 At level 12, upgrade to \mitem{educated staff+}.
    %                 At level 18, upgrade to \mitem{educated staff++}.
    %             \parhead{Mystic Sphere} Channel Divinity, Vivimancy.
    %             \parhead{Suggested Feats} Sphere Focus: Channel Divinity, Sphere Focus: Vivimancy, Boongiver
    %             \parhead{Combat Tactics} Paladins make unusual spellcasters.
    %             They lack the flexibility of more dedicated spellcasting classes because they lack the \ability{metamagic} ability and access to a variety of mystic spheres.
    %             However, your \textit{divine spell versatility} and \textit{divine conduit} give you a unique combat style that rewards staying directly on the front lines.
    %             In addition, you have an unusually high power thanks to \textit{wellspring of power} and \textit{paragon power}, so your damage and healing spells can be quite potent.
    %             Make sure you have single-target damage spells like \spell{mystic bolt} and \spell{lifesteal} to make the most of your abilities.

    %         \subsubsection{Divine Spell Expertise}
    %             Use the typical character for the Divine Magic paladin archetype.
    %             Even if you focus on spells through this archetype, you should generally still rank up your spells before improving your rank in this archetype.

    %         \subsubsection{Stalwart Guardian}

    %             \parhead{Species} Halfling
    %             % 0 3 3 0 0 2 base
    %             \parhead{Attributes} -1 Str, 3 Dex, 3 Con, 0 Int, 0 Per, 3 Wil (after species modifiers).
    %             \parhead{Class} Paladin.
    %             \parhead{Archetypes} Stalwart Guardian first, Divine Magic second, Zealous Warrior third.
    %             \parhead{Insight Points} 1 point.
    %             \parhead{Skills} Endurance, Medicine, Ride
    %             \parhead{Languages} Common, Dwarven, Giantish
    %             \parhead{Equipment} Smallsword, standard shield, scale mail. As you gain levels, buy spare throwing axes and use the best light armor you can afford.
    %             \parhead{Legacy Item} Apparel.
    %                 At level 6, choose \mitem{amulet of divine healing}.
    %                 At level 12, upgrade to \mitem{boots of speed}.
    %                 At level 18, upgrade to \mitem{amulet of divine healing++}.
    %             \parhead{Mystic Sphere} Fabrication.
    %             \parhead{Suggested Feats} Shieldbearer, Sphere Focus: Prayer
    %             \parhead{Combat Tactics} You have exceptionally high defenses, and you can heal your less well protected allies.
    %             Your damage is reasonable thanks to your \ability{smite} ability, which allows you to use your high Willpower in place of your poor Strength.
    %             You can use spells like \spell{mystic barrier} and \spell{blade barrier} to control the battlefield and funnel your foes into positions where you can make the most of your defensive abilities.

    %         \subsubsection{Zealous Warrior}

    %             \parhead{Species} Elf
    %             % 3 0 0 0 2 3 base
    %             \parhead{Attributes} 3 Str, 0 Dex, -1 Con, 0 Int, 3 Per, 3 Wil (after species modifiers).
    %             \parhead{Class} Paladin.
    %             \parhead{Archetypes} Zealous Warrior first, Divine Magic second, Stalwart Guardian third.
    %             \parhead{Insight Points} 1 point.
    %             \parhead{Skills} Awareness, Persuasion, Social Insight
    %             \parhead{Languages} Common, Dwarven, Giantish
    %             \parhead{Equipment} Greatmace, scale mail. As you gain levels, buy a longbow and use the best heavy armor you can afford.
    %             \parhead{Legacy Item} Weapon.
    %                 At level 6, choose \mitem{bloodfuel}.
    %                 At level 12, upgrade to \mitem{honed}.
    %                 At level 18, upgrade to \mitem{honed+}.
    %             \parhead{Mystic Sphere} Channel Divinity.
    %             \parhead{Suggested Feats} Greatweapon Warrior, Spellsword, Celestial Ancestry
    %             \parhead{Combat Tactics} You are an extremely dangerous martial combatant.
    %             Thanks to your \ability{smite} ability, you have high damage and devastatingly powerful critical hits.
    %             Your defenses are unimpressive, but you have enough durability from heavy armor to avoid being too squishy.
    %             The biggest weakness you have is your slow speed and vulnerability to debilitating conditions.
    %             Make sure you take some ranged spells like \spell{retributive judgment} to deal with foes who try to avoid your greatmace.

    %             Alternately, you could focus more heavily on ranged combat, since \ability{smite} can be used with \weapontag{Projectile} weapons.
    %             If that is your goal, consider the Mystic Archer and Sniper feats instead of Greatweapon Warrior and Spellsword.

\sectiongraphic*{Character Advancement and Gaining Levels}{width=\columnwidth}{characters/character advancement}

  As you accomplish challenges and defeats foes, you gain experience.
  If you have enough experience, you gain a level.
  You gain some abilities at specific levels, as described in \trefnp{Character Advancement and Gaining Levels}.

  When you gain a level, the following things happen:
  \begin{raggeditemize}
    \item Your \glossterm{hit points} increase (see \pcref{Hit Points}).
    \item You gain an additional \glossterm{archetype rank} (see \pcref{Archetypes}).
    \item Your \glossterm{accuracy} may increase (see \pcref{Accuracy}).
    \item At even levels, \glossterm{magical power} and \glossterm{mundane power} each increase by 1 (see \pcref{Power}).
    \item At even levels, your bonus with \glossterm{trained skills} increases (see \pcref{Trained Skills}).
    \item At even levels, all of your \glossterm{defenses} increase by 1 (see \pcref{Defenses}).
  \end{raggeditemize}

  In addition, some irregular advancements happen at specific levels:
  \begin{raggeditemize}
    \item At 3rd level, and every 6 levels thereafter, you increase any two of your \glossterm{attributes} by 1 (see \pcref{Attributes}).
    \item At 4th level, and every 3 levels thereafter, your \glossterm{character rank} increases (see \pcref{Character Rank}).
    \item At 6th level, you gain a \glossterm{legacy item} (see \pcref{Legacy Items}).
      Every 6 levels thereafter, your legacy item improves, as shown in \trefnp{Character Advancement and Gaining Levels}.
  \end{raggeditemize}

  These effects are summarized in \trefnp{Character Advancement and Gaining Levels}, which also defines the experience required to gain each level.
  The special effects from gaining levels are also listed in the base class table for each class as a reminder.

  \begin{dtable}
    \lcaption{Character Advancement and Gaining Levels}
    \begin{compresseddtabularx}{\columnwidth}{l l l l X l}
      \tb{Level} & \tb{Rank} & \tb{Durability} & \tb{Bonus}\fn{1} & \tb{Special}             & \tb{XP} \tableheaderrule
      1st        & 1         & \plus0          & \tdash           & \tdash                   & 0      \\
      2nd        & 1         & \plus1          & \plus1           & \tdash                   & 10     \\ % +10 xp
      3rd        & 1         & \plus2          & \plus1           & \plus1 to two attributes & 25     \\ % +15 xp
      4th        & 2         & \plus2          & \plus2           & \tdash                   & 45     \\ % +20 xp
      5th        & 2         & \plus3          & \plus2           & \tdash                   & 70     \\ % +25 xp
      6th        & 2         & \plus4          & \plus3           & Legacy item: rank 3      & 100    \\ % +30 xp
      7th        & 3         & \plus4          & \plus3           & \tdash                   & 140    \\ % +40 xp
      8th        & 3         & \plus5          & \plus4           & \tdash                   & 200    \\ % +60 xp
      9th        & 3         & \plus6          & \plus4           & \plus1 to two attributes & 300    \\ % +100 xp
      10th       & 4         & \plus6          & \plus5           & \tdash                   & 450    \\ % +150 xp
      11th       & 4         & \plus7          & \plus5           & \tdash                   & 700    \\ % +250 xp
      12th       & 4         & \plus8          & \plus6           & Legacy item: rank 5      & 1,000  \\ % +300 xp
      13th       & 5         & \plus8          & \plus6           & \tdash                   & 1,400  \\ % +400 xp
      14th       & 5         & \plus9          & \plus7           & \tdash                   & 2,000  \\ % +600 xp
      15th       & 5         & \plus10         & \plus7           & \plus1 to two attributes & 3,000  \\ % +1000 xp
      16th       & 6         & \plus10         & \plus8           & \tdash                   & 4,500  \\ % +1,500 xp
      17th       & 6         & \plus11         & \plus8           & \tdash                   & 7,000  \\ % +2,500 xp
      18th       & 6         & \plus12         & \plus9           & Legacy item: rank 7      & 10,000 \\ % +3,000 xp
      19th       & 7         & \plus12         & \plus9           & \tdash                   & 14,000 \\
      20th       & 7         & \plus13         & \plus10          & \tdash                   & 20,000 \\
      21st       & 7         & \plus14         & \plus10          & \plus1 to two attributes & 30,000 \\
    \end{compresseddtabularx}
    1. This bonus applies to your \magical power, mundane power, trained skills, and defenses. \\
  \end{dtable}

  At your GM's discretion, you may also change some of the choices you have made about your character when you level up.
  For example, you could change one of your trained skills for a different skill, decrease one attribute to increase another, or change the \glossterm{mystic spheres} you have access to (and corresponding spells).
  The GM may ask for a specific narrative justification for the change, require spending in-game time to retrain, or disallow changing some fundamental aspects of your character.

  \subsection{Character Rank}\label{Character Rank}
    Your character has a \glossterm{character rank}, which is a general indication of their overall power.
    It is determined by your level, as shown in \trefnp{Character Advancement and Gaining Levels}.
    Your maximum rank in any individual \glossterm{archetype} is equal to your character rank (see \pcref{Archetypes}).
    It has no other universal effects, though some abilities reference your character rank to determine their effects.
    

  \subsection{Legacy Items}\label{Legacy Items}

    Over time, items associated with places and people of great power gain magical properties.
    This process takes place for you as you gain levels in addition to in the world as a whole.

    At 6th level, you choose a weapon, body armor, shield, apparel item, or implement you own.
    That item becomes a \glossterm{legacy item}.
    You choose a single magic item property of rank 3 or lower, and your legacy item gains that property.
    You do not have to \glossterm{attune} to your legacy item to gain its benefits.
    However, for each \glossterm{deep attunement} property that your legacy item has, you reduce your maximum \glossterm{attunement points} by one.

    Being a legacy item does not interfere with an item's normal properties, including magical ones.
    If it was already a magic item or made of a special material, it still has those properties in addition to being your legacy item.
    Otherwise, you can imbue it with magic or reforge it to be made of a special material just you could with a mundane item of the same type.

    The property must be appropriate for the category of item you chose: weapon, armor, apparel, or implement.
    You do not have to precisely match the location of an apparel item, just the category.
    For example, you can choose an amulet as your legacy item and give it the effect of the \mitem{boots of translocation}, or apply the effects of a \mitem{hardblock shield} to your body armor.

    \parhead{Legacy Item Scaling}
    Your legacy item increases in power as you gain levels, as described below.
    \begin{raggeditemize}
      \item 6th level: You gain a property with a maximum rank of 3.
      \item 12th level: You can change its property, up to a maximum rank of 5.
      \item 18th level: You can change its property, up to a maximum rank of 7.
    \end{raggeditemize}

    \parhead{Losing Your Legacy Item}
    If you lose your legacy item, you must retrieve it to regain its power.
    There are rituals to facilitate this retrieval such as \ritual{seek legacy} and \ritual{retrieve legacy}.
    If your legacy item is \glossterm{destroyed}, you can designate a new item of the same type to be your legacy item, causing it to gain the effect of your legacy item.
    Designating a new item in this way requires taking a \glossterm{long rest} while holding or wearing the replacement item.

    \parhead{Unique Legacy Items}
    Legacy items are fundamentally a reflection of the character who wields them.
    Their effects can be more unusual and complex than abilities on normal magic items, and they can have a larger effect on the way that character interacts with the world.
    As a player, you can work with your GM to create custom magical effects of an appropriate power that are a better reflection of your character's personality and powers than the magic item abilities that exist.
