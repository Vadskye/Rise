\chapter{Adventuring}\label{Adventuring}

\sectiongraphic*{Resting}{width=\columnwidth}{core mechanics/resting}

  When you have a moment to relax, you can rest to regain some of your expended resources.
  There are two main types of rests: a \glossterm{short rest} and a \glossterm{long rest}.
  The benefits of taking a short rest or long rest happen automatically after you spend enough time avoiding strenuous activity.
  You do not have to declare that you are using the ``short rest'' or ``long rest'' ability.
  Resting at night is often combined with sleeping, but you can rest at any time without sleeping.

  % TODO: clarify interrupted rests

  \subsection{Short Rest}\label{Short Rest}
    Resting for ten minutes is considered a \glossterm{short rest}.
    When you finish a short rest, you gain the following benefits.
    \begin{raggeditemize}
      \item Your \glossterm{hit points} become equal to your maximum hit points.
      \item You regain any \glossterm{attunement points} you released from \glossterm{deep attunement} effects (see \pcref{Deep Attunement}).
      \item You remove all \glossterm{conditions} affecting you.
      \item Some other abilities have specific effects that last until you finish a short rest.
    \end{raggeditemize}

  \subsection{Long Rest}\label{Long Rest}
    Resting for eight hours is considered a \glossterm{long rest}.
    When you finish a long rest, you gain the following benefits.
    \begin{raggeditemize}
      \item You remove one of your vital wounds (see \pcref{Removing Vital Wounds}).
        The Medicine skill can increase this healing (see \pcref{Accelerate Recovery}).
      \item Your \glossterm{fatigue level} becomes 0.
      \item Some other abilities have specific effects that last until you finish a long rest.
    \end{raggeditemize}

    You can take multiple long rests consecutively to recover from extensive vital wounds.

  \subsection{Sleep and Fatigue}\label{Sleep and Fatigue}
    A typical creature needs a minimum of 6 hours of sleep for every 18 hours spent awake, and a minimum of 50 hours of sleep every week.
    You can stay awake beyond those limits with the Endurance skill (see \pcref{Stay Awake}).

\section{Overland Movement}\label{Overland Movement}
  This section provides rules governing overland movement speeds.
  Not every game should think about overland movement travel speed in a detailed way.
  It's fine to just spend ``a few days'' walking around between various important locations.
  However, sometimes the details are important, such as when you are on a strict timetable.
  The GM can tell you when overland movement matters.

  \subsection{Standard Travel Days}
    Characters covering long distances cross-country use overland movement.
    Overland movement is measured in miles per hour or miles per day.
    A day normally represents 8 hours of actual travel time.
    However, sailing ships and other methods of travel that keep moving without requiring a rest are listed with a full 24 hours of travel time.

    You can make an Endurance check to push beyond a standard 8-hour travel day.
    In addition, you can make an Endurance check to travel faster within a normal travel day.
    For details, see \pcref{Endurance}.

    Standard travel distances on foot are listed in \tref{Travel Distance By Movement Speed}.
    When using mounts or ships, \tref{Mounts and Vehicles} will be more convenient.

    \begin{dtable}
      \lcaption{Travel Distance By Movement Speed}
      \begin{dtabularx}{\columnwidth}{>{\lcol}X c c c c}
        & \multicolumn{4}{c}{\tdash\tdash\tdash Speed \tdash\tdash\tdash} \tableheaderrule
        & 15 feet    & 20 feet  & 30 feet  & 40 feet  \\
        One Hour (Overland) &            &          &          &          \\
        Walk                & 1-1/2 mile & 2 miles  & 3 miles  & 4 miles  \\
        Hustle (Exertion)   & 3 miles    & 4 miles  & 6 miles  & 8 miles  \\
        One Day (Overland)  &            &          &          &          \\
        Walk                & 12 miles   & 16 miles & 24 miles & 30 miles \\
        Hustle (Exertion)   & \tdash     & \tdash   & \tdash   & \tdash   \\
      \end{dtabularx}
    \end{dtable}

    \begin{dtable}
      \lcaption{Mounts and Vehicles}
      \begin{dtabularx}{\columnwidth}{>{\lcol}X l l}
        \tb{Mount/Vehicle}                         & \tb{Per Hour} & \tb{Per Day} \tableheaderrule
        Mount (carrying load)                      &             &          \\
        \tind Light horse or warhorse              & 5 miles     & 40 miles \\
        \tind Draft horse                          & 4 miles     & 32 miles \\
        \tind Pony or war pony                      & 4 miles     & 32 miles \\
        \tind Donkey or mule                       & 3 miles     & 24 miles \\
        \tind Dog, riding                          & 4 miles     & 32 miles \\
        \tind Cart or wagon                        & 2 miles     & 16 miles \\
        \tb{Ship}                                  &             &          \\
        \tind Raft or barge (poled or towed)\fn{1} & 1/2 mile    & 4 miles  \\
        \tind Keelboat (rowed)\fn{1}               & 1 mile      & 8 miles \\
        \tind Rowboat (rowed)\fn{1}                & 1-1/2 miles & 12 miles \\
        \tind Sailing ship (sailed)                & 2 miles     & 48 miles \\
        \tind Warship (sailed and rowed)           & 2-1/2 miles & 60 miles \\
        \tind Longship (sailed and rowed)          & 3 miles     & 72 miles \\
        \tind Galley (rowed and sailed)            & 4 miles     & 96 miles \\
      \end{dtabularx}
      1 Rafts, barges, keelboats, and rowboats are used on lakes and rivers.
      If going downstream, add the speed of the current (typically 3 miles per hour) to the speed of the vehicle. In addition to 10 hours of being rowed, the vehicle can also float an additional 14 hours, if someone can guide it, so add an additional 42 miles to the daily distance traveled. These vehicles can't be rowed against any significant current, but they can be pulled upstream by draft animals on the shores.
    \end{dtable}

  \subsection{Overland Terrain}
    Travelling over a flat, paved highway is much faster than trailblazing through a jungle.
    You can use \tref{Terrain and Overland Movement} as a reference for common terrain.

    A highway is a straight, major, paved road.
    A road is typically a dirt track.
    A trail is like a road, except that it allows only single-file travel and does not benefit a party traveling with vehicles.
    Trackless terrain is a wild area with no significant paths.

    \begin{dtable}
      \lcaption{Terrain and Overland Movement}
      \begin{dtabularx}{\columnwidth}{>{\lcol}X c c c}
        \tb{Terrain}   & \tb{Highway} & \tb{Road or Trail} & \tb{Trackless} \tableheaderrule
        Desert, sandy  & \mult1       & \mult1/2           & \mult1/2 \\
        Forest         & \mult1       & \mult1             & \mult1/2 \\
        Hills          & \mult1       & \mult3/4           & \mult1/2 \\
        Jungle         & \mult1       & \mult3/4           & \mult1/4 \\
        Moor           & \mult1       & \mult1             & \mult3/4 \\
        Mountains      & \mult3/4     & \mult3/4           & \mult1/2 \\
        Plains         & \mult1       & \mult1             & \mult3/4 \\
        Swamp          & \mult1       & \mult3/4           & \mult1/2 \\
        Tundra, frozen & \mult1       & \mult3/4           & \mult3/4
      \end{dtabularx}
    \end{dtable}

\section{Communication and Languages}\label{Languages}\label{Communication and Languages}

  \parhead{Literacy}
  All characters with an Intelligence of \minus2 or higher are presumed to be literate, allowing them to read and write any language they speak. Each language has an alphabet, though sometimes several spoken languages share a single alphabet.

  \parhead{Language Rarity}\nonsectionlabel{Language Rarity}
  Some languages are widely spoken in the world, while others are only encountered in unusual circumstances.
  Common languages are summarized on \trefnp{Common Languages}, below.
  Rare languages are summarized on \trefnp{Rare Languages}, below.
  Rare languages are more difficult to learn, and are usually only spoken by unusual creatures.

  \parhead{Learning Languages}\nonsectionlabel{Learning Languages}
  Learning a language is a time-consuming process, and most characters only know a few languages based on their species.
  You can learn two common languages or one rare language in place of training a skill (see \pcref{Skills}).
  In addition, you can talk to your GM about knowing an additional language based on your character's personal background.

  \begin{dtable}
    \lcaption{Common Languages}
    \begin{dtabularx}{\columnwidth}{l >{\lcol}X l}
      \tb{Language} & \tb{Typical Speakers} & \tb{Alphabet} \tableheaderrule
      Common        & Civilized creatures   & Common   \\
      Draconic      & Dragons, kobolds      & Draconic \\
      Dwarven       & Dwarves               & Dwarven  \\
      Elven         & Elves                 & Elven    \\
      Giantish      & Ogres, giants         & Dwarven  \\
      Gnoll         & Gnolls                & Common   \\
      Gnome         & Gnomes                & Dwarven  \\
      Goblin        & Goblins, hobgoblins   & Dwarven  \\
      Halfling      & Halflings             & Common   \\
      Orcish        & Orcs                  & Dwarven  \\
    \end{dtabularx}
  \end{dtable}

  \begin{dtable}
    \lcaption{Rare Languages}
    \begin{dtabularx}{\columnwidth}{l >{\lcol}X l}
      \tb{Language}  & \tb{Typical Speakers}  & \tb{Alphabet} \tableheaderrule
      Abyssal     & Evil planeforged      & Abyssal  \\
      Aquan       & Water-based creatures & Elemental \\
      Auran       & Air-based creatures   & Elemental \\
      Celestial   & Good planeforged      & Celestial \\
      Ignan       & Fire-based creatures  & Elemental \\
      Sylvan      & Dryads, faeries       & Elven     \\
      Terran      & Earth-based creatures & Elemental \\
      Undercommon & Drow                  & Elven
    \end{dtabularx}
  \end{dtable}

\section{Common Magical Effects}

  \subsection{Resurrection}\label{Resurrection}
    Some abilities can return dead creatures to life.
    This is called resurrection.

    When a living creature dies, its soul departs its body, travels through the Astral Plane, and goes to abide on the appropriate plane or Divine Realm.
    This process is enforced by the Nature, who guides souls to ensure they reach their intended destinations.
    Bringing a creature back from the dead means retrieving their soul and returning it to their body.

    A creature has no hit points when it returns to life.
    It is cured of all \glossterm{vital wounds}, \glossterm{conditions}, and other negative effects, but the body's shape is unchanged.
    Any missing or irreparably damaged limbs or organs remain missing or damaged.
    The creature may therefore die shortly after being resurrected if its body is excessively damaged.
    Some resurrection abilities can restore more damaged corpses to life, as indicated in their descriptions.

    Coming back from the dead is an ordeal.
    The creature's maximum \glossterm{fatigue tolerance} is reduced by 2.
    This penalty stacks if the creature is resurrected multiple times.
    Every thirty days, and whenever the creature gains a level, this penalty is reduced by 1.
    % TODO: it's odd that this makes resurrection Con-based, when it seems Willpower-based conceptually?
    If this penalty would reduce a creature's maximum fatigue tolerance below 0, the creature cannot be resurrected.

    Resurrection is always voluntary.
    A soul infallibly knows the name, alignment, and patron deity (if any) of the character attempting to revive it and may refuse to return on that basis.
    If a dead creature's soul refuses to return to life, no effect can compel it to be resurrected.
    Similarly, if a dead creature's soul has been subsumed into the planar essence of its afterlife plane, it has already been resurrected, or the soul is otherwise inaccessible, resurrection is impossible.

    Although you can resurrect creatures who have died of old age, it is usually pointless.
    They will die again before long from some malady resulting from their advanced age.

    \subsubsection{Limits of Resurrection}
      While dead, souls gradually lose their cohesion and independent sense of self.
      A typical creature can maintain its existence in the afterlife for a number of years equal to 5 times the sum of its level and Willpower.
      This can vary significantly for individual creatures, and being tormented in the afterlife can significantly reduce this time.

      Enemies can take steps to make it more difficult for a character to be returned from the dead.
      Except for \spell{true resurrection}, every ritual to raise the dead requires some part of the creature's body, so keeping or destroying the body is an effective deterrent.
      The \spell{soul bind} ritual prevents any sort of revivification unless the soul is first released.

  \subsection{Shapeshifting}\label{Shapeshifting}
    When a creature shapeshifts, its physical body completely transforms into a different shape.
    It generally retains all of its original statistics and abilities, with the following exceptions.
    Some specific abilities that cause a creature to shapeshift have additional effects.
    % TODO: are more exceptions necessary?
    \begin{raggeditemize}
      \item The creature's size changes to match the new form.
        This can change the creature's \glossterm{base speed}, Reflex defense, and other statistics as normal (see \pcref{Size Categories}).
      \item The creature's \glossterm{mundane} movement modes and natural weapons are replaced with the movement modes and natural weapons of its new shape.
        If the new form does not specifically mention a walk speed, it has an average walk speed.
      \item If the new shape is not normally capable of speech, the creature cannot speak.
        This may prevent it from casting spells with \glossterm{verbal components} and using similar abilities.
      \item The creature is limited by the number of \glossterm{free hands} present in the new form.
        In addition, it cannot gain more free hands by shapeshifting than it originally had in its base form.
        Even if you shapeshift to a form with many hands, you do not have the mental coordination necessary to use them all effectively.
      \item Any special properties that a creature had that were originally a result of its pure physical composition change to match its new form.
        For example, a ghost would stop being \trait{incorporeal} if it shapeshifted.
    \end{raggeditemize}

    All of a shapeshifted creature's equipment that is physically incompatible with the creature's new shape meld into its body.
    This does not break \glossterm{attunement}, and the creature still gains the benefit of any magical properties of melded items.
    However, it does not gain the benefit of nonmagical properties from melded items.
    For example, a creature that shapeshifts into an amorphous gas would still benefit from all attuned effects from its equipped items, such as \mitem{boots of speed}.
    However, it would gain no benefit to its Armor defense or durability from any melded body armor, and it would not be able to attack with any of its melded weapons.
    Items exceeding a creature's \glossterm{carrying capacity} are not melded, and simply fall to the ground in place.

    When a shapeshifted creature dies, it returns to its original form.

  \subsection{Teleportation}\label{Teleportation}
    Some abilities can \glossterm{teleport} creatures or objects.
    When you are teleported, you move through the Astral Plane and arrive at a new location.
    You can be teleported between two different locations on the same \glossterm{plane}, or between two different locations on different planes.
    If for some reason you cannot access the Astral Plane, you cannot be teleported.

    Anything being teleported must have both \glossterm{line of sight} and \glossterm{line of effect} to its destination.
    In addition, the destination of the teleportation must be an unoccupied location on a stable surface.
    That surface must be able to support the weight of the teleporting creature or object.
    If any of these conditions is not met, the teleportation fails without effect.
    Some teleportation abilities are less restricted, as indicated in their description.

    In general, you can teleport up slopes that are no more than 45 degrees.
    Steeper slopes prevent you from seeing stable ground to teleport to it.
    The GM can provide guidance for individual slopes, which may be easier or harder to navigate with teleportation.

    \subsubsection{Teleportation Noise}\label{Teleportation Noise}
      Creatures and objects that are teleported make a sound when they depart and arrive.
      This noise is caused by the displacement of air (or other substances) created by the teleportation.
      The base \glossterm{difficulty value} of an Awareness check to hear this sound for a Medium creature or object is 10.
      This difficulty value changes based on the size of the teleported creature or object:

      \begin{raggeditemize}
        \item Fine: 30
        \item Diminutive: 25
        \item Tiny: 20
        \item Small: 15
        \item Medium: 10
        \item Large: 5
        \item Huge: 0
        \item Gargantuan: \minus5
        \item Colossal: \minus10
      \end{raggeditemize}

    \subsubsection{Carrying Objects}
      When a creature is teleported, it can bring along equipment and held objects as long as two conditions are met.
      First, the combined weight of the objects cannot exceed the creature's maximum \glossterm{carrying capacity} (see \pcref{Weight Limits}).
      If a creature is teleported while carrying more than its maximum carrying capacity, all excess objects are left behind, starting with the heaviest object and proceeding in order of weight.

      Second, no object can extend more than two feet away from the creature's body.
      Any objects that extend beyond that distance are left behind.
      For example, a creature wearing handcuffs will arrive at its teleportation destination still wearing the handcuffs.
      However, a creature that is tied to a post by a long rope will arrive at its teleportation destination without the rope.

    \subsubsection{Astral Beacons}\label{Astral Beacons}
      Some abilities allow long-distance teleportation, such as the \ritual{overland teleportation} ritual.
      This sort of teleportation is much easier if you are travelling to an \glossterm{astral beacon}.
      The specific effects of an astral beacon are defined in the teleportation ability being used.
      An astral beacon covers an area, rather than a single point in space.

      Each astral beacon has a unique name.
      The name represents the beacon's precise location in the Astral Plane, so no two beacons can have identical names.
      For example, astral beacons created by rituals have their name defined by the precise color of ritual inks, details of drawn patterns, timing and inflection of ritual incantations, and similar subtleties.

      It is possible, though unlikely, to find astral beacons simply by wandering in the Astral Plane.
      They are similar in size and shape to \glossterm{scrying sensors}, but their appearance is visually distinct (for creatures who can see \trait{invisible} objects).
      Inspecting a beacon can reveal the location it points to, and destroying the beacon in the Astral Plane removes its effects.

    \subsubsection{Horizontal Teleportation}
      Some planes have a curved primary surface.
      On those planes, ``horizontal'' teleportation isn't objectively horizontal.
      Instead, it is horizontal relative to the surface of the plane.

\section{Breaking Objects}
  There are two main ways of breaking objects.
  You can deal damage to objects with attacks, similarly to how you can deal damage to creatures.
  Alternately, you can attempt to sunder the object with sheer strength.

  \subsection{Damaging Objects}
    Objects have \glossterm{hit points} like creatures.
    They can also have \glossterm{hardness}, which reduces damage they take.
    Whenever an object takes damage, it first subtracts it hardness from that damage before applying that damage to its hit points.
    Objects are not normally subject to \glossterm{critical hits}.
    When object lose enough hit points, they can become \glossterm{broken} (see \pcref{Broken and Destroyed Objects}).

  \subsection{Sundering Objects}
    As a standard action, you can attempt to break an object with raw strength instead of damage.
    This requires two hands.
    When you sunder an object, make a Strength check.
    The \glossterm{difficulty value} of the check is equal to the object's \glossterm{hardness}, \plus5 for each \glossterm{weight category} above Tiny.
    Success means the object becomes \glossterm{broken}.
    Critical success means the object becomes \glossterm{destroyed}.

  \subsection{Object Statistics}
    An object's size primarily influences the number of \glossterm{hit points} it has.
    The primary material it is constructed from determines its \glossterm{hardness}, and can modify the number of hit points it has.
    Details are given in \tref{Object Statistics By Size} and \tref{Object Statistics By Material}.

    For objects that have two large dimensions and one small dimension, treat their size as being one size category smaller than the bulk of their larger dimensions.
    For objects that have one large dimension and two smaller dimensions, just use the smaller dimensions.
    Consider these examples:
    \begin{raggeditemize}
      \item A 20 foot cube would have the hit points of a Huge object.
      \item A stone pillar that is 20 feet wide, 20 feet long, and 100 feet tall would have the hit points of a Huge object.
      \item A 20 foot tall, 20 foot wide wall of typical thickness would have the hit points of a Large object.
    \end{raggeditemize}

    These rules are more detailed than you should really need.
    During a typical game session, it's often best to just guess whether a character could plausibly sunder or smash an object rather than consulting these tables.

    \begin{dtable}
      \lcaption{Object Statistics By Size}
      \begin{dtabularx}{\textwidth}{l X X}
        \tb{Size}  & \tb{Hit Points} & \tb{Sunder DV Modifier} \tableheaderrule
        Fine       & 2               & 0  \\
        Diminutive & 5               & 0  \\
        Tiny       & 10              & 0  \\
        Small      & 25              & \plus5  \\
        Medium     & 50              & \plus10 \\
        Large      & 100             & \plus15 \\
        Huge       & 250             & \plus20 \\
        Gargantuan & 500             & \plus25 \\
        Colossal   & 1000            & \plus30 \\
      \end{dtabularx}
    \end{dtable}

    \begin{dtable}
      \lcaption{Object Statistics By Material}
      \begin{dtabularx}{\textwidth}{l X X X}
        \tb{Material}   & \tb{Hardness}\fn{1} & \tb{HP Multiplier}\fn{2} \tableheaderrule
        Adamantine      & 30            & \mult3   \\
        Glass           & 5             & \mult1/2 \\
        Ice             & 0             & \mult1/2 \\
        Iron or steel   & 15            & \mult2   \\
        Leather or hide & 5             & \tdash   \\
        Mithral         & 20            & \mult2   \\
        Packed earth    & 5             & \tdash   \\
        Paper or cloth  & 0             & \mult1/2 \\
        Rope            & 5             & \tdash   \\
        Stone           & 10            & \mult2   \\
        Wood            & 5             & \tdash   \\
      \end{dtabularx}
      1. An object's \glossterm{hardness} also increases the difficulty value of checks to sunder it with raw Strength. \\
      2. Any value here modifies the number of hit points the object would normally have based on its size.
    \end{dtable}

  \subsection{Broken and Destroyed Objects}\label{Broken and Destroyed Objects}
    If an object would gain its first \glossterm{vital wound}, typically by reaching negative hit points, it becomes \glossterm{broken} instead.
    Unlike creatures, objects do not reset their negative hit points to 0 at the end of the round or when they are healed.
    Instead, it continues to track its negative hit points indefinitely.
    If its negative hit points reach ten times its maximum hit points, it becomes \glossterm{destroyed}.

    \parhead{Broken Objects}\nonsectionlabel{Broken Objects}
    Broken objects cannot be used for their intended purpose, but still retain enough of their original form to be repaired without too much work.
    For example, a broken wall lies in pieces on the ground and no longer blocks passage, but can be repaired with far less effort than would be required to create a wall from scratch.
    Magic items that are broken retain their magical properties once fixed.
    Broken (but not destroyed) objects can be repaired with the Craft skill (see \pcref{Craft}).

    Most magical effects that create temporary objects are fully destroyed when they become broken.

    \parhead{Destroyed Objects}\nonsectionlabel{Destroyed Objects}
    Destroyed object have been damaged beyond hope of any sort of repair short of crafting the object again from raw materials.
    For example, a destroyed wall is reduced to dust or small, useless chunks of rubble.
    Magic items that are destroyed irrevocably lose their magical properties.
    The remains of a destroyed object generally occupy a space one size category smaller than the original object.
    Destroyed objects can be rebuilt with the Craft skill, but it requires significant time and investment.

  \subsection{Breaking Equipment}\label{Breaking Equipment}
    Normally, a character's equipment cannot be damaged or otherwise affected by attacks.
    This includes worn items, anything held in your hands, and anything in a secure storage like a small backpack.
    Such items are considered \glossterm{attended}.
    They are unaffected by damage caused by area effects, and cannot be targeted individually.
    Some abilities can specifically target \glossterm{attended} objects, as indicated in their descriptions.

    \subsubsection{Loose Equipment}\label{Loose Equipment}
      Some items are explicitly \glossterm{loose equipment}.
      Loose equipment does not gain the protections listed above while worn as equipment.
      It can be individually targeted by attackers, and is affected by area effects just like any other object in the area.

\section{Poison}\label{Poison}
  Poisons are organic substances that are dangerous to living creatures.
  They can deal damage or inflict debilitating effects.
  Some effects which are not literally poisonous, such as animal venom or fungal spores, are considered poisons.
  Poisons are not \glossterm{conditions}, and cannot be removed by abilities that remove conditions (see \pcref{Conditions}).
  Common poisons are listed in \tref{Consumables}.
  You can use the Craft (poison) skill to create poisons (see \pcref{Crafting Items}).

  \subsection{Poison Effects}\label{Poison Effects}
    When you come in contact with a poison, you become \glossterm{poisoned}.
    As soon as you become poisoned, and at the end of each subsequent round, the poison makes an attack against your Fortitude defense.
    On a hit, the poison progresses to its next stage.
    On a critical hit, the poison progresses by two stages at once, to a maximum of the third stage.
    On a miss, you make progress towards removing the poison entirely.
    Once the poison misses you three times, you stop being poisoned.

    If you become poisoned again by the same poison, it does not intensify the effects of the poison.
    However, it cancels any progress you had made towards removing the poison.
    A poison is considered the same if it has the same name.
    If you are affected by multiple different poisons with the same name, but different accuracy bonuses, use the highest accuracy bonus.

    Poisons have no effect on non-living creatures.

    \subsubsection{Poison Stages}
      Poison effects are divided into stages.
      Becoming poisoned does not have any ill effects until the poison progresses to its first stage.

      Many poisons have a Stage 1 effect.
      This effect happens as soon as the poison's attack first succeeds against you.
      Some poisons also have a Stage 3 effect.

    \subsubsection{Poison Accuracy}
      A poison's accuracy depends on the way it was applied.
      Item-based poisons have a specific accuracy listed in their description.
      This accuracy does not depend on the skill of the creature inflicting the poison.

      Poisons inflicted by creature abilities use the creature's \glossterm{accuracy}.
      They may also have additional modifiers listed in the ability's description.
      For monsters, the poison's accuracy will be listed in the monster description.

  \subsection{Poison Transmission}\label{Poison Transmission}\label{Transmission}

    There are three ways that poisons can be contracted.

    \parhead{Contact} A contact poison affects any creature that touches it with bare skin.
    \parhead{Ingestion} An ingestion poison affects any creature that eats, drinks, or breathes it, depending on the type of poison.
    Ingestion poisons have no effect when touched or used to coat weapons.
    This means you can avoid negative effects from ingestion poisons by holding your breath and avoiding eating or drinking.
    \parhead{Injury} An injury poison affects any creature that loses \glossterm{hit points} from something bearing the poison.
    Almost all injury poisons take liquid form, and are typically used to coat weapons.

  \subsection{Poison Forms}\label{Poison Forms}

    There are four forms of poison.
    If a poison can be thrown, throwing one dose of the poison requires a standard action.

    \parhead{Gas} Gaseous poisons are difficult to store, but easy to affect foes with.
    Unless otherwise noted, a gas poison can be thrown at a location within \shortrange, and affects a \tinyarea radius \glossterm{zone}.
    Once active, gas poisons typically linger for one minute before dispersing, though high wind can clear them much more rapidly.
    Walking through the gas cloud will affect creatures with contact poisons, but generally not with ingestion poisons.
    \parhead{Liquid} Liquid poisons are the most common type of poison.
    Liquid poisons can be used to coat weapons, slipped into food, or simply thrown at foes.
    A dose of a liquid poison is usually about one ounce of the poison.
    Unless otherwise noted, a liquid poison can be thrown at something within \shortrange.
    \parhead{Pellet} Some rare poisons come in small, solid pellets or cubes.
    Typically, these pellets contain a powerful liquid poison that becomes inert quickly after being exposed.
    Pellet poisons are typically applied by being slipped into food.
    \parhead{Powder} Poison in powder form cannot be used to coat weapons, but can be slipped into food or thrown at foes.
    % adjacent or 10 feet?
    Unless otherwise noted, a powder poison can be thrown at something within 10 foot range.

  \subsection{Coating Weapons with Poison}\label{Coating Weapons with Poison}
    As a standard action, you can coat a weapon with a single dose of a liquid contact-based or injury-based poison.
    The next time a creature takes damage from a \glossterm{strike} using that weapon, the struck creature comes in contact with the poison.
    This removes one dose of the poison from the weapon.
    Coated poisons expire and lose their effectiveness after ten minutes.

    An injury-based poison has no effect if the strike does not cause an \glossterm{injury}, but the dose is still removed from the weapon.
    For this reason, injury-based poisons are typically applied to secondary weapons that can be used after the subject is already weakened.

    A weapon can hold up to three poison doses of the same poison.
    Mixing different poison types on the same weapon is ineffective, as each poison dilutes the others.
    Only the highest rank poison on the weapon has any effect.

  \subsection{Poison Materials}\label{Poison Materials}
    Creating a poison requires special materials.
    The type of materials required, and how those materials can be acquired, depend on the type of poison:

    \begin{raggeditemize}
      \itemhead{Alchemical} Alchemical poisons require alchemical materials.
        These normally can't be found in nature.
        In unusual circumstances, these components can be synthesized from natural chemicals or magical materials.
        This takes an hour of work, and requires a Craft (alchemy) check equal to 10 \add twice the poison's \glossterm{rank}.
      \itemhead{Plant} Plant-based poisons can typically be harvested by making a Survival \glossterm{extended check} to search in appropriate terrain.
        This usually takes an hour of work.
        The \glossterm{difficulty value} is usually equal to 10 \add twice the poison's \glossterm{rank}.
      \itemhead{Venom} Venom requires an appropriate body part from a creature -- often, poison it naturally produces.
        This usually takes a minute of work.
        Harvesting a body part from a creature requires a Medicine or Survival check equal to 5 \add twice the poison's rank.
    \end{raggeditemize}
