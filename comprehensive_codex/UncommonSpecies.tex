\chapter{Uncommon Species}\label{Uncommon Species}

\section{Animal Hybrid}
  Animal hybrids are humanoid creatures that are a combination of humans and animals.
  The abilities of an animal hybrid depend on the type of animal it is based on.

  \parhead{Size} Medium.
  \parhead{Attributes} The attributes of an animal hybrid depend on its size.
  \parhead{Special Abilities} As the original animal.
  \parhead{Automatic Languages} Common and any one \glossterm{common language} (see \tref{Common Languages}).

  \subsection{Sample Animal Hybrids}

    \parhead{Hybrid Bee}

    \subparhead{Special Abilities}
    \parhead{Attribute} \plus1 Dexterity, \minus1 Constitution.
    \begin{raggeditemize}
      \itemhead{Low-light Vision} A hybrid bee has \trait{low-light vision}, allowing it to see clearly in \glossterm{shadowy illumination} (see \pcref{Low-light Vision}).
      \itemhead{Stinger} A hybrid bee has a stinger natural weapon (see \pcref{Natural Weapons}).
        Whenever it causes a creature to lose \glossterm{hit points} with that natural weapon, the struck creature is poisoned by giant wasp venom (see \pcref{Poison}).
        % Note that this doesn't have the stage 3 damage of giant wasp venom. That's fine because bees aren't wasps.
        Its stage 1 effect makes the target \slowed while the poison lasts.
      \itemhead{Winged Agility} A hybrid bee has wings that are not strong enough to help it fly.
        However, the wings still help it stabilize its movements.
        It gains a \plus3 bonus to the Balance skill, and it gains a \plus5 foot bonus to its maximum horizontal jump distance (see \pcref{Jumping}).
        This increases its maximum vertical jump distance normally.
    \end{raggeditemize}

    \parhead{Hybrid Shark}

    \subparhead{Special Abilities}
    \begin{raggeditemize}
      \itemhead{Bloodscent} A hybrid shark has the scent ability (see \pcref{Scent}).
        In addition, it gains a \plus10 bonus to Awareness checks to detect blood.
      \itemhead{Bite} A hybrid shark's mouth is elongated, which it can use as a bite attack (see \pcref{Natural Weapons}).
        A hybrid shark's bite deals 1d6 damage.
      \itemhead{Gills} You can breathe water as easily as a human breathes air, preventing you from drowning or suffocating underwater.
      \itemhead{Swim Speed} A hybrid shark has an average \glossterm{swim speed}.
    \end{raggeditemize}

    \parhead{Hybrid Wolf}

    \subparhead{Special Abilities}
    \begin{raggeditemize}
      \itemhead{Scent} A hybrid wolf has the scent ability (see \pcref{Scent}).
      \itemhead{Bite} A hybrid wolf's mouth is elongated, which it can use as a bite attack (see \pcref{Natural Weapons}).
        A hybrid wolf's bite deals 1d6 damage.
      \itemhead{Low-light Vision} A hybrid wolf has \trait{low-light vision}, allowing it to see clearly in \glossterm{shadowy illumination} (see \pcref{Low-light Vision}).
    \end{raggeditemize}

\section{Automaton}
  An automaton appears to be a humanoid construct, like a golem.
  Its body is made from some combination of stone, wood, and metal.
  However, its artificial body is inhabited by a true soul, making it an \trait{indwelt} (see \pcref{Indwelt}).

  \parhead{Size} Medium.
  \parhead{Attributes} \plus1 Constitution or Intelligence, \minus1 Dexterity.
  \parhead{Special Abilities}
  \begin{raggeditemize}
    \itemhead{Artificial Life} Automatons are not alive. They are invalid targets for abilities which only affect living creatures, including poisons and most healing abilities. In addition, they do not need to eat, drink, or sleep.
    \itemhead{Automaton Archetype} Automatons only gain two class archetypes instead of three.
      Instead, they treat the Automaton archetype as one of their archetypes, and they gain ranks in it just like they gain ranks in class archetypes.
    \itemhead{Manual Repair} A Craft skill relevant to the automaton's body can be used to achieve the same effects that the Medicine skill would have on a living creature.
    \itemhead{Mechanical Body} Automatons are considered both objects and creatures, and are affected by abilities which affect either.
      They are always considered to be \glossterm{attended} by themselves, so they are never affected by abilities that only affect unattended objects, even while unconscious.
    \itemhead{Mechanical Intuition} Automatons gain a \plus2 bonus to the Devices skill and one Craft skill of their choice.
  \end{raggeditemize}

  \subsection{Automaton Archetype}

    \cf{Aut}[1]{Modular Carapace} You can adjust the density and layering of your hardened exterior to augment your defenses.
      Changing your configuration in this way requires 10 minutes of work, and spare armor parts that you generally keep with you.
      You can choose to treat your carapace as being either light, medium, or heavy armor.
      To gain the full benefits of your carapace, you must have proficiency with armor of the appropriate usage class.
      The benefits from this ability are considered to come from body armor, and do not stack with actual body armor.
      Your modular carapace cannot be reinforced like ordinary body armor (see \pcref{Reinforcement}).

      You can use magic armor to build your carapace.
      If you do, the magic armor becomes embedded in your body.
      You can attune to it to benefit from its effects.
      If you use light modular carapace, this can allow you to benefit from two different magic armor effects.

      \begin{raggeditemize}
        \item Light armor: You gain a \plus3 bonus to your Armor defense and a \plus2 bonus to your \glossterm{durability}.
          You can wear body armor on top of this carapace.
          Although the benefits of that armor do not stack with the carapace, you can use the higher Armor defense value and durability bonus from either armor.
        \item Medium armor: You gain a \plus5 bonus to your Armor defense, a \plus4 bonus to your \glossterm{durability}, and a \plus1 bonus to \glossterm{vital rolls}.
          However, your Dexterity bonus to your Armor defense is halved, and you cannot wear body armor.
        \item Heavy armor: You gain a \plus6 bonus to your Armor defense, a \plus8 bonus to your \glossterm{durability}, and a \plus2 bonus to \glossterm{vital rolls}.
          However, your Dexterity bonus to your Armor defense is halved, you take a \minus10 foot penalty to your speed with all movement modes, and you cannot wear body armor.
          Unlike normal for heavy body armor, you do not need a minimum Strength to use this armor.
      \end{raggeditemize}

      If you lose your original armor parts, you can create or buy new parts that are suited to your body.
      These parts are considered a Rank 1 (40 gp) item.

    \cf{Aut}[2]{Modular Armaments} You can customize your arms.
      Changing your arm configuration in this way requires 10 minutes of work, and spare arm and weapon parts that you generally keep with you.
      It also requires at least one \glossterm{free hand}.
      You can combine any number of different customizations with this ability.
      Some customizations apply to both of your arms, but others apply to only one arm, as indicated.
      \begin{raggeditemize}
        \item Bulky: You augment both of your arms with additional strength.
          You gain a \plus1 bonus to your \glossterm{mundane power}, and to your Strength for the purpose of determining your \glossterm{weight limits}.
          However, you increase your \glossterm{encumbrance} by 2.
        \item Fortified: You add additional protective plating to both of your arms.
          You gain a \plus1 bonus to your Armor defense, but you increase your \glossterm{encumbrance} by 1.
          This does not require a \glossterm{free hand}, but it is still considered to come from a shield, and it does not stack with the benefit from using a shield.
        \item Slim: You trim away excess muscle from both of your arms to make their movements more precise.
          You gain a \plus1 bonus to your \glossterm{accuracy} with \glossterm{strikes}.
          However, you take a \minus1 penalty to your \glossterm{mundane power}.
        \item Weapon: You convert one of your arms into a manufactured weapon of your choice that you are \glossterm{proficient} with.
          It is considered either a \glossterm{natural weapon}, a manufactured weapon, or both whenever it would be beneficial for you.
          However, that arm no longer has a \glossterm{free hand}.
          You can incorporate a magic weapon into this process or find a smith to imbue your arm parts as if they were a magic weapon.
          If you do, you can attune to the magic weapon property, and it affects this weapon.
      \end{raggeditemize}

      If you lose your original arm and weapon parts, you can create or buy new parts that are suited to your body.
      These parts are considered a Rank 1 (40 gp) item.

    \magicalcf{Aut}[3]{Sharpening Slash}
    \begin{activeability}{Sharpening Slash}
        \abilityusagetime Standard action.
        \rankline
        Make a \glossterm{strike} that deals \glossterm{extra damage} equal to half your \glossterm{power}.
        Then, if you did not get a critical hit with the strike, you are \glossterm{briefly} \honed.

        \rankline
        \rank{4} The extra damage increases to 1d6 \add half your power.
        \rank{5} The extra damage increases to 1d6 \add your power.
        \rank{6} The extra damage increases to 3d6 \add your power.
        \rank{7} The strike deals double \glossterm{weapon damage}.
    \end{activeability}

    \magicalcf{Aut}[4]{Embedded Apparel} You can embed magic apparel into your body with an hour of work.
      The item becomes a part of your body, and would require another hour of work to remove.
      This allows you to use one additional apparel item from each body slot (see \pcref{Body Slots}).
      For example, you could embed one set of magic boots into your feet and then wear another pair of magic boots over them.
      You also gain an additional \glossterm{attunement point} that you can only use to attune to items embedded into your body.
      You cannot embed a \glossterm{legacy item} in this way.

    \cf{Aut}[5]{Reassembly} You can recover from vital wounds more easily by simply replacing broken parts.
      You can remove a vital wound with ten minutes of work.
      This increases your \glossterm{fatigue level} by three, and it requires replacement parts that you generally keep with you.
      The parts are considered a consumable Rank 3 (200 gp) item.

      This can even save you from death, though that is more difficult and requires more advanced parts.
      A creature can spend eight hours replacing broken parts of your corpse to \glossterm{resurrect} you (see \pcref{Resurrection}).
      % Baseline trained bonus at rank 5 is +9.
      This requires a \glossterm{difficulty value} 20 \glossterm{extended check} using a Craft skill appropriate to the composition of your body.
      The parts required to perform this feat are considered a consumable Rank 5 (5,000 gp) item.

    \cf{Aut}[6]{Artificial Mind} You become immune to \abilitytag{Compulsion} and \abilitytag{Emotion} attacks.

    \cf{Aut}[7]{Infinite Edge} You are always \honed.
    When an ability would cause you to become honed, such as your \ability{sharpening slash} ability, you become \empowered instead.

  \subsection{Base Class Effects}
    \highhpprogressiontable

    If you choose automaton as your base class, you gain the following benefits.

    \cf{Aut}{Hit Points}
      You have 8 hit points \add twice your Constitution, plus 2 hit points per level beyond 1.
      This increases as your level increases, as indicated below.
      \begin{raggeditemize}
        \itemhead{Level 7} 20 hit points \add three times your Constitution, plus 3 hit points per level beyond 7.
        \itemhead{Level 13} 40 hit points \add six times your Constitution, plus 6 hit points per level beyond 13.
        \itemhead{Level 19} 80 hit points \add twelve times your Constitution, plus 12 hit points per level beyond 19.
      \end{raggeditemize}

    \cf{Aut}{Defenses}
      You gain a \plus3 bonus to your Brawn, Fortitude, Mental, and Reflex defenses.

    \cf{Aut}{Resources}
      \begin{raggeditemize}
          \item \glossterm{Attunement points}: 3 (see \pcref{Attunement Points}).
          \item \glossterm{Fatigue tolerance}: 5 \add your Constitution (see \pcref{Fatigue}).
          \item \glossterm{Insight points}: 1 \add your Intelligence (see \pcref{Insight Points}).
          \item \glossterm{Trained skills}: 4 from among your \glossterm{class skills}, plus additional trained skills equal to your Intelligence if it is positive (see \pcref{Skills}).
      \end{raggeditemize}

    \cf{Aut}{Weapon Proficiencies}
      You are proficient with all non-exotic weapons.

    \cf{Aut}{Armor Proficiencies}
      You are proficient with light, medium, and heavy armor.

    \cf{Aut}{Starting Items and Equipment}
    You can start with the following items and equipment:
    \begin{raggeditemize}
        \item Any one of the following: buff leather, leather lamellar, or breastplate
        \item Any two of the following: club, dagger, broadsword, two handaxes, or spear
        \item A buckler or standard shield
        \item A standard adventuring kit (see \pcref{Standard Adventuring Kit}).
        \item A rank 0 wealth item (1 gp)
    \end{raggeditemize}

    \cf{Aut}{Skills}
      You have the following \glossterm{class skills}:
      \begin{raggeditemize}
        \itemhead{Strength} Climb, Jump.
        \itemhead{Dexterity} Balance.
        \itemhead{Constitution} Endurance.
        \itemhead{Intelligence} Craft (any), Deduction, Devices, Disguise, Knowledge (engineering, items).
        \itemhead{Perception} Awareness.
      \end{raggeditemize}

\section{Awakened Animal}

  Awakened animals are animals that have been granted sentience by the \spell{awaken} ritual.
  The abilities of an awakened animal depend on the type of animal it is.

  \parhead{Size} Small or Medium, as original animal.
  \parhead{Attributes} The attributes of an awakened animal depend on its size.
  \subparhead{Medium} No change.
  \subparhead{Small} \minus2 Strength, \plus1 Dexterity.
  \parhead{Special Abilities} As the original animal.
  \parhead{Automatic Languages} Common.

  \subsection{Sample Awakened Animals}

    \parhead{Cat}

    \subparhead{Size} Small. This gives a cat a 20 foot \glossterm{base speed} and a \plus4 bonus to the Stealth skill, among other effects (see \pcref{Size Categories}).
    \subparhead{Attributes} \minus2 Strength, \plus1 Dexterity
    \subparhead{Special Abilities}
    \begin{raggeditemize}
      \itemhead{Claws} A cat's paws end in claws, which it can use to attack (see \pcref{Natural Weapons}). A cat's claws have a \plus2 accuracy bonus and deal 1d4 damage.
      \itemhead{Low-light Vision} A cat has \trait{low-light vision}, allowing it to see clearly in \glossterm{shadowy illumination} (see \pcref{Low-light Vision}).
      \itemhead{Multipedal} A cat is \trait{multipedal}, which gives it a \plus10 foot bonus to its \glossterm{movement speed} and a \plus5 bonus to Balance.
      \itemhead{Scent} A cat has the scent ability (see \pcref{Scent}).
    \end{raggeditemize}

\section{Changeling}

  \parhead{Size} Medium.
  \parhead{Attributes} No change.
  \parhead{Special Abilities}
  \begin{raggeditemize}
    \itemhead{Alter Shape} A changling can change its body using the \ability{alter shape} ability.
      \begin{activeability}{Alter Shape}
        \abilityusagetime Standard action.
        \rankline
        You make a Disguise check to alter your appearance (see \pcref{Change Appearance}).
        This physically changes your body to match the results of your disguise.
        You gain a \plus4 bonus on the check, and you ignore penalties for changing your gender, species, and age.
        However, this effect is unable to alter your equipment, size category, or number of limbs.

        This ability lasts until you \glossterm{dismiss} it or until you use it again.
      \end{activeability}
    \itemhead{Skilled} A changeling gains an additional \glossterm{trained skill} (see \pcref{Skills}).
  \end{raggeditemize}
  \parhead{Bonus Languages} Any.
  \parhead{Automatic Languages} Common, any two \glossterm{common languages}.

\section{Dragon}
  Ancient dragons are magical creatures of immense power and wisdom, and are far more powerful than any ordinary character of the same level.
  However, young dragons can be played as characters, though their unique abilities do pose unique challenges.

  \parhead{Creature Type} Unlike most other playable species, dragons are magical beasts instead of humanoids.
  \parhead{Size} Small. This gives a dragon a 20 foot \glossterm{base speed} and a \plus1 bonus to their Reflex defense, among other effects (see \pcref{Size Categories}).
  \parhead{Attributes} \minus2 Strength, \plus1 Dexterity.
  \parhead{Special Abilities}
  \begin{raggeditemize}
    \itemhead{Dragon Archetype} Dragons only gain two class archetypes instead of three.
      Instead, they treat the Dragon archetype as one of their archetypes, and they gain ranks in it just like they gain ranks in class archetypes.
    \itemhead{Draconic Senses} Dragons have \trait{darkvision} with a 60 foot range, allowing them to see in complete darkness (see \pcref{Darkvision}).
      In addition, dragons gain \trait{low-light vision}, allowing them to see clearly in \glossterm{shadowy illumination} (see \pcref{Low-light Vision}).
    \itemhead{Draconic Scales} Dragons gain a \plus2 bonus to their Armor defense.
    \itemhead{Draconic Weapons} Dragons have a bite natural weapon and two claw natural weapons.
      For details, see \pcref{Natural Weapons}.
    \itemhead{Draconic Wings} Dragons have scaly wings that sprout from their backs.
      These wings grant them an average \glossterm{glide speed} (see \pcref{Aerial Movement}).
    \itemhead{Dragon Type} Each dragon has a single type from among the dragon types on \trefnp{Dragon Types}.
      They are immune to attacks with their associated ability tag.
    \itemhead{Limited Equipment} A dragon's claws are not able to effectively wield shields or manufactured weapons.
      They can wear armor, but it is treated as \glossterm{barding} instead of normal armor, reducing its effectiveness (see \pcref{Barding}).
    \itemhead{Multipedal} Dragons are \trait{multipedal}, which grants them a \plus10 foot bonus to their \glossterm{movement speed} and a \plus5 bonus to the Balance skill.
  \end{raggeditemize}
  \parhead{Automatic Languages} Common, Draconic, any one \glossterm{common language}.

  \begin{dtable}
    % Don't use lcaption because there is already a Dragon Types table with a label
    \caption[]{Dragon Types}
    \begin{dtabularx}{\columnwidth}{l >{\lcol}X >{\lcol}X}
      \tb{Dragon} & \tb{Tag} & \tb{Breath Weapon} \tableheaderrule
      Black       & \atAcid             & \areamed, 5 ft. wide line \\
      Blue        & \atElectricity      & \areamed, 5 ft. wide line \\
      Brass       & \atFire             & \areamed, 5 ft. wide line \\
      Bronze      & \atElectricity      & \areamed, 5 ft. wide line \\
      Copper      & \atAcid             & \areamed, 5 ft. wide line \\
      Gold        & \atFire             & \areasmall cone           \\
      Green       & \atAcid             & \areasmall cone           \\
      Red         & \atFire             & \areasmall cone           \\
      Silver      & \atCold             & \areasmall cone           \\
      White       & \atCold             & \areasmall cone           \\
    \end{dtabularx}
  \end{dtable}

  \subsection{Dragon Archetype}

    \cf{Dgn}[1]{Dragon Breath}
      % +1r damage to compensate for cooldown... ish. Kind of awkward scaling right now.
      \begin{activeability}{Dragon Breath}
        \abilityusagetime Standard action.
        \abilitycost You \glossterm{briefly} cannot use this ability again.
        \rankline
        This ability's tag depends on your dragon type (see Dragon Types, above).
        Make an attack vs. Reflex against everything in the area defined by your dragon type.
        \hit \damageranktwo.
        \miss Half damage.

        \rankline
        % T2 area
        \rank{2} The area increases.
        A line breath weapon becomes a \arealarge, 5 ft.\ wide line.
        A cone breath weapon becomes a \areamed cone.
        \rank{3} The damage increases to \damagerankthree.
        \rank{4} The damage increases to \damagerankfour.
        % T4 area
        \rank{5} The area increases.
        A line breath weapon becomes a \areahuge, 10 ft.\ wide line.
        A cone breath weapon becomes a \arealarge cone.
        \rank{6} The damage increases to \damageranksix.
        \rank{7} The damage increases to \damagerankseven.
      \end{activeability}

    \magicalcf{Dgn}[2]{Draconic Flight} Your wings grow larger, granting you a limited ability to fly.
      You gain a slow \glossterm{fly speed} with a maximum height of 10 feet (see \pcref{Flight}).
      As a \glossterm{free action}, you can increase your \glossterm{fatigue level} by one to ignore this height limit until the end of the round.

    \cf{Dgn}[3]{Draconic Body} You gain a \plus1 bonus to your Armor defense.

    \cf{Dgn}[4]{Draconic Bulk} Your size category increases to Medium.
      This increases your \glossterm{base speed} to 30 feet.
      You reduce your Dexterity by 1 and increase your Strength by 2.
      In addition, you gain a \plus1 bonus to your \glossterm{magical power} and \glossterm{mundane power}.

    \cf{Dgn}[5]{Draconic Body+} The Armor bonus from your \textit{draconic body} ability increases to \plus2.

    % TODO: why is this weaker than harpy rank 6?
    \magicalcf{Dgn}[6]{Draconic Flight+} The maximum height increases to 30 feet, and the speed increases to average.

    \cf{Dgn}[7]{Draconic Bulk+} Your size category increases to Large.
      This increases your \glossterm{base speed} to 40 feet.
      In addition, the attribute modifiers to Dexterity and Strength increase to \minus2 and \plus3 respectively, and the power bonus increases to \plus2.
      % Natural dragons treat their tail slam as heavy, but that may not work for a PC? They also treat their Bite as heavy, which these dragons don't do.
      You also gain a tail slam \glossterm{natural weapon}.
      It deals 1d10 damage and has the \abilitytag{Impact} weapon tag (see \pcref{Weapon Tags}).

  \subsection{Base Class Effects}
    \highhpprogressiontable

    If you choose dragon as your base class, you gain the following benefits.

    \cf{Drg}{Hit Points}
      You have 8 hit points \add twice your Constitution, plus 2 hit points per level beyond 1.
      This increases as your level increases, as indicated below.
      \begin{raggeditemize}
        \itemhead{Level 7} 20 hit points \add three times your Constitution, plus 3 hit points per level beyond 7.
        \itemhead{Level 13} 40 hit points \add six times your Constitution, plus 6 hit points per level beyond 13.
        \itemhead{Level 19} 80 hit points \add twelve times your Constitution, plus 12 hit points per level beyond 19.
      \end{raggeditemize}

    \cf{Drg}{Defenses}
      You gain a \plus3 bonus to your Brawn, Fortitude, Mental, and Reflex defenses.

    \cf{Drg}{Resources}
      \begin{raggeditemize}
          \item \glossterm{Attunement points}: 4 (see \pcref{Attunement Points}).
          \item \glossterm{Fatigue tolerance}: 3 \add your Constitution (see \pcref{Fatigue}).
          \item \glossterm{Insight points}: 1 \add your Intelligence (see \pcref{Insight Points}).
          \item \glossterm{Trained skills}: 4 from among your \glossterm{class skills}, plus additional trained skills equal to your Intelligence if it is positive (see \pcref{Skills}).
      \end{raggeditemize}

    \cf{Drg}{Weapon Proficiencies}
      You are not proficient with any manufactured weapons, even simple weapons.
      You are still proficient with your natural weapons.

    \cf{Drg}{Armor Proficiencies}
      You are proficient with light armor.
      Armor shaped appropriately for dragons can be hard to find, and may need to be crafted individually for the dragon.

    \cf{Drg}{Starting Items and Equipment}
    You can start with the following items and equipment:

    \begin{raggeditemize}
        \item Buff leather barding
        \item Two \magicitem{potions of healing}
        \item A standard adventuring kit (see \pcref{Standard Adventuring Kit}).
        \item Two rank 0 wealth items (2 gp)
    \end{raggeditemize}

    \cf{Drg}{Skills}
      You have the following \glossterm{class skills}:
      \begin{raggeditemize}
        \item \subparhead{Strength} Climb, Swim.
        \item \subparhead{Dexterity} Balance, Stealth.
        \item \subparhead{Constitution} Endurance.
        \item \subparhead{Intelligence} Craft, Deduction, Knowledge (arcana, items), Medicine.
        \item \subparhead{Perception} Awareness, Creature Handling, Social Insight, Survival.
        \item \subparhead{Other} Deception, Intimidate, Persuasion.
      \end{raggeditemize}

\section{Drow}

  Drow are an offshoot group of elves that live deep underground.
  The deep caves are a far harsher environment than the surface world.
  Resources are scarce, and dangerous monsters are far more common.
  In order to survive, drow were forced to adopt a variety of practices condemned by surface civilizations.
  The most notorious are their frequent use of poison, their refusal to take prisoners, their willingness to eat any non-drow creatures they kill, even sentient creatures.
  In addition, drow society tends to reward selfishness and ambition more explicitly than surface civilizations, and the vast majority of drow are evil.

  When drow find opportunities to reach the surface world, they seek to conquer territory for themselves, usually with great violence.
  They have always been defeated and banished back to their caves, but surface civilizations still remember the danger that drow pose.
  Even more so than tieflings or orcs, who are already viewed with suspicion, drow are anathema in almost any civilized society.
  Drow who escape the deep caves are more likely to find a peaceful existence on other planes that do not fear an underground invasion.

  \parhead{Size} Medium.
  \parhead{Attributes} \minus1 Constitution, \plus1 Dexterity
  \parhead{Special Abilities}
  \begin{raggeditemize}
    \itemhead{Darkvision} Drow have \trait{darkvision} with a 120 foot range, allowing them to see in complete darkness (see \pcref{Darkvision}).
    \magicalitemhead{Deep Darkness}
      \begin{magicalsustainability}{Deep Darkness}{\abilitytag{Sustain} (standard)}
        \abilityusagetime Standard action.
        \rankline
        You create a void of darkness in a \medarea radius \glossterm{zone} within \medrange.
        \glossterm{Bright illumination} and \glossterm{brilliant illumination} within or passing through that area is dimmed to be no brighter than \glossterm{shadowy illumination}.
        Any object or effect which blocks light also blocks this ability's effect.
      \end{magicalsustainability}
    \itemhead{Drow Prejudice} Almost all surface-dwellers have negative associations with drow.
      Drow have an Opposition relationship with most people that they meet, which influences people's behavior and makes Persuasion checks harder (see \pcref{Persuasion}).
      People in some locations, such as deep underground, do not have this attitude.
    \itemhead{Keen Senses} Drow gain a \plus2 bonus to the Awareness skill (see \pcref{Awareness}).
    \itemhead{Poison Tolerance} Drow are \trait{impervious} to poison.
    \itemhead{Sensitive Eyes} Drow take a \minus2 penalty to \glossterm{accuracy} while they are in \glossterm{bright illumination}.
      This penalty is doubled while they are in \glossterm{brilliant illumination}.
    \itemhead{Trance} Drow do not sleep, and are immune to \magical effects that would cause them to sleep.
      Instead of sleeping, drow can trance for 4 hours.
      An elf in trance may make Perception-based checks at a \minus5 penalty.
      Drow must still avoid strenuous activity for 8 hours to heal and gain other benefits of taking a \glossterm{long rest}.
  \end{raggeditemize}
  \parhead{Automatic Languages} Common, Elven, Undercommon

\section{Dryaidi}

  Dryaidi are humanoid creatures with plantlike characteristics.
  They might have leaves instead of hair, a green skin tone, or rough, barky skin.
  They are descended from dryads, and share some fey heritage and an affinity for trees.

  \parhead{Size} Medium.
  \parhead{Attributes} No change.
  \parhead{Special Abilities}
  \begin{raggeditemize}
    \itemhead{Dryad Archetype} Dryaidi only gain two class archetypes instead of three.
      Instead, they treat the Dryad archetype as one of their archetypes, and they gain ranks in it just like they gain ranks in class archetypes.
    \magicalitemhead{Enchanting Appearance} A dryaidi gains a \plus2 \glossterm{enhancement bonus} to the Creature Handling, Perform, and Persuasion skills.
    \itemhead{Fey Vulnerability} Dryaidi are \vulnerable to cold iron weapons.
    \magicalitemhead{Tree Bond} A dryaidi must be bonded with a specific tree.
      The tree must be at least a hundred years old, healthy, and intact.
      Forming a bond or severing a bond takes one week of meditation and ritual, periodically interrupted by rest.
      Forming a bond also requires asking permission from the tree through the ritual.
      Any individual tree can only be bonded to one dryad or dryaidi in this way.

      As long as the bonded tree remains healthy and intact, the dryaidi gains a \plus1 bonus to Mental defense and a \plus1 bonus to its \glossterm{fatigue tolerance}.
      If the bonded tree becomes unhealthy, is seriously damaged, or is killed, these bonuses are inverted into penalties until the dryaidi forms a bond with a new tree.
      A bonded dryaidi can passively observe the general health and status of the tree it bonded to.
    \magicalitemhead{Verdant Flourishing} Dryaidi can use the \spell{bramblepatch} and \spell{rapid growth} \glossterm{cantrips} from the \sphere{verdamancy} sphere.
      If they already have access to that sphere, they can sustain those cantrips as a \glossterm{free action} instead of as a minor action.
  \end{raggeditemize}
  \parhead{Automatic Languages} Common, Sylvan.

  \subsection{Dryad Archetype}

    \magicalcf{Dry}[1]{Tree Stride} You can walk into and through living trees.
      Moving through a tree does not impede your movement in any way, and you can end your movement inside a tree.
      When you do, you can choose to be partially melded or fully melded with the tree.
      While partially melded, the tree provides \glossterm{cover} against all attacks against you.
      While fully melded, the tree blocks \glossterm{line of sight} or \glossterm{line of effect} between you and the outside world as long as it remains intact.

      At the end of each round, if you are fully or partially melded with a tree that you are bonded with using your \textit{tree bond} ability, you regain hit points equal to half your maximum hit points.

    \magicalcf{Dry}[2]{Natural Speech} You can speak with plants and animals as if they were capable of ordinary speech.
      This ability does not make them any more friendly or cooperative than normal.
      Wary and cunning animals are likely to be terse and evasive, while stupid ones tend to make inane comments and are unlikely to say or understand anything of use.
      Plants do not have complex thought processes, but can provide information about events that have happened near them.
      In general, plants can remember events that happened within the most recent quarter of their lifespan.

    \magicalcf{Dry}[3]{Tree Stride+} You can \glossterm{teleport} between living trees instead of moving using your \glossterm{walk speed}.
      Teleporting a given distance costs movement equal to half that distance.
      If this teleportation fails for any reason, you still expend that movement.

    \magicalcf{Dry}[4]{Fey Charm}
      \begin{magicalsustainability}{Fey Charm}{\abilitytag{Emotion}, \abilitytag{Subtle}, \abilitytag{Sustain} (minor)}
        \abilityusagetime \glossterm{Minor action}.
        \rankline
        Make an attack vs. Mental against a creature within \medrange that is an animal, plant, or humanoid.
        You take a \minus10 penalty to \glossterm{accuracy} with this attack against creatures who have made an attack or been attacked since the start of the last round.
        \hit The target is \charmed by you.
        Any act by you or by creatures that appear to be your allies that threatens or harms the charmed person breaks the effect.
        Harming the target is not limited to dealing it damage, but also includes causing it significant subjective discomfort.
        An observant target may interpret overt threats to its allies as a threat to itself.

        \rankline

        \noindent The attack's \glossterm{accuracy} increases by \plus2 for each rank beyond 4.
      \end{magicalsustainability}

    \magicalcf{Dry}[5]{Tree Bond+} You can bond to a grove of trees instead of a single tree.
      The cumulative age of all trees in the grove must be at least a thousand years, and the grove must fit within a 500 foot radius.
      While bonded to a grove, the bonuses from your \textit{tree bond} and \textit{enchanting appearance} abilities double.

    \magicalcf{Dry}[6]{Tree Union} When you meld with a tree using your \textit{tree stride} ability, you can fully unite with it.
      When you do, you have \glossterm{line of sight} and \glossterm{line of effect} from all areas of the tree simultaneously, as if you were everywhere in the tree's body.
      Attacks against the tree simultaneously affect both you and the tree.
      You and the tree are both \impervious to damaging attacks, but \vulnerable to \atFire attacks and cold iron weapons.

    \magicalcf{Dry}[7]{Acorns of Life} Whenever you visit a tree you are bonded to with your \textit{tree bond} ability, you can gather acorns of life.
      You can have up to ten acorns of life at once.
      As a \glossterm{minor action}, you can throw an acorn of life onto an unoccupied \glossterm{grounded} space within \medrange of you.
      The space must be made of dirt, earth, or stone.
      When the acorn lands, a tree immediately grows in that space.
      The tree has a five foot diameter trunk and grows vertically until it reaches a hundred feet tall or until it encounters a solid obstacle preventing its growth.

  \subsection{Base Class Effects}
    \mediumhpprogressiontable

    If you choose dryad as your base class, you gain the following benefits.

    \cf{Dry}{Hit Points}
      You have 8 hit points \add twice your Constitution, plus 1 hit point per level beyond 1.
      This increases as your level increases, as indicated below.
      \begin{raggeditemize}
        \itemhead{Level 7} 16 hit points \add three times your Constitution, plus 2 hit points per level beyond 7.
        \itemhead{Level 13} 32 hit points \add five times your Constitution, plus 5 hit points per level beyond 13.
        \itemhead{Level 19} 65 hit points \add ten times your Constitution, plus 10 hit points per level beyond 19.
      \end{raggeditemize}

    \cf{Dry}{Defenses}
      You gain a \plus3 bonus to your Brawn, Fortitude, Reflex, and Mental defenses.

    \cf{Dry}{Resources}
      \begin{raggeditemize}
          \item \glossterm{Attunement points}: 4 (see \pcref{Attunement Points}).
          \item \glossterm{Fatigue tolerance}: 2 \add your Constitution (see \pcref{Fatigue}).
          \item \glossterm{Insight points}: 2 \add your Intelligence (see \pcref{Insight Points}).
          \item \glossterm{Trained skills}: 4 from among your \glossterm{class skills}, plus additional trained skills equal to your Intelligence if it is positive (see \pcref{Skills}).
      \end{raggeditemize}

    \cf{Dry}{Weapon Proficiencies}
      You are proficient with simple weapons.

    \cf{Dry}{Armor Proficiencies}
      You are proficient with light armor.

    \cf{Dry}{Starting Items and Equipment}
    You can start with the following items and equipment:

    \begin{raggeditemize}
        \item Buff leather
        \item A club or dagger
        \item A buckler
        \item A standard adventuring kit (see \pcref{Standard Adventuring Kit}).
        \item A rank 0 wealth item (1 gp)
    \end{raggeditemize}

    \cf{Dry}{Skills}
      You have the following \glossterm{class skills}:
      \begin{raggeditemize}
        \item \subparhead{Strength} Climb, Jump, Swim.
        \item \subparhead{Dexterity} Balance, Flexibility, Perform, Stealth.
        \item \subparhead{Intelligence} Craft (wood), Knowledge (arcana, nature), Medicine
        \item \subparhead{Perception} Awareness, Creature Handling, Deception, Persuasion, Social Insight, Survival.
        \item \subparhead{Other} Intimidate.
      \end{raggeditemize}

\section{Eladrin}

  \parhead{Size} Medium.
  \parhead{Attributes} \minus1 Constitution, either \plus1 Dexterity or \plus1 Willpower
  \parhead{Special Abilities}
  \begin{raggeditemize}
    \itemhead{Fae Step}
      \begin{magicalactiveability}{Fae Step}
        \abilityusagetime Standard action.
        \rankline
        You \glossterm{teleport} horizontally to a location within \shortrange.

        \rankline
        This ability improves based on your rank in your highest-rank archetype.
        \rank{3} The range increases to \medrange.
        \rank{5} The range increases to \longrange.
        \rank{7} The range increases to \distrange.
      \end{magicalactiveability}
    \itemhead{Fae Season} Eladrin respond strongly to their emotions, and change their abilities based on the season they currently represent.
      An eladrin must choose one of the following seasons when it finishes a \glossterm{short rest}.
      The chosen season lasts until it changes to a different season.
      \subcf{Spring} \plus1 bonus to Mental defense, \minus1 penalty to Fortitude defense.
      Eladrin expressing the spring season are filled with the joy of a new year.
      However, they are also visibly thinner and more frail, as if recovering from a long winter.
      \subcf{Summer} \plus1 bonus to Fortitude defense, \minus1 penalty to Reflex defense.
      Eladrin expressing the summer season are visibly hearty and a little more plump.
      However, they also move with all the alacrity of a long summer day.
      \subcf{Autumn} \plus1 bonus to all checks, \minus1 penalty to \glossterm{accuracy}.
      Eladrin expressing the autumn season embody the spirit of the harvest.
      They are filled with goodwill towards all creatures, and prefer finding peaceful solutions to problems.
      Their bodies tend to be firm and toned, reflecting the hard work required to prepare for the winter.
      \subcf{Winter} \plus1 bonus to \glossterm{vital rolls}, \minus1 penalty to Mental defense.
      Eladrin expressing the winter season are prepared for the worst.
      They tend to be dour and pessimistic, but they press on despite the certainty of doom.
    \itemhead{Low-light Vision} Eladrin have \trait{low-light vision}, allowing them to see clearly in \glossterm{shadowy illumination} (see \pcref{Low-light Vision}).
    \itemhead{Trance} Eladrin do not sleep, and are immune to \magical effects that would cause them to sleep.
      Instead of sleeping, eladrin can trance for 4 hours.
      An eladrin in trance may make Perception-based checks at a \minus5 penalty.
      Eladrin must still avoid strenuous activity for 8 hours to heal and gain other benefits of taking a \glossterm{long rest}.
  \end{raggeditemize}
  \parhead{Species Feat Options}
  \parhead{Automatic Languages} Common, Sylvan, and any one \glossterm{common language} (see \tref{Common Languages}).

\section{Harpy}
  Harpies are winged creatures with the upper body of a humanoid and the lower body of a bird.
  Most harpies are female, but male harpies do exist.

  \parhead{Creature Type} Unlike most other playable species, harpies are monstrous humanoids instead of humanoids.
  \parhead{Size} Medium.
  \parhead{Attributes} \minus1 Intelligence, \plus1 Dexterity.
  \parhead{Special Abilities}
  \begin{raggeditemize}
    \itemhead{Harpy Archetype} Harpies only gain two class archetypes instead of three.
      Instead, they treat the Harpy archetype as one of their archetypes, and they gain ranks in it just like they gain ranks in class archetypes.
    \itemhead{Limited Equipment} Harpies can wear armor, but it is treated as \glossterm{barding} instead of normal armor, reducing its effectiveness (see \pcref{Barding}).
      Harpy talons are not able to effectively wield shields or manufactured weapons.
    \itemhead{Prehensile Talons} Harpies have a talon natural weapon on each foot (see \pcref{Natural Weapons}).
      In addition, they can use their feet as \glossterm{free hands}.
      They can make short hops to use their feet to attack or manipulate objects without suffering penalties for gliding or flying.
    \itemhead{Wings} Harpies have no arms or hands.
      Instead, they have feathered wings that sprout from their shoulders.
      These wings grant them an average \glossterm{glide speed} (see \pcref{Aerial Movement}).
    \itemhead{Winged Agility} While a harpy is able to use its wings, it gains a \plus2 bonus to Armor defense, a \plus4 bonus to the Balance skill, and a \plus10 foot bonus to its maximum horizontal jump distance (see \pcref{Jumping}).
  \end{raggeditemize}
  \parhead{Automatic Languages} Common.

  \subsection{Harpy Archetype}

    \magicalcf{Hrp}[1]{Luring Song}
      \begin{magicalactiveability}{Luring Song}[\abilitytag{Auditory}, \abilitytag{Compulsion}]
        \abilityusagetime Standard action.
        \rankline
        Make an attack vs. Mental against a creature within \longrange.
        In addition, you begin a vocal performance (see \pcref{Performance Types}).
        \hit As a \glossterm{condition}, the target must move towards you as best it can during each \glossterm{movement phase}.
        In addition, it cannot move farther away from you at any time, except as necessary to get closer to you (such as to avoid an intervening obstacle).
        It can otherwise act freely, and is still able to attack you and your allies.

        The target will risk danger to reach you, such as moving towards your allies or swimming through rough water.
        However, it is not compelled to take actions that are guaranteed to damage harm it, such as jumping off of a cliff.
        If it cannot make any progress towards you, it remains in place.

        If you attack the target with any ability other than this one, or if you stop your vocal performance, this effect is automatically broken.
        When this effect ends, the target becomes immune to this effect until it finishes a \glossterm{short rest}.
        \crit The condition must be removed twice before the effect ends.

        \rankline
        The attack's \glossterm{accuracy} increases by \plus1 for each rank beyond 1.
        In addition:
        \rank{3} You can target an additional creature within range.
        \rank{5} The maximum number of targets increases to 3.
        \rank{7} The maximum number of targets increases to 5.
      \end{magicalactiveability}

    \cf{Hrp}[2]{Harpy Wings} You gain a slow \glossterm{fly speed} with a maximum height of 10 feet (see \pcref{Flight}).
      As a \glossterm{free action}, you can increase your \glossterm{fatigue level} by one to ignore this height limit until the end of the round.

    \cf{Hrp}[3]{Sharp Talons} Your talons deal 1d6 damage.

    \cf{Hrp}[3]{Winged Agility+} The Armor defense bonus increases to \plus3, and the Balance bonus increases to \plus8.

      % TODO: define correct rank
    \magicalcf{Hrp}[4]{Siren Song}
      \begin{magicalsustainability}{Siren Song}{\abilitytag{Auditory}, \abilitytag{Emotion}, \abilitytag{Sustain} (minor)}
        \abilityusagetime Standard action.
        \rankline
        Make an attack vs. Mental against all \glossterm{enemies} within a \medarea radius from you.
        In addition, you begin a vocal performance (see \pcref{Performance Types}).
        \hit The target is both \charmed by you and \stunned as long as it can still hear your vocal performance.
        It remains stunned even if it stops being charmed, such as if you or your allies attack it.
        This ability does not have the \abilitytag{Subtle} tag, so an observant target may notice that it is being influenced.
        \rankline
        The attack's \glossterm{accuracy} increases by \plus1 for each rank beyond 4.
        In addition:
        \rank{6} The area increases to a \largearea radius.
      \end{magicalsustainability}

    \cf{Hrp}[5]{Caress the Enthralled} You gain a \plus2 accuracy bonus against creatures that are affected by either your \ability{luring song} or \ability{siren song} ability.

    \cf{Hrp}[6]{Agile Flight} You reduce the penalties to your Armor and Reflex defenses from gliding or flying by 2.

    \cf{Hrp}[6]{Harpy Wings+} Your maximum height increases to 30 feet, and the speed increases to average.

    \cf{Hrp}[7]{Sharp Talons+} Your talons deal 1d8 damage and gain the \abilitytag{Keen} \glossterm{ability tag}.

    \magicalcf{Hrp}[7]{Mythic Siren} You gain a \plus5 \glossterm{accuracy} bonus with your \ability{luring song} and \ability{siren song} abilities.

  \subsection{Base Class Effects}
    \mediumhpprogressiontable

    If you choose harpy as your base class, you gain the following benefits.

    \cf{Hrp}{Hit Points}
      You have 8 hit points \add  your Constitution, plus 1 hit points per level beyond 1.
      This increases as your level increases, as indicated below.
      \begin{raggeditemize}
        \itemhead{Level 7} 18 hit points \add twice your Constitution, plus 2 hit points per level beyond 7.
        \itemhead{Level 13} 35 hit points \add five times your Constitution, plus 5 hit points per level beyond 13.
        \itemhead{Level 19} 70 hit points \add ten times your Constitution, plus 10 hit points per level beyond 19.
      \end{raggeditemize}

    \cf{Hrp}{Defenses}
      You gain a \plus3 bonus to your Brawn, Fortitude, Mental, and Reflex defenses.

    \cf{Hrp}{Resources}
      \begin{raggeditemize}
          \item \glossterm{Attunement points}: 4 (see \pcref{Attunement Points}).
          \item \glossterm{Fatigue tolerance}: 2 \add your Constitution (see \pcref{Fatigue}).
          \item \glossterm{Insight points}: 1 \add your Intelligence (see \pcref{Insight Points}).
          \item \glossterm{Trained skills}: 5 from among your \glossterm{class skills}, plus additional trained skills equal to your Intelligence if it is positive (see \pcref{Skills}).
      \end{raggeditemize}

    \cf{Hrp}{Weapon Proficiencies}
      You are proficient with simple weapons.

    \cf{Hrp}{Armor Proficiencies}
      You are proficient with light armor.

    \cf{Hrp}{Starting Items and Equipment}
    You can start with the following items and equipment:
    \begin{raggeditemize}
        \item Buff leather barding
        \item A club or dagger
        \item A buckler
        \item A standard adventuring kit (see \pcref{Standard Adventuring Kit}).
        \item A rank 0 wealth item (1 gp)
    \end{raggeditemize}

    \cf{Hrp}{Skills}
      You have the following \glossterm{class skills}:
      \begin{raggeditemize}
        \item \subparhead{Strength} Climb, Jump.
        \item \subparhead{Dexterity} Balance, Flexibility, Perform, Stealth.
        \item \subparhead{Perception} Awareness, Creature Handling, Deception, Persuasion, Survival.
        \item \subparhead{Other} Intimidate.
      \end{raggeditemize}

\section{Incarnation}

  An incarnation is a physical embodiment of an element or form of energy, like an elemental.
  Unlike elementals, incarnations are alive and have souls.
  Most incarnations are created on the plane associated with their element or energy and never leave that plane.
  However, in rare circumstances involving powerful magic, incarnations can sometimes be created in other planes.

  \parhead{Creature Type} Unlike most other playable species, incarnations are planeforged instead of humanoids.
  \parhead{Size} Medium.
  \parhead{Attributes} One attribute gains a \plus1 bonus and another takes a \minus1 penalty, depending on the chosen element or energy.
  \parhead{Special Abilities}
  \begin{raggeditemize}
    \magicalitemhead{Essence Infusion} Each incarnation chooses one of the following tags as its essence infusion: \atAcid, \atAir, \atAuditory, \atCold, \atCompulsion, \atEarth, \atEmotion, \atElectricity, \atFire, \atVisual, or \atWater.
      All of the incarnation's \glossterm{strikes} gain that tag, and it is \impervious to attacks with that tag.
    \magicalitemhead{Essence Vulnerability} Each incarnation chooses a tag to be \vulnerable to.
      It can choose any of the valid tags for an \textit{essence infusion}, including the tag it chose as its essence infusion.
    \itemhead{Glowing} If appropriate to an incarnation's essence, it may naturally shed light as a torch.
      An incarnation cannot willingly disable this light, though the light can be covered by thick clothing.
    \itemhead{Incarnation Archetype} Incarnations only gain two class archetypes instead of three.
      Instead, they treat the Incarnation archetype as one of their archetypes, and they gain ranks in it just like they gain ranks in class archetypes.
    \itemhead{Poison Resistance} An incarnation is \impervious to \atPoison attacks.
    \itemhead{Unusual Body} An incarnation's body can be unusually light or heavy compared to an ordinary creature's body.
      It can be \glossterm{lightweight} or \glossterm{heavyweight}, depending on its essence.
  \end{raggeditemize}
  \parhead{Automatic Languages} Common.

  Incarnations can come in two forms: tethered or untethered.
  Tethered incarnations are bipedal, with two legs and two arms.
  They have the same basic body shape and functionality as a human, though they may have unusual proportions.

  Untethered incarnations have no arms or legs.
  They may appear as a perfect sphere, as intricate shifting arcane glyphs, or any other appearance depending on their nature.
  The appearance of any individual untethered incarnation is consistent.
  They are not shapeshifters or masters of disguise, simply unbound by ordinary conceptions of a ``normal'' body shape.
  An untethered incarnation has the following special abilities:
  \begin{raggeditemize}
    \itemhead{Esoteric Body} The incarnation has no arms, legs, or \glossterm{free hands}.
      This means it is unable to use manufactured weapons or \glossterm{somatic components}.
      It can wear armor, but the armor is treated as \glossterm{barding} instead of normal armor, reducing its effectiveness (see \pcref{Barding}).
      It has no \glossterm{walk speed}.
      Since it has no legs, it is immune to being \prone, but it also cannot jump.
    \magicalitemhead{Flight} The incarnation's only movement mode is an average \glossterm{fly speed} with a 5 foot height limit (see \pcref{Aerial Movement}).
      Since it is native to the air, flying does not penalize its Armor or Reflex defenses.
      It is also \trait{floating}, so it does not need to fly every phase to avoid falling.
    \itemhead{Ram} The incarnation gains a ram \glossterm{natural weapon}.
      It deals 1d6 damage and has the \weapontag{Resonating} weapon tag (see \pcref{Weapon Tags}).
    \itemhead{Uniform Composition} The incarnation is immune to \glossterm{critical hits} from strikes.
  \end{raggeditemize}

  \subsection{Incarnation Archetype}

    \cf{Inc}[1]{Essence Spike}
      \begin{magicalactiveability}{Essence Spike}
        \abilityusagetime Standard action.
        \rankline
        Make an attack against something within \medrange.
        The defense against this attack depends on your \textit{essence infusion}.
        \begin{raggeditemize}
          \item Armor defense: \atWater.
          \item Brawn defense: \atAir, \atEarth.
          \item Fortitude defense: \atAcid, \atAuditory, \atCold.
          \item Reflex defense: \atElectricity, \atFire, \atVisual.
          \item Mental defense: \atCompulsion, \atEmotion.
        \end{raggeditemize}
        \hit \damageranktwo.

        \rankline
        \rank{2} The damage increases to \damagerankthree.
        \rank{3} The damage increases to \damagerankfour.
        \rank{4} The damage increases to \damagerankfive.
        \rank{5} The damage increases to \damageranksix.
        \rank{6} The damage increases to \damagerankseven.
        \rank{7} The damage increases to \damagerankeight.
      \end{magicalactiveability}

    \cf{Inc}[2]{Essence Flare}
      \begin{magicalactiveability}{Essence Flare}
        \abilityusagetime Standard action.
        \rankline
        You are \glossterm{briefly} \focused.
        At the end of the next round, if you hit with an attack that has your \textit{essence infusion} tag during that round, you repeat the full effect of this ability.
        Otherwise, you are \glossterm{briefly} \maximized.
      \end{magicalactiveability}

    \cf{Inc}[3]{Deep Tether} You gain a special ability depending on whether you are tethered or untethered.
      \begin{raggeditemize}
        \item Tethered: Choose an \atAttune spell of rank 3 or lower from any \glossterm{mystic sphere}.
          The spell must have your \textit{essence infusion} tag, and it must not be a \glossterm{deep attunement}.
          You gain the effect of that spell on you permanently.
          If the spell disables itself, you gain its benefit again after 5 minutes.
        \item Untethered: The height limit of your fly speed increases to 10 feet.
          In addition, whenever you use the \ability{sprint} ability, you can become \trait{intangible} during that phase.
          This ability has the \abilitytag{Swift} tag, so it affects attacks against you during the current phase.
      \end{raggeditemize}

    \cf{Inc}[4]{Essence Exemplar} The bonus to an attribute that you gain from being an incarnate increases to \plus2.

    \cf{Inc}[5]{Essence Infusion+} You become \glossterm{immune} instead of impervious to attacks with your \textit{essence infusion} tag.
      In addition, you gain a \plus1 accuracy bonus with all abilities which have that tag.

    \cf{Inc}[6]{Deep Tether+} You gain a special ability depending on whether you are tethered or untethered.
      \begin{raggeditemize}
        \item Tethered: You can choose up two spells with a combined rank of 6 or lower.
        % TODO: weak?
        \item Untethered: The height limit of your fly speed increases to 20 feet.
          In addition, you gain a \plus1 bonus to your \glossterm{mundane power} and \glossterm{magical power}.
      \end{raggeditemize}

    \cf{Inc}[7]{Essence Incarnate}
      \begin{magicalactiveability}{Essence Incarnate}[\abilitytag{Swift}]
        \abilityusagetime \glossterm{Minor action}, and you \glossterm{briefly} cannot use this ability again.
        \abilitycost One \glossterm{fatigue level}.
        \rankline
        You gain a benefit depending on whether you are tethered or untethered:
        \begin{raggeditemize}
          \item Tethered: You \glossterm{briefly} \primed with abilities that have your \textit{essence infusion} tag.
          \item Untethered: You \glossterm{briefly} become \trait{incorporeal}.
            If this effect ends while you are inside of a solid object, you are pushed back in the direction from which you entered that object until you emerge.
            You take 5d10 damage for every 5 feet that you are pushed in this way.
        \end{raggeditemize}

        % don't end with an itemize right before an ability closes
        {}
      \end{magicalactiveability}

  \subsection{Base Class Effects}
    \highhpprogressiontable

    If you choose incarnation as your base class, you gain the following benefits.

    \cf{Inc}{Hit Points}
      You have 8 hit points \add twice your Constitution, plus 2 hit points per level beyond 1.
      This increases as your level increases, as indicated below.
      \begin{raggeditemize}
        \itemhead{Level 7} 20 hit points \add three times your Constitution, plus 3 hit points per level beyond 7.
        \itemhead{Level 13} 40 hit points \add six times your Constitution, plus 6 hit points per level beyond 13.
        \itemhead{Level 19} 80 hit points \add twelve times your Constitution, plus 12 hit points per level beyond 19.
      \end{raggeditemize}

    \cf{Inc}{Defenses}
      You gain a \plus3 bonus to your Brawn, Fortitude, Mental, and Reflex defenses.
      In addition, you gain a \plus2 bonus to your choice of one of those four defenses.
      The defense you choose should be related to your \textit{essence infusion}.

    \cf{Inc}{Resources}
      \begin{raggeditemize}
          \item \glossterm{Attunement points}: 4 (see \pcref{Attunement Points}).
          \item \glossterm{Fatigue tolerance}: 2 \add your Constitution (see \pcref{Fatigue}).
          \item \glossterm{Insight points}: 1 \add your Intelligence (see \pcref{Insight Points}).
          \item \glossterm{Trained skills}: 3 from among your \glossterm{class skills}, plus additional trained skills equal to your Intelligence if it is positive (see \pcref{Skills}).
      \end{raggeditemize}

    \cf{Inc}{Weapon Proficiencies}
      You are proficient with simple weapons.

    \cf{Inc}{Armor Proficiencies}
      You are proficient with light armor.
      If you are untethered, you need \glossterm{barding} instead of regular armor (see \pcref{Barding}).

    \cf{Inc}{Starting Items and Equipment}
    You can start with the following items and equipment:
    \begin{raggeditemize}
        \item Buff leather (barding if appropriate)
        \item A club or dagger
        \item A buckler
        \item A standard adventuring kit (see \pcref{Standard Adventuring Kit}).
        \item A rank 0 wealth item (1 gp)
    \end{raggeditemize}

    \cf{Inc}{Skills}
      If you are tethered, you have the following \glossterm{class skills}:
      \begin{raggeditemize}
        \item \subparhead{Strength} Climb, Jump.
        \item \subparhead{Dexterity} Balance, Flexibility.
        \item \subparhead{Constitution} Endurance.
        \item \subparhead{Intelligence} Craft, Knowledge (arcana, nature, planes).
        \item \subparhead{Perception} Awareness.
        \item \subparhead{Other} Intimidate.
      \end{raggeditemize}

      If you are untethered, you have the following \glossterm{class skills}:
      \begin{raggeditemize}
        \item \subparhead{Dexterity} Flexibility.
        \item \subparhead{Constitution} Endurance.
        \item \subparhead{Intelligence} Deduction, Knowledge (arcana, nature, planes).
        \item \subparhead{Perception} Awareness, Deception, Persuasion, Social Insight.
        \item \subparhead{Other} Intimidate.
      \end{raggeditemize}

\section{Kit}

  Kit are humanoid creatures that have noticeable foxlike characteristics.
  They are descended from natural fox spirits.
  All kit have at least one tail, and some have multiple tails.
  Their tails are distinctly fluffy and fox-like, and most kit put effort into concealing their tails to avoid revealing their true nature.

  \parhead{Size} Medium.
  \parhead{Attributes} No change.
  \parhead{Special Abilities}
  \begin{raggeditemize}
    \itemhead{Foxlike Agility} Kit gain a \plus2 bonus to the Balance and Stealth skills.
    \itemhead{Illusory Guise \sparkle} As a standard action, a kit can magically disguise its physical appearance in minor ways.
      This functions like the \textit{change appearance} ability with a \plus4 bonus, except that a kit cannot change the appearance of its equipment, creature type, or number of limbs, including any tails it may have (see \pcref{Change Appearance}).
      This ability lasts until the kit \glossterm{dismisses} the ability or uses it again.
    \itemhead{Instictive Trickster} Kit gain a \plus2 bonus to the Deception and Social Insight skills.
    \itemhead{Low-light Vision} Kit have \trait{low-light vision}, allowing them to see clearly in \glossterm{shadowy illumination} (see \pcref{Low-light Vision}).
  \end{raggeditemize}
  \parhead{Automatic Languages} Common, any one \glossterm{common language}.

\section{Naiadi}
  Naiadi are humanoid creatures descended from water spirits called naiads.
  Most naiadi are unusually physically appealing, but show no other outward signs of their heritage.

  \parhead{Size} Medium.
  \parhead{Attributes} No change.
  \parhead{Special Abilities}
  \begin{raggeditemize}
    \itemhead{Aquatic Essence} Naiadi are \impervious to \atFire and \atWater attacks.
      However, they are \vulnerable to \atCold and \atElectricity attacks.
    \itemhead{Create Water} A naiadi can cast the \spell{create water} cantrip.
      When they do so, they do not require \glossterm{verbal components} or \glossterm{somatic components}, and their spellcasting rank is considered to be equal to their rank in their highest rank archetype.
    \itemhead{Enchanting Appearance} A naiadi gains a \plus2 \glossterm{enhancement bonus} to the Creature Handling, Perform, and Persuasion skills.
    \itemhead{Fey Vulnerability} Naiadi are \vulnerable to cold iron weapons.
    \itemhead{Low-light Vision} Naiadi have \trait{low-light vision}, allowing them to see clearly in \glossterm{shadowy illumination} (see \pcref{Low-light Vision}).
    \itemhead{Naiad Archetype} Naiadi may choose three class archetypes, as normal.
      However, you may choose the Naiad archetype in place of one of your class archetypes.
      If you do, you gain ranks in it just like you gain ranks in class archetypes.
    \itemhead{Water Affinity} A naiadi has an average \glossterm{swim speed}.
      In addition, they can breathe clean water like a human breathes air.
  \end{raggeditemize}
  \parhead{Automatic Languages} Common, Sylvan, any one \glossterm{common language}.

  \subsection{Naiad Archetype}

    \magicalcf{Nai}[1]{Water Bond} You can form a bond with a fresh stream, lake, or other Gargantuan or larger body of fresh water (not salt water).
      Forming or severing a bond takes one week of meditation and ritual, periodically interrupted by rest.
      Forming a bond also requires asking permission from the water, which you are able to do as part of the ritual.
      Any individual body of water can only be bonded to one naiad or naiadi in this way.

      As long as your bonded water remains clean, pure, and large enough to be a valid subject of bonding, you gain a \plus1 \glossterm{enhancement bonus} to your \glossterm{magical power} and \glossterm{mundane power}, and a \plus2 bonus to your \glossterm{vital rolls}.
      While you are within 60 feet of your bonded body of water, these bonuses double.
      If your bonded water becomes contaminated or shrinks below the minimum size, these bonuses are inverted into penalties until you sever the bond.
      You can passively observe the general health and status of water you are bonded to, including knowing when significant pollutants enter the water and when the water grows or shrinks significantly.

    \magicalcf{Nai}[2]{Fluid Freedom} While your \textit{water bond} is active, all of your \magical attacks have the \atWater tag.
    In addition, whenever you use a \magical \atWater ability, you can choose to \glossterm{exclude} your \glossterm{allies} from it.

    % Expected volume is 25 gallons per cast.
    \magicalcf{Nai}[2]{Freshwater Fountain} The volume of water you can create with the \spell{create water} cantrip is doubled.
    In addition, you do not consider casting that cantrip to be strenuous activity, so you can cast it continuously for longer than five minutes (see \pcref{Maintain Exertion}).
    This generally means that you can create a Small body of water with half a minute of work.

    \magicalcf{Nai}[3]{Bonded Boon} While your \textit{water bond} is active, you gain a benefit based on the body of water you bonded.
    If it's ambiguous how to categorize your body of water, you can choose any one applicable category when you gain this ability and when you form any future bonds.
    \begin{raggeditemize}
      \itemhead{Geyser or spring} You gain a \plus2 accuracy bonus against creatures that are at \unaware or \partiallyunaware of your attacks.
        In addition, when you use the \ability{desperate exertion} ability to affect an attack, the target is considered \partiallyunaware of that attack.
      \itemhead{Lake} You gain a \plus3 bonus to your \glossterm{durability}.
      \itemhead{River or stream} You gain a \plus2 bonus to your Reflex defense.
        In addition, when you \ability{sprint} downhill, you gain a \plus10 foot bonus to your \glossterm{movement speed}.
        This bonus is doubled as normal by the sprint ability.
      \itemhead{Underground reservoir} You gain \trait{darkvision} with a 60 foot range, allowing you to see in complete darkness (see \pcref{Darkvision}).
        In addition, you gain a \plus2 \glossterm{enhancement bonus} to the Deception and Stealth skills.
    \end{raggeditemize}

    \magicalcf{Nai}[4]{Aqueous Form} You can cast the \spell{aqueous form} spell.
      When you do, you do not require \glossterm{verbal components} or \glossterm{somatic components}, and you use your rank in this archetype as your your spellcasting rank.
      In addition, it has the \atAttune tag instead of the \atAttune (deep) tag.

    \magicalcf{Nai}[4]{Enchanting Appearance+} The bonuses from your \textit{enchanting appearance} ability are doubled.

    % Expected volume is 100 gallons per cast.
    % Medium body of water is 935 gallons.
    % Gargantuan body of water is 64k cubic feet, so 478k gallons.
    % That takes 4,780 rounds, or 478 minutes, or 8 hours.
    \magicalcf{Nai}[5]{Freshwater Fountain+} The multiplier from your \textit{freshwater fountain} ability increases to ten times the normal volume of water.
    This generally means that you can create a Medium body of water with one minute of work, or a Gargantuan body of water with 8 hours of work.
    You can bond with a body of water you create with this ability just like any other body of water.

    \magicalcf{Nai}[5]{Fluid Force+} The bonuses increase to \plus2.

    \magicalcf{Nai}[6]{Bonded Boon+} The benefit from your bonded body of water improves.
    \begin{raggeditemize}
      \itemhead{Geyser or spring} The accuracy bonus increases to \plus4.
      % Unlike barbarian, this is their only upgrade at this rank
      \itemhead{Lake} The hit point bonus increases to five times your rank in this archetype.
      \itemhead{River or stream} You gain a \plus10 foot \glossterm{enhancement bonus} to your \glossterm{movement speed}.
      \itemhead{Underground reservoir} The range of your \trait{darkvision} increases by 60 feet.
        In addition, the skill bonuses increase to \plus4.
    \end{raggeditemize}

    \magicalcf{Nai}[7]{Water Bond+} The bonuses from your \textit{water bond} ability increase to \plus4.
    In addition, your bonded body of water becomes effectively impossible to contaminate.
    The entire body of water is continuously purified, as if by the \spell{purify water} ability, with contaminants shunted to the outside.
    It can still be physically destroyed with sufficient effort.

  \subsection{Base Class Effects}
    \mediumhpprogressiontable

    If you choose naiad as your base class, you gain the following benefits.

    \cf{Nai}{Hit Points}
      You have 8 hit points \add twice your Constitution, plus 1 hit point per level beyond 1.
      This increases as your level increases, as indicated below.
      \begin{raggeditemize}
        \itemhead{Level 7} 16 hit points \add three times your Constitution, plus 2 hit points per level beyond 7.
        \itemhead{Level 13} 32 hit points \add five times your Constitution, plus 4 hit points per level beyond 13.
        \itemhead{Level 19} 65 hit points \add ten times your Constitution, plus 8 hit points per level beyond 19.
      \end{raggeditemize}

    \cf{Nai}{Defenses}
      You gain a \plus3 bonus to your Brawn, Fortitude, Mental, and Reflex defenses.

    \cf{Nai}{Resources}
      \begin{raggeditemize}
          \item \glossterm{Attunement points}: 4 (see \pcref{Attunement Points}).
          \item \glossterm{Fatigue tolerance}: 2 \add your Constitution (see \pcref{Fatigue}).
          \item \glossterm{Insight points}: 1 \add your Intelligence (see \pcref{Insight Points}).
          \item \glossterm{Trained skills}: 5 from among your \glossterm{class skills}, plus additional trained skills equal to your Intelligence if it is positive (see \pcref{Skills}).
      \end{raggeditemize}

    \cf{Nai}{Weapon Proficiencies}
      You are proficient with simple weapons.

    \cf{Nai}{Armor Proficiencies}
      You are proficient with light armor.
 
    \cf{Nai}{Starting Items and Equipment}
    You can start with the following items and equipment:
    \begin{raggeditemize}
        \item Buff leather
        \item A club or dagger
        \item A buckler
        \item A standard adventuring kit (see \pcref{Standard Adventuring Kit}).
        \item A rank 0 wealth item (1 gp)
    \end{raggeditemize}

    \cf{Nai}{Skills}
      You have the following \glossterm{class skills}:
      \begin{raggeditemize}
        \item \subparhead{Strength} Swim.
        \item \subparhead{Dexterity} Balance, Flexibility, Perform, Sleight of Hand, Stealth.
        \item \subparhead{Intelligence} Deduction, Knowledge (nature), Medicine
        \item \subparhead{Perception} Awareness, Creature Handling, Deception, Persuasion, Social Insight, Survival.
        \item \subparhead{Other} Intimidate.
      \end{raggeditemize}

\sectiongraphic*{Oozeborn}{width=\columnwidth}{optional rules/oozeborn}
  Oozeborn are ooze creatures that have gained true sentience through a strange quirk of their birth.
  They are very rare to see in civilized lands, as most oozeborn lack the opportunity to discover more than the dark caves in which they were spawned.
  Since they often grow up without mentorship from any civilized creature, oozeborn tend to have odd mannerisms and a poor ability to mask their emotions, even after spending years in civilization.
  Old oozeborn may eventually adapt to societal norms and act perfectly natural, or they may abandon civilized company entirely.

  The body of an oozeborn is amorphous, and they lack any identifiable internal organs.
  Their natural color depends on the nature of the ooze that spawned them, so green and gray are the most common colors.
  Adventuring oozeborn typically assume a bipedal shape for both practical and social convenience, but their natural shape is a loosely spherical blob.
  Unconscious oozeborn revert to their default state automatically, though some learn to maintain a semblance of cohesion while asleep.

  \parhead{Creature Type} Unlike most other playable species, oozeborn are animates instead of humanoids.
  \parhead{Size} Medium.
  \parhead{Attributes} \minus1 Intelligence, \plus1 Constitution.
  \parhead{Special Abilities}
  \begin{raggeditemize}
    \itemhead{Acidic Body} Ooozeborn are \trait{impervious} to \atAcid and \atPoison attacks. However, they are \trait{vulnerable} to \atEarth attacks.
    \itemhead{Amorphous Form} An oozeborn's natural form is a loosely spherical blob.
      They have a \minus10 foot penalty to their \glossterm{movement speed}, but they gain a \plus5 bonus to the Flexibility skill (see \pcref{Flexibility}).
      Since they have no legs, they are immune to being \prone.
      Most legless creatures cannot jump, but oozeborn can jump without penalty.
      They can also use the \ability{mold body} ability to adopt a particular shape.
      \begin{sustainability}{Mold Body}{\abilitytag{Sustain} (free)}
        \abilityusagetime Standard action.
        \rankline
        You make a Disguise check to alter your appearance (see \pcref{Change Appearance}).
        This physically changes your body to match the results of your disguise.
        You gain a \plus4 bonus on the check, and you ignore penalties for changing your gender, species, age, and number of limbs.
        However, this effect is unable to alter your equipment or size category in any way.

        You cannot create more than four limbs with this ability, and a maximum of two \glossterm{free hands}.
        If you add at least two legs, you gain a \plus10 foot bonus to your \glossterm{movement speed}.
        % TODO: awkwardly worded
        This speed bonus does not stack with the bonus for becoming \trait{multipedal}, so the only benefit you gain from creating three or more legs is a \plus5 bonus to the Balance skill.
        If you give yourself a standard humanoid shape, you can wear armor designed for humanoids without suffering the normal penalties for \glossterm{barding} (see \pcref{Barding}).

        You can sustain this ability for any length of time without mental strain, ignoring the normal 5 minute limit.
      \end{sustainability}
    \itemhead{Compressible Body} Oozeborn can compress their head and shoulders down to a minimum of a one inch radius, allowing them to squeeze through very small areas.
      Their clothing or armor is not compressed, so they may limit their ability to move through extremely narrow spaces.
    \itemhead{Darkvision} Oozeborn have \trait{darkvision} with a 60 foot range, allowing them to see in complete darkness (see \pcref{Darkvision}).
    \itemhead{Oozeborn Archetype} Oozeborn only gain two class archetypes instead of three.
      Instead, they treat the Oozeborn archetype as one of their archetypes, and they gain ranks in it just like they gain ranks in class archetypes.
  \end{raggeditemize}
  \parhead{Automatic Languages} Common.

  \subsection{Oozeborn Archetype}

    \cf{Ooz}[1]{Acidic Pseudopod} One of your arms becomes a pseudopod \glossterm{natural weapon}.
      It deals 1d10 damage and has the \atAcid and \weapontag{Long} tags (see \pcref{Weapon Tags}).
      You do not have a \glossterm{free hand} on that arm while using it as a weapon in this way.

      In addition, all of your attacks with natural weapons have the \atAcid tag.
      This does not affect damage you deal with manufactured weapons.

    \cf{Ooz}[2]{Darkborn Senses} You gain \trait{blindsense} with a 60 foot range, allowing you to sense your surroundings without light (see \pcref{Blindsense}).
      If you already have the blindsense ability, you increase its range by 60 feet.
      In addition, you gain \trait{blindsight} with a 15 foot range, allowing you to see without light (see \pcref{Blindsight}).
      If you already have the blindsight ability, you increase its range by 15 feet.

    \cf{Ooz}[2]{Ingest Object}
      \begin{activeability}{Ingest Object}[\atAcid]
        \abilityusagetime Standard action.
        \rankline
        This ability functions like the \spell{absorb object} spell, except that the maximum size of the object is equal to your size.
        Anything you absorb in this way takes a single point of \glossterm{environmental} damage during each of your actions while it remains absorbed.
        This damage is insufficient to hurt most objects made from wood, stone, or metal, but it can destroy more fragile objects like paper or complex mechanical traps.
      \end{activeability}

    \cf{Ooz}[3]{Amorphous Form+} You gain a \plus4 bonus to your defenses when determining whether a \glossterm{strike} gets a \glossterm{critical hit} against you instead of a normal hit.
      In addition, your \ability{mold body} ability loses the \abilitytag{Sustain} (free) tag.
      Instead, it lasts until you choose to \glossterm{dismiss} it.
      This allows you to maintain your shape while unconscious.

    \cf{Ooz}[3]{Compressible Body+} You reduce your penalties for \squeezing by 1.

    \cf{Ooz}[4]{Acidic Body+} You are \trait{immune} to \atAcid and \atPoison attacks.

    \cf{Ooz}[5]{Darkborn Senses+} The range of your \trait{blindsense} increases by 60 feet.
      In addition, the range of your \trait{blindsight} increases by 15 feet.

    \cf{Ooz}[5]{Ingest Object+} The maximum number of objects you can absorb with your \textit{ingest object} ability increases to 2.
      In addition, you may absorb \glossterm{allies} with that ability in addition to unattended objects.

    \cf{Ooz}[6]{Amorphous Form++} The defense bonus against critical hits from strikes increases to \plus10.

    \cf{Ooz}[6]{Compressible Body++} The reduction of squeezing penalties increases to 2.
      This means you take no penalties for squeezing unless you use the \ability{tight squeeze} ability (see \pcref{Flexibility}).

    \cf{Ooz}[7]{Third Arm} When you use your \ability{mold body} ability, you can create three arms instead of two.
      You can use all three hands as free hands.
      For example, this can allow you to use a \weapontag{Heavy} weapon and a shield simultaneously.

      In addition, your arms become stronger and more agile.
      You can use any of your arms as a pseudopod natural weapon, and your pseudopods gain the \weapontag{Light} weapon tag (see \pcref{Weapon Tags}).

  \subsection{Base Class Effects}
    \veryhighhpprogressiontable

    If you choose oozeborn as your base class, you gain the following benefits.

    \cf{Ooz}{Hit Points}
      You have 10 hit points \add twice your Constitution, plus 2 hit points per level beyond 1.
      This increases as your level increases, as indicated below.
      \begin{raggeditemize}
        \itemhead{Level 7} 24 hit points \add four times your Constitution, plus 4 hit points per level beyond 7.
        \itemhead{Level 13} 50 hit points \add eight times your Constitution, plus 8 hit points per level beyond 13.
        \itemhead{Level 19} 100 hit points \add fifteen times your Constitution, plus 15 hit points per level beyond 19.
      \end{raggeditemize}

    \cf{Ooz}{Defenses}
      You gain a \plus3 bonus to your Brawn, Mental, and Reflex defenses.
      In addition, you gain a \plus5 bonus to your Fortitude defense and a \plus1 bonus to your \glossterm{vital rolls}.

    \cf{Ooz}{Resources}
      \begin{raggeditemize}
          \item \glossterm{Attunement points}: 3 (see \pcref{Attunement Points}).
          \item \glossterm{Fatigue tolerance}: 3 \add your Constitution (see \pcref{Fatigue}).
          \item \glossterm{Insight points}: 1 \add your Intelligence (see \pcref{Insight Points}).
          \item \glossterm{Trained skills}: 3 from among your \glossterm{class skills}, plus additional trained skills equal to your Intelligence if it is positive (see \pcref{Skills}).
      \end{raggeditemize}

    \cf{Ooz}{Weapon Proficiencies}
      You are proficient with simple weapons.

    \cf{Ooz}{Armor Proficiencies}
      You are proficient with light armor.
      Depending on whether you are sustaining your \ability{mold body} and the form you choose, you may need \glossterm{barding} instead of regular armor (see \pcref{Barding}).

    \cf{Ooz}{Starting Items and Equipment}
    You can start with the following items and equipment:
    \begin{raggeditemize}
        \item Buff leather (barding if appropriate)
        \item A club or dagger
        \item A buckler
        \item A standard adventuring kit (see \pcref{Standard Adventuring Kit}).
        \item A rank 0 wealth item (1 gp)
    \end{raggeditemize}

    \cf{Ooz}{Skills}
      You have the following \glossterm{class skills}:
      \begin{raggeditemize}
        \item \subparhead{Strength} Climb, Swim.
        \item \subparhead{Dexterity} Balance, Flexibility, Sleight of Hand, Stealth.
        \item \subparhead{Constitution} Endurance.
        \item \subparhead{Intelligence} Craft, Disguise, Knowledge (dungeoneering).
        \item \subparhead{Perception} Awareness, Survival.
        \item \subparhead{Other} Intimidate.
      \end{raggeditemize}

\section{Sapling}

  Saplings are young treants that have left their forest home in search of adventure.
  They tend to be slow to think and act, but resilient once they have made up their mind.

  \parhead{Creature Type} Unlike most other playable species, saplings are considered animates instead of humanoids.
  \parhead{Size} Medium.
  \parhead{Attributes} \plus1 Constitution, \plus1 Willpower, \minus1 Dexterity, \minus1 Intelligence
  \parhead{Special Abilities}
  \begin{raggeditemize}
    \itemhead{Barkskin} A sapling gains a \plus2 bonus to Armor defense.
    \itemhead{Ingrain} A sapling can bury its roots into the ground.
      \begin{activeability}{Ingrain}
        \abilityusagetime \glossterm{Minor action}.
        \rankline
        This ability can only be used if the sapling is \glossterm{grounded}.
        The sapling becomes \braced, and its \glossterm{movement speed} becomes 5 feet, regardless of any modifiers that normally apply.
        It cannot voluntarily stop being \glossterm{grounded} while this ability lasts.

        If the sapling finishes a \glossterm{long rest} with this ability active for the duration of the rest, it acquires nutrients sufficient to replace a day's worth of food and water.
        This ability lasts until the sapling ends it as a standard action, or until it stops being \glossterm{grounded}.
      \end{activeability}
    \itemhead{Limited Equipment} Saplings can wear armor, but it is treated as \glossterm{barding} instead of normal armor, reducing its effectiveness (see \pcref{Barding}).
    \itemhead{Made of Wood} Saplings are \vulnerable to \atFire attacks. In addition, they are both creatures and plants.
    \itemhead{Treant Archetype} Saplings only gain two class archetypes instead of three.
      Instead, they treat the Treant archetype as one of their archetypes, and they gain ranks in it just like they gain ranks in class archetypes.
    \itemhead{Tree Appearance} When a sapling stays perfectly still, observers must make a DV 15 Awareness check to recognize that it is not an ordinary tree.
      Careful observers may still notice that the ordinary tree has appeared where no tree used to be, so they may be suspicious of a sapling even if they do not pass this check.
    \itemhead{Unhurried and Unfaltering} Saplings have a \minus10 penalty to their \glossterm{movement speed}.
      However, a sapling's movement speed cannot be more than 10 feet slower than its \glossterm{base speed}, even while \slowed or under similar effects, except with the \ability{ingrain} ability.
      In addition, saplings are unaffected by \glossterm{difficult terrain} from inanimate natural sources, such as \glossterm{heavy undergrowth}.
  \end{raggeditemize}
  \parhead{Automatic Languages} Common, Sylvan.

  \subsection{Treant Archetype}
    % Nourishing Ingrain:
    % * R1, 2 HP - 25% of fighter base HP (8)
    % * R2, 4 HP - 28.5% of fighter base HP (14)
    % * R3, 6 HP - 30% of fighter base HP (20)
    % * R4, 8 HP - 27.5% (29)
    % * R5, 15 HP - 37.5% (40)
    % * R6, 18 HP - 31% (58)
    % * R7, 21 HP - 26% (80)

    % This intentionally can bring you above half max
    \cf{Tre}[1]{Nourishing Ingrain} At the end of each round while you are \ability{ingrained}, you regain hit points equal to your rank in this archetype, and you may choose to remove a \glossterm{condition}.
      If you do, you increase your \glossterm{fatigue level} by one.

    \cf{Tre}[2]{Sturdy as the Mighty Oak} You gain a \plus3 bonus to your \glossterm{durability}.

    % TODO: proper EA / damage calc
    \magicalcf{Tre}[3]{Animate Plants}
      \begin{magicalactiveability}{Animate Plants}[\abilitytag{Manifestation}]
        \abilityusagetime Standard action.
        \rankline
        Make an attack vs. Reflex against one Large or smaller \glossterm{grounded} creature within \medrange.
        You gain a \plus2 accuracy bonus if the target is in \glossterm{undergrowth}.

        \hit The target is \glossterm{briefly} \slowed.
        In addition, it is attacked by plants as a \glossterm{condition}.
        It takes 1d8 damage immediately, and during each of your subsequent actions while this condition lasts.

        This condition can be removed if the target makes a \glossterm{difficulty value} 10 Strength check as a \glossterm{move action} to break the plants.
        If the target makes this check as a standard action, it gains a \plus5 bonus.
        In addition, this condition is removed if the target takes damage from a \atFire ability.
        \crit The condition must be removed an additional time before the effect ends.
        \rankline
        For each rank beyond 3, the attack's \glossterm{accuracy} increases by \plus2 and the \glossterm{difficulty value} to break the plants increases by 2.
        In addition, the damage increases at each rank as described below.
        \rank{4} 1d10 damage.
        \rank{5} 2d6 damage.
        \rank{6} 2d8 damage.
        \rank{7} 2d10 damage.
      \end{magicalactiveability}

    \cf{Tre}[4]{Tall as the Noble Pine} Your size category increases to Large.
      Unlike normal for increasing your size, this does not increase your \glossterm{base speed}.
      You also gain a \plus1 bonus to your Strength, and a \minus1 penalty to your Dexterity.

    \cf{Tre}[5]{Nourishing Ingrain+} The healing from your \textit{nourishing ingrain} ability increases to three times your rank in this archetype.
      In addition, removing a condition with that ability no longer increases your fatigue level.

    \cf{Tre}[6]{Sturdy as the Mighty Oak+} The durability bonus increases to \plus5.

    \cf{Tre}[7]{Tall as the Noble Pine+} Your size category increases to Huge.
      This increases your \glossterm{base speed} to 40 feet.
      Your normal movement speed is still only 30 feet due to the penalty from \textit{unhurried and unfaltering}.
      The modifiers to Strength and Dexterity increase to \plus2 and \minus2, respectively.

  \subsection{Base Class Effects}
    \veryhighhpprogressiontable

    If you choose treant as your base class, you gain the following benefits.

    \cf{Tre}{Hit Points}
      You have 10 hit points \add twice your Constitution, plus 2 hit points per level beyond 1.
      This increases as your level increases, as indicated below.
      \begin{raggeditemize}
        \itemhead{Level 7} 24 hit points \add four times your Constitution, plus 4 hit points per level beyond 7.
        \itemhead{Level 13} 50 hit points \add eight times your Constitution, plus 8 hit points per level beyond 13.
        \itemhead{Level 19} 100 hit points \add fifteen times your Constitution, plus 15 hit points per level beyond 19.
      \end{raggeditemize}

    \cf{Tre}{Defenses}
      You gain a \plus3 bonus to your Brawn and Reflex defenses.
      In addition, you gain a \plus5 bonus to your Fortitude defense and a \plus4 bonus to your Mental defense.

    \cf{Tre}{Resources}
      \begin{raggeditemize}
          \item \glossterm{Attunement points}: 3 (see \pcref{Attunement Points}).
          \item \glossterm{Fatigue tolerance}: 3 \add your Constitution (see \pcref{Fatigue}).
          \item \glossterm{Insight points}: 1 \add your Intelligence (see \pcref{Insight Points}).
          \item \glossterm{Trained skills}: 3 from among your \glossterm{class skills}, plus additional trained skills equal to your Intelligence if it is positive (see \pcref{Skills}).
      \end{raggeditemize}

    \cf{Tre}{Weapon Proficiencies}
      You are proficient with simple weapons and club-like weapons.

    \cf{Tre}{Armor Proficiencies}
      You are proficient with light, medium, and heavy armor.

    \cf{Tre}{Starting Items and Equipment}
    You can start with the following items and equipment:
    \begin{raggeditemize}
        \item Any one of the following: buff leather, leather lamellar, or breastplate
        \item Any two of the following: club, dagger, greatmace, morning star, or sap
        \item A buckler or standard shield
        \item A standard adventuring kit (see \pcref{Standard Adventuring Kit}).
        \item A rank 0 wealth item (1 gp)
    \end{raggeditemize}

    \cf{Tre}{Skills}
      You have the following \glossterm{class skills}:
      \begin{raggeditemize}
        \item \subparhead{Dexterity} Swim.
        \item \subparhead{Dexterity} Balance.
        \item \subparhead{Constitution} Endurance.
        \item \subparhead{Intelligence} Craft (wood), Knowledge (nature).
        \item \subparhead{Perception} Awareness, Creature Handling, Survival.
        \item \subparhead{Other} Intimidate.
      \end{raggeditemize}

\sectiongraphic*{Tiefling}{width=\columnwidth}{optional rules/tiefling}

  Tieflings are humanoid creatures descended from fiends.
  They inherit a tendency towards evil from their ancestors, and are therefore viewed with great suspicion by most civilized societies.
  Good-aligned tieflings exist, but they may have difficulty using their natural talents for subterfuge and deceit for noble ends, and they often struggle with hidden vices.

  \parhead{Size} Medium.
  \parhead{Attributes} No change.
  \parhead{Special Abilities}
  \begin{raggeditemize}
    \itemhead{Darkvision} Tieflings have \trait{darkvision} with a 60 foot range, allowing them to see in complete darkness (see \pcref{Darkvision}).
    \itemhead{Demonic Prejudice} Most people have negative associations with tieflings thanks to the malign influence that demons have on the world.
      Tieflings have an Opposition relationship with most people that they meet, which influences people's behavior and makes Persuasion checks harder (see \pcref{Persuasion}).
      People in some locations, such as the Abyss, do not have this attitude.
    \itemhead{Hellfire Tolerance} Tieflings are \trait{impervious} to \atFire attacks.
    \itemhead{Infernal Presence} Tieflings gain a \plus2 bonus to the Deception and Intimidate skills.
    \itemhead{Tiefling Archetype} You may choose three class archetypes, as normal.
      However, you may choose the Tiefling archetype in place of one of your class archetypes.
      If you do, you gain ranks in it just like you gain ranks in class archetypes.
      You cannot choose tiefling as your base class.
  \end{raggeditemize}
  \parhead{Automatic Languages} Abyssal, Common, any one \glossterm{common language}.

  \subsection{Tiefling Archetype}
    % A normal short range damage spell would deal drX+1.
    % Drop to drX for the teleport, and drX-1 for the AOE.
    \magicalcf{Tif}[1]{Abyssal Hop}
      \begin{magicalactiveability}{Abyssal Hop}[\atFire]
        \abilityusagetime Standard action.
        \rankline
        You teleport horizontally into an unoccupied location within \shortrange on a stable surface that can support your weight.
        If the destination is invalid, this spell fails with no effect.
        In addition, make an attack vs. Reflex against each \glossterm{enemy} adjacent to your location after you arrive.
        \hit \damagerankzero.
        \miss Half damage.

        \rankline
        % Stay at drX-1, approximately
        \rank{2} The base damage increases to 1d6.
        \rank{3} The damage bonus from your power increases to be equal to your power.
        \rank{4} The base damage increases to 1d8.
        \rank{5} The base damage increases to 2d6.
        \rank{6} The damage bonus increases to 1d6 per 2 power.
        \rank{7} The damage bonus increases to 1d8 per 2 power.
      \end{magicalactiveability}

    \magicalcf{Tif}[2]{Infernal Ancestry} You deepen your connection to a particular aspect of your demonic ancestry.
      Choose one of the following infernal ancestries: hellfire conduit, tempting allure, or unholy might.
      You gain a benefit based on your chosen ancestry.
      \begin{raggeditemize}
        \item Hellfire Conduit: You gain a \plus1 bonus to your \glossterm{magical power}.
          In addition, when you use your \ability{abyssal hop} ability, it deals damage to each enemy in a \smallarea radius from your arrival location.
        \item Tempting Allure: You gain a \plus2 bonus to the Deception, Disguise, and Persuasion skills.
        \item Unholy Might: You gain two claw natural weapons and one bite natural weapon (see \pcref{Natural Weapons}).
          In addition, you gain a \plus1 bonus to your \glossterm{mundane power}.
      \end{raggeditemize}

    \magicalcf{Tif}[3]{Abysswalker} Once per phase, you can teleport horizontally instead of moving using your \glossterm{walk speed}.
    Teleporting a given distance costs movement equal to twice that distance.
    If this teleportation fails for any reason, you still expend that movement.

    \magicalcf{Tif}[4]{Infernal Ancestry+} The benefits of your \textit{infernal ancestry} ability improve.
      \begin{raggeditemize}
        % Normally, double defense is +1dr and burn is +1dr. This is more damage than that, but it's also a class feature, so it's probably fine?
        \item Hellfire Conduit: If your \ability{abyssal hop} ability also hits a target's Fortitude defense, you can repeat the attack against that during your next action.
        \item Tempting Allure: You can charm creatures.
          \begin{magicalactiveability}{Charming Temptation}[\abilitytag{Emotion}, \abilitytag{Subtle}]
            \abilityusagetime Standard action.
            \rankline
            Make an attack vs. Mental against a humanoid creature within \medrange.
            You take a \minus10 penalty to \glossterm{accuracy} with this attack against creatures who have made an attack or been attacked since the start of the last round.
            \hit The target is \charmed by you until it takes a \glossterm{long rest}.
            Any act by you or by creatures that appear to be your allies that threatens or harms the charmed person breaks the effect.
            Harming the target is not limited to dealing it damage, but also includes causing it significant subjective discomfort.
            An observant target may interpret overt threats to its allies as a threat to itself.

            \rankline
            The attack's \glossterm{accuracy} increases by \plus2 for each rank beyond 4.
          \end{magicalactiveability}
        \item Unholy Might: When you use your \ability{abyssal hop} ability as a standard action, you can also make a \glossterm{mundane} \glossterm{strike} at your destination.
      \end{raggeditemize}

    \cf{Tif}[5]{Infernal Resilience} You gain a \plus1 bonus to your Fortitude, Reflex, and Mental defenses.

    \magicalcf{Tif}[6]{Abysswalker+} Teleporting a given distance only costs movement equal to that distance.
    In addition, you can teleport in this way any number of times per phase.

    \magicalcf{Tif}[7]{Infernal Ancestry++} The benefits of your \textit{infernal ancestry} ability reach their peak.
      \begin{raggeditemize}
        \item Hellfire Conduit: The magical power bonus increases to \plus2.
          In addition, your \ability{abyssal hop} ability also affects \glossterm{enemies} in a \smallarea radius from your starting point.
          Creatures in both areas are still only affected by the ability once.
        \item Tempting Allure: The skill bonuses from your \textit{infernal ancestry} ability increase to \plus4.
        \item Unholy Might: The mundane power bonus increases to \plus2.
          In addition, the damage dealt by the strike after you use \ability{abyssal hop} is doubled.
      \end{raggeditemize}

      % \subsection{Base Class Effects}
      %     If you choose tiefling as your base class, you gain the following benefits.

      %     \cf{Tif}{Defenses}
      %     You gain the following bonuses to your \glossterm{defenses}: \plus3 Fortitude, \plus5 Reflex, \plus7 Mental.

      %     \cf{Tif}{Resources} You have the following \glossterm{resources}:
      %     \begin{raggeditemize}
      %         \item Four \glossterm{attunement points}, which are required to use some items and abilities (see \pcref{Attunement Points}).
      %         \item A \plus2 bonus to your \glossterm{fatigue tolerance}, which makes it easier for you to use powerful abilities that fatigue you (see \pcref{Fatigue}).
      %         \item Two \glossterm{insight points}, which you can spend to gain additional abilities or proficiencies (see \pcref{Insight Points}).
      %         \item Five \glossterm{trained skills}, which you can spend to learn skills (see \pcref{Trained Skills}).
      %     \end{raggeditemize}

      %     \cf{Tif}{Weapon Proficiencies} 
      %     You are proficient with simple weapons.

      %     \cf{Tif}{Armor Proficiencies} 
      %     You are proficient with light and medium armor.

      %     \cf{Tif}{Skills}
      %     You have the following \glossterm{class skills}:
      %     \begin{raggeditemize}
      %         \item \subparhead{Strength} Climb.
      %         \item \subparhead{Dexterity} Balance, Sleight of Hand, Stealth.
      %         \item \subparhead{Intelligence} Craft, Disguise, Knowledge (arcana, planes)
      %         \item \subparhead{Perception} Awareness, Social Insight.
      %         \item \subparhead{Other} Deception, Intimidate, Perform, Persuasion.
      %     \end{raggeditemize}

\section{Troll}
  Trolls are large, ugly giants with tusks.
  They are famous for their supernatural regeneration abilities.
  A troll can survive even dismemberment, regrowing severed limbs over time.
  Most trolls have green skin, though other earthy skin tones are also possible.

  Trolls naturally grows lichen or other fungus on their bodies in addition to hair.
  The patterns and type of fungus depends on the troll.
  Some trolls prefer to style or entirely remove their surface fungus, just like humans may style or remove their hair.

  The body of a troll is suffused with fungus that is unique to troll biology.
  This fungus grants trolls their incredible healing abilities.
  However, it can be destroyed by fire and acid once exposed, leading to the troll's true death.
  Trolls can also be killed by pulverizing the body to an extreme extent, leaving nothing sufficiently intact to regenerate from.

  \parhead{Creature Type} Unlike most other playable species, trolls are monstrous humanoids instead of humanoids.
  \parhead{Size} Medium.
  \parhead{Attributes} \plus1 Strength, \plus1 Constitution, \minus1 Intelligence, \minus1 Perception.
  \parhead{Special Abilities}
  \begin{raggeditemize}
    \itemhead{Bite} Trolls have a bite natural weapon (see \tref{Natural Weapons}).
    \itemhead{Fungal Resilience} Trolls are \impervious to \atPoison attacks.
    \itemhead{Fungal Vulnerabilities} Trolls are \vulnerable to \atAcid and \atFire attacks.
    \itemhead{Hard to Kill} Trolls cannot be killed unless their body is almost entirely destroyed.
      If they would die from a vital wound effect of \minus5 or higher, they instead simply stay unconscious until the vital wound is healed.
      While they are unconscious in this way, if they take any damage from a \atAcid or \atFire ability, they immediately die.
    \itemhead{Subspecies} Every troll has a particular subspecies with specific effects, listed below.
      \subcf{Cave} Cave trolls are accustomed to life in deep caves.
      They have \trait{darkvision} with a 60 foot range, allowing them to see in complete darkness (see \pcref{Darkvision}).
      In addition, they gain a \plus2 bonus to the Stealth skill.
      \subcf{Forest} Forest trolls are smaller and crafter than other trolls.
      They gain a \plus1 bonus to their Intelligence, but take a \minus1 penalty to their Strength.
      In addition, they gain a \plus2 bonus to any one Craft skill.
      \subcf{Mountain} Mountain trolls are larger and stronger than even other trolls.
      They gain a \plus1 bonus to their Strength for the purpose of determining their \glossterm{weight limits} (see \pcref{Weight Limits}).
      \subcf{Scrag} Scrag trolls prefer to live in water, though they can breathe air and move on land.
      They gain a \plus3 bonus to the Endurance and Swim skills, and they can hold their breath for ten times the normal limit (see \pcref{Common Endurance Tasks}).
    \itemhead{Troll Archetype} Trolls only gain two class archetypes instead of three.
      Instead, they treat the Troll archetype as one of their archetypes, and they gain ranks in it just like they gain ranks in class archetypes.
  \end{raggeditemize}
  \parhead{Automatic Languages} Common, Giant.

  \subsection{Troll Archetype}

    \cf{Trl}[1]{Regeneration} At the end of each round, if you did not take damage from a \atAcid or \atFire ability that round, you regain hit points equal to your rank in this archetype.

    \cf{Trl}[2]{Tough Hide} You gain a \plus2 bonus to your \glossterm{durability}.
      In addition, you gain a \plus1 bonus to your \glossterm{vital rolls} (see \pcref{Vital Wounds}).

    \cf{Trl}[3]{Subspecies Specialization} You gain a bonus based on your troll subspecies.
      \begin{raggeditemize}
        \item Cave: The range of your darkvision increases by 60 feet.
          In addition, the Stealth bonus increases to \plus4.
        \item Forest: You gain an additional \glossterm{insight point}.
        \item Mountain: You gain a \plus1 bonus to your \glossterm{mundane power}.
        \item Scrag: You gain a \glossterm{swim speed} 10 feet slower than your \glossterm{base speed}.
      \end{raggeditemize}

    \cf{Trl}[3]{Tusks} Your bite natural weapon deals 1d10 damage instead of the normal 1d8.

    \cf{Trl}[4]{Regeneration+} The recovery increases to three times your rank in this archetype.
      In addition, you also automatically remove one \glossterm{vital wound}.
      You can choose to suppress this healing, and it does not function if you took damage from a \atAcid or \atFire ability that round.
      While you are unconscious, this automatically removes your most severe vital wound.
      Whenever you remove a vital wound in this way, you increase your \glossterm{fatigue level} by three.

    \cf{Trl}[5]{Hulking Size} Your size category increases to Large.
    This increases your \glossterm{base speed} to 40 feet, among other effects (see \pcref{Size and Weight}).

    \cf{Trl}[6]{Subspecies Specialization+} Your bonus based on your troll subspecies improves.
      \begin{raggeditemize}
        \item Cave: You gain a \plus1 bonus to your Dexterity.
        \item Forest: You gain a \plus1 bonus to your Intelligence.
        \item Mountain: You gain a \plus1 bonus to your Strength.
        \item Scrag: You gain a \plus1 bonus to your Constitution.
      \end{raggeditemize}

    \cf{Trl}[6]{Tusks+} Your bite natural weapon deals 2d6 damage instead of the normal 1d8.

    \cf{Trl}[7]{Regeneration++} The recovery increases to six times your rank in this archetype.
      In addition, removing a vital wound with this ability only increases your fatigue level by two.

  \subsection{Base Class Effects}
    \veryhighhpprogressiontable

    If you choose troll as your \glossterm{base class}, you gain the following benefits.

    \cf{Trl}{Hit Points}
      You have 10 hit points \add twice your Constitution, plus 2 hit points per level beyond 1.
      This increases as your level increases, as indicated below.
      \begin{raggeditemize}
        \itemhead{Level 7} 24 hit points \add four times your Constitution, plus 4 hit points per level beyond 7.
        \itemhead{Level 13} 50 hit points \add eight times your Constitution, plus 8 hit points per level beyond 13.
        \itemhead{Level 19} 100 hit points \add fifteen times your Constitution, plus 15 hit points per level beyond 19.
      \end{raggeditemize}

    \cf{Trl}{Defenses}
      You gain a \plus3 bonus to your Brawn, Fortitude, Mental, and Reflex defenses.
      In addition, you gain a \plus1 bonus to your \glossterm{vital rolls}.

    \cf{Trl}{Resources}
      \begin{raggeditemize}
          \item \glossterm{Attunement points}: 3 (see \pcref{Attunement Points}).
          \item \glossterm{Fatigue tolerance}: 4 \add your Constitution (see \pcref{Fatigue}).
          \item \glossterm{Insight points}: 1 \add your Intelligence (see \pcref{Insight Points}).
          \item \glossterm{Trained skills}: 4 from among your \glossterm{class skills}, plus additional trained skills equal to your Intelligence if it is positive (see \pcref{Skills}).
      \end{raggeditemize}

    \cf{Trl}{Weapon Proficiencies}
      You are proficient with simple weapons and club-like weapons.

    \cf{Trl}{Armor Proficiencies}
      You are proficient with light and medium armor.

    \cf{Trl}{Starting Items and Equipment}
    You can start with the following items and equipment:
    \begin{raggeditemize}
        \item Any one of the following: buff leather or leather lamellar
        \item Any two of the following: club, dagger, broadsword, two handaxes, or spear
        \item A buckler or standard shield
        \item A standard adventuring kit (see \pcref{Standard Adventuring Kit}).
        \item A rank 0 wealth item (1 gp)
    \end{raggeditemize}

    \cf{Trl}{Class Skills}
      You have the following \glossterm{class skills}:

      \begin{raggeditemize}
        \item \subparhead{Strength} Climb, Jump, Swim.
        \item \subparhead{Dexterity} Stealth.
        \item \subparhead{Constitution} Endurance.
        \item \subparhead{Perception} Awareness, Survival.
        \item \subparhead{Other} Intimidate.
      \end{raggeditemize}

\section{Vampire}
  A vampire is an undead creature that must drink the blood of living creatures to survive.
  Unlike most undead creatures, vampires appear to be alive and human, allowing them to act normally in society.
  Vampires have great power, but also many dangerous weaknesses.

  The unusual blend of life and death in a vampire comes from the strange disease of vampirism.
  This blood-transmitted disease attacks the brain of an infected creature.
  Such destruction is lethal, and it does cause the death of the original creature's mind.
  However, the disease also takes over and replaces most of the brain's autonomous functions.
  The creature's heart keeps pumping, it continues to breathe, and so on, giving the dead creature a convincing imitation of life.
  Vampires are generally pale thanks to their poor circulation, but not impossibly so.

  The half-death inflicted by vampirism can be confusing for a creature's soul.
  Although most of the soul passes on to its normal afterlife, fragments remain behind and cling to their original body.
  These give a newly born vampire a hint of its original personality, but only a hint.

  A newly born vampire, called a vampire spawn, does not have a full, independent soul.
  Its will is completely bound to the vampire that created it, called its sire.
  The sire's soul invades and replaces the vacuum left behind when the vampire's original soul was fractured by death and fled to the afterlife.

  Over time, by feeding on blood to strengthen its body, a vampire spawn can connect more deeply to its original soul.
  If allowed and guided by its sire, it can become strong enough to wrest the rest of its soul back from its afterlife.
  This unifies the creature's soul and makes it a new entity fully independent from its sire.
  A fully ensouled vampire is called a true vampire.

  Only true vampires can create new vampire spawn.
  Since each spawn claims a piece of its sire's soul to sustain its capacity for thought and action, most vampires can create few spawn at once, or else they risk losing their will.
  Despite this risk, many true vampires do not permit their spawn to become ensouled.
  Instead, they may prefer to maintain a small handful of spawn to serve them with no possibility of betrayal.
  A vampire with a soul powerful enough to command many dependent spawn is called a vampire lord.

  \parhead{Creature Type} Unlike most other playable species, vampires are undead instead of humanoids.
  \parhead{Size} Medium.
  \parhead{Attributes} \plus1 Strength and Dexterity, \minus1 Constitution.
  \parhead{Special Abilities}
  \begin{raggeditemize}
    \itemhead{Climb Speed} Vampires have a \glossterm{climb speed} 10 feet slower than their \glossterm{base speed}.
    \itemhead{Darkvision} Vampires have \trait{darkvision} with a 90 foot range, allowing them to see in complete darkness (see \pcref{Darkvision}).
    \itemhead{Fangs} Vampires have a bite natural weapon (see \tref{Natural Weapons}).
      These fangs retract when not in use, so vampires cannot be identified as non-human by their fangs unless they choose to expose them.
    \itemhead{Undead} Vampires are \trait{undead} instead of \glossterm{living}, and they take damage from most healing effects (see \pcref{Undead})).
      Since they are not living, they are immune to \atPoison effects.
    \itemhead{Unnatural Life} Unlike most undead creatures, vampires share some aspects of living creatures.
      They must breathe air, and they must sleep as much as humans do.
    \itemhead{Unnatural Charm} Vampires gain a \plus2 bonus to the Persuasion skill.
    \itemhead{Vampire Archetype} Vampires only gain two class archetypes instead of three.
      Instead, they treat the Vampire archetype as one of their archetypes, and they gain ranks in it just like they gain ranks in class archetypes.
  \end{raggeditemize}

  \parhead{Special Weaknesses}
  Vampires have a number of specific weaknesses.
  \begin{raggeditemize}
    \itemhead{Blood Dependence} For every 24 hours that a vampire remains awake without ingesting at least one pint of blood from living creatures, its maximum hit points are reduced by 20.
      If its maximum hit points are reduced to 0 in this way, it dies and withers away into a pile of ash.
      This penalty is removed as soon as the vampire drinks a pint of blood.
      A vampire can can enter a torpor to survive without blood.
      While in a torpor, it is unconscious until it smells blood nearby.
      It can survive while in torpor for a number of consecutive centuries equal to its \glossterm{rank} before it withers away to dust.
    \itemhead{Consecrated Ground} Whenever a vampire enters consecrated ground, it takes 10 damage and becomes \stunned as a condition.
      This damage is repeated at the during each subsequent \glossterm{action phase} that the vampire spends on consecrated ground.
    \itemhead{Garlic} Whenever a vampire smells or touches garlic, it takes 10 damage and becomes \frightened by any creatures bearing garlic as a condition.
      This damage is repeated at the during each subsequent \glossterm{action phase} that the vampire spends exposed to garlic.
    \itemhead{Holy Water} Whenever a vampire takes damage from holy water, it becomes \stunned as a condition.
    \itemhead{Running Water} Whenever a vampire touches or flies over running water, it takes 10 damage and \glossterm{briefly} becomes \paralyzed.
      This applies as long as the vampire is within 100 feet of the running water, even the water is underground or under a bridge.
      It can use the \ability{struggle} ability to move despite being paralyzed, but only towards the closest shore.
      This damage is repeated at the during each subsequent \glossterm{action phase} that the vampire spends touching or flying over running water.
    \itemhead{Silver} Vampires are \vulnerable to strikes using silvered weapons.
    \itemhead{Sunlight} Whenever a vampire is exposed to sunlight, it takes 10 damage and becomes \stunned as a condition.
      This damage is repeated at the during each subsequent \glossterm{action phase} that the vampire spends in true sunlight.
    \itemhead{Unmirrored} Vampires have no reflection in mirrors, including their clothes and equipment.
      This can allow careful observers to identify vampires.
    \itemhead{Wooden Stakes} If a vampire is \glossterm{injured} by a critical strike using a wooden stake, the stake becomes impaled in its heart.
      The vampire becomes \paralyzed until the stake is removed.
      A wooden stake is a \weapontag{Light} improvised weapon that deals 1d4 damage.
  \end{raggeditemize}

  \subsection{Vampire Archetype}

    \cf{Vmp}[1]{Blood Drain} Whenever a living creature with blood loses hit points from a \glossterm{strike} using your bite natural weapon, you can increase your \glossterm{fatigue level} by one.
      When you do, you regain \glossterm{hit points} at the end of the round.
      The recovery is equal to the hit points the target lost from the attack, ignoring negative hit points and any damage increase from critical hits.

    \cf{Vmp}[2]{Gentle Fangs} Whenever you deal damage using your bite natural weapon, you can choose not to reduce the target's hit points below 0, or you can treat the damage as \glossterm{subdual damage}.
      In addition, damage dealt using your bite natural weapon does not wake sleeping creatures unless you inflict a vital wound.

    \magicalcf{Vmp}[2]{Reviving Coffin} You can designate a coffin as your home by resting in it for 24 consecutive hours.
      When you take a \glossterm{long rest} in your home coffin, you recover two \glossterm{vital wounds} instead of one.
      In addition, you can cross running water without penalty while in your home coffin.

      When you die, if your corpse is placed in your home coffin, you can be resurrected after 24 hours.
      You can only be resurrected in this way if you were killed by a vital wound with a \glossterm{vital roll} of \minus3 or higher.
      If you were killed by a more severe vital wound, or by some other effect, even your home coffin cannot save you.

    \magicalcf{Vmp}[3]{Charming Gaze}
      \begin{magicalsustainability}{Charming Gaze}{\abilitytag{Emotion}, \abilitytag{Subtle}, \abilitytag{Sustain} (minor), \abilitytag{Visual}}
        \abilityusagetime Standard action.
        \rankline
        Make an attack vs. Mental against all humanoid creatures in a \medarea cone from you.
        You take a \minus10 penalty to \glossterm{accuracy} with this attack against creatures who have made an attack or been attacked since the start of the last round.
        \hit The target is \charmed by you.
        Any act by you or by creatures that appear to be your allies that threatens or harms the charmed person breaks the effect.
        Harming the target is not limited to dealing it damage, but also includes causing it significant subjective discomfort.
        An observant target may interpret overt threats to its allies as a threat to itself.

        \rankline
        The attack's \glossterm{accuracy} increases by \plus2 for each rank beyond 3.
      \end{magicalsustainability}

    \magicalcf{Vmp}[4]{Creature of the Night}
      \begin{magicalattuneability}{Creature of the Night}{\abilitytag{Attune}}
        \abilityusagetime Standard action.
        \rankline
        You \glossterm{shapeshift} into the form of a Tiny bat, a Medium cloud of mist, or your normal humanoid form.
        \begin{raggeditemize}
          \item Bat: While in your bat form, you gain \trait{blindsense} (120 ft.) and \trait{blindsight} (30 ft.).
            You cannot speak and have no \glossterm{free hands}.
            All of your normal movement modes are replaced with an average fly speed with a 60 ft. height limit.
          \item Mist: While in your mist form, you become \trait{floating}, \trait{intangible}, and \trait{legless}.
            You cannot speak and have no \glossterm{free hands}.
            All of your normal movement modes are replaced with a slow \glossterm{fly speed} with a 30 foot \glossterm{height limit} (see \pcref{Flight}).
            Since you have no \glossterm{walk speed}, flying in this way does not penalize your defenses.
        \end{raggeditemize}

        In either non-humanoid form, you are unable to take any standard actions, but you can still take \glossterm{move actions} in place of standard actions.
        Since you have no walk speed in those forms, flying does not penalize your Armor or Reflex defenses.
        You cannot use this ability while \paralyzed.
      \end{magicalattuneability}

    \magicalcf{Vmp}[5]{Reviving Coffin+} You can designate up to three home coffins, rather than only one.
      This can allow you to travel with one coffin while keeping others safe for emergencies.

    \cf{Vmp}[5]{Unholy Resilience} You gain a \plus1 bonus to your Armor, Brawn, Reflex, and Mental defenses.
      However, you take a \minus2 penalty to your Fortitude defense.

    % TODO: proper EA math, this is probably too strong
    \magicalcf{Vmp}[6]{Dominating Gaze}
      \begin{magicalactiveability}{Dominating Gaze}[\abilitytag{Emotion}, \abilitytag{Visual}]
        \abilityusagetime Standard action.
        \rankline
        Make an attack vs. Mental against all humanoid \glossterm{enemies} within a \medarea \glossterm{cone} from you.
        \hit The target is \stunned as a \glossterm{condition}.
        While it is \glossterm{injured}, it is \confused instead of stunned.
        \crit If the target was already confused from a previous use of this ability, you may \glossterm{attune} to this ability.
        When you do, the target becomes \dominated by you for the duration of that attunement.
        This attunement only allows you to control one creature, not each target of this spell, and you can only attune to this effect once.

        \rankline
        The attack's \glossterm{accuracy} increases by \plus2 for each rank beyond 6.
      \end{magicalactiveability}

    \magicalcf{Vmp}[7]{Blood Drain+} You can use this ability without increasing your fatigue level.
    After you do, you \glossterm{briefly} cannot do so again.

    \magicalcf{Vmp}[7]{Eternal Undeath} Your \ability{reviving coffin} ability can revive you from any cause of death or severity of vital wound.
      As long as some part of your corpse, even just a pinch of ash, is placed inside one of your home coffins, you will resurrect after 24 hours.
      Only the destruction of all of your home coffins or the total annihilation of your corpse can prevent your return.

  \subsection{Base Class Effects}
    \highhpprogressiontable

    If you choose vampire as your base class, you gain the following benefits.

    \cf{Vmp}{Hit Points}
      You have 8 hit points \add twice your Constitution, plus 2 hit points per level beyond 1.
      This increases as your level increases, as indicated below.
      \begin{raggeditemize}
        \itemhead{Level 7} 20 hit points \add three times your Constitution, plus 3 hit points per level beyond 7.
        \itemhead{Level 13} 40 hit points \add six times your Constitution, plus 6 hit points per level beyond 13.
        \itemhead{Level 19} 80 hit points \add twelve times your Constitution, plus 12 hit points per level beyond 19.
      \end{raggeditemize}

    \cf{Vmp}{Defenses}
      You gain a \plus3 bonus to your Brawn, Fortitude, Mental, and Reflex defenses.

    \cf{Vmp}{Resources}
      \begin{raggeditemize}
          \item \glossterm{Attunement points}: 4 (see \pcref{Attunement Points}).
          \item \glossterm{Fatigue tolerance}: 2 \add your Constitution (see \pcref{Fatigue}).
          \item \glossterm{Insight points}: 1 \add your Intelligence (see \pcref{Insight Points}).
          \item \glossterm{Trained skills}: 4 from among your \glossterm{class skills}, plus additional trained skills equal to your Intelligence if it is positive (see \pcref{Skills}).
      \end{raggeditemize}

    \cf{Vmp}{Weapon Proficiencies}
      You are proficient with all non-exotic weapons.

    \cf{Vmp}{Armor Proficiencies}
      You are proficient with light armor.

    \cf{Trl}{Starting Items and Equipment}
    You can start with the following items and equipment:
    \begin{raggeditemize}
        \item Buff leather
        \item Any two of the following: club, dagger, broadsword, two handaxes, or spear
        \item A buckler
        \item A standard adventuring kit (see \pcref{Standard Adventuring Kit}).
        \item A rank 0 wealth item (1 gp)
    \end{raggeditemize}

    \cf{Vmp}{Skills}
      You have the following \glossterm{class skills}:
      \begin{raggeditemize}
        \itemhead{Strength} Climb, Jump.
        \itemhead{Dexterity} Balance, Stealth.
        \itemhead{Intelligence} Deduction, Disguise, Knowledge (dungeoneering and religion)
        \itemhead{Perception} Awareness, Creature Handling, Deception, Persuasion, Social Insight.
        \itemhead{Other} Intimidate, Persuasion
      \end{raggeditemize}
