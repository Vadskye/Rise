\chapter{Uncommon Species}\label{Uncommon Species}

\section{Animal Hybrid}
  Animal hybrids are humanoid creatures that are a combination of humans and animals.
  The abilities of an animal hybrid depend on the type of animal it is based on.

  \parhead{Size} Medium.
  \parhead{Attributes} The attributes of an animal hybrid depend on its size.
  \parhead{Special Abilities} As the original animal.
  \parhead{Automatic Languages} Common and any one \glossterm{common language} (see \tref{Common Languages}).

  \subsection{Sample Animal Hybrids}

    \parhead{Hybrid Bee}

    \subparhead{Special Abilities}
    \parhead{Attribute} \plus1 Dexterity, \minus1 Constitution.
    \begin{raggeditemize}
      \itemhead{Low-light Vision} A hybrid bee has \sense{low-light vision}, allowing it to see clearly in \glossterm{dim illumination} (see \pcref{Low-light Vision}).
      \itemhead{Stinger} A hybrid bee has a stinger natural weapon (see \pcref{Natural Weapons}).
        Whenever it causes a creature to lose \glossterm{hit points} with that natural weapon, the struck creature is poisoned by giant wasp venom (see \pcref{Poison}).
        % Note that this doesn't have the stage 3 damage of giant wasp venom. That's fine because bees aren't wasps.
        Its stage 1 effect makes the target \slowed while the poison lasts.
      \itemhead{Winged Agility} A hybrid bee has wings that are not strong enough to help it fly.
        However, the wings still help it stabilize its movements.
        It gains a \plus3 bonus to the Balance skill, and it gains a \plus5 foot bonus to its maximum horizontal jump distance (see \pcref{Jumping}).
        This increases its maximum vertical jump distance normally.
    \end{raggeditemize}

    \parhead{Hybrid Shark}

    \subparhead{Special Abilities}
    \begin{raggeditemize}
      \itemhead{Bloodscent} A hybrid shark has the scent ability (see \pcref{Scent}).
        In addition, it gains a \plus10 bonus to Awareness checks to detect blood.
      \itemhead{Bite} A hybrid shark's mouth is elongated, which it can use as a bite attack (see \pcref{Natural Weapons}).
        A hybrid shark's bite deals 1d6 damage.
      \itemhead{Gills} You can breathe water as easily as a human breathes air, preventing you from drowning or suffocating underwater.
      \itemhead{Swim Speed} A hybrid shark has an average \glossterm{swim speed}.
    \end{raggeditemize}

    \parhead{Hybrid Wolf}

    \subparhead{Special Abilities}
    \begin{raggeditemize}
      \itemhead{Scent} A hybrid wolf has the scent ability (see \pcref{Scent}).
      \itemhead{Bite} A hybrid wolf's mouth is elongated, which it can use as a bite attack (see \pcref{Natural Weapons}).
        A hybrid wolf's bite deals 1d6 damage.
      \itemhead{Low-light Vision} A hybrid wolf has \sense{low-light vision}, allowing it to see clearly in \glossterm{dim illumination} (see \pcref{Low-light Vision}).
    \end{raggeditemize}

\section{Automaton}
  An automaton appears to be a humanoid construct, like a golem.
  Its body is made from some combination of stone, wood, and metal.
  However, its artificial body is inhabited by a true soul, making it an \trait{indwelt} (see \pcref{Indwelt}).

  \parhead{Size} Medium.
  \parhead{Attributes} \plus1 Constitution or Intelligence, \minus1 Dexterity.
  \parhead{Special Abilities}
  \begin{raggeditemize}
    \itemhead{Artificial Life} Automatons are not alive. They are invalid targets for abilities which only affect living creatures, including poisons and most healing abilities. In addition, they do not need to eat, drink, or sleep.
    \itemhead{Automaton Archetype} Automatons only gain two class archetypes instead of three.
      Instead, they treat the Automaton archetype as one of their archetypes, and they gain ranks in it just like they gain ranks in class archetypes.
    \itemhead{Manual Repair} A Craft skill relevant to the automaton's body can be used to achieve the same effects that the Medicine skill would have on a living creature.
    \itemhead{Mechanical Body} Automatons are considered both objects and creatures, and are affected by abilities which affect either.
      They are always considered to be \glossterm{attended} by themselves, so they are never affected by abilities that only affect unattended objects, even while unconscious.
    \itemhead{Mechanical Intuition} Automatons gain a \plus2 bonus to the Devices skill and one Craft skill of their choice.
  \end{raggeditemize}

\input{generated/automaton.tex}

\section{Awakened Animal}

  Awakened animals are animals that have been granted sentience by the \spell{awaken} ritual.
  The abilities of an awakened animal depend on the type of animal it is.

  \parhead{Size} Small or Medium, as original animal.
  \parhead{Attributes} The attributes of an awakened animal depend on its size.
  \subparhead{Medium} No change.
  \subparhead{Small} \minus2 Strength, \plus1 Dexterity.
  \parhead{Special Abilities} As the original animal.
  \parhead{Automatic Languages} Common.

  \subsection{Sample Awakened Animals}

    \parhead{Cat}

    \subparhead{Size} Small. This gives a cat a 20 foot \glossterm{base speed} and a \plus4 bonus to the Stealth skill, among other effects (see \pcref{Size Categories}).
    \subparhead{Attributes} \minus2 Strength, \plus1 Dexterity
    \subparhead{Special Abilities}
    \begin{raggeditemize}
      \itemhead{Claws} A cat's paws end in claws, which it can use to attack (see \pcref{Natural Weapons}). A cat's claws have a \plus2 accuracy bonus and deal 1d4 damage.
      \itemhead{Low-light Vision} A cat has \sense{low-light vision}, allowing it to see clearly in \glossterm{dim illumination} (see \pcref{Low-light Vision}).
      \itemhead{Quadrupedal} A cat is \trait{quadrupedal}, which gives it a \plus10 foot bonus to its \glossterm{speed}.
      \itemhead{Scent} A cat has the scent ability (see \pcref{Scent}).
    \end{raggeditemize}

\section{Changeling}

  \parhead{Size} Medium.
  \parhead{Attributes} No change.
  \parhead{Special Abilities}
  \begin{raggeditemize}
    \itemhead{Alter Shape} A changling can change its body using the \ability{alter shape} ability.
    \begin{activeability}{Alter Shape}{Standard action}
        \rankline
        You make a Disguise check to alter your appearance (see \pcref{Change Appearance}).
        This physically changes your body to match the results of your disguise.
        You gain a \plus4 bonus on the check, and you ignore penalties for changing your gender, species, and age.
        However, this effect is unable to alter your equipment, size category, or number of limbs.

        This ability lasts until you \glossterm{dismiss} it or until you use it again.
      \end{activeability}
    \itemhead{Skilled} A changeling gains an additional \glossterm{trained skill} (see \pcref{Skills}).
  \end{raggeditemize}
  \parhead{Bonus Languages} Any.
  \parhead{Automatic Languages} Common, any two \glossterm{common languages}.

\section{Dragon}
  Ancient dragons are magical creatures of immense power and wisdom, and are far more powerful than any ordinary character of the same level.
  However, young dragons can be played as characters, though their unique abilities do pose unique challenges.

  \parhead{Creature Type} Unlike most other playable species, dragons are magical beasts instead of humanoids.
  \parhead{Size} Small. This gives a dragon a 20 foot \glossterm{base speed} and a \plus1 bonus to their Reflex defense, among other effects (see \pcref{Size Categories}).
  \parhead{Attributes} \minus2 Strength, \plus1 Dexterity.
  \parhead{Special Abilities}
  \begin{raggeditemize}
    \itemhead{Dragon Archetype} Dragons only gain two class archetypes instead of three.
      Instead, they treat the Dragon archetype as one of their archetypes, and they gain ranks in it just like they gain ranks in class archetypes.
    \itemhead{Draconic Senses} Dragons have \sense{darkvision} with a 60 foot range, allowing them to see in complete darkness (see \pcref{Darkvision}).
      In addition, dragons gain \sense{low-light vision}, allowing them to see clearly in \glossterm{dim illumination} (see \pcref{Low-light Vision}).
    \itemhead{Draconic Scales} Dragons gain a \plus2 bonus to their Armor defense.
    \itemhead{Draconic Weapons} Dragons have a bite natural weapon and two claw natural weapons.
      For details, see \pcref{Natural Weapons}.
    \itemhead{Draconic Wings} Dragons have scaly wings that sprout from their backs.
      These wings grant them an average \glossterm{glide speed} (see \pcref{Aerial Movement}).
    \itemhead{Dragon Type} Each dragon has a single type from among the dragon types on \trefnp{Dragon Species Types}.
      They are immune to attacks with their associated ability tag.
    \itemhead{Limited Equipment} A dragon's claws are not able to effectively wield shields or manufactured weapons.
      They can wear armor, but it is treated as \glossterm{barding} instead of normal armor, reducing its effectiveness (see \pcref{Barding}).
    \itemhead{Quadrupedal} A dragon is \trait{quadrupedal}, which gives it a \plus10 foot bonus to its \glossterm{speed}.
  \end{raggeditemize}
  \parhead{Automatic Languages} Common, Draconic, any one \glossterm{common language}.

    \begin{columntable}
      % This is the same as the Dragon Types table in the Feats chapter, but it has a different name to avoid label nonsense
      \columncaption{Dragon Species Types}
      \begin{dtabularx}{\columnwidth}{l >{\lcol}X >{\lcol}X}
        \tb{Dragon} & \tb{Tag} & \tb{Breath Weapon} \tableheaderrule
        Black       & \atAcid             & \areamed, 5 ft. wide line \\
        Blue        & \atElectricity      & \areamed, 5 ft. wide line \\
        Brass       & \atFire             & \areamed, 5 ft. wide line \\
        Bronze      & \atElectricity      & \areamed, 5 ft. wide line \\
        Copper      & \atAcid             & \areamed, 5 ft. wide line \\
        Gold        & \atFire             & \areasmall cone           \\
        Green       & \atAcid             & \areasmall cone           \\
        Red         & \atFire             & \areasmall cone           \\
        Silver      & \atCold             & \areasmall cone           \\
        White       & \atCold             & \areasmall cone           \\
      \end{dtabularx}
    \end{columntable}

\input{generated/dragon.tex}

\section{Drow}

  Drow are an offshoot group of elves that live deep underground.
  The deep caves are a far harsher environment than the surface world.
  Resources are scarce, and dangerous monsters are far more common.
  In order to survive, drow were forced to adopt a variety of practices condemned by surface civilizations.
  The most notorious are their frequent use of poison, their refusal to take prisoners, their willingness to eat any non-drow creatures they kill, even sentient creatures.
  In addition, drow society tends to reward selfishness and ambition more explicitly than surface civilizations, and the vast majority of drow are evil.

  When drow find opportunities to reach the surface world, they seek to conquer territory for themselves, usually with great violence.
  They have always been defeated and banished back to their caves, but surface civilizations still remember the danger that drow pose.
  Even more so than tieflings or orcs, who are already viewed with suspicion, drow are anathema in almost any civilized society.
  Drow who escape the deep caves are more likely to find a peaceful existence on other planes that do not fear an underground invasion.

  \parhead{Size} Medium.
  \parhead{Attributes} \minus1 Constitution, \plus1 Dexterity
  \parhead{Special Abilities}
  \begin{raggeditemize}
    \itemhead{Darkvision} Drow have \sense{darkvision} with a 120 foot range, allowing them to see in complete darkness (see \pcref{Darkvision}).
    \magicalitemhead{Deep Darkness}
    \begin{magicalsustainability}{Deep Darkness}{Standard action}
        \abilitytags \abilitytag{Sustain} (standard)
        \rankline
        You create a void of darkness in a \medarea radius \glossterm{zone} within \medrange.
        \glossterm{Bright illumination} and \glossterm{brilliant illumination} within or passing through that area is dimmed to be no brighter than \glossterm{dim illumination}.
        Any object or effect which blocks light also blocks this ability's effect.
      \end{magicalsustainability}
    \itemhead{Drow Prejudice} Almost all surface-dwellers have negative associations with drow.
      Drow have an Opposition relationship with most people that they meet, which influences people's behavior and makes Persuasion checks harder (see \pcref{Persuasion}).
      People in some locations, such as deep underground, do not have this attitude.
    \itemhead{Keen Senses} Drow gain a \plus2 bonus to the Awareness skill (see \pcref{Awareness}).
    \itemhead{Poison Tolerance} Drow are \impervious to poison.
    \itemhead{Sensitive Eyes} Drow take a \minus2 penalty to \glossterm{accuracy} while they are in \glossterm{bright illumination}.
      This penalty is doubled while they are in \glossterm{brilliant illumination}.
    \itemhead{Trance} Drow do not sleep, and are immune to \magical effects that would cause them to sleep.
      Instead of sleeping, drow can trance for 4 hours.
      An elf in trance may make Perception-based checks at a \minus5 penalty.
      Drow must still avoid strenuous activity for 8 hours to heal and gain other benefits of taking a \glossterm{long rest}.
  \end{raggeditemize}
  \parhead{Automatic Languages} Common, Elven, Undercommon

\section{Dryaidi}

  Dryaidi are humanoid creatures with plantlike characteristics.
  They might have leaves instead of hair, a green skin tone, or rough, barky skin.
  They are descended from dryads, and share some fey heritage and an affinity for trees.

  \parhead{Size} Medium.
  \parhead{Attributes} No change.
  \parhead{Special Abilities}
  \begin{raggeditemize}
    \itemhead{Dryad Archetype} Dryaidi only gain two class archetypes instead of three.
      Instead, they treat the Dryad archetype as one of their archetypes, and they gain ranks in it just like they gain ranks in class archetypes.
    \magicalitemhead{Enchanting Appearance} A dryaidi gains a \plus2 \glossterm{enhancement bonus} to the Creature Handling, Perform, and Persuasion skills.
    \itemhead{Fey Vulnerability} Dryaidi are \vulnerable to cold iron weapons.
    \magicalitemhead{Tree Bond} A dryaidi must be bonded with a specific tree.
      The tree must be at least a hundred years old, healthy, and intact.
      Forming a bond or severing a bond takes one week of meditation and ritual, periodically interrupted by rest.
      Forming a bond also requires asking permission from the tree through the ritual.
      Any individual tree can only be bonded to one dryad or dryaidi in this way.

      As long as the bonded tree remains healthy and intact, the dryaidi gains a \plus1 bonus to Mental defense and a \plus1 bonus to its \glossterm{fatigue tolerance}.
      If the bonded tree becomes unhealthy, is seriously damaged, or is killed, these bonuses are inverted into penalties until the dryaidi forms a bond with a new tree.
      A bonded dryaidi can passively observe the general health and status of the tree it bonded to.
    \magicalitemhead{Verdant Flourishing} Dryaidi can use the \spell{bramblepatch} and \spell{rapid growth} \glossterm{cantrips} from the \sphere{verdamancy} sphere.
      If they already have access to that sphere, they can sustain those cantrips as a \glossterm{free action} instead of as a minor action.
  \end{raggeditemize}
  \parhead{Automatic Languages} Common, Sylvan.

\input{generated/dryad.tex}

\section{Eladrin}

  \parhead{Size} Medium.
  \parhead{Attributes} \minus1 Constitution, either \plus1 Dexterity or \plus1 Willpower
  \parhead{Special Abilities}
  \begin{raggeditemize}
    \itemhead{Fae Step}
    \begin{magicalactiveability}{Fae Step}{Standard action}
        \rankline
        You \glossterm{teleport} horizontally to a location within \shortrange.

        \rankline
        This ability improves based on your rank in your highest-rank archetype.
        \rank{3} The range increases to \medrange.
        \rank{5} The range increases to \longrange.
        \rank{7} The range increases to \distrange.
      \end{magicalactiveability}
    \itemhead{Fae Season} Eladrin respond strongly to their emotions, and change their abilities based on the season they currently represent.
      An eladrin must choose one of the following seasons when it finishes a \glossterm{short rest}.
      The chosen season lasts until it changes to a different season.
      \subcf{Spring} \plus1 bonus to Mental defense, \minus1 penalty to Fortitude defense.
      Eladrin expressing the spring season are filled with the joy of a new year.
      However, they are also visibly thinner and more frail, as if recovering from a long winter.
      \subcf{Summer} \plus1 bonus to Fortitude defense, \minus1 penalty to Reflex defense.
      Eladrin expressing the summer season are visibly hearty and a little more plump.
      However, they also move with all the alacrity of a long summer day.
      \subcf{Autumn} \plus1 bonus to all checks, \minus1 penalty to \glossterm{accuracy}.
      Eladrin expressing the autumn season embody the spirit of the harvest.
      They are filled with goodwill towards all creatures, and prefer finding peaceful solutions to problems.
      Their bodies tend to be firm and toned, reflecting the hard work required to prepare for the winter.
      \subcf{Winter} \plus1 bonus to \glossterm{vital rolls}, \minus1 penalty to Mental defense.
      Eladrin expressing the winter season are prepared for the worst.
      They tend to be dour and pessimistic, but they press on despite the certainty of doom.
    \itemhead{Low-light Vision} Eladrin have \sense{low-light vision}, allowing them to see clearly in \glossterm{dim illumination} (see \pcref{Low-light Vision}).
    \itemhead{Trance} Eladrin do not sleep, and are immune to \magical effects that would cause them to sleep.
      Instead of sleeping, eladrin can trance for 4 hours.
      An eladrin in trance may make Perception-based checks at a \minus5 penalty.
      Eladrin must still avoid strenuous activity for 8 hours to heal and gain other benefits of taking a \glossterm{long rest}.
  \end{raggeditemize}
  \parhead{Species Feat Options}
  \parhead{Automatic Languages} Common, Sylvan, and any one \glossterm{common language} (see \tref{Common Languages}).

\section{Harpy}
  Harpies are winged creatures with the upper body of a humanoid and the lower body of a bird.
  Most harpies are female, but male harpies do exist.

  \parhead{Creature Type} Unlike most other playable species, harpies are monstrous humanoids instead of humanoids.
  \parhead{Size} Medium.
  \parhead{Attributes} \minus1 Intelligence, \plus1 Dexterity.
  \parhead{Special Abilities}
  \begin{raggeditemize}
    \itemhead{Harpy Archetype} Harpies only gain two class archetypes instead of three.
      Instead, they treat the Harpy archetype as one of their archetypes, and they gain ranks in it just like they gain ranks in class archetypes.
    \itemhead{Limited Equipment} Harpies can wear armor, but it is treated as \glossterm{barding} instead of normal armor, reducing its effectiveness (see \pcref{Barding}).
      Harpy talons are not able to effectively wield shields or manufactured weapons.
    \itemhead{Prehensile Talons} Harpies have a talon natural weapon on each foot (see \pcref{Natural Weapons}).
      In addition, they can use their feet as \glossterm{free hands}.
      They can make short hops to use their feet to attack or manipulate objects without suffering penalties for gliding or flying.
    \itemhead{Wings} Harpies have no arms or hands.
      Instead, they have feathered wings that sprout from their shoulders.
      These wings grant them an average \glossterm{glide speed} (see \pcref{Aerial Movement}).
    \itemhead{Winged Agility} While a harpy is able to use its wings, it gains a \plus2 bonus to Armor defense, a \plus4 bonus to the Balance skill, and a \plus10 foot bonus to its maximum horizontal jump distance (see \pcref{Jumping}).
  \end{raggeditemize}
  \parhead{Automatic Languages} Common.

% TODO: make sure this is the same as the explicit latex here before removing the explicit latex.
\input{generated/harpy.tex}

\section{Incarnation}

  An incarnation is a physical embodiment of an element or form of energy, like an elemental.
  Unlike elementals, incarnations are alive and have souls.
  Most incarnations are created on the plane associated with their element or energy and never leave that plane.
  However, in rare circumstances involving powerful magic, incarnations can sometimes be created in other planes.

  \parhead{Creature Type} Unlike most other playable species, incarnations are planeforged instead of humanoids.
  \parhead{Size} Medium.
  \parhead{Attributes} One attribute gains a \plus1 bonus and another takes a \minus1 penalty, depending on the chosen element or energy.
  \parhead{Special Abilities}
  \begin{raggeditemize}
    \magicalitemhead{Essence Infusion} Each incarnation chooses one of the following tags as its essence infusion: \atAcid, \atAir, \atAuditory, \atCold, \atCompulsion, \atEarth, \atEmotion, \atElectricity, \atFire, \atVisual, or \atWater.
      All of the incarnation's \glossterm{strikes} gain that tag, and it is \impervious to attacks with that tag.
    \magicalitemhead{Essence Vulnerability} Each incarnation chooses a tag to be \vulnerable to.
      It can choose any of the valid tags for an \textit{essence infusion}, including the tag it chose as its essence infusion.
    \itemhead{Glowing} If appropriate to an incarnation's essence, it may naturally shed light as a torch.
      An incarnation cannot willingly disable this light, though the light can be covered by thick clothing.
    \itemhead{Incarnation Archetype} Incarnations only gain two class archetypes instead of three.
      Instead, they treat the Incarnation archetype as one of their archetypes, and they gain ranks in it just like they gain ranks in class archetypes.
    \itemhead{Poison Resistance} An incarnation is \impervious to \atPoison attacks.
    \itemhead{Unusual Body} An incarnation's body can be unusually light or heavy compared to an ordinary creature's body.
      It can be \glossterm{lightweight} or \glossterm{heavyweight}, depending on its essence.
  \end{raggeditemize}
  \parhead{Automatic Languages} Common.

  Incarnations can come in two forms: tethered or untethered.
  Tethered incarnations are bipedal, with two legs and two arms.
  They have the same basic body shape and functionality as a human, though they may have unusual proportions.

  Untethered incarnations have no arms or legs.
  They may appear as a perfect sphere, as intricate shifting arcane glyphs, or any other appearance depending on their nature.
  The appearance of any individual untethered incarnation is consistent.
  They are not shapeshifters or masters of disguise, simply unbound by ordinary conceptions of a ``normal'' body shape.
  An untethered incarnation has the following special abilities:
  \begin{raggeditemize}
    \itemhead{Esoteric Body} The incarnation has no arms, legs, or \glossterm{free hands}.
      This means it is unable to use manufactured weapons or \glossterm{somatic components}.
      It can wear armor, but the armor is treated as \glossterm{barding} instead of normal armor, reducing its effectiveness (see \pcref{Barding}).
      It has no \glossterm{walk speed}.
      Since it has no legs, it is immune to being \prone, but it also cannot jump.
    \magicalitemhead{Flight} The incarnation's only movement mode is an average \glossterm{fly speed} with a 5 foot height limit (see \pcref{Aerial Movement}).
      Since it is native to the air, flying does not penalize its Armor or Reflex defenses.
      It is also \trait{floating}, so it does not need to fly every phase to avoid falling.
    \itemhead{Ram} The incarnation gains a ram \glossterm{natural weapon}.
      It deals 1d6 damage and has the \weapontag{Resonating} weapon tag (see \pcref{Weapon Tags}).
    \itemhead{Uniform Composition} The incarnation is immune to \glossterm{critical hits} from \glossterm{strikes}.
  \end{raggeditemize}

\input{generated/incarnation.tex}

\section{Kit}

  Kit are humanoid creatures that have noticeable foxlike characteristics.
  They are descended from natural fox spirits.
  All kit have at least one tail, and some have multiple tails.
  Their tails are distinctly fluffy and fox-like, and most kit put effort into concealing their tails to avoid revealing their true nature.

  \parhead{Size} Medium.
  \parhead{Attributes} No change.
  \parhead{Special Abilities}
  \begin{raggeditemize}
    \itemhead{Foxlike Agility} Kit gain a \plus2 bonus to the Balance and Stealth skills.
    \itemhead{Illusory Guise \sparkle} As a standard action, a kit can magically disguise its physical appearance in minor ways.
      This functions like the \textit{change appearance} ability with a \plus4 bonus, except that a kit cannot change the appearance of its equipment, creature type, or number of limbs, including any tails it may have (see \pcref{Change Appearance}).
      This ability lasts until the kit \glossterm{dismisses} the ability or uses it again.
    \itemhead{Instictive Trickster} Kit gain a \plus2 bonus to the Deception and Social Insight skills.
    \itemhead{Low-light Vision} Kit have \sense{low-light vision}, allowing them to see clearly in \glossterm{dim illumination} (see \pcref{Low-light Vision}).
  \end{raggeditemize}
  \parhead{Automatic Languages} Common, any one \glossterm{common language}.

\section{Naiadi}
  Naiadi are humanoid creatures descended from water spirits called naiads.
  Most naiadi are unusually physically appealing, but show no other outward signs of their heritage.

  \parhead{Size} Medium.
  \parhead{Attributes} No change.
  \parhead{Special Abilities}
  \begin{raggeditemize}
    \itemhead{Aquatic Essence} Naiadi are \impervious to \atFire and \atWater attacks.
      However, they are \vulnerable to \atCold and \atElectricity attacks.
    \itemhead{Create Water} A naiadi can cast the \spell{create water} cantrip.
      When they do so, they do not require \glossterm{verbal components} or \glossterm{somatic components}, and their spellcasting rank is considered to be equal to their rank in their highest rank archetype.
    \itemhead{Enchanting Appearance} A naiadi gains a \plus2 \glossterm{enhancement bonus} to the Creature Handling, Perform, and Persuasion skills.
    \itemhead{Fey Vulnerability} Naiadi are \vulnerable to cold iron weapons.
    \itemhead{Low-light Vision} Naiadi have \sense{low-light vision}, allowing them to see clearly in \glossterm{dim illumination} (see \pcref{Low-light Vision}).
    \itemhead{Naiad Archetype} Naiadi may choose three class archetypes, as normal.
      However, you may choose the Naiad archetype in place of one of your class archetypes.
      If you do, you gain ranks in it just like you gain ranks in class archetypes.
    \itemhead{Water Affinity} A naiadi has an average \glossterm{swim speed}.
      In addition, they can breathe clean water like a human breathes air.
  \end{raggeditemize}
  \parhead{Automatic Languages} Common, Sylvan, any one \glossterm{common language}.

\input{generated/naiad.tex}

\sectiongraphic*{Oozeborn}{width=\columnwidth}{classes/oozeborn}
  Oozeborn are ooze creatures that have gained true sentience through a strange quirk of their birth.
  They are very rare to see in civilized lands, as most oozeborn lack the opportunity to discover more than the dark caves in which they were spawned.
  Since they often grow up without mentorship from any civilized creature, oozeborn tend to have odd mannerisms and a poor ability to mask their emotions, even after spending years in civilization.
  Old oozeborn may eventually adapt to societal norms and act perfectly natural, or they may abandon civilized company entirely.

  The body of an oozeborn is amorphous, and they lack any identifiable internal organs.
  Their natural color depends on the nature of the ooze that spawned them, so green and gray are the most common colors.
  Adventuring oozeborn typically assume a bipedal shape for both practical and social convenience, but their natural shape is a loosely spherical blob.
  Unconscious oozeborn revert to their default state automatically, though some learn to maintain a semblance of cohesion while asleep.

  \parhead{Creature Type} Unlike most other playable species, oozeborn are animates instead of humanoids.
  \parhead{Size} Medium.
  \parhead{Attributes} \minus1 Intelligence, \plus1 Constitution.
  \parhead{Special Abilities}
  \begin{raggeditemize}
    \itemhead{Acidic Body} Ooozeborn are \impervious to \atAcid and \atPoison attacks. However, they are \vulnerable to \atEarth attacks.
    \itemhead{Amorphous Form} An oozeborn's natural form is a loosely spherical blob.
      They have a \minus10 foot penalty to their \glossterm{speed}, but they gain a \plus5 bonus to the Flexibility skill (see \pcref{Flexibility}).
      Since they have no legs, they are immune to being \prone.
      Most legless creatures cannot jump, but oozeborn can jump without penalty.
      They can also use the \ability{mold body} ability to adopt a particular shape.
      \begin{sustainability}{Mold Body}{Standard action}
        \abilitytags \abilitytag{Sustain} (free)
        \rankline
        You make a Disguise check to alter your appearance (see \pcref{Change Appearance}).
        This physically changes your body to match the results of your disguise.
        You gain a \plus4 bonus on the check, and you ignore penalties for changing your gender, species, age, and number of limbs.
        However, this effect is unable to alter your equipment or size category in any way.

        You cannot create more than four limbs with this ability, and a maximum of two \glossterm{free hands}.
        If you create four legs, you become \trait{quadrupedal}, so you gain a \plus10 foot bonus to your \glossterm{speed}.
        If you give yourself a standard humanoid shape, you can wear armor designed for humanoids without suffering the normal penalties for \glossterm{barding} (see \pcref{Barding}).

        You can sustain this ability for any length of time without mental strain, ignoring the normal 5 minute limit.
      \end{sustainability}
    \itemhead{Compressible Body} Oozeborn can compress their head and shoulders down to a minimum of a one inch radius, allowing them to squeeze through very small areas.
      Their clothing or armor is not compressed, so they may limit their ability to move through extremely narrow spaces.
    \itemhead{Darkvision} Oozeborn have \sense{darkvision} with a 60 foot range, allowing them to see in complete darkness (see \pcref{Darkvision}).
    \itemhead{Oozeborn Archetype} Oozeborn only gain two class archetypes instead of three.
      Instead, they treat the Oozeborn archetype as one of their archetypes, and they gain ranks in it just like they gain ranks in class archetypes.
  \end{raggeditemize}
  \parhead{Automatic Languages} Common.

\input{generated/oozeborn.tex}

\section{Sapling}

  Saplings are young treants that have left their forest home in search of adventure.
  They tend to be slow to think and act, but resilient once they have made up their mind.

  \parhead{Creature Type} Unlike most other playable species, saplings are considered animates instead of humanoids.
  \parhead{Size} Medium.
  \parhead{Attributes} \plus1 Constitution, \plus1 Willpower, \minus1 Dexterity, \minus1 Intelligence
  \parhead{Special Abilities}
  \begin{raggeditemize}
    \itemhead{Barkskin} A sapling gains a \plus2 bonus to Armor defense.
    \itemhead{Ingrain} A sapling can bury its roots into the ground.
    \begin{activeability}{Ingrain}{\glossterm{Minor action}}
        \rankline
        This ability can only be used if the sapling is \glossterm{grounded}.
        The sapling becomes \braced, and its \glossterm{speed} becomes 5 feet, regardless of any modifiers that normally apply.
        It cannot voluntarily stop being \glossterm{grounded} while this ability lasts.

        If the sapling finishes a \glossterm{long rest} with this ability active for the duration of the rest, it acquires nutrients sufficient to replace a day's worth of food and water.
        This ability lasts until the sapling ends it as a standard action, or until it stops being \glossterm{grounded}.
      \end{activeability}
    \itemhead{Limited Equipment} Saplings can wear armor, but it is treated as \glossterm{barding} instead of normal armor, reducing its effectiveness (see \pcref{Barding}).
    \itemhead{Made of Wood} Saplings are \vulnerable to \atFire attacks. In addition, they are both creatures and plants.
    \itemhead{Treant Archetype} Saplings only gain two class archetypes instead of three.
      Instead, they treat the Treant archetype as one of their archetypes, and they gain ranks in it just like they gain ranks in class archetypes.
    \itemhead{Tree Appearance} When a sapling stays perfectly still, observers must make a DV 15 Awareness check to recognize that it is not an ordinary tree.
      Careful observers may still notice that the ordinary tree has appeared where no tree used to be, so they may be suspicious of a sapling even if they do not pass this check.
    \itemhead{Unhurried and Unfaltering} Saplings have a \minus10 penalty to their \glossterm{speed}.
      However, a sapling's movement speed cannot be more than 10 feet slower than its \glossterm{base speed}, even while \slowed or under similar effects, except with the \ability{ingrain} ability.
      In addition, saplings are unaffected by \glossterm{difficult terrain} from inanimate natural sources, such as \glossterm{heavy undergrowth}.
  \end{raggeditemize}
  \parhead{Automatic Languages} Common, Sylvan.

\input{generated/treant.tex}

\sectiongraphic*{Tiefling}{width=\columnwidth}{optional rules/tiefling}

  Tieflings are humanoid creatures descended from fiends.
  They inherit a tendency towards evil from their ancestors, and are therefore viewed with great suspicion by most civilized societies.
  Good-aligned tieflings exist, but they may have difficulty using their natural talents for subterfuge and deceit for noble ends, and they often struggle with hidden vices.

  \parhead{Size} Medium.
  \parhead{Attributes} No change.
  \parhead{Special Abilities}
  \begin{raggeditemize}
    \itemhead{Darkvision} Tieflings have \sense{darkvision} with a 60 foot range, allowing them to see in complete darkness (see \pcref{Darkvision}).
    \itemhead{Demonic Prejudice} Most people have negative associations with tieflings thanks to the malign influence that demons have on the world.
      Tieflings have an Opposition relationship with most people that they meet, which influences people's behavior and makes Persuasion checks harder (see \pcref{Persuasion}).
      People in some locations, such as the Abyss, do not have this attitude.
    \itemhead{Hellfire Tolerance} Tieflings are \impervious to \atFire attacks.
    \itemhead{Infernal Presence} Tieflings gain a \plus2 bonus to the Deception and Intimidate skills.
    \itemhead{Tiefling Archetype} You may choose three class archetypes, as normal.
      However, you may choose the Tiefling archetype in place of one of your class archetypes.
      If you do, you gain ranks in it just like you gain ranks in class archetypes.
      You cannot choose tiefling as your base class.
  \end{raggeditemize}
  \parhead{Automatic Languages} Abyssal, Common, any one \glossterm{common language}.

  % TODO: convert this into a Rust-style archetype?
  % Not currently easy since there isn't a Tiefling base class.
  \subsection{Tiefling Archetype}
    % A normal short range damage spell would deal drX+1.
    % Drop to drX for the teleport, and drX-1 for the AOE.
    \magicalcf{Tif}[1]{Abyssal Hop}
    \begin{magicalactiveability}{Abyssal Hop}{Standard action}
        \abilitytags \atFire
        \rankline
        You teleport horizontally into an unoccupied location within \shortrange on a stable surface that can support your weight.
        If the destination is invalid, this spell fails with no effect.
        In addition, make an attack vs. Reflex against each \glossterm{enemy} adjacent to your location after you arrive.
        \hit \damagerankzero.
        \miss Half damage.

        \rankline
        % Stay at drX-1, approximately
        \rank{2} The base damage increases to 1d6.
        \rank{3} The damage bonus from your power increases to be equal to your power.
        \rank{4} The base damage increases to 1d8.
        \rank{5} The base damage increases to 2d6.
        \rank{6} The damage bonus increases to 1d6 per 2 power.
        \rank{7} The damage bonus increases to 1d8 per 2 power.
      \end{magicalactiveability}

    \magicalcf{Tif}[2]{Infernal Ancestry} You deepen your connection to a particular aspect of your demonic ancestry.
      Choose one of the following infernal ancestries: hellfire conduit, tempting allure, or unholy might.
      You gain a benefit based on your chosen ancestry.
      \begin{raggeditemize}
        \item Hellfire Conduit: You gain a \plus1 bonus to your \glossterm{magical power}.
          In addition, when you use your \ability{abyssal hop} ability, it deals damage to each enemy in a \smallarea radius from your arrival location.
        \item Tempting Allure: You gain a \plus2 bonus to the Deception, Disguise, and Persuasion skills.
        \item Unholy Might: You gain two claw natural weapons and one bite natural weapon (see \pcref{Natural Weapons}).
          In addition, you gain a \plus1 bonus to your \glossterm{mundane power}.
      \end{raggeditemize}

    \magicalcf{Tif}[3]{Abysswalker} Once per phase, you can teleport horizontally instead of moving using your \glossterm{walk speed}.
    Teleporting a given distance costs movement equal to twice that distance.
    If this teleportation fails for any reason, you still expend that movement.

    \magicalcf{Tif}[4]{Infernal Ancestry+} The benefits of your \textit{infernal ancestry} ability improve.
      \begin{raggeditemize}
        % Normally, double defense is +1dr and burn is +1dr. This is more damage than that, but it's also a class feature, so it's probably fine?
        \item Hellfire Conduit: If your \ability{abyssal hop} ability also hits a target's Fortitude defense, you can repeat the attack against that during your next action.
        \item Tempting Allure: You can charm creatures.
          \begin{magicalactiveability}{Charming Temptation}{Standard action}
            \abilitytags \abilitytag{Emotion}, \abilitytag{Subtle}
            \rankline
            Make an attack vs. Mental against a humanoid creature within \medrange.
            You take a \minus10 penalty to \glossterm{accuracy} with this attack against creatures who have made an attack or been attacked since the start of the last round.
            \hit The target is \charmed by you until it takes a \glossterm{long rest}.
            Any act by you or by creatures that appear to be your allies that threatens or harms the charmed person breaks the effect.
            Harming the target is not limited to dealing it damage, but also includes causing it significant subjective discomfort.
            An observant target may interpret overt threats to its allies as a threat to itself.

            \rankline
            The attack's \glossterm{accuracy} increases by \plus2 for each rank beyond 4.
          \end{magicalactiveability}
        \item Unholy Might: When you use your \ability{abyssal hop} ability as a standard action, you can also make a \glossterm{mundane} \glossterm{strike} at your destination.
      \end{raggeditemize}

    \cf{Tif}[5]{Infernal Resilience} You gain a \plus1 bonus to your Fortitude, Reflex, and Mental defenses.

    \magicalcf{Tif}[6]{Abysswalker+} Teleporting a given distance only costs movement equal to that distance.
    In addition, you can teleport in this way any number of times per phase.

    \magicalcf{Tif}[7]{Infernal Ancestry++} The benefits of your \textit{infernal ancestry} ability reach their peak.
      \begin{raggeditemize}
        \item Hellfire Conduit: The magical power bonus increases to \plus2.
          In addition, your \ability{abyssal hop} ability also affects \glossterm{enemies} in a \smallarea radius from your starting point.
          Creatures in both areas are still only affected by the ability once.
        \item Tempting Allure: The skill bonuses from your \textit{infernal ancestry} ability increase to \plus4.
        \item Unholy Might: The mundane power bonus increases to \plus2.
          In addition, the damage dealt by the strike after you use \ability{abyssal hop} is doubled.
      \end{raggeditemize}

      % \subsection{Base Class Effects}
      %     If you choose tiefling as your base class, you gain the following benefits.

      %     \cf{Tif}{Defenses}
      %     You gain the following bonuses to your \glossterm{defenses}: \plus3 Fortitude, \plus5 Reflex, \plus7 Mental.

      %     \cf{Tif}{Resources} You have the following \glossterm{resources}:
      %     \begin{raggeditemize}
      %         \item Four \glossterm{attunement points}, which are required to use some items and abilities (see \pcref{Attunement Points}).
      %         \item A \plus2 bonus to your \glossterm{fatigue tolerance}, which makes it easier for you to use powerful abilities that fatigue you (see \pcref{Fatigue}).
      %         \item Two \glossterm{insight points}, which you can spend to gain additional abilities or proficiencies (see \pcref{Insight Points}).
      %         \item Five \glossterm{trained skills}, which you can spend to learn skills (see \pcref{Trained Skills}).
      %     \end{raggeditemize}

      %     \cf{Tif}{Weapon Proficiencies} 
      %     You are proficient with simple weapons.

      %     \cf{Tif}{Armor Proficiencies} 
      %     You are proficient with light and medium armor.

      %     \cf{Tif}{Skills}
      %     You have the following \glossterm{class skills}:
      %     \begin{raggeditemize}
      %         \item \subparhead{Strength} Climb.
      %         \item \subparhead{Dexterity} Balance, Sleight of Hand, Stealth.
      %         \item \subparhead{Intelligence} Craft, Disguise, Knowledge (arcana, planes)
      %         \item \subparhead{Perception} Awareness, Social Insight.
      %         \item \subparhead{Other} Deception, Intimidate, Perform, Persuasion.
      %     \end{raggeditemize}

\section{Troll}
  Trolls are large, ugly giants with tusks.
  They are famous for their supernatural regeneration abilities.
  A troll can survive even dismemberment, regrowing severed limbs over time.
  Most trolls have green skin, though other earthy skin tones are also possible.

  Trolls naturally grows lichen or other fungus on their bodies in addition to hair.
  The patterns and type of fungus depends on the troll.
  Some trolls prefer to style or entirely remove their surface fungus, just like humans may style or remove their hair.

  The body of a troll is suffused with fungus that is unique to troll biology.
  This fungus grants trolls their incredible healing abilities.
  However, it can be destroyed by fire and acid once exposed, leading to the troll's true death.
  Trolls can also be killed by pulverizing the body to an extreme extent, leaving nothing sufficiently intact to regenerate from.

  \parhead{Creature Type} Unlike most other playable species, trolls are monstrous humanoids instead of humanoids.
  \parhead{Size} Medium.
  \parhead{Attributes} \plus1 Strength, \plus1 Constitution, \minus1 Intelligence, \minus1 Perception.
  \parhead{Special Abilities}
  \begin{raggeditemize}
    \itemhead{Bite} Trolls have a bite natural weapon (see \tref{Natural Weapons}).
    \itemhead{Fungal Resilience} Trolls are \impervious to \atPoison attacks.
    \itemhead{Fungal Vulnerabilities} Trolls are \vulnerable to \atAcid and \atFire attacks.
    \itemhead{Hard to Kill} Trolls cannot be killed unless their body is almost entirely destroyed.
      If they would die from a vital wound effect of \minus5 or higher, they instead simply stay unconscious until the vital wound is healed.
      While they are unconscious in this way, if they take any damage from a \atAcid or \atFire ability, they immediately die.
    \itemhead{Subspecies} Every troll has a particular subspecies with specific effects, listed below.
      \subcf{Cave} Cave trolls are accustomed to life in deep caves.
      They have \sense{darkvision} with a 60 foot range, allowing them to see in complete darkness (see \pcref{Darkvision}).
      In addition, they gain a \plus2 bonus to the Stealth skill.
      \subcf{Forest} Forest trolls are smaller and crafter than other trolls.
      They gain a \plus1 bonus to their Intelligence, but take a \minus1 penalty to their Strength.
      In addition, they gain a \plus2 bonus to any one Craft skill.
      \subcf{Mountain} Mountain trolls are larger and stronger than even other trolls.
      They gain a \plus1 bonus to their Strength for the purpose of determining their \glossterm{weight limits} (see \pcref{Weight Limits}).
      \subcf{Scrag} Scrag trolls prefer to live in water, though they can breathe air and move on land.
      They gain a \plus3 bonus to the Endurance and Swim skills, and they can hold their breath for ten times the normal limit (see \pcref{Common Endurance Tasks}).
    \itemhead{Troll Archetype} Trolls only gain two class archetypes instead of three.
      Instead, they treat the Troll archetype as one of their archetypes, and they gain ranks in it just like they gain ranks in class archetypes.
  \end{raggeditemize}
  \parhead{Automatic Languages} Common, Giant.

\input{generated/troll.tex}

\section{Vampire}
  A vampire is an undead creature that must drink the blood of living creatures to survive.
  Unlike most undead creatures, vampires appear to be alive and human, allowing them to act normally in society.
  Vampires have great power, but also many dangerous weaknesses.

  The unusual blend of life and death in a vampire comes from the strange disease of vampirism.
  This blood-transmitted disease attacks the brain of an infected creature.
  Such destruction is lethal, and it does cause the death of the original creature's mind.
  However, the disease also takes over and replaces most of the brain's autonomous functions.
  The creature's heart keeps pumping, it continues to breathe, and so on, giving the dead creature a convincing imitation of life.
  Vampires are generally pale thanks to their poor circulation, but not impossibly so.
  If the disease is somehow cured, the vampire immediately dies.

  The half-death inflicted by vampirism can be confusing for a creature's soul.
  Although most of the soul passes on to its normal afterlife, fragments remain behind and cling to their original body.
  These give a newly born vampire a hint of its original personality, but only a hint.

  A newly born vampire, called a vampire spawn, does not have a full, independent soul.
  Its will is completely bound to the vampire that created it, called its sire.
  The sire's soul invades and replaces the vacuum left behind when the vampire's original soul was fractured by death and fled to the afterlife.

  Over time, by feeding on blood to strengthen its body, a vampire spawn can connect more deeply to its original soul.
  If allowed and guided by its sire, it can become strong enough to wrest the rest of its soul back from its afterlife.
  This unifies the creature's soul and makes it a new entity fully independent from its sire.
  A fully ensouled vampire is called a true vampire.

  Only true vampires can create new vampire spawn.
  Since each spawn claims a piece of its sire's soul to sustain its capacity for thought and action, most vampires can create few spawn at once, or else they risk losing their will.
  Despite this risk, many true vampires do not permit their spawn to become ensouled.
  Instead, they may prefer to maintain a small handful of spawn to serve them with no possibility of betrayal.
  A vampire with a soul powerful enough to command many dependent spawn is called a vampire lord.

  \parhead{Creature Type} Unlike most other playable species, vampires are undead instead of humanoids.
  \parhead{Size} Medium.
  \parhead{Attributes} \plus1 Strength and Dexterity, \minus1 Constitution.
  \parhead{Special Abilities}
  \begin{raggeditemize}
    \itemhead{Climb Speed} Vampires have a \glossterm{climb speed} 10 feet slower than their \glossterm{base speed}.
    \itemhead{Darkvision} Vampires have \sense{darkvision} with a 90 foot range, allowing them to see in complete darkness (see \pcref{Darkvision}).
    \itemhead{Fangs} Vampires have a bite natural weapon (see \tref{Natural Weapons}).
      These fangs retract when not in use, so vampires cannot be identified as non-human by their fangs unless they choose to expose them.
    \itemhead{Undead} Vampires are \trait{undead} instead of \trait{living}, and they take damage from most healing effects (see \pcref{Undead})).
      Since they are not living, they are immune to \atPoison effects.
    \itemhead{Unnatural Life} Unlike most undead creatures, vampires share some aspects of living creatures.
      They must breathe air, and they must sleep as much as humans do.
      They are \trait{humanoid}, have blood, and visibly bleed when injured.
    \itemhead{Unnatural Charm} Vampires gain a \plus2 bonus to the Persuasion skill.
    \itemhead{Vampire Archetype} Vampires only gain two class archetypes instead of three.
      Instead, they treat the Vampire archetype as one of their archetypes, and they gain ranks in it just like they gain ranks in class archetypes.
  \end{raggeditemize}

  \parhead{Vampire Weaknesses\sparkle}
  Vampires have a number of specific weaknesses.
  Many vampire weaknesses trigger on exposure to particular substances or circumstances.
  These weaknesses trigger immediately upon first contact, and are repeated at the start of each \glossterm{action phase} in subsequent rounds as long as the vampire remains exposed.
  \begin{raggeditemize}
    \itemhead{Blood Dependence} For every 24 hours that a vampire remains awake without ingesting at least one pint of blood from living creatures, its maximum hit points are reduced by 20.
      If its maximum hit points are reduced to 0 in this way, it dies and withers away into a pile of ash.
      This penalty is removed as soon as the vampire drinks a pint of blood.
      A vampire can enter a torpor to survive without blood.
      While in a torpor, it is unconscious until it smells blood nearby.
      It can survive while in torpor for a number of consecutive centuries equal to its \glossterm{rank} before it withers away to dust.
    \itemhead{Consecrated Ground} A vampire in consecrated ground takes 20 damage and becomes \stunned as a condition if it is not already stunned.
    \itemhead{Garlic} A vampire that smells garlic becomes \frightened by any creatures bearing garlic as a condition.
      In addition, creatures that have eaten garlic recently are treated as not having blood for the purpose of a vampire's abilities, so their blood cannot be drained.
    \itemhead{Holy Water} A vampire that touches holy water takes 20 damage and becomes \stunned as a condition if it is not already stunned.
    \itemhead{Running Water} A vampire that touches or passes over running water takes 10 damage and \glossterm{briefly} becomes \paralyzed.
      This applies as long as the vampire is within 100 feet of the running water, even the water is underground or under a bridge.
      It can use the \ability{struggle} ability to move despite being paralyzed, but only towards the closest shore.
    \itemhead{Silver} Vampires are \vulnerable to strikes using silver weapons.
    \itemhead{Sunlight} A vampire that touches sunlight takes 20 damage and becomes \blinded as a condition if it is not already blinded.
    \itemhead{Unmirrored} Vampires have no reflection in mirrors, including their clothes and equipment.
      This can allow careful observers to identify vampires.
    \itemhead{Wooden Stakes} If a vampire is \glossterm{injured} by a critical strike using a wooden stake, the stake becomes impaled in its heart.
      The vampire becomes \paralyzed until the stake is removed.
      A wooden stake is a \weapontag{Light} improvised weapon that deals 1d4 damage.
  \end{raggeditemize}

\input{generated/vampire.tex}
