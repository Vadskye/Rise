\section{Spell Descriptions}

\small

\pdfbookmark[2]{A}{SpellDescriptionsA}
\begin{comment}
\subsubsection{A}
\end{comment}

\spellsection{Ablate Impact}{2}
\spelldesc{You instantly reduce the force of an incoming blow.}
\spellinfo{Abjur (Shielding)}{Abjur}
\spelltwocol{\spelltime{Immediate action}}{\spellcmp{Verbal only}}
\spelldur{1 round}
\begin{spelltarget}{You}
    \spelleffect You gain physical damage reduction 10/force. This damage reduction increases by 1 per caster level above 4th.
\end{spelltarget}
\spellnotes This spell's damage reduction allows the subject to ignore the first 10 physical damage it takes each round. If it is hit by a attack that deals force damage, such as \spell{magic missile}, it cannot use its damage reduction for 1 round.

You can cast this spell instantaneously, quickly enough react to an opponent attacking you (but before the attack is rolled).

\spellsection{Ablative Shield}{1}
\spelldesc{You instantly encase yourself a shimmering field of magical energy, protecting you from hostile magic.}
\spellinfo{Abjur (Negation) [Magic]}{Abjur, Magic}
\spelltime{Immediate action}
\spellcmp{Verbal only}
\spelldur{1 round}
\begin{spelltarget}{You}
    \spelleffect You gain spell damage reduction 5/force. This damage reduction increases by 1 per caster level above 2nd.
\end{spelltarget}
\spellnotes This spell's damage reduction allows the subject to ignore the first 5 damage it takes each round from spells and spell-like abilities. If it is hit by a attack that deals force damage, such as \spell{magic missile}, it cannot use its damage reduction for 1 round.

Spells that are not subject to spell resistance are not affected by \spell{ablative shield}. You can cast this spell instantly - quickly enough to gain its benefits in an emergency. Casting the spell is an immediate action, so you can use this spell even when it's not your turn.

\spellsection{Acid Arrow}{2}
\spelldesc{You fire a magical arrow of acid from your hand that speeds to its target.}
\spellinfo{Conj (Creation) [Acid]}{Arcane}
\spellrng{\rngmed}
\spelldur{1 round per two caster levels}
\begin{spelltarget}{One creature or object}[Magic vs. Reflex]
    \spellsuccess 2d8 acid damage immediately, and d8 acid damage at the end of each round after the first.
\end{spelltarget}
\spellnotes If the target becomes submerged in water or takes at least ten points of cold or fire damage, this spell's effect ends.

\spellsectioncomma{Acid Arrow}{Greater}{5}
\spelldesc{You fire a magical arrow of acid from your hand that speeds to its target.}
\spellinfo{Conj (Creation) [Acid]}{Arcane}
\spellrng{\rngfar}
\spelldur{1 round per two caster levels}
\begin{spelltarget}{One creature or object}[Magic vs. Ref and Fort]
    \spellsuccess If you beat the target's Reflex defense, it takes 5d8 acid damage immediately, and 2d8 acid damage at the end of each round after the first. If you also beat the target's Fortitude defense, it is \vulnerable.
\end{spelltarget}
\spellnotes As \spell{acid arrow}, except that twenty points of fire or acid damage are required to end the effect.

\spellsection{Acid Fog}{8}
\spelldesc{A billowing mass of acidic vapors fills the area, slowing creatures down and obscuring sight.}
\spellinfo{Conj (Creation) [Acid, Fog]}{Arcane, Destruction}
\spelltwocol{\spellzone{\areamed radius cylinder}}{\spellrng{\rngmed}}
\spelldur{\durshort}
\spellline
\spelleffect Fog in the area, as \spell{solid fog}.
\begin{spelltrigger}{End of every round}
    \begin{spelltarget}*{Everything in the area}[Magic vs. Fortitude]
        \spellsuccess 4d6 acid damage.
        \spellfailure As above, but half damage.
    \end{spelltarget}
\end{spelltrigger}

\spellsection{Agony}{5}
\spelldesc{You inflict debilitating pain on your foe, crippling its ability to act.}
\spellinfo{Necro (Flesh)}{Arcane}
\spellrng{\rngmed}
\spelldur{\durshort}
\spellsr{Yes (Fortitude)}
\begin{spelltarget}{One creature}
    \spelleffect \minus4 penalty to attacks, defenses, and checks.
\end{spelltarget}

\spellsection{Aid}{2}
\spelldesc{You fill your ally with confidence, improving its resilience in combat.}
\spellinfo{Ench (Emotion) [Mind-Affecting, Morale]}{Divine}
\spellrng{\rngclose}
\spelldur{\durshort}
\spellsr{Yes (Will)}
\begin{spelltarget}{One creature}
    \spelleffect 10 temporary hit points \add 1 per caster level above 4th, and a \plus2 enhancement bonus to Will defense. \spellbonusscalingdescription
\end{spelltarget}
\spellnotes If the subject takes life damage, it loses all temporary hit points provided by this spell before applying the damage.

\spellsection{Air Walk}{4}
\spelldesc{You imbue the subject with the ability to walk on nothing but air.}
\spellinfo{Trans (Imbuement) [Air]}{Air, Divine, Nature, Travel}
\spellrng{\rngtouch}
\spelldur{\durshort}
\spellsr{Yes (Will)}
\begin{spelltarget}{One creature (Gargantuan size or smaller)}
    \spelleffect The target can walk on air as if it were solid ground. The magic only affects the target's legs, and does not grant the ability to climb vertically through the air.
    \par Should the spell end while the target is still aloft, the magic fails slowly. The target floats downward 60 feet per round for 1d6 rounds. If it reaches the ground in that amount of time, it lands safely. If not, it falls the rest of the distance, taking 1d6 damage per 10 feet of fall.
\end{spelltarget}

\spellsection{Align Weapon}{2}
\spelldesc{You enhance a weapon while bringing it closer to your ideals.}
\spellschool{Evoc/Trans (Augment, Channeling) [see text]}
\spelllists{Chaos, Evil, Good, Law}
\spellrng{\rngclose}
\begin{spelltarget}{One weapon or fifty projectiles (in a single group)}
    \spelleffect The weapon is enhanced, as \spell{magic weapon}.
    \spellattack{Magic vs. Will (object)}
    \spellsuccess The item becomes good, evil, lawful, or chaotic, as you choose, allowing it to overcome damage reduction of the appropriate type. This overrides any existing alignments.
\end{spelltarget}
\spellnotes When you make a weapon good, evil, lawful, or chaotic, \spell{align weapon} is a good, evil, lawful, or chaotic spell, respectively.

\spellsection{Alter Weapon}{1}
\spelldesc{You transform a weapon into a slightly different form.}
\spellinfo{Trans (Alteration)}{Arcane}
\spellrng{Touch}
\spelldur{\durmed}
\spellsr{Yes (Will)}
\begin{spelltarget}{One weapon}[Magic vs. Will (object)]
    \spellsuccess The weapon transforms into a different weapon from the same weapon group. In addition, you can decrease (but not increase) its size by one size category.
\end{spelltarget}

\spellsectioncomma{Alter Weapon}{Greater}{4}
\spelldesc{You transform a weapon into a completely different shape.}
\spellinfo{Trans (Alteration)}{Arcane}
\spellrng{Touch}
\spelldur{\durmed}
\spellsr{Yes (Will)}
\begin{spelltarget}{One weapon}[Magic vs. Will (object)]
    \spellsuccess The weapon transforms into any other manufactured weapon (but not an improvised weapon). In addition, you can increase or decrease its size by one size category.
    \spelleffect As a standard action that requires concentration, you can touch the weapon to change its shape again.
\end{spelltarget}

\spellsection{Animal Growth}{7}
\spelldesc{You cause a number of animals grow to twice their normal size and eight times their normal weight.}
\spellschool{Trans (Polymorph) [Size-Affecting]}
\spelllists{Nature, Wild}
\spelltime{Full-round action}
\spelltwocol{\spelllimit{\areamed radius}}{\spellrng{\rngmed}}
\begin{spelltarget}{Five animals in the area}[Magic vs. Fortitude]
    \spellsuccess The target grows larger, as \spell{enlarge person}, except that it affects animals.
\end{spelltarget}

\begin{comment}
\spellsection{Animate Objects}{5}
\spelldesc{You imbue inanimate objects with mobility and a semblance of life.}
\spellinfo{Trans (Animation)}{Chaos, Trans}
\spelltwocol{\spelllimit{\areamed radius}}{\spellrng{\rngmed}}
\spelltgts{One Small object/level in the area; see text}
\spelldur{\durshort}
\spelleffect Each animated object immediately attacks whomever or whatever you initially designate. Your control of the objects is limited to simple commands (``Attack,'' ``Defend,'' ``Stop,'' and so forth).
\par An animated object can be of any nonmagical material. You may animate one Small or smaller object or an equivalent number of larger objects per caster level. A Medium object counts as two Small or smaller objects, a Large object as four, a Huge object as eight, a Gargantuan object as sixteen, and a Colossal object as thirty-two. You can give the objects new commands as a move action, as normal for directing an active spell.
\spellnotes This spell cannot animate objects carried or worn by a creature. This spell can be made permanent with a \spell{permanency} ritual.

\spellsection{Animate Plants}{5}
\spelldesc{You imbue inanimate plants with mobility and a semblance of life.}
\spellinfo{Trans (Animation)}{Nature, Plant}
\begin{spelltargets}{One Small plant/level in the area; see text}
    \spelleffect This spell functions like \spell{animate objects}, except that you animate plants instead of inanimate objects.
\end{spelltargets}
\spellnotes \spell{Animate plants} cannot affect plant creatures, nor does it affect nonliving vegetable material.
\end{comment}

\spellsection{Antilife Shell}{7}
\spelldesc{You create an immobile, spherical energy field that hedges out living creatures.}
\spellinfo{Abjur (Interdiction) [Barrier]}{Divine, Nature, Wild}
\spellzone{\areasmall radius centered on you} 
\spelldur{\durlong \dismissable}
\spellline
\spelleffect Living creatures cannot enter the spell's area. Nonliving creatures, such as constructs and undead, suffer no ill effect.
\spellnotes Barrier spells may be used only defensively, not aggressively. Creatures in the area at the time that the spell is cast are unaffected by the spell.
\spellsr{Yes (Will)}

\spellsection{Antimagic Field}{7}
\spelldesc{You create a mobile, spherical energy field that suppresses magic.}
\spellinfo{Abjur (Negation) [Magic]}{Abjur, Divine, Magic}
\spellemanation{\areasmall radius centered on you}
\spelldur{\durlong \dismissable}
\spellline
\spelleffect All spells, spell-like abilities, supernatural abilities, and magic items fail to function in the area of this spell. They cannot be activated from within the field, and any existing effects brought into or cast into the area are suppressed. Time spent within an \spell{antimagic field} counts against a suppressed spell's duration.
\par Summoned creatures of any type and incorporeal undead disappear if they enter an \spell{antimagic field}. They reappear in the same spot once the field goes away. (The effects of instantaneous conjurations, such as \spell{create water}, are not affected by an \spell{antimagic field} because the conjuration itself is no longer in effect, only its result.)
\par Creatures within an \spell{antimagic field} cannot dismiss spells. However, you can dismiss your own antimagic field.
\spellnotes A normal creature can enter the area, as can normal missiles. Furthermore, while a magic sword does not function magically in the area, it is still a sword. The spell has no effect on golems and other constructs that are imbued with magic during their creation process and are thereafter self-supporting (unless they have been summoned, in which case they are treated like any other summoned creatures). Elementals, corporeal undead, and outsiders are likewise unaffected unless summoned.
\par \spell{Dispel magic} does not remove the field. Two or more \spell{antimagic fields} sharing any of the same space have no effect on each other. Certain spells, such as \spell{wall of force}, \spell{prismatic sphere}, and \spell{prismatic wall}, remain unaffected by \spell{antimagic field} (see the individual spell descriptions).
\par Any part of a creature that lies outside the field is unaffected by the field.
\par Artifacts and deities are unaffected by mortal magic such as this. 

\spellsection{Aqueous Blade}{2}
\spelldesc{You transform the active part of your ally's weapon into water, weakening its blows but allowing it penetrate your foe's defenses more easily.}
\spellinfo{Trans (Alteration) [Water]}{Nature, Water}
\spellrng{\rngclose}
\spelldur{\durshort \dismissable}
\spellsr{Yes (Will)}
\begin{spelltarget}{One weapon}[Magic vs. Will]
    \spellsuccess Attacks with the affected weapon are made against Reflex defense instead of Armor defense. However, damage with the weapon is halved, including any bonuses to damage.
\end{spelltarget}

\spellsection{Assimilate}{9}
\spelldesc{Your pointing finger turns black as obsidian. You touch a creature and it dissolves into dust as you assimilate its form into your own body.}
\spellinfo{Necro/Trans (Augment, Life)}{Arcane, Evil}
\spellrng{\rngtouch}
\spelldur{Instantaneous and one hour; see text}
\spellsr{Yes (Fortitude)}
\begin{spelltarget}{One living creature}[Magic vs. Fortitude]
    \spellsuccess 18d8 life damage \add d6 per four caster levels above 18th. 
    \spellfailure As above, but half damage.
    \spelleffect If the target has no hit points remaining after taking damage from this spell, it is entirely assimilated into your form, leaving behind only a trace of fine dust. An assimilated creature's equipment is unaffected.
    \par If the creature has at least 1 hit point following your use of this power, you gain temporary hit points equal to half the damage you dealt for 1 hour.
    \par If the creature is completely assimilated, you gain a number of temporary hit points equal to the damage you dealt and a \plus4 enhancement bonus to each of your attributes for 1 hour. In addition, you gain the appearance of the creature for 1 hour, granting you a \plus10 enhancement bonus on Disguise checks made to appear as that creature during that time.
\end{spelltarget}
\spellnotes If you take life damage, you lose all temporary hit points provided by this spell before applying the damage.

\spellsection{Aversion}{3}
\spelldesc{You make the subject want to avoid something.}
\spellinfo{Ench (Emotion) [Mind-Affecting]}{Ench}
\spellrng{\rngmed}
\spelldur{\durshort}
\spellsr{Yes (Will)}
\begin{spelltarget}{One creature}[Magic vs. Will]
    \spellsuccess The target feels an aversion to a particular person or object. If the object of the implanted aversion is an individual or a physical object, she will prefer not to approach within 30 feet of it. If it is a word, she will try not to utter it; if it is an action, she will not willingly attempt to perform it; and if it is an event, she will not willingly attend it. The target will take reasonable steps to avoid the object of its aversion, but will not put herself in jeopardy by doing so.
    \par If the target is unable to avoid the object of her aversion, she takes a \minus4 penalty to attacks, defenses, and checks for 1 round.
\end{spelltarget}

\pdfbookmark[2]{B}{SpellDescriptionsB}
\begin{comment}
\subsubsection{B}
\end{comment}

\spellsection{Bane}{1}
\spelldesc{You fill your enemies with dismay, impairing their ability to fight.}
\spellinfo{Ench (Emotion) [Mind-Affecting, Morale]}{Divine, Evil, War}
\spellburst{\areamed radius centered on you}
\spelldur{5 rounds}
\spellsr{Yes (Will)}
\begin{spelltargets}{All enemies in the area}[Magic vs. Will]
    \spellsuccess \minus2 penalty to physical attacks.
\end{spelltargets}

\spellsection{Banishment}{6}
\spelldesc{You force extraplanar creatures back to their home plane.}
\spellinfo{Abjur/Conj (Interdiction, Translocation) [Planar]}{Arcane, Divine}
\spellcmp{Verbal, Somatic, and Focus}
\spellrng{\rngmed}
\spelldur{Concentration}
\begin{spelltargets}{One extraplanar creature}[Magic vs. Will]
    \spellsuccess The target is banished, as \spell{dismissal}.
\end{spelltargets}
\spelleffect As long as you concentrate on this spell, you can attack a new target each round. An individual creature can only be targeted once per casting of this spell.
\spellnotes You can improve the spell's chance of success by presenting at least one object or substance that the target hates, fears, or otherwise opposes. For each such object or substance, you gain a \plus2 bonus on your caster level with the spell. For example, if this spell were cast on a demon that hated light and was \vulnerable to holy water and cold iron weapons, you might use iron, holy water, and a torch in the spell. The three items would give you a \plus6 bonus on your caster level. 
\par Certain rare items might work twice as well as a normal item for the purpose of the bonuses (each providing a \plus4 bonus to your caster level).
\spellfocus{Any item that is distasteful to the target (optional, see above)}

\spellsection{Barkskin}{2}
\spelldesc{You toughen a creature's skin, giving it the appearance of tree bark.}
\spellinfo{Trans (Augment)}{Nature, Wild}
\spellrng{Touch}
\spelldur{\durshort}
\spellsr{Yes (Fortitude)}
\begin{spelltarget}{One living creature}
    \spelleffect The target gains a \plus2 enhancement bonus to its armor modifier. \spellbonusscalingdescription In addition, the target gains physical damage reduction 2/adamantine or fire. This damage reduction increases by 1 per two caster levels above 4th.
\end{spelltarget}
\spellnotes This spell's damage reduction allows the target to ignore the first 2 physical damage it takes each round. If it is hit by a adamantine weapon or an attack that deals fire damage, it cannot use its damage reduction for 1 round.

\spellsection{Bestow Curse}{5}
\spelldesc{You place a curse on your foe, crippling its ability to act.}
\spellrng{\rngclose}
\spellinfo{Necro (Life) [Curse]}{Death, Divine, Evil, Necro}
\spelldur{Permanent}
\spellsr{Yes (Will)}
\begin{spelltarget}{One creature}[Magic vs. Will]
    \spellsuccess The target suffers one of the following three effects, chosen by you:
    \begin{itemize}
        \item \minus6 penalty to an attribute.
        \item \minus4 penalty on attacks, defenses, and checks.
        \item Each turn, the target has a 25\% chance to take no action; otherwise, it acts normally.
    \end{itemize}
    \par You may also invent your own curse, but it should be no more powerful than those described above.
\end{spelltarget}
\spellnotes Curses cannot be dispelled.

\spellsection{Black Tentacles}{7}
\spelldesc{You conjure a field of rubbery black tentacles, each 5 feet long. These waving members seem to spring forth from the earth, floor, or whatever surface is underfoot -- including water. They grasp and entwine around creatures that enter the area, holding them fast and crushing them with great strength.}
\spellinfo{Conj/Trans (Animation, Creation)}{Arcane}
\spelltwocol{\spellzone{\areasmall radius}}{\spellrng{\rngmed}}
\spelldur{\durshort \dismissable}
\spellsr{Yes (Fortitude)}
\spellline
\spelleffect The area is considered difficult terrain.
\begin{spelltrigger}{End of every movement phase}
    \begin{spelltargets}*{All creatures in the area within 5 feet of the ground}l[Caster level \add casting attribute vs. Maneuver defense (grapple)]
        \spellsuccess The target is grappled and takes 1d8\plus4 bludgeoning damage. It remains grappled until it escape the tentacle. The tentacle's Maneuver defense is equal to 10 \add your caster level \add your casting attribute.
    \end{spelltargets}
\end{spelltrigger}
\spellnotes The tentacles are immune to all forms of attack.

\spellsection{Blade Barrier}{6}
\spelldesc{You create an immobile, vertical curtain of whirling blades shaped of pure force.}
\spellinfo{Evoc (Energy) [Force, Wall]}{Divine, War}
\spellrng{\rngmed}
\spelltwocol{\spellzone{100 ft. wall, 20 ft. high \shapeable}}{}
\spelldur{\durshort \dismissable}
\spellsr{Yes (Reflex)}
\spellline
\spelleffect You create a wall of blades made of force energy. The wall provides active cover (20\% miss chance) against attacks made through it. Attacks that miss in this way harmlessly strike the wall. The wall is considered difficult terrain.
\begin{spelltrigger}{Creature passes through wall}
    \begin{spelltargets}*{Creature in wall}[Magic vs. Reflex]
        \spellsuccess 6d6 force damage \add d6 per four caster levels above 12th.
        \spellfailure As above, but half damage.
    \end{spelltargets}
\end{spelltrigger}

\spellsection{Blasphemy}{7}
\spelldesc{You speak an unholy utterance of great power, afflicting all those nearby who do not share your allegiance to evil.}
\spellinfo{Evoc (Channeling) [Evil]}{Divine, Evil}
\spellcmp{Verbal only}
\spellburst{\arealarge radius centered on you}
\spellsr{Yes (Will)}
\begin{spelltargets}{All nonevil creatures in the area}
    \spelleffect If the target's level does not exceed your caster level, it is \sickened for 5 rounds.

    If it is also bloodied, it also suffers one or more of the following ill effects, depending on its level.
    \begin{itemize}
        \item Up to caster level \minus5: The creature is also \nauseated for 1 round.
        \item Up to caster level \minus10: The creature is also \paralyzed for 5 rounds.
        \item Up to caster level \minus15: The creature immediately dies. A nonliving creature is destroyed.
    \end{itemize}
\end{spelltargets}

\spellsection{Bless}{2}
\spelldesc{You fill your allies with confidence, improving their prowess in combat.}
\spellinfo{Ench (Emotion) [Mind-Affecting, Morale]}{Divine, Good, War}
\spellburst{\areamed radius}
\spelldur{5 rounds}
\spellsr{Yes (Will)}
\begin{spelltargets}{All allies in the area}
    \spelleffect The target gains a \plus2 enhancement bonus to physical attacks. \spellbonusscalingdescription
\end{spelltargets}

\spellsection{Blindness/Deafness}{2}
\spelldesc{You remove one of your foe's senses.}
\spellinfo{Necro (Flesh)}{Arcane, Divine, Death}
\spellrng{\rngclose}
\spelldur{\durshort \dismissable}
\spellsr{Yes (Fortitude)}
\begin{spelltarget}{One creature}[Magic vs. Fortitude]
    \spellsuccess The target is \sickened.

    If the target becomes \bloodied, it is also \blinded for 1 round or \deafened for the duration of the spell, as you choose.
\end{spelltarget}
\spellnotes The choice of bloodied conditions is made at the time the spell is cast.

\spellsection{Blink}{4}
\spelldesc{You rapidly blink in and out of reality, confounding your foes and protecting you from their attacks.}
\spellinfo{Conj (Translocation) [Planar]}{Arcane}
\spellrng{\rngpers}
\spelldur{\durshort \dismissable}
\begin{spelltarget}{You}
    \spelleffect You ``blink" back and forth between the Material Plane and the Ethereal Plane. This has several effects, as follows.
    \begin{itemize}
        \item All attacks made against you and spells targeted on you have a 50\% chance to fail. This failure chance is reduced to 20\% if the attack can strike ethereal targets or if the attacker can see ethereal targets. If both are true, the attack suffers no chance of failure. Force effects can strike ethereal targets.
        \item You take half damage from area attacks (but full damage from those that extend onto the Ethereal Plane).
        \item You take half damage from falling, since you fall only while you are material.
        \item All of your attacks and spells have a 20\% chance to happen while you are in the Ethereal Plane, which usually means they have no effect.
        \item You can move at only three-quarters speed (because movement on the Ethereal Plane is at half speed, and you spend about half your time there and half your time material.)
        \item You can step through (but not see through) solid objects. For each 5 feet of solid material you walk through, there is a 50\% chance that you become material. If this occurs, you are shunted off to the nearest open space and take 1d6 damage per 5 feet so traveled. 
        \item You can see and interact with ethereal creatures in roughly the same way you interact with material ones.
    \end{itemize}
\end{spelltarget}

\spellsection{Blur}{2}
\spelldesc{You distort an ally's outline so it appears blurred, shifting, and wavering.}
\spellinfo{Illus (Glamer)}{Arcane}
\spellrng{\rngclose}
\spelldur{\durshort \dismissable}
\spellsr{Yes (Will)}
\begin{spelltarget}{One creature}
    \spelleffect Other creatures take a \minus2 penalty on sight-based checks and physical attacks against the target, such as on Perception and Sense Motive checks.
\end{spelltarget}
\spellnotes A \spell{see invisibility} spell does not counteract the blurring effect, but a \spell{true seeing} spell does.

\spellsection{Burning Hands}{1}
\spelldesc{You expel a cone of searing flame shoots from your fingertips, searing creatures in front of you.}
\spellinfo{Evoc (Energy) [Fire]}{Arcane, Destruction, Nature, Fire}
\spellburst{\areamed cone}
\spellsr{Yes (Reflex)}
\begin{spelltargets}{Everything in the area}[Magic vs. Reflex]
    \spellsuccess 1d6 fire damage \add 1d6 per four caster levels above 2nd.
    \spellfailure As above, but half damage.
\end{spelltargets}

\pdfbookmark[2]{C}{SpellDescriptionsC}
\begin{comment}
\subsubsection{C}
\end{comment}

\spellsection{Call Lightning}{3}
\spelldesc{You repeatedly call bolts of lightning that flash down from thin air to smite your foes.}
\spellinfo{Evoc (Energy) [Destructive, Electricity]}{Air, Nature}
\spelltwocol{\spellburst{\arealarge vertical line}}{\spellrng{\rngmed}}
\spelldur{\durmed or until discharged \dismissable}
\spellsr{Yes (Reflex)}
\begin{spelltargets}{Everything in the area}[Magic vs. Reflex]
    \spellsuccess 3d8 electricity damage \add d8 per four caster levels above 6th. If you are outdoors and in a stormy area -- a rain shower, clouds and wind, hot and cloudy conditions, or even a tornado -- this deals 3d10 electricity damage \add d10 per four caster levels above 6th instead.
    \spellfailure As above, but half damage.
\end{spelltargets}
\spelleffect Until the spell is discharged, you can call a new bolt of lightning anywhere within range as a standard action that requires concentration. You may call a total number of bolts equal to your caster level before the spell is discharged.
\spellnotes This spell functions indoors or underground, but not underwater. \destructivespellnotes

\spellsectioncomma{Call Lightning}{Greater}{5}
\spelldesc{You repeatedly call intense bolts of lightning that flash down from thin air to smite your foes.}
\spellinfo{Evoc (Energy) [Destructive, Electricity]}{Air, Nature}
\spelltwocol{\spellburst{\arealarge vertical line}}{\spellrng{\rngmed}}
\spelldur{\durmed or until discharged \dismissable}
\spellsr{Yes (Reflex)}
\begin{spelltargets}{Everything in the area}l[Magic vs. Reflex and Fortitude]
    \spellsuccess If you beat the target's Reflex defense, it takes 5d8 electricity damage \add d8 per four caster levels above 10th. If you are outdoors in a stormy area, it takes 5d10 electricity damage \add d10 per four caster levels above 10th instead.

    If the target is \bloodied, and you also beat its Fortitude defense, it is \staggered for 5 rounds.
    \spellfailure As above, but half damage and the target is not staggered.
\end{spelltargets}
\spelleffect You may call additional bolts, as \spell{call lightning}.
\spellnotes As \spell{call lightning}.

\spellsection{Calm Emotions}{2}
\spelldesc{You calm a group of creatures, preventing the situation from getting out of hand.}
\spellinfo{Ench (Emotion) [Mind-Affecting]}{Arcane}
\spelltwocol{\spellburst{\areamed radius}}{\spellrng{\rngmed}}
\spelldur{Concentration}
\begin{spelltargets}{All creatures in the area}[Magic vs. Will]
    \spellsuccess The target has its emotions calmed. It cannot take violent actions (although it can defend itself) or do anything destructive.
\end{spelltargets}
\spellnotes Any aggressive action against or damage dealt to a calmed creature immediately breaks the spell on all calmed creatures.

This spell automatically suppresses (but does not dispel) any effects of spells or abilities that affect or require emotions, including all other enchantment (emotion) spells.
\spellsr{Yes (Will)}

\spellsection{Chain Lightning}{5}
\spelldesc{You create a stroke of lightning which strikes a single foe before arcing to hit a number of other foes of your choice.}
\spellinfo{Evoc (Energy) [Electricity]}{Arcane, Destruction, Nature}
\spellrng{\rngmed}
\spelllimit{\areamed radius centered on the primary target}
\spellsr{Yes (Fortitude)}
\begin{spelltargets}{One primary target, plus up to five secondary targets in the area}l[Magic vs. Reflex]
    \spellsuccess 5d10 electricity damage \add d10 per four caster levels above 10th. Secondary targets take half damage.
    \spellfailure As above, but half damage.
\end{spelltargets}

\spellsection{Changestaff}{7}
\spelldesc{You plant your staff in the ground and transform it into a massive tree-like creature which obeys your every command.}
\spellinfo{Trans (Alteration, Animation)}{Nature, Wild}
\spellcmp{Verbal, Somatic, and Focus}
\spelltime{Full-round action}
\spelltwocol{\spelltgt{Your touched staff}}{\spellrng{\rngtouch}}
\spelldur{\durmed \dismissable}
\spellline
\spelleffect Your staff turns into a creature that looks and fights just like a treant. The staff-treant defends you and obeys any spoken commands. However, it is by no means a true treant; it cannot converse with actual treants or control trees.

If the staff-treant is reduced to 0 or fewer hit points, it crumbles to powder and the staff is destroyed. Otherwise, the staff returns to its normal form when the spell duration expires (or when the spell is dismissed), and it can be used as the focus for another casting of the spell. The staff-treant is always at full strength when created, despite any wounds it may have incurred the last time it appeared.
\spellline
\spellfocus{The quarterstaff, which must be specially prepared. The staff must be a sound limb cut from an ash, oak, or yew, then cured, shaped, carved, and polished (a process requiring twenty-eight days). You cannot adventure or engage in other strenuous activity during the shaping and carving of the staff.}

\spellsection{Chaos Hammer}{4}
\spelldesc{You unleash a multicolored explosion of leaping, ricocheting energy to smite your foes.}
\spellinfo{Evoc (Channeling) [Chaotic]}{Chaos}
\spellrng{\rngmed}
\spelldur{Instantaneous/5 rounds}
\spellsr{Yes (Will)}
\begin{spelltarget}{One nonchaotic creature}[Magic vs. Will]
    \spellsuccess 8d6 divine damage \add d6 per two caster levels above 8th, and the target is \bewildered for 5 rounds.
    \spellfailure As above, but half damage.
\end{spelltarget}

\spellsection{Charm Monster}{5}
\spelldesc{You manipulate a creature's mind so it thinks of you as a trusted friend and ally.}
\spellinfo{Ench (Emotion) [Charm, Mind-Affecting]}{Ench}
\spellcmp{Somatic only}
\spellrng{\rngmed}
\spelldur{\durlong}
\spellsr{Yes (Will)}
\begin{spelltarget}{One creature}[Magic vs. Will]
    \spellsuccess The target is charmed, as \spell{charm person}, except that the effect is not restricted by creature type.
\end{spelltarget}

\spellsection{Charm Person}{2}
\spelldesc{You manipulate a person's mind so he thinks of you as a trusted friend and ally.}
\spellinfo{Ench (Emotion) [Charm, Mind-Affecting]}{Ench}
\spellcmp{Somatic only}
\spellrng{\rngmed}
\spelldur{\durlong}
\spellsr{Yes (Will)}
\begin{spelltarget}{One humanoid creature}[Magic vs. Will]
    \spellsuccess The target regards you as its trusted friend and ally. If it is currently faced with any obvious threat from you or your allies, such as someone drawing a weapon, casting a spell, or aiming a ranged weapon at the creature, you take a \minus5 penalty on the magic attack.
    \par The spell does not enable you to control the subject as if it were an automaton, but it perceives your words and actions in the most favorable way. You can try to give the subject orders, but you must succeed at a Persuasion check to convince it to do anything it wouldn't ordinarily do. (Retries are not allowed.) Treat the target as a friend (a \plus10 relationship modifier) for the purpose of the Persuasion check. An affected creature never obeys suicidal or obviously harmful orders, but it might be convinced that something very dangerous is worth doing.
\end{spelltarget}
\spellnotes Any act by you or your apparent allies that threatens the \spell{charmed} person breaks the spell. A creature that resists this spell is immune to all further attempts by the same spellcaster for 24 hours.

\spellsectioncomma{Charm Person}{Mass}{6}
\spelldesc{You manipulate the minds of many people so they think of you as a trusted friend and ally.}
\spellinfo{Ench (Emotion) [Charm, Mind-Affecting]}{Ench}
\spelltwocol{\spelllimit{\areamed radius}}{\spellrng{\rngmed}}
\begin{spelltargets}{Five humanoid creatures in the area}
    \spellsuccess This spell functions like \spell{charm person}, except that it affects multiple humanoid creatures.
\end{spelltargets}

\spellsection{Circle of Death}{6}
\spelldesc{You snuff out the life force of your weakened foes by flooding them with negative energy.}
\spellinfo{Necro (Vitalism) [Death, Negative]}{Death, Divine}
\spellcmp{Verbal, Somatic, and Focus}
\spelltwocol{\spelllimit{\areamed radius}}{\spellrng{\rngmed}}
\spellsr{Yes (Fortitude)}
\spellline
\spelleffect This spell affects all \bloodied living creatures in the area, starting with the creature with the lowest level, until it affects a total number of levels equal to twice your caster level. Among creatures with equal levels, those closest to the burst's point of origin are affected first. No creature whose level is greater than half your caster level can be affected, and levels that are not sufficient to affect a creature are wasted. Healthy creatures are not affected by this spell, and do not count against its level limit.
\begin{spelltargets}*{Several bloodied living creatures in the area}l[Magic vs. Fortitude]
    \spellsuccess If the target is \bloodied, it is reduced to 0 hit points and takes 9 critical damage, causing it to begin dying.
\end{spelltargets}
\spellline
\spellmat{The powder of a crushed black pearl with a minimum value of 100 gp.}

\spellsection{Clenched Fist}{9}
\spelldesc{You create a floating, disembodied hand made of magical force that strikes your foe.}
\spellinfo{Evoc (Control) [Force]}{Evoc, Strength}
\spellrng{\rngmed}
\spelldur{\durshort \dismissable}
\spellsr{Yes (Fortitude)}
\spellline
\spelleffect As \spell{interposing hand}, except that the hand attacks its target instead of protecting you from it.
\begin{spelltarget}*{One creature}l[Caster level \add casting attribute vs. Armor defense]
    \spellsuccess 2d10 force damage \add half casting attribute. If the target is \bloodied, you make an additional attack.
    \begin{spelltarget}*{Struck creature}[Magic vs. Fortitude]
        \spellsuccess If the target is \bloodied, it is \dazed for 1 round.
    \end{spelltarget}
\end{spelltarget}
\spellnotes The hand attacks during the action phase, regardless of when you direct it to attack a target.

\spellsection{Cloak of Chaos}{8}
\spelldesc{You shield your allies with an an powerful aura that resembles a random pattern of color -- an affront to your lawful foes.}
\spellinfo{Abjur (Shielding) [Chaotic]}{Chaos, Divine}
\spellcmp{Verbal, Somatic, and Focus}
\spelllimit{\areamed radius centered on you}
\spelldur{\durshort \dismissable}
\spellsr{Yes (Will)}
\begin{spelltarget}{Five creatures in the area}
    The target gains a \plus5 enhancement bonus to its defenses. In addition, it gains spell resistance against lawful spells and spells cast by lawful creatures.
    \par At the end of each round, all lawful creatures within \rngclose range of the subject that attacked the subject with their body or a melee weapon that round take 4d6 points of damage. A creature that attacks multiple creatures shielded by this spell can take this damage multiple times.
\end{spelltarget}
\spellline
\spellfocus{A tiny reliquary containing some sacred relic, such as a scrap of parchment from a chaotic text. The reliquary costs at least 250 gp.}

\spellsection{Cloudkill}{7}
\spelldesc{You conjure a yellowish green fog bank that obscures vision and slowly poisons creatures inside.}
\spellinfo{Conj (Creation) [Fog, Poison]}{Arcane}
\spelltwocol{\spellzone{\areamed radius cylinder}}{\spellrng{\rngmed}}
\spelldur{\durshort}
\spellline
\spelleffect Fog in the area, as \spell{fog cloud}, except that the fog is mobile and poisonous.

\par The fog moves away from you at 10 feet per round, rolling along the surface of the ground. Figure out the cloud's new area each round based on its new point of origin, which is 10 feet farther away from the point of origin where you cast the spell.
\begin{spelltrigger}{End of every round}
    \begin{spelltargets}*{Everything in the area}[Magic vs. Fortitude]
        \spellsuccess 1d4 Constitution damage.
    \end{spelltargets}
\end{spelltrigger}
\spellnotes Holding one's breath doesn't help against the poison, but creatures immune to poison are unaffected.
\par Because the vapors are heavier than air, they sink to the lowest level of the land, even pouring down den or sinkhole openings. This spell cannot penetrate liquids, nor can it be cast underwater.

\spellsection{Color Spray}{1}
\spelldesc{You project a vivid cone of clashing colors from your outstretched hand, striking creatures in front of you.}
\spellinfo{Illus (Figment) [Light, Sight-Dependent]}{Arcane}
\spellburst{\areamed cone}
\spelldur{1d4 rounds}
\spellsr{Yes (Will)}
\begin{spelltargets}[All creatures in the area]
    \spellsuccess The target is \dazzled and \bewildered.
\end{spelltargets}
\spellnotes Creatures who cannot see the light are not affected by this spell. Merely closing one's eyes is insufficient protection, however.

\spellsection{Combat Transformation}{7}
\spelldesc{You become a virtual fighting machine -- stronger, tougher, faster, and more skilled in combat. Your mind-set changes so that you relish combat instead of casting spells.}
\spellinfo{Trans (Augment)}{Arcane}
\spellcmp{Verbal, Somatic, and Material}
\spellrng{\rngpers}
\spelldur{\durshort \dismissable}
\begin{spelltarget}{You}
    \spelleffect You gain a \plus3 enhancement bonus to Strength, Dexterity, Constitution, and Fortitude defense. This bonus increases to \plus4 at 14th caster level and to \plus5 at 20th caster level. In addition, you gain proficiency with any weapons you hold (except exotic weapons).
\end{spelltarget}
\spellnotes If you cast a spell or use a spell activation or spell completion magic item, the spell immediately ends.
\spellmat{A potion of \spell{totemic power} (which costs 40 gp), which you drink (and whose effects are subsumed by the spell effects).}

\spellsection{Command}{1}
\spelldesc{You compel a foe to obey a single command of your choice.}
\spellinfo{Ench (Compulsion) [Language-Dependent, Mind-Affecting, Sound-Dependent]}{Arcane, Divine, Law}
\spellcmp{Verbal only}
\spellrng{\rngmed}
\spelldur{1 round}
\spellsr{Yes (Will)}
\begin{spelltarget}{One creature}[Magic vs. Will]
    \spellsuccess The subject is bewildered. If it is \bloodied, it must also perform one of the following actions, as you choose.
    \par \subspell{Approach} On its turn, the subject moves toward you as quickly and directly as possible. The creature may do nothing but move during its turn, and it provokes attacks of opportunity for this movement as normal.
    \par \subspell{Drop} As soon as possible, the subject drops whatever it is holding. It may act normally on its turn, except that it can't pick up any dropped items.
    \par \subspell{Fall} As soon as possible, the subject falls to the ground. It may act normally on its turn, except that it can't get up from its prone position.
    \par \subspell{Flee} On its turn, the subject moves away from you as quickly as possible. It may do nothing but move during its turn, and it provokes attacks of opportunity for this movement as normal.
    \par \subspell{Halt} On its turn, the subject can take no actions, but it can defend itself normally.
    \par \subspell{Laugh} On its turn, the subject takes a standard action to do nothing but laugh uproariously, provoking attacks of opportunity. After that, it can act normally.
\end{spelltarget}
\spellnotes If the subject can't understand or carry out your command, the spell automatically fails.

\spellsectioncomma{Command}{Mass}{5}
\spelldesc{You compel many foes to obey your command.}
\spellinfo{Ench (Compulsion) [Language-Dependent, Mind-Affecting, Sound-Dependent]}{Divine, Law}
\spellcmp{Verbal only}
\spellrng{\rngmed}
\spelllimit{\areamed radius}
\spelldur{1 round}
\spellsr{Yes (Will)}
\begin{spelltarget}{Five creatures in the area}
    \spellsuccess The target obeys a command, as \spell{command}.
\end{spelltarget}

\spellsection{Cone of Cold}{5}
\spelldesc{You create an area of extreme cold that drains heat from creatures in the area, diminishing their ability to move and fight.}
\spellinfo{Evoc (Energy) [Cold, Destructive]}{Arcane, Nature}
\spellburst{\areamed cone}
\spellsr{Yes (Reflex)}
\begin{spelltarget}{Everything in the area}[Magic vs. Reflex]
    \spellsuccess 5d6 cold damage \add d6 per four caster levels above 10th. In addition, the target is \fatigued for 5 rounds.
    \spellfailure As above, but half damage and the target is not fatigued.
\end{spelltarget}
\spellnotes \destructivespellnotes

\spellsectioncomma{Cone of Cold}{Greater}{8}
\spelldesc{You create a massive area of extreme cold that drains heat from creatures in the area, diminishing their ability to move and fight.}
\spellinfo{Evoc (Energy) [Cold, Destructive]}{Arcane, Nature}
\spellburst{\arealarge cone}
\begin{spelltarget}{Everything in the area}[Magic vs. Reflex]
    \spellsuccess 8d6 cold damage \add d6 per four caster levels above 16th. In addition, the target is \fatigued for 5 rounds.
    \spellfailure As above, but half damage and the target is not fatigued.
\end{spelltarget}
\spellnotes As \spell{cone of cold}.

\spellsection{Confusion}{3}
\spelldesc{You compel a creature to act randomly, sowing confusion in your foes' ranks.}
\spellinfo{Ench (Compulsion) [Mind-Affecting]}{Arcane, Chaos, Trickery}
\spellrng{\rngmed}
\spelldur{\durshort}
\spellsr{Yes (Will)}
\begin{spelltarget}{One creature}[Magic vs. Will]
    \spellsuccess The target is \bewildered.

    If it is \bloodied, it is also \confused.
\end{spelltarget}

\spellsectioncomma{Confusion}{Mass}{7}
\spelldesc{You compel a group of creatures to act randomly, sowing confusion in your foes' ranks.}
\spellinfo{Ench (Compulsion) [Mind-Affecting]}{Arcane, Trickery}
\spellrng{\rngmed}
\spelllimit{\areamed radius}
\begin{spelltarget}{Five creatures in the area}[Magic vs. Will]
    \spellsuccess As \spell{confusion}.
\end{spelltarget}

\spellsection{Control Water}{2}
\spelldesc{You manipulate elemental forces to control water around you.}
\spellinfo{Evoc (Control) [Water]}{Nature, Water}
\spelltwocol{\spellzone{One 5 ft. cube/caster level}}{\spellrng{\rngfar}}
\spelldur{\durmed \dismissable}
\spellline
\spelleffect Depending on the version you choose, this spell raises or lowers water.
\par \subspell{Lower Water} This causes water or similar liquid to reduce its depth by as much as 2 feet per caster level (to a minimum depth of 1 inch). The water is lowered within a squarish depression whose sides are up to caster level \mtimes 10 feet long. In extremely large and deep bodies of water, such as a deep ocean, the spell creates a whirlpool that sweeps ships and similar craft downward, putting them at risk and rendering them unable to leave by normal movement for the duration of the spell.
\par \subspell{Raise Water} This causes water or similar liquid to rise in height, just as the lower water version causes it to lower. Boats raised in this way slide down the sides of the hump that the spell creates. If the area affected by the spell includes riverbanks, a beach, or other land nearby, the water can spill over onto dry land.
\spellnotes With either version, you may reduce one horizontal dimension by half and double the other horizontal dimension.

\spellsection{Create Sound}{1}
\spelldesc{You create false sounds from nowhere.}
\spellinfo{Illus (Figment) [Unreal]}{Illus}
\spelltwocol{\spellzone{\areamed radius}}{\spellrng{\rngmed}}
\spelldur{\durshort \dismissable}
\spellline
\spelleffect This spell creates a volume of sound within the area, as determined by you. As a standard action, you can concentrate to alter the sound within the area.
\par The volume of sound created depends on your caster level. You can produce as much noise as two normal humans per caster level. Thus, talking, singing, shouting, walking, marching, or running sounds can be created. The noise can be virtually any type of sound within the volume limit, including speech. A horde of rats running and squeaking is about the same volume as eight humans running and shouting. A roaring lion is equal to the noise from sixteen humans, while a roaring dire tiger is equal to the noise from twenty humans.
\spellnotes Creatures can identify the illusion, as \spell{silent image}. This spell can be made permanent with a \spell{permanency} ritual.

\spellsection{Creeping Doom}{7}
\spelldesc{You summon uncountable hordes of centipedes to overwhelm your foes.}
\spellinfo{Conj (Summoning)}{Nature}
\spelltime{Full-round action}
\spellrng{\rngmed}
\spelldur{\durmed}
\spellline
\spelleffect This spell creates one centipede swarm per two caster levels. They must all be adjacent at least one other swarm. You may summon the centipede swarms so that they share the area of other creatures. The swarms remain stationary, attacking any creatures in their area, unless you command the creeping doom to move (a standard action). As a standard action, you can command any number of the swarms to move toward any prey within range of you. Any swarm out of range of you remains stationary, attacking any creatures in its area.

\spellsection{Cripple}{6}
\spelldesc{You render your foe's limbs useless.}
\spellinfo{Necro (Flesh)}{Arcane}
\spellrng{\rngmed}
\spelldur{\durshort}
\spellsr{Yes (Fortitude)}
\begin{spelltarget}{One creature}[Magic vs. Fortitude]
    \spellsuccess The target is \staggered. If it is \bloodied, it cannot move its limbs, including any wings. Generally, that means it is paralyzed, except that it can move its head and mouth.
\end{spelltarget}

\spellsection{Crushing Despair}{3}
\spelldesc{You fill a number of creatures with sadness and gloom.}
\spellinfo{Ench (Emotion) [Mind-Affecting, Morale]}{Arcane}
\spellburst{\areamed cone}
\spelldur{\durmed}
\spellsr{Yes (Will)}
\begin{spelltargets}{All creatures in the area}
    The target is \vulnerable.
\end{spelltargets}

\spellsection{Crushing Hand}{8}
\spelldesc{You create a floating, disembodied hand made of magical force that crushes your foe in its grasp.}
\spellinfo{Evoc (Control) [Force]}{Evoc}
\spellrng{\rngmed}
\spelldur{\durshort \dismissable}
\spellsr{Yes (Fortitude)}
\spellline
\spelleffect This spell creates a hand, as \spell{interposing hand}, except that the hand grapples its target instead of protecting you from it.
\begin{spelltarget}*{One creature}l[Caster level \add casting attribute vs. Maneuver defense]
    \spellsuccess The target is grappled. It takes 2d6 bludgeoning damage \add half your casting attribute.
\end{spelltarget}

\spellsection{Cure Critical Wounds}{4}
\spelldesc{You lay your hand on a creature and channel positive energy into it, healing even the most grievous injuries.}
\spellinfo{Necro (Vitalism) [Positive]}{Divine, Life, Nature}
\spellrng{\rngclose}
\spellsr{Yes (Fortitude)}
\begin{spelltarget}{One creature}[Magic vs. Fortitude]
    \spelleffect As \spell{cure light wounds}, except that it heals 8d6 damage \add d6 per two caster levels above 8th. For every 5 points of healing granted by this spell, it can instead cure 1 point of critical damage.
    \spellsuccess If the target is undead, it instead takes that much positive damage.
    \spellfailure As above, but half damage.
\end{spelltarget}

\spellsectioncomma{Cure Critical Wounds}{Mass}{8}
\spelldesc{You stretch out your hand and channel positive energy into all of your allies, healing even their most grievous injuries.}
\spellinfo{Necro (Vitalism) [Positive]}{Divine, Life, Nature}
\spellrng{\rngclose}
\spellsr{Yes (Fortitude)}
\begin{spelltargets}{Five creatures in the area}[Magic vs. Fortitude]
    \spelleffect As \spell{cure critical wounds}, except that it heals 8d6 damage \add d6 per four caster levels above 16th.
    \spellsuccess If the target is undead, it instead takes that much positive damage.
    \spellfailure As above, but half damage.
\end{spelltargets}

\spellsection{Cure Light Wounds}{1}
\spelldesc{You lay your hand on a creature and channel positive energy into it, healing some of its wounds.}
\spellinfo{Necro (Vitalism) [Positive]}{Divine, Nature}
\spellrng{\rngclose}
\spellsr{Yes (Fortitude)}
\begin{spelltarget}{One creature}[Magic vs. Fortitude]
    \spelleffect If the target is living, it is healed for 2d6 damage \add d6 per two caster levels above 2nd.
    \spellsuccess If the target is undead, it instead takes that much positive damage.
    \spellfailure As above, but half damage.
\end{spelltarget}

\spellsectioncomma{Cure Light Wounds}{Mass}{5}
\spelldesc{You stretch out your hand and channel positive energy into all of your allies, healing some of their wounds.}
\spellinfo{Necro (Vitalism) [Positive]}{Divine, Life, Nature}
\spelltwocol{\spelllimit{\areamed radius}}{\spellrng{\rngmed}}
\spellrng{\rngclose}
\spellsr{Yes (Fortitude)}
\begin{spelltargets}{Five creatures in the area}[Magic vs. Fortitude]
    \spelleffect As \spell{cure light wounds}, except that it heals 5d6 damage \add d6 per four caster levels above 10th.
    \spellsuccess As \spell{cure light wounds}.
    \spellfailure As above, but half damage.
\end{spelltargets}

\spellsection{Cure Moderate Wounds}{2}
\spelldesc{You lay your hand on a creature and channel positive energy into it, healing its wounds.}
\spellinfo{Necro (Life) [Positive]}{Divine, Life, Nature}
\spellrng{\rngclose}
\spellsr{Yes (Fortitude)}
\begin{spelltarget}{One creature}[Magic vs. Fortitude]
    \spelleffect As \spell{cure light wounds}, except that it heals 4d6 damage \add d6 per two caster levels above 4th. For every 15 points of healing granted by this spell, it can instead cure 1 point of critical damage.
    \spellsuccess If the target is undead, it instead takes that much positive damage.
    \spellfailure As above, but half damage.
\end{spelltarget}

\spellsectioncomma{Cure Moderate Wounds}{Mass}{6}
\spelldesc{You stretch out your hand and channel positive energy into all of your allies, healing their wounds.}
\spellinfo{Necro (Vitalism) [Positive]}{Divine, Life, Nature}
\begin{spelltargets}{Five creatures in the area}[Magic vs. Fortitude]
    \spelleffect As \spell{cure moderate wounds}, except that it heals 6d6 damage \add d6 per four caster levels above 12th.
    \spellsuccess If the target is undead, it instead takes that much positive damage.
    \spellfailure As above, but half damage.
\end{spelltargets}

\spellsection{Cure Serious Wounds}{3}
\spelldesc{You lay your hand on a creature and channel positive energy into it, healing even serious injuries.}
\spellinfo{Necro (Vitalism) [Positive]}{Divine, Life, Nature}
\spellrng{\rngclose}
\spellsr{Yes (Fortitude)}
\begin{spelltarget}{One creature}[Magic vs. Fortitude]
    \spelleffect As \spell{cure light wounds}, except that it heals 6d6 damage \add d6 per two caster levels above 6th. For every 10 points of healing granted by this spell, it can instead cure 1 point of critical damage.
    \spellsuccess If the target is undead, it instead takes that much positive damage.
    \spellfailure As above, but half damage.
\end{spelltarget}

\spellsectioncomma{Cure Serious Wounds}{Mass}{7}
\spelldesc{You stretch out your hand and channel positive energy into all of your allies, healing even serious injuries.}
\spellinfo{Necro (Vitalism) [Positive]}{Divine, Life, Nature}
\begin{spelltargets}{Five creatures in the area}[Magic vs. Fortitude]
    \spelleffect As \spell{cure serious wounds}, except that it heals 7d6 damage \add d6 per four caster levels above 14th.
    \spellsuccess If the target is undead, it instead takes that much positive damage.
    \spellfailure As above, but half damage.
\end{spelltargets}

\pdfbookmark[2]{D}{SpellDescriptionsD}
\begin{comment}
\subsubsection{D}
\end{comment}

\spellsection{Dancing Lights}{1}
\spelldesc{You create floating lights to guide your way.}
\spellinfo{Illus (Figment) [Light]}{Arcane}
\spelltwocol{\spelllimit{\areamed radius}}{\spellrng{\rngmed}}
\spelldur{\durshort \dismissable}
\spellline
\spelleffect This spell creates mobile sources of light. You can create up to four lights which resemble lanterns or torches, up to four glowing spheres of light, or a single glowing, vaguely humanoid shape. Regardless of their form, each light creates bright illumination in a \areamed radius, as a torch.

As a swift action, you can move the lights as you desire through the air. They can move up to 100 feet per round, but they must always stay within range of you, and all the lights must remain within a single \areamed radius. Any light which goes beyond those limits winks out.
\spellnotes This spell can be made permanent with a \spell{permanency} ritual.

\spellsection{Darkvision}{2}
\spelldesc{You grant an ally the ability to see in complete darkness.}
\spellinfo{Div (Awareness)}{Arcane}
\spellrng{\rngtouch}
\spelldur{\durlong}
\spellsr{Yes (Fortitude)}
\begin{spelltarget}{One creature}
    \spelleffect The subject gains the ability to see 60 feet even in total darkness. Beyond 60 feet, the subject can see dimly, treating areas of darkness as shadowy illumination. Darkvision does not function if a creature is in an area of bright light or is dazzled. Darkvision is black and white only, but otherwise like normal sight.
\end{spelltarget}
\spellnotes This spell does not grant the ability to see in magical darkness. This spell can be made permanent with a \spell{permanency} ritual.

\spellsection{Daylight}{2}
\spelldesc{You infuse an object with the power of the sun, causing it to illuminate a large area.}
\spellinfo{Illus (Figment) [Light]}{Divine}
\spellrng{\rngtouch}
\spelldur{\durlong \dismissable}
\spellsr{Yes (Fortitude)}
\begin{spelltarget}{Object touched}
    \spelleffect The object touched sheds light as bright as full daylight in a \arealarge radius, and dim light for an additional 50 feet beyond that. Creatures that take penalties in bright light also take them while within the radius of this magical light. Despite its name, this spell is not the equivalent of sunlight for the purposes of creatures that are damaged or destroyed by bright light.
    \par If \spell{daylight} is cast on a small object that is then placed inside or under a light-proof covering, the spell's effects are blocked until the covering is removed.
\end{spelltarget}
\spellnotes \spell{Daylight} brought into an area of magical darkness (or vice versa) is temporarily negated, so that the otherwise prevailing light conditions exist in the overlapping areas of effect.

\spellsection{Death Knell}{2}
\spelldesc{You draw forth the ebbing life force of a creature and use it to fuel your own power.}
\spellinfo{Necro (Life) [Death]}{Death, Evil, Necro}
\spellrng{\rngmed}
\spelldur{\durshort; see text}
\begin{spelltarget}{Living creature}[Magic vs. Fortitude]
    \spellsuccess If the subject is bloodied, it is \vulnerable. If it drops to 0 hit points, it dies immediately, and you gain 10 temporary hit points \add 1 per caster level above 4th. These temporary hit points last for 1 round per level the subject had.
\end{spelltarget}
\spellnotes If you take life damage, you lose all temporary hit points provided by this spell before applying the damage.

\spellsection{Death Ward}{3}
\spelldesc{You shield an ally from deadly spells.}
\spellinfo{Abjur/Necro (Shielding, Vitalism) [Positive]}{Death, Divine, Good, Protection}
\spellrng{\rngclose}
\spelldur{\durshort}
\spellsr{Yes (Fortitude)}
\begin{spelltarget}{One living creature}
    \spelleffect The subject is immune to all death spells, magical death effects, energy drain, and any negative energy effects.
\end{spelltarget}
\spellnotes This spell doesn't remove negative levels that the subject has already gained. It does not protect against other sorts of attacks, even if those attacks might be lethal.

\spellsectioncomma{Death Ward}{Mass}{7}
\spelldesc{You shield your allies from deadly spells.}
\spellinfo{Abjur/Necro (Shielding, Vitalism) [Positive]}{Death, Divine}
\spelltwocol{\spelllimit{\areamed radius}}{\spellrng{\rngmed}}
\spelldur{\durshort}
\spellsr{Yes (Fortitude)}
\begin{spelltarget}{Five living creatures in the area}
    \spelleffect The target is protected, as \spell{death ward}.
\end{spelltarget}

\spellsection{Deep Slumber}{7}
\spelldesc{You fill your foe with an overpowering urge to sleep, inevitably rendering him comatose.}
\spellinfo{Ench (Compulsion) [Mind-Affecting]}{Arcane}
\spellrng{\rngmed}
\spelldur{\durlong}
\spellsr{Yes (Will)}
\begin{spelltarget}{One creature}[Magic vs. Will]
    \spellsuccess The target is \bewildered. If it is \bloodied, it immediately falls asleep. If left undisturbed, it will sleep until it dies. As long as it remains bloodied, it cannot be awakened until the spell's duration expires, though it can be awakened normally after that point.
\end{spelltarget}

\spellsection{Deflection}{3}
\spelldesc{You shield yourself from enemy attacks, causing them to deflect away from you harmlessly.}
\spellinfo{Abjur (Shielding)}{Abjur}
\spellrng{Personal}
\spelldur{\durlong}
\begin{spelltarget}{You}
    \spelleffect You gain a \plus2 enhancement bonus to your physical defenses. \spellbonusscalingdescription
\end{spelltarget}
\spellnotes The enhancement bonus from this spell stacks with any enhancement bonuses to defense modifiers, such as enhancement bonuses to your armor. It does not stack with other enhancement bonuses that apply directly to your physical defenses. 

\spellsection{Delay Damage}{5}
\spelldesc{You partially shift yourself into the future, delaying the impact of attacks against you.}
\spellinfo{Abjur/Trans (Shielding, Temporal)}{\spelllists{Arcane}}
\spellrng{Personal}
\spelldur{\durmed}
\begin{spelltarget}{You}
    \spelleffect Whenever you take damage, half of the damage (rounded down) is not dealt to you immediately. This damage is tracked separately. At the end of the spell's duration, you take all of the delayed damage at once. For every point of damage dealt in this way in excess of your hit points, you take 1 point of critical damage.
\end{spelltarget}

\spellsection{Delay Poison}{1}
\spellinfo{Necro (Flesh)}{Divine, Nature}
\spelltime{1 swift action}
\spellrng{\rngclose}
\spelldur{\durshort}
\spellsr{Yes (Fortitude)}
\begin{spelltarget}{One creature}
    \spelleffect The subject becomes temporarily immune to the effects of poison. Poisons the subject is exposed to do not make attacks against it. This effect does not prevent the subject from becoming poisoned, and any poisons in the subject's system when the spell ends will continue their effects normally. 
\end{spelltarget}
\spellnotes This spell does not cure any damage that poison may have already done.

\spellsection{Delayed Blast Fireball}{6}
\spellinfo{Evoc (Energy) [Destructive, Fire]}{Arcane, Fire}
\spelltwocol{\spellburst{\areamed radius}}{\spellrng{\rngmed}}
\spelldur{5 rounds or less; see text}
\spellsr{Yes (Reflex)}
\spellline
\spelleffect You can delay this spell's attack until up to 5 rounds after the spell is cast. You select the amount of delay upon completing the spell, and that time cannot change once it has been set unless someone touches the bead (see below). For every round that this spell is delayed, your caster level with it increases by 2.

If you choose a delay, a glowing bead sits at the point of origin until it detonates. A creature can pick up and hurl the bead as a thrown weapon (range increment 10 feet). If a creature handles and moves the bead within 1 round of its detonation, there is a 25\% chance that the bead detonates while being handled.
\begin{spelltargets}{Everything in the area}[Magic vs. Reflex]
    \spellsuccess 6d6 fire damage \add d6 per four caster levels above 12th.
    \spellfailure As above, but half damage.
\end{spelltargets}

\spellnotes As \spellnotes{fireball}.

\spellsection{Destruction}{7}
\spellinfo{Necro (Flesh) [Death]}{Destruction, Divine}
\spellcmp{Verbal, Somatic, and Focus}
\spellrng{\rngclose}
\spellsr{Yes (Fortitude)}
\begin{spelltarget}{One creature}[Magic vs. Fortitude]
    \spellsuccess The target is \staggered for 5 rounds. If it is bloodied, it loses all its hit points and takes 9 critical damage, causing it to begin dying.
\end{spelltarget}
\spellnotes The remains of a creature killed by this spell are consumed utterly (but not its equipment or possessions). The only way to restore life such a creature is to use \spell{true resurrection}, a carefully worded \spell{wish} spell followed by \spell{resurrection}, or \spell{miracle}.
\spellfocus{A special holy (or unholy) symbol of silver marked with verses of anathema (cost 250 gp).}

\spellsection{Detect Alignment}{3}
\spelldesc{You sense the presence of creatures with a particular alignment.}
\spellinfo{Div (Awareness) [Detection]}{Divine}
\spellemanation{\arealarge cone from you}
\spelldur{Concentration}
\spellline
\spelleffect As you cast this spell, you choose an alignment: good, evil, lawful, or chaotic. Anything within the spell's area that has the chosen alignment has a faint aura, visible only to you.

By concentrating on an aura (a standard action), you can determine the strength of the aura. Most aligned creatures and magic items have a faint aura. Creatures that embody the alignment (such as undead and outsiders with an aligned creature subtype) have a moderate aura. Creatures that act directly on behalf of the alignment (such as paladins and some clerics), and exceptionally potent aligned magic items (primarily artifacts) have a strong aura.
\spellnotes Each round, you can turn to detect objects in a new area. A detection spell can penetrate barriers, but 1 foot of stone, 1 inch of common metal, a thin sheet of lead, or 3 feet of wood or dirt blocks it.

\spellsection{Dictum}{7}
\spellinfo{Evoc (Channeling) [Lawful]}{Divine, Law}
\spellcmp{Verbal only}
\spellburst{\arealarge radius centered on you}
\spelldur{Instantaneous/5 rounds}
\spellsr{Yes (Will)}
\begin{spelltargets}{All nonlawful creatures in the area}
    \spelleffect If the target's level does not exceed your caster level, it is \sickened for 5 rounds.

    If it is also bloodied, it also suffers one or more of the following ill effects, depending on its level.
    \begin{dtable}
        \begin{tabularx}{\columnwidth}{l >{\lcol}X}
            \par \thead{Level} & \thead{Effect} \\
            \par Equal to caster level & Staggered \\
            \par Up to caster level \minus5 & Stunned, staggered \\
            \par Up to caster level \minus10 & Paralyzed, stunned, staggered \\
            \par Up to caster level \minus15 & Killed\fn{1}
        \end{tabularx}
        1 Living creatures die. Nonliving creatures are destroyed.
    \end{dtable}
    \par \subspell{Staggered} The creature is staggered for 5 rounds. It can take a move action or a standard action each round, but not both.
    \par \subspell{Stunned} The creature is stunned for 1 round.
    \par \subspell{Paralyzed} The creature is paralyzed and helpless for 5 rounds.
    \par \subspell{Killed} Living creatures die. Nonliving creatures are destroyed.
\end{spelltargets}

\spellsection{Dimension Door}{4}
\spellinfo{Conj (Translocation) [Teleportation]}{Arcane, Travel}
\spellrng{\rngext \rngunrestricted}
\begin{spelltarget}{You}
    \spelleffect The target is instantly teleported from its current location to any location within range, regardless of intervening obstacles. The destination can be visualized or specified by stating a direction and distance. The target can bring along objects as long as their weight doesn't exceed its maximum load.

    After arriving, the target is dazed until the next action phase.
\end{spelltarget}
\spellnotes \par If you arrive in a place that is already occupied by a solid body, you take 2d6 damage and are shunted to a random open space on a suitable surface within 100 feet of the intended location that is within the range of the spell.
\par  If there is no free space within 100 feet, you take an additional 4d6 damage and the spell simply fails.

\spellsectioncomma{Dimension Door}{Mass}{7}
\spellinfo{Conj (Translocation) [Teleportation]}{Conj, Travel}
\spelltwocol{\spelllimit{\areamed radius centered on you}}{\spellrng{\rngext}}
\begin{spelltarget}{You and up to five other willing creatures in the area}
    \spelleffect The target is teleported, as \spell{dimension door}.
\end{spelltarget}
\spellnotes You can choose the destinations for each target independently, within the range of the spell. 

\spellsection{Dimension Slide}{3}
\spellinfo{Conj (Translocation) [Teleportation]}{Conj, Travel}
\spellrng{\rngclose}
\spellsr{Yes (Will)}
\begin{spelltarget}{One creature}[Magic vs. Will]
    \spelleffect The target is instantly teleported from its current location to any other spot within range, regardless of intervening obstacles. The destination must be an unoccupied space on stable ground. The target can bring along objects as long as their weight doesn't exceed the target's maximum load.
\end{spelltarget}
\spellnotes If you somehow attempt to transfer the creature into a location occupied by a solid body or a location you can't see, the spell simply fails to function.

\spellsection{Dimensional Anchor}{3}
\spelldesc{You surround your foe in a shimmering emerald field that completely blocks extradimensional travel, preventing it from escaping you.}
\spellinfo{Abjur (Negation)}{Arcane, Divine, Magic}
\spellrng{\rngmed}
\spelldur{\durlong/5 rounds}
\spellsr{Yes (Will)}
\begin{spelltarget}{One creature}[Magic vs. Will]
    \spellsuccess  The subject cannot travel extradimensionally for 1 hour. Effects barred by a \spell{dimensional anchor} include \spell{astral projection}, \spell{blink}, \spell{dimension door}, \spell{ethereal jaunt}, \spell{gate}, \spell{maze}, \spell{plane shift}, \spell{shadow walk}, \spell{teleport}, and similar spell-like or supernatural abilities.
    \spellfailure As above, but the effect lasts for 5 rounds.
\end{spelltarget}
\spellnotes This spell does not interfere with the movement of creatures already in ethereal or astral form when the spell is cast, nor does it block extradimensional perception or attack forms, such as summoning monsters. Also, it does not prevent summoned creatures from disappearing at the end of a summoning spell.

\spellsection{Discern Lies}{3}
\spelldesc{You can discern subtle magical disturbances caused by lying.}
\spellinfo{Div (Awareness) [Detection]}{Divine, Law}
\spellemanation{\arealarge cone from you}
\spelldur{Concentration}
\spellline
\spelleffect You know when any creature in the area deliberately and knowingly speaks a lie. The spell does not reveal the truth, uncover unintentional inaccuracies, or necessarily reveal evasions.
\spellnotes Each round, you can turn to discern lies in a new area. A detection spell can penetrate barriers, but 1 foot of stone, 1 inch of common metal, a thin sheet of lead, or 3 feet of wood or dirt blocks it.

\spellsection{Discern Vulnerability}{4}
\spellinfo{Div (Knowledge)}{Arcane}
\spelltime{1 swift action}
\spellrng{\rngmed}
\begin{spelltarget}{One creature}
    \spelleffect You instantly recognize all of the target's vulnerabilities. This grants you a \plus2 bonus to attacks and weapon damage against that creature. In addition, you learn any significant weaknesses the creature has. This includes, but is not limited to, the following information:
    \begin{itemize}
        \item Which of the target's defenses is lowest
        \item If the target has any vulnerabilities to specific damage types
        \item How to overcome the target's damage reduction, regeneration, or other similar abilities
    \end{itemize}
\end{spelltarget}
\spellnotes This spell gives no information about a creature's strengths or abilities -- only its weaknesses.

\spellsection{Disintegrate}{6}
\spelldesc{You shoot a thin, green ray from your pointing finger that completely destroys whatever it hits.}
\spellinfo{Trans (Alteration)}{Arcane, Destruction}
\spellrng{\rngclose}
\spellsr{Yes (Fortitude)}
\begin{spelltarget}{One creature or attended object}l[Magic vs. Reflex and Fortitude (object)]
    \spellsuccess 12d8 physical damage \add d8 per two caster levels above 12th.
    \spellfailure[Fortitude] As above, but half damage.
    \spellfailure[Reflex] No effect.
    \spelleffect Any creature reduced to 0 hit points by this spell is entirely disintegrated, leaving behind only a trace of fine dust. Its equipment is unaffected.
    \par When used against an object, the ray simply disintegrates as much as one 10-foot cube of nonliving matter. Thus, the spell disintegrates only part of any very large object or structure targeted.
\end{spelltarget}
\spellnotes This spell affects even objects constructed entirely of force, such as \spell{wall of force}, but not magical effects such as an \spell{antimagic field}.

\spellsection{Dismissal}{4}
\spellinfo{Abjur/Conj (Interdiction, Translocation) [Planar]}{Arcane, Divine}
\spellrng{\rngclose}
\spellsr{Yes (Will)}
\begin{spelltarget}{One extraplanar creature}[Magic vs. Will]
    \spellsuccess The target is sent back to its proper plane. There is a 20\% chance of actually sending it to a plane other than its own.
\end{spelltarget}

\spellsection{Dispel Magic}{3}
\spellinfo{Abjur (Negation) [Magic]}{Arcane, Divine, Magic, Nature}
\spellrng{\rngmed}
\spellspecial This spell has two versions: a targeted dispel, and an area dispel. Its effects depend on which version is chosen.
\begin{spelltarget}{One creature or object}[Caster level vs. Special]
    \spelleffect For every spell affecting the target, if the attack result beats a DC equal to 10 \add the caster level of the spell, it is dispelled.

    If the target is an object, and the attack result beats a DC equal to 10 \add the caster level of the object, the object is suppressed for 5 rounds. A suppressed object loses all its magical abilities, though it is still treated as being a magical object for the purpose of spells and effects.

    If the target is an effect of an ongoing spell (such as a summoned creature), and the attack result beats a DC equal to 10 \add the caster level of the spell, the target is treated as if the spell that created it was dispelled. This usually causes the target to disappear.
\end{spelltarget}

\spellline
\spelllimit{\areamed radius}
\begin{spelltargets}*{All creatures and unattended objects in the area}l[Caster level vs. Special]
    \spelleffect If the attack result beats a DC equal to 10 \add the caster level of the highest level spell on the target, that effect is dispelled. If there are multiple spells of the same level, choose one randomly.
\end{spelltargets}

\spellnotes A dispelled spell ends as if its duration had expired. If a spell affects multiple targets, it must be dispelled individually on each target. Dispelling the effect on one target does not affect the other targets of the spell.

Some spells, as detailed in their descriptions, can't be dispelled by this spell. A spell without a duration cannot be dispelled, even if it has a lasting effect.

You may choose to automatically succeed or fail on your attack against any spell that you cast yourself.

Spell-like abilities are treated like spells, and this spell affects them in the same way it affects spells.

Artifacts and deities are unaffected by mortal magic such as this.

\spellsectioncomma{Dispel Magic}{Greater}{6}
\spellinfo{Abjur (Negation) [Magic]}{Arcane, Divine, Magic, Nature}
\spelltwocol{\spellarea{\areamed radius limit}}{\spellrng{\rngmed}}
\begin{spelltargets}{All creatures and unattended objects in the area}l[Caster level vs. Special]
    \spelleffect Spells affecting the target are dispelled, as a targeted \spell{dispel magic}.
\end{spelltargets}
\spellnotes As \spell{dispel magic}. In addition, this spell has a chance to dispel any effect that \spell{remove curse} can remove, even if \spell{dispel magic} can't dispel that effect.

\spellsection{Displacement}{4}
\spellinfo{Illus (Glamer)}{Arcane}
\spellrng{\rngclose}
\spelldur{\durshort \dismissable}
\spellsr{Yes (Will)}
\begin{spelltarget}{One creature}
    \spelleffect The target appears to be about 2 feet away from its true location. Attacks against the subject have a 50\% miss chance as if it were invisible. However, unlike invisibility, this spell does not prevent enemies from targeting the creature normally, and it does not allow the creature to hide.
\end{spelltarget}

\spellsection{Disrupting Weapon}{4}
\spellinfo{Necro/Trans (Imbuement, Positive)}{Divine}
\spellrng{\rngclose}
\spelldur{\durshort}
\spellsr{Yes (Fortitude)}
\begin{spelltarget}{One melee weapon}
    \spelleffect The target weapon becomes deadly to undead.
    \begin{spelltrigger}{The target weapon strikes a bloodied undead creature for the first time in a round}
        \begin{spelltarget}*{One bloodied undead creature}[Magic vs. Fortitude]
            \spellsuccess The target creature is utterly destroyed.
        \end{spelltarget}
    \end{spelltrigger}
\end{spelltarget}

\spellsection{Divine Favor}{1}
\spelldesc{You imbue yourself with skill in combat by calling upon the divine power of your patron.}
\spellinfo{Trans (Augment)}{Divine, Strength, War}
\spelldur{\durshort}
\begin{spelltarget}{You}
    \spelleffect \plus2 enhancement bonus on attack and weapon damage rolls. \spellbonusscalingdescription
\end{spelltarget}

\spellsectioncomma{Divine Favor}{Greater}{4}
\spelldesc{You imbue yourself with great strength and skill in combat by calling upon the divine power of your patron.}
\spellinfo{Trans (Augment)}{Divine, Strength, War}
\spelldur{\durshort}
\begin{spelltarget}{You}
    \spelleffect As \spell{divine favor}, except that you also gain a \plus2 enhancement bonus to Strength.
\end{spelltarget}

\spellsection{Dominate Monster}{8}
\spellinfo{Ench (Compulsion) [Domination, Mind-Affecting]}{Ench}
\spellrng{\rngmed}
\spelldur{One day}
\begin{spelltarget}{One creature}[Magic vs. Will]
    \spellsuccess The target is dominated, as \spell{dominate person}, except that the effect does not depend on creature type.
\end{spelltarget}

\spellsection{Dominate Person}{7}
\spellinfo{Ench (Compulsion) [Domination, Mind-Affecting]}{Ench}
\spellrng{\rngmed}
\spelldur{One day}
\begin{spelltarget}{One humanoid creature}[Magic vs. Will]
    \spellsuccess You can control the actions of the target through a telepathic link that you establish with the subject's mind.
    \par If you and the subject have a common language, you can generally force the subject to perform as you desire, within the limits of its abilities. If no common language exists, you can communicate only basic commands, such as ``Come here," ``Go there," ``Fight," and ``Stand still." If you concentrate on the spell, you know what the subject is experiencing, but you do not receive direct sensory input from it, nor can it communicate with you telepathically.
    \par Once you have given a dominated creature a command, it continues to attempt to carry out that command to the exclusion of all other activities except those necessary for day-to-day survival (such as sleeping, eating, and so forth). Because of this limited range of activity, a Sense Motive check against DC 15 (rather than DC 25) can determine that the subject's behavior is being influenced by an enchantment effect (see the Sense Motive skill description).
    It takes time for the link to be established. For the first hour after the spell is cast, you must concentrate on the spell (a standard action) to control the subject's actions. While you are not concentrating on the spell, the creature acts as if confused, as the \spell{confusion} spell, except that it never attacks you. If the subject would randomly attack you, it instead is forced to follow your commands. At the end of the hour, you must make a second Will attack. If you concentrate on the spell during this time, you gain a \plus4 bonus to the attack. If your attack succeeds, you dominate the creature fully for the remainder of the spell duration. Otherwise, the creature is freed.
    \par After the first hour, changing your instructions or giving a dominated creature a new command is the equivalent of redirecting a spell, so it is a move action.
    \par By concentrating fully on the spell (a standard action), you can receive full sensory input as interpreted by the mind of the subject, though it still can't communicate with you. You can't actually see through the subject's eyes, so it's not as good as being there yourself, but you still get a good idea of what's going on.
    \par Subjects resist this control, and you must make a new attack to force the subject to take an action against its nature. Failure means it breaks free. This does not apply when a subject is merely ordered to perform an action it disagrees with -- the action must be directly opposed to the subject's beliefs. Ordering a paladin to murder an innocent would require a new attack, but ordering him to build a bridge that would allow an evil army to cross a river would not. If your command would obviously lead to the creature's death, your attack takes a \minus10 penalty. Once control is established, the range at which it can be exercised is unlimited, as long as you and the subject are on the same plane. You need not see the subject to control it.
    \par If you recast this spell on a subject you have dominated before it escapes your control, you can extend the duration of the spell indefinitely. You do not need to make a new attack when you renew your control in this fashion.
\end{spelltarget}

\pdfbookmark[2]{E}{SpellDescriptionsE}
\begin{comment}
\subsubsection{E}
\end{comment}
\spellsr{Yes (Will)}

\spellsection{Earth's Pull}{1}
\spelldesc{You intensify the pull of gravity on your foe, causing it to feel much heavier and making its movements sluggish.}
\spellinfo{Evoc (Control) [Earth]}{Earth, Nature}
\spellrng{\rngmed}
\spelldur{\durshort}
\spellsr{Yes (Will)}
\begin{spelltarget}{One Large or smaller creature}
    \spelleffect The subject moves at half speed and takes a \minus2 penalty to physical defenses. If it is flying within 10 feet of the ground, the subject falls to the ground.
\end{spelltarget}
\spellnotes If the subject gets farther than 10 feet from the ground, the spell's effect is broken. As a result, the spell cannot affect creatures flying high above the ground.

\spellsection{Earthen Blade}{2}
\spellinfo{Trans (Alteration, Augment) [Earth]}{Earth, Nature}
\spellrng{Touch}
\spelldur{\durlong \dismissable}
\spellsr{Yes (Fortitude)}
\spellline
\spelleffect This spell creates a weapon from the ground. The weapon can be of any type you are proficient with. In addition, the weapon is magical, as the \spell{magic weapon} spell.

\spellsection{Earth Glide}{5}
\spellinfo{Trans (Imbuement) [Earth]}{Earth, Nature}
\spellrng{Touch}
\spelldur{\durshort}
\spellsr{Yes (Fortitude)}
\begin{spelltarget}{One creature}
    \spelleffect The subject gains the earth glide ability, as an earth elemental. This allows it to glide through stone, dirt, or almost any other sort of earth except metal  as if it were air. The subject can walk or climb at any angle in the earth. However, the subject generally cannot breathe, speak, or hear while gliding. While gliding, a creature can remain partially within the earth, granting it cover.
\end{spelltarget}
\spellnotes The subject's burrowing leaves behind no tunnel or hole, nor does it create any ripple or other signs of its presence.

\spellsection{Earthquake}{8}
\spelldesc{An intense but highly localized tremor shakes the ground. The shock knocks creatures down and makes fighting or escaping difficult. Eventually, the ground swallows those foolish enough to remain.}
\spellinfo{Evoc (Control) [Earth]}{Destruction, Divine, Earth, Nature}
\spelltwocol{\spellzone{\arealarge radius}}{\spellrng{\rngmed}}
\spelldur{5 rounds}
\spellline
\spelleffect The area is difficult terrain. Creatures in the area take a \minus2 penalty to physical attacks, defenses, and checks. Casting a spell in the area requires a Concentration check against a DC equal to 10 \add your caster level \add double the level of the spell being cast.
\begin{spelltrigger}{End of every round, except the last round}
    \begin{spelltarget}*{All creatures in the area}[Magic vs. Reflex]
        \spellsuccess The target is knocked prone.
    \end{spelltarget}
    \begin{spelltarget}*{All buildings and structures in the area}
        \spelleffect 8d8 bludgeoning damage \add d8 per four caster levels above 16th.
    \end{spelltarget}
\end{spelltrigger}
\begin{spelltrigger}{End of the last round}
    \begin{spelltarget}*{All creatures in the area}
        \spelleffect 8d6 bludgeoning damage \add d6 per four caster levels above 16th. In addition, the target is grappled by the ground until it escapes. To escape, it must beat a Maneuver defense equal to 10 \add your caster level \add your casting attribute.
    \end{spelltarget}
\end{spelltrigger}
\spellnotes In terrain with unusual ground, such as rivers or swamps, this spell may have different effects.

\spellsection{Earthspike}{3}
\spelldesc{You create a spike from the ground that impales your foe.} 
\spellinfo{Trans (Animation) [Earth]}{Earth, Nature}
\spellrng{\rngmed}
\spellsr{Yes (Fort)}
\begin{spelltarget}{One creature or object within 10 feet of natural earth or stone}l[Caster level \add casting attribute vs. Armor defense and Maneuver defense]
    \spellsuccess[Armor defense] 6d8 piercing damage \add d8 per two caster levels above 6th.
    \spellsuccess[Armor defense and Maneuver defense] The target is \immobilized by a spike from the ground for 5 rounds. It can break free by destroying the spike. The spike's physical defenses are all 10, and it has 2 hit points per caster level.
\end{spelltarget}

\spellsectioncomma{Earthspike}{Mass}{6}
\spellinfo{Trans (Animation) [Earth]}{Earth, Nature}
\spellrng{\rngmed}
\spelllimit{\areasmall radius}
\begin{spelltargets}{Everything in the area within 10 feet of natural earth or stone}l[Caster level \add casting attribute vs. Armor defense and Maneuver defense]
    \spellsuccess The target is damaged and possibly immobilized, as \spell{earthspike}, except that it takes 6d8 piercing damage \add d8 per four caster levels above 12th.
\end{spelltargets}
\spellnotes This spell cannot attack more than one target within a single 5-ft. square. Each immobilized creature must destroy a separate spike to break free.

\spellsection{Edict}{6}
\spellinfo{Abjur/Div (Communication, Interdiction)}{Abjur, Law}
\spellemanation{\arealarge radius centered on you}
\spelldur{\durshort}
\spellsr{Yes (Will)}
\spellline
\spelleffect You loudly declare a single, specific rule which which all creatures must obey, such as ``Do not use ranged weapons'' or ``Do not lie''. If the rule is too complicated, the spell fails. The spell grants all creatures that enter the area an understanding of the rule, even if they were unable to understand the rule as originally stated. If you break the rule, the spell ends -- after you suffer the consequences.
\begin{spelltrigger}{A creature breaks the rule}
    \begin{spelltarget}*{The creature breaking the rule}
        \spelleffect 6d6 damage \add 1d6 per four caster levels above 12th. You know a creature broke the rule, but not which creature.
    \end{spelltarget}
\end{spelltrigger}
\spellnotes Mindless creatures are given no special insight into the rule.

\spellsection{Elemental Swarm}{9}
\spellinfo{Conj (Summoning) [see text]}{Air, Earth, Fire, Nature, Water}
\spelltwocol{\spelllimit{\arealarge radius}}{\spellrng{\rngmed}}
\spelldur{\durlong \dismissable}
\spellline
\spelleffect This spell opens a portal to an Elemental Plane and summons elementals from it. A druid can choose the plane (Air, Earth, Fire, or Water); a cleric opens a portal to the plane matching his domain.
\par When the spell is complete, 2d4 Large elementals appear. Five minutes later, 1d4 Huge elementals appear. Five minutes after that, one greater elemental appears. All creatures initially appear wherever you desire within the spell's area. Once these creatures appear, they serve you for the duration of the spell.
\par The elementals obey you explicitly and never attack you, even if someone else manages to gain control over them. You do not need to concentrate to maintain control over the elementals. You can dismiss them singly or in groups at any time.
\spellnotes When you use a summoning spell to summon an air, earth, fire, or water creature, it is a spell of that type.

\spellsection{Energy Conversion}{7}
\spellinfo{Abjur/Evoc (Energy, Shielding) [see text]}{Arcane, Protection}
\spellrng{Personal and \rngclose; see text}
\spelldur{\durlong or until discharged}
\spellsr{Yes (Fortitude)}
\begin{spelltarget}{You}
    \spelleffect You resist energy damage, as \spell{greater resist energy}, except that you store up the energy you absorb. As a standard action, you can expend some of that energy to make an attack.
\end{spelltarget}
\begin{spelltarget}{One creature}[Magic vs. Reflex]
    \spelleffect Choose an energy type that you have stored. You expend damage of that type of up to three times your caster level.
    \spellsuccess The target takes damage equal to the energy expended.
\end{spelltarget}
\spellattack{None, Physical vs. Reflex}
\spellnotes This spell's descriptor is the same as the type of energy you discharge in a ray; thus, its subtype can change during the course of the spell's duration.

\spellsection{Energy Drain}{8}
\spellinfo{Necro (Vitalism) [Negative]}{Arcane, Death, Divine, Evil}
\spellrng{\rngclose}
\spelldur{\durshort}
\spellsr{Yes (Fortitude)}
\begin{spelltarget}{One creature}[Magic vs. Fortitude]
    \spelleffect If the target is living, it gains six \negativelevels.

    \spelleffect If the target is undead, it gains temporary hit points equal to 40 \add your caster level and physical damage reduction 16/positive.
\end{spelltarget}
\spellnotes The damage reduction allows an undead subject to ignore the first 16 physical damage it takes each round. If it is hit by an attack that deals positive damage, such as \spell{cure light wounds}, it cannot use its damage reduction for 1 round.

\spellsection{Enervation}{4}
\spelldesc{Your foe's body loses its color momentarily as you drain its life force away.}
\spellinfo{Necro (Vitalism) [Negative]}{Arcane, Death, Divine, Evil}
\spellrng{\rngclose}
\spelldur{\durshort}
\spellsr{Yes (Fortitude)}
\begin{spelltarget}{One creature}[Magic vs. Fortitude]
    \spelleffect If the target is living, it gains three \negativelevels.

    If the target is undead, it gains physical damage reduction 8/positive instead. This damage reduction increases by 1 per two caster levels above 8th.
\end{spelltarget}
\spellnotes This spell stacks with any effect that bestows negative levels, including itself.

The damage reduction allows an undead subject to ignore the first 8 physical damage it takes each round. If it is hit by a positive attack, such as \spell{cure light wounds}, it cannot use its damage reduction for 1 round.

\spellsection{Enfeeblement}{1}
%\spelldesc{You fire a coruscating ray from your hand. When it strikes your foe, he becomes weaker.}
\spellinfo{Necro (Flesh)}{Arcane, Death}
\spellrng{\rngmed}
\spelldur{\durshort}
\spellsr{Yes (Fortitude)}
\begin{spelltarget}{One creature}[Magic vs. Fortitude]
    \spellsuccess The target takes a \minus4 penalty to your choice of Strength, Dexterity, or Constitution.
    \spellfailure As above, but the penalty is halved.
\end{spelltarget}
\spellnotes This spell cannot reduce an attribute below \minus9.

\spellsection{Enlarge Person}{3}
\spellinfo{Trans (Polymorph) [Size-Affecting]}{Strength, Trans}
\spelltime{Full-round action}
\spellrng{\rngclose}
\spelldur{\durshort \dismissable}
\spellsr{Yes (Fortitude)}
\begin{spelltarget}{One humanoid creature (Huge or smaller)}l[Magic vs. Fortitude]
    \spelleffect The target instantly grows, doubling its height and multiplying its weight by 8. This increase changes the creature's size category to the next larger one. This has several effects.
    \begin{itemize} 
        \item \plus10 ft. bonus to movement speed.
        \item \plus4 bonus to maneuver attack and defense.
        \item \minus1 penalty to other physical attacks and defenses.
        \item \minus2 penalty to Dexterity.
        \item \plus2 enhancement bonus to Strength.
        \item \minus4 penalty to Stealth checks.
    \end{itemize}
    \par A typical humanoid creature whose size increases to Large has a space of 10 feet and a natural reach of 10 feet.
    \par If insufficient room is available for the desired growth, the creature attains the maximum possible size and may make a Strength check (using its increased Strength) to burst any enclosures in the process. If it fails, it is constrained without harm by the materials enclosing it -- the spell cannot be used to crush a creature by increasing its size.
    \par All equipment worn or carried by a creature is similarly enlarged by the spell. Melee and projectile weapons affected by this spell deal more damage. Other magical properties are not affected by this spell. Any enlarged item that leaves an enlarged creature's possession (including a projectile or thrown weapon) instantly returns to its normal size. This means that thrown weapons deal their normal damage, and projectiles deal damage based on the size of the weapon that fired them. Magical properties of enlarged items are not increased by this spell.
\end{spelltarget}
\spellnotes Multiple magical effects that increase size do not stack. This spell can be made permanent with a \spell{permanency} ritual.

\spellsectioncomma{Enlarge Person}{Mass}{7}
\spellinfo{Trans (Polymorph) [Size-Affecting]}{Strength, Trans}
\spelltime{Full-round action}
\spelltwocol{\spelllimit{\areamed radius}}{\spellrng{\rngmed}}
\spelldur{\durshort \dismissable}
\spellsr{Yes (Fortitude)}
\begin{spelltarget}{Five humanoid creatures in the area (Huge or smaller)}
    \spelleffect The target is enlarged, as \spell{enlarge person}.
\end{spelltarget}

\spellsection{Entangle}{1}
\spelldesc{Grasses, weeds, bushes, and even trees ensnare creatures in the area.}
\spellinfo{Trans (Animation)}{Nature, Wild}
\spelltwocol{\spellzone{\areasmall radius}}{\spellrng{\rngmed}}
\spelldur{\durshort \dismissable}
\spellline
\spelleffect The area is difficult terrain.
\begin{spelltrigger}{End of every movement phase}
    \begin{spelltarget}*{Each creature in the area within 5 feet of plants}l[Magic vs. Reflex]
        \spellsuccess The target is \entangled and \immobilized. It can break free of the effect as a standard action by making a grapple attack or Escape Artist check that beats your magic attack result.
    \end{spelltarget}
\end{spelltrigger}
\spellnotes The effects of this spell may be altered somewhat based on the nature of the plants in the area. If no plants exist in the area, this spell has no effect.

\spellsection{Entangling Growth}{4}
\spelldesc{Grasses, weeds, bushes, and even trees grow out of thin air to ensnare creatures in the area.}
\spellinfo{Trans (Alteration, Animation)}{Nature, Wild}
\spelltwocol{\spellzone{\areamed radius}}{\spellrng{\rngmed}}
\spelldur{\durshort \dismissable}
\spellline
\spelleffect The area is difficult terrain. Plants grow in the area, even if the terrain would not normally support plant life. At the end of the spell's duration, the plants recede into the ground, leaving no trace that they were ever there.
\begin{spelltrigger}{End of every movement phase}
    \begin{spelltarget}{Each creature in the area within 5 feet of the ground}l[Magic vs. Reflex]
        \spellsuccess The target is ensnared, as \spell{entangle}.
    \end{spelltarget}
\end{spelltrigger}
\spellnotes The effects of this spell may be altered somewhat based on the nature of the plants in the area.

\spellsection{Entropic Shield}{2}
\spelldesc{You surround your ally with a magical field that glows with a chaotic blast of multicolored hues. This field deflects incoming ranged attacks, causing them to randomly swerve away from their intended target.}
\spellinfo{Abjur (Shielding)}{Chaos, Divine}
\spellrng{Touch}
\spelldur{\durshort \dismissable}
\begin{spelltarget}{One creature}
    \spelleffect Each physical ranged attack directed at the target has a 50\% miss chance (similar to the effects of active cover). Other attacks that simply work at a distance are not affected.
\end{spelltarget}

\spellsection{Ethereal Jaunt}{7}
\spellinfo{Conj (Translocation) [Planar]}{Arcane, Travel}
\spelldur{\durshort \dismissable}
\begin{spelltarget}{You}
\spelleffect You become ethereal, along with your equipment. For the duration of the spell, you are in a place called the Ethereal Plane, which overlaps the normal, physical, Material Plane. When the spell expires, you return to material existence.
\par An ethereal creature is invisible, insubstantial, and capable of moving in any direction, even up or down, albeit at half normal speed. As an insubstantial creature, you can move through solid objects, including living creatures. An ethereal creature can see and hear on the Material Plane, but everything looks gray and ephemeral. Sight and hearing onto the Material Plane are limited to 60 feet.
\par Force effects and abjurations affect an ethereal creature normally. Their effects extend onto the Ethereal Plane from the Material Plane, but not vice versa. An ethereal creature can't attack material creatures, and spells you cast while ethereal affect only other ethereal things. Certain material creatures or objects have attacks or effects that work on the Ethereal Plane (such as a basilisk's gaze attack). Treat other ethereal creatures and ethereal objects as if they were material. 
\par If you end the spell and become material while inside a material object (such as a solid wall), you are shunted off to the nearest open space and take 1d6 damage per 5 feet that you so travel.
\end{spelltarget}

\spellsection{Etherealness}{9}
\spellinfo{Conj (Translocation) [Planar]}{Arcane, Travel}
\spellrng{Touch}
\spellsr{Yes (Will)}
\begin{spelltarget}{You and up to five willing creatures}
    \spelleffect The target becomes ethereal, as \spell{ethereal jaunt}.
\end{spelltarget}
\spellnotes When the spell expires, all affected creatures on the Ethereal Plane return to material existence.

\spellsection{Expeditious Retreat}{1}
\spellinfo{Trans (Temporal)}{Trans}
\spellrng{\rngclose}
\spelldur{\durshort \dismissable}
\begin{spelltarget}{One creature}
    \spelleffect The target's land speed doubles, to a maximum of a \plus30 foot increase. (This adjustment is treated as an enhancement bonus.) There is no effect on other modes of movement.
\end{spelltarget}
\spellnotes As with any effect that increases your speed, this spell affects your ability to jump (see \pcref{Athletics}).

\pdfbookmark[2]{F}{SpellDescriptionsF}
\begin{comment}
\subsubsection{F}
\end{comment}
\spellsr{Yes (Will)}

\spellsection{Faerie Fire}{1}
\spellinfo{Illus (Figment) [Light]}{Nature}
\spelltwocol{\spelllimit{\areasmall radius}}{\spellrng{\rngmed}}
\spelldur{\durshort \dismissable}
\spellsr{Yes (Will)}
\begin{spelltarget}{Everything in the area}
    \spelleffect A pale glow surrounds and outlines the target, causing it to shed light as a candle. This negates invisibility, concealment, and similar effects.
\end{spelltarget}
\spellnotes Illusory figments such as \spell{silent image} are not outlined, which may reveal them for what they are. The light is too dim to have any special effect on creatures vulnerable to light. The \spell{faerie fire} can be blue, green, or violet, according to your choice at the time of casting. This spell does not cause any harm to the objects or creatures thus outlined.

\spellsection{False Reality}{9}
\spellinfo{Illus (Figment)}{Illus}
\spelllimit{1 mile radius}
\spelldur{\durlong \dismissable}
\spelleffect This spell functions like \spell{persistent image}, except that the illusion affects a massive area for \durlong duration.

\spellsection{Fear}{5}
\spelldesc{You project an invisible cone that drives creatures away from you in abject fear.}
\spellinfo{Ench (Emotion) [Fear, Mind-Affecting]}{Arcane}
\spellburst{\areamed cone}
\spelldur{\durshort \dismissable}
\spellattack{Magic vs. Will}
\begin{spellhealthy}
    Creatures in the area are \shaken.
\end{spellhealthy}
\begin{spellblood}
    Creatures in the area are \frightened.
\end{spellblood}
\spellsr{Yes (Will)}

\spellsection{Feather Fall}{1}
\spellinfo{Evoc (Control) [Air]}{Air, Evoc, Travel}
\spellcmp{Verbal only}
\spelltime{1 swift action}
\spelltwocol{\spelllimit{\areamed radius}}{\spellrng{\rngmed}}
\spelltgts{Five Medium or smaller freefalling objects or willing creatures in the area}
\spelldur{\durshort or until landing}
\spelleffect The affected creatures or objects fall slowly. Feather fall instantly changes the rate at which the targets fall to a mere 60 feet per round (equivalent to the end of a fall from a few feet), and the subjects take no damage upon landing while the spell is in effect. However, when the spell duration expires, a normal rate of falling resumes.
\par The spell affects one or more Medium or smaller creatures (including gear and carried objects up to each creature's maximum load) or objects, or the equivalent in larger creatures: A Large creature or object counts as two Medium creatures or objects, a Huge creature or object counts as two Large creatures or objects, and so forth.
\par If the spell is cast on a falling item, the object does half normal damage based on its weight, with no bonus for the height of the drop.
\spellnotes \par Feather fall works only upon free-falling objects. It no special effect on ranged weapons unless they are falling an extraordinary distance. It does not affect a sword blow or a charging or flying creature.
\spellsr{Yes (Will)}

\spellsection{Feeblemind}{5}
\spellinfo{Ench (Inhibition) [Mind-Affecting]}{Arcane}
\spelltwocol{\spelltgt{One creature}}{\spellrng{\rngtouch}}
\spellattack{Physical vs. Reflex, Magic vs. Will}
\begin{spellhealthy}
    If you touch the target, it is bewildered.
\end{spellhealthy}
\begin{spellblood}
    If you toucht the target, its Intelligence drops to \minus9, giving it roughly the intellect of a lizard. It is unable to use Intelligence-based skills, cast spells, understand language, or communicate coherently. Still, it knows who its friends are and can follow them and even protect them. The subject remains in this state until a \spell{heal}, \spell{limited wish}, \spell{miracle}, or \spell{wish} spell is used to cancel the effect of the \spell{feeblemind}.
\end{spellblood}
\spellnotes The target must be \bloodied when the spell is cast to suffer the \bloodied effect.
\spellsr{Yes (Will)}

\spellsection{Finger of Death}{7}
\spellinfo{Necro (Life) [Death]}{Arcane, Death}
\spelltwocol{\spelltgt{One living creature}}{\spellrng{\rngclose}}
\spellattack{Magic vs. Fortitude}
\begin{spellhealthy}
    The target is \staggered for 5 rounds.
\end{spellhealthy}
\begin{spellblood}
    The target loses all its hit points and takes 9 critical damage, causing it to begin dying.
\end{spellblood}
\spellsr{Yes (Fortitude)}

\spellsection{Fire Seeds}{6}
\spellinfo{Evoc/Trans (Energy, Imbuement) [Fire]}{Fire, Nature, Wild}
\spelltwocol{\spellburst{\areasmall or \areamed radius centered on the touched objects; see text}}{\spellrng{\rngtouch}}
\spelltgts{Up to four touched acorns or up to eight touched holly berries}
\spelldur{\durlong or until used}
\spelldmg{6d6 fire damage \add d6 per four caster levels above 12th (acorn grenades);\par
6d8 fire damage \add d8 per four caster levels above 12th (holly berry bombs)}
\spellattack{Physical vs. Reflex, Magic vs. Reflex; see text}
\spelleffect Depending on the version of fire seeds you choose, you turn acorns into splash weapons that you or another character can throw, or you turn holly berries into bombs that you can detonate on command.
\par \subspell{Acorn Grenades} As many as four acorns turn into special splash weapons that can be hurled as far as 100 feet. When you throw an acorn, you make a physical ranged attack vs. Reflex to strike the intended target. If you miss, the acorn detonates in a random corner of the intended target square. Together, the acorns are capable of dealing 6d6 fire damage \add d6 per four caster levels above 12th, divided up among the acorns as you wish.
\par Each acorn explodes upon striking any hard surface. When it does, you make a magic attack vs. Reflex to deal damage to all creatures in a \areasmall radius burst. A failed attack deals half damage. You automatically succeed on this attack against a creature struck by the acorn directly.
\par \subspell{Holly Berry Bombs} You turn as many as eight holly berries into special bombs. The holly berries are usually placed by hand, since they are too light to make effective thrown weapons (they can be tossed only 5 feet). Together, the holly berries are capable of dealing 6d8 fire damage \add d8 per four caster levels above 12th, divided up among the berries as you wish.
\par If you are within \rngmed range and speak a word of command (as a standard action), each berry instantly bursts into flame. When they do, you make a magic attack vs. Reflex to deal damage to every creature in a \areamed radius burst of the berries. A failed attack deals half damage.
\spellnotes You can only have one \spell{fire seeds} active at any time.
\par \mfx{Material Component} The acorns or holly berries.
\spellsr{Yes (Reflex)}

\spellsection{Fire Shield}{4}
\spelldesc{You appear to immolate yourself in a wreath of flame that lashes out at anyone who tries to harm you.}
\spellinfo{Abjur/Evoc (Energy, Shielding) [Fire or Cold]}{Arcane, Fire}
\spelldur{\durshort \dismissable}
\begin{spelltarget}{You}
    \spelleffect You gain cold damage reduction 20 \add 1 per caster level above 8th, and radiate light as a torch. In addition, creatures that attack you take damage.
\end{spelltarget}
\begin{spelltrigger}{Creature within \rngclose range attacks you with a melee weapon}
    \begin{spelltarget}*{Attacking creature}[Magic vs. Reflex]
        \spellsuccess 4d6 fire or cold damage \add d6 per four caster levels above 8th.
        \spellfailure Half damage.
    \end{spelltarget}
\end{spelltrigger}

\spellnotes The damage reduction allows the subject to ignore the first 20 energy damage it takes each round of the appropriate type.
\spellsr{Yes (Fortitude)}

\spellsection{Fire Storm}{8}
\spelldesc{You fill a massive area with sheets of roaring flame, burning everyone who opposes you.}
\spellinfo{Evoc (Energy) [Fire]}{Destruction, Fire, Nature, War}
\spelltwocol{\spellburst{\arealarge radius}}{\spellrng{\rngmed}}
\spellsr{Yes (Reflex)}
\begin{spelltargets}{Everything in the area, except allied creatures}l[Magic vs. Reflex]
    \spellsuccess 8d6 fire damage \add d6 per four caster levels above 16th.
    \spellfailure As above, but half damage.
\end{spelltargets}

\spellsection{Fireball}{3}
\spelldesc{You create an explosion of flame that detonates with a low roar, damaging nearby creatures and objects.}
\spellinfo{Evoc (Energy) [Destructive, Fire]}{Arcane, Fire}
\spelltwocol{\spellburst{\areasmall radius}}{\spellrng{\rngmed}}
\spellsr{Yes (Reflex)}
\begin{spelltargets}{Everything in the area}[Magic vs. Reflex]
    \spellsuccess 3d6 fire damage \add d6 per four caster levels above 6th.
    \spellfailure As above, but half damage.
\end{spelltargets}
\spellnotes \destructivespellnotes

\firespellnotes

\spellsection{Flame Blade}{2}
\spellinfo{Evoc (Energy) [Fire]}{Nature, Fire}
\spellrng{Touch}
\spelldur{\durlong \dismissable}
\spelleffect A 3 foot long beam of red-hot fire springs forth from your hand. In addition to providing illumination like a torch, you can wield this bladelike beam as a weapon. It is treated like a scimitar, except that all damage dealt with it is fire damage, you add half your casting attribute to damage in place of half your Strength, and it is treated as a light weapon, so you can use Dexterity to attack with it. Alternately, you can hurl flames from the weapon up to \rngmed range as if it were a thrown weapon.
\spellnotes \firespellnotes

Spell resistance applies when a foe is struck by the weapon, but not when the blade is created.
\spellsr{Yes (Fortitude)}

\spellsection{Flame Strike}{5}
\spelldesc{You call a vertical column of divine fire that roars downward, consuming your unworthy foes.}
\spellinfo{Evoc (Channeling, Energy) [Destructive, Fire]}{Destruction, Divine, Fire, War}
\spelltwocol{\spellburst{\areamed radius cylinder, 40 ft. high}}{\spellrng{\rngclose}}
\spelldmg{5d6 fire and divine damage \add d6 per four caster levels above 8th; see text}
\spellattack{Magic vs. Reflex}
\spelleffect Everything in the area takes damage. A failed attack deals half damage. Half the damage is fire damage, but the other half results directly from divine power. Your allies in the area take only the fire damage.
\spellnotes \destructivespellnotes

\firespellnotes
\spellsr{Yes (Fortitude)}

\spellsection{Fly}{4}
\spellinfo{Trans (Imbuement)}{Arcane}
\spelltwocol{\spelltgt{One creature}}{\spellrng{\rngtouch}}
\spelldur{\durshort}
\spelleffect The subject gains a 30 foot fly speed with good maneuverability.
\spellnotes An unencumbered creature with a fly speed can fly through the air. See \pcref{Flying}, for more details.
\spellsr{Yes (Fortitude)}

%priced as buff spell for range
\spellsection{Fog Cloud}{2}
\spelldesc{You conjure a bank of fog from a location you choose, concealing those inside.}
\spellinfo{Conj (Creation) [Fog]}{Arcane, Nature, Water}
\spelltwocol{\spellzone{\areamed radius cylinder}}{\spellrng{\rngmed}}
\spelldur{\durshort}
\spelleffect Fog blocks sight in the area, causing all creatures within or looking through the area to treat everything they see as if it had concealment (\plus4 to physical defenses). The cloud is stationary once created.
\spellnotes Fog spells do not function underwater and can be dispersed by wind. A moderate wind (11\add mph) disperses the fog in 5 rounds; a strong wind (21\add mph) disperses the fog in 1 round. Localized wind, such as from a \spell{gust of wind} spell, only disperses the fog where it overlaps the fog. A fire spell burns away the fog in the area into which it deals damage.

\spellsection{Forcecage}{7}
\spellinfo{Evoc (Control) [Force]}{Evoc}
\spellrng{\rngmed}
\spelldur{\durlong \dismissable}
\spellattack{Magic vs. Reflex}
\spelleffect This powerful spell brings into being an immobile, invisible cubical prison composed of either bars of force or solid walls of force (your choice). You make a magic attack vs. Reflex against any creature adjacent to the cage when it is formed. A failed attack allows the creature to disrupt the cage, preventing it from being formed in any squares adjacent to the creature. The rest of the cage is formed regardless.
\par Creatures in the area are caught and contained unless they are too big to fit inside, in which case the spell automatically fails. Teleportation and other forms of astral travel provide a means of escape, but the force walls or bars extend into the Ethereal Plane, blocking ethereal travel.
\par Like a \spell{wall of force} spell, a forcecage resists \spell{dispel magic}, but it is vulnerable to a \spell{disintegrate} spell, and it can be destroyed by a \magicitem{sphere of annihilation} or a \magicitem{rod of cancellation}.
\par \subspell{Barred Cage} This version of the spell produces a 20-foot cube made of bands of force (similar to a \spell{wall of force} spell) for bars. The bands are a half-inch wide, with half-inch gaps between them. Any creature capable of passing through such a small space can escape; others are confined. You can't attack a creature in a barred cage with a weapon unless the weapon can fit between the gaps. Even against such weapons (including arrows and similar ranged attacks), a creature in the barred cage has cover. All spells and breath weapons can pass through the gaps in the bars.
\par \subspell{Windowless Cell} This version of the spell produces a 10-foot cube with no way in and no way out. Solid walls of force form its six sides.

\spellsection{Foresee Probability}{2}
\spellinfo{Div (Knowledge)}{Div, Knowledge}
\spelltwocol{\spelltgt{One willing creature}}{\spellrng{\rngmed}}
\spelldur{\durshort}
\spelleffect When you cast this spell, you roll a d20 twice. Each time the subject would roll a d20, it instead uses a result you rolled for it. The die results are used in the same order you rolled them. Any die results unused when the spell duration expires are discarded.

At 10th caster level, and every 5 caster levels thereafter, you roll an additional die when you cast this spell.
\spellsr{Yes (Will)}

\spellsection{Foresight}{5}
\spellinfo{Div (Knowledge)}{Arcane, Knowledge, Protection}
\spelltwocol{\spelltgt{One creature}}{\spellrng{Touch}}
\spelldur{\durshort or \durlong; see text \dismissable}
\spelleffect The subject receives instantaneous warnings of impending danger or harm that would befall it. It gains an enhancement bonus to initiative checks equal to your caster level, and a \plus3 enhancement bonus to its dodge modifier and Reflex defense. This bonus increases to \plus4 at 14th caster level, and to \plus5 at 20th caster level.
\par If you cast this spell on yourself, it lasts for \durlong duration. On any other creature, it lasts for \durshort duration.

\spellsectioncomma{Foresight}{Greater}{9}
\spelldesc{You bestow a powerful sixth sense to your ally, giving them clear visions of any imminent danger.}
\spellinfo{Div (Knowledge)}{Arcane, Knowledge, Protection}
\spelleffect This spell functions like \spell{foresight}, except that the subject is also never surprised or unaware.

\spellsection{Forget}{1}
\spellinfo{Ench (Complusion)}{Chaos, Ench}
\spelltwocol{\spelltgt{One creature}}{\spellrng{\rngmed}}
\spelldur{\durlong}
\spellattack{Magic vs. Will}
\spelleffect The subject forgets something simple. You can't make it forget something important, such as its name. You must know what you want it to forget. The spell does not prevent the subject from learning the information again, and it can remember the information normally after the spell's duration.
\spellsr{Yes (Will)}

\spellsection{Freedom}{4}
\spellinfo{Trans (Imbuement)}{Divine, Nature, Travel}
\spelltwocol{\spelltgt{One creature}}{\spellrng{\rngtouch}}
\spelldur{\durshort}
\spelleffect The subject can move and attack normally for the duration of the spell, even under the influence of magic that usually impedes movement, such as paralysis, \spell{solid fog}, \spell{slow}, and \spell{web}. The subject gains a \plus20 enhancement bonus to Maneuver Class against grapple attacks, as well as on grapple attacks or Escape Artist checks made to escape a grapple or a pin.
\par The spell also allows the subject to move and attack with melee weapons normally while underwater.
\spellsr{Yes (Will)}

\spellsectioncomma{Freedom}{Mass}{8}
\spellinfo{Trans (Imbuement)}{Divine, Nature, Travel}
\spelltwocol{\spelllimit{\areamed radius}}{\spellrng{\rngmed}}
\spelltgts{Five creatures in the area}
\spelleffect This spell functions like \spell{freedom}, except that it affects multiple creatures.

\pdfbookmark[2]{G}{SpellDescriptionsG}
\begin{comment}
\subsubsection{G}
\end{comment}

\spellsection{Gaseous Form}{3}
\spellinfo{Trans (Polymorph)}{Arcane, Air, Travel}
\spellcmp{Somatic only}
\spelltwocol{\spelltgt{One willing corporeal creature}}{\spellrng{\rngtouch}}
\spelldur{\durshort \dismissable}
\spelleffect The subject and all its gear become insubstantial, misty, and translucent. Its armor modifier becomes 0, though other defense modifiers continue to apply normally. The subject gains physical damage reduction 10/magic and becomes immune to critical hits. It can't attack or cast spells with verbal, somatic, material, or focus components while in gaseous form. (This does not rule out the use of certain spells that the subject may have prepared using the feats Silent Spell or Still Spell.) If it has a touch spell ready to use, that spell is discharged harmlessly when the gaseous form spell takes effect.
\par A gaseous creature can fly at a speed of 10 feet (special maneuverability). It can pass through small holes or narrow openings, even mere cracks, with all it was wearing or holding in its hands, as long as the spell persists. The creature is subject to the effects of wind, and it can't enter water or other liquid. It also can't manipulate objects or activate items, even those carried along with its gaseous form. Continuously active items remain active, though in some cases their effects may be moot.
\spellnotes This spell's damage reduction allows the subject to ignore the first 10 physical damage it takes each round. If it is hit by a magical attack, such as a damaging spell or magic weapon, it cannot use its damage reduction for 1 round.

\spellsection{Gentle Descent}{2}
\spelldesc{You grant your ally ephemeral wings which allow him to glide.}
\spellinfo{Trans (Imbuement) [Air]}{Air, Nature}
\spelltwocol{\spelltgt{One creature}}{\spellrng{\rngmed}}
\spelldur{\durlong}
\spelleffect The subject gains a 30 foot glide speed.
\spellnotes A creature with a glide speed can glide through the air at the indicated speed. It must be unencumbered (see \pcref{Encumbrance}).

While in the air, a creature with a glide speed can control its fall as a move action. This allows it to move up to its speed horizontally in a direction of its choice while moving only five feet down. If it desires, it can move half as far horizontally and fall down twice as fast. It takes no falling damage if it touches the ground while gliding.
\spellsr{Yes (Will)}

\spellsection{Ghoul Touch}{1}
\spelldesc{Your foe feels the touch of a ghoul's undead hand against its flesh.}
\spellinfo{Necro (Flesh)}{Arcane}
\spelltwocol{\spelltgt{One living creature}}{\spellrng{\rngclose}}
\spelldur{1 round}
\spellattack{Magic vs. Fortitude}
\begin{spellhealthy}
    The subject is \sickened.
\end{spellhealthy}
\begin{spellblood}
    The subject is \nauseated.
\end{spellblood}
\spellsr{Yes (Fortitude)}

\spellsection{Giant Vermin}{4}
\spellinfo{Trans (Polymorph)}{Nature, Wild}
\spelltwocol{\spelllimit{\areamed radius}}{\spellrng{\rngclose}}
\spelltgts{Up to three vermin in the area}
\spelldur{\durmed}
\spelleffect You turn three normal-sized centipedes, two normal-sized spiders, or a single normal-sized scorpion into Large-sized forms. Only one type of vermin can be transmuted (so a single casting cannot affect both a centipede and a spider), and all must be grown to the same size.
\par Any giant vermin created by this spell do not attempt to harm you, but your control of such creatures is limited to simple commands (``Attack," ``Defend," ``Stop," and so forth). Orders to attack a certain creature when it appears or guard against a particular occurrence are too complex for the vermin to understand. Unless commanded to do otherwise, the giant vermin attack whoever or whatever is near them.
\spellsr{Yes (Fortitude)}

\spellsection{Glitterdust}{2}
\spellinfo{Conj (Creation)}{Arcane}
\spelltwocol{\spellburst{\areasmall radius}}{\spellrng{\rngmed}}
\spelldur{\durshort}
\spelleffect A cloud of golden particles covers everyone and everything in the area, visibly outlining invisible things for the duration of the spell. It likewise negates the effects of \spell{blur} and \spell{displacement}, and reveals illusory figments such as \spell{silent image} for what they are. All in the area at the time that the spell is cast are covered by the dust, which continues to sparkle until it fades.
\par Any creature covered by the dust takes a \minus40 penalty on Hide checks.
\spelleffect Water and similar substances can remove the dust.

\spellsection{Grasping Hand}{6}
\spellinfo{Evoc (Control) [Force]}{Arcane}
\spellrng{\rngmed}
\spelldur{\durshort \dismissable}
\spellsr{Yes (Fortitude)}
\spellline
\spelleffect This spell creates a hand, as \spell{interposing hand}, except that the hand grapples its target instead of protecting you from it.
\begin{spelltarget}*{One creature}l[Caster level \add casting attribute vs. Maneuver defense]
    \spellsuccess The target is grappled. The hand deals no damage.
\end{spelltarget}

\spellsection{Grease}{1}
\spelldesc{You conjure a layer of slippery grease on the ground, tripping up your foes.}
\spellinfo{Conj (Creation)}{Arcane}
\spelltwocol{\spellzone{\areasmall radius}}{\spellrng{\rngclose}}
\spelldur{\durshort \dismissable}
\spellattack{Magic vs. Reflex}
\spelleffect All creatures in the area fall prone. For the duration of the spell, the area is treated as difficult terrain, and moving through the area requires a DC 10 Acrobatics check to balance. Failure means a creature's movement is wasted, while failure by 5 or more means it falls.

\invocationsection{Greater (Spell Name)}
\par Any spell whose name begins with greater is alphabetized in this chapter according to the second word of the spell name. Thus, the description of a greater spell appears near the description of the spell on which it is based. Spell chains that have greater spells in them include those based on the spells command, dispel magic, invisibility, magic fang, magic weapon, restoration, scrying, shadow conjuration, shadow evocation, shout, and teleport.

\spellsection{Gust of Wind}{1}
\spelldesc{You create a severe blast of air that knocks your foes flying.}
\spellinfo{Evoc (Control) [Air]}{Air, Nature}
\spellzone{\arealarge line from you}
\spellattack{Special vs. Maneuver Defense}
\spelleffect You use a blast of wind to make a shove attack vs. Maneuver defense against all creatures in the area to push them away from you. Your attack bonus is equal to your caster level \add your casting attribute. The wind is capable of pushing enemies up to the end of the spell area. 
\par In addition to the effects noted, a gust of wind can do anything that a sudden blast of wind would be expected to do. It can extinguish open flames, create a stinging spray of sand or dust, fan a large fire, overturn delicate awnings or hangings, heel over a small boat, and blow gases or vapors to the edge of its range.
\spellnotes This spell can be made permanent with a \spell{permanency} ritual.

\pdfbookmark[2]{H}{SpellDescriptionsH}
\begin{comment}
\subsubsection{H}
\end{comment}

\spellsection{Harm}{6}
\spelldesc{You fill your foe with a massive influx of negative energy, crippling its body.}
\spellinfo{Necro (Vitalism) [Negative]}{Arcane, Death, Divine, Evil, Vitality}
\spelltwocol{\spelltgt{One creature}}{\spellrng{\rngtouch}}
\spelldmg{12d8 negative energy damage \add d8 per two caster levels above 12th}
\spellattack{Physical vs. Reflex, Magic vs. Fortitude}
\spelleffect If you touch the target, you make a magic attack vs. Fortitude to deal damage. A failed attack deals half damage. In addition, the target takes four points of Constitution damage. This damage is not halved on a failed attack.
\spellnotes If used on an undead creature, \spell{harm} acts like \spell{heal}.
\spellsr{Yes (Fortitude)}

\spellsection{Haste}{4}
\spelldesc{You accelerate your ally's motions, causing her to move and act more quickly than normal.}
\spellinfo{Trans (Temporal)}{Trans}
\spelltwocol{\spelltgts{One creature}}{\spellrng{Touch}}
\spelldur{\durshort}
\spelleffect The subject is hasted. It doubles all of its movement speeds (to a maximum of an additional 30 feet of movement), and can take an additional attack at a \minus5 penalty when it makes a standard attack. The increase to movement speed is considered an enhancement bonus.
\spellnotes The extra attack granted is not cumulative with similar effects, such as that provided by a weapon of speed, nor does it actually grant an extra action, so you can't use it to cast a second spell or otherwise take an extra action in the round.
\spellsr{Yes (Will)}

\spellsectioncomma{Haste}{Mass}{8}
\spelldesc{You accelerate your allies' motions, causing them to move and act more quickly than normal.}
\spellinfo{Trans (Temporal)}{Trans}
\spelltwocol{\spelllimit{\areamed radius}}{\spellrng{\rngmed}}
\spelltgts{Five creatures in the area}
\spelleffect This spell functions like \spell{haste}, except that it affects multiple creatures.

\spellsection{Heal}{6}
\spelldesc{You fill the subject with a massive influx of positive energy, restoring its body to its fullest.}
\spellinfo{Necro (Vitalism) [Positive]}{Divine, Good, Nature, Vitality}
\spelltwocol{\spelltgt{One creature}}{\spellrng{\rngtouch}}
\spellheal{12d8 \add d8 per two caster levels above 12th}
\spellattack{Physical vs. Reflex, Magic vs. Fortitude}
\spelleffect If you touch the target, it is healed. All of the following conditions are also removed from the target: ability damage, blinded, confused, dazed, dazzled, deafened, diseased, exhausted, fatigued, feebleminded, insanity, nauseated, sickened, stunned, and poisoned.

\par In addition, for every 10 points of healing granted by the spell, it can instead cure 1 point of critical damage.
\spellnotes \spell{Heal} does not remove negative levels, restore permanently drained levels, or restore permanently drained attribute points.
\par If used against an undead creature, \spell{heal} instead acts like \spell{harm}.
\spellsr{Yes (Fortitude)}

\spellsection{Heat Metal}{2}
\spelldesc{You heat your foe's armor, blistering its skin.}
\spellinfo{Evoc (Energy) [Fire]}{Nature}
\spelltwocol{\spelltgt{One metal object of up to Medium size}}{\spellrng{\rngmed}}
\spelldur{\durshort \dismissable}
\spelldmg{2d6 fire damage per round \add 1d6 per four caster levels above 4th}
\spellattack{Magic vs. Fortitude (object)}
\spelleffect At the end of each round, you make a magic attack vs. Fortitude to deal damage to the target object and any creature in contact with it. A failed attack deals half damage.
\spellnotes If the subject is underwater, this spell deals half damage,  and boils the surrounding water. Any cold intense enough to damage the creature negates fire damage from the spell (and vice versa) on a point-for-point basis. This spell can affect the armor worn by a typical Medium creature, but not generally by a larger creature.
\spellsr{Yes (Fortitude)}

\spellsection{Heroism}{3}
\spelldesc{You imbue your ally with great bravery and morale in battle.}
\spellinfo{Ench (Emotion) [Mind-Affecting, Morale]}{Arcane}
\spelltwocol{\spelltgt{One creature}}{\spellrng{\rngclose}}
\spelldur{\durshort \dismissable}
\spelleffect The subject gains a \plus2 enhancement bonus on physical attacks, all checks, and special defenses. \spellbonusscalingdescription
\spellsr{Yes (Will)}

\spellsectioncomma{Heroism}{Greater}{7}
\spellinfo{Ench (Emotion) [Mind-Affecting, Morale]}{Arcane}
\spelleffect This spell functions like \spell{heroism}, except the subject also gains 35 temporary hit points \add 1 per caster level above 14th. In addition, the subject is immune to hostile morale effects.

\spellsection{Hold Monster}{4}
\spellinfo{Ench (Inhibition) [Mind-Affecting]}{Arcane, Law}
\spelltwocol{\spelltgt{One living creature}}{\spellrng{\rngclose}}
\spelleffect This spell functions like \spell{hold person}, except that it is not limited by creature type.

\spellsectioncomma{Hold Monster}{Mass}{9}
\spellinfo{Ench (Inhibition) [Mind-Affecting]}{Arcane}
\spelltwocol{\spelllimit{\areamed radius}}{\spellrng{\rngclose}}
\spelltgts{Five creatures in the area}
\spelleffect This spell functions like \spell{hold monster}, except that it affects multiple creatures.

\spellsection{Hold Person}{2}
\spellinfo{Ench (Inhibition) [Mind-Affecting]}{Arcane, Divine, War}
\spelltwocol{\spelltgt{One humanoid creature}}{\spellrng{\rngclose}}
\spelldur{\durshort \dismissable; see text}
\spellattack{Magic vs. Will}
\begin{spellhealthy}
    The subject is \bewildered.
\end{spellhealthy}
\begin{spellblood}
    As the healthy effect, and the subject is paralyzed and unable to act. You must make a new attack each round on your turn. Failure means the subject is no longer paralyzed, though it is still bewildered.
\end{spellblood}
\spellsr{Yes (Will)}

\spellsectioncomma{Hold Person}{Mass}{8}
\spellinfo{Ench (Inhibition) [Mind-Affecting]}{Arcane, Divine}
\spelltwocol{\spelllimit{\areamed radius}}{\spellrng{\rngmed}}
\spelltgts{Five creatures in the area}
\spelleffect This spell functions like \spell{hold person}, except that it affects multiple creatures.

\spellsection{Holy Aura}{8}
\spellinfo{Abjur (Interdiction) [Good]}{Divine, Good}
\spellcmp{Verbal, Somatic, and Focus}
\spelllimit{\areamed radius centered on you}
\spelltgts{Five creatures in the area}
\spelldur{\durshort \dismissable}
\spelleffect A brilliant divine radiance surrounds the subjects, protecting them from attacks, granting them resistance to spells cast by evil creatures, and damaging evil creatures when they strike the subjects. This abjuration has three effects.
\par First, each subject gains a \plus5 enhancement bonus to its defenses.
\par Second, each subject gains spell resistance against evil spells and spells cast by evil creatures.
\par Third, at the end of each round, all evil creatures within \rngclose range of the subject that attacked the subject with their body or a melee weapon that round take 4d6 points of damage. A creature that attacks multiple creatures shielded by this spell can take this damage multiple times.
\spellfocus{A tiny reliquary containing some sacred relic. The reliquary costs at least 250 gp.}
\spellsr{Yes (Will)}

\spellsection{Holy Smite}{4}
\spellinfo{Evoc (Channeling) [Good]}{Good}
\spelltwocol{\spelltgt{One creature}}{\spellrng{\rngmed}}
\spelldur{Instantaneous/5 rounds}
\spelldmg{8d6 divine damage \add d6 per two caster levels above 8th}
\spellattack{None/Will half}
\spelleffect If the target is not good, it is bewildered for 5 rounds, and you make a Will attack to deal damage to it. A failed attack deals half damage.
\spellsr{Yes (Will)}

\spellsection{Holy Word}{7}
\spellinfo{Evoc (Channeling) [Good]}{Good, Divine}
\spellcmp{Verbal only}
\spellburst{\arealarge radius centered on you}
\spelltgts{All nongood creatures in the area}
\spelleffect If the target's level does not exceed your caster level, it is \deafened for 5 rounds.

If it is also bloodied, it also suffers one or more of the following ill effects, depending on its level.
\begin{dtable}
    \begin{tabularx}{\columnwidth}{l >{\lcol}X}
        \par \thead{Level} & \thead{Effect} \\
        \par Equal to caster level & Deafened \\
        \par Up to caster level \minus5 & Blinded, deafened \\
        \par Up to caster level \minus10 & Paralyzed, blinded, deafened \\
        \par Up to caster level \minus15 & Killed\fn{1}
    \end{tabularx}
    1 Living creatures die. Nonliving creatures are destroyed.
\end{dtable}
\par \subspell{Deafened} The creature is deafened for 5 rounds.
\par \subspell{Blinded} The creature is blinded for 2 rounds.
\par \subspell{Paralyzed} The creature is paralyzed and helpless for 5 rounds.
\par \subspell{Killed} Living creatures die. Nonliving creatures are destroyed.
\spellsr{Yes (Will)}

\spellsection{Horrid Wilting}{8}
\spelldesc{You dessicate your foes from a great distance, shriveling their bodies.}
\spellinfo{Necro (Flesh)}{Necro, Water}
\spelltwocol{\spelllimit{\arealarge radius}}{\spellrng{\rngfar}}
\spelltgts{Ten living creatures in the area}
\spelldmg{8d6 physical damage \add d6 per four caster levels above 16th}
\spellattack{Fortitude half}
\spelleffect You make a Fortitude attack to deal damage to each affected creature. A failed attack deals half damage. You gain a \plus5 bonus to attack plants and creatures with the water subtype. 
\spellsr{Yes (Fortitude)}

\spellsection{Hypnotic Pattern}{3}
\spelldesc{You create a twisting pattern of subtle, shifting colors that weaves through the air, fascinating creatures within it.}
\spellinfo{Ench/Illus (Compulsion, Figment) [Light, Mind-Affecting]}{Arcane}
\spelltwocol{\spellzone{\areasmall radius}}{\spellrng{\rngmed}}
\spelldur{\durshort}
\spellattack{Magic vs. Will}
\spelleffect Creatures within the spell's area are fascinated. Each fascinated creature stands or sits quietly, taking no actions other than to pay attention to the fascinating effect for the duration of the spell. It takes a \minus4 penalty on skill checks made as reactions, such as Listen and Spot checks. Any obvious threat, such as noticing someone draw a weapon, cast a spell, or aim a ranged weapon at the fascinated creature, automatically breaks the effect. A fascinated creature's ally may shake it free of the spell as a standard action.
\spellnotes Creatures who cannot see the lights are not affected by this spell.

\pdfbookmark[2]{I}{SpellDescriptionsI}
\begin{comment}
\subsubsection{I}
\end{comment}
\spellsr{Yes (Will)}

\spellsection{Ice Storm}{4}
\spelldesc{You conjure magical hailstones that pound down, smashing and chilling creatures in their path.}
\spellinfo{Conj/Evoc (Creation, Energy) [Cold]}{Arcane, Destruction, Nature, Water}
\spelltwocol{\spellburst{\areasmall radius cylinder, 20 ft. high}}{\spellrng{\rngmed}}
\spelldur{Instantaneous/1 round}
\spelldmg{4d4 cold and bludgeoning damage \add d4 per four caster levels above 8th}
\spelleffect You deal damage to everything in the area. The area is difficult terrain for 1 round.
\spellsr{Yes (Reflex)}

\spellsection{Implosion}{9}
\spelldesc{You create a destructive responance in your foe's body that destroys it from the inside out.}
\spellinfo{Evoc (Control)}{Destruction, Divine}
\spelltwocol{\spelltgts{One corporeal creature/round}}{\spellrng{\rngclose}}
\spelldur{Instantaneous and concentration (up to 5 rounds); see text}
\spellattack{None/Magic vs. Fortitude}
\begin{spellhealthy}
    The target creature you concentrate on is staggered for 5 rounds. It can take a move action or a standard action each round, but not both.
\end{spellhealthy}
\begin{spellblood}
    The target is instantly slain.
\end{spellblood}
\spellnotes You can concentrate on one creature per round. You can target a particular creature only once with each casting of the spell.
\par \spell{Implosion} has no effect on creatures in \spell{gaseous form} or on incorporeal creatures.
\spellsr{Yes (Fortitude)}

\spellsection{Imprisonment}{9}
\spellinfo{Conj/Trans (Time, Translocation) [Teleportation]}{Arcane, Earth, Law}
\spelltwocol{\spelltgt{One creature}}{\spellrng{\rngclose}}
\spelldur{See text}
\spelldmg{18d8 physical damage \add d8 per two caster levels above 18th}
\spellattack{Magic vs. Will}
\spelleffect The target takes damage as its body is partially teleported away, and it is slowed for 5 rounds. This damage ignores hardness and damage reduction.
\begin{spellblood}
    If the creature is touching the ground, it becomes permanently entombed in a state of suspended animation (as the \spell{temporal stasis} spell) in a small sphere far beneath the surface of the earth. It remains there unless an \spell{emancipation} spell is cast at the locale where the imprisonment took place.
\end{spellblood}
\spellnotes A slowed creature can take only a single move action or standard action each turn, but not both. It cannot take full-round actions, but it may take swift actions. Additionally, it takes a \minus2 penalty to physical attacks and defenses, as well as Strength and Dexterity-based checks.

The subject must be \bloodied at the time that the spell is cast to be imprisoned. Magical search by a crystal ball, a \spell{locate creature} spell, or some other similar divination does not reveal the fact that a creature is imprisoned, but \spell{discern location} does. A \spell{wish} or \spell{miracle} spell will not free the recipient, but will reveal where it is entombed.
\spellsr{Yes (Will)}

\spellsection{Inertial Shield}{2}
\spelldesc{You create a barrier around your ally that resists physical intrusion.}
\spellinfo{Abjur (Shielding)}{Arcane}
\spelltwocol{\spelltgt{One creature}}{\spellrng{\rngclose}}
\spelldur{\durshort}
\spelleffect The subject gains physical damage reduction 4/force. This damage reduction increases by 1 per two caster levels above 4th.
\spellnotes This spell's damage reduction allows the subject to ignore the first 4 physical damage it takes each round. If it is hit by a attack that deals force damage, such as \spell{magic missile}, it cannot use its damage reduction for 1 round.
\spellsr{Yes (Will)}

\spellsection{Inflict Critical Wounds}{4}
\spellinfo{Necro (Vitalism) [Negative]}{Arcane, Divine}
\spelldmg{8d6 negative energy damage \add d6 per two caster levels above 8th}
\spelleffect This spell functions like \spell{inflict light wounds}, except that for every 10 points of damage dealt in excess of the subject's hit points, it can instead inflict 1 point of critical damage.
\spellnotes This effect can cause a creature to begin dying without being disabled first.

\spellsectioncomma{Inflict Critical Wounds}{Mass}{8}
\spellinfo{Necro (Vitalism) [Negative]}{Arcane, Divine}
\spelldmg{8d6 negative energy damage \add d6 per four caster levels above 10th}
\spelleffect This spell functions like \spell{mass inflict light wounds}, except that for every 10 points of damage dealt in excess of the subject's hit points, it can instead inflict 1 point of critical damage.
\spellnotes This effect can cause a creature to begin dying without being disabled first.

\spellsection{Inflict Light Wounds}{1}
\spellinfo{Necro (Vitalism) [Negative]}{Arcane, Divine}
\spelltwocol{\spelltgt{One creature}}{\spellrng{\rngclose}}
\spelldmg{2d6 negative energy damage \add d6 per two caster levels above 2nd}
\spellattack{Fortitude half}
\spelleffect You make a Fortiude attack to deal damage to the target. A failed attack deals half damage. Since undead are powered by negative energy, this spell heals them instead of dealing damage.
\spellsr{Yes (Fortitude)}

\spellsectioncomma{Inflict Light Wounds}{Mass}{5}
\spellinfo{Necro (Vitalism) [Negative]}{Arcane, Divine}
\spelltwocol{\spelllimit{\areamed radius}}{\spellrng{\rngclose}}
\spelltgts{Five creatures in the area}
\spelldmg{5d6 negative energy damage \add d6 per four caster levels above 10th}
\spellattack{Fortitude half}
\spelleffect You make a Fortitude attack to deal damage to each target. A failed attack deals half damage. Like other \spellindirect{inflict light wounds}{inflict} spells, \spell{mass inflict light wounds} heals undead instead of dealing damage.
\spellsr{Yes (Fortitude)}

\spellsection{Inflict Moderate Wounds}{2}
\spellinfo{Necro (Vitalism) [Negative]}{Arcane, Divine}
\spelldmg{4d6 negative energy damage \add d6 per two caster levels above 4th}
\spelleffect This spell functions like \spell{inflict light wounds}, except that for every 20 points of damage dealt in excess of the subject's hit points, it can instead inflict 1 point of critical damage.
\spellnotes This effect can cause a creature to begin dying without being disabled first.

\spellsectioncomma{Inflict Moderate Wounds}{Mass}{6}
\spellinfo{Necro (Vitalism) [Negative]}{Arcane, Divine}
\spelldmg{6d6 negative energy damage \add d6 per four caster levels above 10th}
\spelleffect This spell functions like \spell{mass inflict light wounds}, except that for every 20 points of damage dealt in excess of the subject's hit points, it can instead inflict 1 point of critical damage.
\spellnotes This effect can cause a creature to begin dying without being disabled first.

\spellsection{Inflict Serious Wounds}{3}
\spellinfo{Necro (Vitalism) [Negative]}{Arcane, Divine}
\spelldmg{6d6 negative energy damage \add d6 per two caster levels above 6th}
\spelleffect This spell functions like \spell{inflict light wounds}, except that for every 15 points of damage dealt in excess of the subject's hit points, it can instead inflict 1 point of critical damage.
\spellnotes This effect can cause a creature to begin dying without being disabled first.

\spellsectioncomma{Inflict Serious Wounds}{Mass}{7}
\spellinfo{Necro (Vitalism) [Negative]}{Arcane, Divine}
\spelldmg{7d6 negative energy damage \add d6 per four caster levels above 10th}
\spelleffect This spell functions like \spell{mass inflict light wounds}, except that for every 15 points of damage dealt in excess of the subject's hit points, it can instead inflict 1 point of critical damage.
\spellnotes This effect can cause a creature to begin dying without being disabled first.

\spellsection{Insanity}{6}
\spellinfo{Ench (Compulsion) [Mind-Affecting]}{Chaos, Ench}
\spelltwocol{\spelltgt{One living creature}}{\spellrng{Touch}}
\spelldur{Permanent}
\spellattack{Magic vs. Will}
\begin{spellhealthy}
    The creature is bewildered, making it \vulnerable.
\end{spellhealthy}
\begin{spellblood}
    \par The affected creature is confused (see the \spell{confusion} spell).
\end{spellblood}
\spellnotes \spell{Remove curse} and \spell{dispel magic} do not remove \spell{insanity}. \spell{Greater restoration}, \spell{heal}, \spell{limited wish}, \spell{miracle}, or \spell{wish} can restore the creature.
\spellsr{Yes (Will)}

\spellsection{Interposing Hand}{2}
\spelldesc{You create a massive hand from thin air that blocks your foe's attacks.}
\spellinfo{Evoc (Control) [Force]}{Arcane}
\spellrng{\rngmed}
\spelldur{\durshort \dismissable}
\spellsr{Yes (Fortitude)}
\spellline
\spelleffect This spell creates a floating, disembodied hand made of magical force. The hand is 10 feet long and about that wide with its fingers outstretched. It has 3 hit points per caster level, an Armor defense of 10 \add half your caster level, and a Maneuver defense of 10 \add your caster level \add your casting attribute. Most effects that don't affect objects do not affect the hand.

Each round, as a swift action, you can direct the hand to protect you from a target. If you do not direct the hand, it remains motionless.
\begin{spelltarget}*{One creature}
    \spelleffect The hand provides you with active cover from the target. Each physical attack the target makes against you has a 20\% chance to strike the hand instead. In addition, if the target is Large size or smaller, it moves at half speed while moving towards you.
\end{spelltarget}
\spellnotes The hand can move up to 60 feet per round. It never provokes attacks of opportunity from opponents. Since the hand is directed by you, its ability to interact with invisible or concealed creatures is no better than yours. Its special defenses are the same as your special defenses.

If the hand goes out of range of you, it winks out. It cannot push through a \spell{wall of force} or enter an \spell{antimagic field}, but it suffers the full effect of a \spell{prismatic wall} or \spell{prismatic sphere}. \spell{Disintegrate} or a successful \spell{dispel magic} destroys it.

\spellsection{Invest Magic}{4}
\spellinfo{Trans (Augment)}{Arcane, Divine, War}
\spelltwocol{\spelltgt{One creature}}{\spellrng{\rngclose}}
\spelldur{\durshort}
\spelleffect All weapons and armor that the subject wields gain a \plus3 enhancement bonus for as long as she wields them. This bonus increases to \plus4 at 14th caster level, and to \plus5 at 20th caster level.
\spellsr{Yes (Will)}

\spellsection{Invisibility}{3}
\spellinfo{Illus (Glamer)}{Arcane, Trickery}
\spelltwocol{\spelltgt{A creature or object weighing no more than 100 lb./caster level}}{\spellrng{\rngclose}}
\spelldur{\durshort \dismissable}
\spellattack{Magic vs. Will (object)}
\spellsuccess The creature or object touched becomes invisible, vanishing from sight, even from darkvision. If the recipient is a creature carrying gear, that vanishes, too.
\par Items dropped or put down by an invisible creature become visible; items picked up disappear if tucked into the clothing or pouches worn by the creature. Light, however, never becomes invisible, although a source of light can become so (thus, the effect is that of a light with no visible source). Any part of an item that the subject carries but that extends more than 5 feet from it becomes visible.
\par Of course, the subject is not magically silenced, and certain other conditions can render the recipient detectable (such as stepping in a puddle). The spell ends if the subject attacks any creature. For purposes of this spell, an attack includes any spell targeting a foe or whose area or effect includes a foe. (Exactly who is a foe depends on the invisible character's perceptions.) Actions directed at unattended objects do not break the spell. Causing harm indirectly is not an attack. Thus, an invisible being can open doors, talk, eat, climb stairs, summon monsters and have them attack, cut the ropes holding a rope bridge while enemies are on the bridge, remotely trigger traps, open a portcullis to release attack dogs, and so forth. If the subject attacks directly, however, it immediately becomes visible along with all its gear. Spells such as \spell{bless} that specifically affect allies but not foes are not attacks for this purpose, even when they include foes in their area.
\spellnotes This spell can be made permanent (on objects only) with a \spell{permanency} ritual.
\spellsr{Yes (Will)}

\spellsectioncomma{Invisibility}{Greater}{6}
\spellinfo{Illus (Glamer)}{Illus}
\spelleffect This spell functions like \spell{invisibility}, except that the subject becomes invisible again at the start of each of its turns, even if it attacked a creature during its previous turn.

\spellsectioncomma{Invisibility}{Mass}{7}
\spellinfo{Illus (Glamer)}{Arcane, Trickery}
\spelltwocol{\spelllimit{\areamed radius}}{\spellrng{\rngmed}}
\spelltgts{Five creatures or objects weighing no more than 100 lb./caster level in the area}
\spelleffect This spell functions like \spell{invisibility}, except that it affects multiple creatures. If the effct is broken for one creature, the other subjects remain invisible.

\spellsection{Invisibility Sphere}{5}
\spellinfo{Illus (Glamer)}{Arcane}
\spellemanation{\areasmall radius centered on the creature or object touched}
\spelleffect This spell functions like \spell{invisibility}, except that this spell confers invisibility upon all creatures within a \areasmall radius emanation of the recipient. The center of the effect is mobile with the recipient.
\par Those affected by this spell can see each other and themselves as if unaffected by the spell. Any affected creature moving out of the area becomes visible, but creatures moving into the area after the spell is cast do not become invisible. Affected creatures (other than the recipient) who attack negate the invisibility only for themselves. If the spell recipient attacks, the \spell{invisibility sphere} ends.

\spellsection{Iron Body}{8}
\spellinfo{Trans (Polymorph)}{Arcane, Earth, Strength}
\spelltwocol{\spelltgt{You}}{\spellrng{\rngpers}}
\spelldur{\durshort \dismissable}
\spelleffect This spell transforms your body into living iron, which grants you several powerful resistances and abilities.
\par You gain physical damage reduction 15/adamantine. You are immune to blindness, critical hits, attribute damage, deafness, disease, drowning, electricity, poison, stunning, and all spells or attacks that affect your physiology or respiration, because you have no physiology or respiration while this spell is in effect. You take only half damage from acid and fire of all kinds.
\par You gain a \plus5 enhancement bonus to your Strength score, but you take a \minus5 penalty to Dexterity as well, and your speed is reduced to half normal. You have a \minus8 armor check penalty. You cannot drink (and thus can't use potions) or play wind instruments.
\par Your unarmed attacks deal damage equal to a warhammer sized for you (1d6 for Small characters or 1d8 for Medium characters), and you are considered armed when making unarmed attacks.
\par Your weight increases by a factor of ten, causing you to sink in water like a stone. However, you could survive the crushing pressure and lack of air at the bottom of the ocean -- at least until the spell duration expires.
\spellnotes This spell's damage reduction allows the subject to ignore the first 15 physical damage it takes each round. If it is hit by an adamantine weapon, it cannot use its damage reduction for 1 round.

\spellsection{Irresistible Dance}{9}
\spelldesc{You fill your enemy with an overpowering urge to dance and caper in place. Against its will, it begins doing so, complete with foot shuffling and tapping.}
\spellinfo{Ench (Compulsion) [Mind-Affecting]}{Arcane, Chaos}
\spelltwocol{\spelltgt{One creature}}{\spellrng{\rngclose}}
\spelldur{1d4 rounds}
\spelleffect The subject is defenseless and must spend a standard action each round to do nothing but dance, which provokes attacks of opportunity.

\pdfbookmark[2]{J-L}{SpellDescriptionsJ-L}
\begin{comment}
\subsubsection{J-L}
\end{comment}
\spellsr{Yes (Will)}

\spellsection{Knock}{2}
\spellinfo{Evoc (Control)}{Evoc}
\spelltwocol{\spelltgt{One Medium or smaller object}}{\spellrng{\rngclose}}
\spelldur{Instantaneous; see text}
\spelleffect The knock spell telekinetically opens stuck, barred, locked, held, or arcane locked objects. If the object is stuck or held, you can immediately make an Strength check to break it open, using your caster level instead of your Strength. Others can aid you on this check as normal. In addition, if the object is locked, you can immediately make a Disable Device check to open the lock as if you had rolled a 20 on the check. You get an enhancement bonus on the Disable Device check equal to half your caster level.
\spellnotes If knock is cast on an \spellindirect{arcane lock}{arcane locked} door, make a caster level check against a DC of 11 \add the caster level of the \spell{arcane lock}. If you succeed, the \spell{arcane lock} is suppressed for 10 minutes. If you fail, you may still bypass the door with the checks above, if possible.

\invocationsection{Lesser (Spell Name)}
\par Any spell whose name begins with lesser is alphabetized in this chapter according to the second word of the spell name. Thus, the description of a lesser spell appears near the description of the spell on which it is based. Spell chains that have lesser spells in them include those based on the spells cone of cold, dispel magic, moment of prescience, precognition, and spelltheft.

\spellsection{Levitate}{3}
\spellinfo{Evoc (Control)}{Arcane}
\spelltwocol{\spelltgt{You or one willing creature or one object (total weight up to 100 lb./caster level)}}{\spellrng{\rngclose}}
\spelldur{\durshort \dismissable}
\spelleffect This spell allows you to telekinetically move the subject up and down as you wish. A creature must be willing to be levitated, and an object must be unattended or possessed by a willing creature. As a swift action, you can mentally direct the subject to move up or down as much as 20 feet each round. You cannot move the recipient horizontally, but the recipient could clamber along the face of a cliff, for example, or push against a ceiling to move laterally (generally at half its base land speed).
\spellsr{Yes (Will)}

\spellsection{Lifeseeking Missile}{3}
\spellinfo{Evoc/Necro (Control, Life) [Force]}{Arcane}
\spellrng{\rngmed}
\spelldmg{3d10 force damage \add d10 per four caster levels above 6th}
\spelleffect This spell functions like \spell{magic missile}, except that the spell creates three missiles that automatically seek out living creatures in the area. Each missile deals 1d10 force damage. If you specify a target for a missile, it will strike the target. Otherwise, it will strike a living creature in the area.

\spell{Invisibility}, \spell{displacement}, and any other forms of cover or concealment do not fool the missiles. You can form one additional missile per four caster levels above 6th.

\spellsection{Lightning Bolt}{3}
\spellinfo{Evoc (Energy) [Destructive, Electricity]}{Arcane, Destruction, Nature}
\spellburst{\arealarge line, 10 ft. wide}
\spelldmg{3d6 electricity damage \add d6 per four caster levels above 6th}
\spellattack{Magic vs. Reflex}
\spelleffect Everything in the area takes damage. A failed attack deals half damage.
\spellnotes \destructivespellnotes
\spellsr{Yes (Reflex)}

\spellsection{Limited Wish}{7}
\spellinfo{Universal}{Arcane}
\spellcmp{Verbal, Somatic, and Material}
\spelltwocol{\spelltgteffarea{See text}}{\spellrng{See text}}
\spelldur{See text}
\spellattack{None; see text}
\spelleffect A limited wish lets you create nearly any type of effect. For example, a limited wish can do any of the following things.
\begin{itemize}
    \item Duplicate any general sorcerer/wizard spell of 6th level or lower, provided the spell is not of a school prohibited to you.
    \item Duplicate any general sorcerer/wizard spell of 5th level or lower, even if it's of a prohibited school.
    \item Duplicate any other spell of 4th level or lower, provided the spell is not of a school prohibited to you.
    \item Duplicate any other spell of 3rd level or lower, even if it's of a prohibited school.
    \item Undo the harmful effects of many spells, such as geas/quest or insanity.
    \item Produce any other effect whose power level is in line with the above effects, such as a single creature automatically hitting on its next attack or taking a \minus5 penalty to its defenses for 5 rounds.
\end{itemize}
\par When casting a limited wish, you do not specify the exact spell or effect you wish to duplicate. Instead, you make a wish, describing what you want to have happen, and make a DC 15 Wisdom check. If the check fails, your intent is redirected or perverted in some way. For example, a \spell{limited wish} to turn a foe to stone would normally mimic the \spell{flesh to stone} effect of the \spell{transmute flesh and stone} spell. However, if the Wisdom check failed, your foe might gain the benefit of a \spell{stoneskin} spell instead.
\par When a limited wish spell duplicates a spell with a material component that costs more than 1,000 gp, you must provide that component (in addition to the 1,000 gp cost for this spell).
\spellmat{A diamond worth no less than 1,000 gp (see above).}
\spellsr{Yes (Will)}

\spellsection{Link Vitality}{3}
\spellinfo{Necro (Life)}{Necro}
\spelllimit{\areamed radius centered on you}
\spelltgts{Any two living creatures in the area}
\spelldur{\durshort}
\spellattack{Magic vs. Will}
\spelleffect If your attack succeeds against both subjects, they become magically linked. If either linked creature experiences pain, both feel it. When one gains or loses hit points, the other gains or loses the same amount.
\spellnotes The loss of hit points caused by this spell is not damage, and is not affected by damage reduction or other abilities which affect damage.
\spellsr{Yes (Will)}

\spellsectioncomma{Link Vitality}{Mass}{7}
\spellinfo{Necro (Life)}{Arcane}
\spelltgts{Five living creatures in the area}
\spelleffect This spell functions as \spell{link vitality}, except that it affects many creatures. The spell links all creatures your Will attack succeeds against. If any of the linked creatures lose or gain hit points, all linked creatures lose or gain the same amount, and so on.

\spellsection{Living Projectile}{3}
\spellinfo{Abjur/Evoc (Control, Shielding)}{Arcane}
\spelltwocol{\spelltgts{One willing creature and one creature}}{\spellrng{\rngclose}}
\spelldur{Instantaneous and 1 round}
\spelleffect You telekinetically fling a willing creature at great speed into a foe. The willing subject is moved adjacent to the foe attacked, gains the benefit of the \spell{ablate impact} spell, and is knocked prone. If the attack hits a different target than intended, such as because of active cover, the ally moves adjacent to that target instead.
\spellattack{Caster level \add casting attribute vs. Armor defense}
\spellsuccess 6d6 damage \add d6 per two caster levels above 6th. The flung creature takes half the damage dealt.
\spellfailure Half damage to the foe. The damage taken by the flung creature is not halved.

\spellsection{Locate Entity}{6}
\spellinfo{Div (Awareness) [Detection]}{Arcane, Knowledge}
\spellrng{\rngext}
\spelldur{\durlong \dismissable}
\spelleffect This spell functions as \spell{locate object}, except that it can also detect creatures, as \spell{locate creature}. When you cast this spell, you choose to locate an object or creature, following the restrictions stated in the respective location spells.

\spellsection{Locate Creature}{2}
\spellinfo{Div (Awareness) [Detection]}{Arcane, Divine}
\spellrng{\rngmed}
\spelldur{\durmed \dismissable}
\spelleffect You sense the direction of a well-known or clearly visualized creature if it is within the spell's range. You can search for general creatures based on visual characteristics (such as ``pointy ears'' or ``looking human''), in which case you locate the nearest one of its kind if more than one is within the range. Attempting to find a certain creature requires a specific and accurate mental image of a distinguishing visual characteristic, such as its clothes or face; if the image is not close enough to the actual creature, the spell fails.
\spellnotes A detection spell can penetrate barriers, but 1 foot of stone, 1 inch of common metal, a thin sheet of lead, or 3 feet of wood or dirt blocks it.

\spellsectioncomma{Locate Creature}{Greater}{4}
\spellinfo{Div (Awareness) [Detection]}{Arcane, Divine, Knowledge}
\spellrng{\rngext}
\spelleffect This spell functions like \spell{locate creature}, except that it detects creatures within \rngext range. In addition, you detect all appropriate creatures within the range, rather than only the nearest creature.

\spellsection{Locate Object}{1}
\spellinfo{Div (Awareness) [Detection]}{Arcane, Divine}
\spellrng{\rngmed}
\spelldur{\durmed \dismissable}
\spelleffect You sense the direction of a well-known or clearly visualized object if it is within the spell's range. You can search for general items, in which case you locate the nearest one of its kind if more than one is within the range. Attempting to find a certain item requires a specific and accurate mental image; if the image is not close enough to the actual object, the spell fails.
\spellnotes A detection spell can penetrate barriers, but 1 foot of stone, 1 inch of common metal, a thin sheet of lead, or 3 feet of wood or dirt blocks it.

\spellsectioncomma{Locate Object}{Greater}{3}
\spellinfo{Div (Awareness) [Detection]}{Arcane, Divine, Knowledge}
\spellrng{\rngext}
\spelleffect This spell functions like \spell{locate object}, except that it detects objects within \rngext range. In addition, you detect all appropriate objects within the range, rather than only the nearest object. 

\spellsection{Longstrider}{1}
\spellinfo{Trans (Augment)}{Nature, Travel}
\spelltwocol{\spelltgt{You}}{\spellrng{\rngpers}}
\spelldur{\durlong \dismissable}
\spelleffect This spell increases your base land speed by 10 feet. (This adjustment counts as an enhancement bonus.) It has no effect on other modes of movement, such as burrow, climb, fly, or swim.

\pdfbookmark[2]{M}{SpellDescriptionsM}
\begin{comment}
\subsubsection{M}
\end{comment}

\spellsection{Mage Armor}{1}
\spelldesc{You create an invisible but tangible field of force that surrounds you, protecting you from attacks.}
\spellinfo{Abjur (Shielding) [Force]}{Arcane}
\spelltwocol{\spelltgt{One creature}}{\spellrng{\rngclose}}
\spelldur{\durshort or \durlong; see text \dismissable}
\spelleffect As you cast this spell, you choose whether to create body armor or a shield. If you choose body armor, the subject gains a \plus2 armor modifier. If you choose a shield, the subject gains a \plus2 shield modifier. The bonus granted increases to \plus3 at 8th caster level, to \plus4 at 14th caster level, and finally to \plus5 at 20th caster level. 
\par Unlike mundane armor, \spell{mage armor} entails no armor check penalty, arcane spell failure chance, or speed reduction.
\par If you cast this spell on yourself, it lasts for \durlong duration. On any other creature, it lasts for \durshort duration.
\spellnotes If you cast this spell on the same creature twice, you can grant the creature both body armor and a shield. The armor created by this spell is treated as a separate piece or armor from any other armor the creature is wearing, so it does not stack with any existing bonuses. Since \spell{mage armor} is made of force, incorporeal creatures can't bypass it the way they do normal armor.
\spellsr{Yes (Will)}

\spellsection{Mage Hand}{1}
\spellinfo{Evoc (Control)}{Arcane}
\spelltwocol{\spelltgt{One nonmagical, unattended object weighing up to 5 lb.}}{\spellrng{\rngclose}}
\spelldur{\durshort}
\spelleffect You point your finger at an object and can lift it and move it in any direction from a distance. By directing the spell as a swift action, you can propel the object as far as 15 feet in any direction each round, though the spell ends if the distance between you and the object ever exceeds the spell's range.
\spellnotes Fine manipulation, including any motion other than simply moving the object in a particular direction, is not possible with this spell.

\spellsection{Mage's Disjunction}{9}
\spellinfo{Abjur (Negation) [Magic]}{Arcane, Magic}
\spelltwocol{\spelltgtorarea{One magic item or \areamed radius burst}}{\spellrng{\rngmed}}
\spellattack{Magic vs. Will (object)}
\spelleffect All magical effects within the radius of the spell, except for those on you, are disjoined. That is, spells and spell-like effects are separated into their individual components (ending the effect as a \spell{dispel magic} spell does).
\par You also have a 2\% chance per caster level of destroying an \spell{antimagic field}.
\par You can also use this spell to target a single item. If you succeed at a Will attack with a \plus5 bonus, the item is permanently rendered nonmagical. Even artifacts are subject to this use of disjunction, though there is only a 1\% chance per caster level of actually affecting such powerful items. Additionally, if an artifact is destroyed, you permanently lose the ability to cast \spell{mage's disjunction}. (This ability cannot be recovered by mortal magic, not even \spell{miracle} or \spell{wish}.)
\par \subspell{Note} Destroying artifacts is a dangerous business, and it is 95\% likely to attract the attention of some powerful being who has an interest in or connection with the device.

\spellsection{Magic Circle against Alignment}{5}
\spellinfo{Abjur (Interdiction) [Barrier, Good]}{Arcane, Chaos, Divine, Evil, Good, Law}
\spelltwocol{\spellemanation{\areasmall radius centered on touched creature}}{\spellrng{Touch}}
\spelldur{\durshort \dismissable}
\spelleffect All creatures in the area gain the effects of a \spell{protection from alignment} spell, using an alignment of your choice. In addition, no summoned creatures can enter the area except summoned creatures of the chosen alignment.
\spellsr{Yes (Will)}

\spellsection{Magic Fang}{2}
\spellinfo{Trans (Augment)}{Nature}
\spelltwocol{\spelltgt{One creature}}{\spellrng{\rngclose}}
\spelldur{\durshort}
\spelleffect This spell makes one of the subject's natural weapons a \plus2 magic weapon, granting a \plus2 enhancement bonus to attack and damage rolls. \spellbonusscalingdescription
\spellnotes The spell does not change an unarmed strike's damage from nonlethal damage to lethal damage. This spell can be made permanent with a \spell{permanency} ritual.
\spellsr{Yes (Will)}

\spellsectioncomma{Magic Fang}{Greater}{4}
\spellinfo{Trans (Augment)}{Nature}
\spelleffect This spell functions like \spell{magic fang}, except that it affects one of the creature's natural weapons per four caster levels.
\spellnotes This spell can be made permanent with a \spell{permanency} ritual.

\spellsection{Magic Missile}{1}
\spellinfo{Evoc (Control) [Force]}{Arcane}
\spelltwocol{\spelllimit{\areamed radius}}{\spellrng{\rngclose}}
\spelltgts{Creatures in the area}
\spelldmg{2d4 force damage \add d4 per two caster levels above 2nd; see text}
\spelleffect Two missiles of magical energy dart forth from your fingertip and strike creatures you designate in the area, dealing 1d4 damage each. A single missile can strike only one creature. For every two caster levels above 2nd, you gain an additional missile.
The missiles strike unerringly, even if the target has cover or concealment. Specific parts of a creature can't be singled out. Inanimate objects are not damaged by the spell. You must designate targets before you check for spell resistance or roll damage.
\spellsr{Yes (Fortitude)}

\spellsection{Magic Vestment}{1}
\spellinfo{Trans (Augment)}{Arcane, Divine}
\spelltwocol{\spelltgt{One suit of armor or shield}}{\spellrng{\rngclose}}
\spelldur{\durmed}
\spelleffect You imbue body armor or a shield with a \plus2 enhancement bonus, increasing the wielder's physical defenses. \spellbonusscalingdescription
\spellnotes An outfit of regular clothing counts as armor that grants no armor defense bonus for the purpose of this spell.
\spellsr{Yes (Will)}

\spellsection{Magic Weapon}{2}
\spellinfo{Trans (Augment)}{Arcane, Divine}
\spelltwocol{\spelltgt{One weapon or fifty projectiles (all of which must be in contact with each other at the time of casting)}}{\spellrng{\rngclose}}
\spelldur{\durshort}
\spelleffect You imbue a weapon or stack of projectiles with a \plus2 enhancement bonus, giving its wielder a \plus2 enhancement bonus to attack and damage. \spellbonusscalingdescription
\spellnotes You can't cast this spell on a natural weapon, such as an unarmed strike (instead, see \spell{magic fang}). A monk's unarmed strike is considered a weapon, and thus it can be enhanced by this spell.
\par If you use this spell to enhance projectiles, the projectiles must be of the same kind, and they have to be together (in the same quiver or other container). Projectiles, but not thrown weapons, lose their transmutation when used. (Treat darts and shuriken as projectiles, rather than as thrown weapons, for the purpose of this spell.)
\spellsr{Yes (Will)}

\spellsection{Major Image}{4}
\spellinfo{Illus (Figment) [Unreal]}{Illus}
\spelltwocol{\spellzone{\arealarge radius}}{\spellrng{\rngmed}}
\spelldur{\durshort}
\spellline
\spelleffect A figment of your design appears within the area, as \spell{silent image}, except that sound, smell, and thermal elements are included.
\spellnotes Creatures can identify the illusion, as \spell{silent image}.

\invocationsection{Mass (Spell Name)}
\par Any spell whose name begins with mass is alphabetized in this chapter according to the second word of the spell name. Thus, the description of a mass spell appears near the description of the spell on which it is based. Spell chains that have mass spells in them include those based on the spells charm monster, cure critical wounds, cure light wounds, cure moderate wounds, cure serious wounds, enlarge person, heal, hold monster, hold person, inflict critical wounds, inflict light wounds, inflict moderate wounds, inflict serious wounds, invisibility, reduce person, suggestion, totemic mind, and totemic power.

\spellsection{Maze}{8}
\spellinfo{Conj (Translocation) [Planar]}{Conj, Trickery}
\spelltwocol{\spelltgt{One creature}}{\spellrng{\rngclose}}
\spelldur{Instantaneous; see text}
\spellattack{Will partial}
\spelleffect You banish the subject into an extradimensional labyrinth of force planes. Each round on its turn, it may attempt an Intelligence check to escape the labyrinth as a full-round action. You make a Will attack to attempt to place it in the middle of the labyrinth. A successful attack means the DC of the Intelligence check is 20, while a failed attack means the DC is 15. If the subject doesn't escape, the maze disappears after 5 minutes, forcing the subject back to the location where it was originally banished.
\par On escaping or leaving the maze, the subject reappears where it had been when the maze spell was cast. If this location is filled with a solid object, the subject appears in the nearest open space.
\spellnotes Spells and abilities that move a creature within a plane, such as \spell{teleport} and \spell{dimension door}, do not help a creature escape a \spell{maze} spell, although a \spell{plane shift} spell allows it to exit to whatever plane is designated in that spell. Minotaurs can escape the spell automatically.
\spellsr{Yes (Will)}

\spellsection{Meld into Stone}{3}
\spellinfo{Trans (Polymorph) [Earth]}{Earth, Nature}
\spelltwocol{\spelltgt{You}}{\spellrng{Personal}}
\spelldur{\durlong}
\spelleffect This spell enables you to meld your body and possessions into a single block of stone. The stone must be large enough to accommodate your body in all three dimensions. When the casting is complete, you and not more than 100 pounds of nonliving gear merge with the stone. If either condition is violated, the spell fails and is wasted.
\par While in the stone, you remain in contact, however tenuous, with the face of the stone through which you melded. You remain aware of the passage of time and can cast spells on yourself while hiding in the stone. Nothing that goes on outside the stone can be seen, but you can still hear what happens around you. Minor physical damage to the stone does not harm you, but its partial destruction (to the extent that you no longer fit within it) expels you and deals you 5d6 points of damage. If the stone is completely destroyed, you are expelled, and you die unless your Fortitude defense is at least 20.
\par Any time before the duration expires, you can step out of the stone through the surface that you entered. If the spell's duration expires or the effect is dispelled before you voluntarily exit the stone, you are violently expelled and take 5d6 points of damage.
\spellnotes The following spells harm you if cast upon the stone that you are occupying: \spell{transmute flesh and stone} expels you and deals 6d6 points of damage. \spellindirect{shape stone}{Shape stone} deals 3d6 points of damage but does not expel you. \spell{Passwall} expels you without damage.

\spellsection{Message}{1}
\spellinfo{Div (Communication)}{Arcane}
\spellcmp{Somatic only}
\spelltwocol{\spelllimit{\areamed radius}}{\spellrng{\rngmed}}
\spelltgts{Five creatures in the area}
\spelldur{\durlong}
\spelleffect You can whisper messages and receive whispered replies with little chance of being overheard. You point your finger at each creature you want to receive the message. When you whisper, the whispered message is audible to all targeted creatures within range. Magical silence, 1 foot of stone, 1 inch of common metal (or a thin sheet of lead), or 3 feet of wood or dirt blocks the spell. The message does not have to travel in a straight line. It can circumvent a barrier if there is an open path between you and the subject, and the path's entire length lies within the spell's range. The creatures that receive the message can whisper a reply that you hear. The spell transmits sound, not meaning. It doesn't transcend language barriers.
\spellnotes To speak a message, you must mouth the words and whisper, possibly allowing observers the opportunity to read your lips.

\spellsection{Meteor Swarm}{9}
\spelldesc{You call a swarm of meteors that streak down from the heavens, leaving a fiery trail behind them. The meteors crash into your foes, driving flying creatures to the ground and knocking your foes off their feet.}
\spellinfo{Evoc (Energy) [Fire]}{Destruction, Evoc, Fire}
\spelltwocol{\spellburst{A \arealarge radius cylinder, 100 ft. high}}{\spellrng{\rngfar}}
\spelldmg{9d6 fire damage \add d6 per four caster levels above 18th}
\spellattack{Magic vs. Reflex/Reflex negates}
\spelleffect You make a Reflex attack to deal damage to everything in the area. A failed attack deals half damage. If you succeed against a flying creature in the area of size Huge or smaller, it is driven to the ground, taking falling damage appropriate to the distance it descends. If you succeed against a creature on the ground, it falls prone.
\spellnotes This spell functions indoors or underground, but not underwater.
\spellsr{Yes (Reflex)}

\spellsection{Mind Fog}{4}
\spelldesc{You conjure a fog bank that hampers the mental acuity of those caught in it.}
\spellinfo{Conj/Ench (Creation, Inhibition) [Fog, Mind-Affecting]}{Arcane, Trickery}
\spelltwocol{\spellzone{\areamed radius cylinder}}{\spellrng{\rngclose}}
\spelldur{\durlong and 5 rounds; see text}
\spellattack{None/Magic vs. Will}
\spelleffect This spell functions like \spell{fog cloud}, except that the fog hampers mental ability. Once per minute, you make a Will attack against all creatures in the area to impose a \minus5 penalty to Wisdom. Affected creatures take the penalty as long as they remain in the fog and for 5 rounds thereafter. The fog is stationary and lasts for 1 hour (or until dispersed by wind).
\spellnotes A moderate wind (11\add mph) disperses the fog in 5 rounds; a strong wind (21\add mph) disperses the fog in 1 round. The penalty to Wisdom imposed by this spell does not stack with itself.

\spellsection{Minor Image}{3}
\spellinfo{Illus (Figment) [Unreal]}{Illus}
\spelltwocol{\spellzone{\areamed radius}}{\spellrng{\rngmed}}
\spelldur{\durshort}
\spellline
\spelleffect A figment of your design appears within the area, as \spell{silent image}, except that sound elements are included.
\spellnotes Creatures can identify the illusion, as \spell{silent image}.

\spellsection{Miracle}{9}
\spellinfo{Evoc (Channeling)}{Divine}
\spelltwocol{\spelltgteffarea{See text}}{\spellrng{See text}}
\spelldur{See text}
\spellattack{See text}
\spelleffect You don't so much cast a miracle as request one. You state what you would like to have happen and request that your deity (or the power you pray to for spells) intercede.
\par A miracle can do any of the following things.
\begin{itemize}
    \item Duplicate any cleric spell of 8th level or lower (including spells to which you have access because of your domains). 
    \item Duplicate any other spell of 7th level or lower.
    \item Undo the harmful effects of certain spells, such as feeblemind or insanity.
    \item Have any effect whose power level is in line with the above effects.
\end{itemize}
\par Alternatively, a cleric can make a very powerful request. Examples of especially powerful miracles of this sort could include the following.
\begin{itemize}
    \item Swiinging the tide of a battle in your favor by raising fallen allies to continue fighting.
    \item Moving you and your allies, with all your and their gear, from one plane to another through planar barriers to a specific locale with no chance of error.
    \item Protecting a city from an earthquake, volcanic eruption, flood, or other major natural disaster.
\end{itemize}
\par In any event, a request that is out of line with the deity's (or alignment's) nature is refused.
\spellnotes If you request a miracle, your deity (or the power you pray to) will expect something of you in return. You must cast commune to learn what this is within 24 hours, or you will lose the ability to cast any cleric spells other than commune. For more moderate miracles, you may be required to offer 25,000gp worth of incense and gems. For especially powerful miracles, or multiple moderate miracles, you may geased with a task to complete.
\par When a miracle spell duplicates a spell with a material component that costs more than 5,000 gp, you must provide that component.
\spellsr{Yes (varies; see text)}

\spellsection{Mirror Image}{2}
\spelldesc{You create illusory duplicates of yourself that make it difficult for enemies to know which image to attack.}
\spellinfo{Illus (Figment)}{Arcane}
\spelltwocol{\spelltgt{You}}{\spellrng{Personal; see text}}
\spelldur{\durshort \dismissable}
\spelleffect This spell creates an illusory duplicate of yourself that mimics your movements perfectly. Enemies attempting to attack you or cast spells at you must select which to attack. Generally, roll randomly to see whether the selected target is real or a figment. An image's physical defenses are 10 \add its size modifier. You gain an additional image at 8th, 14th, and 20th caster level. 
\par If an image is hit, it is destroyed. If you are hit, your attacker knows the attack was successful, and can ignore the image. You can create new images to replace destroyed images as a swift action, preventing your foes from knowing which image to attack.
\par You can move into and through your duplicates on your turn. When you and the image separate, observers can't use vision or hearing to tell which one is you and which the image. The duplicates may also move through each other. The figments mimic your actions, pretending to cast spells when you cast a spell, drink potions when you drink a potion, levitate when you levitate, and so on.
\par Mirror images can be attacked like any other creature. They count as separate creatures, and can be targeted separately by spells like \spell{magic missile} or feats like Whirlwind Attack, though they are not destroyed by area spells. Destroying an image counts as dropping a creature for the purpose of the Cleave feat and similar abilities.
\spellnotes An attacker must be able to see the images to be fooled. If you are invisible or an attacker shuts his or her eyes, the spell has no effect. (Being unable to see means you are blinded.)

\spellsection{Mislead}{6}
\spellinfo{Illus (Figment, Glamer) [Unreal]}{Arcane, Trickery}
\spelltwocol{\spelltgt{You}}{\spellrng{Personal/\rngmed}}
\spelldur{\durshort \dismissable; see text}
\spellattack{Magic check vs. Perception and Will (if interacted with); see text}
\spelleffect You become invisible (as \spell{invisibility}, a glamer), and at the same time, an illusory double of you (as \spell{major image}, an unreal figment) appears. You are then free to go elsewhere while your double moves away. The double appears within range but thereafter moves as you direct it (which requires concentration beginning on the first round after the casting). You can make the figment appear superimposed perfectly over your own body so that observers don't notice an image appearing and you turning invisible. You and the figment can then move in different directions. The double moves at your speed and can talk and gesture as if it were real, but it cannot attack or cast spells, though it can pretend to do so.
\par The illusory double lasts as long as you concentrate upon it, plus 5 additional rounds. After you cease concentration, the illusory double continues to carry out the same activity until the duration expires. The invisibility lasts for 5 minutes, regardless of concentration.

\spellsection{Missile Storm}{7}
\spelldesc{You unleash an immense swarm of missiles which seek out and destroy all of your foes.}
\spellinfo{Evoc (Control) [Force]}{Arcane}
\spelllimit{\arealarge radius centered on you}
\spelltgts{Any number of creatures in the area}
\spelldmg{7d4 force damage \add d4 per four caster levels above 14th}
\spelleffect Each target is struck by seven missiles like those created by the \spell{magic missile} spell. Each missile deals 1d4 damage. Each target is struck by one additional missile per four caster levels above 14th.
\spellsr{Yes (Fortitude)}

\spellsection{Moment of Prescience}{7}
\spellinfo{Div (Knowledge)}{Arcane, Div, Knowledge}
\spelleffect This spell functions like \spell{lesser moment of prescience}, except that you also gain a bonus equal to half your caster level on the roll. Alternately, you can expend the spell to protect yourself. If you do, you gain a bonus to your dodge modifier equal to half your caster level, and you stop being unaware of an attack if you were. This effect can be used even if you are unaware of an attack, which would normally prevent you from taking any combat actions.
\spellnotes You can't have more than one \spellindirect{lesser moment of prescience}{moment of prescience} effect active on you at the same time.

\spellsectioncomma{Moment of Prescience}{Greater}{9}
\spellinfo{Div (Knowledge)}{Div}
\spelleffect This spell functions like \spell{moment of prescience}, except that the bonus and extra rolls apply to all of your physical attacks, opposed checks, and defenses until the beginning of your next turn.
\spellnotes You can't have more than one \spellindirect{lesser moment of prescience}{moment of prescience} effect active on you at the same time.

\spellsectioncomma{Moment of Prescience}{Lesser}{4}
\spellinfo{Div (Knowledge)}{Arcane, Div, Knowledge}
\spelltwocol{\spelltgt{You}}{\spellrng{Personal}}
\spelldur{\durext or until discharged}
\spelleffect This spell grants you a powerful sixth sense in relation to yourself. Once during the spell's duration, you may choose to use its effect. You may roll twice on any single physical attack or opposed check. Activating the effect takes an immediate action, so you can even activate it on another character's turn if needed. Once activated, the spell ends.
\spellnotes You can't have more than one \spellindirect{lesser moment of prescience}{moment of prescience} effect active on you at the same time.

\pdfbookmark[2]{O-P}{SpellDescriptionsO-P}
\begin{comment}
\subsubsection{O-P}
\end{comment}

\spellsection{Obscuring Mist}{1}
\spelldesc{You conjure a bank of fog that arises around you, concealing you and your allies.}
\spellinfo{Conj (Creation) [Fog]}{Arcane, Divine, Nature, Water}
\spellzone{\areamed radius cylinder centered on you, 20 ft. high}
\spelleffect This spell functions like \spell{fog cloud}, except that the fog created is centered on you.

\spellsection{Order's Wrath}{4}
\spellinfo{Evoc (Channeling) [Lawful]}{Law}
\spelltwocol{\spelltgt{One creature}}{\spellrng{\rngmed}}
\spelldur{Instantaneous/5 rounds}
\spelldmg{8d6 divine damage \add d6 per two caster levels above 8th}
\spellattack{None/Will half}
\spelleffect If the target is not lawful, it is bewildered for 5 rounds, and you make a Will attack to deal damage to it. A failed attack deals half damage.
\spellsr{Yes (Will)}

\spellsection{Persistent Image}{6}
\spellinfo{Illus (Figment)}{Illus}
\spelltwocol{\spellzone{\arealarge radius}}{\spellrng{\rngmed}}
\spelldur{\durmed \dismissable}
\spellline
\spelleffect A figment of your design appears within the area, as \spell{silent image}, except that sound, smell, and thermal elements are included. When you cast the spell, you set a script for the figment to follow. It follows that script without you having to concentrate on the spell.
\spellnotes Creatures can identify the illusion, as \spell{silent image}.

\spellsection{Phantasmal Killer}{4}
\spelldesc{You create a phantasmal image of the most fearsome creature imaginable to the subject simply by forming the fears of the subject's subconscious mind into something that its conscious mind can visualize: this most horrible beast.}
\spellinfo{Ench/Illus (Emotion, Phantasm) [Death, Fear, Mind-Affecting]}{Arcane, Trickery}
\spelltwocol{\spelltgt{One creature}}{\spellrng{\rngmed}}
\spellattack{Will disbelief and Magic vs. Fortitude; see text}
\spelleffect You make a Will attack to make the subject shaken for 5 rounds, leaving it \vulnerable.
\begin{spellblood}
    If you succeed at the Will attack, the creature is frightened for 5 rounds. In addition, make a Fortitude attack. If you succeed, the target loses all its hit points and takes 9 critical damage, causing it to begin dying.
\end{spellblood}
\spellsr{Yes (Will)}

\spellsection{Phantasmal Maze}{5}
\spelldesc{You manipulate the subject's perceptions, causing it to believe that it is trapped in a labyrinth.}
\spellinfo{Illus (Phantasm)}{Arcane, Trickery}
\spelltwocol{\spelltgt{One creature}}{\spellrng{\rngclose}}
\spelldur{\durmed}
\spellattack{Will disbelief}
\spelleffect The subject perceives itself to be banished to an extradimensional labyrinth of force planes, as the \spell{maze} spell. It cannot see or hear anything to the contrary, causing it to be treated as if blinded and deafened for most purposes. Typically, this means the subject moves in a random direction each round to escape the maze. If it encounters any physical resistance in its movements or takes any damage, you must make another Will attack to maintain the effect.
\spellsr{Yes (Will)}

\spellsection{Phantasmal Wound}{2}
\spelldesc{You manipulate the subject's perceptions, causing it to believe that it is grievously wounded.}
\spellinfo{Illus (Phantasm)}{Arcane}
\spelltwocol{\spelltgt{One creature}}{\spellrng{\rngmed}}
\spelldur{\durshort}
\spellattack{Will disbelief}
\begin{spellhealthy}
    The subject is sickened.
\end{spellhealthy}
\begin{spellblood}
    The subject perceives itself to have no hit points remaining. It is staggered, and may try to heal itself or take other actions. If its hit points are altered, such as by damage or healing, the creature is merely sickened for the rest of the spell's duration.
\end{spellblood}
\spellnotes A staggered creature may take a single move action or standard action each round, but not both. It cannot take full-round actions, but may take swift actions. In addition, it is \vulnerable, causing it to take a \minus2 penalty on attacks, defenses, and checks.
\spellsr{Yes (Will)}

\spellsection{Poison}{4}
\spelldesc{Calling upon the venomous powers of natural predators, you infect the subject with a horrible poison that drains its life force.}
\spellinfo{Necro (Flesh) [Poison]}{Death, Divine, Nature}
\spelltwocol{\spelltgt{One living creature}}{\spellrng{Touch}}
\spelldur{Instantaneous; see text}
\spellattack{Magic vs. Fortitude; see text}
\spelleffect You make a Fortitude attack against the touched creature. Success means it takes 1d6 points of Constitution damage immediately. Every two rounds, you make another Fortitude attack to deal 1d6 points of Constitution damage. The spell lasts until you fail the Fortitude attack twice.
\spellsr{Yes (Fortitude)}

\spellsection{Polar Ray}{8}
\spelldesc{You fire a blue-white ray of frigid air and ice, freezing your foe in place.}
\spellinfo{Evoc (Energy) [Cold]}{Arcane, Water}
\spelltwocol{\spelltgt{One creature or object}}{\spellrng{\rngclose}}
\spelldur{Instantaneous/5 rounds}
\spelldmg{16d6 cold damage \add d6 per three caster levels above 16th}
\begin{spellhealthy}
    You make a Reflex attack to deal damage to the target and slow it for 5 rounds.
\end{spellhealthy}
\begin{spellblood}
    The struck target is also frozen solid, causing it to be paralyzed for 5 rounds.
\end{spellblood}
\spellnotes A slowed creature can take only a single move action or standard action each turn, but not both. It cannot take full-round actions, but it may take swift actions. Additionally, it takes a \minus2 penalty to physical attacks and defenses, as well as Strength and Dexterity-based checks.

A paralyzed creature cannot take any action that requires motion. It has effective Dexterity and Strength scores of \minus10 and is helpless, but can take purely mental actions. A winged creature flying in the air at the time that it becomes paralyzed cannot flap its wings and falls. A paralyzed swimmer can't swim and may drown. A creature can move through a space occupied by a paralyzed creature -- ally or otherwise. Each square occupied by a paralyzed creature, however, counts as 2 squares.
\spellsr{Yes (Fortitude)}

\spellsection{Power Word Blind}{6}
\spellinfo{Necro (Flesh)}{Arcane}
\spellcmp{Verbal only}
\spelltwocol{\spelltgt{One creature}}{\spellrng{\rngclose}}
\spelldur{Instantaneous/5 rounds}
\begin{spellhealthy}
    The target is sickened, making it \vulnerable for 5 rounds.
\end{spellhealthy}
\begin{spellblood}
    The target is blinded for 5 rounds.
\end{spellblood}
\spellnotes The target must be \bloodied when the spell is cast to suffer the \bloodied effect.

A blinded creature cannot see. It moves at half speed and is defenseless, causing it to provoke attacks of opportunity for its actions. All checks and activities that rely on vision (such as reading and visual Perception checks) automatically fail, and any checks related to vision (such as Climb and Sense Motive checks) take a \minus4 penalty. All opponents are considered to be invisible (50\% miss chance) relative to the blinded creature.
\spellsr{Yes (Fortitude)}

\spellsection{Power Word Kill}{9}
\spelldesc{You utter a single word of power that instantly kills your foe, whether it can hear the word or not.}
\spellinfo{Necro (Life) [Death]}{Arcane, Death}
\spellcmp{Verbal only}
\spelltwocol{\spelltgt{One living creature}}{\spellrng{\rngclose}}
\begin{spellhealthy}
    The target is sickened for 5 rounds, making it \vulnerable.
\end{spellhealthy}
\begin{spellblood}
    If the target's level does not exceed your caster level, it loses all its hit points and takes 9 critical damage, causing it to begin dying. Otherwise, it is sickened for 5 rounds.
\end{spellblood}
\spellnotes The target must be \bloodied when the spell is cast to suffer the \bloodied effect.
\spellsr{Yes (Fortitude)}

\spellsection{Power Word Confuse}{5}
\spellinfo{Ench (Compulsion) [Mind-Affecting]}{Arcane}
\spellcmp{Verbal only}
\spelltwocol{\spelltgt{One creature}}{\spellrng{\rngclose}}
\begin{spellhealthy}
    The target is bewildered, making it \vulnerable for 5 rounds.
\end{spellhealthy}
\begin{spellblood}
    The target is confused for 5 rounds. \confusionexplanation
\end{spellblood}
\spellnotes The target must be \bloodied when the spell is cast to suffer the \bloodied effect.
\spellsr{Yes (Will)}

\spellsection{Power Word Stun}{7}
\spelldesc{You utter a single word of power that instantly causes your foe to become stunned, whether the creature can hear the word or not.}
\spellinfo{Ench (Inhibition) [Mind-Affecting]}{Arcane}
\spellcmp{Verbal only}
\spelltwocol{\spelltgt{One creature}}{\spellrng{\rngclose}}
\begin{spellhealthy}
    The target is bewildered, making it \vulnerable for 5 rounds.
\end{spellhealthy}
\begin{spellblood}
    The target is stunned for 5 rounds.
\end{spellblood}
\spellnotes The target must be \bloodied when the spell is cast to suffer the \bloodied effect.
\spellsr{Yes (Will)}

\spellsectioncomma{Precognition}{Lesser}{2}
\spelldesc{You extend your mind a fraction of a second into the future, allowing you to strike at your foes more effectively.}
\spellinfo{Div (Knowledge)}{Arcane, Div}
\spelltwocol{\spelltgt{You}}{\spellrng{Personal}}
\spelldur{\durshort \dismissable}
\spelleffect You gain a \plus2 enhancement bonus to your attack and weapon damage rolls. \spellbonusscalingdescription

\spellsection{Precognition}{5}
\spelldesc{You extend your mind a fraction of a second into the future, allowing you to strike at your foes more effectively and avoid hostile attacks more easily.}
\spellinfo{Div (Knowledge)}{Arcane, Div}
\spelleffect This spell functions like \spell{lesser precognition}, except that it also affects your special defenses and dodge modifier.

\spellsectioncomma{Precognition}{Greater}{8}
\spelldesc{You extend your mind a short time into the future, allowing you to strike at your foes more effectively and avoid hostile attacks more easily.}
\spellinfo{Div (Knowledge)}{Arcane, Div}
\spelleffect This spell functions like \spell{lesser precognition}, except that it also affects your special defenses and dodge modifier. In addition, when making a full attack, you may make an additional attack at a \minus5 penalty.

\spellsection{Manipulate Probability}{5}
\spellinfo{Div (Knowledge)}{Div, Knowledge}
\spelleffect This spell functions like \spell{foresee probability}, except that you choose the order in which the die results are used. You must still use all die results before the creature rolls for itself.

\spellsection{Prismatic Sphere}{9}
\spellinfo{Evoc (Control, Energy) [Light]}{Arcane}
\spelleffect This spell functions like \spell{prismatic wall}, except you conjure up a \areasmall immobile, opaque sphere of shimmering, multicolored light that surrounds you and protects you from all forms of attack. The sphere flashes in all colors of the visible spectrum. 
\par You can pass into and out of the prismatic sphere and remain near it without harm. However, the sphere blocks any attempt to project something through it (including spells). Other creatures that attempt to attack you or pass through suffer the effects of each color, one at a time. You can fight from partially within the sphere. If you do, you gain cover from anyone outside the sphere.
\par Typically, only the upper hemisphere of the globe will exist, since it is created with you at the center, so the lower half is usually excluded by the floor surface you are standing on.
\par The colors of the sphere have the same effects as the colors of a prismatic wall.
\spellnotes This spell can be made permanent with a \spell{permanency} ritual.

\spellsection{Prismatic Spray}{7}
\spelldesc{This spell causes seven shimmering, intertwined, multicolored beams of light to spray from your hand.}
\spellinfo{Evoc (Control, Energy) [Light]}{Arcane, Chaos}
\spellburst{\arealarge cone}
\spellattack{See text}
\spelleffect You make an attack against every creature in the area. Each creature is randomly affected by a differently colored beam. The beam color determines the effects the creature suffers, as shown on the table below.
\begin{dtable}
    \begin{tabularx}{\columnwidth}{l >{\lcol}p{4em} >{\lcol}X}
        \thead{1d8} & \thead{Color of Beam} & \thead{Effect} \\
        1 & Red & 15 points fire damage (Magic vs. Reflex) \\
        2 & Orange & 30 points acid damage (Magic vs. Reflex) \\
        3 & Yellow & 45 points electricity damage (Magic vs. Reflex) \\
        4 & Green & 40 damage and nauseated for 1 round (Magic vs. Fortitude) \\
        5 & Blue & Petrified if \bloodied, slowed for 5 rounds if healthy (Magic vs. Fortitude) \\
        6 & Indigo & Insane, as \spell{insanity} spell (Magic vs. Will) \\
        7 & Violet & Sent to another plane, as \spell{plane shift} ritual (Magic vs. Will) \\
        8 & & Struck by two rays; roll twice more, ignoring any ``8" results.
    \end{tabularx}
\end{dtable}
\spellsr{Yes (varies)}

\spellsection{Prismatic Wall}{8}
\spellinfo{Evoc (Control, Energy) [Light]}{Arcane, Chaos}
\spelltwocol{\spellzone{\arealarge wall, 20 ft. high}}{\spellrng{\rngclose}}
\spelldur{\durshort \dismissable}
\spellattack{See text}
\spelleffect This spell creates a vertical, opaque wall -- a shimmering, multicolored plane of light that protects you from all forms of attack. The wall flashes with seven colors, each of which has a distinct power and purpose. The wall is immobile, and you can pass through and remain near the wall without harm. However, any other creature with less than 8 levels that is within 20 feet of the wall is blinded for 1 minute by the colors if it looks at the wall.
\par The wall's maximum proportions are 50 feet wide and 30 feet high. A \spell{prismatic wall} spell cast to materialize in a space occupied by a creature is disrupted, and the spell is wasted.
\par Each color in the wall has a special effect. \trefnp{Prismatic Wall Colors} shows the seven colors of the wall, the order in which they appear, their effects on creatures trying to attack you or pass through the wall, and the magic needed to negate each color.
\par The wall can be destroyed, color by color, in consecutive order, by various magical effects; however, the first color must be brought down before the second can be affected, and so on. A rod of cancellation or a mage's disjunction spell destroys a prismatic wall, but an antimagic field fails to penetrate it. Dispel magic and greater dispel magic cannot dispel the wall or anything beyond it. Spell resistance is effective against a prismatic wall, but the caster level check must be repeated for each color present.
\begin{dtable*}
    \lcaption{Prismatic Wall Colors}
    \begin{tabularx}{\textwidth}{l l >{\lcol}X l}
        \thead{Color} & \thead{Order} & \thead{Effect of Color} & \thead{Negated By} \\
        Red & 1st & Stops nonmagical ranged weapons.
        Deals 15 points of fire damage (Magic vs. Reflex). & \spellindirect{cone of cold}{Cone of cold} \\
        Orange & 2nd & Stops magical ranged weapons.
        Deals 30 points of acid damage (Magic vs. Reflex). & \spellindirect{gust of wind}{Gust of wind} \\
        Yellow & 3rd & Stops poisons, gases, and petrification.
        Deals 45 points of electricity damage (Magic vs. Reflex). & \spell{Disintegrate} \\
        Green & 4th & Stops breath weapons.
        Poison (40 damage and nauseated for 1 round; Magic vs. Fortitude). & \spell{Passwall} \\
        Blue & 5th & Stops divination and mental attacks.
        Petrified if \bloodied, slowed and entangled for 1 minute if healthy (Magic vs. Fortitude). & \spellindirect{magic missile}{Magic missile} \\
        Indigo & 6th & Stops all spells.
        Become insane, as \spell{insanity} spell (Magic vs. Will). & \spell{Daylight} \\
        Violet & 7th & Energy field destroys all objects and effects.\footnotetemp{1}
        Creatures sent to another plane (as \spell{plane shift} ritual) (Magic vs. Will). & \spellindirect{dispel magic}{Dispel magic} \\
    \end{tabularx}
    1 The violet effect makes the special effects of the other six colors redundant, but these six effects are included here because certain magical effects can create prismatic effects one color at a time, and spell resistance might render some colors ineffective (see above).
\end{dtable*}
\spellnotes This spell can be made permanent with a \spell{permanency} ritual.
\spellsr{See text}

\spellsection{Prohibition}{9}
\spellinfo{Abjur (Interdiction)}{Abjur, Law}
\spellemanation{100 foot radius centered on you}
\spelldur{\durshort}
\spelldmg{8d6 damage \add 1d6 per four caster levels above 16th}
\spellattack{Will half/Magic vs. Will}
\spelleffect When you cast this spell, you choose an action. Whenever a creature in the area intentionally attempts to take the chosen action, you make a Will attack against them. A successful attack means they are unable to complete the action, while a failed attack means they take half damage. Any individual creature can only be affected by the \spell{prohibition} once per round.

You may choose any action that must be taken intentionally. For example, you could prohibit creatures from attacking, but not from breathing. Likewise, you cannot prohibit actions that have non-voluntary components, such as ``protecting yourself with natural armor''. 
\spellnotes Creatures that take prohibited actions know they were magically prevented from taking their chosen action, and may choose to take a different action. However, they are granted no special insight into the precise nature of the prohibition they violated.
\spellsr{Yes (Will)}

\spellsection{Project Image}{6}
\spellinfo{Illus (Shadow)}{Arcane}
\spellrng{\rngmed}
\spelldur{\durshort \dismissable}
\spellattack{Magic check vs. Perception and Will (if interacted with)}
\spelleffect You tap energy from the Plane of Shadow to create a quasi-real, illusory version of yourself. The projected image looks, sounds, and smells like you but is intangible. The projected image mimics your actions (including speech) unless you direct it to act differently (which is a move action).
\par You can see through its eyes and hear through its ears as if you were standing where it is, and during your turn you can switch from using its senses to using your own, or back again, as a swift action. While you are using its senses, your body is considered blinded and deafened.
\par If you desire, any spell you cast whose range is touch or greater can originate from the projected image instead of from you. The projected image can't cast any spells on itself except for illusion spells. The spells affect other targets normally, despite originating from the projected image.
\spellnotes Objects are always considered to disbelieve this spell.
\par You must maintain line of effect to the projected image at all times. If your line of effect is obstructed, the spell ends. If you use \spell{dimension door}, \spell{teleport}, \spell{plane shift}, or a similar spell that breaks your line of effect, even momentarily, the spell ends.

\spellsection{Protection from Energy}{3}
\spellinfo{Abjur (Shielding)}{Arcane, Divine, Nature, Protection}
\spelltwocol{\spelltgt{One creature}}{\spellrng{Touch}}
\spelldur{\durlong or until discharged}
\spelleffect This spell grants temporary immunity to the type of energy you specify when you cast it (acid, cold, electricity, fire, or sonic). When the spell absorbs 10 points per caster level of energy damage, it is discharged.
\spellnotes Protection from energy overlaps (and does not stack with) \spell{resist energy}. If a character is shielded by \spell{protection from energy} and \spell{resist energy}, the protection spell absorbs damage until its power is exhausted.
\spellsr{Yes (Fortitude)}

\spellsectioncomma{Protection from Energy}{Greater}{6}
\spellinfo{Abjur (Shielding)}{Arcane, Divine, Nature, Protection}
\spelleffect This spell functions like \spell{protection from energy}, except that it protects from all five types of energy. When the spell absorbs 10 points per caster level of damage in total, regardless of its type, it is discharged.

\spellsection{Protection from Alignment}{1}
\spellinfo{Abjur (Interdiction)}{Arcane, Chaos, Divine, Evil, Good, Law}
\spellrng{\rngclose}
\spelldur{\durshort \dismissable}
\begin{spelltarget}{One creature}
    \spelleffect The subject gains a \plus2 enhancement bonus to its special defenses. \spellbonusscalingdescription

    In addition, choose an alignment other than neutral. The enhancement bonus is increased by 2 against attacks from creatures or effects with the chosen alignment.
\end{spelltarget}
\spellnotes This spell has the subtype of the alignment opposed to the chosen alignment.

\pdfbookmark[2]{Q-R}{SpellDescriptionsQR}
\begin{comment}
\subsubsection{Q-R}
\end{comment}

\spellsection{Read Mind}{3}
\spellinfo{Div (Awareness) [Mind-Affecting]}{Arcane, Knowledge}
\spelltwocol{\spelltgt{One creature}}{\spellrng{\rngmed}}
\spelldur{Concentration}
\spellattack{Magic vs. Will}
\spelleffect You make a Will attack against the subject to read its surface thoughts. A failed attack prevents you from reading its thoughts. You gain a \plus4 bonus to Bluff, Persuasion, and Intimidate checks against creatures whose mind you are reading.
\spellsr{Yes (Will)}

\spellsectioncomma{Read Mind}{Greater}{7}
\spellinfo{Div (Awareness) [Mind-Affecting]}{Arcane, Knowledge}
\spelleffect This spell functions like \spell{read mind}, except that no attack is needed.

\spellsectioncomma{Read Mind}{Mass}{8}
\spellinfo{Div (Awareness) [Mind-Affecting]}{Arcane, Knowledge}
\spelllimit{\areamed radius}
\spelltgt{Five creatures in the area}
\spelleffect This spell functions like \spell{read mind}, except that you can read the minds of multiple creatures at once. 

\spellsection{Reduce Person}{1}
\spellinfo{Trans (Polymorph) [Size-Affecting]}{Trans}
\spelltime{Full-round action}
\spelltwocol{\spelltgt{One humanoid creature}}{\spellrng{\rngclose}}
\spelldur{\durshort \dismissable}
\spellattack{Magic vs. Fortitude}
\spelleffect This spell causes instant diminution of a humanoid creature, halving its height, length, and width and dividing its weight by 8. This decrease changes the creature's size category to the next smaller one. This has several effects.
\begin{itemize} 
    \item \minus10 ft. penalty to movement speed.
    \item \minus4 penalty to maneuver attack and defense.
    \item \plus1 bonus to other physical attacks and defenses.
    \item \minus2 penalty to Strength.
    \item \plus2 enhancement bonus to Dexterity.
    \item \plus4 bonus to Stealth checks.
\end{itemize}
\par A Small humanoid creature whose size decreases to Tiny has a space of 2-1/2 feet and a natural reach of 0 feet (meaning that it must enter an opponent's square to attack). A Large humanoid creature whose size decreases to Medium has a space of 5 feet and a natural reach of 5 feet.
\par All equipment worn or carried by a creature is similarly reduced by the spell. Melee and projectile weapons deal less damage. Other magical properties are not affected by this spell. Any reduced item that leaves the reduced creature's possession (including a projectile or thrown weapon) instantly returns to its normal size. This means that thrown weapons deal their normal damage (projectiles deal damage based on the size of the weapon that fired them).
\spellnotes Multiple magical effects that reduce size do not stack. This spell can be made permanent with a \spell{permanency} ritual.
\spellsr{Yes (Fortitude)}

\spellsectioncomma{Reduce Person}{Mass}{5}
\spellinfo{Trans (Polymorph) [Size-Affecting]}{Trans}
\spelltwocol{\spelllimit{\areamed radius}}{\spellrng{\rngmed}}
\spelltgts{Five humanoid creatures in the area}
\spelleffect This spell functions like \spell{reduce person}, except that it affects multiple creatures.

\spellsection{Regenerate}{8}
\spellinfo{Necro (Flesh)}{Divine, Nature}
\spelltwocol{\spelltgt{One living creature}}{\spellrng{Touch}}
\spelldur{\durshort}
\spelleffect You grant immense healing power to a creature with a touch. The target of this spell automatically heals a number of hit points each round equal to your caster level.
\par You can also use this spell to regrow lost portions of the subject's body and to reattach severed limbs or body parts, if you do nothing but concentrate on regrowing the lost body part or reattaching the severed limb for 5 minutes.
\spellsr{Yes (Fortitude)}

\spellsection{Repulsion}{6}
\spellinfo{Abjur (Shielding) [Barrier]}{Arcane, Protection, Travel}
\spellemanation{\arealarge radius centered on you}
\spelldur{\durshort \dismissable}
\spellattack{Magic vs. Will}
\spelleffect An invisible, mobile field surrounds you and prevents creatures from approaching you. You decide how big the field is at the time of casting. When you cast this spell, and at the start of each of your turns, you make a Will attack. Any creature within or entering the field who cannot resist your attack is unable to move towards you that round.

Repelled creatures' actions are not otherwise restricted. They can fight other creatures and can cast spells and attack you with ranged weapons. The creature is free to make melee attacks against you if you come within reach. If a repelled creature moves away from you and then tries to turn back toward you, it cannot move any closer if it is still within the spell's area.
\spellnotes Unlike most barrier spells, this spell does not collapse if you move towards a creature held at bay by the barrier. The spell continues to prevent that creature from approaching you, but the creature suffers no other ill effect.
\spellsr{Yes (Will)}

\spellsection{Resilient Sphere}{5}
\spellinfo{Evoc (Control) [Force]}{Evoc}
\spelltwocol{\spelltgt{One creature or object}}{\spellrng{\rngmed}}
\spelldur{\durshort \dismissable}
\spellattack{Reflex negates}
\spelleffect This spell creates a globe of shimmering force. The sphere persists for the spell's duration, containing any creatures or objects held inside, provided they are small enough to fit within the diameter of the sphere. It is not subject to damage of any sort.
\par The subject may struggle, but the sphere cannot be physically moved either by people outside it or by the struggles of those within.
\spellnotes The sphere can only be affected a \spell{disintegrate} spell, a targeted \spell{dispel magic} spell, or similar effects. These effects destroy the sphere without harm to the subject. Nothing can pass through the sphere, inside or out, though the subject can breathe normally.
\spellsr{Yes (Reflex)}

\spellsection{Resist Energy}{2}
\spellinfo{Abjur (Shielding)}{Arcane, Divine, Nature, Protection}
\spelltwocol{\spelltgt{One creature}}{\spellrng{Touch}}
\spelldur{\durlong or until discharged}
\spelleffect The subject gains energy damage reduction 10 against whichever of the five energy types that you select: acid, cold, electricity, fire, or sonic. This damage reduction increases by 1 per caster level above 4th.
\par The spell can absorb a maximum amount of damage equal to 10 points per caster level. After it absorbs its maximum amount of damage, the spell ends.
\spellnotes This spell's damage reduction allows the subject to ignore the first 10 energy damage it takes each round of the appropriate type.

Resist energy absorbs only damage. The subject could still suffer unfortunate side effects. The spell protects the recipient's equipment as well.
\par \spellindirect{resist energy}{Resist energy} overlaps (and does not stack with) \spell{protection from energy}. If a character is shielded by both spells, the \spellindirect{protection from energy}{protection} spell absorbs damage until its power is exhausted. A character can only be affected by one \spell{resist energy} spell at once.
\spellsr{Yes (Fortitude)}

\spellsectioncomma{Resist Energy}{Greater}{4}
\spellinfo{Abjur (Shielding)}{Arcane, Divine, Nature}
\spelleffect This spell functions like \spell{resist energy}, except that the creature gains protection from all five energy types at once. The spell can absorb a total amount of damage equal to 10 points per caster level.
\spellnotes A character can only be affected by one \spell{resist energy} spell at once.

\spellsection{Retributive Brilliance}{5}
\spellinfo{Abjur/Illus (Figment, Shielding)}{Arcane}
\spelltwocol{\spelltgt{One creature; see text}}{\spellrng{\rngclose; see text}}
\spelldur{\durshort or until discharged}
\spellsr{Yes (Fortitude)}
\begin{spelltarget}{One creature}
    \spelleffect The target is protected by a retributive shield that can be activated when the subject is attacked. Activating the triggered attack requires an immediate action from either you or the subject, and discharges the spell.
    \begin{spelltrigger}{Creature within \rngclose range of the target attacks it with a melee weapon}
        \begin{spelltarget}*{Attacking creature}[Magic vs. Reflex]
            \spellsuccess The target is \dazzled for 5 rounds. If it is \bloodied, it is also \blinded for 5 rounds.
            \spellfailure The target is dazzled for 5 rounds.
        \end{spelltarget}
    \end{spelltrigger}
\end{spelltarget}

\spellsection{Retributive Shield}{4}
\spellinfo{Abjur/Necro (Life, Shielding)}{Arcane}
\spelltwocol{\spelltgt{One creature}}{\spellrng{\rngclose/\rngmed}}
\spelldur{\durshort}
\spelleffect A subject within \rngclose range gains physical damage reduction 8/life. This damage reduction increases by 1 per two caster levels above 8th. In addition, the spell reflects the damage back at the creature's attackers. Any creature within \rngmed range of the subject that attacks it takes life damage equal to the amount of damage resisted by this spell.
\spellnotes This spell's damage reduction allows the subject to ignore the first 8 physical damage it takes each round. If it is hit by an attack that deals life damage, such as \spell{crush life}, it cannot use its damage reduction for 1 round.
\spellsr{Yes (Will)}

\spellsection{Retrieve Ally}{2}
\spellinfo{Conj (Translocation) [Teleportation]}{Conj}
\spelltwocol{\spelltgt{One willing creture}}{\spellrng{\rngmed}}
\spelleffect You teleport the target into a free space adjacent to you. You must have line of sight and line of effect to the destination. If you accidentally attempt to teleport the creature into an invalid location, the spell simply fails.
\spellsr{Yes (Will)}

\spellsection{Retrieve Object}{1}
\spellinfo{Conj (Translocation) [Teleportation]}{Conj}
\spelltwocol{\spelltgt{One object you can hold or carry in one hand, weighing up to 10 lb./caster level}}{\spellrng{\rngclose}}
\spellattack{None (object)}
\spelleffect You teleport an item you can see within range directly to your hand. If the object is attended, this spell automatically fails.
\spellsr{Yes (Will)}

\spellsectioncomma{Retrieve Object}{Greater}{5}
\spellinfo{Conj (Translocation) [Teleportation]}{Conj}
\spellrng{\rngmed}
\spellattack{Magic vs. Will (object)}
\spelleffect This spell functions like \spell{retrieve}, except that if the object is attended, it comes to your hand if you succeed on a Will attack against the attending creature.

\spellsection{Revelation}{9}
\spellinfo{Div (Awareness, Knowledge)}{Arcane, Div, Knowledge}
\spelltwocol{\spelltgt{One creature}}{\spellrng{\rngmed}}
\spelldur{\durshort}
\spelleffect You grant the target a powerful revelatory vision of a possible future. This spell has different effects depending on the version chosen. Creatures without an Intelligence score are not affected by this spell.
\par \subspell{Revelation of Destruction} You inflict a vision of a terrible future upon the target. It takes a \minus4 penalty to attacks, defenses, and checks as it struggles to avoid the certainty of its own doom. If it becomes \bloodied, these penalties increase by 2.
\par \subspell{Revelation of Prowess} You show the target a vision of its success in the combat to come. It gains the benefits of a \spell{greater precognition} spell.
\par \subspell{Revelation of Truth} You show the target the truth of the world around it. It gains the benefits of a \spell{true seeing} spell.
\spellsr{Yes (Will)}

\spellsection{Reverse Gravity}{8}
\spellinfo{Trans}{Air, Arcane, Trickery}
\spelltwocol{\spellzone{\arealarge radius}}{\spellrng{\rngclose}}
\spelldur{Concentration (up to 5 rounds)}
\spelleffect This spell reverses gravity in an area, causing all unattached objects and creatures within that area to fall upward and reach the top of the area in 1 round. If some solid object (such as a ceiling) is encountered in this fall, falling objects and creatures strike it in the same manner as they would during a normal downward fall. If an object or creature reaches the top of the area without striking anything, it remains there, oscillating slightly, until the spell ends. At the end of the spell duration, affected objects and creatures fall downward.
  %an initiative check belongs here \par A creature caught in the area can attempt a Reflex save to react to the shift in gravity. Common reactions include securing oneself if possible, or jumping to reach more stable ground.
\spellnotes Creatures who can fly or levitate can keep themselves from falling, though the shift in gravity can be disorienting. A creature that reacts by jumping does not actually move until its turn, but it moves in the direction of its jump, rather than simply falling upwards.

\spellsection{Revivify}{5}
\spelldesc{You reconnect a corpse's soul with its body before the soul has completely passed on.}
\spellinfo{Necro (Life, Soul)}{Divine}
\spellcmp{Verbal, Somatic, and Material}
\spelltwocol{\spelltgt{One dead creature}}{\spellrng{Touch}}
\spelleffect This spell restores a creature to life like the \spell{raise dead} ritual, except that the affected creature suffers no negative effects for having died. However, the spell must be cast within one round of the creature's death per four caster levels. After that time, it has no effect (and the material components are not consumed).

The creature has 0 hit points and 1 point of critical damage (but is stable) after being restored to life.
\spellmat{Diamonds worth at least 500 gp.}

\spellsection{Righteous Might}{5}
\spellinfo{Trans (Augment, Polymorph) [Size-Affecting]}{Divine, Good, Strength}
\spelltwocol{\spelltgt{You}}{\spellrng{Personal}}
\spelleffect This spell functions like \spell{enlarge person}, except that it affects only you, regardless of your creature type, and does not penalize your Dexterity. In addition, you gain physical damage reduction equal to 10 \add 1 per two caster levels above 10th. This damage reduction is overcome by evil attacks if you are good or neutral, and by good attacks if you are evil.
\spellnotes This spell's damage reduction allows the subject to ignore the first 10 physical damage it takes each round. If it is hit by an attack that deals appropriately aligned damage, such as a weapon affected by \spell{align weapon} or a spell with the appropriate descriptor, it cannot use its damage reduction for 1 round.
Multiple magical effects that increase size do not stack.

\spellsection{Sanctuary}{1}
\spellinfo{Abjur/Ench (Compulsion, Shielding) [Mind-Affecting]}{Arcane, Divine, Protection}
\spelltwocol{\spelltgt{One creature}}{\spellrng{Touch}}
\spelldur{\durshort}
\spellattack{Magic vs. Will; see text}
\spelleffect When you cast this spell, and at the start of each of your turns, you make a Will attack. Any creature attempting to physically attack the subject who cannot resist your attack is unable to complete the attack, and it has no effect on the subject.

If the subject takes any offensive action, the spell immediately ends.

\spellnotes This is considered a mind-affecting effect on any creature that attempts to attack the subject. Creatures immune to mind-affecting effects can attack the subject freely.

\pdfbookmark[2]{S}{SpellDescriptionsS}
\begin{comment}
\subsubsection{S}
\end{comment}
\spellsr{Yes (Will)}

\spellsection{Scintillating Pattern}{8}
\spelldesc{You create a massive spread of colorful lights that spin and whirl in a complex pattern that bewilders your foes.}
\spellinfo{Ench/Illus (Compulsion, Figment) [Light, Mind-Affecting, Sight-Dependent]}{Arcane}
\spellzone{\arealarge radius centered on you}
\spelldur{\durshort}
\spelleffect All enemies within the spell's area are bewildered for as long as they can see the lights, and for 5 rounds thereafter. In addition, the area is illuminated in bright light out to a 100 ft. radius, and dim light extends an additional 100 ft. beyond that.
\spellnotes Your allies, and creatures unable to see the lights, are unaffected.
\spellsr{Yes (Will)}

\spellsection{Scorching Ray}{2}
\spelldesc{You blast your enemies with fiery rays.}
\spellinfo{Evoc (Energy) [Fire]}{Arcane, Fire}
\spelltwocol{\spelltgts{One or more objects or creatures}}{\spellrng{\rngclose}}
\spelldmg{4d6 fire damage \add d6 per two caster levels above 4th}
\spelleffect You may fire up to three rays at the same or separate targets. Each ray requires a Reflex attack to hit. You may split the damage among the rays as you choose. The rays may be fired at the same or different targets, but all must be aimed at targets within 30 feet of each other and fired simultaneously. Precision damage can only be applied with one of the rays.
\spellsr{Yes (Reflex)}

\spellsection{Sea of Fog}{8}
\spellinfo{Conj (Creation) [Fog]}{Arcane, Nature}
\spellzone{200 ft. radius cylinder centered on you, 50 ft. high}
\spelleffect This spell functions like \spell{obscuring mist}, except that the effect is much larger.
\spellnotes A severe wind disperses the fog within 1 minute, a windstorm disperses it within 5 rounds, and a hurricane disperses it within a round.

\spellsection{Searing Light}{3}
\spelldesc{You channel divine power into a searing blast of light that erupts your palm, striking your foe.}
\spellinfo{Evoc (Channeling) [Light]}{Divine}
\spelltwocol{\spelltgt{One creature or object}}{\spellrng{\rngclose}}
\spelldur{Instantaneous and see text}
\spelldmg{6d6 divine damage \add d6 per two caster levels above 6th; see text}
\spellattack{Magic vs. Reflex}
\spelleffect You make a Reflex attack against the target to deal damage and dazzle it for 1 round. You deal extra damage equal to your caster level to undead creatures and to creatures particularly vulnerable to bright light. If you hit an undead creature particularly vulnerable to bright light, you deal extra damage equal to twice your caster level, and the creature is blinded for 1 round instead of being dazzled.
\spellnotes A dazzled creature has a 20\% miss chance on all physical attacks and takes a \minus4 penalty to Spot checks. It is also unable to see with darkvision.
\spellsr{Yes (Reflex)}

\spellsection{See Invisibility}{2}
\spellinfo{Div (Revelation)}{Arcane}
\spelltwocol{\spelltgt{One creature}}{\spellrng{Touch}}
\spelldur{\durlong \dismissable}
\spelleffect You grant the touched creature the ability to see any objects or beings that are invisible within its range of vision, as well as any that are ethereal, as if they were normally visible. Such creatures are visible as translucent shapes, allowing the target to easily discern the difference between visible, invisible, and ethereal creatures.
\spellnotes The spell does not reveal the method used to obtain invisibility. It does not reveal illusions or enable you to see through opaque objects. It does not reveal creatures who are simply hiding, concealed, or otherwise hard to see.
This spell can be made permanent with a \spell{permanency} ritual.

\spellsection{Seeming}{4}
\spellinfo{Illus (Glamer) [Unreal]}{Arcane, Trickery}
\spelltwocol{\spelllimit{\areamed radius}}{\spellrng{\rngclose}}
\spelltgts{One creature per caster level in the area}
\spelldur{\durlong \dismissable}
\spelleffect This spell functions like \spell{disguise self}, except that it affects multiple creatures. Affected creatures resume their normal appearances if slain.
\spellsr{Yes}

\spellsection{Shadow Body}{7}
\spellinfo{Illus/Trans (Polymorph, Shadow)}{Arcane}
\spelltwocol{\spelltgt{You}}{\spellrng{Personal}}
\spelldur{\durmed \dismissable}
\spelleffect Your body and all your equipment are subsumed by your shadow. As a living shadow, you blend perfectly into any other shadow and vanish in darkness. You appear as an unattached shadow in areas of full light.
\par You can move at your normal speed, on any surface, including walls and ceilings, as well as across the surfaces of liquids -- even up the face of a waterfall.
\par You become perfectly flat, potentially allowing you to move into locations you would not normally be able to move into.
\par While in your shadow body, you gain damage reduction 15/solar and darkvision out to 60 feet. You are immune to ability damage, disease, drowning, and poison. You take only half damage from acid, electricity, and fire of all kinds.
\par While affected by this spell, you can be detected by spells that read thoughts, life, or presences (including true seeing), or if you make suspicious movements in lighted areas.
\par You cannot harm anyone physically or manipulate any objects, but you can use your spells normally. Doing so may attract notice, but if you remain in a shadowed area, you get a \plus15 enhancement bonus on your Hide check to remain unnoticed.
\spellnotes This spell's damage reduction allows the subject to ignore the first 15 physical damage it takes each round. If it is hit by an attack that deals solar damage, such as \spell{sunbeam}, it cannot use its damage reduction for 1 round.

\spellsection{Shadow Conj}{7}
\spelldesc{You use material from the Plane of Shadow to shape quasi-real illusions of one or more creatures, objects, or forces.}
\spellinfo{Illus (Shadow)}{Illus}
\spelltwocol{\spelltgtorarea{See text}}{\spellrng{See text}}
\spelldur{See text}
\spellattack{Will disbelief (if interacted with); varies, see text}
\spelleffect Shadow conjuration can mimic any non-restricted sorcerer or wizard conjuration spell of 5th level or lower. If you summon a creature, as with the \spell{summon monster} spells, you may only summon a creature that you know how to summon with such a spell.
\par When you cast this spell, you make a Will attack in addition to any other attacks required for the spell. This attack is made against any creature that interacts with the conjured object, force, or creature. If you fail, the creature disbelieves the spell.
\par A creature that disbelieves the spell takes only half damage from the attack. If the disbelieved attack has a special effect other than damage, that effect is half as strong (if applicable) or only half as likely to occur.
\par A shadow creature has half the hit points of a normal creature of its kind (regardless of whether it's disbelieved). It deals normal damage and has all normal abilities and weaknesses. Against a creature that recognizes it as a shadow creature, however, the shadow creature deals half damage, and all special abilities that do not deal lethal damage are only 50\% likely to work. (Roll for each use separately.)
\spellnotes A creature that disbelieves the effect sees the shadow conjurations as transparent images superimposed on vague, shadowy forms.
\par When you learn this spell, you choose two creatures from the 5th-level list or lower on \tref{Summon Monster List}. You can summon those creatures with this or any \spell{summon monster} spell.
\par Objects are always considered to disbelieve this spell.
\spellsr{Yes (Will); See text}

\spellsection{Shadow Evoc}{5}
\spellinfo{Illus (Shadow)}{Illus}
\spelltwocol{\spelltgtorarea{See text}}{\spellrng{See text}}
\spelldur{See text}
\spellattack{Will disbelief (if interacted with); varies, see text}
\spelleffect You tap energy from the Plane of Shadow to cast a quasi-real, illusory version of a non-restricted sorcerer or wizard evocation spell of 3rd level or lower. (For a spell with more than one level, use the best one applicable to you.)
\par When you cast this spell, you make a Will attack in addition to any other attacks required for the spell. This attack is made against any creature that interacts with the conjured object, force, or creature. If you fail, the creature disbelieves the spell.
\par A creature that disbelieves the spell takes only half damage from the attack. If the disbelieved attack has a special effect other than damage, that effect is half as strong (if applicable) or only half as likely to occur.
\spellnotes A creature that disbelieves the effect sees the shadow conjurations as transparent images superimposed on vague, shadowy forms. Objects are always considered to disbelieve this spell.
\spellsr{Yes (Will); see text}

\spellsection{Shadow Puppet}{9}
\spellinfo{Conj/Illus (Shadow, Translocation) [Planar, Unreal]}{Illus}
\spelltwocol{\spelltgt{You}}{\spellrng{Personal/\rngfar; see text}}
\spelldur{\durmed}
\spellattack{None/Magic check vs. Perception and Will (if interacted with)}
\spelleffect You step into the Plane of Shadow (as \spell{shadow walk}, a planar translocation effect), and at the same time, you create a quasi-real, illusory version of yourself (as \spell{project image}, an unreal shadow effect). The double appears superimposed over your body so that observers don't notice an image appearing and you disappearing. You can then control the image and cast spells through it even though you are on a different plane.
\spellnotes If the image moves farther than \rngfar range away from where it was originally created, or if you leave the Plane of Shadow, the image ceases to exist.

\spellsection{Shadow Umbra}{8}
\spellinfo{Abjur/Illus (Glamer, Shadow, Shielding) [Planar]}{Arcane}
\spelltwocol{\spelltgt{One creature}}{\spellrng{\rngclose}}
\spelldur{\durshort}
\spelleffect The subject is protected by a shadowy umbra that connects directly to the Plane of Shadow. All attacks that would affect the creature, including magical and supernatural attacks, have a 50\% chance to be absorbed by the umbra. Attacks absorbed by the umbra have no effect on the subject. The umbra is selective, and does not inhibit beneficial effects.

Whenever the umbra absorbs an attack, it alters the creature's appearance (including smell, sound, and other senses, as appropriate) with a glamer. This causes the creature to seem as if were affected by the attack. Outside observers have no way of knowing which attacks were absorbed by the umbra unless they can disbelieve the illusion. The spell does not attempt to conceal extraordinary attacks which cannot be disguised, such as attacks which would destroy the creature's body.
\spellsr{Yes (Will)}

\spellsection{Share Pain}{2}
\spellinfo{Abjur/Necro (Life, Shielding)}{Arcane, Divine, Protection}
\spelltwocol{\spelltgts{You and one willing creature}}{\spellrng{\rngmed}}
\spelldur{\durlong \dismissable}
\spelleffect This spell creates a connection between you and a willing subject. As you cast the spell, you decide which creature will be protected. When the protected creature would take damage to its hit points, it instead takes half of that damage (rounded down), and you lose hit points equal to the other half of the damage (rounded up).

If the subject is out of range of you, the effect is suppressed until the subject returns within the spell's range.
\spellnotes The loss of hit points caused by this spell is not damage, and is not affected by damage reduction or other abilities which affect damage. When this spell ends, subsequent damage is no longer divided between the subject and you, but damage already shared is not reassigned.
\spellsr{Yes (Will)}

\spellsectioncomma{Share Pain}{Forced}{3}
\spellinfo{Abjur/Necro (Life, Shielding)}{Arcane, Divine}
\spelltgts{You and one other creature}
\spellattack{Magic vs. Will}
\spelleffect This spell functions like \spell{share pain}, except that you can share your damage with an unwilling creature if you succeed on a Will attack.

\spellsection{Shield of Faith}{1}
\spelldesc{You create a shimmmering, magical shield that protects your ally as long as you maintain faith.}
\spellinfo{Abjur (Shielding)}{Divine, Protection}
\spelleffect This spell functions like \spell{mage armor}, except that it only conjures shields rather than body armor. In addition, it not a force effect, so it does not protect against incorporeal attacks.
\spelleffect You can maintain concentration on this spell as a swift action.

\spellsection{Shield of Law}{8}
\spellinfo{Abjur (Shielding) [Lawful]}{Divine, Law}
\spellcmp{Verbal, Somatic, and Focus}
\spelllimit{\areamed radius centered on you}
\spelltgts{Five creatures in the area}
\spelldur{\durshort \dismissable}
\spellattack{See text}
\spelleffect A dim, blue glow surrounds the subjects, protecting them from attacks, granting them resistance to spells cast by chaotic creatures, and slowing chaotic creatures when they strike the subjects. This abjuration has three effects.
\par First, each subject gains a \plus5 enhancement bonus to its defenses.
\par Second, each subject gains spell resistance 10 against chaotic spells and spells cast by chaotic creatures.
\par Third, at the end of each round, all chaotic creatures within \rngclose range of the subject that attacked the subject with their body or a melee weapon that round take 4d6 points of damage. A creature that attacks multiple creatures shielded by this spell can take this damage multiple times.
\spellfocus{A tiny reliquary containing some sacred relic, such as a scrap of parchment from a lawful text. The reliquary costs at least 250 gp.}
\spellsr{Yes (Will)}

\spellsection{Shillelagh}{1}
\spellinfo{Trans}{Nature}
\spelltwocol{\spelltgt{One touched nonmagical oak club or quarterstaff}}{\spellrng{Touch}}
\spelldur{\durshort}
\spellattack{Magic vs. Will (object)}
\spelleffect Your own nonmagical club or quarterstaff becomes a weapon with a \plus2 enhancement bonus on attack and damage rolls. \spellbonusscalingdescription (A quarterstaff gains this enhancement for both ends of the weapon.) In addition, the weapon deals damage as if it were one size category larger (a Small club or quarterstaff so transmuted deals 1d6 points of damage, a Medium 1d8, and a Large 1d10).
\spellnotes These effects only occur when the weapon is wielded by you. If you do not wield it, the weapon behaves as if unaffected by this spell.
\spellsr{Yes (Will)}

\spellsection{Shocking Grasp}{1}
\spelldesc{You deliver a powerful electrical shock to your foe.}
\spellinfo{Evoc (Energy) [Electricity]}{Arcane, Destruction}
\spelltwocol{\spelltgt{Creature or object touched}}{\spellrng{Touch}}
\spelldmg{2d6 electricity damage \add d6 per two caster levels above 2nd}
\spellattack{None/Magic vs. Fortitude}
\spelleffect You make a Reflex attack to deal damage to the target. If you hit, you also make a Fortitude attack to stagger it for 1 round. You gain a \plus2 bonus to attack with this spell if the target is wearing metal armor or otherwise has a significant quantity of metal.
\spellnotes A staggered character may take a single move action or standard action each round, but not both. She cannot take full-round actions, but she may take swift actions. In addition, she is \vulnerable, causing her to take a \minus2 penalty on attacks, defenses, and checks.
\spellsr{Yes (Fortitude)}

\spellsection{Shout}{4}
\spelldesc{You emit an ear-splitting yell that deafens and damages creatures in its path.}
\spellinfo{Evoc (Energy) [Sonic]}{Arcane, Destruction, Strength}
\spellcmp{Verbal only}
\spellburst{\areamed cone}
\spelldmg{4d6 sonic damage \add d6 per four caster levels above 8th; see text}
\spellattack{Fortitude half/Magic vs. Fortitude}
\spelleffect You make a Fortitude attack to deal damage to everything in the area. A successful attack also deafens the target for 5 rounds, while a failed attack deals half damage. You gain a \plus5 bonus to attack against brittle or crystalline objects and creatures.
\spellsr{Yes (Fortitude)}

\spellsectioncomma{Shout}{Greater}{7}
\spellinfo{Evoc (Energy) [Sonic]}{Arcane, Destruction, Strength}
\spellburst{\arealarge cone}
\spelldmg{7d6 sonic damage \add d6 per four caster levels above 14th; see text}
\spellattack{Fortitude partial or Reflex negates (object); See text}
\spelleffect This spell functions like \spell{shout}, except that it is larger.

\spellsection{Shrink Item}{3}
\spellinfo{Trans (Alteration)}{Trans}
\spelltwocol{\spelltgt{One Small (or larger) nonmagical object; see text}}{\spellrng{Touch}}
\spelldur{24 hours; see text}
\spellattack{Magic vs. Will (object)}
\spelleffect You are able to shrink one nonmagical item to 1/16 of its normal size in each dimension (to about 1/4,000 the original volume and mass). This change effectively reduces the object's size by four categories. Optionally, you can also change its now shrunken composition to a clothlike one. Objects changed by a \spell{shrink item} spell can be returned to normal composition and size merely by tossing them onto any solid surface or by a word of command from the original caster. The object must be resting on a stable surface to return to its original size; if the command word is spoken while the object is not stable (such as while it is in the air), the object returns to its original size as soon as it finds a resting point. Even a burning fire and its fuel can be shrunk by this spell. Restoring the shrunken object to its normal size and composition ends the spell.
\par You can shrink a Medium object at 8th caster level, a Large object at 12th caster level, a Huge object at 16th caster level, or a Gargantuan object at 24th caster level.
\spellnotes This spell can be made permanent with a \spell{permanency} ritual, in which case the affected object can be shrunk and expanded an indefinite number of times, but only by the original caster. If you recast the spell each day on an object, you can keep it at its small size indefinitely.
\spellsr{Yes (Will)}

\spellsection{Silence}{2}
\spellinfo{Illus (Glamer)}{Divine, Trickery}
\spelltwocol{\spelltgt{One creature or object}}{\spellrng{\rngmed}}
\spelldur{\durshort \dismissable}
\spellattack{Magic vs. Will (object)}
\spelleffect If you succeed on a Will attack, the subject becomes unable to make noise. Any noises it makes are inaudible to other creatures. When you cast the spell, you may choose whether the subject can still hear itself normally, potentially causing it to be unaware of the effect of the spell.

Extraordinarily loud noises, such as the yell of a giant or a sonic attack, are merely muffled by the spell, not completely silenced. The DC to hear such sounds produced by the subject is increased by 40.
\spellnotes Spellcasters unable to hear themselves cast are treated as deafened, and suffer a 20\% chance of spell failure when casting spells with verbal components.
\spellsr{Yes (Will)}

\spellsection{Silent Image}{2}
\spellinfo{Illus (Figment) [Unreal]}{Illus}
\spelltwocol{\spellzone{\areamed radius}}{\spellrng{\rngmed}}
\spelldur{\durshort}
\spellline
\spelleffect This spell creates the visual illusion of an object, creature, or force within the area, as determined by you. The illusion does not create sound, smell, texture, or temperature. As a standard action, you can concentrate to alter the image within the area.
\spellnotes Creatures can recognize the figment as unreal by interacting with it physically, or by making a Perception check against a DC equal to 10 \add half your caster level \add your casting attribute. A creature gets a \plus10 bonus on this Perception check when using senses which should be present in the figment, but which are missing.

A creature faced with definitive proof that the figment is unreal can disbelieve it, treating it as if it were not there.

\spellsection{Skysmite}{6}
\spelldesc{You call down lightning from the heavens, unerringly striking your foes, even if you cannot see them.}
\spellinfo{Evoc (Energy) [Electricity]}{Air, Arcane, Destruction, Nature}
\spelltwocol{\spelllimit{\areamed radius}}{\spellrng{\rngext}}
\spellburst{\arealarge vertical line of lightning, 5 ft. wide}
\spelldmg{12d6 electricity damage \add d6 per two caster levels above 12th}
\spellattack{Magic vs. Reflex}
\spelleffect Lightning strikes where you direct, allowing you to make a Reflex attack to deal damage to everything in the burst. A failed attack deals half damage. If no creatures or objects lie in its path, the lightning will instead strike the closest occupied square within a \areamed radius limit.
\spellnotes \spell{Invisibility} and other forms of concealment do not protect creatures from the lightning, but it does not differentiate between friend, foe, and inanimate object.
\spellsr{Yes (Reflex)}

\spellsection{Slay Living}{6}
\spelldesc{Your hand seethes with an eerie dark fire as you reach out to touch your foe, instantly snuffing out his life.}
\spellinfo{Necro (Life) [Death]}{Death, Divine}
\spelltwocol{\spelltgt{One living creature}}{\spellrng{Touch}}
\spellattack{Magic vs. Fortitude}
\begin{spellhealthy}
    The target is staggered for 5 rounds. It can take a move action or a standard action each round, but not both.
\end{spellhealthy}
\begin{spellblood}
    The target loses all its hit points and takes 9 critical damage, causing it to begin dying.
\end{spellblood}
\spellnotes A staggered character may take a single move action or standard action each round, but not both. She cannot take full-round actions, but she may take swift actions. In addition, she is \vulnerable, causing her to take a \minus2 penalty on attacks, defenses, and checks.
\spellsr{Yes (Fortitude)}

\spellsection{Sleep}{1}
\spellinfo{Ench (Compulsion) [Mind-Affecting, Sleep]}{Arcane}
\spelltwocol{\spelltgt{One living creature}}{\spellrng{\rngmed}}
\spelldur{\durshort}
\spellattack{Magic vs. Will}
\spelleffect The subject is fatigued and attempts to go to sleep as soon as possible, though it will not stop fighting to do so. Awakening a creature put to sleep by this spell is difficult, and requires a standard action.
\spellnotes A fatigued character can neither sprint nor charge and is \vulnerable, giving it a \minus2 penalty to attacks, defenses, and checks.
\spellsr{Yes (Will)}

\spellsectioncomma{Sleep}{Mass}{4}
\spellinfo{Ench (Compulsion) [Mind-Affecting, Sleep]}{Arcane}
\spelltgts{Five creatures within a \areamed radius limit}
\spelleffect This spell functions like \spell{sleep}, except that it affects multiple creatures.

\spellsection{Slow}{2}
\spelldesc{You decelerate your enemy's motions, causing her to move and act more slowly than normal.}
\spellinfo{Trans (Temporal)}{Arcane}
\spelltwocol{\spelltgt{One creature}}{\spellrng{\rngmed}}
\spelldur{\durshort}
\spellattack{Magic vs. Will}
\spelleffect A slowed creature can take only a single move action or standard action each turn, but not both. It cannot take full-round actions, but it may take swift actions. Additionally, it takes a \minus2 penalty to physical attacks and defenses, as well as Strength and Dexterity-based checks.
\spellsr{Yes (Will)}

\spellsectioncomma{Slow}{Mass}{6}
\spelldesc{You decelerate your enemies' motions, causing them to move and act more slowly than normal.}
\spellinfo{Trans (Temporal)}{Arcane}
\spelltgts{Five creatures in an \areamed radius}
\spelleffect This spell functions like \spell{slow}, except that it affects multiple creatures.

\spellsection{Solid Fog}{6}
\spellinfo{Conj (Creation)}{Arcane, Druid, Water}
\spelldur{\durmed}
\spelleffect This spell functions like \spell{fog cloud}, but in addition to obscuring sight, the fog is so thick that any creature attempting to move through it progresses at a speed of 5 feet, regardless of its normal speed, and it takes a \minus2 penalty on all melee attack and melee damage rolls. The vapors prevent effective ranged weapon attacks (except for magic rays and the like). A creature or object that falls into solid fog is slowed, so that each 10 feet of vapor that it passes through reduces falling damage by 1d6.
\par A creature in the fog can take a full-round action to make a Strength check, moving 5 feet for every 5 by which the result exceeds DC 0. This movement is affected by any other effects which impede movement, as normal.
\spellnotes A severe wind (31\add mph) disperses the fog in 5 rounds, and a hurricane force wind disperses the fog in 1 round. This spell can be made permanent with a \spell{permanency} ritual. A permanent solid fog dispersed by wind reforms in 10 minutes.

\spellsection{Song of Discord}{6}
\spellinfo{Ench (Compulsion) [Auditory, Mind-Affecting]}{Arcane}
\spellburst{\areamed radius centered on you}
\spelldur{\durshort}
\spellattack{Magic vs. Will}
\spelleffect This spell causes all creatures in the area to turn on each other rather than attack their foes. Each affected creature has a 50\% chance to attack the nearest target each round. (Roll to determine each creature's behavior every round at the beginning of its turn.) A creature that does not attack its nearest neighbor is free to act normally for that round. After each round that a subject is compelled to attack the nearest target, it may make a saving throw to throw off the effect.
\par Creatures forced to attack their fellows employ all methods at their disposal, choosing their deadliest spells and most advantageous combat tactics. They do not, however, harm targets that have fallen unconscious.
\spellnotes Creatures whose level exceeds your caster level are immune to this spell.
\spellsr{Yes (Will)}

\spellsection{Soulrend}{6}
\spelldesc{You attack your foe's soul directly.}
\spellinfo{Necro (Soul)}{Necro}
\spelltwocol{\spelltgt{One living creature}}{\spellrng{\rngfar}}
\spellattack{Will half}
\begin{spellhealthy}
    The target takes 1 Charisma damage per three caster levels.
\end{spellhealthy}
\begin{spellblood}
    The target takes 1 Charisma damage per two caster levels.
\end{spellblood}
\spellnotes A creature with a Charisma of \minus10 is unable to act. Undead can take Charisma damage from this spell despite being immune to ability damage. 
\spellsr{Yes (Will)}

\spellsection{Sound Burst}{2}
\spelldesc{You create a cacophony of sound.}
\spellinfo{Evoc (Energy) [Sonic]}{Arcane}
\spelltwocol{\spellburst{\areasmall radius}}{\spellrng{\rngclose}}
\spelldmg{2d6 sonic damage \add d6 per four caster levels above 4th}
\spellattack{Fortitude half/Magic vs. Fortitude}
\spelleffect You make a Fortitude attack to deal damage to everything in the area. A successful attack also deafens the target for 1 round, while a failed attack deals half damage.
\spellsr{Yes (Fortitude)}

\spellsection{Spell Resistance}{4}
\spellinfo{Abjur (Shielding) [Magic]}{Abjur, Magic, Protection}
\spelltwocol{\spelltgt{One creature}}{\spellrng{\rngclose}}
\spelldur{\durshort}
\spelleffect The subject gains spell resistance against all spells.
\spellnotes A creature with spell resistance may always make a saving throw when a spell is cast on it. If the creature succeeds, the spell has no effect on it. The type of saving throw made is indicated by the spell. If the spell also allows a saving throw of the same type, only one roll is made.
\spellsr{Yes (Will)}

\spellsection{Spelltheft}{5}
\spellinfo{Abjur (Negation) [Magic]}{Abjur, Magic}
\spelleffect This spell functions like \spell{dispel magic}, except that you can choose to gain the effects of any spells you dispel or counterspell as if they had been originally cast on you. The effects last for the remainder of their original durations or for 5 rounds, whichever is shorter. Spells that cannot be cast on you, such as spells which have a range of personal, are simply dispelled.

\spellsectioncomma{Spelltheft}{Greater}{8}
\spellinfo{Abjur (Negation) [Magic]}{Abjur}
\spelleffect This spell functions like \spell{greater dispel magic}, except that you can choose to gain the effects of any spells you dispel or counterspell as if they had been originally cast on you. The effects last for the remainder of their original durations or for 5 rounds, whichever is shorter. Spells that cannot be cast on you, such as spells which have a range of personal, are simply dispelled.

\spellsection{Spell Turning}{7}
\spellinfo{Abjur (Shielding) [Magic]}{Arcane, Magic, Protection}
\spelltwocol{\spelltgt{You}}{\spellrng{Personal}}
\spelldur{\durlong or until expended}
\spelleffect Spells and spell-like effects targeted on you are turned back upon the original caster. The abjuration turns only spells that have you as a target. Effect and area spells are not affected. Spell turning also fails to stop touch range spells. 
\par From seven to ten (1d4\plus6) spell levels are affected by the turning. The exact number is rolled secretly.
\par When you are targeted by a spell of higher level than the amount of spell turning you have left, that spell is partially turned; both you and the caster each take half damage. For all effects other than damage, there is a 50\% chance that you suffer the effects; otherwise, the caster suffers the effects.
\spellnotes If you and a spellcasting attacker are both shielded by spell turning effects in operation, a resonating field is created.
\par Roll randomly to determine the result.
\begin{dtable}
    \begin{tabularx}{\columnwidth}{l >{\lcol}X}
        \thead{d\%} & \thead{Effect} \\
        01--70 & Spell drains away without effect. \\
        71--80 & Spell affects both of you equally at full effect. \\
        81--97 & Both turning effects are rendered nonfunctional for 1d4 minutes. \\
        98--100 & Both of you go through a rift into another plane.
    \end{tabularx}
\end{dtable}

\spellsection{Spider Climb}{2}
\spellinfo{Trans (Imbuement)}{Arcane, Nature, Travel}
\spelltwocol{\spelltgt{One creature}}{\spellrng{Touch}}
\spelldur{\durmed}
\spelleffect The subject can climb and travel on vertical surfaces or even traverse ceilings as well as a spider does. The affected creature must have its hands free to climb in this manner. The subject gains a climb speed of 20 feet; furthermore, it need not make Climb checks to traverse a vertical or horizontal surface (even upside down). See \pcref{Climbing}, for more details.
\spellsr{Yes (Fortitude)}

\spellsection{Spike Growth}{2}
\spellinfo{Trans (Alteration)}{Nature}
\spelltwocol{\spellzone{\areasmall radius}}{\spellrng{\rngmed}}
\spelldur{\durshort \dismissable}
\spellattack{None/Reflex negates}
\spelleffect Any ground-covering vegetation in the spell's area becomes very hard and sharply pointed. In areas of bare earth, roots and rootlets act in the same way. Typically, spike growth can be cast in any outdoor setting except open water, ice, heavy snow, sandy desert, or bare stone. Any foe moving on foot into or through the spell's area takes 1d4 points of physical piercing damage for each 5 feet of movement through the spiked area. Allies suffer no ill effects.

Whenever a creature is damaged by this spell, you make a Fortitude attack against it. A successful attack makes slows its land speed by one-half. This speed penalty lasts for 12 hours or until the injured creature receives magical healing. Another character can remove the penalty by taking 10 minutes to dress the injuries and succeeding on a Heal check that beats your Fortitude attack.
\spellsr{Yes (Reflex)}

\spellsection{Spike Stones}{4}
\spellinfo{Trans (Alteration)}{Nature}
\spellzone{\areamed radius}
\spelleffect This spell functions like \spell{spike growth}, except that it deals 1d8 physical piercing damage to creatures moving through it and it can also be cast on rocky ground, stone floors, and similar surfaces.

%need to run expected damage calculation
\spellsection{Spiritual Weapon}{2}
\spelldesc{You bring into being a weapon made of pure force which attacks your foes of its own volition.}
\spellinfo{Evoc (Energy) [Force]}{Divine, War}
\spellrng{\rngmed}
\spelldur{\durshort \dismissable}
\spelleffect The weapon created by this spell attacks once each round on your turn. This functions just as if you were attacking with the weapon, except that you use your casting ability in place of your Strength and you never get multiple attacks with the weapon.
\par The weapon attacks the same target until you redirect it (a swift action). The weapon is treated as a separate creature for the purpose of overwhelm penalties.
\par If an attacked creature has spell resistance, you make a spell penetration check the first time the spiritual weapon strikes it. If the weapon is successfully resisted, it cannot harm that creature. If not, the weapon has its normal full effect on that creature for the duration of the spell.
\par The weapon takes the shape of a weapon favored by your deity or a weapon with some spiritual significance or symbolism to you (see below), and has the same threat range and critical multipliers as a real weapon of its form.
\spellnotes The \spell{spiritual weapon} strikes as a spell, not as a weapon, so, for example, ignores physical damage reduction. As a force effect, it can strike incorporeal creatures without the normal miss chance associated with incorporeality. If the weapon goes beyond the spell range, if it goes out of your sight, or if you are not directing it, the weapon returns to you and hovers. Even if the spiritual weapon is a ranged weapon, use the spell's range, not the weapon's normal range increment, and switching targets still is a move action.
\par A \spell{spiritual weapon} cannot be attacked or harmed by physical attacks, but \spell{dispel magic}, \spell{disintegrate}, and similar effects can affect it. A spiritual weapon's physical defenses are all 12 (10 \add \plus2 bonus for Tiny size).
\par The weapon that you get is usually a force replica of any weapon from your deity's weapon group. A cleric without a deity gets a weapon based on his alignment. A neutral cleric without a deity can create a spiritual weapon of any alignment, provided he is acting at least generally in accord with that alignment at the time. The weapon groups associated with each alignment are as follows.
\par Chaos: Axes
\par Evil: Flexible weapons
\par Good: Headed weapons
\par Law: Heavy blades
\spellsr{Yes (Will)}

\spellsection{Stampede}{9}
\spellinfo{Conj (Summoning)}{Nature, Wild}
\spelltime{Full-round action}
\spelltwocol{\spelllimit{\arealarge radius}}{\spellrng{\rngfar}}
\spelldur{\durshort \dismissable}
\spelldmg{9d6 bludgeoning damage \add d6 per four caster levels above 18th}
\spellattack{Magic vs. Reflex; see text}
\spelleffect This spell summons a stampede of nine bison to trample your foes. Creatures trampled by the herd of bison take 1d6 damage per bison in the herd. You can summon one additional bison per four caster levels above 18th.
\par The bison are summoned in a place that you designate within the spell's area, with each creature being summoned in the closest free space to the point of origin. If there is insufficient room for all of the bison to appear while standing on stable ground, the spell will summon fewer bison than the maximum. The herd of bison always moves directly away from you, trampling anything of Large size or smaller that gets in their way. If the herd is thinned to fewer than 5 bison, they stop stampeding and scatter in random directions.
\par The bison do not attack, even if cornered; they will only stampede. At the end of the spell's duration, the bison disappear.
\spellnotes Under normal circumstances, the bison can travel 800 feet over the duration of the spell.

\spellsection{Stinking Cloud}{5}
\spellinfo{Conj (Creation)}{Arcane}
\spellattack{None/Magic vs. Fortitude}
\spelleffect This spell functions like \spell{fog cloud}, except that the fog has a putrid stench. When you cast this spell, and at the start of each of your turns, you make a Fortitude attack to sicken all creatures in the area. The condition lasts as long as the creature remains in the cloud and for 5 rounds after it leaves.
\spellnotes This spell can be made permanent with a \spell{permanency} ritual. A permanent \spell{stinking cloud} dispersed by wind reforms in 10 minutes.

\spellsection{Stoneskin}{4}
\spelldesc{You dramatically toughen a creature's skin, giving it the appearance of stone.}
\spellinfo{Trans (Alteration) [Earth]}{Arcane, Earth, Nature, Protection}
\spelltwocol{\spelltgt{One creature}}{\spellrng{Touch}}
\spelldur{\durshort}
\spelleffect The subject gains a \plus3 enhancement bonus to its armor modifier. This bonus increases to \plus4 at 14th caster level, and to \plus5 at 20th caster level. In addition, it gains physical damage reduction 8/adamantine. This damage reduction increases by 1 per two caster levels above 8th.
\spellnotes This spell's damage reduction allows the subject to ignore the first 8 physical damage it takes each round. If it is hit by an adamantine weapon, it cannot use its damage reduction for 1 round.
\spellsr{Yes (Fortitude)}

\spellsection{Storm of Vengeance}{9}
\spellinfo{Conj/Evoc (Energy, Control, Creation)}{Air, Divine, Nature, War, Water}
\spelltime{Full-round action}
\spelltwocol{\spellzone{360 ft. radius cylinder, 200 ft. high}}{\spellrng{\rngfar}}
\spelldur{Concentration (maximum 10 rounds)}
\spelldmg{Varies}
\spellattack{See text}
\spelleffect This spell creates an enormous black storm cloud. Lightning and crashing claps of thunder appear within the storm. You make a Fortitude attack to deafen each creature beneath the cloud for 5 minutes. Violent rain and wind gusts obscure all sight beyond 100 feet. A creature less than 100 feet away has concealment (\plus4 to physical defenses). Ranged attacks in the area of the storm take a \minus4 penalty, and spells cast in the area are disrupted unless the caster succeeds on a Concentration check against a DC equal to 20 \add double the level of the spell.
\par If you do not maintain concentration on the spell after casting it, the spell ends. If you continue to concentrate, the spell generates new effects in each following round, as noted below. Each effect occurs during your turn.
\par 2nd Round: Acid rains down, dealing 1d10 acid damage to everything in the area.
\par 3rd Round: You call three bolts of lightning down from the cloud. You decide where the bolts strike. No two bolts may strike the same target. Each bolt deals 9d6 electricity damage \add d6 per four caster levels above 18th. You make a Reflex attack to deal damage to each creature in the path of the bolt. A failed attack deals half damage. If you do not direct the lightning bolts, each bolt automatically targets the largest available target in the area.
\par 4th Round: Hailstones rain down, dealing 5d6 bludgeoning damage to all enemies in the area.
\par 5th through 10th Rounds: Acid rains down, dealing 1d10 damage to everything in the area.
\spellsr{Yes (varies)}

\spellsection{Stormlord}{7}
\spelldesc{You surround yourself in a whirlwind which deflects ranged attacks and batters your foes.} 
\spellinfo{Abjur/Evoc (Control, Shielding)}{Air, Nature}
\spelltwocol{\spelltgt{You}}{\spellrng{Personal}}
\spelldur{\durshort \dismissable}
\spelldmg{7d6 bludgeoning damage \add d6 per four caster levels above 14th}
\spellattack{None/Fortitude half}
\spelleffect You gain physical damage reduction 35 against ranged attacks such as projectile weapons and thrown weapons. This damage reduction increases by 1 per caster level above 14th. In addition, you can make a Fortitude attack to deal damage to any creature that strikes you with its body or a melee weapon. A failed attack deals half damage. Each individual creature can take this damage only once per round.
\spellnotes This spell's damage reduction allows the subject to ignore the first 35 physical damage it takes each round from ranged attacks. The saving throw and spell resistance apply against the damage dealt, but not against this spell's other effects.

\spellsection{Strip the Flesh}{7}
\spelldesc{You rend parts of your foe's skin off its body, inflicting grievous wounds and leaving it \vulnerable.}
\spellinfo{Necro (Flesh)}{Arcane}
\spelltwocol{\spelltgt{One creature}}{\spellrng{\rngclose}}
\spelldur{Instantaneous/5 rounds}
\spelldmg{7d10 physical damage \add d10 per four caster levels above 14th}
\spellattack{None/Magic vs. Fortitude}
\spelleffect You deal damage to the target. In addition, if you succeed at a Fortitude attack, for 5 rounds all damage it takes is doubled. This does not double the initial damage dealt by this spell.
\spellnotes A Heal check that beats your Fortitude attack negates the doubling of damage. 
\spellsr{Yes (Fortitude)}

\spellsection{Suggestion}{4}
\spellinfo{Ench (Compulsion) [Language-Dependent, Mind-Affecting, Sound-Dependent]}{Ench}
\spellcmp{V, M}
\spelltwocol{\spelltgt{One living creature}}{\spellrng{\rngclose}}
\spelldur{\durext or until completed}
\spellattack{Magic vs. Will}
\spelleffect You influence the actions of the target creature by suggesting a course of activity (limited to a sentence or two). The suggestion must be worded in such a manner as to make the activity sound reasonable. Asking the creature to do some obviously harmful act automatically negates the effect of the spell. Additionally, any obvious threat, such as someone drawing a weapon, casting a spell, or aiming a ranged weapon at the fascinated creature, grants the creature a new saving throw with a \plus5 bonus.
\par The suggested course of activity can continue for the entire duration. If the suggested activity can be completed in a shorter time, the spell ends when the subject finishes what it was asked to do. You can instead specify conditions that will trigger a special activity during the duration. If the condition is not met before the spell duration expires, the activity is not performed.
\spellnotes A very reasonable suggestion can grant a \plus2 or greater bonus on the Will attack. A creature that resists this spell is immune to all further attempts by the same spellcaster for 24 hours.
\spellsr{Yes (Will)}

\spellsectioncomma{Suggestion}{Mass}{8}
\spellinfo{Ench (Compulsion) [Language-Dependent, Mind-Affecting, Sound-Dependent]}{Ench}
\spelltwocol{\spelllimit{\areamed radius}}{\spellrng{\rngclose}}
\spelltgts{Five creatures in the area}
\spelldur{\durmed}
\spelleffect This spell functions like \spell{suggestion}, except that it can affect multiple creatures and has a shorter duration. The same suggestion applies to all subjects.

\spellsection{Summon Monster I}{1}\hypertarget{spell:summon monster}{}
\spellinfo{Conj (Summoning) [see text]}{Arcane, Divine}
\spelltwocol{\spelltime{Full-round action}}{\spellrng{\rngclose}}
\spelldur{\durshort \dismissable}
\spelleffect This spell summons an extraplanar creature (typically an outsider, elemental, or magical beast native to another plane). It appears where you designate and acts on your next turn. You must spend a swift action each round to control the creature summoned by this spell. If you do, it attacks your opponents to the best of its ability. You can direct the creature not to attack, to attack particular enemies, or to perform other actions if you can communicate with it. If you do not actively control the creature summoned by this spell, it acts according to its nature.
\par When you learn this spell, you choose two creatures from the 1st-level list on \tref{Summon Monster List}. You can summon those creatures with this or any other summon monster spell.
\par A summoned monster cannot summon or otherwise conjure another creature, nor can it use any teleportation or planar travel abilities. Creatures cannot be summoned into an environment that cannot support them.
\par When you use a summoning spell to summon an air, chaotic, earth, evil, fire, lawful, or water creature, it is a spell of that type.

\spellsection{Summon Monster II}{2}
\spellinfo{Conj (Summoning)}{Arcane, Divine}
\spelltwocol{\spelllimit{\areamed radius}}{\spellrng{\rngclose}}
\spelleffect This spell functions like \spell{summon monster I}, except that you can summon one creature from the 2nd-level list or 1d3 creatures of the same kind from the 1st-level list. When you learn this spell, you choose two creatures from the 2nd-level list or lower on \tref{Summon Monster List}. You can summon those creatures with this or any other \spell{summon monster} spell.

\spellsection{Summon Monster III}{3}
\spellinfo{Conj (Summoning)}{Arcane, Chaos, Divine, Evil, Good, Law}
\spelltwocol{\spelllimit{\areamed radius}}{\spellrng{\rngclose}}
\spelleffect This spell functions like \spell{summon monster I}, except that you can summon one creature from the 3rd-level list or 1d3 creatures of the same kind from a lower-level list. When you learn this spell, you choose two creatures from the 3rd-level list or lower on \tref{Summon Monster List}. You can summon those creatures with this or any other \spell{summon monster} spell.

\spellsection{Summon Monster IV}{4}
\spellinfo{Conj (Summoning)}{Arcane, Divine}
\spelltwocol{\spelllimit{\areamed radius}}{\spellrng{\rngclose}}
\spelleffect This spell functions like \spell{summon monster I}, except that you can summon one creature from the 4th-level list or 1d3 creatures of the same kind from a lower-level list. When you learn this spell, you choose two creatures from the 4th-level list or lower on \tref{Summon Monster List}. You can summon those creatures with this or any other \spell{summon monster} spell.

\spellsection{Summon Monster V}{4}
\spellinfo{Conj (Summoning)}{Air, Arcane, Divine, Earth, Fire, Water}
\spelltwocol{\spelllimit{\areamed radius}}{\spellrng{\rngclose}}
\spelleffect This spell functions like \spell{summon monster I}, except that you can summon one creature from the 5th-level list or 1d3 creatures of the same kind from a lower-level list. When you learn this spell, you choose two creatures from the 5th-level list or lower on \tref{Summon Monster List}. You can summon those creatures with this or any other \spell{summon monster} spell.

\spellsection{Summon Monster VI}{6}
\spellinfo{Conj (Summoning)}{Arcane, Chaos, Divine, Evil, Good, Law}
\spelltwocol{\spelllimit{\areamed radius}}{\spellrng{\rngclose}}
\spelleffect This spell functions like \spell{summon monster I}, except you can summon one creature from the 6th-level list or 1d3 creatures of the same kind from a lower-level list. When you learn this spell, you choose two creatures from the 6th-level list or lower on \tref{Summon Monster List}. You can summon those creatures with this or any other \spell{summon monster} spell.

\spellsection{Summon Monster VII}{7}
\spellinfo{Conj (Summoning)}{Arcane, Divine}
\spelltwocol{\spelllimit{\areamed radius}}{\spellrng{\rngclose}}
\spelleffect This spell functions like \spell{summon monster I}, except that you can summon one creature from the 7th-level list or 1d3 creatures of the same kind from a lower-level list. When you learn this spell, you choose two creatures from the 7th-level list or lower on \tref{Summon Monster List}. You can summon those creatures with this or any other \spell{summon monster} spell.

\spellsection{Summon Monster VIII}{7}
\spellinfo{Conj (Summoning)}{Air, Arcane, Divine, Earth, Fire, Water}
\spelltwocol{\spelllimit{\areamed radius}}{\spellrng{\rngclose}}
\spelleffect This spell functions like \spell{summon monster I}, except that you can summon one creature from the 8th-level list or 1d3 creatures of the same kind from a lower-level list. When you learn this spell, you choose two creatures from the 8th-level list or lower on \tref{Summon Monster List}. You can summon those creatures with this or any other \spell{summon monster} spell.

\spellsection{Summon Monster IX}{9}
\spellinfo{Conj (Summoning)}{Arcane, Chaos, Divine, Evil, Good, Law}
\spelltwocol{\spelllimit{\areamed radius}}{\spellrng{\rngclose}}
\spelleffect This spell functions like \spell{summon monster I}, except that you can summon one creature from the 9th-level list or 1d3 creatures of the same kind from a lower-level list. When you learn this spell, you choose two creatures from the 9th-level list or lower on \tref{Summon Monster List}. You can summon those creatures with this or any other \spell{summon monster} spell.

\begin{dtable!*}
    \lcaption{Summon Monster List}
    \begin{tabularx}{\textwidth}{>{\lcol}X c >{\lcol}X c >{\lcol}X c}
        \thead{1st Level} &  & \thead{4th Level} &  & Fiendish monstrous spider, Huge & CE \\
        Celestial dog & LG & Archon, lantern & LG & Fiendish snake, giant constrictor & CE \\
        Celestial owl & LG & Celestial giant owl & LG &  &  \\
        Celestial giant fire beetle & NG & Celestial giant eagle & CG & \thead{7th Level} &  \\
        Celestial porpoise\fn{1} & NG & Celestial lion & CG & Celestial elephant & LG \\
        Celestial badger & CG & Mephit (any)\fn{2} & N & Avoral (guardinal) & NG \\
        Celestial monkey & CG & Fiendish dire wolf & LE & Celestial baleen whale\fn{1} & NG \\
        Fiendish dire rat & LE & Fiendish giant wasp & LE & Djinni (genie) & CG \\
        Fiendish raven & LE & Fiendish giant praying mantis & NE & Elemental, Huge (any)\fn{2} & N \\
        Fiendish monstrous centipede, Medium & NE & Fiendish shark, Large\fn{1} & NE & Invisible stalker & N \\
        Fiendish monstrous scorpion, Small & NE & Yeth hound & NE & Devil, bone & LE \\
        Fiendish hawk & CE & Fiendish monstrous spider, Large & CE & Fiendish megaraptor & LE \\
        Fiendish monstrous spider, Small & CE & Fiendish snake, Huge viper & CE & Fiendish monstrous scorpion, Huge & \\ NE
        Fiendish octopus\fn{1} & CE & Howler & CE & Babau (demon) & CE \\
        Fiendish snake, Small viper & CE &  &  & Fiendish giant octopus\fn{1} & CE \\
        &  & \thead{5th Level} &  & Fiendish girallon & CE \\
        \thead{2nd Level} &  & Archon, hound & K &  &  \\
        Celestial giant bee & LG & Celestial brown bear & LG &  &  \\
        Celestial giant bombardier beetle & NG & Celestial giant stag beetle & LG & \thead{8th Level} &  \\
        Celestial riding dog & NG & Celestial sea cat\fn{1} & NG & Celestial dire bear & LG \\
        Celestial eagle & CG & Celestial griffon & NG & Celestial cachalot whale\fn{1} & NG \\
        Lemure (devil) & LE & Elemental, Medium (any)\fn{2} & CG & Celestial triceratops & NG \\
        Fiendish squid\fn{1} & LE & Achaierai & N & Lillend & CG \\
        Fiendish wolf & LE & Devil, bearded & LE & Elemental, greater (any)\fn{2} & N \\
        Fiendish monstrous centipede, Large & NE & Fiendish deinonychus & LE & Fiendish giant squid\fn{1} & LE \\
        Fiendish monstrous scorpion, Medium & NE & Fiendish dire ape & LE & Hellcat & LE \\
        Fiendish shark, Medium\fn{1} & NE & Fiendish dire boar & LE & Fiendish monstrous centipede, Colossal & NE \\
        Fiendish monstrous spider, Medium & CE & Fiendish shark, Huge & NE & Fiendish dire tiger & CE \\
        Fiendish snake, Medium viper & CE & Fiendish monstrous scorpion, Large & NE & Fiendish monstrous spider, Gargantuan & CE \\
        &  & Shadow mastiff & NE & Fiendish tyrannosaurus & CE \\
        \thead{3rd Level} &  & Fiendish dire wolverine & NE & Vrock (demon) & CE \\
        Celestial black bear & LG & Fiendish giant crocodile & CE &  &  \\
        Celestial bison & NG & Fiendish tiger & CE &  &  \\
        Celestial dire badger & CG &  &  & \thead{9th Level} &  \\
        Celestial hippogriff & CG & \thead{6th Level} &  & Couatl & LG \\
        Elemental, Small (any)\fn{2} & N & Celestial polar bear & LG & Leonal (guardinal) & NG \\
        Fiendish ape & LE & Celestial orca whale\fn{1} & NG & Celestial roc & CG \\
        Fiendish dire weasel & LE & Bralani (eladrin) & CG & Elemental, elder (any)\fn{2} & N \\
        Hell hound & LE & Celestial dire lion & CG & Devil, barbed & LE \\
        Fiendish snake, constrictor  & LE & Elemental, Large (any)\fn{2} & N & Fiendish dire shark\fn{1} & NE \\
        Fiendish boar & NE & Janni (genie) & N & Fiendish monstrous scorpion, Gargantuan & NE \\
        Fiendish dire bat & NE & Chaos beast & CN & Night hag & NE \\
        Fiendish monstrous centipede, Huge & NE & Devil, chain & LE & Bebilith (demon) & CE \\
        Fiendish crocodile & CE & Xill & LE & Fiendish monstrous spider, Colossal & CE \\
        Dretch (demon) & CE & Fiendish monstrous centipede, Gargantuan & NE & Hezrou (demon) & CE \\
        Fiendish snake, Large viper & CE & Fiendish rhinoceros & NE & & \\
        Fiendish wolverine & CE & Fiendish elasmosaurus\fn{1} & CE & &
    \end{tabularx}
    1 May be summoned only into an aquatic or watery environment. \\
    2 Each variety must be learned individually.
\end{dtable!*}

\spellsection{Summon Nature's Ally I}{1}\hypertarget{spell:summon nature's ally}{}
\spellinfo{Conj (Summoning)}{Nature}
\spelltime{Full-round action}
\spellrng{\rngclose}
\spelldur{\durshort \dismissable}
\spelleffect This spell summons a natural creature. It appears where you designate and acts on your next turn. You must spend a swift action each round to control the creature summoned by this spell. If you do, it attacks your opponents to the best of its ability. You can direct the creature not to attack, to attack particular enemies, or to perform other actions if you can communicate with it. If you do not actively control the creature summoned by this spell, it acts according to its nature.
\par When you learn this spell, you choose two creatures from the 1st-level list on \tref{Summon Nature's Ally List}. You can summon those creatures with this or any other \spell{summon nature's ally} spell.
\par A summoned monster cannot summon or otherwise conjure another creature, nor can it use any teleportation or planar travel abilities. Creatures cannot be summoned into an environment that cannot support them.
\par All the creatures on the table are neutral unless otherwise noted.

\spellsection{Summon Nature's Ally II}{2}
\spellinfo{Conj (Summoning)}{Nature}
\spelltwocol{\spelllimit{\areamed radius}}{\spellrng{\rngclose}}
\spelleffect This spell functions like \spellindirect{summon nature's ally i}{summon nature's ally I}, except that you can summon one 2nd-level creature or 1d3 1st-level creatures of the same kind. When you learn this spell, you choose two creatures from the 2nd-level list or lower on \tref{Summon Nature's Ally List}. You can summon those creatures with this or any other \spell{summon nature's ally} spell.

\spellsection{Summon Nature's Ally III}{3}
\spellinfo{Conj (Summoning) [see text]}{Nature, Wild}
\spelltwocol{\spelllimit{\areamed radius}}{\spellrng{\rngclose}}
\spelleffect This spell functions like \spellindirect{summon nature's ally i}{summon nature's ally I}, except that you can summon one 3rd-level creature, 1d3 2nd-level creatures of the same kind, or 1d4\plus1 1st-level creatures of the same kind. When you learn this spell, you choose two creatures from the 3rd-level list or lower on \tref{Summon Nature's Ally List}. You can summon those creatures with this or any other \spell{summon nature's ally} spell.

\spellsection{Summon Nature's Ally IV}{4}
\spellinfo{Conj (Summoning) [see text]}{Nature}
\spelltwocol{\spelllimit{\areamed radius}}{\spellrng{\rngclose}}
\spelleffect This spell functions like \spellindirect{summon nature's ally i}{summon nature's ally I}, except that you can summon one 4th-level creature or 1d3 creatures of the same kind from a lower-level list. When you learn this spell, you choose two creatures from the 4th-level list or lower on \tref{Summon Nature's Ally List}. You can summon those creatures with this or any other \spell{summon nature's ally} spell.

\spellsection{Summon Nature's Ally V}{5}
\spellinfo{Conj (Summoning) [see text]}{Nature}
\spelltwocol{\spelllimit{\areamed radius}}{\spellrng{\rngclose}}
\spelleffect This spell functions like \spellindirect{summon nature's ally i}{summon nature's ally I}, except that you can summon one 5th-level creature or 1d3 creatures of the same kind from a lower-level list. When you learn this spell, you choose two creatures from the 5th-level list or lower on \tref{Summon Nature's Ally List}. You can summon those creatures with this or any other \spell{summon nature's ally} spell.

\spellsection{Summon Nature's Ally VI}{6}
\spellinfo{Conj (Summoning) [see text]}{Nature, Wild}
\spelltwocol{\spelllimit{\areamed radius}}{\spellrng{\rngclose}}
\spelleffect This spell functions like \spellindirect{summon nature's ally i}{summon nature's ally I}, except that you can summon one 6th-level creature or 1d3 creatures of the same kind from a lower-level list. When you learn this spell, you choose two creatures from the 6th-level list or lower on \tref{Summon Nature's Ally List}. You can summon those creatures with this or any other \spell{summon nature's ally} spell.

\spellsection{Summon Nature's Ally VII}{7}
\spellinfo{Conj (Summoning) [see text]}{Nature}
\spelltwocol{\spelllimit{\areamed radius}}{\spellrng{\rngclose}}
\spelleffect This spell functions like \spellindirect{summon nature's ally i}{summon nature's ally I}, except that you can summon one 7th-level creature or 1d3 creatures of the same kind from a lower-level list. When you learn this spell, you choose two creatures from the 7th-level list or lower on \tref{Summon Nature's Ally List}. You can summon those creatures with this or any other \spell{summon nature's ally} spell.

\spellsection{Summon Nature's Ally VIII}{8}
\spellinfo{Conj (Summoning) [see text]}{Nature}
\spelltwocol{\spelllimit{\areamed radius}}{\spellrng{\rngclose}}
\spelleffect This spell functions like \spellindirect{summon nature's ally i}{summon nature's ally I}, except that you can summon one 8th-level creature or 1d3 creatures of the same kind from a lower-level list. When you learn this spell, you choose two creatures from the 8th-level list or lower on \tref{Summon Nature's Ally List}. You can summon those creatures with this or any other \spell{summon nature's ally} spell.

\spellsection{Summon Nature's Ally IX}{9}
\spellinfo{Conj (Summoning) [see text]}{Nature, Wild}
\spelltwocol{\spelllimit{\areamed radius}}{\spellrng{\rngclose}}
\spelleffect This spell functions like \spellindirect{summon nature's ally i}{summon nature's ally I}, except that you can summon one 9th-level creature or 1d3 creatures of the same kind from a lower-level list. When you learn this spell, you choose two creatures from the 9th-level list or lower on \tref{Summon Nature's Ally List}. You can summon those creatures with this or any other \spell{summon nature's ally} spell.

\begin{dtable*}
    \lcaption{Summon Nature's Ally List}
    \begin{tabularx}{\textwidth}{>{\lcol}X >{\lcol}X >{\lcol}X >{\lcol}X}
        \thead{1st Level} & Eagle, giant [NG] & \thead{5th Level} & \thead{7th Level} \\
        Dire rat & Lion & Arrowhawk, adult & Arrowhawk, elder \\
        Eagle (animal) & Owl, giant [NG] & Bear, polar (animal) & Dire tiger \\
        Monkey (animal) & Satyr [CN; without pipes] & Dire lion & Elemental, greater (any)\fn{2} \\
        Octopus\fn{1} (animal) & Shark, Large\fn{1} (animal) & Elasmosaurus\fn{1} (dinosaur) & Djinni (genie) [NG] \\
        Owl (animal) & Snake, constrictor (animal) & Elemental, Large (any)\fn{2} & Invisible stalker \\
        Porpoise\fn{1} (animal) & Snake, Large viper (animal) & Griffon & Pixie\fn{3} (sprite) [NG; with sleep arrows] \\
        Snake, Small viper (animal) & Thoqqua & Janni (genie) & Squid, giant\fn{1} (animal) \\
        Wolf (animal) &  & Rhinoceros (animal) & Triceratops (dinosaur) \\
        & \thead{4th Level} & Satyr [CN; with pipes] & Tyrannosaurus (dinosaur) \\
        \thead{2nd Level} & Arrowhawk, juvenile & Snake, giant constrictor (animal) & Whale, cachalot\fn{1} (animal) \\
        Bear, black (animal) & Bear, brown (animal) & Nixie (sprite) & Xorn, elder \\
        Crocodile (animal) & Crocodile, giant (animal) & Tojanida, adult\fn{1} &  \\
        Dire badger & Deinonychus (dinosaur) & Whale, orca\fn{1} (animal) & \thead{8th Level} \\
        Dire bat & Dire ape &  & Dire shark\fn{1} \\
        Elemental, Small (any)\fn{2} & Dire boar & \thead{6th Level} & Roc \\
        Hippogriff & Dire wolverine & Dire bear & Salamander, noble [NE] \\
        Shark, Medium\fn{1} (animal) & Elemental, Medium (any)\fn{2} & Elemental, Huge (any)\fn{2} & Tojanida, elder \\
        Snake, Medium viper (animal) & Salamander, flamebrother [NE] & Elephant (animal) &  \\
        Squid\fn{1} (animal) & Sea cat\fn{1} & Girallon & \thead{9th Level} \\
        Wolverine (animal) & Shark, Huge\fn{1} (animal) & Megaraptor (dinosaur) & Elemental, elder \\
        & Snake, Huge viper (animalo) & Octopus, giant\fn{1} (animal) & Grig [NG; with fiddle] (sprite) \\
        \thead{3rd Level} & Tiger (animal) & Pixie\fn{3} (sprite) [NG; no special arrows] & Pixie\fn{4} (sprite) [NG; with sleep and memory loss arrows] \\
        Ape (animal) & Tojanida, juvenile\fn{1} & Salamander, average [NE] & Unicorn, celestial charger \\
        Dire weasel & Unicorn [CG] & Whale, baleen\fn{1} &  \\
        Dire wolf & Xorn, minor & Xorn, average & 
    \end{tabularx}
    1 May be summoned only into an aquatic or watery environment. \\
    2 Each variety must be learned individually. \\
    3 Can't cast irresistible dance \\
    4 Can cast irresistible dance \\
\end{dtable*}

\spellsection{Summon Nature's Army}{8}
\spellinfo{Conj (Summoning)}{Nature, Wild}
\spelltwocol{\spelllimit{\areamed radius}}{\spellrng{\rngclose}}
\spelleffect This spell functions like \spellindirect{summon nature's ally i}{summon nature's ally I}, except that you can summon up to one creature per caster level from the 4th-level list or lower.
\par When you learn this spell, you choose a creature from the 4th-level list or lower on the Summon Nature's Ally table. That is the only creature you can summon with this spell.

\spellsection{Sunbeam}{5}
\spelldesc{You evoke a dazzling beam of intense light, blinding your foes with the power of the sun itself.}
\spellinfo{Evoc (Control) [Light]}{Nature}
\spellburst{\arealarge line, 10 ft. wide}
\spelldur{Instantaneous/5 rounds}
\spelldmg{5d6 solar damage \add d6 per four caster levels above 10th; see text}
\spellattack{Magic vs. Reflex/Reflex negates}
\spelleffect You make a Reflex attack to deal damage to everything in the area. A failed attack deals half damage. A failed attack deals half damage. You deal extra damage equal to twice your caster level against any creatures to which sunlight is harmful or unnatural, and they are blinded for 5 rounds if your attack succeeds.
\spellnotes A dazzled creature has a 20\% miss chance on all physical attacks and takes a \minus4 penalty to Spot checks. It is also unable to see with darkvision. 
\spellsr{Yes (Reflex)}

\spellsection{Sunburst}{8}
\spelldesc{You cause a globe of searing radiance to explode silently from a point you select.}
\spellinfo{Evoc (Control) [Light]}{Nature}
\spelltwocol{\spellburst{\areamed radius}}{\spellrng{\rngmed}}
\spelldmg{8d6 solar damage \add d6 per four caster levels above 16th; see text}
\spelleffect This spell functions as \spell{sunbeam}, except that it affects a \areamed radius and deals more damage.

\pdfbookmark[2]{T}{SpellDescriptionsT}
\begin{comment}
\subsubsection{T}
\end{comment}

\spellsection{Telekinesis}{6}
\spelldesc{You move objects or creatures by concentrating on them.}
\spellinfo{Evoc (Control)}{Evoc}
\spelltwocol{\spelltgtortgts{See text}}{\spellrng{\rngmed}}
\spelldur{Concentration, up to \durmed/Instantaneous; see text}
\spellattack{Magic vs. Will (object)/None; see text}
\spelleffect Depending on the version selected, the spell can provide a gentle, sustained force, perform a variety of maneuvers, or exert a single short, violent thrust.
\par \subspell{Sustained Force} As the \spell{telekinetic force} spell.

\par \subspell{Combat Maneuver} As the \spell{telekinetic maneuver} spell.

\par \subspell{Violent Thrust} Alternatively, the spell energy can be spent in a single round, as the \spell{telekinetic thrust} spell.

\spellsection{Telekinetic Force}{4}
\spellinfo{Evoc (Control)}{Evoc}
\spelltwocol{\spelltgt{One object or creature at a time}}{\spellrng{\rngmed}}
\spelldur{Concentration, up to 5 minutes}
\spellattack{Magic vs. Will (object); see text}
\spelleffect You may manipulate objects or creatures at a distance as if you were holding the object in your hands. When doing so, your effective Strength is equal to half your casting attribute, and your effective Dexterity is equal to half your Intelligence. You can move objects at a speed of up to 20 feet per round in any direction.

To affect a creature or an attended object, you must make a successful Will attack. This attack must be repeated each round. If you are prevented from affecting a target in this way, it and any of its possessions are immune to your attempts for the duration of the spell, though you can still attempt to affect other creatures or objects. 
\spellnotes This spell generally moves objects too slowly for them to be used as weapons. However, some indirect weapons, such as crossbows, may be used to attack with this spell. 
\spellsr{Yes (Will)}

\spellsection{Telekinetic Maneuver}{3}
\spellinfo{Evoc (Control)}{Evoc}
\spelltwocol{\spelltgt{One creature}}{\spellrng{\rngmed}}
\spelldur{Concentration, up to \durmed}
\spelleffect Once per round, you can telekinetically attack a foe of your choice. You can perform a disarm, dirty trick, grapple (including a pin, if you have already grappled a foe), shove, or trip attack. Resolve these attempts as normal, except that they don't provoke attacks of opportunity, you use your caster level in place of your base attack bonus, and you use your casting attribute in place of your Strength. In addition, you get a \plus2 enhancement bonus to maneuvers with this spell. \spellbonusscalingdescription
\spellsr{Yes (Will)}

\spellsection{Telekinetic Thrust}{5}
\spellinfo{Evoc (Control)}{Evoc}
\spelltwocol{\spelltgtortgts{Five objects or creatures in a \areamed radius \add one per four caster levels above 8th}}{\spellrng{\rngmed}}
\spellattack{Magic vs. Will (object); see text}
\spelleffect You can throw the affected objects or creatures anywhere within the spell's range. All subjects of this spell must be thrown to the same place. You can hurl up to a total weight of 25 pounds per caster level.
\par You must succeed on physical ranged attacks (one per creature or object thrown) to hit the target of the hurled items with the items, with an attack bonus equal to your caster level \add your casting attribute. Hurled weapons deal their normal damage. Other objects deal damage ranging from 1 point per 25 pounds of weight (for less dangerous objects such as an empty barrel) to 1d6 points per 25 pounds of weight (for hard, dense objects such as a boulder).
\par To affect a creature or attended item, you must make a successful Will attack.
\par If you use this power to hurl a creature against a solid surface, it takes damage as if it had fallen 50 feet (5d6 damage).
\spellsr{Yes (Will)}

\spellsection{Telepathy}{5}
\spellinfo{Div (Communication)}{Arcane}
\spelltwocol{\spelltgt{You}}{\spellrng{Personal}}
\spelldur{\durlong}
\spelleffect You gain telepathy out to a range of 100 feet. This allows you to send mental messages to any creature within range that has a language. Non-telepathic creatures can reply mentally to your messages, but they cannot initiative a telepathic conversation with you.

You can address multiple creatures at once with telepathy, but maintaining separate mental conversations is just as difficult as simultaneously speaking and listening to multiple creatures at the same time. 

\spellsection{Temporal Stasis}{8}
\spellinfo{Trans (Temporal)}{Arcane}
\spelltwocol{\spelltgt{One creature}}{\spellrng{Touch}}
\spelldur{\durshort/Permanent}
\spellattack{None/Magic vs. Will}
\spelleffect The subject is slowed for a \durshort duration.
\begin{spellblood}
    In addition, you make a Will attack against the touched creature. A successful attack causes the subject to be placed into a state of suspended animation. For the creature, time ceases to flow and its condition becomes fixed. The creature does not grow older. Its body functions virtually cease, and no force or effect can harm it. This state persists until the magic is removed (such as by a successful \spell{dispel magic} spell or an \spell{emancipation} spell).
\end{spellblood}
\spellnotes A slowed creature can take only a single move action or standard action each turn, but not both. It cannot take full-round actions, but it may take swift actions. Additionally, it takes a \minus2 penalty to physical attacks and defenses, as well as Strength and Dexterity-based checks.
\spellsr{Yes (Will)}

\spellsection{Power Word Fear}{6}
\spell{You fill your foe with an inescapable fear, forcing it to flee from your presence.}
\spellinfo{Ench (Emotion) [Fear, Mind-Affecting]}{Arcane}
\spelltwocol{\spelltgt{One creature}}{\spellrng{\rngclose}}
\spelldur{\durshort}
\begin{spellhealthy}
    The subject is shaken, causing it to be \vulnerable.
\end{spellhealthy}
\begin{spellblood}
    The subject is frightened.
\end{spellblood}
\spellnotes A \vulnerable character takes a \minus2 penalty on attacks, defenses, and checks. A frightened creature is the same, except that it also flees from the source of its fear as best it can. If unable to flee, it may fight.
\par A character shaken by multiple sources becomes frightened. A character frightened by multiple sources becomes panicked.
\spellsr{Yes (Will)}

\spellsection{Time Stop}{9}
\spellinfo{Trans (Temporal)}{Arcane}
\spelltwocol{\spelltgt{You}}{\spellrng{Personal}}
\spelldur{1d3\plus1 rounds (apparent time); see text}
\spelleffect This spell seems to make time cease to flow for everyone but you. In fact, you step into an alternate timestream, causing you to speed up so greatly that all other creatures seem frozen, though they are actually still moving at their normal speeds. You are free to act for 1d3\plus1 rounds of apparent time. You are still \vulnerable to danger, such as from heat or dangerous gases, but your actions have no effect on anything in the world other than yourself. Objects and creatures appear frozen in place. You cannot cast spells that affect any targets except yourself; the temporal magic is too strong to permit interference from lesser magic, and attempts to cast magic beyond the accelerated time surrounding you simply fail. The only exception is for temporal spells, which can be cast normally inside a \spell{time stop}. The subjects are not affected and do not attempt to resist the effects until the end of the \spell{time stop}, so you do not know whether they are affected by any spells you cast until the effect has expired.
\spellnotes Most spellcasters use the additional time to improve their defenses or flee from combat. You are undetectable while \spell{time stop} lasts. You cannot enter an area protected by an \spell{antimagic field} while under the effect of \spell{time stop}.

\spellsection{Totemic Mind}{2}
\spellinfo{Trans (Augment)}{Arcane, Divine, Nature}
\spelltwocol{\spelltgt{One creature}}{\spellrng{\rngtouch}}
\spelldur{\durshort}
\spelleffect This spell grants the subject the mental power of a totem animal. It has three forms, each of which grants a \plus2 enhancement bonus to a mental attribute.
\par \subspell{Eagle's Splendor} The transmuted creature becomes more persuasive and personally forceful, gaining a bonus to Charisma.
\par \subspell{Fox's Cunning} The transmuted creature becomes smarter, gaining a bonus to Intelligence.
\par \subspell{Owl's Wisdom} The transmuted creature becomes more perceptive, gaining a bonus to Wisdom.
\spellsr{Yes (Will)}

\spellsectioncomma{Totemic Mind}{Greater}{5}
\spellinfo{Trans (Augment)}{Arcane, Divine, Nature}
\spelleffect This spell functions like \spell{totemic mind}, except that it grants a \plus4 enhancement bonus to the chosen attribute instead. Alternately, you can grant the subject a \plus2 enhancement bonus to all mental attributes.

\spellsectioncomma{Totemic Mind}{Mass}{6}
\spellinfo{Trans (Augment)}{Arcane, Divine, Nature}
\spelltwocol{\spelllimit{\areamed radius}}{\spellrng{\rngmed}}
\spelltgts{Five creatures in the area}
\spelleffect This spell functions like \spell{totemic mind}, except that it affects multiple creatures. All affected creatures must gain a bonus to the same attribute.

\spellsection{Totemic Power}{2}
\spellinfo{Trans (Augment)}{Arcane, Divine, Nature, Strength}
\spelltwocol{\spelltgt{One creature}}{\spellrng{\rngtouch}}
\spelldur{\durshort}
\spelleffect This spell grants the subject the physical power of an animal. It has three forms, each of which grants a \plus2 enhancement bonus to a physical attribute.
\par \subspell{Bear's Endurance} The transmuted creature gains greater vitality and stamina, gaining a bonus to Constitution. Hit points gained by a temporary increase in Constitution score are not temporary hit points. They go away when the subject's Constitution drops back to normal. They are not lost first as temporary hit points are.
\par \subspell{Bull's Strength} The transmuted creature becomes stronger, gaining a bonus to Strength.
\par \subspell{Cat's Grace} The transmuted creature becomes more graceful, agile, and coordinated, gaining a bonus to Dexterity.
\spellsr{Yes (Fortitude)}

\spellsectioncomma{Totemic Power}{Greater}{5}
\spellinfo{Trans (Augment)}{Arcane, Divine, Nature, Strength}
\spelleffect This spell functions like \spell{totemic power}, except that it grants a \plus4 enhancement bonus to the chosen attribute instead. Alternately, you can grant the subject a \plus2 enhancement bonus to all physical attributes.

\spellsectioncomma{Totemic Power}{Mass}{6}
\spellinfo{Trans (Augment)}{Arcane, Divine, Nature}
\spelltwocol{\spelllimit{\areamed radius}}{\spellrng{\rngmed}}
\spelltgts{Five creatures in the area}
\spelleffect This spell functions like \spell{totemic power}, except that it affects multiple creatures. Each affected creature must have the same attribute increased. 

\spellsection{Touch of Idiocy}{2}
\spelldesc{With a touch, you reduce the target's mental faculties.}
\spellinfo{Ench (Inhibition) [Mind-Affecting]}{Arcane}
\spelltwocol{\spelltgt{One creature}}{\spellrng{One}}
\spelldur{\durshort}
\spellattack{Will half}
\spelleffect The subject takes a \minus4 penalty to the target's Intelligence, Wisdom, and Charisma scores. This penalty can't reduce any of these scores below \minus9.
\spellnotes This spell's effect may make it impossible for the target to cast some or all of its spells, if the requisite attribute drops below the minimum required to cast spells of that level.
\spellsr{Yes (Will)}

\spellsection{Transmute Any Object}{9}
\spellinfo{Trans (Alteration, Polymorph)}{Arcane}
\spelltwocol{\spelltgt{One creature, or one nonmagical object of up to 1000 cu. ft.}}{\spellrng{\rngmed}}
\spelldur{See text}
\spellattack{Magic vs. Fortitude (object); see text}
\spelleffect This spell can be used to duplicate the effects of \spell{fabricate}, \spell{major creation}, \spell{passwall}, \spell{shape stone}, \spell{transmute flesh and stone}, or \spell{wall of stone}. The object or creature to be transformed must meet any requirements of the spell to be duplicated, except that it must be within \rngmed range.
\spellsr{Yes (Fortitude)}

\spellsection{Transmute Flesh and Stone}{6}
\spellinfo{Trans (Polymorph)}{Arcane, Earth}
\spelltwocol{\spelltgt{One creature or a cylinder of stone from 1 ft. to 3 ft. in diameter and up to 10 ft. long}}{\spellrng{\rngmed}}
\spelldur{\durshort/Instantaneous}
\spelldmg{3d8 damage per round; see text}
\spellattack{Magic vs. Fortitude (object); see text}
\spelleffect This spell has different effects depending on the version chosen.
\par \subspell{Flesh to Stone} The subject is slowed for the duration of the spell, and takes 3d8 physical damage each round as its body gradually turns to stone. If the subject reaches 0 hit points before the spell ends, it becomes a mindless, inert statue, along with all its carried gear. If the statue resulting from this effect is broken or damaged, the subject (if ever returned to its original state) has similar damage or deformities. The creature is not dead, but it is not considered alive either.
\par Only creatures made of flesh are affected by this effect.
\par \subspell{Stone to Flesh} This effect restores a petrified creature to its normal state, restoring life and goods. Any petrified creature, regardless of size, can be restored. A restored creature has as many hit points as it had when it was petrified. Stone which was not originally a petrified creature is unaffected.
\spellnotes A slowed creature can take only a single move action or standard action each turn, but not both. It cannot take full-round actions, but it may take swift actions. Additionally, it takes a \minus2 penalty to physical attacks and defenses, as well as Strength and Dexterity-based checks.
\spellsr{Yes (Fortitude)}

\spellsection{Tree Shape}{2}
\spellinfo{Trans (Polymorph)}{Nature}
\spelltwocol{\spelltgt{You}}{\spellrng{Personal}}
\spelldur{\durext \dismissable}
\spelleffect You become able to assume the form of a Large living tree or shrub or a Large dead tree trunk with a small number of limbs. The closest inspection cannot reveal that the tree in question is actually a magically concealed creature. To all normal tests you are, in fact, a tree or shrub, although a Spellcraft check can reveal a faint transmutation on the tree. While in tree form, you can observe all that transpires around you just as if you were in your normal form, and your hit points and defenses remain unaffected. You gain a \plus10 enhancement bonus to your armor modifier, but you have an effective Dexterity score of \minus10 and cannot move. You are immune to critical hits while in tree form. All clothing and gear carried or worn changes with you.
\spellnotes You can dismiss tree shape as a free action (instead of as a standard action).

\spellsection{Tremorsense}{1}
\spellinfo{Trans (Imbuement)}{Nature, Earth}
\spelltwocol{\spelltgt{You}}{\spellrng{Personal/\arealarge limit}}
\spelldur{Concentration}
\spelleffect You gain the tremorsense ability. If you are touching a surface, you can automatically pinpoint the location of anything in the area of the spell that is in contact with the surface, including inanimate objects.
\spellnotes Tremorsense functions on surfaces of any kind, regardless of lighting conditions.

\spellsection{True Seeing}{6}
\spellinfo{Div (Awareness)}{Arcane, Divine, Knowledge}
\spellcmp{Verbal, Somatic, and Material}
\spelltwocol{\spelltgt{One creature}}{\spellrng{Touch}}
\spelldur{\durshort}
\spelleffect You confer on the subject the ability to see all things as they actually are. The subject sees through normal and magical darkness, notices secret doors hidden by magic, sees the truth behind visual figments and glamers, and sees the true form of polymorphed, changed, or transmuted things. Further, the subject can focus its vision to see into the Ethereal Plane (but not into extradimensional spaces). The effect extends out to \rngmed range.
\spellnotes \spellindirect{true seeing}{True seeing} does not penetrate solid objects. It in no way confers X-ray vision or its equivalent. It does not negate concealment, including that caused by fog and the like. True seeing does not help the viewer see through mundane disguises, spot creatures who are simply hiding, or notice secret doors hidden by mundane means. In addition, the spell effects cannot be further enhanced with known magic, so one cannot use \spell{true seeing} through a scrying effect.
\spellmat{An ointment for the eyes that costs 100 gp and is made from mushroom powder, saffron, and fat.}
\spellsr{Yes (Will)}

\spellsection{True Strike}{6}
\spelldesc{You grant your ally a temporary, intuitive insight into the immediate future during their next attack.}
\spellinfo{Div (Knowledge)}{Arcane}
\spelltime{1 swift action}
\spellcmp{Verbal only}
\spelltwocol{\spelltgt{one creature}}{\spellrng{\rngmed}}
\spelldur{See text}
\spelleffect The subject's next single physical attack (if it is made before the end of the next round) gains a \plus20 enhancement bonus, and ignores all miss chances. After casting this spell, you cannot cast it again for 5 rounds.

\pdfbookmark[2]{U-Z}{SpellDescriptionsU-Z}
\begin{comment}
\subsubsection{U-Z}
\end{comment}

\spellsection{Unholy Aura}{8}
\spellinfo{Abjur (Interdiction) [Evil]}{Divine, Evil}
\spellcmp{Verbal, Somatic, and Focus}
\spelllimit{\areamed radius centered on you}
\spelltgts{Five creatures in the area}
\spelldur{\durshort \dismissable}
\spellattack{See text}
\spelleffect A malevolent darkness surrounds the subjects, protecting them from attacks, granting them resistance to spells cast by good creatures, and weakening good creatures when they strike the subjects. This abjuration has three effects.
\par First, each subject gains a \plus5 enhancement bonus to its defenses.
\par Second, each subject gains spell resistance 10 against chaotic spells and spells cast by good creatures.
\par Third, at the end of each round, all good creatures within \rngclose range of the subject that attacked the subject with their body or a melee weapon that round take 4d6 points of damage. A creature that attacks multiple creatures shielded by this spell can take this damage multiple times.
\spellfocus{A tiny reliquary containing some sacred relic, such as a piece of parchment from an unholy text. The reliquary costs at least 250 gp.}
\spellsr{Yes (Will)}

\spellsection{Unholy Blight}{4}
\spellinfo{Evoc (Channeling) [Evil]}{Evil}
\spelltwocol{\spelltgt{One creature}}{\spellrng{\rngmed}}
\spelldur{Instantaneous/5 rounds}
\spelldmg{8d6 divine damage \add d6 per two caster levels above 8th}
\spellattack{Will half/Magic vs. Will}
\spelleffect If the target is not evil, it is sickened for 5 rounds, and you make a Will attack to deal damage to it. A failed attack deals half damage.
\spellsr{Yes (Will)}

\spellsection{Unliving Eyes}{3}
\spellinfo{Div/Necro (Awareness, Life)}{Arcane}
\spelltwocol{\spelltgt{One creature}}{\spellrng{Touch}}
\spelldur{\durlong \dismissable}
\spelleffect The subject gains the ability to ``see'' any living creatures and their equipment within 30 feet perfectly, regardless of lighting conditions, physical barriers, invisibility, or any other means of concealment.

If you cast this spell on an undead creature, the range of the vision is doubled to 60 feet.
\spellsr{Yes (Will)}

\spellsection{Unliving Heart}{1}
\spelldesc{You harness the power of unlife to grant yourself a limited ability to avoid death.}
\spellinfo{Necro (Life)}{Necro}
\spelltwocol{\spelltgt{You}}{\spellrng{\rngpers}}
\spelldur{\durlong}
\spelleffect You gain 5 temporary hit points \add 1 per caster level above 2nd. If you take life damage, you lose all temporary hit points provided by this spell before applying the damage.

In addition, you are treated as being undead for the purpose of spells or abilities which affect undead. This causes some unintelligent undead, such as skeletons and zombies, to avoid attacking you.

%gaining full temp HP would be level 5
\spellsection{Vampiric Touch}{4}
\spellinfo{Necro (Life)}{Necro}
\spelltwocol{\spelltgt{One living creature}}{\spellrng{Touch}}
\spelldur{Instantaneous/\durlong}
\spelldmg{8d8 life damage \add d8 per two caster levels above 8th}
\spellattack{Fortitude half}
\spelleffect You make a Fortitude attack to deal damage to the touched creature. A failed attack deals half damage. You gain temporary hit points equal to half the damage you deal. You can't gain more hit points than the subject has. The temporary hit points disappear 1 hour later. If you take life damage, you lose all temporary hit points provided by this spell before applying the damage.

As long as you have temporary hit points from this spell, you are treated as being undead for the purpose of spells or abilities which affect undead. This causes some unintelligent undead, such as skeletons and zombies, to avoid attacking you.
\spellsr{Yes (Fortitude)}

\spellsection{Ventriloquism}{1}
\spellinfo{Illus (Figment)}{Arcane, Trickery}
\spellcmp{V, F}
\spellrng{\rngclose}
\spelldur{\durshort \dismissable}
\spellattack{Magic check vs. Perception and Will (if interacted with)}
\spelleffect You can make your voice (or any sound that you can normally make vocally) seem to issue from someplace else. You can speak in any language you know. With respect to such voices and sounds, anyone who disbelieves the sound recognizes it as illusory (but still hears it).
\spellnotes When you cast this spell, you make a magic check, with the same bonus as your magic attack bonus. This check is opposed by the Will defense of any creature that interacts with the effect. If you fail, the creature disbelieves the spell. In order to interact with the illusion with a Perception check, the creature must make a Perception check that beats your magic check.

\begin{comment}
\spellsection{Vestments of the Mage}{2}
\spelldesc{You imbue a set of armor with magical power, preventing it from interfering with your spellcasting.}
\spellinfo{Trans (Imbuement)}{Arcane}
\spelltwocol{\spelltgt{One nonmagical armor or shield}}{\spellrng{Touch}}
\spelldur{\durext \dismissable}
\spelleffect The armor or shield's chance of arcane spell failure decreases by 10\% as long as you are wearing or using it. If any other creature wears the armor, it receives no benefit from this spell.
\spellnotes This decrease is considered an enhancement enhancement bonus.
\spellsr{Yes (Will)}
\end{comment}

%complicated
\spellsection{Wail of the Banshee}{9}
\spelldesc{You emit a terrible scream that kills anyone that hears it.}
\spellinfo{Necro (Life) [Death, Sound-Dependent]}{Death, Necro}
\spellcmp{Verbal only}
\spellburst{\arealarge radius centered on you}
\spelltgts{Up to five living creatures in the area}
\spelldur{Concentration, up to 2 rounds; see text}
\spellattack{Magic vs. Fortitude}
\begin{spellhealthy}
    The subjects are sickened, making them \vulnerable for 5 rounds. If you concentrate for a second round, subjects still in the area are nauseated for 1 round.
\end{spellhealthy}
\begin{spellblood}
    The subjects are nauseated for 1 round. If you concentrate for a second round, subjects still in the area are reduced to 0 hit points and take 9 critical damage, causing them to begin dying.
\end{spellblood}
\spellnotes This spell affects a maximum number of creatures equal to your caster level. Creatures closest to you are affected first, so creatures farther away may be unaffected if there are enough intervening creatures. Each creature makes only one saving throw against the effect.
\spellsr{Yes (Fortitude)}

\spellsection{Wall of Fire}{5}
\spellinfo{Evoc (Energy) [Fire, Wall]}{Arcane, Nature, Fire}
\spelltwocol{\spellzone{100 ft. wall, 20 ft. high \shapeable}}{\spellrng{\rngmed}}
\spelldur{\durshort}
\spelldmg{4d6 \add d6 per four caster levels above 8th; see text}
\spellattack{Magic vs. Reflex; see text}
\spelleffect An immobile, blazing curtain of shimmering violet fire springs into existence. Whenever a creature passes through the wall, you make a Reflex attack to deal damage to it. A failed attack deals half damage. In addition, the wall radiates heat, dealing 2d6 points of fire damage to creatures within 10 feet and 1d6 points of fire damage to those past 10 feet but within 20 feet. This damage does not require an attakc, and is dealt at the start of each of your turns after you create the wall.
\par If you evoke the wall so that it appears where creatures are, each creature takes damage as if passing through the wall. If any 5-foot length of wall takes 20 points of cold damage or more in 1 round, that length goes out.
\spellnotes This spell can be made permanent with a \spell{permanency} ritual. A permanent \spell{wall of fire} that is extinguished by cold damage becomes inactive for 10 minutes, then reforms at normal strength.
\spellsr{Yes (Reflex)}

\spellsection{Wall of Force}{6}
\spellinfo{Evoc (Control) [Force, Wall]}{Arcane}
\spelltwocol{\spellzone{100 ft. wall, 10 ft. high}}{\spellrng{\rngmed}}
\spelldur{\durshort \dismissable}
\spelleffect This spell creates an invisible wall made of force. Nothing can pass through or alter the wall. It forms a flat, vertical plane, and it must be continuous and unbroken when formed. If the surface is broken by any object or creature, the spell fails.
\spellnotes The wall is unaffected by most spells, including \spell{dispel magic}. However, \spell{disintegrate} immediately destroys it, as does a \magicitem{rod of cancellation}, a \magicitem{sphere of annihilation}, or a \spell{mage's disjunction} spell. As a force effect, it blocks ethereal creatures as well as material ones.
This spell can be made permanent with a \spell{permanency} ritual.

\spellsection{Wall of Stone}{5}
\spellinfo{Conj/Trans (Alteration, Creation) [Earth, Wall]}{Arcane, Earth, Nature}
\spelltwocol{\spellzone{\arealarge wall, 5 ft. high \shapeable}}{\spellrng{\rngmed}}
\spellattack{See text}
\spelleffect This spell forms a wall of stone atop existing rock surfaces. A wall of stone is 4 inches thick and composed of up to ten 5-foot squares. You can double the wall's area by halving its thickness. The wall cannot be conjured so that it occupies the same space as a creature or another object.
\par You can create a wall of stone in almost any shape you desire. The wall created need not be vertical, nor rest upon any firm foundation; however, it must merge with and be solidly supported by existing stone. It can be used to bridge a chasm, for instance, or as a ramp. For this use, if the span is more than 20 feet, the wall must be arched and buttressed. This requirement reduces the spell's area by half. The wall can be crudely shaped to allow crenellations, battlements, and so forth by likewise reducing the area.
\par Like any other stone wall, this one can be destroyed by a \spell{disintegrate} spell or by normal means such as breaking and chipping. Each 5-foot square of the wall has 15 hit points per inch of thickness and hardness 8. A section of wall whose hit points drop to 0 is breached. If a creature tries to break through the wall with a single attack, the DC for the Strength check is 20 \add 2 per inch of thickness.

\spellsection{Wall of Thorns}{5}
\spellinfo{Conj (Creation) [Wall]}{Nature, Wild}
\spelltwocol{\spellzone{100 ft. wall, 10 ft. high \shapeable}}{\spellrng{\rngmed}}
\spelldur{\durlong \dismissable}
\spelleffect This spell creates a barrier of very tough, pliable, tangled brush bearing needle-sharp thorns as long as a human's finger. Any creature forced into or attempting to move through a wall of thorns takes slashing damage per square of movement equal to twice your caster level, minus the creature's Armor defense.
\par You can make the wall as thin as 5 feet thick, which allows you to shape the wall as twenty 10\mult10\mult5 foot blocks. This has no effect on the damage dealt by the thorns, but any creature attempting to break through takes that much less time to force its way through the barrier.
\par Creatures can force their way slowly through the wall by making a grapple attack or Escape Artist check as a full-round action. The creature moves 5 feet for each full 5 points by which the check result exceeds 20, up to a maximum distance equal to its normal land speed. Of course, moving or attempting to move through the thorns incurs damage as described above. A creature trapped in the thorns can choose to remain motionless in order to avoid taking any more damage.
\par If you have at least 5 feet of thorns between you and an opponent, it provides cover. If you have at least 20 feet of thorns between you, it provides total cover.
\par Any creature in the area of the spell when it is cast takes damage as if it had moved into the wall and is caught inside. In order to escape, it must attempt to push its way free, or it can wait until the spell ends. Creatures with the ability to pass throughs overgrown areas unhindered can pass through a wall of thorns at normal speed without taking damage.
\spellnotes A \spell{wall of thorn} can be breached by slow work with edged weapons or fire. It has hardness 8 and 30 hit points per square foot of thickness.
\par Despite its appearance, a \spell{wall of thorns} is not actually a living plant, and thus is unaffected by spells that affect plants.

\spellsection{Warp Wood}{2}
\spellinfo{Trans (Alteration)}{Destruction, Nature}
\spelltwocol{\spelllimit{\areamed radius}}{\spellrng{\rngclose}}
\spelltgt{1 Small nonmagical wooden object/caster level in the area}
\spellattack{Magic vs. Will (object)}
\spelleffect You cause wood to bend and warp, permanently destroying its straightness, form, and strength. A warped door springs open (or becomes stuck, requiring a Strength check to open, at your option). A boat or ship springs a leak. Warped ranged weapons are useless. A warped melee weapon imposes a \minus4 penalty on physical attacks.
\par You may warp one Small or smaller object or its equivalent per caster level. A Medium object counts as two Small objects, a Large object as four, a Huge object as eight, a Gargantuan object as sixteen, and a Colossal object as thirty-two.
\par Alternatively, you can unwarp wood (effectively warping it back to normal) with this spell, straightening wood that has been warped by this spell or by other means. \spellindirect{make whole}{Make whole}, on the other hand, does no good in repairing a warped item.
\spellnotes You can combine multiple consecutive \spell{warp wood} spells to warp (or unwarp) an object that is too large for you to warp with a single spell. Until the object is completely warped, it suffers no ill effects.
\spellsr{Yes (Will)}

\spellsection{Water Walk}{3}
\spellinfo{Trans (Imbuement) [Water]}{Druid, Water}
\spelltwocol{\spelltgts{Five touched creatures}}{\spellrng{Touch}}
\spelldur{\durlong \dismissable}
\spelleffect The transmuted creatures can tread on any liquid as if it were firm ground. Mud, oil, snow, quicksand, running water, ice, and even lava can be traversed easily, since the subjects' feet hover an inch or two above the surface. (Creatures crossing molten lava still take damage from the heat because they are near it.) The subjects can move across the surface as if it were normal ground.
\par If the spell is cast underwater (or while the subjects are partially or wholly submerged in whatever liquid they are in), the subjects are borne toward the surface at 60 feet per round until they can stand on it.
\spellsr{Yes (Fortitude)}

\spellsection{Waves of Exhaustion}{8}
\spellinfo{Necro (Flesh)}{Arcane, Death, War}
\spellburst{\arealarge cone}
\spelldur{\durshort}
\spellattack{Fortitude partial}
\spelleffect You make a Fortitude attack to exhaust all creatures in the area. A failed attack means they are fatigued instead. This spell has no effect on a creature that is already exhausted.
\spellnotes An exhausted character cannot sprint or charge, moves at half speed, and takes a \minus4 penalty to attacks, defenses, and checks. A fatigued character can neither sprint nor charge and is \vulnerable, giving it a \minus2 penalty to attacks, defenses, and checks.
\spellsr{Yes (Fortitude)}

\spellsection{Waves of Fatigue}{5}
\spellinfo{Necro (Flesh)}{Arcane, Death, War}
\spellburst{\arealarge cone}
\spelldur{\durshort}
\spellattack{No}
\spelleffect Living creatures in the area are fatigued. This spell has no effect on a creature that is already fatigued.
\spellnotes A fatigued character can neither sprint nor charge and is \vulnerable, giving it a \minus2 penalty to attacks, defenses, and checks.
\spellsr{Yes (Fortitude)}

\spellsection{Web}{3}
\spelldesc{You create a many-layered mass of strong, stricky strands that entangle creatures caught within them. The strands are similar to spider webs, but larger and tougher.}
\spellinfo{Conj (Creation)}{Arcane}
\spelltwocol{\spellzone{\areamed radius}}{\spellrng{\rngclose}}
\spelldur{\durshort \dismissable}
\spellattack{Reflex negates; see text}
\spelleffect You make a Reflex attack to entangle all creatures in the area. This attack is repeated each round that a creature moves or fights in the area. An entangled creature can spend a standard action to make a grapple attack or Escape Artist attempt against your Reflex attack to break the webs holding it, preventing it from being entangled. A creature entangled by the spell remains entangled until it breaks the webs holding it or escapes the spell's area.
\par If the strands can be anchored to two or more solid and diametrically opposed structures, such as walls, the strands are much more sturdy. A creature entangled within a sturdy web is unable to move from its square until it stops being entangled.
\spellnotes An entangled creature moves at half speed, cannot sprint or charge, and takes a \minus2 penalty to physical attacks and defenses, as well as Strength and Dexterity-based checks. If it attempts to cast a spell must make a Concentration check (DC 10 \add double the spell's level) or lose the spell.
The strands are too widely spaced to significantly obscure sight, but are flammable. A magic flaming sword can slash them away as easily as a hand brushes away cobwebs. Any fire can set the webs alight and burn away 5 square feet in 1 round. All creatures within flaming webs take 2d4 points of fire damage from the flames.

This spell can be made permanent with a \spell{permanency} ritual. A permanent \spell{web} that is destroyed regrows in 10 minutes.

\spellsection{Weird}{9}
\spellinfo{Ench/Illus (Emotion, Phantasm) [Death, Fear, Mind-Affecting, Unreal]}{Arcane, Trickery}
\spelltwocol{\spelllimit{\areamed radius}}{\spellrng{\rngmed}}
\spelltgts{Five creatures in the area}
\spelleffect This spell functions like \spell{phantasmal killer}, except that it affects multiple creatures.

\spellsection{Windstrike}{2}
\spelldesc{You command the air to bludgeon the target, sending it flying.}
\spellinfo{Evoc (Control) [Air]}{Air, Nature}
\spelltwocol{\spelltgt{One creature or object}}{\spellrng{\rngmed}}
\spelldmg{4d6 bludgeoning damage \add d6 per two caster levels above 4th}
\spellattack{Fortitude half}
\spelleffect You make a Fortitude attack to deal damage to the target. A failed attack deals half damage. In addition, you may make a shove attack with an attack bonus equal to your caster level \add your casting attribute. If you succeed, you may have the wind shove the target in any direction -- even vertically. Moving the target up takes twice as much movement as moving the target horizontally.
\spellsr{Yes (Fortitude)}

\spellsection{Windstrike, Greater}{5}
\spelldesc{You command the air to bludgeon the target with tremendous force, sending it flying.}
\spellinfo{Evoc (Control) [Air]}{Air, Nature}
\spelldmg{10d6 bludgeoning damage \add d6 per two caster levels above 10th}
\spelleffect This spell functions like \spell{windstrike}, except that the shove is much more powerful. You make a shove attack with an enhancement bonus equal to your caster level \add your casting attribute \add 12, treating the wind as a Gargantuan creature.

If you succeed, you knock the target prone and may have the wind shove the target in any direction -- even vertically. Moving the target up does not require more movement than moving the target horizontally.

\spellsection{Wish}{9}
\spellinfo{Universal}{Arcane, Magic}
\spellcmp{Verbal, Somatic, and Material}
\spelltwocol{\spelltgteffarea{See text}}{\spellrng{See text}}
\spelldur{See text}
\spellattack{See text}
\spelleffect This spell is the mightiest spell a wizard or sorcerer can cast. By simply speaking your desires aloud, you can alter reality to better suit you.
\par Even wish, however, has its limits.
\par A wish can produce any one of the following effects.
\begin{itemize}
    \item Duplicate any general wizard or sorcerer spell of 8th level or lower, provided the spell is not of a school prohibited to you.
    \item Duplicate any general wizard or sorcerer spell of 7th level or lower even if it's of a prohibited school.
    \item Duplicate any other spell of 6th level or lower, provided the spell is not of a school prohibited to you.
    \item Duplicate any other spell of 5th level or lower even if it's of a prohibited school. 
    \item Undo the harmful effects of many other spells, such as geas/quest or insanity.
    \item Create a nonmagical item of up to 10,000 gp in value.
    \item Create a magic item, or add to the powers of an existing magic item.
    \item Remove injuries and afflictions. A single wish can aid one creature per caster level, and all subjects are cured of the same kind of affliction. For example, you could heal all the damage you and your companions have taken, or remove all poison effects from everyone in the party, but not do both with the same wish. A wish can never restore the experience point loss from casting a spell or the level or Constitution loss from being raised from the dead.
    \item Revive the dead. A wish can bring a dead creature back to life by duplicating a resurrection spell. A wish can revive a dead creature whose body has been destroyed, but the task takes two wishes, one to recreate the body and another to infuse the body with life again.
    \item Transport travelers. A wish can lift one creature per caster level from anywhere on any plane and place those creatures anywhere else on any plane regardless of local conditions. You must make a Will attack to affect unwilling targets.
    \item Undo misfortune. A wish can undo a single recent event. The wish forces a reroll of any roll made within the last round (including your last turn). Reality reshapes itself to accommodate the new result. For example, a wish could undo a foe's successful critical hit (either the attack roll or the critical roll), a friend's failed attack, and so on. The reroll, however, may be as bad as or worse than the original roll. You must make a Will attack to affect an unwilling target.
\end{itemize}
\par When casting a wish, you do not specify the exact spell or effect you wish to duplicate. Instead, you make a wish, describing what you want to have happen, and make a DC 20 Wisdom check. If the check fails, your intent is redirected or perverted in some way. For example, a wish to turn a foe to stone would normally mimic the flesh to stone effect of the transmute flesh and stone spell. However, if the Wisdom check failed, your foe might gain the benefit of a \spell{stoneskin} spell instead.
\par You may try to use a wish to produce greater effects than these, but doing so is dangerous. The DC of the Wisdom check increases to 25, and the negative consequences for failing the check increase in proportion to the potency of the effect you try to create.
\spellmat{10,000gp of diamonds. In addition, when a \spell{wish} duplicates a spell with a material component that costs more than 10,000 gp, you must provide that component.}
\spellsr{Yes (varies)}

\spellsection{Word of Chaos}{7}
\spellinfo{Evoc (Channeling) [Chaotic]}{Chaos}
\spellcmp{Verbal only}
\spellburst{\arealarge radius centered on you}
\spelltgts{All nonlawful creatures in the area}
\spelleffect If the target's level does not exceed your caster level, it is \vulnerable for 5 rounds.

If it is \bloodied, it also suffers one or more of the following ill effects, depending on its level.
\begin{dtable}
    \begin{tabularx}{\columnwidth}{l >{\lcol}X}
        \par \thead{Level} & \thead{Effect} \\
        \par Equal to caster level & Bewildered \\
        \par Up to caster level \minus5 & Confused, bewildered \\
        \par Up to caster level \minus10 & Paralyzed, nauseated, sickened \\
        \par Up to caster level \minus15 & Killed\fn{1}
        1 Living creatures die. Nonliving creatures are destroyed.
    \end{tabularx}
\end{dtable}
\par \subspell{Bewildered} The creature is bewildered, making it \vulnerable for 5 rounds.
\par \subspell{Confused} The creature is confused for 2 rounds.
\par \subspell{Paralyzed} The creature is paralyzed and helpless for 5 rounds.
\par \subspell{Killed} Living creatures die. Nonliving creatures are destroyed.
\spellsr{Yes (Will)}

\spellsection{Word of Recall}{6}
\spellinfo{Conj (Translocation) [Teleportation]}{Divine}
\spellcmp{Verbal only}
\spelltwocol{\spelltgt{You}}{\spellrng{Unlimited}}
\spelleffect This spell teleports you instantly back to your sanctuary. You must designate the sanctuary when you ready the spell for the day, and it must be a very familiar place. The actual point of arrival is a designated area no larger than 10 feet by 10 feet. You can be transported any distance within a plane but cannot travel between planes. You can transport, in addition to yourself, any objects you carry, as long as their weight doesn't exceed your maximum load. Exceeding this limit causes the spell to fail.

\spellsection{Zephyr Blade}{3}
\spelldesc{You imbue a weapon with the power of the wind, allowing it to manipulate air currents as it strikes.}
\spellinfo{Evoc/Trans (Augment, Control) [Air]}{Air, Nature}
\spelltwocol{\spelltgt{One melee weapon}}{\spellrng{Touch}}
\spelldur{\durshort}
\spelleffect This spell functions as \spell{magic weapon}, except that the affected weapon also gains an additional five feet of reach, extending the wielder's threatened area. Attacks outside the weapon's normal range deal half damage, but are otherwise treated exactly as if the wielder was attacking with the weapon normally.
\spellnotes Despite the name of the spell, it can affect melee weapons of any type, even reach weapons. The weapon's extended reach is visible, and opponents can defend themselves normally against the attacks.
\spellsr{Yes (Will)}

\spellsectioncomma{Zephyr Blade}{Greater}{6}
\spelldesc{You imbue a weapon with the full might of the wind, allowing it to shred opponents with nothing but the air itself.}
\spellinfo{Evoc/Trans (Augment, Control) [Air]}{Air, Nature}
\spelleffect This spell functions like \spell{zephyr blade}, except that it extends the weapon's reach by ten feet, and attacks outside the weapon's normal range deal full damage.
