\section{Spell Descriptions}

\pdfbookmark[2]{A}{SpellDescriptionsA}
\begin{comment}
\subsubsection{A}
\end{comment}

\spellsection{Ablate Impact}
\spellschool{Abjuration (Shielding)}
\spelllvl{Abjur 2}
\spelltime{1 immediate action}
\spellrng{Personal}
\spelltgt{You}
\spelldur{1 round}
\begin{spelleffect}
  You gain physical damage reduction 10/force. This damage reduction increases by 1 per caster level above 4th.
\end{spelleffect}
\begin{spellnotes}
  This spell's damage reduction allows the subject to ignore the first 10 physical damage it takes each round. If it is hit by a attack that deals force damage, such as \spell{magic missile}, it cannot use its damage reduction for 1 round.

  You can cast this spell instantaneously, quickly enough react to an opponent attacking you (but before the attack is rolled).
\end{spellnotes}

\spellsection{Ablative Shield}
\spelldesc{You instantly encase yourself a shimmering field of magical energy, protecting you from hostile magic.}
\spellschool{Abjuration (Negation) [Magic]}
\spelllvl{Abjur 1, Magic 1}
\spellcmp{V}
\spelltime{1 immediate action}
\spellrng{\rngpers}
\spelltgt{You}
\spelldur{1 round}
\begin{spelleffect}
  You gain spell damage reduction 5/force. This damage reduction increases by 1 per caster level above 2nd.
\end{spelleffect}
\begin{spellnotes}
  This spell's damage reduction allows the subject to ignore the first 5 spell damage it takes each round, such as from spells and spell-like abilities. If it is hit by a attack that deals force damage, such as \spell{magic missile}, it cannot use its damage reduction for 1 round.
  
  Spells that are not subject to spell resistance are not affected by \spell{ablative shield}. You can cast this spell instantly - quickly enough to gain its benefits in an emergency. Casting the spell is an immediate action, so you can use this spell even when it's not your turn.
\end{spellnotes}

\spellsection{Acid Arrow}
\spelldesc{You fire a magical arrow of acid from your hand that speeds to its target.}
\spellschool{Conjuration (Creation) [Acid]}
\spelllvl{Sor/Wiz 2}
\spellrng{\rngmed}
\spelleff{One arrow of acid}
\spelldur{1 round per two levels}
\spellsave{None}
\spellsr{No}
\spelldmg{2d6 acid damage \add d6 per round}
\begin{spelleffect}
  You must succeed on a ranged touch attack to hit your target. The acid remains on the target after the initial impact, dealing damage each round on your turn.
\end{spelleffect}
\begin{spellnotes}
  If the target becomes submerged in water or takes at least ten points of cold or fire damage, this spell's effect ends.
\end{spellnotes}

\spellsection{Aid}
\spelldesc{You fill the target with confidence, improving its resilience and stamina in combat.}
\spellschool{Enchantment (Emotion) [Mind-Affecting, Morale]}
\spelllvl{Clr 2, Pal 2}
\spellrng{\rngclose}
\spelltgt{One creature}
\spelldur{\durshort}
\spellsave{Will negates (harmless)}
\spellsr{Yes (Will)}
\begin{spelleffect}
  The subject gains a \plus2 bonus to attack rolls and temporary hit points equal 10 \add 1 per caster level above 4th. \bonusscalingdescription  If you take life damage, you lose all temporary hit points provided by this spell before applying the damage.
\end{spelleffect}

\spellsection{Align Weapon}
\spelldesc{You enhance a weapon while bringing it closer to your ideals.}
\spellschool{Evocation/Transmutation (Augment, Channeling) [see text]}
\spelllvl{Chaos 2, Evil 2, Good 2, Law 2}
\spellrng{Touch}
\begin{spelleffect}
  This spell functions like \spell{magic weapon}, except that it also makes a weapon good, evil, lawful, or chaotic, as you choose, allowing it to overcome damage reduction of the appropriate type. When cast on a weapon that already has an alignment, this spell overrides the alignment of the weapon unless the weapon makes a Will save.
\end{spelleffect}
\begin{spellnotes}
  When you make a weapon good, evil, lawful, or chaotic, \spell{align weapon} is a good, evil, lawful, or chaotic spell, respectively.
\end{spellnotes}

\spellsection{Aqueous Blade}
\spelldesc{You transform the active part of your ally's weapon into water, weakening its blows but allowing it penetrate your foe's defenses more easily.}
\spellschool{Transmutation (Alteration) [Water]}
\spelllvl{Drd 2, Water 2}
\spellrng{\rngclose}
\spelltgt{One weapon}
\spelldur{\durshort (D)}
\spellsave{Will negates}
\spellsr{Yes (Will)}
\begin{spelleffect}
  Attacks with the affected weapon are made as touch attacks. However, damage with the weapon is halved, including any bonuses to weapon damage.
\end{spelleffect}

\spellsection{Arcane Sight}
\spelldesc{Your eyes glow blue with power. All nearby magical auras become apparent to you.}
\spellschool{Divination (Awareness) [Magic]}
\spelllvl{Sor/Wiz 2}
\spellrng{\rngpers}
\spelltgt{You}
\spelldur{\durlong (D)}
\begin{spelleffect}
  You know the location and power of all magical auras that you can see within \rngmed range of you. An aura's power depends on a spell's functioning level or an item's caster level, as noted in the description of the Spellcraft skill. If the items or creatures bearing the auras are in line of sight, you can make Spellcraft skill checks to determine the school of magic involved in each. (Make one check per aura; DC 15 \add spell level, or 15 \add one-half caster level for a nonspell effect.)
  \par If you concentrate on a specific creature within \rngmed range of you as a standard action, you can determine whether it has any spellcasting or spell-like abilities, whether these are arcane or divine (spell-like abilities register as arcane), and the strength of the most powerful spell or spell-like ability the creature currently has available for use.
\end{spelleffect}
\begin{spellnotes}
  \spell{Arcane sight} can be made permanent with a \spell{permanency} spell.
\end{spellnotes}

\spellsection{Attraction}
\spelldesc{You cause the subject to feel attracted to something.}
\spellschool{Enchantment (Emotion) [Mind-Affecting]}
\spelllvl{Ench 1}
\spellrng{\rngmed}
\spelltgt{One creature}
\spelldur{\durext}
\spellsave{Will negates}
\spellsr{Yes (Will)}
\begin{spelleffect}
  An affected creature feels attracted to a particular person or object. The subject will take reasonable steps to meet, get close to, attend, or find the object of its implanted attraction. For the purpose of this spell, ``reasonable'' means that, while attracted, the subject doesn't suffer from blind obsession. He will act on this attraction only when not engaged in combat. The subject won't perform obviously suicidal actions. He can still recognize danger but will not flee unless the threat is immediate. If you make the subject feel an attraction to yourself, you can't command him indiscriminately, although he will be willing to listen to you (even if he disagrees).
  \par This spell grants you a \plus4 circumstance bonus on any social interaction checks you make involving the subject (such as Bluff, Diplomacy, Intimidate, and Sense Motive).
\end{spelleffect}

\spellsection{Aversion}
\spelldesc{You make the subject want to avoid something.}
\spellschool{Enchantment (Emotion) [Mind-Affecting]}
\spelllvl{Ench 2}
\spellrng{\rngmed}
\spelltgt{One creature}
\spelldur{\durext}
\spellsave{Will negates}
\spellsr{Yes (Will)}
\begin{spelleffect}
  An affected creature feels an aversion to a particular person or object. If the object of the implanted aversion is an individual or a physical object, she will prefer not to approach within 30 feet of it. If it is a word, she will try not to utter it; if it is an action, she will not willingly attempt to perform it; and if it is an event, she will not willingly attend it. The subject will take reasonable steps to avoid the object of its aversion, but will not put herself in jeopardy by doing so.
  \par If the subject is forced into taking an action she has an aversion to, she is bewildered as long as she performs the action, making her vulnerable.
\end{spelleffect}
\begin{spellnotes}
  A vulnerable creature takes a \minus2 penalty to attack rolls, saving throws, checks, DCs, and AC.
\end{spellnotes}

\pdfbookmark[2]{B}{SpellDescriptionsB}
\begin{comment}
\subsubsection{B}
\end{comment}

\spellsection{Backbiter}
\spelldesc{You subtly animate a weapon so that it strikes its wielder instead of its intended target.}
\spellschool{Transmutation (Animation)}
\spelllvl{Trans 1}
\spellrng{\rngmed}
\spelltgt{One weapon}
\spelldur{\durshort or until discharged}
\spellsave{Will negates (object)}
\spellsr{Yes (Will)}
\begin{spelleffect}
  The next time the affected weapon is used to make a melee attack, it twists around so that the weapon automatically strikes the wielder instead. The wielder gets no warning or knowledge of the spell's effect on his weapon, and though he makes the attack, the self-dealt damage can't be consciously reduced (though damage reduction applies) or changed to nonlethal damage.
  \par Once the weapon attacks its wielder (whether successfully or not), the spell is discharged.
\end{spelleffect}

\spellsection{Bane}
\spelldesc{You fill your enemies with dismay, impairing their ability to fight.}
\spellschool{Enchantment (Emotion) [Mind-Affecting, Morale]}
\spelllvl{Clr 1, Evil 1, War 1}
\spellarea{\areamed radius burst}
\spelldur{Instantaneous}
\spellsave{None}
\spellsr{Yes (Will)}
\begin{spelleffect}
  All enemies within the area take a \minus2 penalty to attack rolls for 5 rounds.
\end{spelleffect}
\begin{spellnotes}
  \spell{Bane} counters and dispels \spell{bless}.
\end{spellnotes}

\spellsection{Barkskin}
\spelldesc{You toughen a creature's skin, giving it the appearance of tree bark.}
\spellschool{Transmutation (Augment)}
\spelllvl{Drd 2, Nature 2}
\spellrng{\rngtouch}
\spelltgt{Living creature touched}
\spelldur{\durshort}
\spellsave{Fortitude negates (harmless)}
\spellsr{Yes (Fortitude)}
\begin{spelleffect}
  The subject gains a \plus2 bonus to its natural armor modifier. \bonusscalingdescription In addition, the subject gains physical damage reduction 2/adamantine or fire. This damage reduction increases by 1 for every four levels above 4th.
\end{spelleffect}
\begin{spellnotes}
  This spell's damage reduction allows the subject to ignore the first 2 physical damage it takes each round. If it is hit by a adamantine weapon or an attack that deals fire damage, it cannot use its damage reduction for 1 round.
\end{spellnotes}

\spellsection{Bless}
\spelldesc{You fill your allies with confidence, improving their prowess in combat.}
\spellschool{Enchantment (Emotion) [Mind-Affecting, Morale]}
\spelllvl{Clr 2, Good 2, Pal 2, War 2}
\spellarea{\areamed radius burst}
\spelldur{Instantaneous}
\spellsave{None}
\spellsr{Yes (Will)}
\begin{spelleffect}
  All allies within the area gain a \plus2 bonus to attack rolls for 5 rounds. \bonusscalingdescription
\end{spelleffect}
\begin{spellnotes}
  \spell{Bless} counters and dispels \spell{bane}.
\end{spellnotes}

\spellsection{Bless Weapon}
\spelldesc{You imbue a weapon with divine power, causing it to strike true against evil foes.}
\spellschool{Evocation/Transmutation (Channeling, Imbuement) [Good]}
\spelllvl{Pal 2}
\spellcmp{V}
\begin{spelleffect}
  This spell functions like \spell{magic weapon}, except that the weapon also becomes good, which means it can bypass the damage reduction of certain creatures. (This effect overrides and suppresses any other alignment the weapon might have.)
\end{spelleffect}

\spellsection{Blur}
\spelldesc{You distort the subject's outline so it appears blurred, shifting, and wavering.}
\spellschool{Illusion (Glamer)}
\spelllvl{Sor/Wiz 2}
\spellrng{\rngclose}
\spelltgt{One creature}
\spelldur{\durshort (D)}
\spellsave{Will negates (harmless)}
\spellsr{Yes (Will)}
\begin{spelleffect}
  The subject gains concealment, granting it a \plus4 circumstance bonus to AC. This concealment allows the subject to use Stealth without other cover or concealment, though other restrictions apply as normal.
\end{spelleffect}
\begin{spellnotes}
  A \spell{see invisibility} spell does not counteract the blurring effect, but a \spell{true seeing} spell does.
  \par Opponents that cannot see the subject ignore the spell's effect (though fighting an unseen opponent carries penalties of its own).
\end{spellnotes}

\spellsection{Burning Hands}
\spelldesc{You expel a cone of searing flame shoots from your fingertips, searing creatures in front of you.}
\spellschool{Evocation (Energy) [Fire]}
\spelllvl{Destruction 1, Fire 1, Sor/Wiz 1}
\spellarea{\areamed cone-shaped burst}
\spelldur{Instantaneous}
\spellsave{Reflex half}
\spellsr{Yes (Reflex)}
\spelldmg{1d6 fire damage \add 1d6 per four caster levels above 2nd}
\begin{spelleffect}
  Everything in the area takes damage. Unattended flammable objects burn if the flames touch them. A character can extinguish burning items as a full-round action.
\end{spelleffect}

\pdfbookmark[2]{C}{SpellDescriptionsC}
\begin{comment}
\subsubsection{C}
\end{comment}

\spellsection{Calm Emotions}
\spelldesc{You calm a group of creatures, preventing the situation from getting out of hand.}
\spellschool{Enchantment (Emotion) [Mind-Affecting]}
\spelllvl{Sor/Wiz 2}
\spellrng{\rngmed}
\spellarea{\areamed radius spread}
\spelldur{Concentration}
\spellsave{Will negates}
\spellsr{Yes (Will)}
\begin{spelleffect}
  Creatures in the area have their emotions calmed. Creatures so affected cannot take violent actions (although they can defend themselves) or do anything destructive.
\end{spelleffect}
\begin{spellnotes}
  Any aggressive action against or damage dealt to a calmed creature immediately breaks the spell on all calmed creatures.

  This spell automatically suppresses (but does not dispel) any effects of spells or abilities that affect or require emotions, including all other enchantment (emotion) spells.
\end{spellnotes}

\spellsection{Cause Fear}
\spelldesc{You fill your enemy with fear.}
\spellschool{Enchantment (Emotion) [Fear, Mind-Affecting]}
\spelllvl{Clr 1, Ench 1}
\spellrng{\rngclose}
\spelltgt{One creature}
\spelldur{\durshort/1 round (D)}
\spellsave{Will negates}
\spellsr{Yes (Will)}
\begin{spellhealthy}
  The subject is shaken, causing it to be vulnerable.
\end{spellhealthy}
\begin{spellblood}
  As the healthy effect, plus the subject is frightened for 1 round.
\end{spellblood}
\begin{spellnotes}
  A vulnerable creature takes a \minus2 penalty to attack rolls, saving throws, checks, DCs, and AC.
\end{spellnotes}

\spellsection{Charm Person}
\spelldesc{You manipulate a person's mind so he thinks of you as a trusted friend and ally.}
\spellschool{Enchantment (Emotion) [Charm, Mind-Affecting]}
\spelllvl{Ench 2}
\spellrng{\rngmed}
\spelltgt{One humanoid creature}
\spelldur{\durlong}
\spellsave{Will negates}
\spellsr{Yes (Will)}
\begin{spelleffect}
  This charm makes a humanoid creature regard you as its trusted friend and ally. If it is currently faced with any obvious threat from you or your allies, such as someone drawing a weapon, casting a spell, or aiming a ranged weapon at the creature, it receives a \plus5 circumstance bonus on its saving throw.
  \par The spell does not enable you to control the subject as if it were an automaton, but it perceives your words and actions in the most favorable way. You can try to give the subject orders, but you must succeed at a Diplomacy check to convince it to do anything it wouldn't ordinarily do. (Retries are not allowed.) Treat the target as a friend (a \plus10 relationship modifier) for the purpose of the Diplomacy check. An affected creature never obeys suicidal or obviously harmful orders, but it might be convinced that something very dangerous is worth doing.
\end{spelleffect}
\begin{spellnotes}
  Any act by you or your apparent allies that threatens the \spell{charmed} person breaks the spell. A creature that makes its saving throw against \spell{charm person} is immune to all further attempts by the same spellcaster for 24 hours.
\end{spellnotes}

\spellsection{Color Spray}
\spelldesc{You project a vivid cone of clashing colors from your outstretched hand, striking creatures in front of you.}
\spellschool{Illusion (Figment) [Light]}
\spelllvl{Sor/Wiz 1}
\spellarea{\areamed cone-shaped burst}
\spelldur{1d4 rounds}
\spellsave{Will negates}
\spellsr{Yes (Will)}
\begin{spelleffect}
  Creatures in the area are dazzled and bewildered.
\end{spelleffect}
\begin{spellnotes}
  A dazzled creature has a 20\% miss chance on all attack rolls and takes a \minus4 penalty to Spot checks. He is also unable to see with darkvision. A bewildered creature is mentally affected in a way that detracts from its ability to act, causing it to be vulnerable. It takes a \minus2 penalty to attack rolls, saving throws, checks, DCs, and AC.

  Creatures who cannot see the light are not affected by this spell. Merely closing one's eyes is insufficient protection.
\end{spellnotes}

\spellsection{Command}
\spelldesc{You compel a foe to obey a single command of your choice.}
\spellschool{Enchantment (Compulsion) [Language-Dependent, Mind-Affecting, Sound-Dependent]}
\spelllvl{Clr 1, Law 1, Pal 1, Sor/Wiz 1}
\spellcmp{V}
\spellrng{\rngmed}
\spelltgt{One creature}
\spelldur{1 round}
\spellsave{Will negates}
\spellsr{Yes (Will)}
\begin{spellhealthy}
  The subject is bewildered, making it vulnerable.
\end{spellhealthy}
\begin{spellblood}
  The subject must perform one of the following actions of your choice.
  \par \subspell{Approach} On its turn, the subject moves toward you as quickly and directly as possible. The creature may do nothing but move during its turn, and it provokes attacks of opportunity for this movement as normal.
  \par \subspell{Drop} As soon as possible, the subject drops whatever it is holding. It may act normally on its turn, except that it can't pick up any dropped items.
  \par \subspell{Fall} As soon as possible, the subject falls to the ground. It may act normally on its turn, except that it can't get up from its prone position.
  \par \subspell{Flee} On its turn, the subject moves away from you as quickly as possible. It may do nothing but move during its turn, and it provokes attacks of opportunity for this movement as normal.
  \par \subspell{Halt} On its turn, the subject can take no actions, but it can defend itself normally.
\end{spellblood}
\begin{spellnotes}
  A vulnerable creature takes a \minus2 penalty to attack rolls, saving throws, checks, DCs, and AC.
  If the subject can't understand or carry out your command, the spell automatically fails.
\end{spellnotes}

\spellsectioncomma{Cone of Cold}{Lesser}
\spelldesc{You create an area of extreme cold that drains heat from creatures in the area.}
\spellschool{Evocation (Energy) [Cold]}
\spelllvl{Drd 2, Sor/Wiz 2}
\spellarea{\areamed cone-shaped burst}
\spelldur{Instantaneous and 1 round}
\spellsave{None/Reflex half}
\spellsr{Yes (Reflex)}
\spelldmg{2d6 cold damage \add d6 per four caster levels above 4th.}
\begin{spelleffect}
  Everything in the area takes damage. Creatures damaged by the spell are fatigued for 1 round.
\end{spelleffect}

\spellsection{Control Water}
\spelldesc{You manipulate elemental forces to control water around you.}
\spellschool{Evocation (Control) [Water]}
\spelllvl{Drd 2, Water 2}
\spellrng{\rngfar}
\spellarea{Water in one volume/level of 10 ft. by 10 ft. by 2 ft. (S)}
\spelldur{\durmed (D)}
\spellsave{None; see text}
\spellsr{No}
\begin{spelleffect}
  Depending on the version you choose, the \spell{control water} spell raises or lowers water.
  \par \subspell{Lower Water} This causes water or similar liquid to reduce its depth by as much as 2 feet per caster level (to a minimum depth of 1 inch). The water is lowered within a squarish depression whose sides are up to caster level \mtimes 10 feet long. In extremely large and deep bodies of water, such as a deep ocean, the spell creates a whirlpool that sweeps ships and similar craft downward, putting them at risk and rendering them unable to leave by normal movement for the duration of the spell.
  \par \subspell{Raise Water} This causes water or similar liquid to rise in height, just as the lower water version causes it to lower. Boats raised in this way slide down the sides of the hump that the spell creates. If the area affected by the spell includes riverbanks, a beach, or other land nearby, the water can spill over onto dry land.
\end{spelleffect}
\begin{spellnotes}
  With either version, you may reduce one horizontal dimension by half and double the other horizontal dimension.
\end{spellnotes}

\spellsection{Create Sound}
\spellschool{Illusion (Figment) [Unreal]}
\spelllvl{Illus 1}
\spellrng{\rngclose}
\spelleff{Illusory sounds}
\spelldur{\durshort (D)}
\spellsave{Will disbelief (if interacted with)}
\spellsr{No}
\begin{spelleffect}
  This spell allows you to create a volume of sound that rises, recedes, approaches, or remains at a fixed place. You choose what type of sound this spell creates when casting it and cannot thereafter change the sound's basic character.
  \par The volume of sound created depends on your level. You can produce as much noise as two normal humans per caster level. Thus, talking, singing, shouting, walking, marching, or running sounds can be created. The noise a \spell{ghost sound} spell produces can be virtually any type of sound within the volume limit, including speech. A horde of rats running and squeaking is about the same volume as eight humans running and shouting. A roaring lion is equal to the noise from sixteen humans, while a roaring dire tiger is equal to the noise from twenty humans.
\end{spelleffect}
\begin{spellnotes}
  \spell{Create sound} can be made permanent with a \spell{permanency} spell.
\end{spellnotes}

\spellsection{Crush Life}
\spelldesc{You attack the life force of a single foe directly, allowing no possibility for escape.}
\spellschool{Necromancy (Life)}
\spelllvl{Death 1, Necro 1}
\spellrng{\rngmed}
\spelltgt{One living creature}
\spelldur{Instantaneous}
\spellsave{None}
\spellsr{Yes (Fort)}
\spelldmg{1d10 life damage \add d10 per four caster levels above 2nd}
\begin{spelleffect}
  The target takes damage.
\end{spelleffect}

\spellsection{Cure Light Wounds}
\spelldesc{You lay your hand on a creature and channel positive energy into it, healing some of its wounds.}
\spellschool{Necromancy (Vitalism) [Healing, Positive]}
\spelllvl{Clr 1, Drd 1, Pal 1}
\spellrng{\rngclose}
\spelltgt{One creature}
\spelldur{Instantaneous}
\spellsave{Fortitude half (harmless) or Fortitude half; see text}
\spellsr{Yes (Fortitude)}
\spellheal{2d6 damage \add d6 per two caster levels above 2nd}
\begin{spelleffect}
  You heal the target. Since undead are powered by negative energy, this spell deals positive damage to them instead of curing their wounds.
\end{spelleffect}

\spellsection{Cure Moderate Wounds}
\spelldesc{You lay your hand on a creature and channel positive energy into it, healing its wounds.}
\spellschool{Necromancy (Life) [Healing, Positive]}
\spelllvl{Clr 2, Drd 2, Life 2, Pal 2}
\spellheal{4d6 damage \add d6 per two caster levels above 4th}
\begin{spelleffect}
  This spell functions like \spell{cure light wounds}, except that for every 20 points of healing granted by the spell, it can instead cure 1 point of critical damage.
\end{spelleffect}

\spellsection{Dancing Lights}
\spellschool{Illusion (Figment) [Light]}
\spelllvl{Sor/Wiz 1}
\spellrng{\rngmed}
\spellarea{\areasmall radius limit}
\spelleff{Up to four lights within the area}
\spelldur{\durshort (D)}
\spellsave{None}
\spellsr{No}
\begin{spelleffect}
  Depending on the version selected, you create up to four lights that resemble lanterns or torches (and cast that amount of light), or up to four glowing spheres of light (which look like will-o'-wisps), or one faintly glowing, vaguely humanoid shape. The \spell{dancing lights} must stay within a \areasmall radius in relation to each other. You can spend a swift action on your turn to move the lights as you desire: forward or back, up or down, straight or turning corners, or the like. The lights can move up to 100 feet per round. A light winks out if the distance between you and it exceeds the spell's range.
\end{spelleffect}
\begin{spellnotes}
  \spell{Dancing lights} can be made permanent with a \spell{permanency} spell.
\end{spellnotes}

\spellsection{Darkness}
\spelldesc{You create a dark aura around an object of your choosing, keeping light without. Even a dwarf's vision is hampered by this shadowy sphere.}
\spellschool{Illusion (Glamer) [Darkness]}
\spelllvl{Sor/Wiz 2, Trickery 2}
\spellcmp{V}
\spellrng{\rngtouch}
\spelltgt{Object touched}
\spelldur{\durmed (D)}
\spellsave{None}
\spellsr{No}
\begin{spelleffect}
  This spell causes an object to radiate shadowy illumination out to a \areamed radius. This causes the level of illumination to drop to shadowy illumination or the current prevailing condition, whichever is lower. Darkvision is ineffective in magical darkness, and confers no advantage over normal vision.
\end{spelleffect}
\begin{spellnotes}
  If \spell{darkness} is cast on a small object that is then placed inside or under a lightproof covering, the spell's effect is blocked until the covering is removed.

  Normal lights (torches, candles, lanterns, and so forth) are incapable of brightening the area or shining through it, as are light spells of lower level. Such effects are also suppressed if they originate from within the area of the darkness, preventing them from shining light elsewhere. Higher level light spells are not affected by darkness.

  \par \spell{Darkness} counters or dispels any light spell of equal or lower spell level.
\end{spellnotes}

\spellsection{Darkvision}
\spellschool{Divination (Awareness)}
\spelllvl{Sor/Wiz 2}
\spellrng{\rngtouch}
\spelltgt{Creature touched}
\spelldur{\durlong}
\spellsave{Fortitude negates (harmless)}
\spellsr{Yes (Fortitude)}
\begin{spelleffect}
  The subject gains the ability to see 60 feet even in total darkness. Beyond 60 feet, the subject can see dimly, treating areas of darkness as shadowy illumination. Darkvision does not function if a creature is in an area of bright light or is dazzled. Darkvision is black and white only, but otherwise like normal sight.
\end{spelleffect}
\begin{spellnotes}
  \spell{Darkvision} does not grant one the ability to see in magical darkness.

  \par \spell{Darkvision} can be made permanent with a \spell{permanency} spell.
\end{spellnotes}

\spellsection{Daylight}
\spellschool{Illusion (Figment) [Light]}
\spelllvl{Clr 2, Pal 2}
\spellrng{\rngtouch}
\spelltgt{Object touched}
\spelldur{\durlong (D)}
\spellsave{None}
\spellsr{No}
\begin{spelleffect}
  The object touched sheds light as bright as full daylight in a \arealarge radius, and dim light for an additional 50 feet beyond that. Creatures that take penalties in bright light also take them while within the radius of this magical light. Despite its name, this spell is not the equivalent of sunlight for the purposes of creatures that are damaged or destroyed by bright light.
  \par If \spell{daylight} is cast on a small object that is then placed inside or under a light-proof covering, the spell's effects are blocked until the covering is removed.
\end{spelleffect}
\begin{spellnotes}
  \spell{Daylight} brought into an area of magical darkness (or vice versa) is temporarily negated, so that the otherwise prevailing light conditions exist in the overlapping areas of effect.
  \par \spell{Daylight} counters or dispels any darkness spell of equal or lower level, such as darkness.
\end{spellnotes}

\spellsection{Death Knell}
\spelldesc{You draw forth the ebbing life force of a creature and use it to fuel your own power.}
\spellschool{Necromancy (Life) [Death]}
\spelllvl{Death 2, Evil 2, Necro 2}
\spellrng{\rngmed}
\spelltgt{Living creature}
\spelldur{\durshort; see text}
\spellsave{Fortitude negates}
\spellsr{Yes (Fortitude)}
\begin{spellblood}
  The subject becomes vulnerable. If it drops to 0 hit points, it dies immediately, and you gain 20 temporary hit points \add 2 per caster level above 4th. These temporary hit points last for 1 round per HV the subject had.

 If you take life damage, you lose all temporary hit points provided by this spell before applying the damage.
\end{spellblood}
\begin{spellnotes}
    A vulnerable creature takes a \minus2 penalty to attack rolls, saving throws, checks, DCs, and AC.
\end{spellnotes}

\spellsection{Delay Poison}
\spellschool{Necromancy (Flesh)}
\spelllvl{Clr 1, Drd 1, Pal 1}
\spelltime{1 swift action}
\spellrng{\rngclose}
\spelltgt{Creature touched}
\spelldur{\durshort}
\spellsave{Fortitude negates (harmless)}
\spellsr{Yes (Fortitude)}
\begin{spelleffect}
  The subject becomes temporarily immune to the effects of poison. It does not make any saving throws against poison during this spell's duration. This effect does not prevent the subject from becoming poisoned, and any poisons in the subject's system when the spell ends will continue their effects normally. 
\end{spelleffect}
\begin{spellnotes}
  This spell does not cure any damage that poison may have already done.
\end{spellnotes}

\spellsection{Detect Animals or Plants}
\spellschool{Divination (Awareness) [Detection]}
\spelllvl{Drd 1, Nature 1}
\spellarea{\arealarge cone-shaped emanation from you}
\spelldur{Concentration}
\spellsave{None}
\spellsr{No}
\begin{spelleffect}
  You know the direction to any animals in the area by seeing their auras. If you concentrate on a particular aura, you learn its location. You must choose to detect either animals or plants. Alternately, you can choose to detect a particular kind of animal or plant. Each round, you can change what you are trying to detect.
\end{spelleffect}
\begin{spellnotes}
  Each round, you can turn to detect animals or plants in a new area. A detection spell can penetrate barriers, but 1 foot of stone, 1 inch of common metal, a thin sheet of lead, or 3 feet of wood or dirt blocks it.
\end{spellnotes}

\spellsection{Detect Chaos}
\spellschool{Divination (Awareness) [Detection]}
\spelllvl{Clr 1, Pal 1}
\begin{spelleffect}
  This spell functions like \spell{detect evil}, except that it detects chaotic auras, and you are vulnerable to an overwhelming chaotic aura if you are lawful.
\end{spelleffect}

\spellsection{Detect Evil}
\spelldesc{You sense the presence of evil.}
\spellschool{Divination (Awareness) [Detection]}
\spelllvl{Clr 1, Pal 1}
\spellarea{\arealarge cone-shaped emanation from you}
\spelldur{Concentration}
\spellsave{None}
\spellsr{No}
\begin{spelleffect}
  You know the direction to any evil creatures or objects in the area by seeing their auras. If you concentrate on a particular aura, you learn how powerful it is, as determined by the table below.
  \par If the HV or level of the aura's source is at least twice your caster level, the power of the aura increases by one step, with strong auras becoming overwhelming. If you are good, and you concentrate on a creature with an overwhelming aura, you must make a Will save or be stunned for 1 round (which typically breaks your concentration, ending the spell).
  \begin{dtable}
    \begin{tabularx}{\columnwidth}{l >{\lcol}X}
      \thead{Creature/Object} & \thead{Aura Power} \\
      Evil creature & Faint \\
      Undead & Moderate \\
      Evil magic item or spell & Moderate\footnotetemp{1} \\
      Evil outsider & Strong \\
      Cleric of an evil deity\footnotetemp{2} & Strong
    \end{tabularx}
    \par 1 Use the item or spell's caster level to determine whether the power of the aura us unusually strong.
    \par 2 Some characters who are not clerics (such as blackguards) may radiate an aura of equivalent power. The class description will indicate whether this applies.
  \end{dtable}
  \par \subspell{Lingering Aura} An evil aura can linger after its original source dissipates (in the case of a spell) or is destroyed (in the case of a creature or magic item). If \spell{detect evil} is cast and directed at such a location, the spell indicates an aura strength of dim (even weaker than a faint aura). Most auras only linger for a few rounds, but strong or overwhelming auras can linger for days.
\end{spelleffect}
\begin{spellnotes}
  Animals, traps, poisons, and other potential perils are not evil, and as such this spell does not detect them.
  \par Each round, you can turn to detect evil in a new area. A detection spell can penetrate barriers, but 1 foot of stone, 1 inch of common metal, a thin sheet of lead, or 3 feet of wood or dirt blocks it.
\end{spellnotes}

\spellsection{Detect Good}
\spellschool{Divination (Awareness) [Detection]}
\spelllvl{Clr 1}
\begin{spelleffect}
  This spell functions like \spell{detect evil}, except that it detects good auras, and you are vulnerable to an overwhelming good aura if you are evil.
\end{spelleffect}
\begin{spellnotes}
  Healing potions, antidotes, and similar beneficial items are not good, and as such this spell does not detect them.
\end{spellnotes}

\spellsection{Detect Law}
\spellschool{Divination (Awareness) [Detection]}
\spelllvl{Clr 1}
\begin{spelleffect}
  This spell functions like \spell{detect evil}, except that it detects lawful auras, and you are vulnerable to an overwhelming lawful aura if you are chaotic.
\end{spelleffect}

\spellsection{Detect Secret Doors}
\spelldesc{You can detect secret doors, compartments, caches, and so forth.}
\spellschool{Divination (Awareness) [Detection]}
\spelllvl{Sor/Wiz 1}
\spellarea{\arealarge cone-shaped emanation from you}
\spelldur{Concentration}
\spellsave{None}
\spellsr{No}
\begin{spelleffect}
  You know the direction to any hidden passages, doors, or openings in the area. If you concentrate on a particular aura, you learn its location. This does not automatically grant you the ability to see or open the door -- merely the knowledge that such a door exists in that location.
\end{spelleffect}
\begin{spellnotes}
  Each round, you can turn to detect secret doors in a new area. A detection spell can penetrate barriers, but 1 foot of stone, 1 inch of common metal, a thin sheet of lead, or 3 feet of wood or dirt blocks it.
\end{spellnotes}

\spellsection{Detect Undead}
\spellschool{Divination (Awareness) [Detection]}
\spelllvl{Clr 1, Pal 1, Sor/Wiz 1}
\spellarea{\arealarge cone-shaped emanation from you}
\spelldur{Concentration}
\spellsave{None}
\spellsr{No}
\begin{spelleffect}
  You know the direction of all undead creatures in the spell's area. If you concentrate on a particular undead creature, you learn the strength of its aura, determined by the table below.
  You can detect the aura that surrounds undead creatures. The amount of information revealed depends on how long you study a particular area.
  \par \subspell{1st Round} Presence or absence of undead auras.
  \par \subspell{2nd Round} Number of undead auras in the area and the strength of the strongest undead aura present. If you are of good alignment, and the strongest undead aura's strength is overwhelming (see below), and the creature has HV of at least twice your character level, you are stunned for 1 round and the spell ends.
  \par \subspell{3rd Round} The strength and location of each undead aura. If an aura is outside your line of sight, then you discern its direction but not its exact location.
  \par \subspell{Aura Strength} The strength of an undead aura is determined by the HV of the undead creature, as given on the following table:
  \begin{dtable}
    \begin{tabularx}{\columnwidth}{*{2}{>{\lcol}X}}
      \par HV & Strength \\ 
      \par 1 or lower & Faint \\ 
      \par 2--4 & Moderate \\ 
      \par 5--10 & Strong \\ 
      \par 11 or higher & Overwhelming
    \end{tabularx}
  \end{dtable}
  \par \subspell{Lingering Aura} An undead aura can linger after its original source is destroyed. If detect undead is cast and directed at such a location, the spell indicates an aura strength of dim (even weaker than a faint aura). How long the aura lingers at this dim level depends on its original power. Most auras only linger for a few rounds, but strong or overwhelming auras can linger for days.
\end{spelleffect}
\begin{spellnotes}
  Each round, you can turn to detect undead in a new area. A detection spell can penetrate barriers, but 1 foot of stone, 1 inch of common metal, a thin sheet of lead, or 3 feet of wood or dirt blocks it.
\end{spellnotes}

\spellsection{Disguise Self}
\spellschool{Illusion (Glamer) [Unreal]}
\spelllvl{Illus 1, Trickery 1}
\spellrng{\rngpers}
\spelltgt{You}
\spelldur{\durlong (D)}
\begin{spelleffect}
  You make yourself -- including clothing, armor, weapons, and equipment -- look different. You can seem 20\% (about 1 foot for an average human) shorter or taller, thin, fat, or in between. You cannot change your body type. Otherwise, the extent of the apparent change is up to you. You could add or obscure a minor feature or look like an entirely different person.
  \par The spell does not provide the abilities or mannerisms of the chosen form, nor does it alter the perceived tactile (touch) or audible (sound) properties or  you or your equipment. 
  \par If you use this spell to create a disguise, you get a \plus10 bonus on the Disguise check.
\end{spelleffect}
\begin{spellnotes}
  A creature that interacts with the effect gets a Will save to recognize it as an illusion. In order to interact with the illusion with a Perception check, the creature must make a Perception check that beats your saving throw DC with this spell or your Disguise check (if used as part of a disguise), whichever is higher. You cannot change your disguise once the spell is cast. 
\end{spellnotes}

\spellsectioncomma{Dispel Magic}{Lesser}
\spellschool{Abjuration (Negation) [Magic]}
\spelllvl{Clr 1, Drd 2, Magic 1, Sor/Wiz 1}
\begin{spelleffect}
  This spell functions like a targeted \spell{dispel magic}, except that you add half your caster level to your dispel check.
\end{spelleffect}

\spellsection{Dissipating Touch}
\spelldesc{Your mere touch can disperse the surface material of your foe, sending a tiny portion of it far away.}
\spellschool{Conjuration (Translocation) [Teleportation]}
\spelllvl{Conj 2}
\spellrng{\rngtouch}
\spelltgt{Creature or object touched}
\spelldur{Instantaneous/1 round}
\spellsave{Will half (object)}
\spellsr{Yes (Will)}
\spelldmg{4d8 physical damage \add d8 per two caster levels above 4th}
\begin{spelleffect}
  The touched target takes damage and is sickened for 1 round. This damage ignores the hardness and damage reduction.
\end{spelleffect}

\spellsection{Divine Favor}
\spelldesc{You imbue yourself with skill in combat by calling upon the divine power of your patron.}
\spellschool{Transmutation (Augment)}
\spelllvl{Clr 1, Pal 1, Strength 1, War 1}
\spellrng{\rngpers}
\spelltgt{You}
\spelldur{\durshort}
\begin{spelleffect}
  You gain a \plus2 bonus on attack and weapon damage rolls. \bonusscalingdescription
\end{spelleffect}

\spellsection{Earth's Pull}
\spelldesc{You intensify the pull of gravity on your foe, causing it to feel much heavier and making its movements sluggish.}
\spellschool{Evocation (Control) [Earth]}
\spelllvl{Drd 1, Earth 1}
\spellrng{\rngmed}
\spelltgt{One Large or smaller creature}
\spelldur{\durshort}
\spellsave{No}
\spellsr{Yes (Will)}
\begin{spelleffect}
  The subject moves at half speed and takes a \minus2 penalty to armor class. If it is flying within 10 feet of the ground, the subject falls to the ground.
\end{spelleffect}
\begin{spellnotes}
  If the subject gets farther than 10 feet from the ground, the spell's effect is broken. As a result, the spell cannot affect creatures flying high above the ground.
\end{spellnotes}

\spellsection{Earthen Blade}
\spellschool{Transmutation (Alteration, Augment) [Earth]}
\spelllvl{Drd 2, Earth 2}
\spellrng{0 ft.}
\spelleff{Earthen weapon}
\spelldur{\durlong (D)}
\spellsave{None}
\spellsr{Yes (Fortitude)}
\begin{spelleffect}
    This spell creates a weapon from the ground. The weapon can be of any type you are proficient with. In addition, the weapon is magical, as the \spell{magic weapon} spell.
\end{spelleffect}

\spellsection{Enfeeblement}
%\spelldesc{You fire a coruscating ray from your hand. When it strikes your foe, he becomes weaker.}
\spellschool{Necromancy (Flesh)}
\spelllvl{Death 1, Sor/Wiz 1}
\spellrng{\rngmed}
\spelltgt{One creature}
\spelldur{\durshort}
\spellsave{Fortitude negates}
\spellsr{Yes (Fortitude)}
\begin{spelleffect}
  The subject takes a \minus4 penalty to your choice of Strength, Dexterity, or Constitution.
\end{spelleffect}
\begin{spellnotes}
  This spell cannot reduce the subject's attributes below \minus9.
\end{spellnotes}

\spellsection{Entangle}
\spellschool{Transmutation (Animation)}
\spelllvl{Drd 1, Nature 1}
\spellrng{\rngmed}
\spellarea{\areasmall radius spread}
\spelldur{\durshort (D)}
\spellsave{Reflex partial}
\spellsr{No}
\begin{spelleffect}
  Grasses, weeds, bushes, and even trees wrap, twist, and entwine about creatures in the area or those that enter the area, holding them fast and causing them to become entangled. The creature can break free and move half its normal speed by using a standard action to make a grapple attack or an Escape Artist check against this spell's save DC. A creature that succeeds on a Reflex save is not entangled but can still move at only half speed through the area. Each round on your turn, the plants once again attempt to entangle all creatures that have avoided or escaped entanglement.
\end{spelleffect}
\begin{spellnotes}
  The effects of the spell may be altered somewhat based on the nature of the entangling plants. If no plants exist in the area, the spell has no effect.
\end{spellnotes}

\spellsection{Entropic Shield}
\spelldesc{You surround your ally with a magical field that glows with a chaotic blast of multicolored hues. This field deflects incoming ranged attacks, causing them to randomly swerve away from their intended target.}
\spellschool{Abjuration (Shielding)}
\spelllvl{Chaos 2, Clr 2}
\spellrng{\rngclose}
\spelltgt{Touched creature}
\spelldur{\durshort (D)}
\begin{spelleffect}
  Each ranged attack directed at the subject for which the attacker must make an attack roll has a 50\% miss chance (similar to the effects of active cover). Other attacks that simply work at a distance are not affected.
\end{spelleffect}

\spellsection{Expeditious Retreat}
\spellschool{Transmutation (Temporal)}
\spelllvl{Trans 1}
\spellrng{\rngclose}
\spelltgt{One creature}
\spelldur{\durshort (D)}
\spellsave{Will negates (harmless)}
\spellsr{Yes (Will)}
\begin{spelleffect}
  Your base land speed doubles, to a maximum of a \plus30 foot increase. (This adjustment is treated as an enhancement bonus.) There is no effect on other modes of movement.
\end{spelleffect}
\begin{spellnotes}
 As with any effect that increases your speed, this spell affects your jumping distance.
 \end{spellnotes}

\pdfbookmark[2]{F}{SpellDescriptionsF}
\begin{comment}
\subsubsection{F}
\end{comment}

\spellsection{Faerie Fire}
\spellschool{Illusion (Figment) [Light, Unreal]}
\spelllvl{Drd 1}
\spellrng{\rngmed}
\spellarea{\areasmall radius limit}
\spelleff{Dim lights in the area}
\spelldur{\durshort (D)}
\spellsave{None}
\spellsr{Yes (Will)}
\begin{spelleffect}
  A pale glow surrounds and outlines all creatures and objects in the area. Outlined subjects shed light as candles. Outlined creatures do not benefit from the concealment normally provided by darkness (though a 3rd-level or higher magical darkness effect functions normally), \spell{blur}, \spell{displacement}, \spell{invisibility}, or similar effects. Illusionary figments such as \spell{silent image} are not outlined, which may reveal them for what they are.
  
  The light is too dim to have any special effect on undead or dark-dwelling creatures vulnerable to light. The \spell{faerie fire} can be blue, green, or violet, according to your choice at the time of casting. This spell does not cause any harm to the objects or creatures thus outlined.
\end{spelleffect}

\spellsection{False Life}
\spelldesc{You harness the power of life to grant yourself a limited ability to avoid death.}
\spellschool{Necromancy (Life)}
\spelllvl{Necro 1}
\spellrng{\rngpers}
\spelltgt{You}
\spelldur{\durshort}
\begin{spelleffect}
  You gain 10 temporary hit points \add 2 per caster level above 2nd. If you take life damage, you lose all temporary hit points provided by this spell before applying the damage.
\end{spelleffect}

\spellsection{Farsight}
\spelldesc{You grant the subject the ability to see farther and more accurately.}
\spellschool{Divination (Awareness)}
\spelllvl{Div 1}
\spellrng{\rngtouch}
\spelltgt{Creature touched}
\spelldur{\durlong (D)}
\spellsave{Will negates (harmless)}
\spellsr{Yes (Will)}
\begin{spelleffect}
  The subject gains a \plus2 bonus to Spot checks and takes half the normal penalty for range increments and for Spot checks made at a distance. \bonusscalingdescription
\end{spelleffect}

\spellsection{Feather Fall}
\spellschool{Evocation (Control) [Air]}
\spelllvl{Air 1, Evoc 1, Travel 1}
\spellcmp{V}
\spelltime{1 immediate action}
\spellrng{\rngmed}
\spellarea{\areamed radius limit}
\spelltgts{Five Medium or smaller freefalling object or creatures within the area}
\spelldur{\durshort or until landing}
\spellsave{Will negates (harmless) or Will negates (object)}
\spellsr{Yes (Will)}
\begin{spelleffect}
  The affected creatures or objects fall slowly. Feather fall instantly changes the rate at which the targets fall to a mere 60 feet per round (equivalent to the end of a fall from a few feet), and the subjects take no damage upon landing while the spell is in effect. However, when the spell duration expires, a normal rate of falling resumes.
  \par The spell affects one or more Medium or smaller creatures (including gear and carried objects up to each creature's maximum load) or objects, or the equivalent in larger creatures: A Large creature or object counts as two Medium creatures or objects, a Huge creature or object counts as two Large creatures or objects, and so forth.
  \par If the spell is cast on a falling item, the object does half normal damage based on its weight, with no bonus for the height of the drop.
\end{spelleffect}
\begin{spellnotes}
  You can cast this spell instantaneously, quickly enough to save yourself if you unexpectedly fall. 
  \par Feather fall works only upon free-falling objects. It no special effect on ranged weapons unless they are falling quite a distance. It does not affect a sword blow or a charging or flying creature.
\end{spellnotes}

\spellsection{Flame Blade}
\spellschool{Evocation (Energy) [Fire]}
\spelllvl{Drd 2, Fire 2}
\spellrng{0 ft.}
\spelleff{Sword-like beam}
\spelldur{\durlong (D)}
\spellsave{None}
\spellsr{Yes (Fortitude)}
\begin{spelleffect}
  A 3 foot long beam of red-hot fire springs forth from your hand. In addition to providing illumination like a torch, you can wield this bladelike beam as a weapon. It is treated like a scimitar, except that all damage dealt with it is fire damage, you add half your casting attribute to damage in place of half your Strength, and it is treated as a light weapon, so you can use Dexterity to attack with it. Alternately, you can hurl flames from the weapon up to \rngmed range as if it were a thrown weapon.
\end{spelleffect}
\begin{spellnotes}
  Fire spells do not function underwater. A \spell{flame weapon} can ignite combustible materials such as parchment, straw, dry sticks, and cloth. Spell resistance applies when a foe is struck by the weapon, but not when the blade is created.
\end{spellnotes}

\spellsection{Gentle Descent}
\spelldesc{You grant your ally ephemeral wings which allow him to glide.}
\spellschool{Transmutation (Imbuement) [Air]}
\spelllvl{Air 2, Drd 2}
\spellrng{\rngmed}
\spelltgt{One creature}
\spelldur{\durlong}
\spellsave{Will negates (harmless)}
\spellsr{Yes (Will)}
\begin{spelleffect}
  The subject gains a 30 foot glide speed. It must spend a move action each round to glide.
\end{spelleffect}
\begin{spellnotes}
  A creature with a glide speed can glide while in the air. Each round, a gliding creature moves forward at a rate equal to its glide speed and moves five feet down. It may choose to move slower, to a minimum of half its normal glide speed. It may alternately choose to dive, allowing it to move forward at a rate equal to twice its glide speed but also moving twenty feet down. A gliding creature cannot run.
\end{spellnotes}

\spellsection{Glitterdust}
\spellschool{Conjuration (Creation)}
\spelllvl{Sor/Wiz 2}
\spellrng{\rngmed}
\spellarea{\areasmall radius spread}
\spelleff{Glittering particles in the area}
\spelldur{\durshort}
\spellsave{None}
\spellsr{No}
\begin{spelleffect}
  A cloud of golden particles covers everyone and everything in the area, visibly outlining invisible things for the duration of the spell. It likewise negates the effects of \spell{blur} and \spell{displacement}, and reveals illusionary figments such as \spell{silent image} for what they are. All within the area at the time that the spell is cast are covered by the dust, which continues to sparkle until it fades.
  \par Any creature covered by the dust takes a \minus40 penalty on Hide checks.
\end{spelleffect}
\begin{spelleffect}
  Water and similar substances can remove the dust.
\end{spelleffect}

\spellsection{Grease}
\spelldesc{You conjure a layer of slippery grease on the ground, tripping up your foes.}
\spellschool{Conjuration (Creation)}
\spelllvl{Sor/Wiz 1}
\spellrng{\rngclose}
\spelltgtorarea{One object or a 10 ft. square}
\spelldur{\durshort (D)}
\spellsave{See text}
\spellsr{No}
\begin{spelleffect}
  Any creature in the area when the spell is cast must make a successful Reflex save or fall. A creature can walk within or through the area of grease at half normal speed with a DC 10 Balance check. Failure means it can't move that round, while failure by 5 or more means it falls (see the Balance skill for details). A creature standing in a greased area loses its Dexterity and dodge modifiers to AC due to the slippery surface.
  \par The spell can also be used to create a greasy coating on an item. Material objects not in use are always affected by this spell, while an object wielded or employed by a creature receives a Reflex saving throw to avoid the effect entirely. If the initial saving throw fails, the creature immediately drops the item. If the item is successfully greased, a saving throw must be made in each round that the creature attempts to pick up or use the greased item. A creature wearing greased armor or clothing gains a \plus10 bonus on Escape Artist checks and on grapple attacks made to resist or escape a grapple or to escape a pin.
\end{spelleffect}

\spellsection{Greater (Spell Name)}
\par Any spell whose name begins with greater is alphabetized in this chapter according to the second word of the spell name. Thus, the description of a greater spell appears near the description of the spell on which it is based. Spell chains that have greater spells in them include those based on the spells arcane sight, command, dispel magic, glyph of warding, invisibility, magic fang, magic weapon, planar ally, planar binding, prying eyes, restoration, scrying, shadow conjuration, shadow evocation, shout, and teleport.

\spellsection{Gust of Wind}
\spellschool{Evocation (Control) [Air]}
\spelllvl{Air 1, Drd 1, Evoc 1}
\spellarea{\arealarge line-shaped emanation from you}
\spelleff{Wind within the area}
\spelldur{1 round}
\spellsave{Fortitude partial; see text.}
\spellsr{No}
\begin{spelleffect}
  This spell creates a severe blast of air (approximately 50 mph) that originates from you, affecting all creatures in its path. Creatures are affected according to their size category. A successful Fortitude save causes a creature to be affected as if it were one size category larger. Flying creatures are affected as if one size category smaller.
  \begin{itemize*}
    \item Tiny or smaller creatures are knocked prone and blown to the edge of the spell's range.
    \item Small creatures are knocked prone by the force of the wind.
    \item Medium creatures are unable to move forward against the force of the wind.
    \item Large or larger creatures may move normally.
  \end{itemize*}
  \par Any creature, regardless of size, takes a \minus4 penalty on ranged attacks and Listen checks in the spell's area.
  \par In addition to the effects noted, a gust of wind can do anything that a sudden blast of wind would be expected to do. It can extinguish open flames, create a stinging spray of sand or dust, fan a large fire, overturn delicate awnings or hangings, heel over a small boat, and blow gases or vapors to the edge of its range.
\end{spelleffect}
\begin{spellnotes}
  \spell{Gust of wind} can be made permanent with a \spell{permanency} ritual.
\end{spellnotes}

\pdfbookmark[2]{H}{SpellDescriptionsH}
\begin{comment}
\subsubsection{H}
\end{comment}

\spellsection{Heat Metal}
\spellschool{Evocation (Energy) [Fire]}
\spelllvl{Drd 2}
\spellrng{\rngmed}
\spelltgt{Metal equipment of one creature within the area}
\spelldur{\durshort (D)}
\spellsave{See text}
\spellsr{Yes (Will)}
\spelldmg{2d6 fire damage per round \add 1d6 per four levels above 4th; see text}
\begin{spelleffect}
  This spell makes metal burning hot, causing it to deal damage each round. A creature not touching metal takes no damage from this spell. A creature wielding metal equipment can attempt a Fortitude save for half damage each round. A creature wearing metal armor receives no saving throw, and is also vulnerable for the duration of the spell.
\end{spelleffect}
\begin{spellnotes}
A vulnerable creature takes a \minus2 penalty to attack rolls, saving throws, checks, DCs, and AC.

  If the subject is underwater, this spell deals half damage, boiling the surrounding water, and the subject is not vulnerable. Any cold intense enough to damage the creature negates fire damage from the spell (and vice versa) on a point-for-point basis.
\end{spellnotes}

\spellsection{Hideous Laughter}
\spelldesc{You force the subject to collapse into gales of manic laughter with an unnaturally amusing joke.}
\spellschool{Enchantment (Compulsion) [Mind-Affecting]}
\spelllvl{Brd 2}
\spellrng{\rngclose}
\spelltgt{One creature}
\spelldur{\durshort}
\spellsave{Will negates}
\spellsr{Yes (Will)}
\begin{spelleffect}
  The subject is bewildered, making it vulnerable.
\end{spelleffect}
\begin{spellblood}
  In addition, the subject is flat-footed and must spend a standard action each round to do nothing but laugh uncontrollably. After each time it laughs, the affected creature can attempt a new saving throw. If it succeeds, it can stop laughing, though it is still bewildered.
\end{spellblood}
\begin{spellnotes}
  A creature with an Intelligence score of \minus8 or lower is not affected. A creature whose type is different from the caster's receives a \plus4 circumstance bonus on its saving throw, because humor doesn't ``translate'' well.
\end{spellnotes}
\begin{spellnotes}
  A vulnerable creature takes a \minus2 penalty to attack rolls, saving throws, checks, DCs, and AC.
\end{spellnotes}

\spellsection{Hold Person}
\spellschool{Enchantment (Inhibition) [Mind-Affecting]}
\spelllvl{Clr 2, Pal 2, Sor/Wiz 2, War 2}
\spellrng{\rngclose}
\spelltgt{One humanoid creature}
\spelldur{\durshort (D); see text}
\spellsave{Will negates; see text}
\spellsr{Yes (Will)}
\begin{spellhealthy}
  The subject is bewildered, making it vulnerable.
\end{spellhealthy}
\begin{spellblood}
  As the healthy effect, and the subject is paralyzed and unable to act. Each round on its turn, the subject may attempt a new saving throw to end the paralysis. If it succeeds, it is no longer paralyzed, though it is still bewildered and can take no other actions that round.
\end{spellblood}
\begin{spellnotes}
  A vulnerable creature takes a \minus2 penalty to attack rolls, saving throws, checks, DCs, and AC.
\end{spellnotes}

\spellsection{Inertial Shield}
\spelldesc{You create a barrier around your ally that resists physical intrusion.}
\spellschool{Abjuration (Shielding)}
\spelllvl{Sor/Wiz 2}
\spellrng{\rngclose}
\spelltgt{One creature}
\spelldur{\durshort}
\spellsave{Will negates (harmless)}
\spellsr{Yes (Will)}
\begin{spelleffect}
  The subject gains physical damage reduction 4/force. This damage reduction increases by 1 per two caster levels above 4th.
\end{spelleffect}
\begin{spellnotes}
  This spell's damage reduction allows the subject to ignore the first 4 physical damage it takes each round. If it is hit by a attack that deals force damage, such as \spell{magic missile}, it cannot use its damage reduction for 1 round.
\end{spellnotes}

\spellsection{Inflict Light Wounds}
\spellschool{Necromancy (Vitalism) [Negative]}
\spelllvl{Clr 1, Sor/Wiz 1}
\spellrng{\rngtouch}
\spelltgt{Creature touched}
\spelldur{Instantaneous}
\spellsave{Fortitude half}
\spellsr{Yes (Fortitude)}
\spelldmg{2d6 negative energy damage \add d6 per two caster levels above 2nd}
\begin{spelleffect}
  The touched creature takes damage. Since undead are powered by negative energy, this spell heals them instead of dealing damage. You must succeed on a melee touch attack to hit a target that does not allow you to touch it. 
\end{spelleffect}

\spellsection{Inflict Moderate Wounds}
\spellschool{Necromancy (Vitalism) [Negative]}
\spelllvl{Clr 2, Sor/Wiz 2}
\spelldmg{4d6 negative energy damage \add d6 per two caster levels above 4th}
\begin{spelleffect}
  This spell functions like \spell{inflict light wounds}, except that for every 20 points of damage dealt in excess of the subject's hit points, it can instead inflict 1 point of critical damage.
\end{spelleffect}
\begin{spellnotes}
  This effect can cause a creature to begin dying without being disabled first.
\end{spellnotes}

\spellsection{Interposing Hand}
\spelldesc{You create a floating, disembodied hand made of magical force that shields you from your foe's blows.}
\spellschool{Evocation (Control) [Force]}
\spelllvl{Evoc 2}
\spellrng{\rngmed}
\spelleff{Large hand made of force}
\spelldur{\durshort (D)}
\spellsave{None}
\spellsr{Yes (Fortitude)}
\begin{spelleffect}
  The hand created by this spell stays between you and one opponent, providing you with cover (\plus4 AC) from that creature. In addition, if the creature is Large size or smaller, it moves at half speed while moving towards you. 
  \par If you cannot see the hand's target, it will stop moving until it is directed to a visible target. The hand does not pursue opponents.
  \par An interposing hand is 10 feet long and about that wide with its fingers outstretched. It has half as many hit points as you do when you're undamaged, and its AC is 15 (\minus1 size, \plus6 natural). It takes damage as a normal creature, but most magical effects that don't cause damage do not affect it.
\end{spelleffect}
\begin{spellnotes}
  \par The hand never provokes attacks of opportunity from opponents. It cannot push through a \spell{wall of force} or enter an \spell{antimagic field}, but it suffers the full effect of a \spell{prismatic wall} or \spell{prismatic sphere}. The hand makes saving throws as its caster.
  \spell{Disintegrate} or a successful \spell{dispel magic} destroys the hand without a saving throw. Directing the hand to a new target is a swift action.
\end{spellnotes}

\spellsection{Invisibility Purge}
\spellschool{Abjuration (Negation)}
\spelllvl{Clr 2, Sor/Wiz 2}
\spellarea{\arealarge radius emanation, centered on you}
\spelldur{\durlong (D)}
\begin{spelleffect}
  You surround yourself with a mobile sphere of power that suppresses all forms of invisibility. Anything invisible becomes visible while in the area.
\end{spelleffect}

\spellsection{Knock}
\spellschool{Evocation (Control)}
\spelllvl{Evoc 2}
\spellcmp{V}
\spellrng{\rngclose}
\spelltgt{One Medium or smaller object}
\spelldur{Instantaneous; see text}
\spellsave{None}
\spellsr{No}
\begin{spelleffect}
  The knock spell telekinetically opens stuck, barred, locked, held, or arcane locked objects. If the object is stuck or held, you can immediately make an Strength check to break it open, using your caster level instead of your Strength. Others can aid you on this check as normal. In addition, if the object is locked, you can immediately make a Disable Device check to open the lock as if you had rolled a 20 on the check. You get a bonus on the Disable Device check equal to half your caster level.
\end{spelleffect}
\begin{spellnotes}
  If knock is cast on an \spellindirect{arcane lock}{arcane locked} door, make a caster level check against a DC of 11 \add the caster level of the \spell{arcane lock}. If you succeed, the \spell{arcane lock} is suppressed for 10 minutes. If you fail, you may still bypass the door with the checks above, if possible.
\end{spellnotes}

\spellsection{Lesser (Spell Name)}
\par Any spell whose name begins with lesser is alphabetized in this chapter according to the second word of the spell name. Thus, the description of a lesser spell appears near the description of the spell on which it is based. Spell chains that have lesser spells in them include those based on the spells confusion, geas, globe of invulnerability, planar ally, planar binding, and restoration.

\spellsection{Lifelink}
\spelldesc{You bind your foe's life force to yours, leaving them vulnerable to your magic.}
\spellschool{Necromancy (Life)}
\spelllvl{Necro 1}
\spellrng{\rngfar}
\spelltgt{One living creature}
\spelldur{\durshort (D)}
\spellsave{Will negates}
\spellsr{Yes (Will)}
\begin{spelleffect}
  The subject is considered to be within \rngclose range of you for determining the range of your spells and spell-like abilities.
\end{spelleffect}

\spellsection{Light}
\spellschool{Illusion (Figment) [Light]}
\spelllvl{Clr 1, Drd 1, Pal 1, Sor/Wiz 1}
\spellrng{\rngtouch}
\spelltgt{Object touched}
\spelldur{\durlong (D)}
\spellsave{None}
\spellsr{No}
\begin{spelleffect}
  This spell causes an object to glow like a torch, shedding bright light in a \areamed radius (and dim light for an additional 20 feet) from the point you touch. The effect is immobile, but it can be cast on a movable object.
  
  As a swift action, you can suppress or intensify the light, preventing the object from shedding light or causing it to shed light in up to a \arealarge radius (and dim light for an additional 50 feet). Either effect lasts for 1 round.
\end{spelleffect}
\begin{spellnotes}
  A light spell (one with the light descriptor) counters and dispels a darkness spell (one with the darkness descriptor) of an equal or lower level. \spell{Light} taken into an area of magical darkness does not function.
\end{spellnotes}

\spellsection{Locate Object}
\spellschool{Divination (Awareness) [Detection]}
\spelllvl{Clr 2, Knowledge 2, Sor/Wiz 2}
\spellrng{\rngfar}
\spelldur{\durmed (D)}
\spellsave{None}
\spellsr{No}
\begin{spelleffect}
  You sense the direction of a well-known or clearly visualized object. You can search for general items, in which case you locate the nearest one of its kind if more than one is within range. Attempting to find a certain item requires a specific and accurate mental image; if the image is not close enough to the actual object, the spell fails. You cannot specify a unique item unless you have observed that particular item firsthand (not through divination).
\end{spelleffect}
\begin{spellnotes}
  The spell is blocked by even a thin sheet of lead, but not by other materials. Creatures cannot be found by this spell.
\end{spellnotes}

\spellsection{Longstrider}
\spellschool{Transmutation (Augment)}
\spelllvl{Drd 1, Travel 1}
\spellrng{\rngpers}
\spelltgt{You}
\spelldur{\durlong (D)}
\begin{spelleffect}
  This spell increases your base land speed by 10 feet. (This adjustment counts as an enhancement bonus.) It has no effect on other modes of movement, such as burrow, climb, fly, or swim.
\end{spelleffect}

\pdfbookmark[2]{M}{SpellDescriptionsM}
\begin{comment}
\subsubsection{M}
\end{comment}

\spellsection{Mage Armor}
\spelldesc{You create an invisible but tangible field of force that surrounds you, protecting you from attacks.}
\spellschool{Abjuration (Shielding) [Force]}
\spelllvl{Sor/Wiz 1}
\spellrng{Personal}
\spelltgt{You}
\spelldur{\durlong (D)}
\begin{spelleffect}
  You gain a \plus2 armor modifier to AC. \bonusscalingdescription
  \par Unlike mundane armor, \spell{mage armor} entails no armor check penalty, arcane spell failure chance, or speed reduction.
\end{spelleffect}
\begin{spellnotes}
  This armor is treated as a separate piece or armor from any other armor the creature is wearing, so it does not stack with any existing armor modifier. Since \spell{mage armor} is made of force, incorporeal creatures can't bypass it the way they do normal armor.
  
  If you become subject to the \spell{shield} spell during the duration of this spell, the \spell{shield} spell lasts until this spell's duration ends.
\end{spellnotes}

\spellsection{Mage Hand}
\spellschool{Evocation (Control)}
\spelllvl{Sor/Wiz 1}
\spellrng{\rngclose}
\spelltgt{One nonmagical, unattended object weighing up to 5 lb.}
\spelldur{\durshort}
\spellsave{None}
\spellsr{No}
\begin{spelleffect}
  You point your finger at an object and can lift it and move it in any direction from a distance. By directing the spell as a swift action, you can propel the object as far as 15 feet in any direction each round, though the spell ends if the distance between you and the object ever exceeds the spell's range.
\end{spelleffect}
\begin{spellnotes}
  Fine manipulation, including any motion other than simply moving the object in a particular direction, is not possible with this spell.
\end{spellnotes}

\spellsection{Magic Fang}
\spellschool{Transmutation (Augment)}
\spelllvl{Drd 2}
\spellrng{\rngclose}
\spelltgt{One creature}
\spelldur{\durshort}
\spellsave{Will negates (harmless)}
\spellsr{Yes (Will)}
\begin{spelleffect}
  This spell makes one of the subject's natural weapons a \plus2 magic weapon, granting a \plus2 bonus to attack and damage rolls. \bonusscalingdescription
\end{spelleffect}
\begin{spellnotes}
  The spell does not change an unarmed strike's damage from nonlethal damage to lethal damage. \spell{Magic fang} can be made permanent with a \spell{permanency} spell.
\end{spellnotes}

\spellsection{Magic Missile}
\spellschool{Evocation (Control) [Force]}
\spelllvl{Sor/Wiz 1}
\spellrng{\rngclose}
\spellarea{\areamed radius limit}
\spelltgts{Creatures in the area}
\spelldur{Instantaneous}
\spellsave{None}
\spellsr{Yes (Fortitude)}
\spelldmg{2d4 force damage \add d4 per two levels above 2nd; see text}
\begin{spelleffect}
  Two missiles of magical energy dart forth from your fingertip and strike creatures you designate in the area, dealing 1d4 damage each. A single missile can strike only one creature. For every two caster levels above 2nd, you gain an additional missile.
  The missiles strike unerringly, even if the target has cover or concealment. Specific parts of a creature can't be singled out. Inanimate objects are not damaged by the spell. You must designate targets before you check for spell resistance or roll damage.
\end{spelleffect}

\spellsection{Magic Vestment}
\spellschool{Transmutation (Augment)}
\spelllvl{Clr 1, Sor/Wiz 1}
\spellrng{\rngclose}
\spelltgt{One suit of armor or shield}
\spelldur{\durmed}
\spellsave{Will negates (harmless, object)}
\spellsr{Yes (Will)}
\begin{spelleffect}
  You imbue body armor or a shield with a \plus2 enhancement bonus, giving its bearer a \plus2 bonus to AC. \bonusscalingdescription
\end{spelleffect}
\begin{spellnotes}
  An outfit of regular clothing counts as armor that grants no AC bonus for the purpose of this spell.
\end{spellnotes}

\spellsection{Magic Weapon}
\spellschool{Transmutation (Augment)}
\spelllvl{Clr 2, Sor/Wiz 2}
\spellrng{\rngclose}
\spelltgt{One weapon or fifty projectiles (all of which must be in contact with each other at the time of casting)}
\spelldur{\durshort}
\spellsave{Will negates (harmless, object)}
\spellsr{Yes (Will)}
\begin{spelleffect}
  You imbue a weapon or stack of projectiles with a \plus2 enhancement bonus, giving its wielder a \plus2 bonus to attack and damage. \bonusscalingdescription
\end{spelleffect}
\begin{spellnotes}
  You can't cast this spell on a natural weapon, such as an unarmed strike (instead, see \spell{magic fang}). A monk's unarmed strike is considered a weapon, and thus it can be enhanced by this spell.
  \par If you use this spell to enhance projectiles, the projectiles must be of the same kind, and they have to be together (in the same quiver or other container). Projectiles, but not thrown weapons, lose their transmutation when used. (Treat darts and shuriken as projectiles, rather than as thrown weapons, for the purpose of this spell.)
\end{spellnotes}

\spellsection{Mass (Spell Name)}
\par Any spell whose name begins with mass is alphabetized in this chapter according to the second word of the spell name. Thus, the description of a mass spell appears near the description of the spell on which it is based. Spell chains that have mass spells in them include those based on the spells charm monster, cure critical wounds, cure light wounds, cure moderate wounds, cure serious wounds, enlarge person, heal, hold monster, hold person, inflict critical wounds, inflict light wounds, inflict moderate wounds, inflict serious wounds, invisibility, reduce person, suggestion, totemic mind, and totemic power.

\spellsection{Mental Retribution}
\spellschool{Abjuration/Enchantment (Inhibition, Shielding) [Mind-Affecting]}
\spelllvl{Sor/Wiz 1}
\spellrng{\rngclose}
\spelltgt{One creature; see text}
\spellarea{\rngmed radius limit centered on the subject; see text}
\spelldur{\durshort or until discharged/5 rounds}
\spellsave{Will negates (harmless)}
\spellsr{Yes (Will)}
\begin{spelleffect}
  The subject gains a faintly shimmering aura. The first time it is attacked by a creature within the area, the spell is discharged, and the attacking creature is bewildered for 5 rounds. A successful Will save can prevent the subject from gaining the aura, but there is no saving throw against the bewildering effect.
\end{spelleffect}

\spellsection{Message}
\spellschool{Divination (Communication)}
\spelllvl{Sor/Wiz 1}
\spellcmp{S}
\spellrng{\rngmed}
\spellarea{\areamed radius limit}
\spelltgts{Five creatures within the area}
\spelldur{\durlong}
\spellsave{None}
\spellsr{No}
\begin{spelleffect}
  You can whisper messages and receive whispered replies with little chance of being overheard. You point your finger at each creature you want to receive the message. When you whisper, the whispered message is audible to all targeted creatures within range. Magical silence, 1 foot of stone, 1 inch of common metal (or a thin sheet of lead), or 3 feet of wood or dirt blocks the spell. The message does not have to travel in a straight line. It can circumvent a barrier if there is an open path between you and the subject, and the path's entire length lies within the spell's range. The creatures that receive the message can whisper a reply that you hear. The spell transmits sound, not meaning. It doesn't transcend language barriers.
\end{spelleffect}
\begin{spellnotes}
  To speak a message, you must mouth the words and whisper, possibly allowing observers the opportunity to read your lips.
\end{spellnotes}

\spellsection{Mirror Image}
\spelldesc{You create illusory duplicates of yourself that make it difficult for enemies to know which image to attack.}
\spellschool{Illusion (Figment)}
\spelllvl{Illus 2}
\spellrng{Personal; see text}
\spelltgt{You}
\spelldur{\durshort (D)}
\begin{spelleffect}
  This spell creates an illusory duplicate of yourself that mimics your movements perfectly. Enemies attempting to attack you or cast spells at you must select which to attack. Generally, roll randomly to see whether the selected target is real or a figment. An image's AC is 10 \add your size modifier. You gain an additional image at 8th, 14th, and 20th level. 
  \par If an image is hit, it is destroyed. If you are hit, your attacker knows the attack was successful, and can ignore the image. You can create new images to replace destroyed images as a swift action, preventing your foes from knowing which image to attack.
  \par You can move into and through your duplicates on your turn. When you and the image separate, observers can't use vision or hearing to tell which one is you and which the image. The duplicates may also move through each other. The figments mimic your actions, pretending to cast spells when you cast a spell, drink potions when you drink a potion, levitate when you levitate, and so on.
  \par Mirror images can be attacked like any other creature. They count as separate creatures, and can be targeted separately by spells like \spell{magic missile} or feats like Whirlwind Attack, though they are not destroyed by area spells. Destroying an image counts as dropping a creature for the purpose of the Cleave feat and similar abilities.
\end{spelleffect}
\begin{spellnotes}
  An attacker must be able to see the images to be fooled. If you are invisible or an attacker shuts his or her eyes, the spell has no effect. (Being unable to see carries the same penalties as being blinded.)
\end{spellnotes}

\spellsection{Obscuring Mist}
\spelldesc{You conjure a bank of fog that arises around you, concealing you and your allies.}
\spellschool{Conjuration (Creation) [Fog]}
\spelllvl{Clr 1, Drd 1, Sor/Wiz 1, Water 1}
\spellarea{\areamed radius cylinder-shaped spread centered on you}
\begin{spelleffect}
  This spell functions like \spell{fog cloud}, except that the fog created is centered on you.
\end{spelleffect}

\spellsectioncomma{Precognition}{Lesser}
\spelldesc{You extend your mind a fraction of a second into the future, allowing you to strike at your foes more effectively.}
\spellschool{Divination (Knowledge)}
\spelllvl{Div 1}
\spellrng{Personal}
\spelltgt{You}
\spelldur{\durshort (D)}
\begin{spelleffect}
  You gain a \plus2 bonus to your attack and weapon damage rolls. \bonusscalingdescription
\end{spelleffect}

\spellsection{Protection from Chaos}
\spellschool{Abjuration (Interdiction) [Lawful]}
\spelllvl{Clr 1, Law 1, Pal 1, Sor/Wiz 1}
\begin{spelleffect}
  This spell functions like \spell{protection from evil}, except that it protects against lawful effects.
\end{spelleffect}

\spellsection{Protection from Evil}
\spelldesc{You guard your ally with a faint pure white aura, shielding him from evil influence.}
\spellschool{Abjuration (Interdiction) [Good]}
\spelllvl{Clr 1, Good 1, Pal 1, Sor/Wiz 1}
\spellrng{\rngclose}
\spelltgt{One creature}
\spelldur{\durshort (D)}
\spellsave{Will negates (harmless)}
\spellsr{No; See text}
\begin{spelleffect}
  The subject gains a \plus2 bonus on saving throws. \bonusscalingdescription

  In addition, the spell blocks any evil attempt to possess or exercise mental control over the creature (such as any domination effect). The protection does not prevent such effects from targeting the protected creature, but it suppresses the effect for the duration of the \spell{protection from evil} spell. If the \spell{protection from evil} spell ends before the effect granting mental control does, the would-be controller would then be able to mentally command the controlled creature. Likewise, the barrier keeps out a possessing life force but does not expel one if it is in place before the spell is cast. This effect works only against attacks by evil creatures or from evil effects.
\end{spelleffect}

\spellsection{Protection from Good}
\spellschool{Abjuration (Interdiction) [Evil]}
\spelllvl{Clr 1, Evil 1, Sor/Wiz 1}
\begin{spelleffect}
  This spell functions like\spell{protection from evil}, except that it protects against good effects.
\end{spelleffect}

\spellsection{Protection from Law}
\spellschool{Abjuration (Interdiction) [Chaotic]}
\spelllvl{Chaos 1, Clr 1, Sor/Wiz 1}
\begin{spelleffect}
  This spell functions like \spell{protection from evil}, except that it protects against chaotic effects.
\end{spelleffect}

\spellsection{Quiet Mind}
\spellschool{Transmutation (Augment)}
\spelllvl{Sor/Wiz 1}
\spelltime{1 swift action}
\spellrng{Personal}
\spelltgt{You}
\spelldur{1 round or until discharged}
\begin{spelleffect}
  You gain a \plus10 bonus to Concentration checks. After you cast a spell, this spell ends.
\end{spelleffect}

\pdfbookmark[2]{Q-R}{SpellDescriptionsQR}
\begin{comment}
\subsubsection{Q-R}
\end{comment}

\spellsection{Ray of Clumsiness}
\spelldesc{You fire a coruscating ray from your hand. When it strikes your foe, he becomes clumsier and less agile.}
\spellschool{Necromancy (Flesh)}
\spelllvl{Sor/Wiz 1}
\spellrng{\rngclose}
\spelleff{Ray}
\spelldur{\durshort}
\spellsave{Fortitude half}
\spellsr{Yes (Fortitude)}
\begin{spelleffect}
  You must succeed on a ranged touch attack. The subject takes a \minus4 penalty to Dexterity.
\end{spelleffect}
\begin{spellnotes}
  The subject's Dexterity score cannot drop below 1.
\end{spellnotes}

\spellsection{Reduce Person}
\spellschool{Transmutation (Polymorph) [Size-Affecting]}
\spelllvl{Trans 2}
\spelltime{Full-round action}
\spellrng{\rngclose}
\spelltgt{One humanoid creature}
\spelldur{\durshort (D)}
\spellsave{Fortitude negates}
\spellsr{Yes (Fortitude)}
\begin{spelleffect}
  This spell causes instant diminution of a humanoid creature, halving its height, length, and width and dividing its weight by 8. This decrease changes the creature's size category to the next smaller one. This has several effects.
  \begin{itemize*} 
    \item \minus10 ft. inherent bonus to movement speed.
    \item \plus1 inherent bonus to attack rolls and AC due to its decreased size.
  \item \minus2 penalty to Strength.
  \item \plus2 bonus to Dexterity. \bonusscalingdescription
  \end{itemize*}
  \par A Small humanoid creature whose size decreases to Tiny has a space of 2-1/2 feet and a natural reach of 0 feet (meaning that it must enter an opponent's square to attack). A Large humanoid creature whose size decreases to Medium has a space of 5 feet and a natural reach of 5 feet.
  \par All equipment worn or carried by a creature is similarly reduced by the spell. Melee and projectile weapons deal less damage. Other magical properties are not affected by this spell. Any reduced item that leaves the reduced creature's possession (including a projectile or thrown weapon) instantly returns to its normal size. This means that thrown weapons deal their normal damage (projectiles deal damage based on the size of the weapon that fired them).
\end{spelleffect}
\begin{spellnotes}
  Multiple magical effects that reduce size do not stack.
  \par \spellindirect{reduce person}{Reduce person} counters and dispels \spell{enlarge person}.
  \par \spellindirect{reduce person}{Reduce person} can be made permanent with a \spell{permanency} spell.
\end{spellnotes}

\spellsection{Resist Energy}
\spellschool{Abjuration (Shielding)}
\spelllvl{Clr 2, Drd 2, Pal 2, Protection 2, Sor/Wiz 2}
\spellrng{Touch}
\spelltgt{Creature touched}
\spelldur{\durlong or until discharged}
\spellsave{Fortitude negates (harmless)}
\spellsr{Yes (Fortitude)}
\begin{spelleffect}
  The subject gains energy damage reduction 10 against whichever of the five energy types that you select: acid, cold, electricity, fire, or sonic. This damage reduction increases by 1 per caster level above 4th.
  \par The spell can absorb a maximum amount of damage equal to 10 points per caster level. After it absorbs its maximum amount of damage, the spell ends.
\end{spelleffect}
\begin{spellnotes}
  This spell's damage reduction allows the subject to ignore the first 10 energy damage it takes each round of the appropriate type.

  Resist energy absorbs only damage. The subject could still suffer unfortunate side effects. The spell protects the recipient's equipment as well.
  \par \spellindirect{resist energy}{Resist energy} overlaps (and does not stack with) \spell{protection from energy}. If a character is shielded by both spells, the \spellindirect{protection from energy}{protection} spell absorbs damage until its power is exhausted. A character can only be affected by one \spell{resist energy} spell at once.
\end{spellnotes}

\spellsection{Retrieve}
\spellschool{Conjuration (Translocation) [Teleportation]}
\spelllvl{Conj 1}
\spellrng{\rngclose}
\spelltgt{One object you can hold or carry in one hand, weighing up to 10 lb./level}
\spelldur{Instantaneous}
\spellsave{None (object)}
\spellsr{Yes (Will)}
\begin{spelleffect}
  You teleport an item you can see within range directly to your hand. If the object is attended, this spell automatically fails.
\end{spelleffect}

\spellsection{Reveal Death}
\spelldesc{You grant a creature a vision of its death - whether immediate or far in the future.}
\spellschool{Divination (Knowledge)}
\spelllvl{Death 2, Div 2}
\spellrng{\rngmed}
\spelltgt{One creature}
\spelldur{\durshort}
\spellsave{None}
\spellsr{Yes (Will)}
\begin{spelleffect}
    This spell has different effects depending on the version chosen.
    \par \subspell{Distant Demise} The subject gains a \plus2 bonus to saving throws. In addition, it is not staggered while at 0 hit points. Further damage is still critical damage and can cause the creature to begin dying as normal. 
    \par \subspell{Imminent Demise} The subject becomes vulnerable.
\end{spelleffect}
\begin{spellnotes}
  \vulnerableexplanation
\end{spellnotes}

\spellsection{Sanctuary}
\spellschool{Abjuration/Enchantment (Compulsion, Shielding)}
\spelllvl{Abjur 1, Clr 1, Pal 1, Protection 1}
\spellrng{Touch}
\spelltgt{Creature touched}
\spelldur{\durshort}
\spellsave{Will negates (harmless) and Will negates; see text}
\spellsr{Yes (Will)}
\begin{spelleffect}
  Any opponent attempting to strike or otherwise directly attack the shielded creature, even with a targeted spell, must attempt a Will save. If the save succeeds, the opponent can attack normally and is unaffected by that casting of the spell. If the save fails, the opponent can't follow through with the attack, that part of its action is lost, and it can't directly attack the shielded creature for the duration of the spell. Those not attempting to attack the subject remain unaffected. This spell does not prevent the shielded creature from being attacked or affected by area or effect spells. The subject cannot attack without breaking the spell but may use nonattack spells or otherwise act.
\end{spelleffect}

\pdfbookmark[2]{S}{SpellDescriptionsS}
\begin{comment}
\subsubsection{S}
\end{comment}

\spellsection{Scorching Ray}
\spelldesc{You blast your enemies with fiery rays.}
\spellschool{Evocation (Energy) [Fire]}
\spelllvl{Fire 2, Sor/Wiz 2}
\spellrng{\rngclose}
\spelleff{One or more rays}
\spelldur{Instantaneous}
\spellsave{None}
\spellsr{Yes (Reflex)}
\spelldmg{4d6 fire damage \add d6 per two caster levels above 4th}
\begin{spelleffect}
  You may fire up to three rays at the same or separate targets. Each ray requires a ranged touch attack to hit. You may split the damage among the rays as you choose. The rays may be fired at the same or different targets, but all must be aimed at targets within 30 feet of each other and fired simultaneously. Precision damage can only be applied with one of the rays.
\end{spelleffect}

\spellsection{See Invisibility}
\spellschool{Divination (Revelation)}
\spelllvl{Sor/Wiz 2}
\spellrng{Touch}
\spelltgt{Touched creature}
\spelldur{\durlong (D)}
\begin{spelleffect}
  You grant the touched creature the ability to see any objects or beings that are invisible within its range of vision, as well as any that are ethereal, as if they were normally visible. Such creatures are visible as translucent shapes, allowing the target to easily discern the difference between visible, invisible, and ethereal creatures.
\end{spelleffect}
\begin{spellnotes}
  The spell does not reveal the method used to obtain invisibility. It does not reveal illusions or enable you to see through opaque objects. It does not reveal creatures who are simply hiding, concealed, or otherwise hard to see.
  \par See invisibility can be made permanent with a permanency spell.
\end{spellnotes}

\spellsection{Shape Wood}
\spellschool{Transmutation (Alteration)}
\spelllvl{Drd 2}
\spellrng{Touch}
\spelltgt{One touched piece of wood no larger than 10 cu. ft. \add 1 cu. ft./level}
\spelldur{Instantaneous}
\spellsave{Fortitude negates (object)}
\spellsr{Yes (Fortitude)}
\begin{spelleffect}
  This spell enables you to form one existing piece of wood into any shape that suits your purpose. While it is possible to make crude coffers, doors, and so forth, fine detail isn't possible. There is a 30\% chance that any shape that includes moving parts simply doesn't work.
\end{spelleffect}

\spellsection{Share Pain}
\spellschool{Abjuration/Necromancy (Life, Shielding)}
\spelllvl{Clr 2, Pal 2, Protection 2, Sor/Wiz 2}
\spellrng{\rngmed}
\spelltgts{You and one willing creature}
\spelldur{\durlong (D)}
\spellsave{None}
\spellsr{Yes (Will)}
\begin{spelleffect}
  This spell creates a connection between you and a willing subject. As you cast the spell, you decide which creature will be protected. When the protected creature would take damage to its hit points, it instead takes half of that damage (rounded down), and you lose hit points equal to the other half of the damage (rounded up).

  If the subject is out of range of you, the effect is suppressed until the subject returns within the spell's range.
\end{spelleffect}
\begin{spellnotes}
  The loss of hit points caused by this spell is not damage, and is not affected by damage reduction or other abilities which affect damage. When this spell ends, subsequent damage is no longer divided between the subject and you, but damage already shared is not reassigned.
\end{spellnotes}

\spellsection{Shatter}
\spelldesc{You create a loud, ringing noise that sunders solid objects.}
\spellschool{Evocation (Energy) [Sonic]}
\spelllvl{Destruction 2, Sor/Wiz 2}
\spellrng{\rngclose}
\spelltgtorarea{One solid object or one crystalline creature; or \areasmall radius spread}
\spelldur{Instantaneous}
\spellsave{Will negates (object)/Will negates (object) or Fortitude half; see text}
\spellsr{Yes (Will)}
\spelldmg{4d6 sonic damage \add d6 per two levels after 4th}
\begin{spelleffect}
  Used as an area attack, shatter destroys nonmagical objects of crystal, glass, ceramic, or porcelain. All such objects within a \areasmall radius of the point of origin are smashed into dozens of pieces by the spell. Objects weighing more than 1 pound per your level are not affected, but all other objects of the appropriate composition are shattered.
  \par Alternatively, you can target a single solid object or crystalline creature. In the case of large objects, such as walls, you target a 5 ft. cube. The target takes damage, with a Fortitude save for half damage.
  \par A creature holding vulnerable objects can attempt a Will save to negate any effect on to those objects.
\end{spelleffect}

\spellsection{Shield}
\spelldesc{You create an invisible, heavy shield-sized mobile disk of force. It hovers in front of your ally, automatically moving to ward off enemy blows.}
\spellschool{Abjuration (Shielding) [Force]}
\spelllvl{Sor/Wiz 1}
\spellrng{Touch}
\spelltgt{Creature touched}
\spelldur{\durshort (D); see text}
\spellsave{Will negates (harmless)}
\spellsr{Yes (Will)}
\begin{spelleffect}
  The subject gains a \plus2 shield modifier to AC. \bonusscalingdescription The subject is not encumbered or hindered in any way by the shield.
\end{spelleffect}
\begin{spellnotes}
  This shield is considered to be separate from any other shields the creature is using, so it never stacks with existing shield modifiers. Since the \spell{shield} is made of force, incorporeal creatures can't bypass it the way they do normal shields.
  
  If you cast this spell on a creature subject to the \spell{mage armor} spell, its duration lasts until the \spell{mage armor} spell expires. 
\end{spellnotes}

\spellsection{Shield of Faith}
\spelldesc{You create a shimmmering, magical shield that protects your ally as long as you maintain faith.}
\spellschool{Abjuration (Shielding)}
\spelllvl{Clr 1, Pal 1, Protection 1}
\begin{spelleffect}
  This spell functions like \spell{shield}, except that it is not a force effect, so it does not protect against incorporeal touch attacks. It has no special effect when cast on a creature with \spell{mage armor}.
\end{spelleffect}
\begin{spelleffect}
  You can maintain concentration on this spell as a swift action.
\end{spelleffect}

\spellsection{Shillelagh}
\spellschool{Transmutation}
\spelllvl{Drd 1}
\spellrng{Touch}
\spelltgt{One touched nonmagical oak club or quarterstaff}
\spelldur{\durshort}
\spellsave{Will negates (object)}
\spellsr{Yes (Will)}
\begin{spelleffect}
  Your own nonmagical club or quarterstaff becomes a weapon with a \plus2 enhancement bonus on attack and damage rolls. \bonusscalingdescription (A quarterstaff gains this enhancement for both ends of the weapon.) In addition, the weapon deals damage as if it were one size category larger (a Small club or quarterstaff so transmuted deals 1d6 points of damage, a Medium 1d8, and a Large 1d10).
\end{spelleffect}
\begin{spellnotes}
  These effects only occur when the weapon is wielded by you. If you do not wield it, the weapon behaves as if unaffected by this spell.
\end{spellnotes}

\spellsection{Shocking Grasp}
\spelldesc{You deliver a powerful electrical shock to your foe.}
\spellschool{Evocation (Energy) [Electricity]}
\spelllvl{Destruction 1, Sor/Wiz 1}
\spellrng{Touch}
\spelltgt{Creature or object touched}
\spelldur{Instantaneous}
\spellsave{None/Fortitude negates}
\spellsr{Yes (Fortitude)}
\spelldmg{2d6 electricity damage \add d6 per two caster levels above 2nd}
\begin{spelleffect}
  If you hit with a touch attack, the target takes damage. If it fails a Fortitude save, it is also staggered for 1 round. When delivering the jolt, you gain a \plus2 circumstance bonus to attack if the opponent is wearing metal armor (or made out of metal, carrying a lot of metal, or the like).
\end{spelleffect}
\begin{spellnotes}
   A staggered character may take a single move action or standard action each round, but not both. She cannot take full-round actions, but she may take swift actions. In addition, she is vulnerable, causing her to take a \minus2 penalty on attack rolls, saving throws, checks, DCs, and AC.
\end{spellnotes}

\spellsection{Silence}
\spellschool{Illusion (Glamer)}
\spelllvl{Clr 2, Trickery 2}
\spellrng{\rngmed}
\spellarea{\areamed radius emanation centered on a creature, object, or point in space}
\spelldur{\durshort (D)}
\spellsave{Will negates; see text or none (object)}
\spellsr{Yes (Will); see text}
\begin{spelleffect}
  Upon the casting of this spell, complete silence prevails in the affected area. No sound can be heard or made in the area, but sound passes through the area normally. Spellcasters are treated as being deafened for the purpose of casting spells with verbal components, and suffer a 20\% chance of spell failure. The spell can be cast on a point in space, but the effect is stationary unless cast on a mobile object. The spell can be centered on a creature, and the effect then radiates from the creature and moves as it moves. An unwilling creature who enters the spell's area can attempt a Will save to negate the spell's effect on them and can use spell resistance, if any. A creature who successfully resists the spell can hear and make sound normally, but still cannot be hear or be heard by other creatures in the area (unless they also resisted the spell). Items in a creature's possession or magic items that emit sound receive the benefits of saves and spell resistance, but unattended objects and points in space do not. 
\end{spelleffect}
\begin{spellnotes}
  This spell provides a defense against sound-dependent effects. Sonic effects are too powerful for magic such as this to muffle, and function normally.
\end{spellnotes}

\spellsection{Silent Image}
\spellschool{Illusion (Figment)}
\spelllvl{Illus 2}
\spellrng{\rngmed}
\spellarea{\areamed radius limit}
\spelleff{Visual figment within the area}
\spelldur{Concentration}
\spellsave{Will disbelief (if interacted with)}
\spellsr{No}
\begin{spelleffect}
  This spell creates the visual illusion of an object, creature, or force, as visualized by you. The illusion does not create sound, smell, texture, or temperature. You can move the image within the limits of the size of the effect.
\end{spelleffect}

\spellsection{Sleep}
\spellschool{Enchantment (Compulsion) [Mind-Affecting, Sleep]}
\spelllvl{Sor/Wiz 1}
\spellrng{\rngmed}
\spelltgt{One living creature}
\spelldur{\durshort}
\spellsave{Will negates}
\spellsr{Yes (Will)}
\begin{spelleffect}
  The subject is fatigued and attempts to go to sleep as soon as possible, though it will not stop fighting to do so. Awakening a creature put to sleep by this spell is difficult, and requires a standard action.
\end{spelleffect}

\spellsection{Slow}
\spelldesc{You decelerate your enemy's motions, causing her to move and act more slowly than normal.}
\spellschool{Transmutation (Temporal)}
\spelllvl{Sor/Wiz 2}
\spellrng{\rngclose}
\spelltgt{One creature}
\spelldur{\durshort}
\spellsave{Will negates}
\spellsr{Yes (Will)}
\begin{spelleffect}
  The subject is slowed. This has two effects.
  \par A slowed creature can take only a single move action or standard action each turn, but not both (nor may it take full-round actions).
  \par A slowed creature takes a \minus2 penalty to attack rolls, Strength and Dexterity-based checks, and armor class.
\end{spelleffect}
\begin{spellnotes}
  \spell{Slow} counters and dispels \spell{haste}.
\end{spellnotes}

\spellsection{Soften Earth and Stone}
\spellschool{Transmutation (Alteration) [Earth]}
\spelllvl{Drd 2, Earth 2}
\spellrng{\rngclose}
\spellarea{\arealarge radius}
\spelldur{Instantaneous}
\spellsave{None}
\spellsr{No}
\begin{spelleffect}
  When this spell is cast, all natural, undressed earth or stone in the spell's area is softened. Wet earth becomes thick mud, dry earth becomes loose sand or dirt, and stone becomes soft clay that is easily molded or chopped. You affect a 10-foot square area to a depth of 1 to 4 feet, depending on the toughness or resilience of the ground at that spot. Magical, enchanted, dressed, or worked stone cannot be affected. Earth or stone creatures are not affected.
  \par A creature in mud must succeed on a Reflex save or be caught for 1 round and unable to move, attack, or cast spells. A creature that succeeds on its save can move through the mud at half speed, and it can't run or charge.
  \par Loose dirt is not as troublesome as mud, but All creatures within the area can move at only half their normal speed and can't run or charge over the surface.
  \par Stone softened into clay does not hinder movement, but it does allow characters to cut, shape, or excavate areas they may not have been able to affect before.
  \par While \spell{soften earth and stone} does not affect dressed or worked stone, cavern ceilings or vertical surfaces such as cliff faces can be affected. Usually, this causes a moderate collapse or landslide as the loosened material peels away from the face of the wall or roof and falls.
  \par A moderate amount of structural damage can be dealt to a manufactured structure by softening the ground beneath it, causing it to settle. However, most well-built structures will only be damaged by this spell, not destroyed.
\end{spelleffect}

\spellsection{Sound Burst}
\spelldesc{You blast an area with a cacophony of sound.}
\spellschool{Evocation (Energy) [Sonic]}
\spelllvl{Brd 2}
\spellrng{\rngclose}
\spellarea{\areasmall radius spread}
\spelldur{Instantaneous}
\spellsave{Fortitude half/Fortitude negates}
\spellsr{Yes (Fortitude)}
\spelldmg{2d6 sonic damage \add d6 per four levels above 4th}
\begin{spelleffect}
  Creatures in the area take damage and are deafened for 5 rounds. A successful Fortitude save halves the damage and negates the deafening.
\end{spelleffect}

\spellsectioncomma{Spelltheft}{Lesser}
\spellschool{Abjuration (Negation) [Magic]}
\spelllvl{Abjur 2, Magic 2}
\spelltgt{One spellcaster, creature, or object}
\begin{spelleffect}
  This spell functions like \spell{lesser dispel magic}, except that you can choose to gain the effects of any spells you dispel as if they had been originally cast on you. The effects last for the remainder of their original durations or for 5 rounds, whichever is shorter. Spells that cannot be cast on you, such as spells which have a range of personal, are simply dispelled.
\end{spelleffect}

\spellsection{Spider Climb}
\spellschool{Transmutation (Imbuement)}
\spelllvl{Drd 2, Sor/Wiz 2, Travel 2}
\spellrng{Touch}
\spelltgt{Creature touched}
\spelldur{\durmed}
\spellsave{Fortitude negates (harmless)}
\spellsr{Yes (Fortitude)}
\begin{spelleffect}
  The subject can climb and travel on vertical surfaces or even traverse ceilings as well as a spider does. The affected creature must have its hands free to climb in this manner. The subject gains a climb speed of 20 feet; furthermore, it need not make Climb checks to traverse a vertical or horizontal surface (even upside down). A spider climbing creature retains its Dexterity and dodge modifiers to Armor Class (if any) while climbing, and opponents get no special bonus to their attacks against it. It cannot, however, use the run action while climbing.
\end{spelleffect}

\spellsection{Spike Growth}
\spellschool{Transmutation (Alteration)}
\spelllvl{Drd 2}
\spellrng{\rngmed}
\spellarea{\areasmall radius}
\spelldur{\durshort (D)}
\spellsave{None/Reflex negates}
\spellsr{Yes (Reflex)}
\begin{spelleffect}
  Any ground-covering vegetation in the spell's area becomes very hard and sharply pointed. In areas of bare earth, roots and rootlets act in the same way. Typically, spike growth can be cast in any outdoor setting except open water, ice, heavy snow, sandy desert, or bare stone. Any foe moving on foot into or through the spell's area takes 1d4 points of physical piercing damage for each 5 feet of movement through the spiked area. Allies suffer no ill effects.
  \par Any creature that takes damage from this spell must also succeed on a Reflex save or suffer injuries to its feet and legs that slow its land speed by one-half. The Reflex save must be repeated each round that the creature moves through the area. This speed penalty lasts for 12 hours or until the injured creature receives magical healing. Another character can remove the penalty by taking 10 minutes to dress the injuries and succeeding on a Heal check against the spell's save DC.
\end{spelleffect}

\spellsection{Spiritual Weapon}
\spelldesc{You bring into being a weapon made of pure force which attacks your foes of its own volition.}
\spellschool{Evocation (Energy) [Force]}
\spelllvl{Clr 2, Pal 2, War 2}
\spellrng{\rngmed}
\spelleff{Magic weapon of force}
\spelldur{\durshort (D)}
\spellsave{None}
\spellsr{Yes (Will)}
\begin{spelleffect}
  The weapon created by this spell attacks once each round on your turn. This functions just as if you were attacking with the weapon, except that you use your casting ability in place of your Strength and you never get multiple attacks with the weapon.
  \par The weapon attacks the same target until you redirect it (a swift action). The weapon is treated as a separate creature for the purpose of overwhelm penalties.
  \par If an attacked creature has spell resistance, you make a spell penetration check the first time the spiritual weapon strikes it. If the weapon is successfully resisted, it cannot harm that creature. If not, the weapon has its normal full effect on that creature for the duration of the spell.
  \par The weapon takes the shape of a weapon favored by your deity or a weapon with some spiritual significance or symbolism to you (see below), and has the same threat range and critical multipliers as a real weapon of its form.
\end{spelleffect}
\begin{spellnotes}
  The \spell{spiritual weapon} strikes as a spell, not as a weapon, so, for example, ignores physical damage reduction. As a force effect, it can strike incorporeal creatures without the normal miss chance associated with incorporeality. If the weapon goes beyond the spell range, if it goes out of your sight, or if you are not directing it, the weapon returns to you and hovers. Even if the spiritual weapon is a ranged weapon, use the spell's range, not the weapon's normal range increment, and switching targets still is a move action.
  \par A \spell{spiritual weapon} cannot be attacked or harmed by physical attacks, but \spell{dispel magic}, \spell{disintegrate}, and similar effects can affect it. A spiritual weapon's AC against touch attacks is 12 (10 \add size bonus for Tiny object).
  \par The weapon that you get is usually a force replica of any weapon from your deity's weapon group. A cleric without a deity gets a weapon based on his alignment. A neutral cleric without a deity can create a spiritual weapon of any alignment, provided he is acting at least generally in accord with that alignment at the time. The weapon groups associated with each alignment are as follows.
  \par Chaos: Axes
  \par Evil: Flexible weapons
  \par Good: Headed weapons
  \par Law: Heavy blades
\end{spellnotes}

\spellsection{Summon Monster I}
\spellschool{Conjuration (Summoning) [see text]}
\spelllvl{Clr 1, Sor/Wiz 1}
\spelltime{Full-round action}
\spellrng{\rngclose}
\spelleff{One summoned creature}
\spelldur{\durshort (D)}
\spellsave{None}
\spellsr{No}
\begin{spelleffect}
  This spell summons an extraplanar creature (typically an outsider, elemental, or magical beast native to another plane). It appears where you designate and acts on your next turn. You must spend a swift action each round to control the creature summoned by this spell. If you do, it attacks your opponents to the best of its ability. You can direct the creature not to attack, to attack particular enemies, or to perform other actions if you can communicate with it. If you do not actively control the creature summoned by this spell, it acts according to its nature.
  \par When you learn this spell, you choose a creature from the 1st-level list on the Summon Monster table. In the case of creatures with multiple options, such as elementals, you must choose one specific kind of creature. You can summon that creature with this or any other summon monster spell.
  \par A summoned monster cannot summon or otherwise conjure another creature, nor can it use any teleportation or planar travel abilities. Creatures cannot be summoned into an environment that cannot support them.
  \par When you use a summoning spell to summon an air, chaotic, earth, evil, fire, lawful, or water creature, it is a spell of that type.
\end{spelleffect}

\spellsection{Summon Monster II}
\spellschool{Conjuration (Summoning) [see text for summon monster I]}
\spelllvl{Clr 2, Conj 1, Sor/Wiz 2}
\spellarea{\areamed radius limit}
\spelleff{One or more summoned creatures within the area}
\begin{spelleffect}
  This spell functions like \spell{summon monster I}, except that you can summon one creature from the 2nd-level list or 1d3 creatures of the same kind from the 1st-level list. When you learn this spell, you choose two creatures from the 2nd-level or lower lists on the Summon Monster table, one at each level. You can summon those creatures with this or any other \spellindirect{summon monster I}{summon monster} spell.
\end{spelleffect}

\spellsection{Summon Nature's Ally I}
\spellschool{Conjuration (Summoning)}
\spelllvl{Drd 1}
\spelltime{Full-round action}
\spellrng{\rngclose}
\spelleff{One summoned creature}
\spelldur{\durshort (D)}
\spellsave{None}
\spellsr{No}
\begin{spelleffect}
  This spell summons a natural creature. It appears where you designate and acts on your next turn. You must spend a swift action each round to control the creature summoned by this spell. If you do, it attacks your opponents to the best of its ability. You can direct the creature not to attack, to attack particular enemies, or to perform other actions if you can communicate with it. If you do not actively control the creature summoned by this spell, it acts according to its nature.
  \par When you learn this spell, you choose a creature from the 1st-level list on the Summon Nature's Ally table. In the case of creatures with multiple options, such as elementals, you must choose one specific kind. You can summon that creature with this or any other \spellindirect{summon nature's ally i}{summon nature's ally} spell. 
  \par A summoned monster cannot summon or otherwise conjure another creature, nor can it use any teleportation or planar travel abilities. Creatures cannot be summoned into an environment that cannot support them.
  \par All the creatures on the table are neutral unless otherwise noted.
\end{spelleffect}

\spellsection{Summon Nature's Ally II}
\spellschool{Conjuration (Summoning)}
\spelllvl{Drd 2}
\spellarea{\areamed radius limit}
\spelleff{One or more summoned creatures within the area}
\begin{spelleffect}
  This spell functions like \spellindirect{summon nature's ally i}{summon nature's ally I}, except that you can summon one 2nd-level creature or 1d3 1st-level creatures of the same kind. When you learn this spell, you choose two creatures from the 2nd-level or lower lists on the Summon Nature's Ally table, one at each level. You can summon those creatures with this or any other \spellindirect{summon nature's ally i}{summon nature's ally} spell.
\end{spelleffect}

\spellsection{Totemic Mind}
\spellschool{Transmutation (Augment)}
\spelllvl{Clr 2, Drd 2, Sor/Wiz 2}
\spellrng{\rngclose}
\spelltgt{One creature}
\spelldur{\durshort}
\spellsave{Will negates (harmless)}
\spellsr{Yes (Will)}
\begin{spelleffect}
  This spell grants creatures the mental power of a totem animal. It has three forms, each of which grants a \plus2 bonus to a mental attribute. \bonusscalingdescription
  \par \subspell{Eagle's Splendor} The transmuted creature becomes more persuasive and personally forceful, gaining a bonus to Charisma.
  \par \subspell{Fox's Cunning} The transmuted creature becomes smarter, gaining a bonus to Intelligence.
  \par \subspell{Owl's Wisdom} The transmuted creature becomes more perceptive, gaining a bonus to Wisdom.
\end{spelleffect}

\spellsection{Totemic Power}
\spellschool{Transmutation (Augment)}
\spelllvl{Clr 2, Drd 2, Sor/Wiz 2, Strength 2}
\spellrng{\rngclose}
\spelltgt{One creature}
\spelldur{\durshort}
\spellsave{Fortitude negates (harmless)}
\spellsr{Yes (Fortitude)}
\begin{spelleffect}
  This spell grants creatures the physical power of an animal. It has three forms, each of which grants a \plus2 bonus to a mental attribute. \bonusscalingdescription
  \par \subspell{Bear's Endurance} The transmuted creature gains greater vitality and stamina, gaining a bonus to Constitution. Hit points gained by a temporary increase in Constitution score are not temporary hit points. They go away when the subject's Constitution drops back to normal. They are not lost first as temporary hit points are.
  \par \subspell{Bull's Strength} The transmuted creature becomes stronger, gaining a bonus to Strength.
  \par \subspell{Cat's Grace} The transmuted creature becomes more graceful, agile, and coordinated, gaining a bonus to Dexterity.
\end{spelleffect}

\spellsection{Touch of Idiocy}
\spellschool{Enchantment (Inhibition) [Mind-Affecting]}
\spelllvl{Sor/Wiz 2}
\spellrng{\rngclose}
\spelltgt{One creature}
\spelldur{\durshort}
\spellsave{Will half}
\spellsr{Yes (Will)}
\begin{spelleffect}
  With a touch, you reduce the target's mental faculties. Your successful melee touch attack applies a \minus4 penalty to the target's Intelligence, Wisdom, and Charisma scores. This penalty can't reduce any of these scores below \minus9.
\end{spelleffect}
\begin{spellnotes}
  This spell's effect may make it impossible for the target to cast some or all of its spells, if the requisite attribute drops below the minimum required to cast spells of that level.
\end{spellnotes}

\spellsection{Tree Shape}
\spellschool{Transmutation (Polymorph)}
\spelllvl{Drd 2}
\spellrng{Personal}
\spelltgt{You}
\spelldur{\durext (D)}
\begin{spelleffect}
  You become able to assume the form of a Large living tree or shrub or a Large dead tree trunk with a small number of limbs. The closest inspection cannot reveal that the tree in question is actually a magically concealed creature. To all normal tests you are, in fact, a tree or shrub, although a Spellcraft check can reveal a faint transmutation on the tree. While in tree form, you can observe all that transpires around you just as if you were in your normal form, and your hit points and save bonuses remain unaffected. You gain a \plus10 bonus to natural armor, but you have an effective Dexterity score of 0 and a speed of 0 feet. You are immune to critical hits while in tree form. All clothing and gear carried or worn changes with you.
\end{spelleffect}
\begin{spellnotes}
  You can dismiss tree shape as a free action (instead of as a standard action).
\end{spellnotes}

\spellsection{Tremorsense}
\spellschool{Transmutation (Imbuement)}
\spelllvl{Drd 1, Earth 1}
\spellrng{Personal/\arealarge limit}
\spelltgt{You}
\spelldur{Concentration}
\begin{spelleffect}
  You gain the tremorsense ability. If you are touching a surface, you can automatically pinpoint the location of anything within the area of the spell that is in contact with the surface, including inanimate objects.
\end{spelleffect}
\begin{spellnotes}
  Tremorsense functions on surfaces of any kind, regardless of lighting conditions.
\end{spellnotes}

\spellsection{Unliving Eyes}
\spellschool{Divination/Necromancy (Awareness, Life)}
\spelllvl{Necro 2}
\spellrng{Touch}
\spelltgt{One creature}
\spelldur{\durshort (D)}
\spellsave{Will negates (harmless)}
\spellsr{Yes (Will)}
\begin{spelleffect}
  The subject gains the ability to ``see'' any living creatures and their equipment within 60 feet perfectly, regardless of lighting conditions, physical barriers, invisibility, or any other means of concealment.
\end{spelleffect}

\spellsection{Veil}
\spellschool{Illusion (Glamer) [Unreal]}
\spelllvl{Sor/Wiz 2}
\spellrng{\rngclose}
\spelltgt{One humanoid creature}
\spelldur{\durshort}
\spellsave{Will negates}
\spellsr{Yes (Will)}
\begin{spelleffect}
  The subject's arms and torso are masked in illusion, causing onlookers to perceive whatever movements you project instead of the creature's true actions. For example, the subject might draw a dagger and attack another creature, but anyone watching would only see the subject folding its arms, even as the dagger strikes true.
\end{spelleffect}
\begin{spellnotes}
  A creature that interacts with the effect gets a Will save to recognize it as an illusion. In order to interact with the illusion with a Perception check, the creature must make a Perception check that beats your saving throw DC with this spell. Anyone witnessing the subject perform an impossible action, such as attacking or climbing without the use of its hands, receives a Will save with a \plus10 bonus.
\end{spellnotes}

\spellsection{Ventriloquism}
\spellschool{Illusion (Figment)}
\spelllvl{Sor/Wiz 1, Trickery 1}
\spellcmp{V, F}
\spellrng{\rngclose}
\spelleff{Intelligible sound, usually speech}
\spelldur{\durshort (D)}
\spellsave{Will disbelief (if interacted with)}
\spellsr{No}
\begin{spelleffect}
  You can make your voice (or any sound that you can normally make vocally) seem to issue from someplace else. You can speak in any language you know. With respect to such voices and sounds, anyone who hears the sound and rolls a successful save recognizes it as illusory (but still hears it).
\end{spelleffect}

\begin{comment}
\spellsection{Vestments of the Mage}
\spelldesc{You imbue a set of armor with magical power, preventing it from interfering with your spellcasting.}
\spellschool{Transmutation (Imbuement)}
\spelllvl{Sor/Wiz 2}
\spellrng{Touch}
\spelltgt{Touched nonmagical armor or shield}
\spelldur{\durext (D)}
\spellsave{Will negates (harmless, object)}
\spellsr{Yes (Will)}
\begin{spelleffect}
  The armor or shield's chance of arcane spell failure decreases by 10\% as long as you are wearing or using it. If any other creature wears the armor, it receives no benefit from this spell.
\end{spelleffect}
\begin{spellnotes}
  This is considered an enhancement bonus.
\end{spellnotes}
\end{comment}

\spellsection{Warp Wood}
\spellschool{Transmutation (Alteration)}
\spelllvl{Destruction 2, Drd 2}
\spellrng{\rngclose}
\spellarea{\areamed radius limit}
\spelltgt{1 Small nonmagical wooden object/level within the area}
\spelldur{Instantaneous}
\spellsave{Will negates (object)}
\spellsr{Yes (Will)}
\begin{spelleffect}
  You cause wood to bend and warp, permanently destroying its straightness, form, and strength. A warped door springs open (or becomes stuck, requiring a Strength check to open, at your option). A boat or ship springs a leak. Warped ranged weapons are useless. A warped melee weapon imposes a \minus4 penalty on attack rolls.
  \par You may warp one Small or smaller object or its equivalent per caster level. A Medium object counts as two Small objects, a Large object as four, a Huge object as eight, a Gargantuan object as sixteen, and a Colossal object as thirty-two.
  \par Alternatively, you can unwarp wood (effectively warping it back to normal) with this spell, straightening wood that has been warped by this spell or by other means. \spellindirect{make whole}{Make whole}, on the other hand, does no good in repairing a warped item.
\end{spelleffect}
\begin{spellnotes}
  You can combine multiple consecutive \spell{warp wood} spells to warp (or unwarp) an object that is too large for you to warp with a single spell. Until the object is completely warped, it suffers no ill effects.
\end{spellnotes}

\spellsection{Windstrike}
\spelldesc{You command the air to bludgeon the target, sending it flying.}
\spellschool{Evocation (Control) [Air]}
\spelllvl{Air 2, Drd 2}
\spellrng{\rngmed}
\spelltgt{One creature or object}
\spelldur{Instantaneous}
\spellsave{Fortitude half}
\spellsr{Yes (Fortitude)}
\spelldmg{4d6 bludgeoning damage \add d6 per two levels after 4th}
\begin{spelleffect}
  The target takes damage from the powerful winds. A successful Fortitude save halves the damage. In addition, you may make a bull rush attack with a bonus equal to your caster level \add your casting attribute. If you succeed, you may have the wind bull rush the target in any direction -- even vertically. Moving the target up takes twice as much movement as moving the target horizontally.
\end{spelleffect}

\spellsection{Zone of Truth}
\spellschool{Enchantment (Inhibition) [Mind-Affecting]}
\spelllvl{Clr 2, Law 2, Pal 2}
\spellrng{\rngmed}
\spellarea{\areamed radius emanation}
\spelldur{\durmed}
\spellsave{Will negates}
\spellsr{Yes (Will)}
\begin{spelleffect}
  Creatures within the emanation area (or those who enter it) can't speak any deliberate and intentional lies. Each potentially affected creature is allowed a save to avoid the effects when the spell is cast or when the creature first enters the emanation area. Affected creatures are aware of this enchantment. Therefore, they may avoid answering questions to which they would normally respond with a lie, or they may be evasive as long as they remain within the boundaries of the truth. Creatures who leave the area are free to speak as they choose.
\end{spelleffect}
