\section{Spell Descriptions}

\small

\pdfbookmark[2]{A}{SpellDescriptionsA}
\begin{comment}
\subsubsection{A}
\end{comment}

\spellsection{Ablate Impact}{2}
\spelldesc{You instantly reduce the force of an incoming blow.}
\spellinfo{Abjur (Shielding)}{Abjur}
\spelltwocol{\spelltime{Immediate action}}{\spellcmp{Verbal only}}
\spelldur{1 round}
\spellsr{No}
\begin{spelltarget}{You}
    \spelleffect You gain physical damage reduction, reducing the physical damage you take each round by 8 \add 1 per caster level above 4th. If you take force damage, you cannot use your damage reduction for 1 round.
\end{spelltarget}
\spellnotes After casting this spell, you cannot cast it again for 5 rounds. You can cast this spell in response to an opponent attacking you, before the attack is rolled.

\spellsection{Ablative Shield}{1}
\spelldesc{You instantly encase yourself a shimmering field of magical energy, protecting you from hostile magic.}
\spellinfo{Abjur (Negation) [Magic]}{Abjur, Magic}
\spelltwocol{\spelltime{Immediate action}}{\spellcmp{Verbal only}}
\spelldur{1 round}
\spellsr{No}
\begin{spelltarget}{You}
    \spelleffect You gain spell damage reduction, reducing the damage you take each round from spells by 4 \add 1 per caster level above 2nd. If you take force damage, you cannot use your damage reduction for 1 round.
\end{spelltarget}
\spellnotes After casting this spell, you cannot cast it again for 5 rounds. You can cast this spell in response to an opponent attacking you, before the attack is rolled.

Spells that are not subject to spell resistance are not affected by \spell{ablative shield}.

\spellsection{Acid Arrow}{2}
\spelldesc{You fire a magical arrow of acid from your hand that speeds to its target.}
\spellinfo{Conj (Creation) [Acid]}{Arcane}
\spellrng{\rngmed}
\spelldur{1 round per two caster levels}
\spellsr{No}
\begin{spelltarget}{One creature or object}[Magic vs. Reflex]
    \spellsuccess 2d8 acid damage immediately, and d8 acid damage at the end of each round after the first.
    \spellfailure As above, but no lingering damage.
\end{spelltarget}
\spellnotes If the target becomes submerged in water or takes at least ten points of cold or fire damage, this spell's effect ends.

\spellsectioncomma{Acid Arrow}{Greater}{5}
\spelldesc{You fire a magical arrow of acid from your hand that speeds to its target.}
\spellinfo{Conj (Creation) [Acid]}{Arcane}
\spellrng{\rngfar}
\spelldur{1 round per two caster levels}
\spellsr{No}
\begin{spelltarget}{One creature or object}l[Magic vs. Reflex and Fortitude]
    \spellsuccess[Reflex] 5d8 acid damage immediately, and 2d8 acid damage at the end of each round after the first.
    \spellfailure[Reflex] As above, but no lingering damage.
    \spellsuccess[Fortitude] The target is \sickened.
\end{spelltarget}
\spellnotes As \spell{acid arrow}, except that twenty points of cold or fire damage are required to end the effect.

\spellsection{Acid Fog}{8}
\spelldesc{A billowing mass of acidic vapors fills the area, slowing creatures down and obscuring sight.}
\spellinfo{Conj (Creation) [Acid, Fog]}{Arcane, Destruction}
\spelltwocol{\spellzone{\areamed radius cylinder}}{\spellrng{\rngmed}}
\spelldur{\durshort}
\spellsr{No}
\spellline
\spelleffect Fog fills the area, as \spell{solid fog}, except that the fog is acidic.
\begin{spelltrigger}{End of every round}
    \begin{spelltarget}*{Everything in the area}[Magic vs. Fortitude]
        \spellsuccess 4d6 acid damage.
        \spellfailure As above, but half damage.
    \end{spelltarget}
\end{spelltrigger}
\spellnotes As \spell{solid fog}.

\spellsection{Agony}{5}
\spelldesc{You inflict debilitating pain on your foe, crippling its ability to act.}
\spellinfo{Necro (Flesh)}{Arcane}
\spellrng{\rngmed}
\spelldur{\durshort}
\spellsr{Yes}
\begin{spelltarget}{One creature}
    \spelleffect The target takes a \minus4 penalty to attacks, defenses, and checks.
\end{spelltarget}

\spellsection{Aid}{2}
\spelldesc{You fill your ally with confidence, improving its resilience in combat.}
\spellinfo{Ench (Emotion) [Mind-Affecting, Morale]}{Divine}
\spellrng{\rngclose}
\spelldur{\durshort}
\spellsr{Yes}
\begin{spelltarget}{One creature}
    \spelleffect The target gains 10 temporary hit points \add 1 per caster level above 4th, and a \plus2 enhancement bonus to Will defense. \spellbonusscalingdescription
\end{spelltarget}
\spellnotes If the target takes life damage, it loses all temporary hit points provided by this spell before applying the damage.

\spellsection{Air Walk}{4}
\spelldesc{You imbue an ally with the ability to walk on nothing but air.}
\spellinfo{Trans (Imbuement) [Air]}{Air, Divine, Nature, Travel}
\spellrng{\rngtouch}
\spelldur{\durshort}
\spellsr{Yes}
\begin{spelltarget}{One creature (Gargantuan size or smaller)}
    \spelleffect The target can walk on air as if it were solid ground. The magic only affects the target's legs, and does not grant the ability to climb vertically through the air.
    \par Should the spell end while the target is still aloft, the magic fails slowly. The target floats downward 60 feet per round for 1d6 rounds. If it reaches the ground in that amount of time, it lands safely. If not, it falls the rest of the distance, taking 1d6 damage per 10 feet of fall.
\end{spelltarget}

\spellsection{Align Weapon}{2}
\spelldesc{You enhance a weapon while bringing it closer to your ideals.}
\spellschool{Evoc/Trans (Augment, Channeling) [see text]}
\spelllists{Chaos, Evil, Good, Law}
\spellrng{\rngclose}
\spellsr{No}
\begin{spelltarget}{One weapon or fifty projectiles (in a single group)}
    \spelleffect The weapon is enhanced, as \spell{magic weapon}.
    \spellsuccess The item becomes good, evil, lawful, or chaotic, as you choose, allowing it to overcome damage reduction of the appropriate type. This overrides any existing alignments.
\end{spelltarget}
\spellnotes As \spell{magic weapon}.

When you make a weapon good, evil, lawful, or chaotic, \spell{align weapon} is a good, evil, lawful, or chaotic spell, respectively.

\spellsection{Alter Weapon}{1}
\spelldesc{You transform a weapon into a slightly different form.}
\spellinfo{Trans (Alteration)}{Arcane}
\spellrng{Touch}
\spelldur{\durmed}
\spellsr{Yes}
\begin{spelltarget}{One weapon}[Magic vs. Will]
    \spellsuccess The weapon transforms into a different weapon from the same weapon group. In addition, you can decrease (but not increase) its size by one size category.
\end{spelltarget}
\spellnotes This spell has no effect on natural attacks or unarmed strikes.

\spellsectioncomma{Alter Weapon}{Greater}{4}
\spelldesc{You transform a weapon into a completely different shape.}
\spellinfo{Trans (Alteration)}{Arcane}
\spellrng{Touch}
\spelldur{\durmed}
\spellsr{Yes}
\begin{spelltarget}{One weapon}[Magic vs. Will]
    \spellsuccess The weapon transforms into any other manufactured weapon (but not an improvised weapon). In addition, you can increase or decrease its size by one size category.
    \spelleffect As a standard action that requires concentration, you can touch the weapon to change its shape again.
\end{spelltarget}
\spellnotes As \spell{alter weapon}.

\spellsection{Animal Growth}{7}
\spelldesc{You cause a number of animals grow to twice their normal size and eight times their normal weight.}
\spellschool{Trans (Polymorph) [Size-Affecting]}
\spelllists{Nature, Wild}
\spelltime{Full-round action}
\spelltwocol{\spelllimit{\areamed radius}}{\spellrng{\rngmed}}
\spellsr{No}
\begin{spelltarget}{Five animals in the area}[Magic vs. Fortitude]
    \spellsuccess The target grows larger, as \spell{enlarge person}, except that it affects animals.
\end{spelltarget}

\begin{comment}
\spellsection{Animate Objects}{5}
\spelldesc{You imbue inanimate objects with mobility and a semblance of life.}
\spellinfo{Trans (Animation)}{Chaos, Trans}
\spelltwocol{\spelllimit{\areamed radius}}{\spellrng{\rngmed}}
\spelltgts{One Small object/level in the area; see text}
\spelldur{\durshort}
\spelleffect Each animated object immediately attacks whomever or whatever you initially designate. Your control of the objects is limited to simple commands (``Attack,'' ``Defend,'' ``Stop,'' and so forth).
\par An animated object can be of any nonmagical material. You may animate one Small or smaller object or an equivalent number of larger objects per caster level. A Medium object counts as two Small or smaller objects, a Large object as four, a Huge object as eight, a Gargantuan object as sixteen, and a Colossal object as thirty-two. You can give the objects new commands as a move action, as normal for directing an active spell.
\spellnotes This spell cannot animate objects carried or worn by a creature. This spell can be made permanent with a \spell{permanency} ritual.

\spellsection{Animate Plants}{5}
\spelldesc{You imbue inanimate plants with mobility and a semblance of life.}
\spellinfo{Trans (Animation)}{Nature, Plant}
\spellsr{No}
\begin{spelltargets}{Up to one Small plant/level in the area; see text}
    \spelleffect This spell functions like \spell{animate objects}, except that you animate plants instead of inanimate objects.
\end{spelltargets}
\spellnotes \spell{Animate plants} cannot affect plant creatures, nor does it affect nonliving vegetable material.
\end{comment}

\spellsection{Antilife Shell}{7}
\spelldesc{You create an immobile, spherical energy field that hedges out living creatures.}
\spellinfo{Abjur (Interdiction) [Barrier]}{Divine, Nature, Wild}
\spellzone{\areasmall radius centered on you} 
\spelldur{\durlong \dismissable}
\spellline
\spelleffect Living creatures cannot enter the spell's area. Nonliving creatures, such as constructs and undead, suffer no ill effect.
\spellnotes Barrier spells may be used only defensively, not aggressively. Creatures in the area at the time that the spell is cast are unaffected by the spell.
\spellsr{Yes}

\spellsection{Antimagic Field}{7}
\spelldesc{You create a mobile, spherical energy field that suppresses magic.}
\spellinfo{Abjur (Negation) [Magic]}{Abjur, Divine, Magic}
\spellemanation{\areasmall radius centered on you}
\spelldur{\durlong \dismissable}
\spellsr{No}
\spellline
\spelleffect All spells, spell-like abilities, and magic items fail to function in the area. They cannot be activated from within the field, and any existing effects brought into or cast into the area are suppressed. Time spent within an \spell{antimagic field} counts against a suppressed spell's duration.
\par Summoned creatures of any type disappear if they enter an \spell{antimagic field}. They reappear in the same spot once the field goes away. 
\par Creatures within an \spell{antimagic field} cannot dismiss spells. However, you can dismiss your own antimagic field.
\spellnotes This spell has no effect on golems and other constructs that are imbued with magic during their creation process and are thereafter self-supporting (unless they have been summoned, in which case they are treated like any other summoned creatures).

The effects of instantaneous conjurations, such as \spell{create water}, are not affected by this spell because the conjuration itself is no longer in effect, only its result.

\par \spell{Dispel magic} does not remove the field. Two or more \spell{antimagic fields} sharing any of the same space have no effect on each other.
\par Any part of a creature that lies outside the field is unaffected by the field.
\par Artifacts and deities are unaffected by mortal magic such as this. 

\spellsection{Aqueous Blade}{2}
\spelldesc{You transform the active part of your ally's weapon into water, weakening its blows but allowing it penetrate your foe's defenses more easily.}
\spellinfo{Trans (Alteration) [Water]}{Nature, Water}
\spellrng{\rngclose}
\spelldur{\durshort \dismissable}
\spellsr{Yes}
\begin{spelltarget}{One weapon}[Magic vs. Will]
    \spellsuccess Attacks with the affected weapon are made against Reflex defense instead of Armor defense. However, damage with the weapon is halved, including any bonuses to damage.
\end{spelltarget}

\spellsection{Assimilate}{9}
\spelldesc{Your pointing finger turns black as obsidian. You touch a creature and it dissolves into dust as you assimilate its form into your own body.}
\spellinfo{Necro/Trans (Augment, Life)}{Arcane, Evil}
\spellrng{\rngtouch}
\spelldur{Instantaneous and one hour; see text}
\spellsr{Yes}
\begin{spelltarget}{One living creature}[Magic vs. Fortitude]
    \spellsuccess 18d8 life damage \add d6 per four caster levels above 18th. 
    \spellfailure As above, but half damage.
    \spelleffect If the target has no hit points remaining after taking damage from this spell, it is entirely assimilated into your form, leaving behind only a trace of fine dust. An assimilated creature's equipment is unaffected.
    \par If the creature has at least 1 hit point following your use of this power, you gain temporary hit points equal to half the damage you dealt for 1 hour.
    \par If the creature is completely assimilated, you gain a number of temporary hit points equal to the damage you dealt and a \plus4 enhancement bonus to each of your attributes for 1 hour. In addition, you gain the appearance of the creature for 1 hour, granting you a \plus10 enhancement bonus on Disguise checks made to appear as that creature during that time.
\end{spelltarget}
\spellnotes If you take life damage, you lose all temporary hit points provided by this spell before applying the damage.

\spellsection{Aversion}{3}
\spelldesc{You make a creature want to avoid something.}
\spellinfo{Ench (Emotion) [Mind-Affecting]}{Ench}
\spellrng{\rngmed}
\spelldur{\durshort}
\spellsr{Yes}
\begin{spelltarget}{One creature}[Magic vs. Will]
    \spellsuccess The target feels an aversion to a particular person or object. If the object of the implanted aversion is an individual or a physical object, she will prefer not to approach within 30 feet of it. If it is a word, she will try not to utter it; if it is an action, she will not willingly attempt to perform it; and if it is an event, she will not willingly attend it. The target will take reasonable steps to avoid the object of its aversion, but will not put herself in jeopardy by doing so.
    \par If the target is unable to avoid the object of her aversion, she takes a \minus4 penalty to attacks, defenses, and checks for 1 round.
\end{spelltarget}

\pdfbookmark[2]{B}{SpellDescriptionsB}
\begin{comment}
\subsubsection{B}
\end{comment}

\spellsection{Bane}{1}
\spelldesc{You fill your enemies with dismay, impairing their ability to fight.}
\spellinfo{Ench (Emotion) [Mind-Affecting, Morale]}{Divine, Evil, War}
\spellburst{\areamed radius centered on you}
\spelldur{5 rounds}
\spellsr{Yes}
\begin{spelltargets}{All enemies in the area}[Magic vs. Will]
    \spellsuccess \minus2 penalty to physical attacks.
\end{spelltargets}

\spellsection{Banishment}{6}
\spelldesc{You force extraplanar creatures back to their home plane.}
\spellinfo{Abjur/Conj (Interdiction, Translocation) [Planar, Teleportation]}{Arcane, Divine}
\spellrng{\rngmed}
\spelldur{Concentration}
\spellsr{No}
\spellline
\begin{spelltrigger}{End of every round}
    \begin{spelltarget}*{One extraplanar creature}[Magic vs. Will]
        \spellsuccess The target is banished, as \spell{dismissal}.
    \end{spelltarget}
\end{spelltrigger}
\spelleffect An individual creature can only be targeted once per casting of this spell.

\spellsection{Barkskin}{2}
\spelldesc{You toughen a creature's skin, giving it the appearance of tree bark.}
\spellinfo{Trans (Augment)}{Nature, Wild}
\spellrng{Touch}
\spelldur{\durshort}
\spellsr{Yes}
\begin{spelltarget}{One living creature}
    \spelleffect The target gains a \plus2 enhancement bonus to its armor modifier. \spellbonusscalingdescription In addition, the target gains physical damage reduction, reducing the physical damage it takes each round by 2 \add 1 per two caster levels above 4th. If it is hit by an adamatine weapon or takes fire damage, it cannot use its damage reduction for 1 round.
\end{spelltarget}

\spellsection{Bestow Curse}{4}
\spelldesc{You place a curse on your foe, crippling its ability to act.}
\spellrng{\rngclose}
\spellinfo{Necro (Life) [Curse]}{Death, Divine, Evil, Necro}
\spelldur{Permanent}
\spellsr{Yes}
\begin{spelltarget}{One creature}[Magic vs. Will]
    \spellsuccess The target suffers one of the following three effects, chosen by you:
    \begin{itemize}
        \item \minus6 penalty to an attribute.
        \item \minus4 penalty on attacks, defenses, and checks.
        \item Each turn, the target has a 25\% chance to take no action; otherwise, it acts normally.
    \end{itemize}
    \par You may also invent your own curse, but it should be no more powerful than those described above.
\end{spelltarget}
\spellnotes \cursespellnotes

\spellsection{Black Tentacles}{7}
\spelldesc{You conjure a field of rubbery black tentacles, each 5 feet long. These waving members seem to spring forth from the earth, floor, or whatever surface is underfoot -- including water. They grasp and entwine around creatures that enter the area, holding them fast and crushing them with great strength.}
\spellinfo{Conj/Trans (Animation, Creation)}{Arcane}
\spelltwocol{\spellzone{\areasmall radius}}{\spellrng{\rngmed}}
\spelldur{\durshort \dismissable}
\spellsr{Yes}
\spellline
\spelleffect The area is considered difficult terrain.
\begin{spelltrigger}{End of every movement phase}
    \begin{spelltargets}*{All creatures in the area within 5 feet of the ground}l[Caster level \add casting attribute vs. Maneuver defense (grapple)]
        \spellsuccess The target is grappled and takes 1d8\plus4 bludgeoning damage. It remains grappled until it escape the tentacle. The tentacle's Maneuver defense is equal to 10 \add your caster level \add your casting attribute.
    \end{spelltargets}
\end{spelltrigger}
\spellnotes The tentacles are immune to all forms of attack.

\spellsection{Blade Barrier}{6}
\spelldesc{You create an immobile, vertical curtain of whirling blades shaped of pure force.}
\spellinfo{Evoc (Energy) [Force, Wall]}{Divine, War}
\spelltwocol{\spellzone{100 ft. wall, 20 ft. high \shapeable}}{\spellrng{\rngmed}}
\spelldur{\durshort \dismissable}
\spellsr{Yes}
\spellline
\spelleffect This spell creates a wall of blades made of force energy. The wall provides active cover (20\% miss chance) against attacks made through it. Attacks that miss in this way harmlessly strike the wall. The wall is considered difficult terrain.
\begin{spelltrigger}{A creature passes through the wall}
    \begin{spelltarget}*{The creature in the wall}[Magic vs. Reflex]
        \spellsuccess 6d6 force damage \add d6 per four caster levels above 12th.
        \spellfailure As above, but half damage.
    \end{spelltarget}
\end{spelltrigger}

\spellsection{Blasphemy}{7}
\spelldesc{You speak an unholy utterance of great power, afflicting all those nearby who do not share your allegiance to evil.}
\spellinfo{Evoc (Channeling) [Evil]}{Divine, Evil}
\spellcmp{Verbal only}
\spellburst{\arealarge radius centered on you}
\spellsr{Yes}
\begin{spelltargets}{All nonevil creatures in the area}
    \spelleffect If the target's level does not exceed your caster level, it is \sickened for 5 rounds.

    If the target is also \bloodied, it also suffers one or more of the following ill effects, depending on its level.
    \begin{itemize}
        \item Up to caster level \minus5: The target is also \nauseated for 1 round.
        \item Up to caster level \minus10: The target is also \paralyzed for 5 rounds.
        \item Up to caster level \minus15: A living target loses all its hit points and takes critical damage equal to your caster level, causing it to begin dying. A nonliving target is destroyed.
    \end{itemize}
\end{spelltargets}

\spellsection{Bless}{2}
\spelldesc{You fill your allies with confidence, improving their prowess in combat.}
\spellinfo{Ench (Emotion) [Mind-Affecting, Morale]}{Divine, Good, War}
\spellburst{\areamed radius}
\spelldur{5 rounds}
\spellsr{Yes}
\begin{spelltargets}{All allies in the area}
    \spelleffect The target gains a \plus2 enhancement bonus to physical attacks. \spellbonusscalingdescription
\end{spelltargets}

\spellsection{Blindness/Deafness}{2}
\spelldesc{You remove one of your foe's senses.}
\spellinfo{Necro (Flesh)}{Arcane, Divine, Death}
\spellrng{\rngmed}
\spelldur{\durshort \dismissable}
\spellsr{Yes}
\begin{spelltarget}{One creature}[Magic vs. Fortitude]
    \spellsuccess The target is \sickened.

    If the target is \bloodied, it is also \blinded for 1 round or \deafened for the duration of the spell, as you choose.
\end{spelltarget}

\spellsection{Blink}{4}
\spelldesc{You rapidly blink in and out of reality, confounding your foes and protecting you from their attacks.}
\spellinfo{Conj (Translocation) [Planar]}{Arcane}
\spellrng{\rngpers}
\spelldur{\durshort \dismissable}
\begin{spelltarget}{You}
    \spelleffect You ``blink" back and forth between the Material Plane and the Ethereal Plane. This has several effects, as follows.
    \begin{itemize}
        \item All attacks made against you and spells targeted on you have a 50\% chance to fail. This failure chance is reduced to 20\% if the attack can strike ethereal targets or if the attacker can see ethereal targets. If both are true, the attack suffers no chance of failure. Force effects can strike ethereal targets.
        \item You take half damage from area attacks (but full damage from those that extend onto the Ethereal Plane).
        \item You take half damage from falling, since you fall only while you are material.
        \item All of your attacks and spells have a 20\% chance to happen while you are in the Ethereal Plane, which usually means they have no effect.
        \item You can move at only three-quarters speed (because movement on the Ethereal Plane is at half speed, and you spend about half your time there and half your time material.)
        \item You can step through (but not see through) solid objects. For each 5 feet of solid material you walk through, there is a 50\% chance that you become material. If this occurs, you are shunted off to the nearest open space and take 1d6 damage per 5 feet so traveled. 
        \item You can see and interact with ethereal creatures in roughly the same way you interact with material ones.
    \end{itemize}
\end{spelltarget}
\spellnotes If you are not on the Material Plane when you cast this spell, it has no effect.

\spellsection{Blur}{2}
\spelldesc{You distort an ally's outline so it appears blurred, shifting, and wavering.}
\spellinfo{Illus (Glamer)}{Arcane}
\spellrng{\rngclose}
\spelldur{\durshort \dismissable}
\spellsr{Yes}
\begin{spelltarget}{One creature}
    \spelleffect Other creatures take a \minus2 penalty on sight-based checks and physical attacks against the target, such as on Perception and Sense Motive checks.
\end{spelltarget}
\spellnotes A \spell{see invisibility} spell does not counteract the blurring effect, but a \spell{true seeing} spell does.

\spellsection{Burning Hands}{1}
\spelldesc{You expel a cone of searing flame shoots from your fingertips, searing creatures in front of you.}
\spellinfo{Evoc (Energy) [Fire]}{Arcane, Destruction, Nature, Fire}
\spellburst{\areamed cone}
\spellsr{Yes}
\begin{spelltargets}{Everything in the area}[Magic vs. Reflex]
    \spellsuccess 1d6 fire damage \add 1d6 per four caster levels above 2nd.
    \spellfailure As above, but half damage.
\end{spelltargets}

\pdfbookmark[2]{C}{SpellDescriptionsC}
\begin{comment}
\subsubsection{C}
\end{comment}

\spellsection{Call Lightning}{3}
\spelldesc{You repeatedly call bolts of lightning that flash down from thin air to smite your foes.}
\spellinfo{Evoc (Energy) [Destructive, Electricity]}{Air, Nature}
\spelltwocol{\spellburst{\arealarge vertical line}}{\spellrng{\rngmed}}
\spelldur{\durmed or until discharged \dismissable}
\spellsr{Yes}
\begin{spelltargets}{Everything in the area}[Magic vs. Reflex]
    \spellsuccess 3d8 electricity damage \add d8 per four caster levels above 6th. If you are outdoors and in a stormy area -- a rain shower, clouds and wind, hot and cloudy conditions, or even a tornado -- this deals 3d10 electricity damage \add d10 per four caster levels above 6th instead.
    \spellfailure As above, but half damage.
\end{spelltargets}
\spelleffect Until the spell is discharged, you can call a new bolt of lightning anywhere within range as a standard action that requires concentration. You may call a total number of bolts equal to your caster level before the spell is discharged.
\spellnotes This spell functions indoors or underground, but not underwater. \destructivespellnotes

\spellsectioncomma{Call Lightning}{Greater}{5}
\spelldesc{You repeatedly call intense bolts of lightning that flash down from thin air to smite your foes.}
\spellinfo{Evoc (Energy) [Destructive, Electricity]}{Air, Nature}
\spelltwocol{\spellburst{\arealarge vertical line}}{\spellrng{\rngmed}}
\spelldur{\durmed or until discharged \dismissable}
\spellsr{Yes}
\begin{spelltargets}{Everything in the area}l[Magic vs. Reflex and Fortitude]
    \spellsuccess If you beat the target's Reflex defense, it takes 5d8 electricity damage \add d8 per four caster levels above 10th. If you are outdoors in a stormy area, it takes 5d10 electricity damage \add d10 per four caster levels above 10th instead.

    If the target is \bloodied after the damage is dealt, and you also beat its Fortitude defense, it is \staggered for 5 rounds.
    \spellfailure As above, but half damage and the target is not staggered.
\end{spelltargets}
\spelleffect You may call additional bolts, as \spell{call lightning}.
\spellnotes As \spell{call lightning}.

\spellsection{Calm Emotions}{2}
\spelldesc{You calm a group of creatures, preventing the situation from getting out of hand.}
\spellinfo{Ench (Emotion) [Mind-Affecting]}{Arcane}
\spelltwocol{\spellburst{\areamed radius}}{\spellrng{\rngmed}}
\spelldur{Concentration}
\begin{spelltargets}{All creatures in the area}[Magic vs. Will]
    \spellsuccess The target has its emotions calmed. It cannot take violent actions (although it can defend itself) or do anything destructive.
\end{spelltargets}
\spellnotes Any aggressive action against or damage dealt to a calmed creature immediately breaks the spell on all calmed creatures.

This spell automatically suppresses (but does not dispel) any effects of spells or abilities that affect or require emotions, including all other enchantment (emotion) spells.
\spellsr{Yes}

\spellsection{Chain Lightning}{5}
\spelldesc{You create a stroke of lightning which strikes a single foe before arcing to hit a number of other foes of your choice.}
\spellinfo{Evoc (Energy) [Electricity]}{Arcane, Destruction, Nature}
\spellrng{\rngmed}
\spelllimit{\areamed radius centered on the primary target}
\spellsr{Yes}
\begin{spelltarget}[Primary]{One creature}
    \spellsuccess 5d10 electricity damage \add d10 per four caster levels above 10th. Secondary targets take half damage.
    \spellfailure As above, but half damage.
\end{spelltarget}
\begin{spelltargets}*[Secondary]{Up to five creatures in the area}l[Magic vs. Reflex]
    \spellsuccess The target takes half the damage dealt to the primary target.
\end{spelltargets}

\spellsection{Changestaff}{7}
\spelldesc{You plant your staff in the ground and transform it into a massive tree-like creature which obeys your every command.}
\spellinfo{Trans (Alteration, Animation)}{Nature, Wild}
\spellcmp{Verbal, Somatic, and Material}
\spelltime{Full-round action}
\spelltwocol{\spelltgt{Your touched staff}}{\spellrng{\rngtouch}}
\spelldur{\durmed \dismissable}
\spellline
\spelleffect Your staff turns into a creature that looks and fights just like a treant. The staff-treant defends you and obeys any spoken commands. However, it is by no means a true treant; it cannot converse with actual treants or control trees.

If the staff-treant is reduced to 0 or fewer hit points, it crumbles to powder and the staff is destroyed. Otherwise, the staff returns to its normal form when the spell duration expires (or when the spell is dismissed), and it can be used as the material component for another casting of the spell. The staff-treant is always at full strength when created, despite any wounds it may have incurred the last time it appeared.
\spellline
\spellmat{The quarterstaff, which must be specially prepared. The staff must be a sound limb cut from an ash, oak, or yew, then cured, shaped, carved, and polished (a process requiring twenty-eight days). You cannot adventure or engage in other strenuous activity during the shaping and carving of the staff.}

\spellsection{Chaos Hammer}{4}
\spelldesc{You unleash a multicolored explosion of leaping, ricocheting energy to smite your foes.}
\spellinfo{Evoc (Channeling) [Chaotic]}{Chaos}
\spellrng{\rngmed}
\spelldur{5 rounds}
\spellsr{Yes}
\begin{spelltarget}{One nonchaotic creature}[Magic vs. Will]
    \spellsuccess 8d6 divine damage \add d6 per two caster levels above 8th, and the target is \bewildered for 5 rounds.
    \spellfailure As above, but half damage.
\end{spelltarget}

\spellsection{Charm Monster}{5}
\spelldesc{You manipulate a creature's mind so it thinks of you as a trusted friend and ally.}
\spellinfo{Ench (Emotion) [Charm, Mind-Affecting]}{Ench}
\spellcmp{Somatic only}
\spellrng{\rngmed}
\spelldur{\durlong}
\spellsr{Yes}
\begin{spelltarget}{One creature}[Magic vs. Will]
    \spellsuccess The target is charmed, as \spell{charm person}, except that the effect is not restricted by creature type.
\end{spelltarget}

\spellsection{Charm Person}{2}
\spelldesc{You manipulate a person's mind so he thinks of you as a trusted friend and ally.}
\spellinfo{Ench (Emotion) [Charm, Mind-Affecting]}{Ench}
\spellcmp{Somatic only}
\spellrng{\rngmed}
\spelldur{\durlong}
\spellsr{Yes}
\begin{spelltarget}{One humanoid creature}[Magic vs. Will]
    \spellsuccess The target regards you as its trusted friend and ally. If it is currently faced with any obvious threat from you or your allies, such as someone drawing a weapon, casting a spell, or aiming a ranged weapon at the creature, you take a \minus5 penalty on the magic attack.
    \par The spell does not enable you to control the target as if it were an automaton, but it perceives your words and actions in the most favorable way. You can try to give the target orders, but you must succeed at a Persuasion check to convince it to do anything it wouldn't ordinarily do. (Retries are not allowed.) Treat the target as a friend (a \plus10 relationship modifier) for the purpose of the Persuasion check. An affected creature never obeys suicidal or obviously harmful orders, but it might be convinced that something very dangerous is worth doing.
\end{spelltarget}
\spellnotes Any act by you or your apparent allies that threatens the \spell{charmed} person breaks the spell.

\norepeatspellnotes

\spellsectioncomma{Charm Person}{Mass}{6}
\spelldesc{You manipulate the minds of many people so they think of you as a trusted friend and ally.}
\spellinfo{Ench (Emotion) [Charm, Mind-Affecting]}{Ench}
\spelltwocol{\spelllimit{\areamed radius}}{\spellrng{\rngmed}}
\begin{spelltargets}{Up to five humanoid creatures in the area}
    \spellsuccess This spell functions like \spell{charm person}, except that it affects multiple humanoid creatures.
\end{spelltargets}

\spellsection{Circle of Death}{6}
\spelldesc{You snuff out the life force of your weakened foes by flooding them with negative energy.}
\spellinfo{Necro (Vitalism) [Death, Negative]}{Death, Divine}
\spelltwocol{\spelllimit{\areamed radius}}{\spellrng{\rngmed}}
\spellsr{Yes}
\spellline
\spelleffect This spell affects all \bloodied living creatures in the area, starting with the creature with the lowest level, until it affects a total number of levels equal to twice your caster level. Among creatures with equal levels, those closest to the burst's point of origin are affected first. No creature whose level is greater than half your caster level can be affected, and levels that are not sufficient to affect a creature are wasted. Healthy creatures are not affected by this spell, and do not count against its level limit.
\begin{spelltargets}*{Several bloodied living creatures in the area}l[Magic vs. Fortitude]
    \spellsuccess If the target is bloodied, it is reduced to 0 hit points and takes critical damage equal to your caster level, causing it to begin dying.
\end{spelltargets}
\spellline
\spellmat{The powder of a crushed black pearl with a minimum value of 100 gp.}

\spellsection{Clenched Fist}{9}
\spelldesc{You create a floating, disembodied hand made of magical force that strikes your foe.}
\spellinfo{Evoc (Control) [Force]}{Evoc, Strength}
\spellrng{\rngmed}
\spelldur{\durshort \dismissable}
\spellsr{Yes}
\spellline
\spelleffect This spell creates a hand, as \spell{interposing hand}, except that the hand attacks its target instead of protecting you from it.
\begin{spelltarget}*{One creature}l[Caster level \add casting attribute vs. Armor defense]
    \spellsuccess 2d10 force damage \add half casting attribute.

    If the target is \bloodied after the damage is dealt, you make an additional attack.
    \begin{spelltarget}*{Struck bloodied creature}[Magic vs. Fortitude]
        \spellsuccess The target is \dazed for 1 round.
    \end{spelltarget}
\end{spelltarget}
\spellnotes As \spell{interposing hand}. The hand attacks during the action phase, regardless of when you direct it to attack a target.

\spellsection{Cloak of Chaos}{8}
\spelldesc{You shield your allies with an an powerful aura that resembles a random pattern of color -- an affront to your lawful foes.}
\spellinfo{Abjur (Shielding) [Chaotic]}{Chaos, Divine}
\spelllimit{\areamed radius centered on you}
\spelldur{\durshort \dismissable}
\spellsr{Yes}
\begin{spelltargets}{Up to five creatures in the area}
    The target gains a \plus5 enhancement bonus to its defenses. In addition, it gains spell resistance against lawful spells and spells cast by lawful creatures.
    \par At the end of each round, all lawful creatures within \rngclose range of the target that attacked it with their body or a melee weapon that round take 4d6 points of damage. A creature that attacks multiple creatures shielded by this spell can take this damage multiple times.
\end{spelltargets}

\spellsection{Cloudkill}{7}
\spelldesc{You conjure a yellowish green fog bank that obscures vision and slowly poisons creatures inside.}
\spellinfo{Conj (Creation) [Fog, Poison]}{Arcane}
\spelltwocol{\spellzone{\areamed radius cylinder}}{\spellrng{\rngmed}}
\spelldur{\durshort}
\spellsr{No}
\spellline
\spelleffect Fog in the area, as \spell{fog cloud}, except that the fog is mobile and poisonous.

\par The fog moves away from you at 10 feet per round, rolling along the surface of the ground. Figure out the cloud's new area each round based on its new point of origin, which is 10 feet farther away from the point of origin where you cast the spell.
\begin{spelltrigger}{End of every round}
    \begin{spelltargets}*{Everything in the area}[Magic vs. Fortitude]
        \spellsuccess 1d4 Constitution damage.
    \end{spelltargets}
\end{spelltrigger}
\spellnotes As \spell{fog cloud}.

Holding one's breath doesn't help against the poison, but creatures immune to poison are unaffected. Because the vapors are heavier than air, they sink to the lowest level of the land, even pouring down den or sinkhole openings. This spell cannot penetrate liquids, nor can it be cast underwater.

\spellsection{Color Spray}{1}
\spelldesc{You project a vivid cone of clashing colors from your outstretched hand, striking creatures in front of you.}
\spellinfo{Illus (Figment) [Light, Sight-Dependent]}{Arcane}
\spellburst{\areamed cone}
\spelldur{1d4 rounds}
\spellsr{Yes}
\begin{spelltargets}[All creatures in the area]
    \spellsuccess The target is \dazzled and \bewildered.
\end{spelltargets}
\spellnotes Creatures who cannot see the light are not affected by this spell. Merely closing one's eyes is insufficient protection, however.

\spellsection{Combat Transformation}{7}
\spelldesc{You become a virtual fighting machine -- stronger, tougher, faster, and more skilled in combat. Your mind-set changes so that you relish combat instead of casting spells.}
\spellinfo{Trans (Augment)}{Arcane}
\spellcmp{Verbal, Somatic, and Material}
\spellrng{\rngpers}
\spelldur{\durshort \dismissable}
\begin{spelltarget}{You}
    \spelleffect You gain a \plus3 enhancement bonus to Strength, Dexterity, Constitution, and Fortitude defense. This bonus increases to \plus4 at 14th caster level, and to \plus5 at 20th caster level. In addition, you gain proficiency with any weapons you hold (except exotic weapons).
\end{spelltarget}
\spellnotes If you cast a spell or use a spell activation or spell completion magic item, the spell immediately ends.
\spellmat{A potion of \spell{totemic power} (which costs 40 gp), which you drink (and whose effects are subsumed by the spell effects).}

\spellsection{Command}{1}
\spelldesc{You compel a foe to obey a single command of your choice.}
\spellinfo{Ench (Compulsion) [Language-Dependent, Mind-Affecting, Sound-Dependent]}{Arcane, Divine, Law}
\spellcmp{Verbal only}
\spellrng{\rngmed}
\spelldur{1 round}
\spellsr{Yes}
\begin{spelltarget}{One creature}[Magic vs. Will]
    \spellsuccess The target is \bewildered.

    If the target is \bloodied, it must also perform one of the following actions, as you choose.
    \par \subspell{Approach}: On its turn, the target moves toward you as quickly and directly as possible. The creature may do nothing but move during its turn, and it provokes attacks of opportunity for this movement as normal.
    \par \subspell{Drop}: As soon as possible, the target drops whatever it is holding. It may act normally on its turn, except that it can't pick up any dropped items.
    \par \subspell{Fall}: As soon as possible, the target falls to the ground. It may act normally on its turn, except that it can't get up from its prone position.
    \par \subspell{Flee}: On its turn, the target moves away from you as quickly as possible. It may do nothing but move during its turn, and it provokes attacks of opportunity for this movement as normal.
    \par \subspell{Halt}: On its turn, the target can take no actions, but it can defend itself normally.
    \par \subspell{Laugh}: On its turn, the target takes a standard action to do nothing but laugh uproariously, provoking attacks of opportunity. After that, it can act normally.
\end{spelltarget}
\spellnotes If the target can't understand or carry out your command, the spell automatically fails.

\spellsectioncomma{Command}{Mass}{5}
\spelldesc{You compel many foes to obey your command.}
\spellinfo{Ench (Compulsion) [Language-Dependent, Mind-Affecting, Sound-Dependent]}{Divine, Law}
\spellcmp{Verbal only}
\spellrng{\rngmed}
\spelllimit{\areamed radius}
\spelldur{1 round}
\spellsr{Yes}
\begin{spelltarget}{Five creatures in the area}
    \spellsuccess The target obeys a command, as \spell{command}.
\end{spelltarget}

\spellsection{Cone of Cold}{5}
\spelldesc{You create an area of extreme cold that drains heat from creatures in the area, diminishing their ability to move and fight.}
\spellinfo{Evoc (Energy) [Cold, Destructive]}{Arcane, Nature}
\spellburst{\areamed cone}
\spelldur{\durshort}
\spellsr{Yes}
\begin{spelltarget}{Everything in the area}[Magic vs. Reflex]
    \spellsuccess 5d6 cold damage \add d6 per four caster levels above 10th. In addition, the target is \fatigued.
    \spellfailure As above, but half damage and the target is not fatigued.
\end{spelltarget}
\spellnotes \destructivespellnotes

\spellsectioncomma{Cone of Cold}{Greater}{8}
\spelldesc{You create a massive area of extreme cold that drains heat from creatures in the area, diminishing their ability to move and fight.}
\spellinfo{Evoc (Energy) [Cold, Destructive]}{Arcane, Nature}
\spellburst{\arealarge cone}
\begin{spelltarget}{Everything in the area}[Magic vs. Reflex]
    \spellsuccess 8d6 cold damage \add d6 per four caster levels above 16th. In addition, the target is \fatigued for 5 rounds.
    \spellfailure As above, but half damage and the target is not fatigued.
\end{spelltarget}
\spellnotes As \spell{cone of cold}.

\spellsection{Confusion}{3}
\spelldesc{You compel a creature to act randomly, sowing confusion in your foes' ranks.}
\spellinfo{Ench (Compulsion) [Mind-Affecting]}{Arcane, Chaos, Trickery}
\spellrng{\rngmed}
\spelldur{\durshort}
\spellsr{Yes}
\begin{spelltarget}{One creature}[Magic vs. Will]
    \spellsuccess The target is \bewildered.

    As long as the target is \bloodied, it is instead \confused.
\end{spelltarget}

\spellsectioncomma{Confusion}{Mass}{7}
\spelldesc{You compel a group of creatures to act randomly, sowing confusion in your foes' ranks.}
\spellinfo{Ench (Compulsion) [Mind-Affecting]}{Arcane, Trickery}
\spellrng{\rngmed}
\spelllimit{\areamed radius}
\begin{spelltarget}{Five creatures in the area}[Magic vs. Will]
    \spellsuccess As \spell{confusion}.
\end{spelltarget}

\spellsection{Control Water}{2}
\spelldesc{You manipulate elemental forces to control water around you.}
\spellinfo{Evoc (Control) [Water]}{Nature, Water}
\spelltwocol{\spellzone{One 5 ft. cube/caster level}}{\spellrng{\rngfar}}
\spelldur{\durmed \dismissable}
\spellline
\spelleffect Depending on the version you choose, this spell raises or lowers water.
\par \subspell{Lower Water} This causes water or similar liquid to reduce its depth by as much as 2 feet per caster level (to a minimum depth of 1 inch). The water is lowered within a squarish depression whose sides are up to caster level \mtimes 10 feet long. In extremely large and deep bodies of water, such as a deep ocean, the spell creates a whirlpool that sweeps ships and similar craft downward, putting them at risk and rendering them unable to leave by normal movement for the duration of the spell.
\par \subspell{Raise Water} This causes water or similar liquid to rise in height, just as the lower water version causes it to lower. Boats raised in this way slide down the sides of the hump that the spell creates. If the area affected by the spell includes riverbanks, a beach, or other land nearby, the water can spill over onto dry land.
\spellnotes With either version, you may reduce one horizontal dimension by half and double the other horizontal dimension.

\spellsection{Create Sound}{1}
\spelldesc{You create false sounds from nowhere.}
\spellinfo{Illus (Figment) [Unreal]}{Illus}
\spelltwocol{\spellzone{\areamed radius}}{\spellrng{\rngmed}}
\spelldur{\durshort \dismissable}
\spellsr{No}
\spellline
\spelleffect This spell creates a volume of sound within the area, as determined by you. As a standard action, you can concentrate to alter the sound within the area.
\par The volume of sound created depends on your caster level. You can produce as much noise as two normal humans per caster level. Thus, talking, singing, shouting, walking, marching, or running sounds can be created. The noise can be virtually any type of sound within the volume limit, including speech. A horde of rats running and squeaking is about the same volume as eight humans running and shouting. A roaring lion is equal to the noise from sixteen humans, while a roaring dire tiger is equal to the noise from twenty humans.
\spellnotes Creatures can identify the illusion, as \spell{silent image}. This spell can be made permanent with a \spell{permanency} ritual.

\spellsection{Creeping Doom}{7}
\spelldesc{You summon uncountable hordes of centipedes to overwhelm your foes.}
\spellinfo{Conj (Summoning)}{Nature}
\spelltime{Full-round action}
\spellrng{\rngmed}
\spelldur{\durmed}
\spellsr{No}
\spellline
\spelleffect This spell creates one centipede swarm per two caster levels. They must all be adjacent at least one other swarm. You may summon the centipede swarms so that they share the area of other creatures. The swarms remain stationary, attacking any creatures in their area, unless you command the creeping doom to move (a standard action). As a standard action, you can command any number of the swarms to move toward any prey within range of you. Any swarm out of range of you remains stationary, attacking any creatures in its area.

\spellsection{Cripple}{6}
\spelldesc{You render your foe's limbs useless.}
\spellinfo{Necro (Flesh)}{Arcane}
\spellrng{\rngmed}
\spelldur{\durshort}
\spellsr{Yes}
\begin{spelltarget}{One creature}[Magic vs. Fortitude]
    \spellsuccess The target is \staggered.

    As long as the target is \bloodied, it cannot move its limbs, including any wings. Generally, that means it is paralyzed, except that it can move its head and mouth.
\end{spelltarget}

\spellsection{Crushing Despair}{3}
\spelldesc{You fill a number of creatures with sadness and gloom.}
\spellinfo{Ench (Emotion) [Mind-Affecting, Morale]}{Arcane}
\spellburst{\areamed cone}
\spelldur{\durmed}
\spellsr{Yes}
\begin{spelltargets}{All creatures in the area}
    The target is \vulnerable.
\end{spelltargets}

\spellsection{Crushing Hand}{8}
\spelldesc{You create a floating, disembodied hand made of magical force that crushes your foe in its grasp.}
\spellinfo{Evoc (Control) [Force]}{Evoc}
\spellrng{\rngmed}
\spelldur{\durshort \dismissable}
\spellsr{Yes}
\spellline
\spelleffect This spell creates a hand, as \spell{interposing hand}, except that the hand grapples its target instead of protecting you from it.
\begin{spelltarget}*{One creature}l[Caster level \add casting attribute vs. Maneuver defense]
    \spellsuccess The target is grappled. It takes 2d6 bludgeoning damage \add half your casting attribute.
\end{spelltarget}
\spellnotes As \spell{interposing hand}. The hand attacks during the action phase, regardless of when you direct it to attack a target.

\spellsection{Cure Critical Wounds}{4}
\spelldesc{You lay your hand on a creature and channel positive energy into it, healing even the most grievous injuries.}
\spellinfo{Necro (Vitalism) [Positive]}{Divine, Life, Nature}
\spellrng{\rngclose}
\spellsr{Yes}
\begin{spelltarget}{One creature}[Magic vs. Fortitude]
    \spelleffect If the target is living, it is healed for 8d6 damage \add d6 per two caster levels above 8th. For every 5 points of healing granted by this spell, it can instead cure 1 point of critical damage.
    \spellsuccess If the target is undead, it takes that much positive damage.
    \spellfailure As above, but half damage.
\end{spelltarget}

\spellsectioncomma{Cure Critical Wounds}{Mass}{8}
\spelldesc{You stretch out your hand and channel positive energy into all of your allies, healing even their most grievous injuries.}
\spellinfo{Necro (Vitalism) [Positive]}{Divine, Life, Nature}
\spellrng{\rngclose}
\spellsr{Yes}
\begin{spelltargets}{Up to five creatures in the area}[Magic vs. Fortitude]
    \spelleffect As \spell{cure critical wounds}, except that it heals 8d6 damage \add d6 per four caster levels above 16th.
    \spellsuccess If the target is undead, it takes that much positive damage.
    \spellfailure As above, but half damage.
\end{spelltargets}

\spellsection{Cure Light Wounds}{1}
\spelldesc{You lay your hand on a creature and channel positive energy into it, healing some of its wounds.}
\spellinfo{Necro (Vitalism) [Positive]}{Divine, Nature}
\spellrng{\rngclose}
\spellsr{Yes}
\begin{spelltarget}{One creature}[Magic vs. Fortitude]
    \spelleffect If the target is living, it is healed for 2d6 damage \add d6 per two caster levels above 2nd.
    \spellsuccess If the target is undead, it takes that much positive damage.
    \spellfailure As above, but half damage.
\end{spelltarget}

\spellsectioncomma{Cure Light Wounds}{Mass}{5}
\spelldesc{You stretch out your hand and channel positive energy into all of your allies, healing some of their wounds.}
\spellinfo{Necro (Vitalism) [Positive]}{Divine, Life, Nature}
\spelltwocol{\spelllimit{\areamed radius}}{\spellrng{\rngmed}}
\spellrng{\rngclose}
\spellsr{Yes}
\begin{spelltargets}{Up to five creatures in the area}[Magic vs. Fortitude]
    \spelleffect As \spell{cure light wounds}, excpet that it heals 5d6 damage \add d6 per four caster levels above 10th.
    \spellsuccess As \spell{cure light wounds}.
    \spellfailure As above, but half damage.
\end{spelltargets}

\spellsection{Cure Moderate Wounds}{2}
\spelldesc{You lay your hand on a creature and channel positive energy into it, healing its wounds.}
\spellinfo{Necro (Life) [Positive]}{Divine, Life, Nature}
\spellrng{\rngclose}
\spellsr{Yes}
\begin{spelltarget}{One creature}[Magic vs. Fortitude]
    \spelleffect If the target is living, it is healed for 4d6 damage \add d6 per two caster levels above 4th. For every 15 points of healing granted by this spell, it can instead cure 1 point of critical damage.
    \spellsuccess If the target is undead, it takes that much positive damage.
    \spellfailure As above, but half damage.
\end{spelltarget}

\spellsectioncomma{Cure Moderate Wounds}{Mass}{6}
\spelldesc{You stretch out your hand and channel positive energy into all of your allies, healing their wounds.}
\spellinfo{Necro (Vitalism) [Positive]}{Divine, Life, Nature}
\begin{spelltargets}{Up to five creatures in the area}[Magic vs. Fortitude]
    \spelleffect As \spell{cure moderate wounds}, except that it heals 6d6 damage \add d6 per four caster levels above 12th.
    \spellsuccess If the target is undead, it takes that much positive damage.
    \spellfailure As above, but half damage.
\end{spelltargets}

\spellsection{Cure Serious Wounds}{3}
\spelldesc{You lay your hand on a creature and channel positive energy into it, healing even serious injuries.}
\spellinfo{Necro (Vitalism) [Positive]}{Divine, Life, Nature}
\spellrng{\rngclose}
\spellsr{Yes}
\begin{spelltarget}{One creature}[Magic vs. Fortitude]
    \spelleffect If the target is living, it is healed for 6d6 damage \add d6 per two caster levels above 6th. For every 10 points of healing granted by this spell, it can instead cure 1 point of critical damage.
    \spellsuccess If the target is undead, it takes that much positive damage.
    \spellfailure As above, but half damage.
\end{spelltarget}

\spellsectioncomma{Cure Serious Wounds}{Mass}{7}
\spelldesc{You stretch out your hand and channel positive energy into all of your allies, healing even serious injuries.}
\spellinfo{Necro (Vitalism) [Positive]}{Divine, Life, Nature}
\begin{spelltargets}{Up to five creatures in the area}[Magic vs. Fortitude]
    \spelleffect As \spell{cure serious wounds}, except that it heals 7d6 damage \add d6 per four caster levels above 14th.
    \spellsuccess If the target is undead, it takes that much positive damage.
    \spellfailure As above, but half damage.
\end{spelltargets}

\pdfbookmark[2]{D}{SpellDescriptionsD}
\begin{comment}
\subsubsection{D}
\end{comment}

\spellsection{Dancing Lights}{1}
\spelldesc{You create floating lights to guide your way.}
\spellinfo{Illus (Figment) [Light]}{Arcane}
\spelltwocol{\spelllimit{\areamed radius}}{\spellrng{\rngmed}}
\spelldur{\durshort \dismissable}
\spellsr{No}
\spellline
\spelleffect This spell creates mobile sources of light. You can create up to four lights which resemble lanterns or torches, up to four glowing spheres of light, or a single glowing, vaguely humanoid shape. Regardless of their form, each light creates bright illumination in a \areamed radius, as a torch.

As a swift action, you can move the lights as you desire through the air. They can move up to 100 feet per round, but they must always stay within range of you, and all the lights must remain within a single \areamed radius. Any light which goes beyond those limits winks out.
\spellnotes This spell can be made permanent with a \spell{permanency} ritual.

\spellsection{Darkvision}{2}
\spelldesc{You grant an ally the ability to see in complete darkness.}
\spellinfo{Div (Awareness)}{Arcane}
\spellrng{\rngtouch}
\spelldur{\durlong}
\spellsr{Yes}
\begin{spelltarget}{One creature}
    \spelleffect The target gains the ability to see 60 feet even in total darkness. Beyond 60 feet, the target can see dimly, treating areas of darkness as shadowy illumination. Darkvision does not function if a creature is in an area of bright light or is dazzled. Darkvision is black and white only, but otherwise like normal sight.
\end{spelltarget}
\spellnotes This spell does not grant the ability to see in magical darkness. This spell can be made permanent with a \spell{permanency} ritual.

\spellsection{Daylight}{2}
\spelldesc{You infuse an object with the power of the sun, causing it to illuminate a large area.}
\spellinfo{Illus (Figment) [Light]}{Divine}
\spellrng{\rngtouch}
\spelldur{\durlong \dismissable}
\spellsr{Yes}
\begin{spelltarget}{Object touched}
    \spelleffect The object touched sheds light as bright as full daylight in a \arealarge radius, and dim light for an additional 50 feet beyond that. Creatures that take penalties in bright light also take them while within the radius of this magical light. Despite its name, this spell is not the equivalent of sunlight for the purposes of creatures that are damaged or destroyed by bright light.
    \par If \spell{daylight} is cast on a small object that is then placed inside or under a light-proof covering, the spell's effects are blocked until the covering is removed.
\end{spelltarget}
\spellnotes \spell{Daylight} brought into an area of magical darkness (or vice versa) is temporarily negated, so that the otherwise prevailing light conditions exist in the overlapping areas of effect.

\spellsection{Death Knell}{2}
\spelldesc{You draw forth the ebbing life force of a creature and use it to fuel your own power.}
\spellinfo{Necro (Life) [Death]}{Death, Evil, Necro}
\spellrng{\rngmed}
\spelldur{\durshort; see text}
\begin{spelltarget}{Living creature}[Magic vs. Fortitude]
    \spellsuccess As long as the target is bloodied, it is \vulnerable. If it drops to 0 hit points, it dies immediately, and you gain 10 temporary hit points \add 1 per caster level above 4th. These temporary hit points last for 1 round per level the target had.
\end{spelltarget}
\spellnotes If you take life damage, you lose all temporary hit points provided by this spell before applying the damage.

\spellsection{Death Ward}{3}
\spelldesc{You shield an ally from deadly spells.}
\spellinfo{Abjur/Necro (Shielding, Vitalism) [Positive]}{Death, Divine, Good, Protection}
\spellrng{\rngclose}
\spelldur{\durshort}
\spellsr{Yes}
\begin{spelltarget}{One living creature}
    \spelleffect The target is immune to all death spells, magical death effects, energy drain, and any negative energy effects.
\end{spelltarget}
\spellnotes This spell doesn't remove negative levels that the target has already gained. It does not protect against other sorts of attacks, even if those attacks might be lethal.

\spellsectioncomma{Death Ward}{Mass}{7}
\spelldesc{You shield your allies from deadly spells.}
\spellinfo{Abjur/Necro (Shielding, Vitalism) [Positive]}{Death, Divine}
\spelltwocol{\spelllimit{\areamed radius}}{\spellrng{\rngmed}}
\spelldur{\durshort}
\spellsr{Yes}
\begin{spelltarget}{Five living creatures in the area}
    \spelleffect The target is protected, as \spell{death ward}.
\end{spelltarget}

\spellsection{Deep Slumber}{7}
\spelldesc{You fill your foe with an overpowering urge to sleep, inevitably rendering him comatose.}
\spellinfo{Ench (Compulsion) [Mind-Affecting]}{Arcane}
\spellrng{\rngmed}
\spelldur{\durlong}
\spellsr{Yes}
\begin{spelltarget}{One creature}[Magic vs. Will]
    \spellsuccess The target is \bewildered.

    If the target becomes \bloodied at any point during the spell's duration, it immediately falls asleep. If left undisturbed, it will sleep until it dies. As long as it remains bloodied, it cannot be awakened until the spell's duration expires, though it can be awakened normally after that point.
\end{spelltarget}

\spellsection{Deflection}{3}
\spelldesc{You shield yourself from enemy attacks, causing them to deflect away from you harmlessly.}
\spellinfo{Abjur (Shielding)}{Abjur}
\spelldur{\durlong}
\begin{spelltarget}{You}
    \spelleffect You gain a \plus2 enhancement bonus to your physical defenses. \spellbonusscalingdescription
\end{spelltarget}
\spellnotes The enhancement bonus from this spell stacks with any enhancement bonuses to defense modifiers, such as enhancement bonuses to your armor. It does not stack with other enhancement bonuses that apply directly to your physical defenses. 

\spellsection{Delay Damage}{5}
\spelldesc{You partially shift yourself into the future, delaying the impact of attacks against you.}
\spellinfo{Abjur/Trans (Shielding, Temporal)}{\spelllists{Arcane}}
\spelldur{\durmed}
\begin{spelltarget}{You}
    \spelleffect Whenever you take damage, half of the damage (rounded down) is not dealt to you immediately. This damage is tracked separately. At the end of the spell's duration, you take all of the delayed damage at once. For every point of damage dealt in this way in excess of your hit points, you take 1 point of critical damage.
\end{spelltarget}

\spellsection{Delay Poison}{1}
\spellinfo{Necro (Flesh)}{Divine, Nature}
\spelltime{1 swift action}
\spellrng{\rngclose}
\spelldur{\durshort}
\spellsr{Yes}
\begin{spelltarget}{One creature}
    \spelleffect The target becomes temporarily immune to the effects of poison. Poisons it is exposed to do not make attacks against it. This effect does not prevent the target from becoming poisoned, and any poisons in the target's system when the spell ends will continue their effects normally. 
\end{spelltarget}
\spellnotes This spell does not cure any damage that poison may have already done.

\spellsection{Delayed Blast Fireball}{6}
\spellinfo{Evoc (Energy) [Destructive, Fire]}{Arcane, Fire}
\spelltwocol{\spellburst{\areamed radius}}{\spellrng{\rngmed}}
\spelldur{5 rounds or less; see text}
\spellsr{Yes}
\spellline
\spelleffect You can delay this spell's attack until up to 5 rounds after the spell is cast. You select the amount of delay upon completing the spell, and that time cannot change once it has been set unless someone touches the bead (see below). For every round that this spell is delayed, your caster level with it increases by 2.

If you choose a delay, a glowing bead sits at the point of origin until it detonates. A creature can pick up and hurl the bead as a thrown weapon (range increment 10 feet). If a creature handles and moves the bead within 1 round of its detonation, there is a 25\% chance that the bead detonates while being handled.
\begin{spelltargets}{Everything in the area}[Magic vs. Reflex]
    \spellsuccess 6d6 fire damage \add d6 per four caster levels above 12th.
    \spellfailure As above, but half damage.
\end{spelltargets}

\spellnotes As \spellnotes{fireball}.

\spellsection{Destruction}{7}
\spellinfo{Necro (Flesh) [Death]}{Destruction, Divine}
\spellrng{\rngclose}
\spellsr{Yes}
\begin{spelltarget}{One creature}[Magic vs. Fortitude]
    \spellsuccess The target is \staggered for 5 rounds.

    If the target is \bloodied, it loses all its hit points and takes critical damage equal to your caster level, causing it to begin dying.
\end{spelltarget}
\spellnotes The remains of a creature killed by this spell are consumed utterly (but not its equipment or possessions). The only way to restore life such a creature is to use \spell{true resurrection}, a carefully worded \spell{wish} spell followed by \spell{resurrection}, or \spell{miracle}.

\spellsection{Detect Alignment}{3}
\spelldesc{You sense the presence of creatures with a particular alignment.}
\spellinfo{Div (Awareness) [Detection]}{Divine}
\spellemanation{\arealarge cone from you}
\spelldur{Concentration}
\spellsr{No}
\spellline
\spelleffect As you cast this spell, you choose an alignment: good, evil, lawful, or chaotic. Anything within the spell's area that has the chosen alignment has a faint aura, visible only to you.

By concentrating on an aura (a standard action), you can determine the strength of the aura. Most aligned creatures and magic items have a faint aura. Creatures that embody the alignment (such as undead and outsiders with an aligned creature subtype) have a moderate aura. Creatures that act directly on behalf of the alignment (such as paladins and some clerics), and exceptionally potent aligned magic items (primarily artifacts) have a strong aura.
\spellnotes Each round, you can turn to detect objects in a new area. A detection spell can penetrate barriers, but 1 foot of stone, 1 inch of common metal, a thin sheet of lead, or 3 feet of wood or dirt blocks it.

\spellsection{Dictum}{7}
\spellinfo{Evoc (Channeling) [Lawful]}{Divine, Law}
\spellcmp{Verbal only}
\spellburst{\arealarge radius centered on you}
\spelldur{Instantaneous/5 rounds}
\spellsr{Yes}
\begin{spelltargets}{All nonlawful creatures in the area}
    \spelleffect If the target's level does not exceed your caster level, it is \sickened for 5 rounds.

    If the target is also \bloodied when the spell is cast, it also suffers one or more of the following ill effects, depending on its level.
    \begin{dtable}
        \begin{tabularx}{\columnwidth}{l >{\lcol}X}
            \par \thead{Level} & \thead{Effect} \\
            \par Equal to caster level & Staggered \\
            \par Up to caster level \minus5 & Stunned, staggered \\
            \par Up to caster level \minus10 & Paralyzed, stunned, staggered \\
            \par Up to caster level \minus15 & Killed\fn{1}
        \end{tabularx}
        1 Living creatures die. Nonliving creatures are destroyed.
    \end{dtable}
    \par \subspell{Staggered} The creature is staggered for 5 rounds. It can take a move action or a standard action each round, but not both.
    \par \subspell{Stunned} The creature is stunned for 1 round.
    \par \subspell{Paralyzed} The creature is paralyzed and helpless for 5 rounds.
    \par \subspell{Killed} Living creatures die. Nonliving creatures are destroyed.
\end{spelltargets}

\spellsection{Dimension Door}{4}
\spellinfo{Conj (Translocation) [Teleportation]}{Arcane, Travel}
\spellrng{\rngext \rngunrestricted}
\begin{spelltarget}{You}
    \spelleffect You teleport to a destination within range. You must clearly visualize the destination. After arriving, you cannot act until the next action phase.

    If the destination is occupied, or dramatically different from how you visualized it, the spell fails.
\end{spelltarget}

\spellsectioncomma{Dimension Door}{Mass}{7}
\spellinfo{Conj (Translocation) [Teleportation]}{Conj, Travel}
\spelltwocol{\spelllimit{\areamed radius centered on you}}{\spellrng{\rngext}}
\begin{spelltarget}{Up to five willing creatures in the area}
    \spelleffect The target teleports to a destination you specify, as \spell{dimension door}.
\end{spelltarget}
\spellnotes You can choose the destinations for each target independently, within the range of the spell. 

\spellsection{Dimension Slide}{3}
\spellinfo{Conj (Translocation) [Teleportation]}{Conj, Travel}
\spellrng{\rngmed}
\spellsr{Yes}
\begin{spelltarget}{One creature}[Magic vs. Will]
    \spelleffect The target teleports to a destination in range. The destination must be an unoccupied space on stable ground. If the destination is invalid, the spell fails.
\end{spelltarget}

\spellsection{Dimensional Anchor}{3}
\spelldesc{You surround your foe in a shimmering emerald field that completely blocks extradimensional travel, preventing it from escaping you.}
\spellinfo{Abjur (Negation)}{Arcane, Divine, Magic}
\spellrng{\rngmed}
\spelldur{\durlong/5 rounds}
\spellsr{Yes}
\begin{spelltarget}{One creature}[Magic vs. Will]
    \spellsuccess  The target cannot travel extradimensionally for 1 hour. Effects barred by a \spell{dimensional anchor} include \spell{astral projection}, \spell{blink}, \spell{dimension door}, \spell{ethereal jaunt}, \spell{gate}, \spell{maze}, \spell{plane shift}, \spell{shadow walk}, \spell{teleport}, and similar spell-like or supernatural abilities.
    \spellfailure As above, but the effect lasts for 5 rounds.
\end{spelltarget}
\spellnotes This spell does not interfere with the movement of creatures already in ethereal or astral form when the spell is cast, nor does it block extradimensional perception or attack forms, such as summoning monsters. Also, it does not prevent summoned creatures from disappearing at the end of a summoning spell.

\spellsection{Discern Lies}{3}
\spelldesc{You can discern subtle magical disturbances caused by lying.}
\spellinfo{Div (Awareness) [Detection]}{Divine, Law}
\spellemanation{\arealarge cone from you}
\spelldur{Concentration}
\spellsr{No}
\spellline
\spelleffect You know when any creature in the area deliberately and knowingly speaks a lie. The spell does not reveal the truth, uncover unintentional inaccuracies, or necessarily reveal evasions.
\spellnotes Each round, you can turn to discern lies in a new area. A detection spell can penetrate barriers, but 1 foot of stone, 1 inch of common metal, a thin sheet of lead, or 3 feet of wood or dirt blocks it.

\spellsection{Discern Vulnerability}{4}
\spellinfo{Div (Knowledge)}{Arcane}
\spelltime{1 swift action}
\spellrng{\rngmed}
\spellsr{No}
\begin{spelltarget}{One creature}
    \spelleffect You instantly recognize all of the target's vulnerabilities. This grants you a \plus2 bonus to attacks and weapon damage against that creature. In addition, you learn any significant weaknesses the creature has. This includes, but is not limited to, the following information:
    \begin{itemize}
        \item Which of the target's defenses is lowest
        \item If the target has any vulnerabilities to specific damage types
        \item How to overcome the target's damage reduction, regeneration, or other similar abilities
    \end{itemize}
\end{spelltarget}
\spellnotes This spell gives no information about a creature's strengths or abilities -- only its weaknesses.

\spellsection{Disintegrate}{6}
\spelldesc{You shoot a thin, green ray from your pointing finger that completely destroys whatever it hits.}
\spellinfo{Trans (Alteration)}{Arcane, Destruction}
\spellrng{\rngclose}
\spellsr{Yes}
\begin{spelltarget}{One creature or attended object}l[Magic vs. Reflex and Fortitude]
    \spellsuccess 12d8 physical damage \add d8 per two caster levels above 12th.
    \spellfailure[Fortitude] As above, but half damage.
    \spellfailure[Reflex] No effect.
    \spelleffect Any creature reduced to 0 hit points by this spell is entirely disintegrated, leaving behind only a trace of fine dust. Its equipment is unaffected.
    \par When used against an object, the ray simply disintegrates as much as one 10-foot cube of nonliving matter. Thus, the spell disintegrates only part of any very large object or structure targeted.
\end{spelltarget}
\spellnotes This spell affects even objects constructed entirely of force, such as \spell{wall of force}, but not magical effects such as an \spell{antimagic field}.

\spellsection{Dismissal}{4}
\spellinfo{Abjur/Conj (Interdiction, Translocation) [Planar, Teleportation]}{Arcane, Divine}
\spellrng{\rngclose}
\spellsr{Yes}
\begin{spelltarget}{One extraplanar creature}[Magic vs. Will]
    \spellsuccess The target teleported back to its proper plane. There is a 20\% chance of actually sending it to a plane other than its own.
\end{spelltarget}

\spellsection{Dispel Magic}{3}
\spellinfo{Abjur (Negation) [Magic]}{Arcane, Divine, Magic, Nature}
\spellrng{\rngmed}
\spellsr{No}
\spellspecial This spell has two versions: a targeted dispel, and an area dispel. Its effects depend on which version is chosen.
\begin{spelltarget}{One creature or object}[Caster level vs. Special]
    \spelleffect For every spell affecting the target, if the attack result beats a DC equal to 10 \add the caster level of the spell, it is dispelled.

    If the target is an object, and the attack result beats a DC equal to 10 \add the caster level of the object, the object is suppressed for 5 rounds. A suppressed object loses all its magical abilities, though it is still treated as being a magical object for the purpose of spells and effects.

    If the target is an effect of an ongoing spell (such as a summoned creature), and the attack result beats a DC equal to 10 \add the caster level of the spell, the target is treated as if the spell that created it was dispelled. This usually causes the target to disappear.
\end{spelltarget}

\spellline
\spellburst{\areamed radius}
\begin{spelltargets}*{All creatures and unattended objects in the area}l[Caster level vs. Special]
    \spelleffect If the attack result beats a DC equal to 10 \add the caster level of the highest level spell on the target, that effect is dispelled. If there are multiple spells of the same level, choose one randomly.
\end{spelltargets}

\spellnotes A dispelled spell ends as if its duration had expired. If a spell affects multiple targets, it must be dispelled individually on each target. Dispelling the effect on one target does not affect the other targets of the spell.

Some spells, as detailed in their descriptions, can't be dispelled by this spell. A spell without a duration cannot be dispelled, even if it has a lasting effect.

You may choose to automatically succeed or fail on your attack against any spell that you cast yourself.

Spell-like abilities are treated like spells, and this spell affects them in the same way it affects spells.

Artifacts and deities are unaffected by mortal magic such as this.

\spellsectioncomma{Dispel Magic}{Greater}{6}
\spellinfo{Abjur (Negation) [Magic]}{Arcane, Divine, Magic, Nature}
\spelltwocol{\spellarea{\areamed radius limit}}{\spellrng{\rngmed}}
\spellsr{No}
\begin{spelltargets}{All creatures and unattended objects in the area}l[Caster level vs. Special]
    \spelleffect Spells affecting the target are dispelled, as a targeted \spell{dispel magic}.
\end{spelltargets}
\spellnotes As \spell{dispel magic}. In addition, this spell has a chance to dispel any effect that \spell{remove curse} can remove, even if \spell{dispel magic} can't dispel that effect.

\spellsection{Displacement}{4}
\spellinfo{Illus (Glamer)}{Arcane}
\spellrng{\rngclose}
\spelldur{\durshort \dismissable}
\spellsr{Yes}
\begin{spelltarget}{One creature}
    \spelleffect The target appears to be about 2 feet away from its true location. Attacks against it have a 50\% miss chance, as if it were invisible. However, unlike invisibility, this spell does not prevent enemies from targeting the creature normally, and it does not allow the creature to hide.
\end{spelltarget}

\spellsection{Disrupting Weapon}{4}
\spelldesc{You imbue a weapon with positive energy, making it deadly to undead.}
\spellinfo{Necro/Trans (Imbuement, Positive)}{Divine}
\spellrng{\rngclose}
\spellfocus{One melee weapon}
\spelldur{\durshort}
\spellsr{Yes}
\begin{spelltrigger}{The focus weapon strikes a \bloodied undead creature for the first time in a round}
    \begin{spelltarget}*{The bloodied undead creature}l[Magic vs. Fortitude]
        \spellsuccess The target creature is utterly destroyed.
    \end{spelltarget}
\end{spelltrigger}

\spellsection{Divine Favor}{1}
\spelldesc{You imbue yourself with skill in combat by calling upon the divine power of your patron.}
\spellinfo{Trans (Augment)}{Divine, Strength, War}
\spelldur{\durshort}
\begin{spelltarget}{You}
    \spelleffect \plus2 enhancement bonus on attack and weapon damage rolls. \spellbonusscalingdescription
\end{spelltarget}

\spellsectioncomma{Divine Favor}{Greater}{4}
\spelldesc{You imbue yourself with great strength and skill in combat by calling upon the divine power of your patron.}
\spellinfo{Trans (Augment)}{Divine, Strength, War}
\spelldur{\durshort}
\begin{spelltarget}{You}
    \spelleffect As \spell{divine favor}, except that you also gain a \plus2 enhancement bonus to Strength.
\end{spelltarget}

\spellsection{Dominate Monster}{8}
\spellinfo{Ench (Compulsion) [Domination, Mind-Affecting]}{Ench}
\spellrng{\rngmed}
\spelldur{One day}
\begin{spelltarget}{One creature}[Magic vs. Will]
    \spellsuccess The target is dominated, as \spell{dominate person}, except that the effect does not depend on creature type.
\end{spelltarget}

\spellsection{Dominate Person}{7}
\spellinfo{Ench (Compulsion) [Domination, Mind-Affecting]}{Ench}
\spellrng{\rngmed}
\spelldur{One day}
\begin{spelltarget}{One humanoid creature}[Magic vs. Will]
    \spellsuccess You can control the actions of the target through a telepathic link that you establish with the target's mind.
    \par If you and the target have a common language, you can generally force the target to perform as you desire, within the limits of its abilities. If no common language exists, you can communicate only basic commands, such as ``Come here," ``Go there," ``Fight," and ``Stand still." If you concentrate on the spell, you know what the target is experiencing, but you do not receive direct sensory input from it, nor can it communicate with you telepathically.
    \par Once you have given a dominated creature a command, it continues to attempt to carry out that command to the exclusion of all other activities except those necessary for day-to-day survival (such as sleeping, eating, and so forth). Because of this limited range of activity, a Sense Motive check against DC 15 (rather than DC 25) can determine that the target's behavior is being influenced by an enchantment effect (see the Sense Motive skill description).
    It takes time for the link to be established. For the first hour after the spell is cast, you must concentrate on the spell (a standard action) to control the target's actions. While you are not concentrating on the spell, the creature acts as if confused, as the \spell{confusion} spell, except that it never attacks you. If the target would randomly attack you, it instead is forced to follow your commands. At the end of the hour, you must make a second Will attack. If you concentrate on the spell during this time, you gain a \plus4 bonus to the attack. If your attack succeeds, you dominate the creature fully for the remainder of the spell duration. Otherwise, the creature is freed.
    \par After the first hour, changing your instructions or giving a dominated creature a new command is the equivalent of redirecting a spell, so it is a move action.
    \par By concentrating fully on the spell (a standard action), you can receive full sensory input as interpreted by the mind of the target, though it still can't communicate with you. You can't actually see through the target's eyes, so it's not as good as being there yourself, but you still get a good idea of what's going on.
    \par The target resists this control, and you must make a new attack to force it to take an action against its nature. Failure means it breaks free. This does not apply when a target is merely ordered to perform an action it disagrees with -- the action must be directly opposed to the target's beliefs. Ordering a paladin to murder an innocent would require a new attack, but ordering him to build a bridge that would allow an evil army to cross a river would not. If your command would obviously lead to the creature's death, your attack takes a \minus10 penalty. Once control is established, the range at which it can be exercised is unlimited, as long as you and the target are on the same plane. You need not see the target to control it.
    \par If you recast this spell on a target you have dominated before it escapes your control, you can extend the duration of the spell indefinitely. You do not need to make a new attack when you renew your control in this fashion.
\end{spelltarget}

\spellsection{Drain Life}{5}
\spellinfo{Necro (Life)}{Necro}
\spellrng{\rngclose}
\spelldur{\durlong}
\spellsr{Yes}
\begin{spelltarget}{One living creature}[Magic vs. Fortitude]
    \spellsuccess 10d6 life damage \add d6 per two caster levels above 8th. You gain temporary hit points equal to half the damage you deal. You can't gain more hit points than the target had. The temporary hit points disappear after 1 hour. If you take life damage, you lose all temporary hit points provided by this spell before applying the damage.

    As long as you have temporary hit points from this spell, you are treated as being undead for the purpose of spells or abilities which affect undead. This can cause some unintelligent undead, such as skeletons and zombies, to avoid attacking you.
    \spellfailure As above, but half damage.
\end{spelltarget}

\pdfbookmark[2]{E}{SpellDescriptionsE}
\begin{comment}
\subsubsection{E}
\end{comment}
\spellsr{Yes}

\spellsection{Earth's Pull}{1}
\spelldesc{You intensify the pull of gravity on your foe, causing it to feel much heavier and making its movements sluggish.}
\spellinfo{Evoc (Control) [Earth]}{Earth, Nature}
\spellrng{\rngmed}
\spelldur{\durshort}
\spellsr{Yes}
\begin{spelltarget}{One Large or smaller creature}
    \spelleffect The target moves at half speed and takes a \minus2 penalty to physical defenses. If it is flying within 10 feet of the ground, it falls to the ground.
\end{spelltarget}
\spellnotes If the target gets farther than 10 feet from the ground, the spell's effect is broken. As a result, the spell cannot affect creatures flying high above the ground.

\spellsection{Earthen Blade}{2}
\spellinfo{Trans (Alteration, Augment) [Earth]}{Earth, Nature}
\spellrng{Touch}
\spelldur{\durlong \dismissable}
\spellsr{Yes}
\spellline
\spelleffect This spell creates a weapon from the ground. The weapon can be of any type you are proficient with. In addition, the weapon is magical, as the \spell{magic weapon} spell.

\spellsection{Earth Glide}{5}
\spellinfo{Trans (Imbuement) [Earth]}{Earth, Nature}
\spellrng{Touch}
\spelldur{\durshort}
\spellsr{Yes}
\begin{spelltarget}{One creature}
    \spelleffect The target gains the earth glide ability, as an earth elemental. This allows it to glide through stone, dirt, or almost any other sort of earth except metal  as if it were air. It can walk or climb at any angle in the earth. However, the target generally cannot breathe, speak, or hear while gliding. While gliding, the target can remain partially within the earth, granting it cover.
\end{spelltarget}
\spellnotes The target's burrowing leaves behind no tunnel or hole, nor does it create any ripple or other signs of its presence.

\spellsection{Earthquake}{8}
\spelldesc{An intense but highly localized tremor shakes the ground. The shock knocks creatures down and makes fighting or escaping difficult. Eventually, the ground swallows those foolish enough to remain.}
\spellinfo{Evoc (Control) [Earth]}{Destruction, Divine, Earth, Nature}
\spelltwocol{\spellzone{\arealarge radius}}{\spellrng{\rngmed}}
\spelldur{5 rounds}
\spellsr{No}
\spellline
\spelleffect The area is difficult terrain. Creatures in the area take a \minus2 penalty to physical attacks, defenses, and checks. Casting a spell in the area requires a Concentration check against a DC equal to 10 \add your caster level \add double the level of the spell being cast.
\begin{spelltrigger}{End of every round, except the last round}
    \begin{spelltarget}*{All creatures in the area}[Magic vs. Reflex]
        \spellsuccess The target is knocked prone.
    \end{spelltarget}
    \begin{spelltarget}*{All buildings and structures in the area}
        \spelleffect 8d8 bludgeoning damage \add d8 per four caster levels above 16th.
    \end{spelltarget}
\end{spelltrigger}
\begin{spelltrigger}{End of the last round}
    \begin{spelltarget}*{All creatures in the area}
        \spelleffect 8d6 bludgeoning damage \add d6 per four caster levels above 16th. In addition, the target is grappled by the ground until it escapes. To escape, it must beat a Maneuver defense equal to 10 \add your caster level \add your casting attribute.
    \end{spelltarget}
\end{spelltrigger}
\spellnotes In terrain with unusual ground, such as rivers or swamps, this spell may have different effects.

\spellsection{Earthspike}{3}
\spelldesc{You create a spike from the ground that impales your foe.} 
\spellinfo{Trans (Animation) [Earth]}{Earth, Nature}
\spellrng{\rngmed}
\spellsr{No}
\begin{spelltarget}{One creature or object within 10 feet of natural earth or stone}l[Caster level \add casting attribute vs. Armor defense and Maneuver defense]
    \spellsuccess[Armor defense] 6d8 piercing damage \add d8 per two caster levels above 6th.
    \spellsuccess[Armor defense and Maneuver defense] The target is \immobilized by a spike from the ground for 5 rounds. It can break free by destroying the spike. The spike's physical defenses are all 10, and it has 2 hit points per caster level.
\end{spelltarget}

\spellsectioncomma{Earthspike}{Mass}{6}
\spellinfo{Trans (Animation) [Earth]}{Earth, Nature}
\spellrng{\rngmed}
\spelllimit{\areasmall radius}
\spellsr{No}
\begin{spelltargets}{Everything in the area within 10 feet of natural earth or stone}l[Caster level \add casting attribute vs. Armor defense and Maneuver defense]
    \spellsuccess The target is damaged and possibly immobilized, as \spell{earthspike}, except that it takes 6d8 piercing damage \add d8 per four caster levels above 12th.
\end{spelltargets}
\spellnotes This spell cannot attack more than one target within a single 5-ft. square. Each immobilized creature must destroy a separate spike to break free.

\spellsection{Elemental Swarm}{9}
\spellinfo{Conj (Summoning) [see text]}{Air, Earth, Fire, Nature, Water}
\spelltwocol{\spelllimit{\arealarge radius}}{\spellrng{\rngmed}}
\spelldur{\durlong \dismissable}
\spellsr{No}
\spellline
\spelleffect This spell opens a portal to an Elemental Plane and summons elementals from it. A druid can choose the plane (Air, Earth, Fire, or Water); a cleric opens a portal to the plane matching his domain.
\par When the spell is complete, 2d4 Large elementals appear. Five minutes later, 1d4 Huge elementals appear. Five minutes after that, one greater elemental appears. All creatures initially appear wherever you desire within the spell's area. Once these creatures appear, they serve you for the duration of the spell.
\par The elementals obey you explicitly and never attack you, even if someone else manages to gain control over them. You do not need to concentrate to maintain control over the elementals. You can dismiss them singly or in groups at any time.
\spellnotes When you use a summoning spell to summon an air, earth, fire, or water creature, it is a spell of that type.

\spellsection{Energy Conversion}{7}
\spellinfo{Abjur/Evoc (Energy, Shielding) [see text]}{Arcane, Protection}
\spelldur{\durlong or until discharged}
\spellsr{Yes}
\begin{spelltarget}{You}
    \spelleffect You resist energy damage, as \spell{greater resist energy}, except that you store up the energy you absorb. As a standard action, you can expend some of that energy to make an attack.
\end{spelltarget}
\spellrng{\rngclose}
\begin{spelltarget}{One creature}[Magic vs. Reflex]
    \spelleffect Choose an energy type that you have stored. You expend damage of that type of up to three times your caster level.
    \spellsuccess The target takes damage equal to the energy expended.
\end{spelltarget}
\spellnotes This spell's descriptor is the same as the type of energy you discharge in a ray; thus, its subtype can change during the course of the spell's duration.

\spellsection{Energy Drain}{8}
\spellinfo{Necro (Vitalism) [Negative]}{Arcane, Death, Divine, Evil}
\spellrng{\rngclose}
\spelldur{\durshort}
\spellsr{Yes}
\begin{spelltarget}{One creature}[Magic vs. Fortitude]
    \spelleffect If the target is living, it gains six \negativelevels.

    \spelleffect If the target is undead, it gains temporary hit points equal to 40 \add your caster level. In addition, it gains physical damage reduction, reducing the physical damage it takes each round by 16 \add 1 per caster level above 16th. If it takes positive damage, it cannot use its damage reduction for 1 round.
\end{spelltarget}

\spellsection{Enervation}{4}
\spelldesc{Your foe's body loses its color momentarily as you drain its life force away.}
\spellinfo{Necro (Vitalism) [Negative]}{Arcane, Death, Divine, Evil}
\spellrng{\rngclose}
\spelldur{\durshort}
\spellsr{Yes}
\begin{spelltarget}{One creature}[Magic vs. Fortitude]
    \spelleffect If the target is living, it gains three \negativelevels.

    If the target is undead, it gains physical damage reduction, reducing the physical damage it takes each round by 8 \add 1 per caster level above 8th. If it takes positive damage, it cannot use its damage reduction for 1 round.
\end{spelltarget}
\spellnotes This spell stacks with any effect that bestows negative levels, including itself.

\spellsection{Enfeeblement}{1}
%\spelldesc{You fire a coruscating ray from your hand. When it strikes your foe, he becomes weaker.}
\spellinfo{Necro (Flesh)}{Arcane, Death}
\spellrng{\rngmed}
\spelldur{\durshort}
\spellsr{Yes}
\begin{spelltarget}{One creature}[Magic vs. Fortitude]
    \spellsuccess The target takes a \minus4 penalty to your choice of Strength, Dexterity, or Constitution.
    \spellfailure As above, but the penalty is halved.
\end{spelltarget}
\spellnotes This spell cannot reduce an attribute below \minus9.

\spellsection{Enlarge Person}{3}
\spellinfo{Trans (Polymorph) [Size-Affecting]}{Strength, Trans}
\spelltime{Full-round action}
\spellrng{\rngclose}
\spelldur{\durshort \dismissable}
\spellsr{Yes}
\begin{spelltarget}{One humanoid creature (Huge or smaller)}l[Magic vs. Fortitude]
    \spelleffect The target instantly grows, doubling its height and multiplying its weight by 8. This increase changes the creature's size category to the next larger one. This has several effects.
    \begin{itemize} 
        \item \plus10 ft. bonus to movement speed.
        \item \plus4 bonus to maneuver attack and defense.
        \item \minus1 penalty to other physical attacks and defenses.
        \item \minus2 penalty to Dexterity.
        \item \plus2 enhancement bonus to Strength.
        \item \minus4 penalty to Stealth checks.
    \end{itemize}
    \par A typical humanoid creature whose size increases to Large has a space of 10 feet and a natural reach of 10 feet.
    \par If insufficient room is available for the desired growth, the creature attains the maximum possible size and may make a Strength check (using its increased Strength) to burst any enclosures in the process. If it fails, it is constrained without harm by the materials enclosing it -- the spell cannot be used to crush a creature by increasing its size.
    \par All equipment worn or carried by a creature is similarly enlarged by the spell. Melee and projectile weapons affected by this spell deal more damage. Other magical properties are not affected by this spell. Any enlarged item that leaves an enlarged creature's possession (including a projectile or thrown weapon) instantly returns to its normal size. This means that thrown weapons deal their normal damage, and projectiles deal damage based on the size of the weapon that fired them. Magical properties of enlarged items are not increased by this spell.
\end{spelltarget}
\spellnotes Multiple magical effects that increase size do not stack. This spell can be made permanent with a \spell{permanency} ritual.

\spellsectioncomma{Enlarge Person}{Mass}{7}
\spellinfo{Trans (Polymorph) [Size-Affecting]}{Strength, Trans}
\spelltime{Full-round action}
\spelltwocol{\spelllimit{\areamed radius}}{\spellrng{\rngmed}}
\spelldur{\durshort \dismissable}
\spellsr{Yes}
\begin{spelltarget}{Five humanoid creatures in the area (Huge or smaller)}
    \spelleffect The target is enlarged, as \spell{enlarge person}.
\end{spelltarget}

\spellsection{Entangle}{1}
\spelldesc{Grasses, weeds, bushes, and even trees ensnare creatures in the area.}
\spellinfo{Trans (Animation)}{Nature, Wild}
\spelltwocol{\spellzone{\areasmall radius}}{\spellrng{\rngmed}}
\spelldur{\durshort \dismissable}
\spellsr{No}
\spellline
\spelleffect The area is difficult terrain.
\begin{spelltrigger}{End of every movement phase}
    \begin{spelltarget}*{All creatures in the area within 5 feet of plants}l[Magic vs. Reflex]
        \spellsuccess The target is \entangled and \immobilized. It can break free of the effect as a standard action by making a grapple attack or Escape Artist check that beats your magic attack result.
    \end{spelltarget}
\end{spelltrigger}
\spellnotes The effects of this spell may be altered somewhat based on the nature of the plants in the area. If no plants exist in the area, this spell has no effect.

\spellsection{Entangling Growth}{4}
\spelldesc{Grasses, weeds, bushes, and even trees grow out of thin air to ensnare creatures in the area.}
\spellinfo{Trans (Alteration, Animation)}{Nature, Wild}
\spelltwocol{\spellzone{\areamed radius}}{\spellrng{\rngmed}}
\spelldur{\durshort \dismissable}
\spellsr{No}
\spellline
\spelleffect The area is difficult terrain. Plants grow in the area, even if the terrain would not normally support plant life. At the end of the spell's duration, the plants recede into the ground, leaving no trace that they were ever there.
\begin{spelltrigger}{End of every movement phase}
    \begin{spelltarget}{All creatures in the area within 5 feet of the ground}l[Magic vs. Reflex]
        \spellsuccess The target is ensnared, as \spell{entangle}.
    \end{spelltarget}
\end{spelltrigger}
\spellnotes The effects of this spell may be altered somewhat based on the nature of the plants in the area.

\spellsection{Entropic Shield}{2}
\spelldesc{You surround your ally with a magical field that glows with a chaotic blast of multicolored hues. This field deflects incoming ranged attacks, causing them to randomly swerve away from their intended target.}
\spellinfo{Abjur (Shielding)}{Chaos, Divine}
\spellrng{Touch}
\spelldur{\durshort \dismissable}
\begin{spelltarget}{One creature}
    \spelleffect Each physical ranged attack directed at the target has a 50\% miss chance (similar to the effects of active cover). Other attacks that simply work at a distance are not affected.
\end{spelltarget}

\spellsection{Ethereal Jaunt}{7}
\spellinfo{Conj (Translocation) [Planar]}{Arcane, Travel}
\spelldur{\durshort \dismissable}
\begin{spelltarget}{You}
    \spelleffect You become ethereal, along with your equipment. For the duration of the spell, you are in a place called the Ethereal Plane, which overlaps the normal, physical, Material Plane. When the spell expires, you return to material existence.
    \par An ethereal creature is invisible, insubstantial, and capable of moving in any direction, even up or down, albeit at half normal speed. As an insubstantial creature, you can move through solid objects, including living creatures. An ethereal creature can see and hear on the Material Plane, but everything looks gray and ephemeral. Sight and hearing onto the Material Plane are limited to 60 feet.
    \par Force effects and abjurations affect an ethereal creature normally. Their effects extend onto the Ethereal Plane from the Material Plane, but not vice versa. An ethereal creature can't attack material creatures, and spells you cast while ethereal affect only other ethereal things. Certain material creatures or objects have attacks or effects that work on the Ethereal Plane (such as a basilisk's gaze attack). Treat other ethereal creatures and ethereal objects as if they were material. 
    \par If you end the spell and become material while inside a material object (such as a solid wall), you are shunted off to the nearest open space and take 1d6 damage per 5 feet that you so travel.
\end{spelltarget}
\spellnotes If you are not on the Material Plane when you cast this spell, it has no effect.

\spellsection{Etherealness}{9}
\spellinfo{Conj (Translocation) [Planar]}{Arcane, Travel}
\spellrng{Touch}
\spellsr{Yes}
\begin{spelltarget}{You and up to five willing creatures}
    \spelleffect The target becomes ethereal, as \spell{ethereal jaunt}.
\end{spelltarget}
\spellnotes When the spell expires, all affected creatures on the Ethereal Plane return to the Material Plane. If you are not on the Material Plane when you cast this spell, it has no effect.

\spellsection{Expeditious Retreat}{1}
\spellinfo{Trans (Temporal)}{Trans}
\spellrng{\rngclose}
\spelldur{\durshort \dismissable}
\begin{spelltarget}{One creature}
    \spelleffect The target's land speed doubles, to a maximum of a \plus30 foot increase. (This adjustment is treated as an enhancement bonus.) There is no effect on other modes of movement.
\end{spelltarget}
\spellnotes As with any effect that increases your speed, this spell affects your ability to jump (see \pcref{Athletics}).

\pdfbookmark[2]{F}{SpellDescriptionsF}
\begin{comment}
\subsubsection{F}
\end{comment}
\spellsr{Yes}

\spellsection{Faerie Fire}{1}
\spellinfo{Illus (Figment) [Light]}{Nature}
\spelltwocol{\spellburst{\areasmall radius}}{\spellrng{\rngmed}}
\spelldur{\durshort \dismissable}
\spellsr{Yes}
\begin{spelltarget}{Everything in the area}
    \spelleffect A pale glow surrounds and outlines the target, causing it to shed light as a candle. This imposes a \minus20 penalty to Stealth checks, and negates invisibility, concealment, and similar effects. Illusory figments, such as those created by the \spell{silent image} spell, are not outlined, which may reveal their false nature.
\end{spelltarget}
\spellnotes The light is too dim to have any special effect on creatures vulnerable to light.

\spellsection{False Reality}{9}
\spellinfo{Illus (Figment) [Unreal]}{Illus}
\spellzone{1 mile radius centered on you}
\spelldur{\durlong \dismissable}
\spellline
\spelleffect A scripted figment of your design appears within the area, as \spell{persistent image}.
\spellnotes Creatures can identify the illusion, as \spell{silent image}.

\spellsection{Fear}{5}
\spelldesc{You project an invisible cone that drives creatures away from you in abject fear.}
\spellinfo{Ench (Emotion) [Fear, Mind-Affecting]}{Arcane}
\spellburst{\areamed cone}
\spelldur{\durshort \dismissable}
\spellsr{Yes}
\begin{spelltarget}{All creatures in the area}[Magic vs. Will]
    \spellsuccess The target is \shaken.

    As long as the target is \bloodied, it is \frightened instead.
\end{spelltarget}

\spellsection{Feather Fall}{1}
\spellinfo{Evoc (Control) [Air]}{Air, Evoc, Travel}
\spelltwocol{\spelltime{1 swift action}}{\spellcmp{Verbal only}}
\spelltwocol{\spelllimit{\areamed radius}}{\spellrng{\rngmed}}
\spelldur{\durshort or until landing}
\spellsr{Yes}
\begin{spelltargets}{Up to five freefalling objects or willing creatures in the area (Medium or smaller)}
    \spelleffect This spell changes at which the target falls to a mere 60 feet per round (equivalent to the end of a fall from a few feet). It takes no damage from falling. If it is heavy enough to deal falling damage to other creatures and objects, it deals half its normal falling damage, with no bonus for the height of the drop.
\end{spelltargets}
\spellnotes This spell affects up to five Medium or smaller creatures (including gear and carried objects up to each creature's maximum load) or objects, or the equivalent in larger creatures: A Large creature or object counts as two Medium creatures or objects, a Huge creature or object counts as two Large creatures or objects, and so forth.

This spell works only upon free-falling objects. It no special effect on ranged weapons unless they are falling an extraordinary distance. It does not affect a sword blow or a charging or flying creature.

\spellsection{Feeblemind}{7}
\spellinfo{Ench (Inhibition) [Mind-Affecting]}{Arcane}
\spellrng{\rngclose}
\spelldur{\durshort or permanent}
\spellsr{Yes}
\begin{spelltarget}{One creature}[Magic vs. Will]
    \spellsuccess The target is \bewildered for \durshort duration.

    If the target is \bloodied, its Intelligence drops to \minus9, giving it roughly the intellect of a lizard. It is unable to use Intelligence-based skills, cast spells, understand language, or communicate coherently. Still, it knows who its friends are and can follow them and even protect them. The target remains in this state until a \spell{heal}, \spell{limited wish}, \spell{miracle}, or \spell{wish} spell is used to cancel the effect of the \spell{feeblemind}.
\end{spelltarget}

\spellsection{Finger of Death}{7}
\spellinfo{Necro (Life) [Death]}{Arcane, Death}
\spellrng{\rngclose}
\spellsr{Yes}
\begin{spelltarget}{One living creature}[Magic vs. Fortitude]
    \spellsuccess The target is \staggered for 5 rounds.

    If it is \bloodied, it instead loses all its hit points and takes critical damage equal to your caster level, causing it to begin dying.
\end{spelltarget}

\spellsection{Fire Seeds}{6}
\spellinfo{Evoc/Trans (Energy, Imbuement) [Fire]}{Fire, Nature, Wild}
\spelldur{\durext or until discharged}
\spellsr{Yes}
\spellline
\spelleffect You can imbue up to five acorns, holly berries, or other seeds with fiery energy capable of dealing up to 6d6 damage \add d6 per four caster levels above 12th. You may freely decide the distribution of damage between the target berries.

As a standard action, you can say a command word to detonate any number of affected seeds. In addition, the seeds detonate on impact. They may be thrown, with a range increment of 20 feet.
\spellburst{\areasmall radius centered on a detonating seed}
\begin{spelltarget}*{Everything in the area}[Magic vs. Reflex]
    \spellsuccess The target takes the damage imbued into the seed.
    \spellfailure As above, but half damage.
    \spellspecial This attack automatically succeeds against a creature that is struck directly by a thrown seed or holding a seed when it detonates.
\end{spelltarget}
\mfx{Material Component} The seeds.
\spellnotes You can only have one \spell{fire seeds} spell active at any time.

\spellsection{Fire Shield}{4}
\spelldesc{You appear to immolate yourself in a wreath of flame that lashes out at anyone who tries to harm you.}
\spellinfo{Abjur/Evoc (Energy, Shielding) [Fire or Cold]}{Arcane, Fire}
\spelldur{\durshort \dismissable}
\spellsr{Yes}
\begin{spelltarget}{You}
    \spelleffect You gain cold damage reduction, reducing the cold damage you take each round by 20 \add 2 per caster level above 8th. In addition, you radiate light as a torch, and creatures that attack you take damage.
\end{spelltarget}
\begin{spelltrigger}{Creature within \rngclose range makes a physical attack against you}
    \begin{spelltarget}*{The attacking creature}[Magic vs. Reflex]
        \spellsuccess 5d6 fire damage \add d6 per four caster levels above 10th.
        \spellfailure Half damage.
    \end{spelltarget}
\end{spelltrigger}

\spellsection{Fire Storm}{8}
\spelldesc{You fill a massive area with sheets of roaring flame, burning everyone who opposes you.}
\spellinfo{Evoc (Energy) [Fire]}{Destruction, Fire, Nature, War}
\spelltwocol{\spellburst{\arealarge radius}}{\spellrng{\rngmed}}
\spellsr{Yes}
\begin{spelltargets}{Everything in the area, except allied creatures}l[Magic vs. Reflex]
    \spellsuccess 8d6 fire damage \add d6 per four caster levels above 16th.
    \spellfailure As above, but half damage.
\end{spelltargets}

\spellsection{Fireball}{3}
\spelldesc{You create an explosion of flame that detonates with a low roar, damaging nearby creatures and objects.}
\spellinfo{Evoc (Energy) [Destructive, Fire]}{Arcane, Fire}
\spelltwocol{\spellburst{\areasmall radius}}{\spellrng{\rngmed}}
\spellsr{Yes}
\begin{spelltargets}{Everything in the area}[Magic vs. Reflex]
    \spellsuccess 3d6 fire damage \add d6 per four caster levels above 6th.
    \spellfailure As above, but half damage.
\end{spelltargets}
\spellnotes \destructivespellnotes

\firespellnotes

\spellsection{Flame Blade}{2}
\spellinfo{Evoc (Energy) [Fire]}{Nature, Fire}
\spellsr{Yes}
\spelldur{\durlong \dismissable}
\spelleffect A 3 foot long beam of red-hot fire springs forth from your hand. In addition to providing illumination like a torch, you can wield this bladelike beam as a weapon. It is treated like a scimitar, except that all damage dealt with it is fire damage, you add half your casting attribute to damage in place of half your Strength, and it is treated as a light weapon, so you can use Dexterity to attack with it. Alternately, you can hurl flames from the weapon up to \rngmed range as if it were a thrown weapon.
\spellnotes \firespellnotes

Spell resistance applies when a foe is struck by the weapon, but not when the blade is created.

\spellsection{Flame Strike}{5}
\spelldesc{You call a vertical column of divine fire that roars downward, consuming your unworthy foes.}
\spellinfo{Evoc (Channeling, Energy) [Destructive, Fire]}{Destruction, Divine, Fire, War}
\spellrng{\rngclose}
\spellburst{\areamed radius cylinder, 40 ft. high}
\spellsr{Yes}
\begin{spelltarget}{Everything in the area}[Magic vs. Reflex]
    \spellsuccess 5d6 fire and divine damage \add d6 per four caster levels above 8th. If the target is an ally, it takes half damage, and all of the damage it takes is fire damage.
    \spellfailure As above, but half damage.
\end{spelltarget}
\spellnotes \destructivespellnotes

\firespellnotes

\spellsection{Fly}{4}
\spellinfo{Trans (Imbuement)}{Arcane}
\spellrng{\rngtouch}
\spelldur{\durshort}
\spellsr{Yes}
\begin{spelltarget}{One creature}
    \spelleffect The target gains a 30 foot fly speed with good maneuverability.
\end{spelltarget}
\spellnotes An unencumbered creature with a fly speed can fly through the air. See \pcref{Flying}, for more details.

%priced as buff spell for range
\spellsection{Fog Cloud}{2}
\spelldesc{You conjure a bank of fog, concealing those inside.}
\spellinfo{Conj (Creation) [Fog]}{Arcane, Nature, Water}
\spelltwocol{\spellzone{\areamed radius cylinder}}{\spellrng{\rngmed}}
\spelldur{\durshort}
\spellsr{No}
\spellline
\spelleffect Fog blocks sight in the area, causing all creatures within or looking through the area to treat everything they see as if it had \concealment.
\spellnotes \fogspellnotes \fogwindspellnotes

\spellsection{Forcecage}{7}
\spellinfo{Evoc (Control) [Force]}{Evoc}
\spellrng{\rngmed}
\spellzone{Solid wall in 20-foot cube or 10-foot cube}
\spelldur{\durlong \dismissable}
\spellline
\spelleffect An immobile, invisible cubical prison appears. You choose whether the walls made of bars of force or perfect sheets of force. A barred prison is a 20-foot cube. The bars are a half-inch wide, with half-inch gaps between them. An unbarred prison is a 10-foot cube.
\begin{spelltarget}*{All creatures adjacent to the prison}[Magic vs. Reflex]
    \spellsuccess The prison is formed normally.
    \spellfailure The target can disrupt the prison, preventing it from being formed in any squares adjacent to the target. The rest of the prison is unaffected.
\end{spelltarget}
\spellnotes As \spell{wall of force}.

\spellsection{Foresee Probability}{2}
\spellinfo{Div (Knowledge)}{Div, Knowledge}
\spellrng{\rngmed}
\spelldur{\durshort or until expended}
\spellsr{Yes}
\spellspecial When you cast this spell, you roll a d20 twice. Store the results in order. At 10th caster level, and every 5 caster levels thereafter, you roll an additional die when you cast this spell.
\begin{spelltarget}{One creature}[Magic vs. Will]
    \spellsuccess Each time the target would roll a d20, it instead uses a die result you rolled for it. The die results are used in the same order you rolled them. When all the die results have been used, the spell is expended.
\end{spelltarget}
\spellnotes Any die results unused when the spell duration expires are discarded.

\spellsection{Foresight}{5}
\spellinfo{Div (Knowledge)}{Arcane, Knowledge, Protection}
\spellrng{Touch}
\spelldur{\durshort or \durlong; see text \dismissable}
\begin{spelltarget}{One creature}
    \spelleffect The target receives instantaneous warnings of impending danger or harm that would befall it. It gains an enhancement bonus to initiative checks equal to your caster level, and a \plus3 enhancement bonus to its dodge modifier and Reflex defense. This bonus increases to \plus4 at 14th caster level, and to \plus5 at 20th caster level.
    \par If you cast this spell on yourself, it lasts for \durlong duration. On any other creature, it lasts for \durshort duration.
\end{spelltarget}

\spellsectioncomma{Foresight}{Greater}{9}
\spelldesc{You bestow a powerful sixth sense to your ally, giving them clear visions of any imminent danger.}
\spellinfo{Div (Knowledge)}{Arcane, Knowledge, Protection}
\spellrng{Touch}
\spelldur{\durshort or \durlong; see text \dismissable}
\begin{spelltarget}{One creature}
    \spelleffect As \spell{foresight}, except that the target is also never surprised or unaware.
\end{spelltarget}

\spellsection{Forget}{1}
\spellinfo{Ench (Complusion)}{Chaos, Ench}
\spellrng{\rngmed}
\spelldur{\durlong}
\spellsr{Yes}
\begin{spelltarget}{One creature}[Magic vs. Will]
    \spelleffect The target forgets something simple. You can't make it forget something important, such as its name. You must know what you want it to forget. The spell does not prevent the target from learning the information again, and it can remember the information normally after the spell's duration.
\end{spelltarget}

\spellsection{Freedom}{4}
\spellinfo{Trans (Imbuement)}{Divine, Nature, Travel}
\spellrng{\rngtouch}
\spelldur{\durshort}
\spellsr{Yes}
\begin{spelltarget}{One creature}
    \spelleffect The target can move and attack normally for the duration of the spell, even under the influence of magic that usually impedes movement, such as paralysis, \spell{solid fog}, \spell{slow}, and \spell{web}. The target gains a \plus20 enhancement bonus to Maneuver Class against grapple attacks, as well as on grapple attacks or Escape Artist checks made to escape a grapple or a pin.
    \par The spell also allows the target to move and attack with melee weapons normally while underwater.
\end{spelltarget}

\spellsectioncomma{Freedom}{Mass}{8}
\spellinfo{Trans (Imbuement)}{Divine, Nature, Travel}
\spelltwocol{\spelllimit{\areamed radius}}{\spellrng{\rngmed}}
\spellsr{Yes}
\begin{spelltargets}{Up to five creatures in the area}
    \spelleffect The target can move freely, as \spell{freedom}.
\end{spelltargets}

\pdfbookmark[2]{G}{SpellDescriptionsG}
\begin{comment}
\subsubsection{G}
\end{comment}

\spellsection{Gaseous Form}{3}
\spellinfo{Trans (Polymorph)}{Arcane, Air, Travel}
\spellcmp{Somatic only}
\spellrng{\rngtouch}
\spelldur{\durshort \dismissable}
\begin{spelltarget}{One willing corporeal creature}
    \spelleffect The target and all its equipment becomes insubstantial, misty, and translucent. Its armor modifier becomes 0, though other defense modifiers continue to apply normally. It gains physical damage reduction, reducing the physical damage it takes each round by 15 \add 2 per caster level above 6th. If it takes damage from a spell or magic weapon, it cannot use its damage reduction for 1 round. It becomes immune to critical hits.

    The target can't attack or cast spells with verbal, somatic, or material components while in gaseous form. If it has a touch spell ready to use, that spell is discharged harmlessly when the gaseous form spell takes effect.
    \par A gaseous creature can fly at a speed of 10 feet (special maneuverability). It can pass through small holes or narrow openings, even mere cracks, with all it was wearing or holding in its hands, as long as the spell persists. The creature is subject to the effects of wind, and it can't enter water or other liquid. It also can't manipulate objects or activate items, even those carried along with its gaseous form. Continuously active items remain active, though in some cases their effects may be moot.
\end{spelltarget}

\spellsection{Gentle Descent}{2}
\spelldesc{You grant your ally ephemeral wings which allow him to glide.}
\spellinfo{Trans (Imbuement) [Air]}{Air, Nature}
\spellrng{\rngmed}
\spelldur{\durlong}
\spellsr{Yes}
\begin{spelltarget}{One creature}
    \spelleffect The target gains a 30 foot glide speed.
\end{spelltarget}
\spellnotes A creature with a glide speed can glide through the air at the indicated speed (see \pcref{Gliding}).

\spellsection{Ghoul Touch}{1}
\spelldesc{Your foe feels the touch of a ghoul's undead hand against its flesh.}
\spellinfo{Necro (Flesh)}{Arcane}
\spellrng{\rngmed}
\spelldur{1 round}
\spellsr{Yes}
\begin{spelltarget}{One living creature}[Magic vs. Fortitude]
    \spellsuccess The target is \sickened.

    If it is \bloodied, it is \nauseated instead.
\end{spelltarget}

\spellsection{Giant Vermin}{4}
\spellinfo{Trans (Polymorph)}{Nature, Wild}
\spelltwocol{\spelllimit{\areamed radius}}{\spellrng{\rngclose}}
\spelldur{\durmed}
\spellsr{Yes}
\begin{spelltarget}{Up to three vermin in the area}
    \spelleffect You turn three normal-sized centipedes, two normal-sized spiders, or a single normal-sized scorpion into Large-sized forms. Only one type of vermin can be transmuted (so a single casting cannot affect both a centipede and a spider), and all must be grown to the same size.
    \par Any giant vermin created by this spell do not attempt to harm you, but your control of such creatures is limited to simple commands (``Attack," ``Defend," ``Stop," and so forth). Orders to attack a certain creature when it appears or guard against a particular occurrence are too complex for the vermin to understand. Unless commanded to do otherwise, the giant vermin attack whoever or whatever is near them.
\end{spelltarget}

\spellsection{Glitterdust}{2}
\spellinfo{Conj (Creation)}{Arcane}
\spelltwocol{\spellburst{\areasmall radius}}{\spellrng{\rngmed}}
\spelldur{\durshort}
\begin{spelltarget}{Everything in the area}
    \spelleffect Golden particles surround and outline the target. This imposes a \minus20 penalty to Stealth checks, and negates invisibility, concealment, and similar effects. Illusory figments, such as those created by the \spell{silent image} spell, are not outlined, which may reveal their false nature.
\end{spelltarget}
\spellnotes Water and similar substances can remove the dust.

\spellsection{Grasping Hand}{6}
\spellinfo{Evoc (Control) [Force]}{Arcane}
\spellrng{\rngmed}
\spelldur{\durshort \dismissable}
\spellsr{Yes}
\spellline
\spelleffect This spell creates a hand, as \spell{interposing hand}, except that the hand grapples its target instead of protecting you from it.
\begin{spelltarget}*{One creature or object}l[Caster level \add casting attribute vs. Maneuver defense]
    \spellsuccess The target is grappled. The hand deals no damage.
\end{spelltarget}
\spellnotes As \spell{interposing hand}. The hand attacks during the action phase, regardless of when you direct it to attack a target.

\spellsection{Grease}{1}
\spelldesc{You conjure a layer of slippery grease on the ground, tripping up your foes.}
\spellinfo{Conj (Creation)}{Arcane}
\spelltwocol{\spellzone{\areasmall radius}}{\spellrng{\rngmed}}
\spelldur{\durshort \dismissable}
\spellsr{No}
\spellline
\spelleffect Any creature moving through the area must make an Acrobatics check to balance. The DC is equal to half your caster level \add your casting attribute. Success means it moves normally (though usually at half speed, because it is balancing). Failure means the creature's movement is wasted. Failure by 10 or more means it also falls prone.

\invocationsection{Greater (Spell Name)}
\par Any spell whose name begins with greater is alphabetized in this chapter according to the second word of the spell name. Thus, the description of a greater spell appears near the description of the spell on which it is based. Spell chains that have greater spells in them include those based on the spells command, dispel magic, invisibility, magic fang, magic weapon, restoration, scrying, shadow conjuration, shadow evocation, shout, and teleport.

\spellsection{Gust of Wind}{1}
\spelldesc{You create a severe blast of air that knocks your foes flying.}
\spellinfo{Evoc (Control) [Air]}{Air, Nature}
\spellzone{\arealarge line from you}
\begin{spelltarget}{Everything in the area}l[Caster level \add casting attribute vs. Maneuver Defense]
    \spellsuccess The target is affected by a shove attack, pushing it back by 5 feet \add 5 feet per 5 points by which your attack exceeded its defense. If it is pushed outside the spell's area, it is not pushed farther.
\end{spelltarget}
\spellnotes In addition to the effect noted, a \spell{gust of wind} can do anything that a sudden blast of wind would be expected to do. It can extinguish open flames, create a stinging spray of sand or dust, fan a large fire, overturn delicate awnings or hangings, heel over a small boat, and blow gases or vapors to the edge of its range.

This spell can be made permanent with a \spell{permanency} ritual.

\pdfbookmark[2]{H}{SpellDescriptionsH}
\begin{comment}
\subsubsection{H}
\end{comment}

\spellsection{Harm}{6}
\spelldesc{You fill your foe with a massive influx of negative energy, crippling its body.}
\spellinfo{Necro (Vitalism) [Negative]}{Arcane, Death, Divine, Evil, Vitality}
\spellrng{\rngclose}
\spellsr{Yes}
\begin{spelltarget}{One creature}[Magic vs. Fortitude]
    \spelleffect If the target is undead, it is healed for 12d6 damage \add d6 per two caster levels above 12th.
    \spellsuccess If the target is not undead, it takes that much negative energy damage, as well as four points of Constitution damage.
    \spellfailure As above, but both the negative energy damage and constitution damage is halved.
\end{spelltarget}

\spellsection{Haste}{4}
\spelldesc{You accelerate your ally's motions, causing her to move and act more quickly than normal.}
\spellinfo{Trans (Temporal)}{Trans}
\spellrng{Touch}
\spelldur{\durshort}
\spellsr{Yes}
\begin{spelltarget}{One creature}
    \spelleffect The target is hasted. It doubles all of its movement speeds (to a maximum of an additional 30 feet of movement), and can take an additional attack at a \minus5 penalty when it makes a standard attack. The increase to movement speed is considered an enhancement bonus.
\end{spelltarget}
\spellnotes The extra attack granted is not cumulative with similar effects.

\spellsectioncomma{Haste}{Mass}{8}
\spelldesc{You accelerate your allies' motions, causing them to move and act more quickly than normal.}
\spellinfo{Trans (Temporal)}{Trans}
\spelllimit{\areamed radius centered on you}
\begin{spelltargets}{Up to five creatures in the area}
    \spelleffect The target moves and acts more quickly, as \spell{haste}.
\end{spelltargets}

\spellsection{Heal}{6}
\spelldesc{You fill an ally with a massive influx of positive energy, restoring its body to its fullest.}
\spellinfo{Necro (Vitalism) [Positive]}{Divine, Good, Nature, Vitality}
\spellrng{\rngclose}
\spellsr{Yes}
\begin{spelltarget}{One creature}[Magic vs. Fortitude]
    \spelleffect If the target is living, it gains three benefits. First, it is healed for 12d6 damage \add d6 per two caster levels above 12th. Second, for every 5 points of healing granted by the spell, it can instead cure 1 point of critical damage. Third, all of the following conditions are also removed from the target: ability damage, bewildered, blinded, confused, dazed, dazzled, deafened, diseased, exhausted, fatigued, feebleminded, insanity, nauseated, sickened, stunned, and poisoned.
    \spellsuccess If the target is undead, it takes that much positive damage.
    \spellfailure As above, but half damage.
\end{spelltarget}

\spellsection{Heat Metal}{2}
\spelldesc{You heat your foe's armor, blistering its skin.}
\spellinfo{Evoc (Energy) [Fire]}{Nature}
\spellrng{\rngmed}
\spellfocus{One metal object}
\spelldur{\durshort \dismissable}
\spellsr{Yes}
\begin{spelltrigger}{End of every round}
    \begin{spelltargets}*{The focus and everything touching it}l[Magic vs. Fortitude]
        \spellsuccess The target takes 2d6 fire damage \add d6 per four caster levels above 4th.
        \spellfailure As above, but half damage.
    \end{spelltargets}
\end{spelltrigger}
\spellnotes If the target is underwater, this spell deals half damage, and boils the surrounding water. Any cold intense enough to damage the creature negates fire damage from the spell (and vice versa) on a point-for-point basis. This spell can affect the armor worn by a typical Medium creature, but not generally by a larger creature.

\spellsection{Heroism}{3}
\spelldesc{You imbue your ally with great bravery and morale in battle.}
\spellinfo{Ench (Emotion) [Mind-Affecting, Morale]}{Arcane}
\spellrng{\rngclose}
\spelldur{\durshort \dismissable}
\spellsr{Yes}
\begin{spelltarget}{One creature}
    \spelleffect The target gains a \plus2 enhancement bonus on physical attacks, all checks, and special defenses. \spellbonusscalingdescription
\end{spelltarget}

\spellsectioncomma{Heroism}{Greater}{7}
\spellinfo{Ench (Emotion) [Mind-Affecting, Morale]}{Arcane}
\spellrng{\rngclose}
\spelldur{\durshort \dismissable}
\spellsr{Yes}
\begin{spelltarget}{One creature}
    \spelleffect As \spell{heroism}, except the target also gains 35 temporary hit points \add 1 per caster level above 14th. In addition, it is immune to hostile morale effects.
\end{spelltarget}

\spellsection{Hold Monster}{4}
\spellinfo{Ench (Inhibition) [Mind-Affecting]}{Arcane, Law}
\spellrng{\rngclose}
\spelldur{\durshort \dismissable; see text}
\spellsr{Yes}
\begin{spelltarget}{One creature}
    \spellsuccess As \spell{hold person}, except that the target can have any creature type.
\end{spelltarget}

\spellsectioncomma{Hold Monster}{Mass}{9}
\spellinfo{Ench (Inhibition) [Mind-Affecting]}{Arcane}
\spelltwocol{\spelllimit{\areamed radius}}{\spellrng{\rngclose}}
\spelldur{\durshort \dismissable; see text}
\spellsr{Yes}
\begin{spelltargets}{Up to five creatures in the area}
    \spellsuccess As \spell{hold person}, except that the target can have any creature type.
\end{spelltargets}

\spellsection{Hold Person}{2}
\spellinfo{Ench (Inhibition) [Mind-Affecting]}{Arcane, Divine, Law, War}
\spellrng{\rngclose}
\spelldur{\durshort \dismissable; see text}
\spellsr{Yes}
\begin{spelltarget}{One humanoid creature}[Magic vs. Will]
    \spellsuccess The target is \bewildered.

    As long as the target is \bloodied, it is also \dazed and unable to act. At the end of every round where the creature is bloodied, you must make a new Magic vs. Will attack against it to sustain the effect. If this attack fails, the creature is no longer dazed for the remainder of the spell, though it remains bewildered.
\end{spelltarget}

\spellsectioncomma{Hold Person}{Mass}{8}
\spellinfo{Ench (Inhibition) [Mind-Affecting]}{Arcane, Divine}
\spelltwocol{\spelllimit{\areamed radius}}{\spellrng{\rngmed}}
\spelldur{\durshort \dismissable; see text}
\spellsr{Yes}
\begin{spelltargets}{Up to five creatures in the area}
    \spellsuccess As \spell{hold person}.
\end{spelltargets}

\spellsection{Holy Aura}{8}
\spellinfo{Abjur (Interdiction) [Good]}{Divine, Good}
\spelllimit{\areamed radius centered on you}
\spelldur{\durshort \dismissable}
\spellsr{Yes}
\begin{spelltargets}{Up to five creatures in the area}
    The target gains a \plus5 enhancement bonus to its defenses. In addition, it gains spell resistance against evil spells and spells cast by evil creatures.
    \par At the end of each round, all evil creatures within \rngclose range of the target that attacked it with their body or a melee weapon that round take 4d6 points of damage. A creature that attacks multiple creatures shielded by this spell can take this damage multiple times.
\end{spelltargets}

\spellsection{Holy Smite}{4}
\spellinfo{Evoc (Channeling) [Good]}{Good}
\spellrng{\rngmed}
\spelldur{5 rounds}
\spellsr{Yes}
\begin{spelltarget}{One nongood creature}[Magic vs. Will]
    \spellsuccess 8d6 divine damage \add d6 per two caster levels above 8th, and the target is \bewildered for 5 rounds.
    \spellfailure As above, but half damage.
\end{spelltarget}

\spellsection{Holy Word}{7}
\spellinfo{Evoc (Channeling) [Good]}{Good, Divine}
\spellcmp{Verbal only}
\spellburst{\arealarge radius centered on you}
\spellsr{Yes}
\begin{spelltargets}{All nongood creatures in the area}
    \spelleffect If the target's level does not exceed your caster level, it is \deafened for 5 rounds.

    If it is also \bloodied, it suffers one or more of the following ill effects, depending on its level.
    \begin{itemize}
        \item Up to caster level \minus5: The target is also \blinded for 1 round.
        \item Up to caster level \minus10: The target is also \paralyzed for 5 rounds.
        \item Up to caster level \minus15: A living target loses all its hit points and takes critical damage equal to your caster level, causing it to begin dying. A nonliving target is destroyed.
    \end{itemize}
\end{spelltargets}

\spellsection{Horrid Wilting}{8}
\spelldesc{You dessicate your foes from a great distance, shriveling their bodies.}
\spellinfo{Necro (Flesh)}{Necro, Water}
\spelltwocol{\spelllimit{\arealarge radius}}{\spellrng{\rngfar}}
\spellsr{Yes}
\begin{spelltargets}{Up to ten living creatures in the area}[Magic vs. Fortitude]
    \spellspecial You gain a \plus5 bonus to attack plants and creatures with the water subtype.
    \spellsuccess 8d6 physical damage \add d6 per four caster levels above 16th.
    \spellfailure As above, but half damage.
\end{spelltargets}

\spellsection{Hypnotic Pattern}{3}
\spelldesc{You create a twisting pattern of subtle, shifting colors that weaves through the air, fascinating creatures within it.}
\spellinfo{Ench/Illus (Compulsion, Figment) [Light, Mind-Affecting, Sight-Dependent]}{Arcane}
\spelltwocol{\spellzone{\areasmall radius}}{\spellrng{\rngmed}}
\spelldur{\durshort}
\begin{spelltarget}{All creatures in the area}[Magic vs. Will]
    \spellsuccess The target is \fascinated.
\end{spelltarget}
\spellnotes Creatures who cannot see the lights are not affected by this spell.

\pdfbookmark[2]{I}{SpellDescriptionsI}
\begin{comment}
\subsubsection{I}
\end{comment}
\spellsr{Yes}

\spellsection{Ice Storm}{4}
\spelldesc{You conjure magical hailstones that pound down, smashing and chilling creatures in their path.}
\spellinfo{Conj/Evoc (Creation, Energy) [Cold, Destructive]}{Arcane, Destruction, Nature, Water}
\spellrng{\rngmed}
\spellburst{\areasmall radius cylinder, 20 ft. high}
\spelldur{1 round}
\spellline
\spelleffect The area is difficult terrain for 1 round.
\spellsr{Yes}
\begin{spelltargets}*{Everything in the area}
    \spellsuccess 4d4 cold and bludgeoning damage \add d4 per four caster levels above 8th.
\end{spelltargets}
\spellnotes \destructivespellnotes

\spellsection{Implosion}{9}
\spelldesc{You create a destructive responance in your foe's body that destroys it from the inside out.}
\spellinfo{Evoc (Control)}{Destruction, Divine}
\spellrng{\rngclose}
\spelldur{Concentration (up to 5 rounds)}
\spellsr{Yes}
\spellline
\begin{spelltrigger}{End of every round}
    \begin{spelltarget}{One creature}*[Magic vs. Fortitude]
        \spellsuccess The target is \staggered for 5 rounds.

        If the target is \bloodied, it is instead instantly slain.
        \spellspecial You cannot target the same creature more than once per casting of this spell.
    \end{spelltarget}
\end{spelltrigger}
\spellnotes \spell{Implosion} has no effect on creatures in \spell{gaseous form} or on incorporeal creatures.

\spellsection{Imprisonment}{9}
\spelldesc{You teleport your foe deep beneath the earth, leaving it in stasis forever.}
\spellinfo{Conj/Trans (Time, Translocation) [Teleportation]}{Arcane, Earth, Law}
\spellrng{\rngclose}
\spelldur{See text}
\spellsr{Yes}
\begin{spelltarget}{One creature touching the ground}[Magic vs. Will]
    \spellsuccess The target is \slowed for 5 rounds.

    \spellsuccess If the target is \bloodied, it instead becomes permanently entombed in a state of suspended animation (as the \spell{temporal stasis} spell) in a small sphere far beneath the surface of the earth. It remains there until an \spell{emancipation} spell is cast at the location where the imprisonment took place.
\end{spelltarget}

\spellnotes If the target becomes inprisoned beneath the earth, it is very difficult to find. Magical search by a crystal ball, a \spell{locate creature} spell, or some other similar divination does not reveal the fact that a creature is imprisoned, but \spell{discern location} does. A \spell{wish} or \spell{miracle} spell will not free the recipient, but will reveal where it is entombed.

\spellsection{Inertial Shield}{2}
\spelldesc{You create a barrier around your ally that resists physical intrusion.}
\spellinfo{Abjur (Shielding)}{Arcane}
\spellrng{\rngclose}
\spelldur{\durshort}
\spellsr{Yes}
\begin{spelltarget}{One creature}
    \spelleffect The target gains physical damage reduction, reducing the physical damage it takes each round by 4 \add 1 per caster level above 4th. If it takes force damage, it cannot use its damage reduction for 1 round.
\end{spelltarget}

\spellsection{Inflict Critical Wounds}{4}
\spellinfo{Necro (Vitalism) [Negative]}{Arcane, Divine}
\spellrng{\rngclose}
\spellsr{Yes}
\begin{spelltarget}{One creature}[Magic vs. Fortitude]
    \spellsuccess If the target is not undead, this spell inflicts 8d6 negative energy damage \add d6 per two caster levels above 8th. For every 5 points of damage dealt in excess of the target's hit points, it can instead inflict 1 point of critical damage.
    \spellfailure As above, but half damage.
    \spelleffect If the target is undead, it is instead healed for that much damage.
\end{spelltarget}

\spellsectioncomma{Inflict Critical Wounds}{Mass}{8}
\spellinfo{Necro (Vitalism) [Negative]}{Arcane, Divine}
\spelltwocol{\spelllimit{\areamed radius}}{\spellrng{\rngmed}}
\spellsr{Yes}
\begin{spelltargets}{Up to five creatures in the area}[Magic vs. Fortitude]
    \spellsuccess As \spell{inflict light wounds}, except that it deals 8d6 negative energy damage \add d6 per four caster levels above 16th.
    \spellfailure As above, but half damage.
    \spelleffect If the target is undead, it is instead healed for that much damage.
\end{spelltargets}

\spellsection{Inflict Light Wounds}{1}
\spellinfo{Necro (Vitalism) [Negative]}{Arcane, Divine}
\spellrng{\rngclose}
\spellsr{Yes}
\begin{spelltarget}{One creature}[Magic vs. Fortitude]
    \spellsuccess If the target is not undead, it takes 2d6 negative energy damage \add d6 per two caster levels above 2nd.
    \spellfailure As above, but half damage.
    \spelleffect If the target is undead, it is instead healed for that much damage.
\end{spelltarget}

\spellsectioncomma{Inflict Light Wounds}{Mass}{5}
\spellinfo{Necro (Vitalism) [Negative]}{Arcane, Divine}
\spelltwocol{\spelllimit{\areamed radius}}{\spellrng{\rngmed}}
\spellsr{Yes}
\begin{spelltargets}{Up to five creatures in the area}[Magic vs. Fortitude]
    \spellsuccess As \spell{inflict light wounds}, except that it deals 5d6 negative energy damage \add d6 per four caster levels above 10th.
    \spellfailure As above, but half damage.
    \spelleffect If the target is undead, it is instead healed for that much damage.
\end{spelltargets}

\spellsection{Inflict Moderate Wounds}{2}
\spellrng{\rngclose}
\spellsr{Yes}
\spellinfo{Necro (Vitalism) [Negative]}{Arcane, Divine}
\begin{spelltarget}{One creature}[Magic vs. Fortitude]
    \spellsuccess If the target is not undead, this spell inflicts 4d6 negative energy damage \add d6 per two caster levels above 4th. For every 15 points of damage dealt in excess of the target's hit points, it can instead inflict 1 point of critical damage.
    \spellfailure As above, but half damage.
    \spelleffect If the target is undead, it is instead healed for that much damage.
\end{spelltarget}

\spellsectioncomma{Inflict Moderate Wounds}{Mass}{6}
\spellinfo{Necro (Vitalism) [Negative]}{Arcane, Divine}
\spelltwocol{\spelllimit{\areamed radius}}{\spellrng{\rngmed}}
\spellsr{Yes}
\begin{spelltargets}{Up to five creatures in the area}[Magic vs. Fortitude]
    \spellsuccess As \spell{inflict moderate wounds}, except that it deals 6d6 negative energy damage \add d6 per four caster levels above 12th.
    \spellfailure As above, but half damage.
    \spelleffect If the target is undead, it is instead healed for that much damage.
\end{spelltargets}

\spellsection{Inflict Serious Wounds}{3}
\spellinfo{Necro (Vitalism) [Negative]}{Arcane, Divine}
\spellrng{\rngclose}
\spellsr{Yes}
\begin{spelltarget}{One creature}[Magic vs. Fortitude]
    \spellsuccess If the target is not undead, this spell inflicts 6d6 negative energy damage \add d6 per two caster levels above 6th. For every 10 points of damage dealt in excess of the target's hit points, it can instead inflict 1 point of critical damage.
    \spellfailure As above, but half damage.
    \spelleffect If the target is undead, it is instead healed for that much damage.
\end{spelltarget}

\spellsectioncomma{Inflict Serious Wounds}{Mass}{7}
\spellinfo{Necro (Vitalism) [Negative]}{Arcane, Divine}
\spelltwocol{\spelllimit{\areamed radius}}{\spellrng{\rngmed}}
\spellsr{Yes}
\begin{spelltargets}{Up to five creatures in the area}[Magic vs. Fortitude]
    \spellsuccess As \spell{inflict serious wounds}, except that it deals 7d6 negative energy damage \add d6 per four caster levels above 14th.
    \spellfailure As above, but half damage.
    \spelleffect If the target is undead, it is instead healed for that much damage.
\end{spelltargets}

\spellsection{Insanity}{6}
\spellinfo{Ench (Compulsion) [Mind-Affecting]}{Arcane}
\spellrng{\rngclose}
\spelldur{Permanent}
\spellsr{Yes}
\begin{spelltarget}{One creature}[Magic vs. Will]
    \spellsuccess As long as the target is \bloodied, it is \confused.
\end{spelltarget}
\spellnotes \spell{Remove curse} and \spell{dispel magic} do not remove \spell{insanity}. \spell{Greater restoration}, \spell{heal}, \spell{limited wish}, \spell{miracle}, or \spell{wish} can restore the creature.

\spellsection{Interposing Hand}{2}
\spelldesc{You create a massive hand from thin air that blocks your foe's attacks.}
\spellinfo{Evoc (Control) [Force]}{Arcane}
\spellrng{\rngmed}
\spelldur{\durshort \dismissable}
\spellsr{Yes}
\spellline
\spelleffect This spell creates a floating, disembodied hand made of magical force. Each round, as a swift action, you can direct the hand to protect you from a target. If you do not direct the hand, it remains motionless.

The hand is 10 feet long and about that wide with its fingers outstretched. It has 3 hit points per caster level, an Armor defense of 10 \add half your caster level, and a Maneuver defense of 10 \add your caster level \add your casting attribute. Most effects that don't affect objects do not affect the hand.

\begin{spelltarget}*{One creature}
    \spelleffect The hand provides you with active cover from the target. Each physical attack the target makes against you has a 20\% chance to strike the hand instead. In addition, if the target is Large size or smaller, it moves at half speed while moving towards you.
\end{spelltarget}
\spellnotes The hand can move up to 60 feet per round. It never provokes attacks of opportunity from opponents. Since the hand is directed by you, its ability to interact with invisible or concealed creatures is no better than yours. Its special defenses are the same as your special defenses. If the hand goes out of range of you, it winks out.

\spellsection{Invest Magic}{4}
\spellinfo{Trans (Augment)}{Arcane, Divine, War}
\spellrng{\rngclose}
\spelldur{\durshort}
\spellsr{Yes}
\begin{spelltarget}{One creature}
    \spelleffect All weapons and armor that the target wields gain a \plus3 enhancement bonus for as long as it wields them. This bonus increases to \plus4 at 14th caster level, and to \plus5 at 20th caster level.
\end{spelltarget}

\spellsection{Invisibility}{3}
\spellinfo{Illus (Glamer)}{Arcane, Trickery}
\spellrng{\rngclose}
\spelldur{\durshort \dismissable}
\spellsr{Yes}
\begin{spelltarget}{One creature or object (Large or smaller)}
    \spelleffect The target and its equipment become invisible. An invisible creature cannot be seen, even by darkvision. Invisible creatures can be detected with the Perception skill (see \pcref{Perception}).

    If the target attacks any creature, such as by casting any spell that affects an unwilling creature, it becomes visible.
\end{spelltarget}
\spellnotes This spell can be made permanent (on objects only) with a \spell{permanency} ritual.

\spellsectioncomma{Invisibility}{Greater}{6}
\spellinfo{Illus (Glamer)}{Illus}
\spellrng{\rngclose}
\spelldur{\durshort \dismissable}
\spellsr{Yes}
\begin{spelltarget}{One creature or object (Large or smaller)}
    \spelleffect The target becomes invisible, as \spell{invisibility}. At the end of every round, if the target did not attack a creature that round, it becomes invisible again.
\end{spelltarget}

\spellsectioncomma{Invisibility}{Mass}{7}
\spellinfo{Illus (Glamer)}{Arcane, Trickery}
\spelltwocol{\spelllimit{\areamed radius}}{\spellrng{\rngmed}}
\spelldur{\durshort \dismissable}
\spellsr{Yes}
\begin{spelltargets}{Up to five creatures or objects in the area (Large or smaller)}
    \spelleffect The target becomes invisible, as \spell{invisibility}.
\end{spelltargets}

\spellsection{Iron Body}{8}
\spellinfo{Trans (Polymorph)}{Arcane, Earth, Strength}
\spelldur{\durshort \dismissable}
\begin{spelltarget}{You}
    \spelleffect This spell transforms your body into living iron, which grants you several powerful resistances and abilities.
    \par You gain physical damage reduction, reducing the physical damage you take each round by 16 \add 1 per caster level above 16th. If you are hit by an adamantine weapon, you cannot use your damage reduction for 1 round.
    \par You are immune to blindness, critical hits, attribute damage, deafness, disease, drowning, electricity, poison, stunning, and all spells or attacks that affect your physiology or respiration, because you have no physiology or respiration while this spell is in effect. You take only half damage from acid and fire of all kinds.
    \par You gain a \plus5 enhancement bonus to your Strength score, but you take a \minus5 penalty to Dexterity as well, and your speed is reduced to half normal. You have a \minus8 armor check penalty. You cannot drink (and thus can't use potions) or play wind instruments.
    \par Your unarmed attacks deal damage equal to a warhammer sized for you (1d6 for Small characters or 1d8 for Medium characters), and you are considered armed when making unarmed attacks.
    \par Your weight increases by a factor of ten, causing you to sink in water like a stone. However, you could survive the crushing pressure and lack of air at the bottom of the ocean -- at least until the spell duration expires.
\end{spelltarget}

\spellsection{Irresistible Dance}{9}
\spelldesc{You fill your enemy with an overpowering urge to dance and caper in place. Against its will, it begins doing so, complete with foot shuffling and tapping.}
\spellinfo{Ench (Compulsion) [Mind-Affecting]}{Arcane, Chaos}
\spellrng{\rngclose}
\spelldur{1 round}
\spellsr{Yes}
\begin{spelltarget}{One creature}
    \spelleffect The target is defenseless and must spend a standard action to do nothing but dance, which provokes attacks of opportunity.
\end{spelltarget}

\pdfbookmark[2]{J-L}{SpellDescriptionsJ-L}
\begin{comment}
\subsubsection{J-L}
\end{comment}

\spellsection{Knock}{2}
\spellinfo{Evoc (Control)}{Evoc}
\spellrng{\rngclose}
\spellsr{Yes}
\begin{spelltarget}{One object (Medium or smaller)}
    \spelleffect This spell telekinetically opens stuck, barred, locked, held, or arcane locked objects. If the target object is stuck or held, you can immediately make an Strength check to break it open, using your caster level instead of your Strength. Others can aid you on this check as normal. In addition, if the target object is locked, you can immediately make a Disable Device check to open the lock as if you had rolled a 20 on the check. You get an enhancement bonus on the Disable Device check equal to half your caster level.
\end{spelltarget}
\spellnotes If knock is cast on an \spellindirect{arcane lock}{arcane locked} door, make a caster level check against a DC of 11 \add the caster level of the \spell{arcane lock}. If you succeed, the \spell{arcane lock} is suppressed for 10 minutes. If you fail, you may still bypass the door with the checks above, if possible.

\invocationsection{Lesser (Spell Name)}
\par Any spell whose name begins with lesser is alphabetized in this chapter according to the second word of the spell name. Thus, the description of a lesser spell appears near the description of the spell on which it is based. Spell chains that have lesser spells in them include those based on the spells cone of cold, dispel magic, moment of prescience, precognition, and spelltheft.

\spellsection{Levitate}{3}
\spellinfo{Evoc (Control)}{Arcane}
\spellrng{\rngclose}
\spelldur{\durshort \dismissable}
\spellsr{Yes}
\begin{spelltarget}{One object or willing creature (total weight up to 100 lb./caster level)}
    \spelleffect You can telekinetically move the target vertically. A creature must be willing to be levitated, and an object must be unattended or possessed by a willing creature. As a swift action, you can mentally direct the target to move up or down as much as 20 feet each round. You cannot move the recipient horizontally, but the recipient could clamber along the face of a cliff, for example, or push against a ceiling to move laterally (generally at half its land speed).
\end{spelltarget}

\spellsection{Lifebreaker Curse}{7}
\spelldesc{You permanently cripple your foe's life force.}
\spellinfo{Necro (Life) [Curse]}{Necro}
\spellrng{\rngmed}
\spelldur{Permanent}
\spellsr{Yes}
\begin{spelltarget}{One living creature}[Magic vs. Will]
    \spellsuccess 12d6 life damage \add d6 per two caster levels above 12th. The target's maximum hit points are reduced by the amount of life damage it takes from this attack (to a minimum of 1).
    \spellfailure As above, but half damage, and the target's maximum hit points are not reduced.
\end{spelltarget}
\spellnotes If this spell is cast multiple times on the same target, the reduction of maximum hit points does not stack. Only the largest reduction applies.

\cursespellnotes

\spellsection{Lifeseeking Missile}{3}
\spellinfo{Evoc/Necro (Control, Life) [Force]}{Arcane}
\spelltwocol{\spelllimit{\areamed radius}}{\spellrng{\rngmed}}
\spellsr{Yes}
\spellline
\spellspecial You create three missiles \add one missile per four caster levels above 6th. Each missile can deal d10 force damage to a single creature.

Any missiles you do not explicitly target will automatically strike a living creature within the area. The missiles are able to unerringly strike creatures you cannot see or are not aware of, including invisible or concealed creatures. You can direct the missiles to avoid specific targets, allowing you to strike a hidden foe among your allies.
\begin{spelltargets}*{Any number of creatures in the area}
    \spelleffect The target is struck by as many missiles as you choose.
\end{spelltargets}

\spellsection{Lightning Bolt}{3}
\spellinfo{Evoc (Energy) [Destructive, Electricity]}{Arcane, Destruction, Nature}
\spellburst{\arealarge line, 10 ft. wide}
\spellsr{Yes}
\begin{spelltarget}{Everything in the area}[Magic vs. Reflex]
    \spellsuccess 3d6 electricity damage \add d6 per four caster levels above 6th.
    \spellfailure As above, but half damage.
\end{spelltarget}
\spellnotes \destructivespellnotes

\spellsection{Limited Wish}{7}
\spellinfo{Universal}{Arcane}
\spellcmp{Verbal, Somatic, and Material}
\spelltwocol{\spelltgteffarea{See text}}{\spellrng{See text}}
\spelldur{See text}
\spelleffect A limited wish lets you create nearly any type of effect. For example, a limited wish can do any of the following things.
\begin{itemize}
    \item Duplicate any general sorcerer/wizard spell of 6th level or lower, provided the spell is not of a school prohibited to you.
    \item Duplicate any general sorcerer/wizard spell of 5th level or lower, even if it's of a prohibited school.
    \item Duplicate any other spell of 4th level or lower, provided the spell is not of a school prohibited to you.
    \item Duplicate any other spell of 3rd level or lower, even if it's of a prohibited school.
    \item Undo the harmful effects of many spells, such as geas/quest or insanity.
    \item Produce any other effect whose power level is in line with the above effects, such as a single creature automatically hitting on its next attack or taking a \minus5 penalty to its defenses for 5 rounds.
\end{itemize}
\par When casting a limited wish, you do not specify the exact spell or effect you wish to duplicate. Instead, you make a wish, describing what you want to have happen, and make a DC 15 Wisdom check. If the check fails, your intent is redirected or perverted in some way. For example, a \spell{limited wish} to turn a foe to stone would normally mimic the \spell{flesh to stone} effect of the \spell{transmute flesh and stone} spell. However, if the Wisdom check failed, your foe might gain the benefit of a \spell{stoneskin} spell instead.
\par When a limited wish spell duplicates a spell with a material component that costs more than 1,000 gp, you must provide that component (in addition to the 1,000 gp cost for this spell).
\spellmat{A diamond worth no less than 1,000 gp (see above).}
\spellsr{Yes}

\spellsection{Link Vitality}{3}
\spellinfo{Necro (Life)}{Necro}
\spelllimit{\areamed radius centered on you}
\spelltgts{Two living creatures in the area}
\spelldur{\durshort}
\spellattack{Magic vs. Will}
\spellsr{Yes}
\spellspecial This spell has no effect unless the attack succeeds against both targets.
\spellsuccess Whenever one target gains or loses hit points, the other target also gains or loses the same amount of hit points.
\spellnotes The loss of hit points caused by this spell is not damage, and is not affected by damage reduction or other abilities which affect damage.

\spellsectioncomma{Link Vitality}{Mass}{6}
\spellinfo{Necro (Life)}{Arcane}
\spelllimit{\areamed radius centered on you}
\spelldur{\durshort}
\spellsr{Yes}
\begin{spelltargets}{Up to five living creatures in the area}
    \spellspecial This spell has no effect unless the attack succeeds against at least two targets.
    \spellsuccess Whenever one target gains or loses hit points, all other affected targets also gain or lose the same amount of hit points.
\end{spelltargets}

\spellsection{Living Projectile}{3}
\spelldesc{You telekinetically fling an ally at great speed towards your foe.}
\spellinfo{Abjur/Evoc (Control, Shielding)}{Arcane}
\spellrng{\rngclose}
\spelltgt{One willing creature}[Primary]
\spelltgt{One creature or object}[Secondary]
\spelldur{1 round}
\spellsr{Yes (primary target), No (secondary target)}
\spellline
\spelleffect The primary target is moved adjacent to the secondary target, gains the benefit of the \spell{ablate impact} spell for 1 round, and is knocked prone.
\spellattack{Caster level \add casting attribute vs. Armor defense (secondary target)}
\begin{spellmargin}
    \spellsuccess The secondary target takes 6d6 damage \add d6 per two caster levels above 6th. The primary target takes half this amount of damage.
    \spellfailure As above, but half damage to the secondary target. The damage taken by the primary target is not halved.
\end{spellmargin}

\spellsection{Locate Entity}{6}
\spellinfo{Div (Awareness) [Detection]}{Arcane, Knowledge}
\spellrng{\rngext}
\spelldur{\durlong \dismissable}
\spelleffect This spell functions as \spell{locate object}, except that it can also detect creatures, as \spell{locate creature}. When you cast this spell, you choose to locate an object or creature, following the restrictions stated in the respective location spells.

\spellsection{Locate Creature}{2}
\spellinfo{Div (Awareness) [Detection]}{Arcane, Divine}
\spellrng{\rnglong}
\spelldur{\durmed \dismissable}
\spelleffect You sense the direction of a well-known or clearly visualized creature if it is within the spell's range. You can search for general creatures based on visual characteristics (such as ``pointy ears'' or ``looking human''), in which case you locate the nearest one of its kind if more than one is within the range. Attempting to find a certain creature requires a specific and accurate mental image of a distinguishing visual characteristic, such as its clothes or face; if the image is not close enough to the actual creature, the spell fails.
\spellnotes A detection spell can penetrate barriers, but 1 foot of stone, 1 inch of common metal, a thin sheet of lead, or 3 feet of wood or dirt blocks it.

\spellsectioncomma{Locate Creature}{Greater}{4}
\spellinfo{Div (Awareness) [Detection]}{Arcane, Divine, Knowledge}
\spellrng{\rngext}
\spelldur{\durmed \dismissable}
\spelleffect As \spell{locate creature}, except that it detects creatures within \rngext range. In addition, you detect all appropriate creatures within the range, rather than only the nearest creature.

\spellsection{Locate Object}{1}
\spellinfo{Div (Awareness) [Detection]}{Arcane, Divine}
\spellrng{\rnglong}
\spelldur{\durmed \dismissable}
\spelleffect You sense the direction of a well-known or clearly visualized object if it is within the spell's range. You can search for general items, in which case you locate the nearest one of its kind if more than one is within the range. Attempting to find a certain item requires a specific and accurate mental image; if the image is not close enough to the actual object, the spell fails.
\spellnotes A detection spell can penetrate barriers, but 1 foot of stone, 1 inch of common metal, a thin sheet of lead, or 3 feet of wood or dirt blocks it.

\spellsectioncomma{Locate Object}{Greater}{3}
\spellinfo{Div (Awareness) [Detection]}{Arcane, Divine, Knowledge}
\spellrng{\rngext}
\spelldur{\durmed \dismissable}
\spelleffect This spell functions like \spell{locate object}, except that it detects objects within \rngext range. In addition, you detect all appropriate objects within the range, rather than only the nearest object. 

\spellsection{Longstrider}{1}
\spellinfo{Trans (Augment)}{Nature, Travel}
\spelldur{\durlong \dismissable}
\begin{spelltarget}{You}
    \spelleffect You gain a \plus10 enhancement bonus to your land speed. This has no effect on other modes of movement, such as burrow, climb, fly, or swim.
\end{spelltarget}

\pdfbookmark[2]{M}{SpellDescriptionsM}
\begin{comment}
\subsubsection{M}
\end{comment}

\spellsection{Mage Armor}{1}
\spelldesc{You create an invisible but tangible field of force that surrounds you, protecting you from attacks.}
\spellinfo{Abjur (Shielding) [Force]}{Arcane}
\spellrng{\rngclose}
\spelldur{\durshort or \durlong; see text \dismissable}
\spellsr{Yes}
\begin{spelltarget}{One creature}
    \spelleffect As you cast this spell, you choose whether to create body armor or a shield. If you choose body armor, the target gains a \plus2 armor modifier. If you choose a shield, the target gains a \plus2 shield modifier. The bonus granted increases to \plus3 at 8th caster level, to \plus4 at 14th caster level, and finally to \plus5 at 20th caster level. 
    \par Unlike mundane armor, \spell{mage armor} entails no armor check penalty, arcane spell failure chance, or speed reduction.
    \par If you are the target, this spell lasts for \durlong duration. On any other creature, it lasts for \durshort duration.
\end{spelltarget}
\spellnotes If you cast this spell on the same creature twice, you can grant the creature both body armor and a shield. The armor created by this spell is treated as a separate piece or armor from any other armor the creature is wearing, so it does not stack with any existing bonuses. Since \spell{mage armor} is made of force, incorporeal creatures can't bypass it the way they do normal armor.

\spellsection{Mage Hand}{1}
\spellinfo{Evoc (Control)}{Arcane}
\spellrng{\rngclose}
\spelldur{\durshort}
\begin{spelltarget}{One nonmagical, unattended object weighing up to 5 lb.}
    \spelleffect You point your finger at an object and can lift it and move it in any direction from a distance. By directing the spell as a swift action, you can propel the object as far as 15 feet in any direction each round, though the spell ends if the distance between you and the object ever exceeds the spell's range.
\end{spelltarget}
\spellnotes Fine manipulation, including any motion other than simply moving the object in a particular direction, is not possible with this spell.

\spellsection{Mage's Disjunction}{9}
\spellinfo{Abjur (Negation) [Magic]}{Arcane, Magic}
\spelltwocol{\spelltgtorarea{One magic item or \areamed radius burst}}{\spellrng{\rngmed}}
\spellspecial This spell has two versions: an area dispel, and a targetted destruction of a magic item. Its effects depend on which version is chosen.
\spellline
\spellburst{\areamed radius}
\begin{spelltargets}*{All spells in the area}
    \spelleffect The target spell is dispelled. If you cast the target spell, you may choose not to dispel it.
\end{spelltargets}
\begin{spelltarget}{One magic item}[Caster level vs. Special]
    \spellspecial The attack is made against a DC equal to 10 \add the caster level of the item.

    If the item is an artifact, there is only a 1\% chance per caster level that the spell works. If you destroy an artifact in this way, you permanently lose the ability to cast \spell{mage's disjunction}.
    \spellsuccess The target item is permanently rendered nonmagical.
    \spellfailure The target item is suppressed for 5 rounds. A suppressed object loses all its magical abilities, though it is still treated as being a magical object for the purpose of spells and effects.
\end{spelltarget}
\spellnotes Destroying artifacts is dangerous, and it is likely to attract the attention of some powerful being who has an interest in or connection with the device.

\spellsection{Magic Circle against Alignment}{5}
\spellinfo{Abjur (Interdiction) [Barrier, Good]}{Arcane, Chaos, Divine, Evil, Good, Law}
\spellrng{Touch}
\spellfocus{One creature}
\spellemanation{\areasmall radius from the focus}
\spelldur{\durshort \dismissable}
\spellsr{Yes}
\spellspecial When you cast this spell, choose an alignment (chaotic, good, lawful, or evil).
\spellline
\spelleffect Summoned creatures which have the chosen alignment cannot enter the area.
\begin{spelltargets}*{All creatures in the area}
    \spelleffect The target is protected from the chosen alignment, as \spell{protection from alignment}.
\end{spelltargets}

\spellsection{Magic Fang}{2}
\spellinfo{Trans (Augment)}{Nature}
\spellrng{\rngclose}
\spelldur{\durshort}
\spellsr{Yes}
\begin{spelltarget}{One creature}
    \spelleffect One of the target's natural weapons gains a \plus2 enhancement bonus to attack and damage. \spellbonusscalingdescription
\end{spelltarget}
\spellnotes This spell can be cast multiple times on the same creature. Each time, you can choose a new natural weapon. It does not change an unarmed strike's damage from nonlethal damage to lethal damage.

This spell can be made permanent with a \spell{permanency} ritual.

\spellsectioncomma{Magic Fang}{Greater}{4}
\spellinfo{Trans (Augment)}{Nature}
\spellrng{\rngclose}
\spelldur{\durshort}
\spellsr{Yes}
\begin{spelltarget}{One creature}
    \spelleffect All of the target's natural weapons gain a \plus3 enhancement bonus to attack and damage. This bonus increases to \plus4 at 14th caster level, and to \plus5 at 20th caster level.
\end{spelltarget}
\spellnotes This spell can be made permanent with a \spell{permanency} ritual.

\spellsection{Magic Missile}{1}
\spellinfo{Evoc (Control) [Force]}{Arcane}
\spelltwocol{\spelllimit{\areamed radius}}{\spellrng{\rngclose}}
\spellsr{Yes}
\spellline
\spellspecial You create two missiles \add one missile per two caster levels above 2nd. Each missile can deal d4 force damage to a single creature.
\begin{spelltargets}*{Any number of creatures in the area}
    \spelleffect The target is struck by as many missiles as you choose.
\end{spelltargets}

\spellsection{Magic Vestment}{1}
\spellinfo{Trans (Augment)}{Arcane, Divine}
\spellrng{\rngclose}
\spelldur{\durmed}
\spellsr{Yes}
\begin{spelltarget}{One suit of armor or shield}
    \spelleffect The target gains a \plus2 enhancement bonus, increasing the defense bonus it provides. \spellbonusscalingdescription
\end{spelltarget}
\spellnotes An outfit of regular clothing counts as armor that grants no armor defense bonus for the purpose of this spell.

\spellsection{Magic Weapon}{2}
\spellinfo{Trans (Augment)}{Arcane, Divine}
\spellrng{\rngclose}
\spelldur{\durshort}
\spellsr{Yes}
\begin{spelltarget}{One weapon or fifty projectiles (in a single group)}
    \spelleffect The target gains a \plus2 enhancement bonus to attack and damage. \spellbonusscalingdescription
\end{spelltarget}
\spellnotes You can't cast this spell on a natural weapon, such as an unarmed strike (instead, see \spell{magic fang}). A monk's unarmed strike is considered a weapon, and thus it can be enhanced by this spell.
\par If you use this spell to enhance projectiles, the projectiles must be of the same kind, and they have to be together (in the same quiver or other container). Projectiles, but not thrown weapons, lose their transmutation when used. (Treat darts and shuriken as projectiles, rather than as thrown weapons, for the purpose of this spell.)

\spellsection{Major Image}{4}
\spellinfo{Illus (Figment) [Unreal]}{Illus}
\spelltwocol{\spellzone{\arealarge radius}}{\spellrng{\rngmed}}
\spelldur{\durshort}
\spellsr{No}
\spellline
\spelleffect A figment of your design appears within the area, as \spell{silent image}, except that sound, smell, and thermal elements are included.
\spellnotes Creatures can identify the illusion, as \spell{silent image}.

\spellsection{Manipulate Probability}{5}
\spellinfo{Div (Knowledge)}{Div, Knowledge}
\spellrng{\rngmed}
\spelldur{\durshort or until expended}
\spellsr{Yes}
\spellspecial When you cast this spell, you roll a d20 three times. Store the results. At 10th caster level, and every 5 caster levels thereafter, you roll an additional die when you cast this spell.
\begin{spelltarget}{One creature}[Magic vs. Will]
    \spellsuccess Your results replace the target's die results, as \spell{foresee probability}, except that you choose the order in which the die results are used. In addition, you may choose to let the creature roll for itself rather than use any of your stored results.
\end{spelltarget}
\spellnotes As \spell{foresee probability}.

\invocationsection{Mass (Spell Name)}
\par Any spell whose name begins with mass is alphabetized in this chapter according to the second word of the spell name. Thus, the description of a mass spell appears near the description of the spell on which it is based. Spell chains that have mass spells in them include those based on the spells charm monster, cure critical wounds, cure light wounds, cure moderate wounds, cure serious wounds, enlarge person, heal, hold monster, hold person, inflict critical wounds, inflict light wounds, inflict moderate wounds, inflict serious wounds, invisibility, reduce person, suggestion, totemic mind, and totemic power.

\spellsection{Maze}{8}
\spellinfo{Conj (Translocation) [Planar, Teleportation]}{Conj, Trickery}
\spellrng{\rngmed}
\spelldur{Instantaneous; see text}
\spellsr{Yes}
\begin{spelltarget}{One creature}[Magic vs. Will]
    \spellsuccess The target is teleported into an extradimensional labyrinth of force planes. Each round, as a full-round action, it may attempt a DC 20 Intelligence check to escape the labyrinth. If the target doesn't escape, the maze disappears after 5 minutes, forcing the target back to the location where it was originally banished.
    \spellfailure As above, but the DC of the Intelligence check to escape is 10.
\end{spelltarget}
\spellnotes Spells and abilities that move a creature within a plane, such as \spell{teleport} and \spell{dimension door}, do not help a creature escape a \spell{maze} spell, although a \spell{plane shift} spell allows it to exit to whatever plane is designated in that spell. Minotaurs can escape the spell automatically.

When leaving the maze, the target reappears where it had been when the maze spell was cast. If this location is filled with a solid object, the target appears in the nearest open space.

\norepeatspellnotes

\spellsection{Meld into Stone}{3}
\spellinfo{Trans (Polymorph) [Earth]}{Earth, Nature}
\spelldur{\durlong}
\begin{spelltarget}{You}
    \spelleffect You can meld your body and possessions into a single block of stone. The stone must be large enough to accommodate your body in all three dimensions. When the casting is complete, you and not more than 100 pounds of nonliving gear merge with the stone. If either condition is violated, the spell fails and is wasted.
    \par While in the stone, you remain in contact, however tenuous, with the face of the stone through which you melded. You remain aware of the passage of time and can cast spells on yourself while hiding in the stone. Nothing that goes on outside the stone can be seen, but you can still hear what happens around you. Minor physical damage to the stone does not harm you, but its partial destruction (to the extent that you no longer fit within it) expels you and deals you 5d6 points of damage. If the stone is completely destroyed, you are expelled, and you die unless your Fortitude defense is at least 20.
    \par At any time before the duration expires, you can step out of the stone through the surface that you entered. If the spell's duration expires or the effect is dispelled before you voluntarily exit the stone, you are violently expelled and take 5d6 points of damage.
\end{spelltarget}
\spellnotes The following spells harm you if cast upon the stone that you are occupying: \spell{transmute flesh and stone} expels you and deals 6d6 points of damage. \spellindirect{shape stone}{Shape stone} deals 3d6 points of damage but does not expel you. \spell{Passwall} expels you without damage.

\spellsection{Message}{1}
\spellinfo{Div (Communication)}{Arcane}
\spellcmp{Somatic only}
\spelldur{\durlong}
\spellline
\spelleffect Whenever you whisper, you may cause other creatures to hear the message.
\spellrng{\rngmed}
\begin{spelltargets}*{Up to five creatures}
    \spelleffect The target hears what you whispered as if you were whispering in its ear.
\end{spelltargets}
\spellnotes This is not telepathic communication, and observers can still read your lips. Very close observers may also hear the message.

\spellsection{Meteor Swarm}{9}
\spelldesc{You call a swarm of meteors that streak down from the heavens, leaving a fiery trail behind them. The meteors crash into your foes, driving flying creatures to the ground and knocking creatures off their feet.}
\spellinfo{Evoc (Energy) [Fire]}{Destruction, Evoc, Fire}
\spellrng{\rnglong}
\spellburst{\arealarge radius cylinder, 100 ft. high}
\spellsr{Yes}
\begin{spelltargets}{Everything in the area}[Magic vs. Reflex]
    \spellsuccess 9d6 fire damage \add d6 per four caster levels above 18th. If the target is on the ground, it falls prone.

    If the target is in the air, and is Gargantuan or smaller, it is driven to the ground. It takes falling damage as appropriate for the distance descended.
    \spellfailure As above, but half damage, and the target is not knocked prone or driven to the ground.
\end{spelltargets}
\spellnotes \firespellnotes

\destructivespellnotes

\spellsection{Minor Image}{3}
\spellinfo{Illus (Figment) [Unreal]}{Illus}
\spelltwocol{\spellzone{\areamed radius}}{\spellrng{\rngmed}}
\spelldur{\durshort}
\spellsr{No}
\spellline
\spelleffect A figment of your design appears within the area, as \spell{silent image}, except that sound elements are included.
\spellnotes Creatures can identify the illusion, as \spell{silent image}.

\spellsection{Miracle}{9}
\spellinfo{Evoc (Channeling)}{Divine}
\spelltwocol{\spelltgteffarea{See text}}{\spellrng{See text}}
\spelldur{See text}
\spellattack{See text}
\spelleffect You don't so much cast a miracle as request one. You state what you would like to have happen and request that your deity (or the power you pray to for spells) intercede.
\par A miracle can do any of the following things.
\begin{itemize}
    \item Duplicate any cleric spell of 8th level or lower (including spells to which you have access because of your domains). 
    \item Duplicate any other spell of 7th level or lower.
    \item Undo the harmful effects of certain spells, such as feeblemind or insanity.
    \item Have any effect whose power level is in line with the above effects.
\end{itemize}
\par Alternatively, a cleric can make a very powerful request. Examples of especially powerful miracles of this sort could include the following.
\begin{itemize}
    \item Swiinging the tide of a battle in your favor by raising fallen allies to continue fighting.
    \item Moving you and your allies, with all your and their gear, from one plane to another through planar barriers to a specific locale with no chance of error.
    \item Protecting a city from an earthquake, volcanic eruption, flood, or other major natural disaster.
\end{itemize}
\par In any event, a request that is out of line with the deity's (or alignment's) nature is refused.
\spellnotes If you request a miracle, your deity (or the power you pray to) will expect something of you in return. You must cast \spell{commune}commune to learn what this is within 24 hours, or you will lose the ability to cast any cleric spells other than \spell{commune}. For more moderate miracles, you may be required to offer 25,000gp worth of incense and gems. For especially powerful miracles, or multiple moderate miracles, you may geased with a task to complete.
\par When a miracle spell duplicates a spell with a material component that costs more than 5,000 gp, you must provide that component.
\spellsr{Yes}

\spellsection{Mirror Image}{2}
\spelldesc{You create illusory duplicates of yourself that make it difficult for enemies to know which image to attack.}
\spellinfo{Illus (Figment)}{Arcane}
\spelldur{\durshort or until expended \dismissable}
\begin{spelltarget}{You}
    \spelleffect Illusory duplicates mirror your every move. You gain four images \add one per four caster levels above 2nd. For each image, you gain a \plus1 bonus to your physical defenses. At the end of each round, you lose one image for each physical attack that missed you that round. If you run out of images, the spell is expended.
\end{spelltarget}
\spellnotes If you are invisible, this spell has no effect. If an creature is unable to see the images, such as by closing their eyes or with the \spell{true seeing} spell, you gain no bonus to your defenses against that creature's attacks.

\spellsectioncomma{Mirror Image}{Greater}{5}
\spellinfo{Illus (Figment)}{Arcane}
\spelldur{\durshort \dismissable}
\begin{spelltarget}{You}
    \spelleffect You gain illusory duplicates, as \spell{mirror image}, except that the spell is not expended when you run out of images. At the end of each round, you gain two additional images, up to the number of images created when the spell is first cast.
\end{spelltarget}
\spellnotes As \spell{mirror image}.

\spellsection{Mislead}{6}
\spellinfo{Illus (Figment, Glamer) [Unreal]}{Arcane, Trickery}
\spelldur{\durshort \dismissable}
\begin{spelltarget}{You}
    \spelleffect You become invisible, as \spell{invisibility}. At the same time, an illusory double of you appears, as \spell{major image}.

    You can control the image of yourself as you would control any other figment with \spell{major image}. If not directed, it remains stationary.
\end{spelltarget}

\spellsection{Missile Storm}{7}
\spelldesc{You unleash an immense swarm of missiles which seek out and destroy all of your foes.}
\spellinfo{Evoc (Control) [Force]}{Arcane}
\spelllimit{\arealarge radius centered on you}
\spellsr{Yes}
\begin{spelltargets}{Any number of creatures in the area}
    \spelleffect 7d4 force damage \add d4 per four caster levels above 14th.
\end{spelltargets}

\spellsection{Moment of Prescience}{7}
\spellinfo{Div (Knowledge)}{Arcane, Div, Knowledge}
\spelldur{\durext or until discharged}
\begin{spelltarget}{You}
    \spelleffect As \spell{lesser moment of prescience}, except that you also gain a bonus equal to half your caster level on the roll.

    Alternately, you can dischage the spell to protect yourself. As an immediate action, when you are attacked by a physical attack, you can gain a bonus to your dodge modifier equal to half your caster level. Unlike normal, you can still take this immediate action even if if you are unaware of the attack. Doing so makes you aware of the attack, allowing you to defend yourself normally.
\end{spelltarget}
\spellnotes As \spell{lesser moment of prescience}.

\spellsectioncomma{Moment of Prescience}{Greater}{9}
\spellinfo{Div (Knowledge)}{Div}
\spelldur{\durext or until discharged}
\begin{spelltarget}{You}
    \spelleffect As \spell{moment of prescience}, except that the bonus and extra rolls apply to all of your physical attacks, opposed checks, and defenses until the beginning of your next turn.
\end{spelltarget}
\spellnotes As \spell{lesser moment of prescience}.

\spellsectioncomma{Moment of Prescience}{Lesser}{4}
\spelldesc{You gain a powerful sixth sense in relation to yourself.}
\spellinfo{Div (Knowledge)}{Arcane, Div, Knowledge}
\spelldur{\durext or until discharged}
\begin{spelltarget}{You}
    \spelleffect As an immediate action, when you make a single physical attack or opposed check, you can roll twice and take the result you prefer. Once activated once, the spell is discharged.
\end{spelltarget}
\spellnotes You can't have more than one \spellindirect{lesser moment of prescience}{moment of prescience} effect active on you at the same time.

\pdfbookmark[2]{O-P}{SpellDescriptionsO-P}
\begin{comment}
\subsubsection{O-P}
\end{comment}

\spellsection{Obscuring Mist}{1}
\spelldesc{You conjure a bank of fog that arises around you, concealing you and your allies.}
\spellinfo{Conj (Creation) [Fog]}{Arcane, Divine, Nature, Water}
\spellzone{\areamed radius cylinder centered on you}
\spelldur{\durshort}
\spellsr{No}
\spellline
\spelleffect Fog blocks sight in the area, as \spell{fog cloud}.
\spellnotes As \spell{fog cloud}.

\spellsection{Order's Wrath}{4}
\spellinfo{Evoc (Channeling) [Lawful]}{Law}
\spellrng{\rngmed}
\spelldur{5 rounds}
\spellsr{Yes}
\begin{spelltarget}{One nonlawful creature}
    \spellsuccess 8d6 divine damage \add d6 per two caster levels above 8th, and the target is \bewildered for 5 rounds.
    \spellfailure As above, but half damage.
\end{spelltarget}

\spellsection{Persistent Image}{6}
\spellinfo{Illus (Figment)}{Illus}
\spelltwocol{\spellzone{\arealarge radius}}{\spellrng{\rngmed}}
\spelldur{\durmed \dismissable}
\spellline
\spelleffect A figment of your design appears within the area, as \spell{silent image}, except that sound, smell, and thermal elements are included. When you cast the spell, you set a script for the figment to follow. It follows that script without you having to concentrate on the spell.
\spellnotes Creatures can identify the illusion, as \spell{silent image}.

\spellsection{Phantasmal Killer}{4}
\spelldesc{You create a phantasmal image of the most fearsome creature imaginable to your foe simply by forming the fears of its subconscious mind into something that its conscious mind can visualize: this most horrible beast.}
\spellinfo{Ench/Illus (Emotion, Phantasm) [Death, Fear, Mind-Affecting]}{Arcane, Trickery}
\spellrng{\rngmed}
\spellsr{Yes}
\begin{spelltarget}{One creature}[Magic vs. Will and Fortitude]
    \spellsuccess[Will] The target is \shaken for 5 rounds.

    \spellsuccess[Will and Fortitude] If the target is \bloodied, it loses all its hit points and takes critical damage equal to your caster level, causing it to begin dying.
\end{spelltarget}

\spellsection{Phantasmal Maze}{5}
\spelldesc{You manipulate a foe's perceptions, causing it to believe that it is trapped in a labyrinth.}
\spellinfo{Illus (Phantasm)}{Arcane, Trickery}
\spellrng{\rngmed}
\spelldur{\durmed}
\spellsr{Yes}
\begin{spelltarget}{One creature}[Magic vs. Will]
    \spellsuccess The target perceives itself to be banished to an extradimensional labyrinth of force planes, as the \spell{maze} spell. It cannot see or hear anything to the contrary, causing it to be treated as if blinded and deafened for most purposes. However, it can still see and hear itself. Typically, this means the target moves in a random direction each round to escape the maze.

    If the target encounters any physical resistance in its movements or takes any damage, you must make another attack to maintain the effect. Failure means the target disbelieves the phantasm, ending the spell.
\end{spelltarget}

\spellsection{Phantasmal Wound}{2}
\spelldesc{You manipulate a foe's perceptions, causing it to believe that it is grievously wounded.}
\spellinfo{Illus (Phantasm)}{Arcane}
\spellrng{\rngmed}
\spelldur{\durshort}
\spellsr{Yes}
\begin{spelltarget}{One creature}[Magic vs. Will]
    \spellsuccess The target is \sickened.

    As long as the target is \bloodied, it also perceives itself to have no hit points remaining. It is \staggered, and may try to heal itself or take other appropriate actions. If its hit points are altered, such as by damage or healing, the creature disbelieves the effect automatically. It is immune to this effect for the rest of the duration, though it remains sickened.
\end{spelltarget}

\spellsection{Poison}{4}
\spelldesc{Calling upon the venomous powers of natural predators, you infect your foe with a horrible poison that drains its life force.}
\spellinfo{Necro (Flesh) [Poison]}{Death, Divine, Nature}
\spellrng{\rngclose}
\spelldur{5 minutes \undispellable}
\spellsr{Yes}
\begin{spelltarget}{One creature}
    \spelleffect The target is poisoned. At the end of every round, you make an attack against it to determine the poison's effects.
\end{spelltarget}
\begin{spelltrigger}{End of every round until the poison is resisted}
    \begin{spelltarget}{The poisoned creature}[Magic vs. Fortitude]
        \spellsuccess If this is the first successful attack, the target is \sickened. If this is the second successful attack, the target is \staggered. If this is the third successful attack, the target is \paralyzed.
        \spellfailure If this is the second failed attack, the target resists the poison. No further attacks are made, though the effects of any previous attacks linger until the end of the spell.
    \end{spelltarget}
\end{spelltrigger}

\spellsection{Polar Ray}{8}
\spelldesc{You fire a blue-white ray of frigid air and ice, freezing your foe in place.}
\spellinfo{Evoc (Energy) [Cold]}{Arcane, Water}
\spellrng{\rngclose}
\spelldur{5 rounds}
\spellsr{Yes}
\begin{spelltarget}{One creature or object}[Magic vs. Reflex]
    \spellsuccess 16d6 cold damage \add d6 per three caster levels above 16th. In addition, the target is \slowed for 5 rounds.

    If the target is \bloodied after the damage is dealt, it is also \paralyzed for 5 rounds.
\end{spelltarget}

\spellsection{Power Word Blind}{7}
\spellinfo{Necro (Flesh)}{Arcane}
\spellcmp{Verbal only}
\spellrng{\rngclose}
\spelldur{\durshort}
\spellsr{Yes}
\begin{spelltarget}{One creature}
    \spelleffect The target is \sickened.

    If the target is \bloodied, it is instead \blinded.
\end{spelltarget}

\spellsection{Power Word Confuse}{6}
\spellinfo{Ench (Compulsion) [Mind-Affecting]}{Arcane}
\spellcmp{Verbal only}
\spellrng{\rngclose}
\spelldur{\durshort}
\spellsr{Yes}
\begin{spelltarget}{One creature}
    \spelleffect The target is \bewildered.

    If the target is \bloodied, it is instead \confused.
\end{spelltarget}

\spellsection{Power Word Kill}{9}
\spelldesc{You utter a single word of power that instantly kills your foe, whether it can hear the word or not.}
\spellinfo{Necro (Life) [Death]}{Arcane, Death}
\spellcmp{Verbal only}
\spellrng{\rngclose}
\spelldur{\durshort}
\spellsr{Yes}
\begin{spelltarget}{One living creature}
    \spelleffect The target is \sickened.

    If the target is \bloodied, and its level does not exceed your caster level, it dies.
\end{spelltarget}

\spellsection{Power Word Stun}{8}
\spelldesc{You utter a single word of power that instantly causes your foe to become stunned, whether the creature can hear the word or not.}
\spellinfo{Ench (Inhibition) [Mind-Affecting]}{Arcane}
\spellcmp{Verbal only}
\spellrng{\rngclose}
\spelldur{\durshort}
\spellsr{Yes}
\begin{spelltarget}{One creature}
    \spelleffect The target is \bewildered.

    If the target is \bloodied, it is instead \stunned.
\end{spelltarget}

\spellsectioncomma{Precognition}{Lesser}{2}
\spelldesc{You extend your mind a fraction of a second into the future, allowing you to strike at your foes more effectively.}
\spellinfo{Div (Knowledge)}{Arcane, Div}
\spelldur{\durshort \dismissable}
\begin{spelltarget}{You}
    \spelleffect You gain a \plus2 enhancement bonus to your physical attack and damage rolls. \spellbonusscalingdescription
\end{spelltarget}

\spellsection{Precognition}{5}
\spelldesc{You extend your mind a fraction of a second into the future, allowing you to strike at your foes more effectively and avoid hostile attacks more easily.}
\spellinfo{Div (Knowledge)}{Arcane, Div}
\spelldur{\durshort \dismissable}
\begin{spelltarget}{You}
    \spelleffect You gain a \plus3 enhancement bonus to your physical attacks and damage rolls, special defenses, and dodge defense modifier. This bonus increases to \plus4 at 14th caster level, and to \plus5 at 20th caster level.
\end{spelltarget}

\spellsectioncomma{Precognition}{Greater}{8}
\spelldesc{You extend your mind a short time into the future, allowing you to strike at your foes more effectively and avoid hostile attacks more easily.}
\spellinfo{Div (Knowledge)}{Arcane, Div}
\spelldur{\durshort \dismissable}
\begin{spelltarget}{You}
    \spelleffect You gain combat bonuses, as \spell{precognition}. In addition, In addition, when making a standard attack, you may make an additional attack at a \minus5 penalty.
\end{spelltarget}

\spellsection{Prismatic Beam}{3}
\spellinfo{Universal [Light]}{Arcane}
\spellrng{\rngmed}
\spellsr{Yes}
\begin{spelltarget}{One creature}
    \spellspecial The target is struck by a randomly colored beam of light. The beam color determines the effect and the defense used, as shown on \tref{Prismatic Beam Effects}. The damaging effects deal 6d6 damage \add d6 damage per two caster levels above 6th.
\end{spelltarget}

\begin{dtable*}
    \lcaption{Prismatic Beam Effects}
    \begin{tabularx}{\textwidth}{l >{\lcol}p{3.6em} l >{\lcol}X l}
        \thead{1d8} & \thead{Color of Beam} & \thead{Defense} & \thead{Success}\fn{1} & \thead{Failure} \\
        1 & Red     & Reflex    & Fire damage and ignited for 5 rounds & Half damage \\
        2 & Orange  & Fortitude & Blinded for 1 round & No effect \\
        3 & Yellow  & Reflex    & Electricity damage and staggered for 1 round & Half damage, not staggered \\
        4 & Green   & Fortitude & Acid damage and sickened for 5 rounds & Half damage \\
        5 & Blue    & Will      & Slowed for 5 rounds & Slowed for 1 round \\
        6 & Indigo  & Will      & Confused for 1 round & No effect \\
        7 & Violet  & None & Damage of all energy types (acid, cold, electricity, fire) & \x \\
        8 & Octamarine & \x & Struck by two beams; roll twice more, ignoring any ``8'' results.
    \end{tabularx}
    1 See \pcref{Conditions} for a summary of the conditions imposed.
\end{dtable*}

\spellsection{Prismatic Storm}{9}
\spellinfo{Universal [Light]}{Arcane}
\spellburst{\arealarge radius centered on you}
\spellsr{Yes}
\begin{spelltargets}{All creatures in the area}[Magic vs. Special]
    \spellspecial The target is struck by a randomly colored beam of light. The beam color determines the effect and the defense used, as shown on \tref{Prismatic Beam Effects}. The damaging effects deal 9d6 damage \add d6 damage per four caster levels above 18th.
\end{spelltargets}

\spellsection{Prismatic Spray}{7}
\spelldesc{This spell causes seven shimmering, intertwined, multicolored beams of light to spray from your hand.}
\spellinfo{Universal [Light]}{Arcane, Chaos}
\spellburst{\arealarge cone}
\spellsr{Yes}
\begin{spelltargets}{All creatures in the area}[Magic vs. Special]
    \spellspecial The target is struck by a randomly colored beam of light. The beam color determines the effect and the defense used, as shown on \tref{Prismatic Beam Effects}. The damaging effects deal 7d6 damage \add d6 damage per four caster levels above 14th.
\end{spelltargets}

\spellsection{Prismatic Wall}{5}
\spellinfo{Universal [Light]}{Arcane, Chaos}
\spelltwocol{\spellzone{\arealarge wall, 20 ft. high}}{\spellrng{\rngclose}}
\spelldur{\durshort \dismissable}
\spellline
\spelleffect This spell creates a shimmering, multicolored plane of light that blocks all sight. It harms any creature that attempts to pass through it.
\begin{spelltrigger}{A creature passes through the wall}
    \begin{spelltarget}*{Creature in wall}[Magic vs. Reflex]
        \spellspecial The target is struck by a randomly colored beam of light. The beam color determines the effect and the defense used, as shown on \tref{Prismatic Beam Effects}. The damaging effects deal 5d6 damage \add d6 damage per four caster levels above 10th.
    \end{spelltarget}
\end{spelltrigger}
\spellnotes This spell can be made permanent with a \spell{permanency} ritual.

\spellsection{Prohibition}{6}
\spellinfo{Abjur/Div (Communication, Interdiction)}{Abjur, Law}
\spellemanation{\arealarge radius centered on you}
\spelldur{\durshort}
\spellsr{Yes}
\spellline
\spelleffect You loudly declare a prohibition on single, specific action which creatures must not take, such as ``Do not use ranged weapons'' or ``Do not lie''. You may choose any action that must be taken intentionally, but not involuntary actions or states of being, such as breathing or wearing armor. If the rule is too complicated, the spell fails.

The spell grants all creatures that enter the area an understanding of the prohibition, even if they were unable to understand the rule as originally stated. If you break the rule, the spell ends -- after you suffer the consequences.
\begin{spelltrigger}{A creature breaks the rule}
    \begin{spelltarget}*{The creature breaking the rule}
        \spelleffect 6d6 damage \add 1d6 per four caster levels above 12th. You know a creature broke the rule, but not which creature.
    \end{spelltarget}
\end{spelltrigger}
\spellnotes Mindless creatures are given no special insight into the rule. Any individual creature can only take damage for breaking the rule once per round.

\spellsectioncomma{Prohibition}{Greater}{9}
\spellinfo{Abjur (Interdiction)}{Abjur, Law}
\spellemanation{\arealarge radius centered on you}
\spelldur{\durshort}
\spellsr{Yes}
\spellline
\spelleffect You declare a rule that creatures must follow, as \spell{prohibition}.
\begin{spelltrigger}{A creature breaks the rule}
    \begin{spelltarget}*{The creature breaking the rule}[Magic vs. Will]
        \spelleffect 9d6 damage \add 1d6 per four caster levels above 18th. You know a creature broke the rule, but not which creature.
        \spellsuccess The target fails to take the prohibited action. Instead, it does nothing.
    \end{spelltarget}
\end{spelltrigger}
\spellnotes As \spell{prohibition}.

\spellsection{Project Image}{6}
\spellinfo{Illus (Shadow)}{Arcane}
\spellrng{\rngmed}
\spelldur{\durmed \dismissable}
\spellline
\spelleffect You tap energy from the Plane of Shadow to create a quasi-real version of yourself. The projected image looks, sounds, and smells like you, but is intangible. Normally, it mimics your actions perfectly, including speech.
\par As a swift action, you can attune to the projected image. This has several effects.
\begin{itemize}
    \item You see and hear from the image's location, rather from where your body is.
    \item Any spells you cast originate from the image instead of from you. This causes you to measure range, line of effect, and so on from the image's location, rather than from your location.
    \item You can control the image's actions independently from your own actions. Each round, it can move up to 100 feet in any direction, including vertically.
\end{itemize}

As a free action, you can stop attuning to the projected image.

\spellnotes You must maintain line of effect to the projected image at all times. If your line of effect is obstructed, the spell ends. If you teleport or use a similar effect that breaks your line of effect, even momentarily, the spell ends.

Since the image is not a creature, it is difficult to disrupt, and many spells have no effect on it.

\spellsection{Protection from Alignment}{1}
\spellinfo{Abjur (Interdiction)}{Arcane, Chaos, Divine, Evil, Good, Law}
\spellrng{\rngclose}
\spelldur{\durshort \dismissable}
\spellspecial Choose an alignment other than neutral (chaotic, good, evil, lawful).
\begin{spelltarget}{One creature}
    \spelleffect The target gains a \plus2 enhancement bonus to its special defenses. \spellbonusscalingdescription

    This bonus is increased by 2 against attacks from creatures or effects with the chosen alignment.
\end{spelltarget}
\spellnotes This spell has the subtype of the alignment opposed to the chosen alignment.

\spellsection{Protection from Energy}{3}
\spellinfo{Abjur (Shielding)}{Arcane, Divine, Nature, Protection}
\spellrng{Touch}
\spelldur{\durlong or until expended}
\spellspecial Choose a type of energy (acid, cold, electricity, or fire).
\spellsr{Yes}
\begin{spelltarget}{One creature}
    \spelleffect The target becomes immune to damage from the chosen type of energy. When the spell absorbs 10 points per caster level of energy damage, it is discharged.
\end{spelltarget}
\spellnotes This spell overlaps (and does not stack with) \spell{resist energy}. If a character is shielded by \spell{protection from energy} and \spell{resist energy}, the protection spell absorbs damage until its power is exhausted.

\spellsectioncomma{Protection from Energy}{Greater}{6}
\spellinfo{Abjur (Shielding)}{Arcane, Divine, Nature, Protection}
\spellrng{Touch}
\spelldur{\durlong or until expended}
\spellsr{Yes}
\begin{spelltarget}{One creature}
    \spelleffect The target becomes immune to all energy damage (acid, cold, electriticity, and fire). When this spell absorbs 10 points per caster level of damage in total, regardless of its type, it is discharged.
\end{spelltarget}

\pdfbookmark[2]{Q-R}{SpellDescriptionsQR}
\begin{comment}
\subsubsection{Q-R}
\end{comment}

\spellsection{Read Mind}{3}
\spellinfo{Div (Awareness) [Mind-Affecting]}{Arcane, Knowledge}
\spellrng{\rngmed}
\spelldur{Concentration}
\spellsr{Yes}
\begin{spelltarget}{One creature}[Magic vs. Will]
    \spellsuccess You can read the target's surface thoughts. You gain a \plus4 bonus to Bluff, Persuasion, and Intimidate checks against creatures whose mind you are reading.
\end{spelltarget}

\spellsectioncomma{Read Mind}{Greater}{7}
\spellinfo{Div (Awareness) [Mind-Affecting]}{Arcane, Knowledge}
\spellrng{\rngmed}
\spelldur{Concentration}
\spellsr{Yes}
\begin{spelltarget}{One creature}[Magic vs. Will]
    \spelleffect You can read the target's surface thoughts, as \spell{read mind}.
\end{spelltarget}

\spellsectioncomma{Read Mind}{Mass}{8}
\spellinfo{Div (Awareness) [Mind-Affecting]}{Arcane, Knowledge}
\spelltwocol{\spelllimit{\areamed radius}}{\spellrng{\rngmed}}
\begin{spelltargets}{Up to five creatures in the area}[Magic vs. Will]
    \spellsuccess You can read the target's surface thoughts. You gain a \plus4 bonus to Bluff, Persuasion, and Intimidate checks against creatures whose mind you are reading.
\end{spelltargets}

\spellsection{Reduce Person}{1}
\spellinfo{Trans (Polymorph) [Size-Affecting]}{Trans}
\spelltime{Full-round action}
\spellrng{\rngclose}
\spelldur{\durshort \dismissable}
\spellsr{Yes}
\begin{spelltarget}{One humanoid creature}[Magic vs. Fortitude]
    \spellsuccess This spell causes instant diminution of a humanoid creature, halving its height, length, and width and dividing its weight by 8. This decrease changes the creature's size category to the next smaller one. This has several effects.
    \begin{itemize} 
        \item \minus10 ft. penalty to movement speed.
        \item \minus4 penalty to maneuver attack and defense.
        \item \plus1 bonus to other physical attacks and defenses.
        \item \minus2 penalty to Strength.
        \item \plus2 enhancement bonus to Dexterity.
        \item \plus4 bonus to Stealth checks.
    \end{itemize}
    \par All equipment worn or carried by a creature is similarly reduced by the spell. Melee and projectile weapons deal less damage. Other magical properties are not affected by this spell. Any reduced item that leaves the reduced creature's possession (including a projectile or thrown weapon) instantly returns to its normal size. This means that thrown weapons deal their normal damage (projectiles deal damage based on the size of the weapon that fired them).
\end{spelltarget}
\spellnotes Multiple magical effects that reduce size do not stack. This spell can be made permanent with a \spell{permanency} ritual.
\par A Small humanoid creature whose size decreases to Tiny has a space of 2-1/2 feet and a natural reach of 0 feet (meaning that it must enter an opponent's square to attack). A Large humanoid creature whose size decreases to Medium has a space of 5 feet and a natural reach of 5 feet.

\spellsectioncomma{Reduce Person}{Mass}{5}
\spellinfo{Trans (Polymorph) [Size-Affecting]}{Trans}
\spelltwocol{\spelllimit{\areamed radius}}{\spellrng{\rngmed}}
\spelldur{\durshort \dismissable}
\spellsr{Yes}
\begin{spelltargets}{Up to five humanoid creatures in the area}[Magic vs. Fortitude]
    \spellsuccess The target shrinks, as \spell{reduce person}.
\end{spelltargets}

\spellsection{Regenerate}{8}
\spelldesc{You grant immense healing power to a creature with a touch.}
\spellinfo{Necro (Flesh)}{Divine, Nature}
\spellrng{Touch}
\spelldur{\durshort}
\spellsr{Yes}
\begin{spelltarget}{One living creature}
    \spelleffect The target automatically heals a number of hit points each round equal to your caster level.
    \par You can also use this spell to regrow lost portions of the target's body and to reattach severed limbs or body parts, if you do nothing but concentrate on regrowing the lost body part or reattaching the severed limb for 5 minutes.
\end{spelltarget}

\spellsection{Repulsion}{6}
\spelldesc{An invisible, mobile field surrounds you and prevents creatures from approaching you.}
\spellinfo{Abjur (Shielding) [Barrier]}{Arcane, Protection, Travel}
\spellemanation{\arealarge radius centered on you}
\spelldur{\durshort \dismissable}
\spellsr{Yes}
\begin{spelltrigger}{A creature in the area moves towards you}
    \begin{spelltarget}*{The moving creature}[Magic vs. Will]
        \spellsuccess The target is unable to move towards you. It can stand still, or alter the direction of its movement to move parallel towards you or away from you.
    \end{spelltarget}
\end{spelltrigger}
\spellnotes Unlike most barrier spells, this spell does not collapse if you move towards a creature held at bay by the barrier. The spell continues to prevent that creature from approaching you, but the creature suffers no other ill effect.

\spellsection{Resilient Sphere}{5}
\spelldesc{You trap a foe in a globe of shimmering force, removing it from the battle.}
\spellinfo{Evoc (Control) [Force]}{Evoc}
\spellrng{\rngmed}
\spellzone{5 ft. radius}
\spelldur{\durshort \dismissable}
\spellsr{Yes}
\spellline
\spelleffect The area is surrounded by a sphere made of force. The sphere cannot be moved or broken, preventing anything in the area from interacting with anything outside the area.
\begin{spelltarget}*{Everything in the area}[Magic vs. Reflex]
    \spellsuccess The target is unable to disrupt the sphere.
    \spellfailure The target disrupts the sphere, preventing it from being formed.
\end{spelltarget}
\spellnotes Force effects can be destroyed by \spell{disintegrate}.

\spellsection{Resist Energy}{2}
\spellinfo{Abjur (Shielding)}{Arcane, Divine, Nature, Protection}
\spellrng{Touch}
\spelldur{\durlong or until expended}
\spellsr{Yes}
\begin{spelltarget}{One creature}
    \spelleffect The target gains gains damage reduction against a single energy type of your choice (acid, cold, electricity, fire). This reduces the damage it takes each round from that energy type by 10 \add 2 per caster level above 4th.
    \par The spell can absorb a maximum amount of damage equal to 10 points per caster level. After it absorbs its maximum amount of damage, the spell ends.
\end{spelltarget}
\spellnotes This spell overlaps (and does not stack with) \spell{protection from energy}. If a character is shielded by both spells, the \spellindirect{protection from energy}{protection} spell absorbs damage until its power is exhausted. A character can only be affected by one \spell{resist energy} spell at once.

\spellsectioncomma{Resist Energy}{Greater}{4}
\spellinfo{Abjur (Shielding)}{Arcane, Divine, Nature}
\spellrng{Touch}
\spelldur{\durlong or until expended}
\spellsr{Yes}
\begin{spelltarget}{One creature}
    \spelleffect The target gains damage reduction against all energy types (acid, cold, electricity, fire). This reduces the energy damage it takes each round by 20 \add 2 per caster level above 8th.
    \par The spell can absorb a maximum amount of damage equal to 10 points per caster level. After it absorbs its maximum amount of damage, the spell ends.
\end{spelltarget}
\spellnotes As \spell{resist energy}.

\spellsection{Retributive Brilliance}{5}
\spellinfo{Abjur/Illus (Figment, Shielding)}{Arcane}
\spellrng{\rngclose}
\spellfocus{One creature}
\spelldur{\durshort or until discharged}
\spellsr{Yes}
\begin{spelltrigger}{A creature attacks the focus creature with a melee weapon}
    \parhead*{Trigger Action} Immediate action by you or the focus creature
    \spellrng{\rngclose}
    \begin{spelltarget}*{The attacking creature}[Magic vs. Reflex]
        \spellsuccess The target is \dazzled for 5 rounds.

        If the target is \bloodied, it is also \blinded for 5 rounds.
        \spellfailure The target is dazzled for 5 rounds.
    \end{spelltarget}
\end{spelltrigger}

\spellsection{Retributive Shield}{4}
\spellinfo{Abjur/Necro (Life, Shielding)}{Arcane}
\spellrng{\rngclose}
\spelldur{\durshort}
\spellsr{Yes}
\begin{spelltarget}{One creature}
    \spelleffect The target gains physical damage reduction, reducing the physical damage it takes each round by 8 \add 1 per caster level above 8th. If it takes life damage, it cannot use its damage reduction for 1 round. In addition, any creature within \rngmed range of the target that deals damage to it takes life damage equal to the damage resisted by this damage reduction.
\end{spelltarget}

\spellsection{Retrieve Ally}{2}
\spelldesc{You save your ally from danger by teleporting it next to you.}
\spellinfo{Conj (Translocation) [Teleportation]}{Conj}
\spellrng{\rngmed}
\spellsr{Yes}
\begin{spelltarget}{One willing creture}
    \spelleffect The target teleports into a free space adjacent to you. You must have line of sight and line of effect to both the target and its destination. If you accidentally attempt to teleport the creature into an invalid location, the spell simply fails.
\end{spelltarget}

\spellsection{Retrieve Object}{1}
\spelldesc{You teleport an object into your hand.}
\spellinfo{Conj (Translocation) [Teleportation]}{Conj}
\spellrng{\rngclose}
\spellsr{Yes}
\begin{spelltarget}{One unattended object (Small or smaller)}[Magic vs. Will]
    \spellsuccess The target teleports into your hands.
\end{spelltarget}
\spellnotes Most items do not have a Will defense. In that case, this spell succeeds automatically.

\spellsectioncomma{Retrieve Object}{Greater}{5}
\spellinfo{Conj (Translocation) [Teleportation]}{Conj}
\spellrng{\rngmed}
\spellsr{Yes}
\begin{spelltarget}{One object (Medium or smaller)}[Magic vs. Will]
    \spelleffect The target teleports into your hands.
\end{spelltarget}
\spellnotes As \spell{retrieve object}.

\spellsection{Revelation}{9}
\spelldesc{You grant the target a powerful vision of a possible future.}
\spellinfo{Div (Awareness, Knowledge)}{Arcane, Div, Knowledge}
\spellrng{\rngmed}
\spelldur{\durshort}
\spellsr{Yes}
\spellline
\spellspecial This spell has three versions. Its effects depend on which version is chosen.
\begin{spelltarget}*{One creature}
    \spelleffect[\subspell{Revelation of Destruction}] You inflict a vision of a terrible future upon the target. It takes a \minus4 penalty to attacks, defenses, and checks as it struggles to avoid the certainty of its own doom.

    As long as the target is \bloodied, these penalties increase by 2.
    \spelleffect[\subspell{Revelation of Prowess}] You show the target a vision of its success in the combat to come. It gains the benefits of a \spell{greater precognition} spell.
    \spelleffect[\subspell{Revelation of Truth}] You show the target the truth of the world around it. It gains the benefits of a \spell{true seeing} spell.
\end{spelltarget}
\spellnotes Creatures without an Intelligence score are not affected by this spell.

\spellsection{Reverse Gravity}{8}
\spellinfo{Trans}{Air, Arcane, Trickery}
\spelltwocol{\spellzone{\arealarge radius cylinder}}{\spellrng{\rngclose}}
\spelldur{Concentration (up to 5 rounds)}
\begin{spelltarget}{Everything in the area}
    \spelleffect The target falls upwards, reaching the top of the area within 1 round. If the target strikes a solid object, such as a ceiling, it is affected in the same way as it would be during a normal fall. Otherwise, it floats at the top of the area, oscillating slightly. When the spell ends, if the target is still floating, it falls, potentially taking damage for the fall.
\end{spelltarget}
\spellnotes Creatures who can fly or levitate can keep themselves from falling, though the shift in gravity can be disorienting.

\spellsection{Revivify}{5}
\spelldesc{You reconnect a corpse's soul with its body before the soul has completely passed on.}
\spellinfo{Necro (Life, Soul)}{Divine}
\spellcmp{Verbal, Somatic, and Material}
\spellrng{Touch}
\begin{spelltarget}{One dead creature}
    \spelleffect If the target has been dead for no more than one round per four caster levels, it is restored to life. This functions like \spell{lesser resurrection} ritual, except that tthe target suffers no negative effects for having died.
\end{spelltarget}
\spellline
\spellmat{Diamonds worth at least 500 gp.}

\spellsection{Righteous Might}{5}
\spellinfo{Trans (Augment, Polymorph) [Size-Affecting]}{Divine, Good, Strength}
\spelldur{\durshort \dismissable}
\begin{spelltarget}{You}
    \spellsuccess You become larger, as \spell{enlarge person}. In addition, you gain physical damage reduction, reducing the physical damage you take each round by 10 \add 1 per caster level above 10th. This damage reduction is overcome by evil attacks if you are good or neutral, and by good attacks if you are evil. If you are hit by an appropraitely aligned attack, you cannot use your damage reduction for 1 round.
\end{spelltarget}
\spellnotes Multiple magical effects that increase size do not stack.

\pdfbookmark[2]{S}{SpellDescriptionsS}
\begin{comment}
\subsubsection{S}
\end{comment}

\spellsection{Sanctuary}{1}
\spellinfo{Abjur/Ench (Compulsion, Shielding) [Mind-Affecting]}{Arcane, Divine, Protection}
\spellrng{Touch}
\spellfocus{One creature}
\spelldur{\durshort or until discharged}
\spellsr{Yes}
\spellspecial If the focus creature attacks any other creature, this spell immediately ends.
\begin{spelltrigger}{The focus creature is attacked}
    \begin{spelltarget}*{The attacking creature}[Magic vs. Will]
        \spellsuccess The target is unable to attack the focus creature. It can attack a different creature instead.
    \end{spelltarget}
\end{spelltrigger}

\spellnotes This is considered a mind-affecting effect on any creature that attempts to attack the target. Creatures immune to mind-affecting effects can attack the target freely.

\spellsection{Scintillating Pattern}{8}
\spelldesc{You create a massive spread of colorful lights that spin and whirl in a complex pattern that bewilders your foes.}
\spellinfo{Ench/Illus (Compulsion, Figment) [Light, Mind-Affecting, Sight-Dependent]}{Arcane}
\spellzone{\arealarge radius centered on you}
\spelldur{\durshort}
\spellline
\spelleffect The area is brightly illuminated, and dim illumination extends for an additional 50 feet.
\spellsr{Yes}
\begin{spelltarget}*{All enemies in the area}
    \spelleffect The target is \bewildered.
\end{spelltarget}
\spellnotes Creatures unable to see the lights are not bewildered.

\spellsection{Scorching Ray}{2}
\spelldesc{You blast your foe with a fiery ray.}
\spellinfo{Evoc (Energy) [Fire]}{Arcane, Destruction, Fire}
\spellrng{\rngclose}
\spelldur{\durshort}
\spellsr{Yes}
\begin{spelltarget}{One creature or object}[Magic vs. Reflex]
    \spellsuccess 4d6 fire damage \add d6 per two caster levels above 4th. In addition, the target is \ignited.
    \spellfailure As above, but half damage, and the target is not ignited.
\end{spelltarget}

\spellsection{Sea of Fog}{8}
\spellinfo{Conj (Creation) [Fog]}{Arcane, Nature}
\spellzone{200 ft. radius cylinder centered on you, 50 ft. high}
\spellsr{No}
\spellline
\spelleffect Fog fills the area, as \spell{fog cloud}.
\spellnotes \fogspellnotes A severe wind disperses the fog within 1 minute, a windstorm disperses it within 5 rounds, and a hurricane disperses it within a round.

\spellsection{Searing Light}{3}
\spelldesc{You channel divine power into a searing blast of light that erupts your palm, striking your foe.}
\spellinfo{Evoc (Channeling) [Light]}{Divine}
\spellrng{\rngclose}
\spellsr{Yes}
\begin{spelltarget}{One creature or object}[Magic vs. Reflex]
    \spelleffect If the target is undead or particularly vulnerable to bright light, it is also \dazzled for 5 rounds.
    \spellsuccess 6d6 divine damage \add d6 per two caster levels above 6th.
    \spellfailure As above, but half damage.
\end{spelltarget}

\spellsection{See Invisibility}{2}
\spellinfo{Div (Revelation)}{Arcane}
\spellrng{Touch}
\spelldur{\durlong \dismissable}
\begin{spelltarget}{One creature}
    \spelleffect The target can see any objects or beings that are invisible within its range of vision, as well as any that are ethereal, as if they were normally visible. Such creatures are visible as translucent shapes, allowing the target to easily discern the difference between visible, invisible, and ethereal creatures.
\end{spelltarget}
\spellnotes The spell does not reveal the method used to obtain invisibility. It does not reveal illusions other than invisibility. It does not reveal creatures who are simply hiding, concealed, or otherwise hard to see.

This spell can be made permanent with a \spell{permanency} ritual.

\spellsection{Shadow Body}{7}
\spellinfo{Illus/Trans (Polymorph, Shadow)}{Arcane}
\spelldur{\durmed \dismissable}
\begin{spelltarget}{You}
    \spelleffect Your body and all your equipment are subsumed by your shadow. As a living shadow, you blend perfectly into any other shadow and vanish in darkness. You appear as an unattached shadow in areas of full light.
    \par You can move at your normal speed, on any surface, including walls and ceilings, as well as across the surfaces of liquids -- even up the face of a waterfall.
    \par You become perfectly flat, potentially allowing you to move into locations you would not normally be able to move into.
    \par While in your shadow body, you gain physical damage reduction, reducing the physical damage you take each round by 14 \add 1 per caster level above 14th. If you take solar damage, you cannot use your damage reduction for 1 round. You are immune to ability damage, disease, drowning, and poison. You take only half damage from energy attacks (acid, cold, electricity, and fire).
    \par While affected by this spell, you can be detected by spells that read thoughts, life, or presences (including true seeing), or if you make suspicious movements in lighted areas.
    \par You cannot harm anyone physically or manipulate any objects, but you can use your spells normally. Doing so may attract notice, but if you remain in a shadowed area, you get a \plus15 enhancement bonus on your Hide check to remain unnoticed.
\end{spelltarget}

\spellsection{Shadow Puppet}{9}
\spellinfo{Conj/Illus (Shadow, Translocation) [Planar, Unreal]}{Illus}
\spelldur{\durmed}
\spellline
\spelleffect You step into the Plane of Shadow, as \spell{shadow walk}. At the same time, you create a quasi-real version of yourself, as \spell{project image}. The duplicate image appears superimposed over your body so that observers don't notice an image appearing and you disappearing. You can then control the image and cast spells through it even though you are on a different plane.
\spellnotes If the image moves farther than \rnglong range away from where it was originally created, or if you leave the Plane of Shadow, the image ceases to exist.

If you are not on the Material Plane when you cast this spell, it has no effect.

\spellsection{Shadow Umbra}{8}
\spelldesc{You shield your ally with a dark umbra that connects directly to the Plane of Shadow.}
\spellinfo{Abjur/Illus (Glamer, Shadow, Shielding) [Planar]}{Arcane}
\spellrng{\rngclose}
\spelldur{\durshort}
\spellsr{Yes}
\begin{spelltarget}{One creature}
    \spelleffect All attacks that would affect the creature, including magical and supernatural attacks, have a 50\% chance to be absorbed by the umbra. Attacks absorbed by the umbra have no effect on the target. The umbra is selective, and does not inhibit beneficial effects.

    Whenever the umbra absorbs an attack, it alters the creature's appearance (including smell, sound, and other senses, as appropriate) with a glamer. This causes the creature to seem as if were affected by the attack. Outside observers have no way of knowing which attacks were absorbed by the umbra unless they can recognize the illusion. The spell does not attempt to mimic the effects of extraordinary attacks which cannot be disguised, such as attacks which would destroy the creature's body.
\end{spelltarget}
\spellnotes If you are not on the Material Plane when you cast this spell, it has no effect.

\spellsection{Share Pain}{2}
\spellinfo{Abjur/Necro (Life, Shielding)}{Arcane, Divine, Protection}
\spellrng{\rngmed}
\spelltgts{Two willing creatures}
\spelldur{\durlong \dismissable}
\spellsr{Yes}
\spellspecial When you cast this spell, you choose which target will be protected.
\spellline
\spelleffect When the protected creature would take hit point damage, it instead loses half that many hit points (rounded down), and the other target loses hit points equal to the other half of the damage (rounded up).

If the targets get out of range of each other, the effect is suppressed until they return within range.
\spellnotes The loss of hit points caused by this spell is not damage, and is not affected by damage reduction or other abilities which affect damage.

\spellsectioncomma{Share Pain}{Forced}{3}
\spellinfo{Abjur/Necro (Life, Shielding)}{Arcane, Divine, Evil}
\spellrng{\rngmed}
\spelltgts{Two creatures}
\spelldur{\durlong \dismissable}
\spellsr{Yes}
\spellspecial When you cast this spell, you choose which target will be protected.
\spellline
\spellattack{Magic vs. Will}
\begin{spellmargin}
    \spellsuccess The targets share damage, as \spell{share pain}.
\end{spellmargin}

\spellsection{Shield of Faith}{1}
\spelldesc{You create a shimmmering, magical shield that protects your ally as long as you maintain faith.}
\spellinfo{Abjur (Shielding)}{Divine, Protection}
\spellrng{\rngclose}
\spelldur{\durshort or \durlong; see text \dismissable}
\spellsr{Yes}
\begin{spelltarget}{One creature}
    \spelleffect The target gains a \plus2 shield modifier. \spellbonusscalingdescription Unlike a mundane shield, this shield does not require a free hand and has no armor check penalty or arcane spell failure chance.

    \par If you are the target, this spell lasts for \durlong duration. On any other creature, it lasts for \durshort duration.
\end{spelltarget}
\spelleffect You can maintain concentration on this spell as a swift action.

\spellsection{Shield of Law}{8}
\spelldesc{A dim, blue glow surrounds your allies, protecting them from attacks.}
\spellinfo{Abjur (Shielding) [Lawful]}{Divine, Law}
\spelllimit{\areamed radius centered on you}
\spelldur{\durshort \dismissable}
\spellsr{Yes}
\begin{spelltargets}{Up to five creatures in the area}
    The target gains a \plus5 enhancement bonus to its defenses. In addition, it gains spell resistance against lawful spells and spells cast by lawful creatures.
    \par At the end of each round, all lawful creatures within \rngclose range of the target that attacked it with their body or a melee weapon that round take 4d6 points of damage. A creature that attacks multiple creatures shielded by this spell can take this damage multiple times.
\end{spelltargets}

\spellsection{Shillelagh}{1}
\spellinfo{Trans}{Nature}
\spellrng{Touch}
\spelldur{\durshort}
\spellsr{Yes}
\begin{spelltarget}{One nonmagical oak club or quarterstaff}
    \spelleffect As long as you wield it, the target is enhanced, as \spell{magic weapon}. In addition, it deals damage as if it were one size category larger. If you stop wielding the weapon, this spell has no effect on it.
\end{spelltarget}
\spellnotes A typical club or quarterstaff wielded by a Medium creature would deal d8 damage under this spell.

\spellsection{Shocking Grasp}{1}
\spelldesc{You deliver a powerful electrical shock to your foe.}
\spellinfo{Evoc (Energy) [Electricity]}{Arcane, Destruction}
\spellrng{5 ft.}
\spellsr{Yes}
\begin{spelltarget}{One creature or object}l[Magic vs. Reflex and Fortitude]
    \spellspecial You gain a \plus2 bonus to attack if the target is wearing metal armor or otherwise has a significant quantity of metal.
    \spellsuccess[Reflex] 2d6 electricity damage \add d6 per two caster levels above 2nd.
    \spellfailure[Reflex] As above, but half damage.
    \spellsuccess[Reflex and Fortitude] The target is \staggered for 1 round.
\end{spelltarget}

\spellsection{Shout}{4}
\spelldesc{You emit an ear-splitting yell that deafens and damages creatures in its path.}
\spellinfo{Evoc (Energy) [Sonic]}{Arcane, Destruction, Strength}
\spellcmp{Verbal only}
\spellburst{\areamed cone}
\spellsr{Yes}
\begin{spelltargets}{Everything in the area}
    \spellspecial You gain a \plus5 bonus to attack against brittle or crystalline objects and creatures.
    \spellsuccess 4d6 sonic damage \add d6 per four caster levels above 8th. In addition, the target is \deafened.
    \spellfailure As above, but half damage, and the target is not deafened.
\end{spelltargets}

\spellsectioncomma{Shout}{Greater}{7}
\spellinfo{Evoc (Energy) [Sonic]}{Arcane, Destruction, Strength}
\spellcmp{Verbal only}
\spellburst{\arealarge cone}
\spellsr{Yes}
\begin{spelltargets}{Everything in the area}
    \spellspecial You gain a \plus5 bonus to attack against brittle or crystalline objects and creatures.
    \spellsuccess 7d6 sonic damage \add d6 per four caster levels above 8th. In addition, the target is \deafened.
    \spellfailure As above, but half damage, and the target is not deafened.
\end{spelltargets}

\spellsection{Shrink Item}{3}
\spellinfo{Trans (Alteration)}{Trans}
\spellrng{\rngclose}
\spelldur{24 hours or until discharged}
\spellsr{Yes}
\spellspecial As you cast this spell, choose a command word.
\begin{spelltarget}{One nonmagical object (Medium or smaller; see text)}l[Magic vs. Will]
    \spellsuccess The target shrinks to 1/16 of its normal size in each dimension (to about 1/4,000 the original volume and mass). This change effectively reduces its size by four size categories. If the target is physically unable to shrink, such as a ring on a finger, it shrinks as much as it can without causing harm to itself or the physical impediment.

    As a standard action, any creature can speak the command word to return the target to its original size. It must be resting on a stable surface. If the command word is spoken while the target is not stable, such as while it is in the air, it returns to its original size as soon as it finds a resting point. Restoring the target to its normal size ends the spell.
\end{spelltarget}
\spellnotes This spell can normally target an object of up to Medium size. You can shrink a Large object at 10th caster level, a Huge object at 16th caster level, or a Gargantuan object at 24th caster level.
\par This spell can be made permanent with a \spell{permanency} ritual, in which case the affected object can be shrunk and expanded an indefinite number of times, but only by the original caster. If you recast the spell each day on an object, you can keep it at its small size indefinitely.

\spellsection{Silence}{2}
\spellinfo{Illus (Glamer)}{Divine, Trickery}
\spellrng{\rngmed}
\spelldur{\durshort \dismissable}
\spellsr{Yes}
\begin{spelltarget}{One creature or object}[Magic vs. Will]
    \spelleffect The target becomes unable to make noise. Any noises it makes are inaudible to other creatures. When you cast the spell, you may choose whether the target can still hear itself normally, potentially causing it to be unaware of the effect of the spell.

    Extraordinarily loud noises, such as the yell of a giant, are merely muffled by the spell rather than completely silenced. The DC to hear such sounds produced by the target is increased by 40. Sonic attacks function normally.
\end{spelltarget}
\spellnotes Spellcasters unable to hear themselves cast are treated as deafened, and suffer a 20\% chance of spell failure when casting spells with verbal components.

\spellsection{Silent Image}{2}
\spellinfo{Illus (Figment) [Unreal]}{Illus}
\spelltwocol{\spellzone{\areamed radius}}{\spellrng{\rngmed}}
\spelldur{\durshort}
\spellsr{No}
\spellline
\spelleffect This spell creates the visual illusion of an object, creature, or force within the area, as determined by you. The illusion does not create sound, smell, texture, or temperature. As a standard action, you can concentrate to alter the image within the area.
\spellnotes Creatures can recognize the figment as unreal by interacting with it physically, or by making a Perception check against a DC equal to 10 \add half your caster level \add your casting attribute. A creature gets a \plus10 bonus on this Perception check when using senses which should be present in the figment, but which are missing.

A creature faced with definitive proof that the figment is unreal can disbelieve it, treating it as if it were not there.

\spellsection{Skysmite}{6}
\spelldesc{You call down lightning from the heavens, unerringly striking your foes, even if you cannot see them.}
\spellinfo{Evoc (Energy) [Electricity]}{Air, Arcane, Destruction, Nature}
\spellrng{\rngext}
\spellburst{\arealarge vertical line of lightning, 5 ft. wide}
\spellsr{Yes}
\spellspecial If no creature or objects lie in the area, the lightning strikes elsewhere instead. It strikes the  occupied square within the spell's range that lies closest to its original destination. If multiple occupied squares are equally close, it strikes the largest target.
\begin{spelltargets}{Everything in the area}[Magic vs. Reflex]
    \spellsuccess 12d6 electricity damage \add d6 per two caster levels above 12th.
    \spellfailure As above, but half damage.
\end{spelltargets}
\spellnotes The lightning can unerringly identify invisible and concealed creatures, but it does not differentiate between friend, foe, and inanimate object.

\spellsection{Sleep}{1}
\spellinfo{Ench (Compulsion) [Mind-Affecting, Sleep]}{Arcane}
\spellrng{\rngmed}
\spelldur{\durshort}
\spellsr{Yes}
\begin{spelltarget}{One living creature}[Magic vs. Will]
    \spellsuccess The target is \fatigued and attempts to go to sleep as soon as possible, though it will not stop fighting to do so. Awakening a creature put to sleep by this spell is difficult, and requires a standard action.
\end{spelltarget}

\spellsectioncomma{Sleep}{Mass}{4}
\spellinfo{Ench (Compulsion) [Mind-Affecting, Sleep]}{Arcane}
\spelltwocol{\spelllimit{\areamed radius}}{\spellrng{\rngmed}}
\spelldur{\durshort}
\spellsr{Yes}
\begin{spelltargets}{Up to five creatures within the area}
    \spellsuccess The target becomes tired, as \spell{spell}.
\end{spelltargets}

\spellsection{Slow}{2}
\spelldesc{You decelerate your enemy's motions, causing her to move and act more slowly than normal.}
\spellinfo{Trans (Temporal)}{Arcane}
\spellrng{\rngmed}
\spelldur{\durshort}
\spellsr{Yes}
\begin{spelltarget}{One creature}[Magic vs. Will]
    \spellsuccess The target is \slowed.
\end{spelltarget}

\spellsectioncomma{Slow}{Mass}{6}
\spelldesc{You decelerate your enemies' motions, causing them to move and act more slowly than normal.}
\spellinfo{Trans (Temporal)}{Arcane}
\spelltwocol{\spelllimit{\areamed radius}}{\spellrng{\rngmed}}
\spelldur{\durshort}
\spellsr{Yes}
\begin{spelltargets}{Up to five creatures in the area}
    \spelleffect The target is \slowed.
\end{spelltargets}

\spellsection{Solid Fog}{6}
\spelldesc{You conjure a bank of immensely thick fog, concealing those inside.}
\spellinfo{Conj (Creation) [Fog]}{Arcane, Nature, Water}
\spelltwocol{\spellzone{\areamed radius cylinder}}{\spellrng{\rngmed}}
\spelldur{\durshort}
\spellsr{No}
\spellline
\spelleffect Fog blocks sight in the area, as \spell{fog cloud}. In addition to obscuring sight, the fog is so thick that any creature attempting to move through it progresses at a speed of 5 feet, regardless of its normal speed, and it takes a \minus2 penalty on all melee attack and melee damage rolls. The vapors prevent effective ranged weapon attacks (except for magic rays and the like). A creature or object that falls into solid fog is slowed, so that each 10 feet of vapor that it passes through reduces falling damage by 1d6.
\par A creature in the fog can take a full-round action to make a Strength check, moving 5 feet for every 5 by which the result exceeds DC 0. This movement is affected by any other effects which impede movement, as normal.
\spellnotes \fogspellnotes A severe wind disperses the fog within 1 minute, a windstorm disperses it within 5 rounds, and a hurricane disperses it within a round.

This spell can be made permanent with a \spell{permanency} ritual. A permanent solid fog dispersed by wind reforms in 10 minutes.

\spellsection{Song of Discord}{6}
\spelldesc{Magical music fills the air, sowing confusion among your foes.}
\spellinfo{Ench (Compulsion) [Auditory, Mind-Affecting]}{Arcane}
\spelltwocol{\spellburst{\areamed radius}}{\spellrng{\rngmed}}
\spelldur{Concentration}
\spellsr{Yes}
\begin{spelltarget}{All creatures in the area}[Magic vs. Will]
    \spellsuccess The target is \bewildered.

    As long as the target is \bloodied, it is instead \confused.
\end{spelltarget}

\spellsection{Sound Burst}{2}
\spelldesc{You create a cacophony of sound.}
\spellinfo{Evoc (Energy) [Sonic]}{Arcane}
\spelltwocol{\spellburst{\areasmall radius}}{\spellrng{\rngclose}}
\spellsr{Yes}
\begin{spelltarget}{Everything in the area}[Magic vs. Fortitude]
    \spellsuccess 2d6 sonic damage \add d6 per four caster levels above 4th.
    \spellfailure As above, but half damage.
\end{spelltarget}

\spellsection{Spell Resistance}{4}
\spellinfo{Abjur (Shielding) [Magic]}{Abjur, Magic, Protection}
\spellrng{\rngclose}
\spelldur{\durshort}
\spellsr{Yes}
\begin{spelltarget}{One creature}
    \spelleffect The target gains spell resistance against all spells.
\end{spelltarget}
\spellnotes A creature with spell resistance may always make a saving throw when a spell is cast on it. If the creature succeeds, the spell has no effect on it. The type of saving throw made is indicated by the spell. If the spell also allows a saving throw of the same type, only one roll is made.

\spellsection{Spelltheft}{5}
\spellinfo{Abjur (Negation) [Magic]}{Abjur, Magic}
\spellspecial This spell functions like \spell{dispel magic}, except that you can choose to gain the effects of any spells you dispel as if they had been originally cast on you. The effects last for the remainder of their original durations or for 5 rounds, whichever is shorter. Spells that cannot be cast on you, such as spells which only affect the caster, are simply dispelled.

\spellsectioncomma{Spelltheft}{Greater}{8}
\spellinfo{Abjur (Negation) [Magic]}{Abjur}
\spellspecial This spell functions like \spell{greater dispel magic}, except that you can choose to gain the effects of any spells you dispel as if they had been originally cast on you. The effects last for the remainder of their original durations or for 5 rounds, whichever is shorter. Spells that cannot be cast on you, such as spells which only affect the caster, are simply dispelled.

\spellsection{Spell Turning}{7}
\spellinfo{Abjur (Shielding) [Magic]}{Arcane, Magic, Protection}
\spelldur{\durlong or until expended}
\begin{spelltarget}{You}
    \spelleffect If you would be a target of a spell or spell-like ability, the caster is targeted instead. If the spell affects multiple targets, the other targets are affected normally. If the caster is not a valid target, the spell simply has no effect on you.

    After three spells have been reflected in this way, the spell ends. You can reflect one additional spell per 7 caster levels above 14th.
\end{spelltarget}

\spellsection{Spider Climb}{2}
\spelldesc{You grant your ally the ability climb on walls and ceilings as well as a spider does.}
\spellinfo{Trans (Imbuement)}{Arcane, Nature, Travel}
\spellrng{Touch}
\spelldur{\durmed}
\spellsr{Yes}
\begin{spelltarget}{One creature}
    \spelleffect The target gains a climb speed of 20 feet. It must have its hands free to climb in this manner.
\end{spelltarget}
\spellnotes See \pcref{Climbing}, for more details.

%need to run expected damage calculation
\spellsection{Spiritual Weapon}{2}
\spelldesc{You bring into being a weapon made of pure force which attacks your foes of its own volition.}
\spellinfo{Evoc (Energy) [Force]}{Divine, War}
\spellrng{\rngmed}
\spelldur{\durshort \dismissable}
\spellsr{Yes}
\spellline
\spelleffect This spell creates a floating weapon made of magical force. Each round, as a swift action, you can direct the weapon to attack a target. If you do not direct the weapon, it remains motionless.

The weapon is sized for you, and can be any type of weapon you are proficient with, though the weapon's shape does not alter this spell's effects. It has 3 hit points per caster level, an Armor defense of 10 \add half your caster level, and a Maneuver defense of 10 \add your caster level \add your casting attribute. Most effects that don't affect objects do not affect the weapon.

\begin{spelltarget}*{One creature or object}l[Caster level \add casting attribute vs. Armor defense]
    \spellsuccess d8 force damage \add half your casting attribute.
\end{spelltarget}
\spellnotes The weapon attacks during the action phase, regardless of when you direct it to attack a target.

The weapon can move up to 60 feet per round. It never provokes attacks of opportunity from opponents. Since the weapon is directed by you, its ability to interact with invisible or concealed creatures is no better than yours. Its special defenses are the same as your special defenses. If the weapon goes out of range of you, it winks out.

\spellsection{Stampede}{9}
\spelldesc{You summon a stampede of bison that trample your foes before disappearing as quickly as they arrived.}
\spellinfo{Conj (Summoning)}{Nature, Wild}
\spelltime{Full-round action}
\spellarea{100 ft. line, 20 ft. wide}
\spelldur{\durshort \dismissable}
\spellsr{No}
\begin{spelltarget}{Everything in the area}[Magic vs. Reflex]
    \spellsuccess 9d6 bludgeoning damage \add d6 per four caster levels above 18th, and the target is knocked \prone.
    \spellfailure Half damage, and the target is not knocked prone.
\end{spelltarget}

\spellsection{Stinking Cloud}{5}
\spelldesc{You create putrid vapors which obscure sight and sicken creatures.}
\spellinfo{Conj (Creation)}{Arcane}
\spelltwocol{\spellzone{\areamed radius cylinder}}{\spellrng{\rngmed}}
\spelldur{\durshort}
\spellsr{No}
\spellline
\spelleffect Fog fills the area, as \spell{fog cloud}, except that the fog has a putrid stench.
\begin{spelltrigger}{End of every round}
    \begin{spelltarget}*{Everything in the area}[Magic vs. Fortitude]
        \spellsuccess The target is \sickened for as long as it remains in the cloud, and for 5 rounds after it leaves.
    \end{spelltarget}
\end{spelltrigger}
\spellnotes This spell can be made permanent with a \spell{permanency} ritual. A permanent \spell{stinking cloud} dispersed by wind reforms in 10 minutes. \fogspellnotes \fogwindspellnotes

\spellsection{Stoneskin}{4}
\spelldesc{You dramatically toughen a creature's skin, giving it the appearance of stone.}
\spellinfo{Trans (Alteration) [Earth]}{Arcane, Earth, Nature, Protection}
\spellrng{Touch}
\spelldur{\durshort}
\spellsr{Yes}
\begin{spelltarget}{One creature}
    \spelleffect The target gains a \plus3 enhancement bonus to its armor modifier. This bonus increases to \plus4 at 14th caster level, and to \plus5 at 20th caster level. In addition, it gains physical damage reduction, reducing the damage it takes each round from physical attacks by 8 \add 1 per caster level above 8th. If it is hit by an adamantine weapon, it cannot use its damage reduction for 1 round.
\end{spelltarget}

\spellsection{Storm of Vengeance}{9}
\spellinfo{Conj/Evoc (Energy, Control, Creation)}{Air, Divine, Nature, War, Water}
\spelltime{Full-round action}
\spelltwocol{\spellzone{500 ft. radius cylinder}}{\spellrng{\rngfar}}
\spelldur{Concentration (maximum 10 rounds)}
\spellsr{Yes}
\spellline
\spelleffect An enormous storm cloud occupies the top 200 feet of the area, as \spell{fog cloud}. Within the area, lightning strikes and thunder rolls. Sunlight is blocked by the dark cloud. This may cause the area to have shadowy illumination, granting everything in it \concealment.

At the end of every round, the storm has an additional effect, as shown on \tref{Storm of Vengeance Effects}. Damaging effects deal 9d6 damage \add d6 per four caster levels above 18th.

\begin{dtable*}
    \lcaption{Storm of Vengeance Effects}
    \begin{tabularx}{\textwidth}{l l l >{\lcol}X l}
        \thead{Rounds} & \thead{Effect} & \thead{Defense} & \thead{Success} & \thead{Failure} \\
        Odd (1, 3, 5, 7, 9)   & Lightning  & Reflex    & Electricity damage (foes only) & Half damage \\
        Even (2, 4, 6, 8, 10) & Thunder    & Fortitude & Deafened for 5 rounds & No effect \\
        2, 6, 10              & Hail       & Reflex    & Bludgeoning damage & Half damage \\
        4, 8                  & Acid rain  & None      & Acid damage & \x \\
    \end{tabularx}
\end{dtable*}
\spellnotes When the storm has multiple effects in the same round, roll a single attack and compare the result to all relevant defenses.

\spellsection{Stormlord}{7}
\spelldesc{You surround yourself in a whirlwind which deflects ranged attacks and batters your foes.} 
\spellinfo{Abjur/Evoc (Control, Shielding)}{Air, Nature}
\spelldur{\durshort \dismissable}
\spellsr{Yes}
\begin{spelltarget}[Primary]{You}
    \spelleffect You gain physical damage reduction against ranged attacks, reducing the physical damage you take each round from ranged attacks by 35 \add 2 per caster level above 14th.
\end{spelltarget}
\begin{spelltrigger}{Creature within \rngclose range makes a physical attack against you}
    \begin{spelltarget}[Secondary]{Attacking creature}[Magic vs. Fortitude]
        \spellsuccess 7d6 bludgeoning damage \add d6 per four caster levels above 14th.
        \spellfailure As above, but half damage.
    \end{spelltarget}
\end{spelltrigger}

\spellsection{Strip the Flesh}{7}
\spelldesc{You rend parts of your foe's skin off its body, inflicting grievous wounds and leaving it vulnerable.}
\spellinfo{Necro (Flesh)}{Arcane}
\spellrng{\rngclose}
\spelldur{\durshort}
\spellsr{Yes}
\begin{spelltarget}{One creature}[Magic vs. Fortitude]
    \spelleffect 7d10 physical damage \add d10 per four caster levels above 14th
    \spellsuccess All damage the target takes is doubled. This does not apply to the initial damage dealt by this spell.
\end{spelltarget}
\spellnotes The doubling of damage can be negated by a Heal check that beats your magic attack result.

\spellsection{Suggestion}{4}
\spellinfo{Ench (Compulsion) [Language-Dependent, Mind-Affecting, Sound-Dependent]}{Ench}
\spellcmp{Verbal only}
\spellrng{\rngclose}
\spelldur{\durext or until completed}
\spellsr{Yes}
\begin{spelltarget}{One creature}[Magic vs. Will]
    \spelleffect You influence the actions of the target creature by suggesting a course of activity (limited to a sentence or two). The suggestion must be worded in such a manner as to make the activity sound reasonable. Asking the creature to do some obviously harmful act automatically negates the effect of the spell. Additionally, any obvious threat, such as someone drawing a weapon, casting a spell, or aiming a ranged weapon at the fascinated creature, grants the creature a new saving throw with a \plus5 bonus.
    \par The suggested course of activity can continue for the entire duration. If the suggested activity can be completed in a shorter time, the spell ends when the target finishes what it was asked to do. You can instead specify conditions that will trigger a special activity during the duration. If the condition is not met before the spell duration expires, the activity is not performed.
\end{spelltarget}
\spellnotes A very reasonable suggestion can grant a \plus2 or greater bonus on the magic attack.

\norepeatspellnotes

\spellsectioncomma{Suggestion}{Mass}{8}
\spellinfo{Ench (Compulsion) [Language-Dependent, Mind-Affecting, Sound-Dependent]}{Ench}
\spelltwocol{\spelllimit{\areamed radius}}{\spellrng{\rngclose}}
\spelldur{\durmed}
\begin{spelltargets}{Up to five creatures in the area}[Magic vs. Will]
    \spelleffect The target obeys a suggestion, as \spell{suggestion}.
\end{spelltargets}
\spellnotes All targets must receive the same suggestion.

\spellsection{Summon Monster I}{1}\hypertarget{spell:summon monster}{}
\spellinfo{Conj (Summoning) [see text]}{Arcane, Divine}
\spelltwocol{\spelltime{Full-round action}}{\spellrng{\rngclose}}
\spelldur{\durshort \dismissable}
\spellsr{No}
\spellline
\spelleffect This spell summons an extraplanar creature (typically an outsider, elemental, or magical beast native to another plane). It appears where you designate and acts on your next turn. You must spend a swift action each round to control the creature summoned by this spell. If you do, it attacks your opponents to the best of its ability. You can direct the creature not to attack, to attack particular enemies, or to perform other actions if you can communicate with it. If you do not actively control the creature summoned by this spell, it acts according to its nature.
\par When you learn this spell, you choose two creatures from the 1st-level list on \tref{Summon Monster List}. You can summon those creatures with this or any other summon monster spell.
\par A summoned monster cannot summon or otherwise conjure another creature, nor can it use any teleportation or planar travel abilities. Creatures cannot be summoned into an environment that cannot support them.
\par When you use a summoning spell to summon an air, chaotic, earth, evil, fire, lawful, or water creature, it is a spell of that type.

\spellsection{Summon Monster II}{2}
\spellinfo{Conj (Summoning)}{Arcane, Divine}
\spelltwocol{\spelllimit{\areamed radius}}{\spellrng{\rngclose}}
\spelldur{\durshort \dismissable}
\spellsr{No}
\spellline
\spelleffect This spell functions like \spell{summon monster I}, except that you can summon one creature from the 2nd-level list or 1d3 creatures of the same kind from the 1st-level list. When you learn this spell, you choose two creatures from the 2nd-level list or lower on \tref{Summon Monster List}. You can summon those creatures with this or any other \spell{summon monster} spell.

\spellsection{Summon Monster III}{3}
\spellinfo{Conj (Summoning)}{Arcane, Chaos, Divine, Evil, Good, Law}
\spelltwocol{\spelllimit{\areamed radius}}{\spellrng{\rngclose}}
\spelldur{\durshort \dismissable}
\spellsr{No}
\spellline
\spelleffect This spell functions like \spell{summon monster I}, except that you can summon one creature from the 3rd-level list or 1d3 creatures of the same kind from a lower-level list. When you learn this spell, you choose two creatures from the 3rd-level list or lower on \tref{Summon Monster List}. You can summon those creatures with this or any other \spell{summon monster} spell.

\spellsection{Summon Monster IV}{4}
\spellinfo{Conj (Summoning)}{Arcane, Divine}
\spelltwocol{\spelllimit{\areamed radius}}{\spellrng{\rngclose}}
\spelldur{\durshort \dismissable}
\spellsr{No}
\spellline
\spelleffect This spell functions like \spell{summon monster I}, except that you can summon one creature from the 4th-level list or 1d3 creatures of the same kind from a lower-level list. When you learn this spell, you choose two creatures from the 4th-level list or lower on \tref{Summon Monster List}. You can summon those creatures with this or any other \spell{summon monster} spell.

\spellsection{Summon Monster V}{4}
\spellinfo{Conj (Summoning)}{Air, Arcane, Divine, Earth, Fire, Water}
\spelltwocol{\spelllimit{\areamed radius}}{\spellrng{\rngclose}}
\spelldur{\durshort \dismissable}
\spellsr{No}
\spellline
\spelleffect This spell functions like \spell{summon monster I}, except that you can summon one creature from the 5th-level list or 1d3 creatures of the same kind from a lower-level list. When you learn this spell, you choose two creatures from the 5th-level list or lower on \tref{Summon Monster List}. You can summon those creatures with this or any other \spell{summon monster} spell.

\spellsection{Summon Monster VI}{6}
\spellinfo{Conj (Summoning)}{Arcane, Chaos, Divine, Evil, Good, Law}
\spelltwocol{\spelllimit{\areamed radius}}{\spellrng{\rngclose}}
\spelldur{\durshort \dismissable}
\spellsr{No}
\spellline
\spelleffect This spell functions like \spell{summon monster I}, except you can summon one creature from the 6th-level list or 1d3 creatures of the same kind from a lower-level list. When you learn this spell, you choose two creatures from the 6th-level list or lower on \tref{Summon Monster List}. You can summon those creatures with this or any other \spell{summon monster} spell.

\spellsection{Summon Monster VII}{7}
\spellinfo{Conj (Summoning)}{Arcane, Divine}
\spelltwocol{\spelllimit{\areamed radius}}{\spellrng{\rngclose}}
\spelldur{\durshort \dismissable}
\spellsr{No}
\spellline
\spelleffect This spell functions like \spell{summon monster I}, except that you can summon one creature from the 7th-level list or 1d3 creatures of the same kind from a lower-level list. When you learn this spell, you choose two creatures from the 7th-level list or lower on \tref{Summon Monster List}. You can summon those creatures with this or any other \spell{summon monster} spell.

\spellsection{Summon Monster VIII}{7}
\spellinfo{Conj (Summoning)}{Air, Arcane, Divine, Earth, Fire, Water}
\spelltwocol{\spelllimit{\areamed radius}}{\spellrng{\rngclose}}
\spelldur{\durshort \dismissable}
\spellsr{No}
\spellline
\spelleffect This spell functions like \spell{summon monster I}, except that you can summon one creature from the 8th-level list or 1d3 creatures of the same kind from a lower-level list. When you learn this spell, you choose two creatures from the 8th-level list or lower on \tref{Summon Monster List}. You can summon those creatures with this or any other \spell{summon monster} spell.

\spellsection{Summon Monster IX}{9}
\spellinfo{Conj (Summoning)}{Arcane, Chaos, Divine, Evil, Good, Law}
\spelltwocol{\spelllimit{\areamed radius}}{\spellrng{\rngclose}}
\spelldur{\durshort \dismissable}
\spellsr{No}
\spellline
\spelleffect This spell functions like \spell{summon monster I}, except that you can summon one creature from the 9th-level list or 1d3 creatures of the same kind from a lower-level list. When you learn this spell, you choose two creatures from the 9th-level list or lower on \tref{Summon Monster List}. You can summon those creatures with this or any other \spell{summon monster} spell.

\begin{dtable!*}
    \lcaption{Summon Monster List}
    \begin{tabularx}{\textwidth}{>{\lcol}X c >{\lcol}X c >{\lcol}X c}
        \thead{1st Level} &  & \thead{4th Level} &  & Fiendish monstrous spider, Huge & CE \\
        Celestial dog & LG & Archon, lantern & LG & Fiendish snake, giant constrictor & CE \\
        Celestial owl & LG & Celestial giant owl & LG &  &  \\
        Celestial giant fire beetle & NG & Celestial giant eagle & CG & \thead{7th Level} &  \\
        Celestial porpoise\fn{1} & NG & Celestial lion & CG & Celestial elephant & LG \\
        Celestial badger & CG & Mephit (any)\fn{2} & N & Avoral (guardinal) & NG \\
        Celestial monkey & CG & Fiendish dire wolf & LE & Celestial baleen whale\fn{1} & NG \\
        Fiendish dire rat & LE & Fiendish giant wasp & LE & Djinni (genie) & CG \\
        Fiendish raven & LE & Fiendish giant praying mantis & NE & Elemental, Huge (any)\fn{2} & N \\
        Fiendish monstrous centipede, Medium & NE & Fiendish shark, Large\fn{1} & NE & Invisible stalker & N \\
        Fiendish monstrous scorpion, Small & NE & Yeth hound & NE & Devil, bone & LE \\
        Fiendish hawk & CE & Fiendish monstrous spider, Large & CE & Fiendish megaraptor & LE \\
        Fiendish monstrous spider, Small & CE & Fiendish snake, Huge viper & CE & Fiendish monstrous scorpion, Huge & \\ NE
        Fiendish octopus\fn{1} & CE & Howler & CE & Babau (demon) & CE \\
        Fiendish snake, Small viper & CE &  &  & Fiendish giant octopus\fn{1} & CE \\
        &  & \thead{5th Level} &  & Fiendish girallon & CE \\
        \thead{2nd Level} &  & Archon, hound & K &  &  \\
        Celestial giant bee & LG & Celestial brown bear & LG &  &  \\
        Celestial giant bombardier beetle & NG & Celestial giant stag beetle & LG & \thead{8th Level} &  \\
        Celestial riding dog & NG & Celestial sea cat\fn{1} & NG & Celestial dire bear & LG \\
        Celestial eagle & CG & Celestial griffon & NG & Celestial cachalot whale\fn{1} & NG \\
        Lemure (devil) & LE & Elemental, Medium (any)\fn{2} & CG & Celestial triceratops & NG \\
        Fiendish squid\fn{1} & LE & Achaierai & N & Lillend & CG \\
        Fiendish wolf & LE & Devil, bearded & LE & Elemental, greater (any)\fn{2} & N \\
        Fiendish monstrous centipede, Large & NE & Fiendish deinonychus & LE & Fiendish giant squid\fn{1} & LE \\
        Fiendish monstrous scorpion, Medium & NE & Fiendish dire ape & LE & Hellcat & LE \\
        Fiendish shark, Medium\fn{1} & NE & Fiendish dire boar & LE & Fiendish monstrous centipede, Colossal & NE \\
        Fiendish monstrous spider, Medium & CE & Fiendish shark, Huge & NE & Fiendish dire tiger & CE \\
        Fiendish snake, Medium viper & CE & Fiendish monstrous scorpion, Large & NE & Fiendish monstrous spider, Gargantuan & CE \\
        &  & Shadow mastiff & NE & Fiendish tyrannosaurus & CE \\
        \thead{3rd Level} &  & Fiendish dire wolverine & NE & Vrock (demon) & CE \\
        Celestial black bear & LG & Fiendish giant crocodile & CE &  &  \\
        Celestial bison & NG & Fiendish tiger & CE &  &  \\
        Celestial dire badger & CG &  &  & \thead{9th Level} &  \\
        Celestial hippogriff & CG & \thead{6th Level} &  & Couatl & LG \\
        Elemental, Small (any)\fn{2} & N & Celestial polar bear & LG & Leonal (guardinal) & NG \\
        Fiendish ape & LE & Celestial orca whale\fn{1} & NG & Celestial roc & CG \\
        Fiendish dire weasel & LE & Bralani (eladrin) & CG & Elemental, elder (any)\fn{2} & N \\
        Hell hound & LE & Celestial dire lion & CG & Devil, barbed & LE \\
        Fiendish snake, constrictor  & LE & Elemental, Large (any)\fn{2} & N & Fiendish dire shark\fn{1} & NE \\
        Fiendish boar & NE & Janni (genie) & N & Fiendish monstrous scorpion, Gargantuan & NE \\
        Fiendish dire bat & NE & Chaos beast & CN & Night hag & NE \\
        Fiendish monstrous centipede, Huge & NE & Devil, chain & LE & Bebilith (demon) & CE \\
        Fiendish crocodile & CE & Xill & LE & Fiendish monstrous spider, Colossal & CE \\
        Dretch (demon) & CE & Fiendish monstrous centipede, Gargantuan & NE & Hezrou (demon) & CE \\
        Fiendish snake, Large viper & CE & Fiendish rhinoceros & NE & & \\
        Fiendish wolverine & CE & Fiendish elasmosaurus\fn{1} & CE & &
    \end{tabularx}
    1 May be summoned only into an aquatic or watery environment. \\
    2 Each variety must be learned individually.
\end{dtable!*}

\spellsection{Summon Nature's Ally I}{1}\hypertarget{spell:summon nature's ally}{}
\spellinfo{Conj (Summoning)}{Nature}
\spelltime{Full-round action}
\spellrng{\rngclose}
\spelldur{\durshort \dismissable}
\spellsr{No}
\spellline
\spelleffect This spell summons a natural creature. It appears where you designate and acts on your next turn. You must spend a swift action each round to control the creature summoned by this spell. If you do, it attacks your opponents to the best of its ability. You can direct the creature not to attack, to attack particular enemies, or to perform other actions if you can communicate with it. If you do not actively control the creature summoned by this spell, it acts according to its nature.
\par When you learn this spell, you choose two creatures from the 1st-level list on \tref{Summon Nature's Ally List}. You can summon those creatures with this or any other \spell{summon nature's ally} spell.
\par A summoned monster cannot summon or otherwise conjure another creature, nor can it use any teleportation or planar travel abilities. Creatures cannot be summoned into an environment that cannot support them.
\par All the creatures on the table are neutral unless otherwise noted.

\spellsection{Summon Nature's Ally II}{2}
\spellinfo{Conj (Summoning)}{Nature}
\spelltwocol{\spelllimit{\areamed radius}}{\spellrng{\rngclose}}
\spelldur{\durshort \dismissable}
\spellsr{No}
\spellline
\spelleffect This spell functions like \spellindirect{summon nature's ally i}{summon nature's ally I}, except that you can summon one 2nd-level creature or 1d3 1st-level creatures of the same kind. When you learn this spell, you choose two creatures from the 2nd-level list or lower on \tref{Summon Nature's Ally List}. You can summon those creatures with this or any other \spell{summon nature's ally} spell.

\spellsection{Summon Nature's Ally III}{3}
\spellinfo{Conj (Summoning) [see text]}{Nature, Wild}
\spelltwocol{\spelllimit{\areamed radius}}{\spellrng{\rngclose}}
\spelldur{\durshort \dismissable}
\spellsr{No}
\spellline
\spelleffect This spell functions like \spellindirect{summon nature's ally i}{summon nature's ally I}, except that you can summon one 3rd-level creature, 1d3 2nd-level creatures of the same kind, or 1d4\plus1 1st-level creatures of the same kind. When you learn this spell, you choose two creatures from the 3rd-level list or lower on \tref{Summon Nature's Ally List}. You can summon those creatures with this or any other \spell{summon nature's ally} spell.

\spellsection{Summon Nature's Ally IV}{4}
\spellinfo{Conj (Summoning) [see text]}{Nature}
\spelltwocol{\spelllimit{\areamed radius}}{\spellrng{\rngclose}}
\spelldur{\durshort \dismissable}
\spellsr{No}
\spellline
\spelleffect This spell functions like \spellindirect{summon nature's ally i}{summon nature's ally I}, except that you can summon one 4th-level creature or 1d3 creatures of the same kind from a lower-level list. When you learn this spell, you choose two creatures from the 4th-level list or lower on \tref{Summon Nature's Ally List}. You can summon those creatures with this or any other \spell{summon nature's ally} spell.

\spellsection{Summon Nature's Ally V}{5}
\spellinfo{Conj (Summoning) [see text]}{Nature}
\spelltwocol{\spelllimit{\areamed radius}}{\spellrng{\rngclose}}
\spelldur{\durshort \dismissable}
\spellsr{No}
\spellline
\spelleffect This spell functions like \spellindirect{summon nature's ally i}{summon nature's ally I}, except that you can summon one 5th-level creature or 1d3 creatures of the same kind from a lower-level list. When you learn this spell, you choose two creatures from the 5th-level list or lower on \tref{Summon Nature's Ally List}. You can summon those creatures with this or any other \spell{summon nature's ally} spell.

\spellsection{Summon Nature's Ally VI}{6}
\spellinfo{Conj (Summoning) [see text]}{Nature, Wild}
\spelltwocol{\spelllimit{\areamed radius}}{\spellrng{\rngclose}}
\spelldur{\durshort \dismissable}
\spellsr{No}
\spellline
\spelleffect This spell functions like \spellindirect{summon nature's ally i}{summon nature's ally I}, except that you can summon one 6th-level creature or 1d3 creatures of the same kind from a lower-level list. When you learn this spell, you choose two creatures from the 6th-level list or lower on \tref{Summon Nature's Ally List}. You can summon those creatures with this or any other \spell{summon nature's ally} spell.

\spellsection{Summon Nature's Ally VII}{7}
\spellinfo{Conj (Summoning) [see text]}{Nature}
\spelltwocol{\spelllimit{\areamed radius}}{\spellrng{\rngclose}}
\spelldur{\durshort \dismissable}
\spellsr{No}
\spellline
\spelleffect This spell functions like \spellindirect{summon nature's ally i}{summon nature's ally I}, except that you can summon one 7th-level creature or 1d3 creatures of the same kind from a lower-level list. When you learn this spell, you choose two creatures from the 7th-level list or lower on \tref{Summon Nature's Ally List}. You can summon those creatures with this or any other \spell{summon nature's ally} spell.

\spellsection{Summon Nature's Ally VIII}{8}
\spellinfo{Conj (Summoning) [see text]}{Nature}
\spelltwocol{\spelllimit{\areamed radius}}{\spellrng{\rngclose}}
\spelldur{\durshort \dismissable}
\spellsr{No}
\spellline
\spelleffect This spell functions like \spellindirect{summon nature's ally i}{summon nature's ally I}, except that you can summon one 8th-level creature or 1d3 creatures of the same kind from a lower-level list. When you learn this spell, you choose two creatures from the 8th-level list or lower on \tref{Summon Nature's Ally List}. You can summon those creatures with this or any other \spell{summon nature's ally} spell.

\spellsection{Summon Nature's Ally IX}{9}
\spellinfo{Conj (Summoning) [see text]}{Nature, Wild}
\spelltwocol{\spelllimit{\areamed radius}}{\spellrng{\rngclose}}
\spelldur{\durshort \dismissable}
\spellsr{No}
\spellline
\spelleffect This spell functions like \spellindirect{summon nature's ally i}{summon nature's ally I}, except that you can summon one 9th-level creature or 1d3 creatures of the same kind from a lower-level list. When you learn this spell, you choose two creatures from the 9th-level list or lower on \tref{Summon Nature's Ally List}. You can summon those creatures with this or any other \spell{summon nature's ally} spell.

\begin{dtable*}
    \lcaption{Summon Nature's Ally List}
    \begin{tabularx}{\textwidth}{>{\lcol}X >{\lcol}X >{\lcol}X >{\lcol}X}
        \thead{1st Level} & Eagle, giant [NG] & \thead{5th Level} & \thead{7th Level} \\
        Dire rat & Lion & Arrowhawk, adult & Arrowhawk, elder \\
        Eagle (animal) & Owl, giant [NG] & Bear, polar (animal) & Dire tiger \\
        Monkey (animal) & Satyr [CN; without pipes] & Dire lion & Elemental, greater (any)\fn{2} \\
        Octopus\fn{1} (animal) & Shark, Large\fn{1} (animal) & Elasmosaurus\fn{1} (dinosaur) & Djinni (genie) [NG] \\
        Owl (animal) & Snake, constrictor (animal) & Elemental, Large (any)\fn{2} & Invisible stalker \\
        Porpoise\fn{1} (animal) & Snake, Large viper (animal) & Griffon & Pixie\fn{3} (sprite) [NG; with sleep arrows] \\
        Snake, Small viper (animal) & Thoqqua & Janni (genie) & Squid, giant\fn{1} (animal) \\
        Wolf (animal) &  & Rhinoceros (animal) & Triceratops (dinosaur) \\
        & \thead{4th Level} & Satyr [CN; with pipes] & Tyrannosaurus (dinosaur) \\
        \thead{2nd Level} & Arrowhawk, juvenile & Snake, giant constrictor (animal) & Whale, cachalot\fn{1} (animal) \\
        Bear, black (animal) & Bear, brown (animal) & Nixie (sprite) & Xorn, elder \\
        Crocodile (animal) & Crocodile, giant (animal) & Tojanida, adult\fn{1} &  \\
        Dire badger & Deinonychus (dinosaur) & Whale, orca\fn{1} (animal) & \thead{8th Level} \\
        Dire bat & Dire ape &  & Dire shark\fn{1} \\
        Elemental, Small (any)\fn{2} & Dire boar & \thead{6th Level} & Roc \\
        Hippogriff & Dire wolverine & Dire bear & Salamander, noble [NE] \\
        Shark, Medium\fn{1} (animal) & Elemental, Medium (any)\fn{2} & Elemental, Huge (any)\fn{2} & Tojanida, elder \\
        Snake, Medium viper (animal) & Salamander, flamebrother [NE] & Elephant (animal) &  \\
        Squid\fn{1} (animal) & Sea cat\fn{1} & Girallon & \thead{9th Level} \\
        Wolverine (animal) & Shark, Huge\fn{1} (animal) & Megaraptor (dinosaur) & Elemental, elder \\
        & Snake, Huge viper (animalo) & Octopus, giant\fn{1} (animal) & Grig [NG; with fiddle] (sprite) \\
        \thead{3rd Level} & Tiger (animal) & Pixie\fn{3} (sprite) [NG; no special arrows] & Pixie\fn{4} (sprite) [NG; with sleep and memory loss arrows] \\
        Ape (animal) & Tojanida, juvenile\fn{1} & Salamander, average [NE] & Unicorn, celestial charger \\
        Dire weasel & Unicorn [CG] & Whale, baleen\fn{1} &  \\
        Dire wolf & Xorn, minor & Xorn, average & 
    \end{tabularx}
    1 May be summoned only into an aquatic or watery environment. \\
    2 Each variety must be learned individually. \\
    3 Can't cast irresistible dance \\
    4 Can cast irresistible dance \\
\end{dtable*}

\spellsection{Summon Nature's Army}{8}
\spellinfo{Conj (Summoning)}{Nature, Wild}
\spelltwocol{\spelllimit{\areamed radius}}{\spellrng{\rngclose}}
\spelldur{\durshort \dismissable}
\spellsr{No}
\spellline
\spelleffect This spell functions like \spellindirect{summon nature's ally i}{summon nature's ally I}, except that you can summon up to one creature per caster level from the 4th-level list or lower.
\par When you learn this spell, you choose a creature from the 4th-level list or lower on the Summon Nature's Ally table. That is the only creature you can summon with this spell.

\spellsection{Sunbeam}{5}
\spelldesc{You evoke a dazzling beam of intense light, blinding your foes with the power of the sun itself.}
\spellinfo{Evoc (Control) [Light]}{Nature}
\spellburst{\arealarge line, 10 ft. wide}
\spellsr{Yes}
\begin{spelltarget}{Everything in the area}[Magic vs. Reflex]
    \spellsuccess 5d6 solar damage \add d6 per four caster levels above 10th.

    If the target is vulnerable to sunlight, it is also \blinded for 5 rounds.
    \spellfailure As above, but half damage, and the target is not blinded.
\end{spelltarget}

\spellsection{Sunburst}{8}
\spelldesc{You cause a globe of searing radiance to explode silently from a point you select.}
\spellinfo{Evoc (Control) [Light]}{Nature}
\spelltwocol{\spellburst{\areamed radius}}{\spellrng{\rngmed}}
\begin{spelltarget}{Everything in the area}[Magic vs. Reflex]
    \spellsuccess 8d6 solar damage \add d6 per four caster levels above 1l6th.

    If the target is vulnerable to sunlight, it is also \blinded for 5 rounds.
    \spellfailure As above, but half damage, and the target is not blinded.
\end{spelltarget}

\pdfbookmark[2]{T}{SpellDescriptionsT}
\begin{comment}
\subsubsection{T}
\end{comment}

\spellsection{Telekinesis}{6}
\spelldesc{You move objects or creatures by concentrating on them.}
\spellinfo{Evoc (Control)}{Evoc}
\spellrng{\rngmed}
\spelldur{Concentration or until discharged}
\spellline
\spellspecial Each round, you choose one of three effects.
\spelleffect[Sustained Force] As \spell{telekinetic force}.
\spelleffect[Combat Maneuver] As \spell{telekinetic maneuver}.
\spelleffect[Violent Thrust] As \spell{telekinetic thrust}. Using this effect ends the spell.

\spellsection{Telekinetic Force}{4}
\spellinfo{Evoc (Control)}{Evoc}
\spellrng{\rngmed}
\spelldur{Concentration}
\spellsr{Yes}
\begin{spelltrigger}{End of every round}
    \begin{spelltarget}*{One creature or object}[Magic vs. Will]
        \spellsuccess You can take a standard action using the target as if you were holding it in your hands. Your effective Strength is equal to your Charisma, and your effective Dexterity is equal to your Intelligence. You can move the target up to thirty feet per round.
    \end{spelltarget}
\end{spelltrigger}
\spellnotes If a target resists your attempt to control it, it (and its equipment) is immune to any further attempts you make for the duration of the spell.

\spellsection{Telekinetic Maneuver}{3}
\spellinfo{Evoc (Control)}{Evoc}
\spellrng{\rngmed}
\spelldur{Concentration}
\spellsr{Yes}
\spellline
\begin{spelltrigger}{End of every round}
    \spellspecial Choose one of the following combat maneuvers: disarm, dirty trick, grapple, shove, or trip.
    \begin{spelltarget}*{One creature or object}l[Caster level \add casting attribute vs. Maneuver defense]
        \spellsuccess The target is affected by the chosen maneuver.
    \end{spelltarget}
\end{spelltrigger}

\spellsection{Telekinetic Thrust}{5}
\spellinfo{Evoc (Control)}{Evoc}
\spelltwocol{\spelllimit{\areamed radius}}{\spellrng{\rngmed}}
\spellsr{Yes}
\begin{spelltarget}{One creature or object}[Magic vs. Will]
    \spellsuccess You throw the target in any direction a number of feet equal to 50 \add 5 per two caster levels above 10th. You can only throw it half as far vertically.

    If the target strikes a solid obstacle, such as another creature, both the target and the struck obstacle take 1d6 physical bludgeoning damage per 10 feet of movement the target had left to travel.
\end{spelltarget}

\spellsection{Telepathy}{5}
\spellinfo{Div (Communication)}{Arcane}
\spelldur{\durlong}
\begin{spelltarget}{You}
    \spelleffect You gain telepathy out to a range of 100 feet. This allows you to send mental messages to any creature within range that has a language. Non-telepathic creatures can reply mentally to your messages, but they cannot initiative a telepathic conversation with you.

    You can address multiple creatures at once with telepathy, but maintaining separate mental conversations is just as difficult as simultaneously speaking and listening to multiple creatures at the same time. 
\end{spelltarget}

\spellsection{Temporal Stasis}{8}
\spellinfo{Trans (Temporal)}{Arcane}
\spellrng{5 feet}
\spelldur{\durlong/Permanent}
\spellsr{Yes}
\begin{spelltarget}{One creature}[Magic vs. Will]
    \spellsuccess The target is \slowed for a \durlong duration.

    If the target is \bloodied, it is instead permanently placed into a state of suspended animation. For the target, time ceases to flow and its condition becomes fixed. It does not grow older. Its body functions virtually cease, and no force or effect can move or harm it.
\end{spelltarget}

\spellsection{Time Stop}{9}
\spellinfo{Trans (Temporal)}{Arcane}
\spelldur{1d3\plus1 rounds (apparent time); see text}
\spellline
\spelleffect This spell seems to make time cease to flow for everyone but you. In fact, you step into an alternate timestream, causing you to speed up so greatly that all other creatures seem frozen, though they are actually still moving at their normal speeds. You are free to act for 1d3\plus1 rounds of apparent time. You are still \vulnerable to danger, such as from heat or dangerous gases, but your actions have no effect on anything in the world other than yourself. Objects and creatures appear frozen in place. You cannot cast spells that affect any targets except yourself; the temporal magic is too strong to permit interference from lesser magic, and attempts to cast magic beyond the accelerated time surrounding you simply fail. The only exception is for temporal spells, which can be cast normally inside a \spell{time stop}. The targets are not affected and do not attempt to resist the effects until the end of the \spell{time stop}, so you do not know whether they are affected by any spells you cast until the effect has expired.
\spellnotes Most spellcasters use the additional time to improve their defenses or flee from combat. You are undetectable while \spell{time stop} lasts. You cannot enter an area protected by an \spell{antimagic field} while under the effect of \spell{time stop}.

\spellsection{Totemic Mind}{2}
\spelldesc{You grant your ally the mental prowess of a totem animal.}
\spellinfo{Trans (Augment)}{Arcane, Divine, Nature}
\spellrng{\rngtouch}
\spelldur{\durshort}
\spellsr{Yes}
\begin{spelltarget}{One creature}
    \spelleffect The target gains a \plus2 enhancement bonus to a mental attribute. The attribute depends on which version is chosen.
    \par \subspell{Eagle's Splendor}: The transmuted creature becomes more persuasive and personally forceful, gaining a bonus to Charisma.
    \par \subspell{Fox's Cunning}: The transmuted creature becomes smarter, gaining a bonus to Intelligence.
    \par \subspell{Owl's Wisdom}: The transmuted creature becomes more perceptive, gaining a bonus to Wisdom.
\end{spelltarget}

\spellsectioncomma{Totemic Mind}{Greater}{5}
\spellinfo{Trans (Augment)}{Arcane, Divine, Nature}
\spellrng{\rngtouch}
\spelldur{\durshort}
\spellsr{Yes}
\begin{spelltarget}{One creature}
    \spelleffect The target's mind improves, as \spell{totemic mind}, except that it gains a \plus4 enhancement bonus. Alternately, you can grant the target a \plus2 enhancement bonus to all its mental attributes.
\end{spelltarget}

\spellsectioncomma{Totemic Mind}{Mass}{6}
\spellinfo{Trans (Augment)}{Arcane, Divine, Nature}
\spelltwocol{\spelllimit{\areamed radius}}{\spellrng{\rngmed}}
\spellsr{Yes}
\begin{spelltargets}{Up to five creatures in the area}
    \spelleffect The target's mind improves, as \spell{totemic mind}. 
\end{spelltargets}
\spellnotes All affected creatures must gain a bonus to the same attribute.

\spellsection{Totemic Power}{2}
\spelldesc{You grant your ally the physical prowess of a totem animal.}
\spellinfo{Trans (Augment)}{Arcane, Divine, Nature, Strength}
\spellrng{\rngtouch}
\spelldur{\durshort}
\spellsr{Yes}
\begin{spelltarget}{One creature}
    \spelleffect The target gains a \plus2 enhancement bonus to a physical attribute. The attribute depends on which version is chosen.
    \par \subspell{Bear's Endurance}: The transmuted creature gains greater vitality and stamina, gaining a bonus to Constitution. Hit points gained by a temporary increase in Constitution score are not temporary hit points. They go away when the target's Constitution drops back to normal. They are not lost first as temporary hit points are.
    \par \subspell{Bull's Strength}: The transmuted creature becomes stronger, gaining a bonus to Strength.
    \par \subspell{Cat's Grace}: The transmuted creature becomes more graceful, agile, and coordinated, gaining a bonus to Dexterity.
\end{spelltarget}

\spellsectioncomma{Totemic Power}{Greater}{5}
\spellinfo{Trans (Augment)}{Arcane, Divine, Nature, Strength}
\spellrng{\rngtouch}
\spelldur{\durshort}
\spellsr{Yes}
\begin{spelltarget}{One creature}
    \spelleffect The target's body improves, as \spell{totemic power}, except that it gains a \plus4 enhancement bonus. Alternately, you can grant the target a \plus2 enhancement bonus to all its physical attributes.
\end{spelltarget}

\spellsectioncomma{Totemic Power}{Mass}{6}
\spellinfo{Trans (Augment)}{Arcane, Divine, Nature}
\spelltwocol{\spelllimit{\areamed radius}}{\spellrng{\rngmed}}
\spellsr{Yes}
\begin{spelltargets}{Up to five creatures in the area}
    \spelleffect The target's body improves, as \spell{totemic power}. 
\end{spelltargets}
\spellnotes All affected creatures must gain a bonus to the same attribute.

\spellsection{Touch of Idiocy}{2}
\spelldesc{With a touch, you reduce the target's mental faculties.}
\spellinfo{Ench (Inhibition) [Mind-Affecting]}{Arcane}
\spellrng{5 feet}
\spelldur{\durshort}
\spellsr{Yes}
\begin{spelltarget}{One creature}[Magic vs. Reflex and Will]
    \spellsuccess[Reflex] The target takes a \minus4 penalty to its Intelligence, Wisdom, and Charisma. This penalty can't reduce any of these attributes below \minus9.
    \spellfailure[Will] As above, but the penalty is halved.
\end{spelltarget}
\spellnotes This spell's effect may make it impossible for the target to cast some or all of its spells, if its casting attribute drops below the minimum required to cast spells of that level.

\spellsection{Transmute Any Object}{9}
\spellinfo{Trans (Alteration, Polymorph)}{Arcane}
\spellrng{\rngmed}
\spellsr{Yes}
\spellspecial This spell can be used to duplicate the effects of \spell{fabricate}, \spell{passwall}, \spell{shape metal}, \spell{shape stone}, \spell{shape wood}, \spell{transmute flesh and stone}, or \spell{wall of stone}. The object or creature to be transformed must meet any requirements of the spell being duplicated, other than range.

\spellsection{Transmute Flesh and Stone}{6}
\spellinfo{Trans (Polymorph)}{Arcane, Earth}
\spellrng{\rngmed}
\spellsr{Yes}
\spellspecial This spell has two versions: transmuting flesh into stone, and transmuting stone into flesh. Its effects depend on which version is chosen.
\begin{spelltarget}{One creature (Huge or smaller)}[Magic vs. Fortitude]
    \spellspecial If the target is not made of flesh (such as a golem), it is unaffected.
    \spellsuccess 6d6 physical damage immediately, and 3d6 physical damage at the end of each round after the first.

    If the target loses all its hit points, it becomes a mindless, inert statue, along with all its equipment. If the statue resulting from this effect is broken or damaged, the target (if ever returned to its original state) has similar damage or deformities. The target is neither alive nor dead in this state.
    \spellfailure As above, but half damage.
\end{spelltarget}

\begin{spelltarget}{One petrified creature (Huge or smaller)}
    \spellsuccess The target is restored to its normal state, including its equipment. Stone which was not originally a petrified creature is unaffected.
\end{spelltarget}

\spellsection{Tree Shape}{2}
\spellinfo{Trans (Polymorph)}{Nature}
\spelldur{\durext \dismissable}
\begin{spelltarget}{You}
    \spelleffect You become able to assume the form of a Large living tree or shrub or a Large dead tree trunk with a small number of limbs. The closest inspection cannot reveal that the tree in question is actually a magically concealed creature. To all normal tests you are, in fact, a tree or shrub, although a Spellcraft check can reveal the magical aura on the tree. While in tree form, you can observe all that transpires around you just as if you were in your normal form. You gain a \plus10 enhancement bonus to your armor modifier, but you have an effective Dexterity score of \minus10 and cannot move. You are immune to critical hits while in tree form. All clothing and gear carried or worn changes with you.
\end{spelltarget}

\spellsection{Tremorsense}{1}
\spellinfo{Trans (Imbuement)}{Nature, Earth}
\spelldur{Concentration}
\begin{spelltarget}{You}
    \spelleffect You gain the tremorsense ability with a range of 50 feet. If you are touching a surface, you can automatically pinpoint the location of anything within 50 feet that is in contact with the surface, including inanimate objects.
\end{spelltarget}
\spellnotes Tremorsense functions on surfaces of any kind, regardless of lighting conditions.

\spellsection{True Seeing}{6}
\spelldesc{You grant your ally the ability to see all things as they actually are.}
\spellinfo{Div (Awareness)}{Arcane, Divine, Knowledge}
\spellcmp{Verbal, Somatic, and Material}
\spellrng{Touch}
\spelldur{\durshort}
\spellsr{Yes}
\begin{spelltarget}{One creature}
    \spelleffect The target sees through normal and magical darkness, notices secret doors hidden by magic, sees the truth behind visual figments and glamers, and sees the true form of polymorphed, changed, or transmuted things. In addition, the target can see into the Ethereal Plane (but not into extradimensional spaces). The effect extends out to \rngmed range.
\end{spelltarget}
\spellnotes This effect does not allow the target to see into or through solid objects. It does not negate concealment, including that caused by fog and the like. True seeing does not help the viewer see through mundane disguises, spot creatures who are simply hiding, or notice secret doors hidden by mundane means. In addition, the spell effects cannot be further enhanced with known magic, so one cannot use \spell{true seeing} through a scrying effect.
\spellmat{An ointment for the eyes that costs 100 gp and is made from mushroom powder, saffron, and fat.}

\spellsection{True Strike}{6}
\spelldesc{You grant your ally a temporary, intuitive insight into the immediate future during their next attack.}
\spellinfo{Div (Knowledge)}{Arcane}
\spelltwocol{\spelltime{1 swift action}}{\spellcmp{Verbal only}}
\spellrng{\rngmed}
\spelldur{1 round or until discharged}
\begin{spelltarget}{One creature}
    \spelleffect The target's next physical attack gains a \plus20 enhancement bonus, and ignores all miss chances. 
\end{spelltarget}
\spellnotes After casting this spell, you cannot cast it again for 5 rounds.

\pdfbookmark[2]{U-Z}{SpellDescriptionsU-Z}
\begin{comment}
\subsubsection{U-Z}
\end{comment}

\spellsection{Unholy Aura}{8}
\spelldesc{You shield your allies with malevolent darkness, protecting them from good foes.}
\spellinfo{Abjur (Interdiction) [Evil]}{Divine, Evil}
\spelllimit{\areamed radius centered on you}
\spelldur{\durshort \dismissable}
\spellsr{Yes}
\begin{spelltargets}{Up to five creatures in the area}
    The target gains a \plus5 enhancement bonus to its defenses. In addition, it gains spell resistance against good spells and spells cast by good creatures.
    \par At the end of each round, all good creatures within \rngclose range of the target that attacked it with their body or a melee weapon that round take 4d6 points of damage. A creature that attacks multiple creatures shielded by this spell can take this damage multiple times.
\end{spelltargets}
\spellline

\spellsection{Unholy Blight}{4}
\spellinfo{Evoc (Channeling) [Evil]}{Evil}
\spellrng{\rngmed}
\spelldur{5 rounds}
\spellsr{Yes}
\begin{spelltarget}{One nonevil creature}[Magic vs. Will]
    \spellsuccess 8d6 divine damage \add d6 per two caster levels above 8th, and the target is \sickened for 5 rounds.
    \spellfailure As above, but half damage.
\end{spelltarget}

\spellsection{Unliving Eyes}{3}
\spellinfo{Div/Necro (Awareness, Life)}{Arcane}
\spellrng{Touch}
\spelldur{\durlong \dismissable}
\spellsr{Yes}
\begin{spelltarget}{One creature}
    \spelleffect The target gains the ability to ``see'' any living creatures and their equipment within 30 feet perfectly, regardless of lighting conditions, physical barriers, invisibility, or any other means of concealment.

    If the target is undead, the range of the vision is doubled to 60 feet.
\end{spelltarget}

\spellsection{Unliving Heart}{1}
\spelldesc{You harness the power of unlife to grant yourself a limited ability to avoid death.}
\spellinfo{Necro (Life)}{Necro}
\spelldur{\durlong}
\begin{spelltarget}{You}
    \spelleffect You gain 5 temporary hit points \add 1 per two caster levels above 2nd. If you take life damage, you lose all temporary hit points provided by this spell before applying the damage.

    In addition, you are treated as being undead for the purpose of spells or abilities which affect undead. This can cause some unintelligent undead, such as skeletons and zombies, to avoid attacking you.
\end{spelltarget}

\spellsection{Ventriloquism}{1}
\spellinfo{Illus (Figment)}{Arcane, Trickery}
\spellcmp{Somatic only}
\spellrng{\rngmed}
\spelldur{\durshort \dismissable}
\spellsr{No}
\begin{spelltarget}{You}
    \spelleffect Your voice (or any sound that you can normally make vocally) originates from another location within range. As a swift action, you can change the apparent origin of your voice.
\end{spelltarget}

\begin{comment}
\spellsection{Vestments of the Mage}{2}
\spelldesc{You imbue a set of armor with magical power, preventing it from interfering with your spellcasting.}
\spellinfo{Trans (Imbuement)}{Arcane}
\spelltwocol{\spelltgt{One nonmagical armor or shield}}{\spellrng{Touch}}
\spelldur{\durext \dismissable}
\spelleffect The armor or shield's chance of arcane spell failure decreases by 10\% as long as you are wearing or using it. If any other creature wears the armor, it receives no benefit from this spell.
\spellnotes This decrease is considered an enhancement enhancement bonus.
\spellsr{Yes}
\end{comment}

%complicated
\spellsection{Wail of the Banshee}{9}
\spelldesc{You emit a terrible scream that kills anyone that hears it.}
\spellinfo{Necro (Life) [Death, Sound-Dependent]}{Death, Necro}
\spellcmp{Verbal only}
\spellburst{\arealarge radius centered on you}
\spelldur{Concentration, up to 2 rounds; see text}
\spellsr{Yes}
\begin{spelltargets}{Living creatures in the area}[Magic vs. Fortitude]
    \spellsuccess The target is \staggered for 1 round.
\end{spelltargets}
\spellspecial If you concentrate for a second round, you make another attack.
\begin{spelltargets}{Bloodied living creatures in the area}[Magic vs. Fortitude]
    \spellsuccess The target loses all its hit points and takes critical damage equal to your caster level, causing it to begin dying.
\end{spelltargets}

\spellsection{Wall of Fire}{4}
\spellinfo{Evoc (Energy) [Fire, Wall]}{Arcane, Nature, Fire}
\spelltwocol{\spellzone{100 ft. wall, 10 ft. high \shapeable}}{\spellrng{\rngmed}}
\spelldur{\durshort}
\spellsr{Yes}
\spellline
\spelleffect This spell creates a wall made of fire.
\begin{spelltrigger}{A creature passes through the wall}
    \begin{spelltarget}*{Creature in wall}[Magic vs. Reflex]
        \spellsuccess 4d6 fire damage \add d6 per four caster levels above 10th.
        \spellfailure As above, but half damage.
    \end{spelltarget}
\end{spelltrigger}
\spellnotes Any part of the wall takes 20 cold damage is extinguished.

This spell can be made permanent with a \spell{permanency} ritual. A permanent \spell{wall of fire} that is extinguished by cold damage becomes inactive for 10 minutes, then reforms at normal strength.

\spellsection{Wall of Force}{6}
\spellinfo{Evoc (Control) [Force, Wall]}{Arcane}
\spelltwocol{\spellzone{100 ft. solid wall, 10 ft. high}}{\spellrng{\rngmed}}
\spelldur{\durshort \dismissable}
\spellsr{No}
\spellline
\spelleffect This spell creates an invisible wall made of force. A 5-foot square of wall has 5 hit points per caster level, and hardness equal to your caster level.
\spellnotes \forcespellnotes

This spell can be made permanent with a \spell{permanency} ritual.

\spellsection{Wall of Stone}{5}
\spellinfo{Conj/Trans (Alteration, Creation) [Earth, Wall]}{Arcane, Earth, Nature}
\spelltwocol{\spellzone{\arealarge solid wall, 5 ft. high \shapeable}}{\spellrng{\rngmed}}
\spellsr{No}
\spellline
\spelleffect This spell creates a wall made of stone atop an existing rock surface. A 5-foot square of wall has 60 hit points, and hardness 8. The wall is four inches thick.
\par Unlike most walls, a \spell{wall of stone} need not be vertical. It need not rest entirely on solid ground, as long as it is solidly supported by existing stone.
\par A wall of stone can be crudely shaped to add crenellations, battlements, and so forth.
\spellnotes Once created, the stone is nonmagical, and can be destroyed like any other stone.

\spellsection{Wall of Thorns}{5}
\spellinfo{Conj (Creation) [Wall]}{Nature, Wild}
\spelltwocol{\spellzone{100 ft. line, 5 ft. wide, 10 ft. high \shapeable}}{\spellrng{\rngmed}}
\spelldur{\durlong \dismissable}
\spelleffect This spell creates a thicket of thorns in the area. A 5-foot cube of wall has 10 hit points per caster level. Moving into or out of a square in the area costs 20 feet of movement. The wall can be created where creatures are.

The wall provides total cover against attacks through the wall. A creature in the wall has cover from attacks on either side of the wall.
\begin{spelltrigger}{A creature enters or exits a square in the area}
    \begin{spelltarget}{The moving creature}
        \spelleffect 3d6 physical piercing damage.
    \end{spelltarget}
\end{spelltrigger}
\spellnotes A \spell{wall of thorn} can be breached by slow work with edged weapons or fire. It has hardness 8 and 30 hit points per square foot of thickness.

\spellsection{Water Walk}{3}
\spellinfo{Trans (Imbuement) [Water]}{Druid, Water}
\spellrng{Touch}
\spelldur{\durlong \dismissable}
\spellsr{Yes}
\begin{spelltargets}{Up to five creatures}
    \spelleffect The target can tread on any liquid as if it were firm ground. Mud, oil, snow, quicksand, running water, ice, and even lava can be traversed easily, since the target's feet hover an inch or two above the surface.
    \par If the target is underwater, it rises toward the surface at 60 feet per round it can stand on it.
\end{spelltargets}

\spellsection{Waves of Exhaustion}{8}
\spellinfo{Necro (Flesh)}{Arcane, Death, War}
\spellburst{\arealarge cone}
\spelldur{\durshort}
\spellsr{Yes}
\begin{spelltargets}{All creatures in the area}[Magic vs. Fortitude]
    \spellsuccess The target is \exhausted.
    \spellfailure The target is \fatigued.
\end{spelltargets}

\spellsection{Waves of Fatigue}{5}
\spellinfo{Necro (Flesh)}{Arcane, Death, War}
\spellburst{\arealarge cone}
\spelldur{\durshort}
\spellsr{Yes}
\begin{spelltargets}{All creatures in the area}
    \spelleffect The target is \fatigued.
\end{spelltargets}

\spellsection{Web}{3}
\spelldesc{You create a many-layered mass of strong, stricky strands that entangle creatures caught within them. The strands are similar to spider webs, but larger and tougher.}
\spellinfo{Conj (Creation)}{Arcane}
\spelltwocol{\spellzone{\areamed radius}}{\spellrng{\rngclose}}
\spelldur{\durshort \dismissable}
\spellsr{No}
\begin{spelltrigger}{A creature in the area moves}
    \begin{spelltarget}{The moving creature}[Magic vs. Reflex]
        \spellsuccess The target is \entangled. As a standard action, it can make a grapple attack or Escape Artist check. If its result beats your attack result, it is no longer entangled.
    \end{spelltarget}
\end{spelltrigger}
\spellnotes The strands are too widely spaced to significantly obscure sight, but are flammable. Any five-foot square that takes 5 points of fire damage is destroyed. A magic flaming sword can slash them away as easily as a hand brushes away cobwebs. Any fire can set the webs alight and burn away 5 square feet in 1 round. All creatures within flaming webs take 2d4 points of fire damage from the flames.

This spell can be made permanent with a \spell{permanency} ritual. A permanent \spell{web} that is destroyed regrows in 10 minutes.

\spellsection{Weird}{9}
\spellinfo{Ench/Illus (Emotion, Phantasm) [Death, Fear, Mind-Affecting, Unreal]}{Arcane, Trickery}
\spelltwocol{\spelllimit{\areamed radius}}{\spellrng{\rngmed}}
\begin{spelltarget}{Five creatures in the area}[Magic vs. Will and Fortitude]
    \spellsuccess[Will] The target is \shaken for 5 rounds.

    \spellsuccess[Will and Fortitude] If the target is \bloodied, it loses all its hit points and takes critical damage equal to your caster level, causing it to begin dying.
\end{spelltarget}

\spellsection{Windstrike}{2}
\spelldesc{You command the air to bludgeon the target, sending it flying.}
\spellinfo{Evoc (Control) [Air]}{Air, Nature}
\spellrng{\rngmed}
\spellsr{Yes}
\begin{spelltarget}{One creature or object}l[Magic vs. Fortitude]
    \begin{spellmargin}
        \spellsuccess 4d6 bludgeoning damage \add d6 per two caster levels above 4th.
        \spellfailure Half damage.
    \end{spellmargin}
    \spellattack Caster level \add casting attribute vs. Maneuver defense (shove)
    \begin{spellmargin}
        \spellsuccess You shove the target in any direction -- even vertically. Moving the target up takes twice as much movement as moving the target horizontally.
    \end{spellmargin}
\end{spelltarget}

\spellsection{Windstrike, Greater}{5}
\spelldesc{You command the air to bludgeon the target with tremendous force, sending it flying.}
\spellinfo{Evoc (Control) [Air]}{Air, Nature}
\spellrng{\rngmed}
\spellsr{Yes}
\begin{spelltarget}{One creature or object}l[Magic vs. Fortitude]
    \begin{spellmargin}
        \spellsuccess 10d6 bludgeoning damage \add d6 per two caster levels above 10th.
        \spellfailure Half damage.
    \end{spellmargin}
    \spellattack Caster level \add casting attribute \add 12 vs. Maneuver defense (shove)
    \begin{spellmargin}
        \spellsuccess You shove the target in any direction -- even vertically.
    \end{spellmargin}
\end{spelltarget}

\spellsection{Wish}{9}
\spellinfo{Universal}{Arcane, Magic}
\spellcmp{Verbal, Somatic, and Material}
\spelltwocol{\spelltgteffarea{See text}}{\spellrng{See text}}
\spelldur{See text}
\spellattack{See text}
\spelleffect This spell is the mightiest spell a wizard or sorcerer can cast. By simply speaking your desires aloud, you can alter reality to better suit you.
\par Even wish, however, has its limits.
\par A wish can produce any one of the following effects.
\begin{itemize}
    \item Duplicate any general wizard or sorcerer spell of 8th level or lower, provided the spell is not of a school prohibited to you.
    \item Duplicate any general wizard or sorcerer spell of 7th level or lower even if it's of a prohibited school.
    \item Duplicate any other spell of 6th level or lower, provided the spell is not of a school prohibited to you.
    \item Duplicate any other spell of 5th level or lower even if it's of a prohibited school. 
    \item Undo the harmful effects of many other spells, such as geas/quest or insanity.
    \item Create a nonmagical item of up to 10,000 gp in value.
    \item Create a magic item, or add to the powers of an existing magic item.
    \item Remove injuries and afflictions. A single wish can aid one creature per caster level, and all subjects are cured of the same kind of affliction. For example, you could heal all the damage you and your companions have taken, or remove all poison effects from everyone in the party, but not do both with the same wish. A wish can never restore the experience point loss from casting a spell or the level or Constitution loss from being resurrected.
    \item Revive the dead. A wish can bring a dead creature back to life by duplicating a resurrection spell. A wish can revive a dead creature whose body has been destroyed, but the task takes two wishes, one to recreate the body and another to infuse the body with life again.
    \item Transport travelers. A wish can lift one creature per caster level from anywhere on any plane and place those creatures anywhere else on any plane regardless of local conditions. You must make a Will attack to affect unwilling targets.
    \item Undo misfortune. A wish can undo a single recent event. The wish forces a reroll of any roll made within the last round (including your last turn). Reality reshapes itself to accommodate the new result. For example, a wish could undo a foe's successful critical hit (either the attack roll or the critical roll), a friend's failed attack, and so on. The reroll, however, may be as bad as or worse than the original roll. You must make a Will attack to affect an unwilling target.
\end{itemize}
\par When casting a wish, you do not specify the exact spell or effect you wish to duplicate. Instead, you make a wish, describing what you want to have happen, and make a DC 20 Wisdom check. If the check fails, your intent is redirected or perverted in some way. For example, a wish to turn a foe to stone would normally mimic the flesh to stone effect of the transmute flesh and stone spell. However, if the Wisdom check failed, your foe might gain the benefit of a \spell{stoneskin} spell instead.
\par You may try to use a wish to produce greater effects than these, but doing so is dangerous. The DC of the Wisdom check increases to 25, and the negative consequences for failing the check increase in proportion to the potency of the effect you try to create.
\spellmat{10,000gp of diamonds. In addition, when a \spell{wish} duplicates a spell with a material component that costs more than 10,000 gp, you must provide that component.}
\spellsr{Yes}

\spellsection{Word of Chaos}{7}
\spellinfo{Evoc (Channeling) [Chaotic]}{Chaos}
\spellcmp{Verbal only}
\spellburst{\arealarge radius centered on you}
\spellsr{Yes}
\begin{spelltargets}{All nonchaotic creatures in the area}
    \spelleffect If the target's level does not exceed your caster level, it is \bewildered for 5 rounds.

    If the target is also \bloodied, it also suffers one or more of the following ill effects, depending on its level.
    \begin{itemize}
        \item Up to caster level \minus5: The target is also \confused for 1 round.
        \item Up to caster level \minus10: The target is also \paralyzed for 5 rounds.
        \item Up to caster level \minus15: A living target loses all its hit points and takes critical damage equal to your caster level, causing it to begin dying. A nonliving target is destroyed.
    \end{itemize}
\end{spelltargets}

\spellsection{Word of Recall}{6}
\spellinfo{Conj (Translocation) [Teleportation]}{Divine}
\spellcmp{Verbal only}
\spellrng{Unlimited \rngunrestricted}
\begin{spelltarget}{You}
    \spelleffect This spell teleports you instantly back to your sanctuary. You must designate the sanctuary when you ready the spell for the day, and it must be a very familiar place. The actual point of arrival is a designated area no larger than 10 feet by 10 feet. You can be transported any distance within a plane but cannot travel between planes. You can transport, in addition to yourself, any objects you carry, as long as their weight doesn't exceed your maximum load. Exceeding this limit causes the spell to fail.
\end{spelltarget}

\spellsection{Zephyr Blade}{3}
\spelldesc{You imbue a weapon with the power of the wind, allowing it to manipulate air currents as it strikes.}
\spellinfo{Evoc/Trans (Augment, Control) [Air]}{Air, Nature}
\spellrng{Touch}
\spelldur{\durshort}
\spellsr{Yes}
\begin{spelltarget}{One melee weapon}
    \spelleffect The target weapon is enhanced, as \spell{magic weapon}. In addition, it also gains an additional five feet of reach, extending the wielder's threatened area. Attacks outside the weapon's normal range deal half damage, but are otherwise treated exactly as if the wielder was attacking with the weapon normally.
\end{spelltarget}
\spellnotes Despite the name of the spell, it can affect melee weapons of any type, even reach weapons. The weapon's extended reach is visible, and opponents can defend themselves normally against the attacks.

\spellsectioncomma{Zephyr Blade}{Greater}{6}
\spelldesc{You imbue a weapon with the full might of the wind, allowing it to shred opponents with nothing but the air itself.}
\spellinfo{Evoc/Trans (Augment, Control) [Air]}{Air, Nature}
\spellrng{Touch}
\spelldur{\durshort}
\spellsr{Yes}
\begin{spelltarget}{One melee weapon}
    \spelleffect The target weapon is enhanced and extended, as \spell{zephyr blade}, except that the weapon's rach increases by ten feet, and attacks outside the weapon's normal range deal full damage.
\end{spelltarget}
