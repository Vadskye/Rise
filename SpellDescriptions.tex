\section{Spell Descriptions}

\small

\pdfbookmark[2]{A}{SpellDescriptionsA}

\begin{spellsection}{Ablate Impact}[2]
    \begin{spellheader}
        \spelldesc{You instantly reduce the force of an incoming blow.}
    \end{spellheader}
    \begin{spellcontent}
        \begin{spelltargetinginfo}
            \spelltgt{You}
            \spelltwocol{\spelltime{Immediate action}}{\spellcmp{Verbal only}}
        \end{spelltargetinginfo}
        \begin{spelleffects}
            \spelleffect You gain nonlethal damage reduction against physical damage equal to twice your spellpower. This damage reduction converts damage into nonlethal damage rather than completely negating it. Force damage ignores this damage reduction and negates it for 1 round.
            \spelldur Until end of round
        \end{spelleffects}
    \end{spellcontent}
    \begin{spellfooter}
        \spellinfo{Abjuration [Shielding]}{Abjuration}
        \spellnotes After casting this spell, you cannot cast it again for 5 rounds. You can cast this spell in response to an opponent attacking you, before the attack is rolled.
        \miscastexplode
    \end{spellfooter}
\end{spellsection}

\begin{spellsection}{Ablative Spellshield}[1]

    \begin{spellheader}
        \spelldesc{You instantly encase yourself a shimmering field of magical energy, protecting you from hostile magic.}
    \end{spellheader}
    \begin{spellcontent}
        \begin{spelltargetinginfo}
            \spelltgt{You}
            \spelltwocol{\spelltime{Immediate action}}{\spellcmp{Verbal only}}
        \end{spelltargetinginfo}
        \begin{spelleffects}
            \spelleffect You gain nonlethal damage reduction against spells equal to twice your spellpower. This damage reduction converts damage into nonlethal damage rather than completely negating it. Force damage ignores this damage reduction and negates it for 1 round.
            \spelldur Until end of round
        \end{spelleffects}
    \end{spellcontent}
    \begin{spellfooter}
        \spellinfo{Abjuration [Antimagic]}{Abjuration, Magic}
        \spellnotes After casting this spell, you cannot cast it again for 5 rounds. You can cast this spell in response to an opponent attacking you, before the attack is rolled.

        Spells that are not subject to spell resistance are not affected by \spell{spellshield}.
        \miscastexplode
    \end{spellfooter}
\end{spellsection}

\begin{spellsection}{Acid Arrow}[2]
    \begin{spellheader}
        \spelldesc{You fire a magical arrow of acid from your hand that speeds to its target.}
    \end{spellheader}
    \begin{spellcontent}
        \begin{spelltargetinginfo}
            \spelltwocol{\spelltgt{One creature or object}}{\spellrng{\rngmed}}
        \end{spelltargetinginfo}
        \begin{spelleffects}
            \begin{spellattack}{Spellpower vs. Reflex}
                \spellsuccess 1d6 acid damage per spellpower immediately, and again at the end of the next round.
            \end{spellattack}
        \end{spelleffects}
    \end{spellcontent}
    \begin{spellfooter}
        \spellinfo{Conjuration [Acid, Creation, Physical]}{Arcane}
        \spellnotes If the target becomes submerged in water or is affected by a cold, fire, or water effect, it takes no secondary damage.

        \physicalspellnotes
        \miscastrandom
    \end{spellfooter}
\end{spellsection}

\begin{spellsection}[Greater]{Acid Arrow}[6]
    \begin{spellheader}
        \spelldesc{You fire a magical arrow of acid from your hand that speeds to its target.}
    \end{spellheader}
    \begin{spellcontent}
        \begin{spelleffects}
            \spellspecial This spell functions like \spell{acid arrow}, except that the target is \staggered for 1 round each time it takes damage.
        \end{spelleffects}
    \end{spellcontent}
    \begin{spellfooter}
        \spellinfo{Conjuration [Acid, Creation, Physical]}{Arcane}
        \spellnotes As \spell{acid arrow}, except that ten points of cold or fire damage are required to end the effect.
        \miscastrandom
    \end{spellfooter}
\end{spellsection}

\begin{spellsection}{Acid Fog}[7]
    \begin{spellheader}
        \spelldesc{A billowing mass of acidic vapors fills the area, slowing creatures down and obscuring sight.}
    \end{spellheader}
    \begin{spellcontent}
        \begin{spelltargetinginfo}
            \spelltwocol{\spellzone{\areamed radius cylinder}}{\spellrng{\rngmed}}
        \end{spelltargetinginfo}
        \begin{spelleffects}
            \spelleffect Fog fills the area, as \spell{fog cloud}, except that the fog is acidic.
            \spelldur \durshort
        \end{spelleffects}
    \end{spellcontent}
    \begin{spellsubcontent}
        \begin{spelltargetinginfo}
            \spelltgr{At the end of every round}
            \spelltgts{Everything in the fog}
        \end{spelltargetinginfo}
        \begin{spelleffects}
            \begin{spellattack}{Spellpower vs. Fortitude}
                \spellsuccess 1d6 acid damage per four spellpower.
                \spellfailure Half damage.
            \end{spellattack}
        \end{spelleffects}
    \end{spellsubcontent}
    \begin{spellfooter}
        \spellinfo{Conjuration [Acid, Creation, Fog, Physical]}{Arcane, Destruction}
        \spellnotes \physicalspellnotes
        \miscastyou
    \end{spellfooter}
\end{spellsection}

\begin{spellsection}{Agony}[2]
    \begin{spellheader}
        \spelldesc{You inflict debilitating pain on your foe.}
    \end{spellheader}
    \begin{spellcontent}
        \begin{spelltargetinginfo}
            \spelltwocol{\spelltgt{One creature}}{\spellrng{\rngmed}}
        \end{spelltargetinginfo}
        \begin{spelleffects}
            \begin{spellattack}{Spellpower vs. Mental}
                \spellsuccess The target takes double damage from all physical damage.
                \spellfailure The target takes double damage the next time it takes physical damage.
            \end{spellattack}
            \spelldur \durshort
        \end{spelleffects}
    \end{spellcontent}
    \begin{spellfooter}
        \spellinfo{Enchantment [Delusion, Mind]}{Arcane}
        \miscastrandom
    \end{spellfooter}
\end{spellsection}

\begin{spellsection}{Aid}[2]
    \begin{spellheader}
        \spelldesc{You fill your ally with confidence, improving its resilience in combat.}
    \end{spellheader}
    \begin{spellcontent}
        \begin{spelltargetinginfo}
            \spelltwocol{\spelltgt{One creature}}{\spellrng{Touch}}
        \end{spelltargetinginfo}
        \begin{spelleffects}
            \spelleffect The target gains temporary hit points equal to your spellpower. In addition, it is immune to fear effects.
            \spelldur \durpersonallong
        \end{spelleffects}
    \end{spellcontent}
    \begin{spellfooter}
        \spellinfo{Enchantment [Mind, Morale]}{Divine}
        \spellnotes If the target takes life damage, it loses all temporary hit points provided by this spell before applying the damage.
        \miscastexplode
    \end{spellfooter}
\end{spellsection}

\begin{spellsection}{Air Walk}[4]
    \begin{spellheader}
        \spelldesc{You imbue an ally with the ability to walk on nothing but air.}
    \end{spellheader}
    \begin{spellcontent}
        \begin{spelltargetinginfo}
            \spelltwocol{\spelltgt{One creature (Gargantuan size or smaller)}}{\spellrng{\rngtouch}}
        \end{spelltargetinginfo}
        \begin{spelleffects}
            \spelleffect The target can walk on air as if it were solid ground. The magic only affects the target's legs, and does not grant the ability to climb vertically through the air.
            \par If the spell ends while the target is still aloft, the magic fails slowly. The target floats downward 60 feet per round for 1d6 rounds. If it reaches the ground in that amount of time, it lands safely. If not, it falls the rest of the distance, taking falling damage if appropriate.
            \spelldur \durshort
        \end{spelleffects}
    \end{spellcontent}
    \begin{spellfooter}
        \spellinfo{Transmutation [Air, Augment]}{Air, Divine, Nature, Travel}
        \miscastexplode
    \end{spellfooter}
\end{spellsection}

\begin{spellsection}{Antilife Shell}[7]
    \begin{spellheader}
        \spelldesc{You create an immobile, spherical energy field that hedges out living creatures.}
    \end{spellheader}
    \begin{spellcontent}
        \begin{spelltargetinginfo}
            \spellzone{\areasmall radius centered on you} 
        \end{spelltargetinginfo}
        \begin{spelleffects}
            \spelleffect Living creatures cannot enter the spell's area. Nonliving creatures, such as constructs and undead, suffer no ill effect.
            \spelldur \durmed \dismissable
        \end{spelleffects}
    \end{spellcontent}
    \begin{spellfooter}
        \spellinfo{Abjuration [Barrier]}{Divine, Nature, Wild}
        \spellnotes Barrier spells may be used only defensively, not aggressively. Creatures in the area at the time that the spell is cast are unaffected by the spell.
        \miscastexplode
    \end{spellfooter}
\end{spellsection}

\begin{spellsection}{Antimagic Field}[7]
    \begin{spellheader}
        \spelldesc{You create a mobile, spherical energy field that suppresses magic.}
    \end{spellheader}
    \begin{spellcontent}
        \begin{spelltargetinginfo}
            \spellzone{\areasmall radius centered on you} 
        \end{spelltargetinginfo}
        \begin{spelleffects}
            \spelleffect All spells, spell-like abilities, and magic items fail to function in the area. They cannot be activated from within the field, and any existing effects brought into or cast into the area are suppressed. Time spent within an \spell{antimagic field} counts against a suppressed spell's duration.
            \par Creatures within an \spell{antimagic field} cannot concentrate on or dismiss spells. However, you can concentrate on and dismiss your own \spell{antimagic field}.
            \spelldur \durlong \dismissable
        \end{spelleffects}
    \end{spellcontent}
    \begin{spellfooter}
        \spellinfo{Abjuration [Antimagic]}{Abjuration, Divine, Magic}
        \spellnotes The effects of instantaneous conjurations, such as \spell{create water}, are not affected by this spell because the conjuration itself is no longer in effect, only its result.

        \par \spell{Dispel magic} and similar magic has no effect on an \spell{antimagic field}. Two or more \spellindirect{antimagicfield}{antimagic fields} sharing any of the same space have no effect on each other.
    %\par Any part of a creature that lies outside the field is unaffected by the field.
        \par Artifacts and deities are unaffected by mortal magic such as this.
        \miscastexplode
    \end{spellfooter}
\end{spellsection}

\begin{spellsection}{Aqueous Blade}[2]
    \begin{spellheader}
        \spelldesc{You transform the active part of a weapon into water, weakening its blows but allowing it penetrate defenses more easily.}
    \end{spellheader}
    \begin{spellcontent}
        \begin{spelltargetinginfo}
            \spelltwocol{\spelltgt{One weapon}}{\spellrng{\rngclose}}
        \end{spelltargetinginfo}
        \begin{spelleffects}
            \begin{spellattack}{Spellpower vs. Mental}
                \spellsuccess Attacks with the affected weapon are made against Reflex defense instead of Armor defense. However, damage with the weapon is halved, including any bonuses to damage.
            \end{spellattack}
            \spelldur \durshort
        \end{spelleffects}
    \end{spellcontent}
    \begin{spellfooter}
        \spellinfo{Transmutation [Alteration, Water]}{Nature, Water}
        \miscastrandom
    \end{spellfooter}
\end{spellsection}

\begin{spellsection}{Assimilate}[9]
    \begin{spellheader}
        \spelldesc{Your pointing finger turns black as obsidian. You touch a creature and it dissolves into dust as you assimilate its form into your own body.}
    \end{spellheader}
    \begin{spellcontent}
        \begin{spelltargetinginfo}
            \spelltwocol{\spelltgt{One living creature}}{\spellrng{\rngclose}}
        \end{spelltargetinginfo}
        \begin{spelleffects}
            \begin{spellattack}{Spellpower vs. Fortitude}
                \spellsuccess 1d10 life damage per two spellpower. If the target has no hit points remaining, it immediately dies, and you are transformed to mimic its appearance for 12 hours. This grants you a \plus10 bonus on Disguise checks made to appear as that creature.
                \spellfailure Half damage, and no additional effects.
            \end{spellattack}
            \spelldur \durext; see text
        \end{spelleffects}
    \end{spellcontent}
    \begin{spellfooter}
        \spellinfo{Transmutation/Vivimancy [Alteration]}{Arcane, Evil}
        \miscastrandom
    \end{spellfooter}
\end{spellsection}

\begin{spellsection}{Aversion}[3]
    \begin{spellheader}
        \spelldesc{You make a creature want to avoid something.}
    \end{spellheader}
    \begin{spellcontent}
        \begin{spelltargetinginfo}
            \spelltwocol{\spelltgt{One creature}}{\spellrng{\rngmed}}
        \end{spelltargetinginfo}
        \begin{spelleffects}
            \begin{spellattack}{Spellpower vs. Mental}
                \spellsuccess The target feels an aversion to a particular person or object. If the object of the implanted aversion is an individual or a physical object, she will prefer not to approach within 30 feet of it. If it is a word, she will try not to utter it; if it is an action, she will not willingly attempt to perform it; and if it is an event, she will not willingly attend it. The target will take reasonable steps to avoid the object of its aversion, but will not put herself in jeopardy by doing so.
                \par If the target is unable to avoid the object of her aversion, she takes a \minus4 penalty to attacks, defenses, and checks as long as she is close to it.
                \spellcritical As above, but the effect is permanent.
            \end{spellattack}
            \spelldur \durshort or \durperm
        \end{spelleffects}
    \end{spellcontent}
    \begin{spellfooter}
        \spellinfo{Enchantment [Mind]}{Enchantment}
        \miscastrandom
    \end{spellfooter}
\end{spellsection}

\pdfbookmark[2]{B}{SpellDescriptionsB}

\begin{spellsection}{Bane}[1]
    \begin{spellheader}
    %\spelldesc{}
    \end{spellheader}
    \begin{spellcontent}
        \begin{spelltargetinginfo}
            \spelltwocol{\spelltgt{One creature}}{\spellrng{\rngclose}}
        \end{spelltargetinginfo}
        \begin{spelleffects}
            \spelleffect The target is \impaired with all actions.

            \spelldur \durshort or until discharged \dismissable
        \end{spelleffects}
    \end{spellcontent}
    \begin{spellfooter}
        \spellinfo{Enchantment [Compulsion, Mind]}{Divine, Evil, War}
        \miscastrandom
    \end{spellfooter}
\end{spellsection}

\begin{spellsection}[Mass]{Bane}[5]
    \begin{spellheader}
    %\spelldesc{}
    \end{spellheader}
    \begin{spellcontent}
        \begin{spelltargetinginfo}
            \spelltwocol{\spelltgts{Up to five creatures}}{\spellrng{\rngmed}}
        \end{spelltargetinginfo}
        \begin{spelleffects}
            \spellspecial This spell functions like \spell{bane}, except that it affects multiple creatures.
        \end{spelleffects}
    \end{spellcontent}
    \begin{spellfooter}
        \spellinfo{Enchantment [Compulsion, Mind]}{Divine, Evil, War}
        \miscastexplode
    \end{spellfooter}
\end{spellsection}

\begin{spellsection}{Barkskin}[2]
    \begin{spellheader}
        \spelldesc{You toughen a creature's skin, giving it the appearance of tree bark.}
    \end{spellheader}
    \begin{spellcontent}
        \begin{spelltargetinginfo}
            \spelltwocol{\spelltgt{One living creature}}{\spellrng{\rngclose}}
        \end{spelltargetinginfo}
        \begin{spelleffects}
            \spelleffect The target gains damage reduction against physical damage equal to half your spellpower. Slashing weapons and fire damage ignore this damage reduction and negate it for 1 round.
            \spelldur \durpersonallong
        \end{spelleffects}
    \end{spellcontent}
    \begin{spellfooter}
        \spellinfo{Transmutation [Augment]}{Nature, Wild}
        \miscastexplode
    \end{spellfooter}
\end{spellsection}

\begin{spellsection}{Black Tentacles}[5]
    \begin{spellheader}
        \spelldesc{You create a field of rubbery black tentacles, each 5 feet long. These waving members seem to spring forth from the earth, floor, or whatever surface is underfoot -- including water. They grasp and entwine around creatures that enter the area, holding them fast and crushing them with great strength.}
    \end{spellheader}
    \begin{spellcontent}
        \begin{spelltargetinginfo}
            \spelltwocol{\spellzone{\areasmall radius}}{\spellrng{\rngmed}}
        \end{spelltargetinginfo}
        \begin{spelleffects}
            \spelleffect Ground within the area is considered difficult terrain.
            \spelldur Concentration
        \end{spelleffects}
    \end{spellcontent}
    \begin{spellsubcontent}
        \begin{spelltargetinginfo}
            \spelltgr{At the end of every round}
            \spelltgts{All creatures in the area within 5 feet of the ground.}
        \end{spelltargetinginfo}
        \begin{spelleffects}
            \begin{spellattack}{Spellpower vs. Maneuver defense}
                \spellsuccess The target is grappled and takes 1d8\plus4 bludgeoning damage. It remains grappled until it escapes the tentacle. The DC to escape the grapple is equal to 10 \add your spellpower.
            \end{spellattack}
        \end{spelleffects}
    \end{spellsubcontent}
    \begin{spellfooter}
        \spellinfo{Evocation}{Arcane}
        \spellnotes The tentacles are immune to all forms of attack.
        \miscastyou
    \end{spellfooter}
\end{spellsection}

\begin{spellsection}{Blade Barrier}[4]
    \begin{spellheader}
        \spelldesc{You create an immobile, vertical curtain of whirling blades shaped of pure force.}
    \end{spellheader}
    \begin{spellcontent}
        \begin{spelltargetinginfo}
            \spelltwocol{\spellzone{100 ft. wall, 20 ft. high \shapeable}}{\spellrng{\rngmed}}
        \end{spelltargetinginfo}
        \begin{spelleffects}
            \spelleffect This spell creates a wall of blades made of force energy. The wall provides active cover (20\% miss chance) against attacks made through it. Attacks that miss in this way harmlessly strike the wall. Passing through the wall costs twice as much movement as normal.
            \spelldur \durshort \dismissable
        \end{spelleffects}
    \end{spellcontent}
    \begin{spellsubcontent}
        \begin{spelltargetinginfo}
            \spelltwocol{\spelltgr{A creature passes through the wall}}{\spelltgt{The moving creature}}
        \end{spelltargetinginfo}
        \begin{spelleffects}
            \begin{spellattack}{Spellpower vs. Reflex}
                \spellsuccess \spelldamage{4}{force}[d6].
                \spellfailure Half damage.
            \end{spellattack}
        \end{spelleffects}
    \end{spellsubcontent}
    \begin{spellfooter}
        \spellinfo{Evocation [Force, Wall]}{Divine, War}
        \miscastexplode
    \end{spellfooter}
\end{spellsection}

\begin{spellsection}{Blasphemy}[6]
    \begin{spellheader}
        \spelldesc{You speak an unholy utterance of great power, afflicting all those nearby who do not share your allegiance to evil.}
    \end{spellheader}
    \begin{spellcontent}
        \begin{spelltargetinginfo}
            \spellburst{\arealarge radius centered on you}
            \spelltgts{All nonevil creatures in the area}
            \spellcmp{Verbal only}
        \end{spelltargetinginfo}
        \begin{spelleffects}
            \begin{spellattack}{Spellpower vs. Mental}
                \spellsuccess \spelldamage{7}{divine}[d6].
                \spellcritical As above, and the target is \staggered for 5 rounds.
                \spellfailure Half damage.
            \end{spellattack}
        \end{spelleffects}
    \end{spellcontent}
    \begin{spellfooter}
        \spellinfo{Evocation [Evil]}{Divine, Evil}
        \miscastexplode
    \end{spellfooter}
\end{spellsection}

\begin{spellsection}{Bless}[1]
    \begin{spellheader}
        \spelldesc{You fill your ally with confidence, improving his prowess in combat.}
    \end{spellheader}
    \begin{spellcontent}
        \begin{spelltargetinginfo}
            \spelltwocol{\spelltgts{One creature}}{\spellrng{\rngmed}}
        \end{spelltargetinginfo}
        \begin{spelleffects}
            \spelleffect The target gains an offensive legend point.
            \spelldur \durshort or until expended
        \end{spelleffects}
    \end{spellcontent}
    \begin{spellfooter}
        \spellinfo{Enchantment [Mind, Morale]}{Divine, Good, War}
        \miscastrandom
    \end{spellfooter}
\end{spellsection}

\begin{spellsection}[Mass]{Bless}[4]
    \begin{spellheader}
        \spelldesc{You fill your allies with confidence, improving their prowess in combat.}
    \end{spellheader}
    \begin{spellcontent}
        \begin{spelltargetinginfo}
            \spelltwocol{\spelltgts{Up to five creatures}}{\spellrng{\rngmed}}
        \end{spelltargetinginfo}
        \begin{spelleffects}
            \spelleffect The target gains an offensive legend point.
            \spelldur \durshort or until expended
        \end{spelleffects}
    \end{spellcontent}
    \begin{spellfooter}
        \spellinfo{Enchantment [Mind, Morale]}{Divine, Good, War}
        \miscastexplode
    \end{spellfooter}
\end{spellsection}

\begin{spellsection}{Blink}[5]
    \begin{spellheader}
        \spelldesc{You rapidly blink in and out of reality, confounding your foes and protecting you from their attacks.}
    \end{spellheader}
    \begin{spellcontent}
        \begin{spelltargetinginfo}
            \spelltgt{You}
        \end{spelltargetinginfo}
        \begin{spelleffects}
            \spelleffect You spend half your time on the Astral Plane. All attacks against you have a 50\% chance to fail.
            \spelldur \durshort \dismissable
        \end{spelleffects}
    \end{spellcontent}
    \begin{spellfooter}
        \spellinfo{Conjuration [Planar]}{Arcane}
        \spellnotes If you are on the Astral Plane when you cast this spell, it has no effect.
        \miscastexplode
    \end{spellfooter}
\end{spellsection}

\begin{spellsection}{Blur}[2]
    \begin{spellheader}
        \spelldesc{You distort an ally's outline so it appears blurred, shifting, and wavering.}
    \end{spellheader}
    \begin{spellcontent}
        \begin{spelltargetinginfo}
            \spelltwocol{\spelltgt{One creature}}{\spellrng{\rngclose}}
        \end{spelltargetinginfo}
        \begin{spelleffects}
            \spelleffect The target gains a \plus5 bonus to Stealth checks, and is always treated as having concealment for the purpose of making Stealth checks.
            \spelldur \durpersonallong
        \end{spelleffects}
    \end{spellcontent}
    \begin{spellfooter}
        \spellinfo{Illusion [Glamer]}{Arcane}
        \miscastrandom
    \end{spellfooter}
\end{spellsection}

\begin{spellsection}{Burning Hands}[1]
    \begin{spellheader}
        \spelldesc{You expel a cone of searing flame shoots from your fingertips, searing creatures in front of you.}
    \end{spellheader}
    \begin{spellcontent}
        \begin{spelltargetinginfo}
            \spelltwocol{\spellburst{\areamed cone}}{\spelltgts{Everything in the area}}
        \end{spelltargetinginfo}
        \begin{spelleffects}
            \begin{spellattack}{Spellpower vs. Reflex}
                \spellsuccess \spelldamage{1}{fire}[d6]
                \spellfailure Half damage.
            \end{spellattack}
        \end{spelleffects}
    \end{spellcontent}
    \begin{spellfooter}
        \spellinfo{Evocation [Destructive, Fire]}{Arcane, Destruction, Nature, Fire}
        \miscastexplode
    \end{spellfooter}
\end{spellsection}

\pdfbookmark[2]{C}{SpellDescriptionsC}

\begin{spellsection}{Cacaphonic Word}[6]
    \begin{spellheader}
        \spelldesc{You utter an incoherent burst of noise, disorienting your foes.}
    \end{spellheader}
    \begin{spellcontent}
        \begin{spelltargetinginfo}
            \spellburst{\arealarge radius centered on you}
            \spelltgts{All nonchaotic creatures in the area}
            \spellcmp{Verbal only}
        \end{spelltargetinginfo}
        \begin{spelleffects}
            \begin{spellattack}{Spellpower vs. Mental}
                \spellsuccess \spelldamage{6}{divine}[d6].
                \spellcritical As above, and the target is \disoriented for 5 rounds.
                \spellfailure Half damage.
            \end{spellattack}
        \end{spelleffects}
    \end{spellcontent}
    \begin{spellfooter}
        \spellinfo{Evocation [Chaotic]}{Chaos, Divine}
        \miscastexplode
    \end{spellfooter}
\end{spellsection}

\begin{spellsection}{Call Lightning}[4]
    \begin{spellheader}
        \spelldesc{You repeatedly call bolts of lightning that flash down from thin air to smite your foes.}
    \end{spellheader}
    \begin{spellcontent}
        \begin{spelltargetinginfo}
            \spelltwocol{\spellburst{\arealarge vertical line}}{\spellrng{\rngmed}}
            \spelltgts{Everything in the area}
        \end{spelltargetinginfo}
        \begin{spelleffects}
            \begin{spellattack}{Spellpower vs. Reflex}
                \spellsuccess \spelldamage{4}{electricity}[d8].

                If you are outdoors in cloudy or stormy weather, roll d10s instead of d8s for damage.
                \spellfailure Half damage.
            \end{spellattack}
            \spelleffect You can concentrate as a standard action to call down another bolt of lightning. You may call a total number of bolts equal to your spellpower before the spell is discharged.
            \spelldur \durmed or until discharged \dismissable
        \end{spelleffects}
    \end{spellcontent}
    \begin{spellfooter}
        \spellinfo{Evocation [Destructive, Electricity]}{Air, Nature}
        \spellnotes This spell functions indoors or underground, but not underwater. \destructivespellnotes
        \miscastexplode
    \end{spellfooter}
\end{spellsection}

\begin{spellsection}[Greater]{Call Lightning}[8]
    \begin{spellheader}
        \spelldesc{You repeatedly call intense bolts of lightning that flash down from thin air to smite your foes.}
    \end{spellheader}
    \begin{spellcontent}
        \begin{spelltargetinginfo}
            \spelltwocol{\spellburst{\arealarge vertical line}}{\spellrng{\rngmed}}
            \spelltgts{Everything in the area}
        \end{spelltargetinginfo}
        \begin{spelleffects}
            \spellspecial This spell functions like \spell{call lightning}, except that creatures struck are also \staggered for 5 rounds if the attack succeeds.
        \end{spelleffects}
    \end{spellcontent}
    \begin{spellfooter}
        \spellinfo{Evocation [Destructive, Electricity]}{Air, Nature}
        \spellnotes As \spell{call lightning}.
        \miscastexplode
    \end{spellfooter}
\end{spellsection}

\begin{spellsection}{Calm Emotions}[3]
    \begin{spellheader}
        \spelldesc{You calm a group of creatures, preventing the situation from getting out of hand.}
    \end{spellheader}
    \begin{spellcontent}
        \begin{spelltargetinginfo}
            \spelltwocol{\spellburst{\arealarge radius}}{\spellrng{\rngmed}}
            \spelltgts{All creatures in the area}
        \end{spelltargetinginfo}
        \begin{spelleffects}
            \begin{spellattack}{Spellpower vs. Mental}
                \spellsuccess The target has its emotions calmed. It cannot take violent actions (although it can defend itself) or do anything destructive.
            \end{spellattack}
            \spelldur Concentration
        \end{spelleffects}
    \end{spellcontent}
    \begin{spellfooter}
        \spellinfo{Enchantment [Mind]}{Arcane}
        \spellnotes Any aggressive action against or damage dealt to a calmed creature immediately breaks the spell on all calmed creatures.

        This spell automatically suppresses (but does not dispel) any effects of spells or abilities that affect or require emotions, including all other enchantment (emotion) spells.
        \miscastyou
    \end{spellfooter}
\end{spellsection}

\begin{spellsection}{Chain Lightning}[6]
    \begin{spellheader}
        \spelldesc{You create a stroke of lightning which strikes a single foe before arcing to hit a number of other foes of your choice.}
    \end{spellheader}
    \begin{spellcontent}
        \begin{spelltargetinginfo}
            \spelltwocol{\spelltgt{One creature or object}[Primary]}{\spellrng{\rngmed}}
            \spelllimit{\areamed radius centered on the primary target}
            \spelltgts{Any number of creatures or objects within the area}[Secondary]
        \end{spelltargetinginfo}
        \begin{spelleffects}
            \begin{spellattack}{Spellpower vs. Reflex}
                \spellsuccess \spelldamage{5}{electricity}[d6].
                \spellfailure Half damage.
                \spellspecial This attack automatically succeeds against the primary target.
            \end{spellattack}
        \end{spelleffects}
    \end{spellcontent}
    \begin{spellfooter}
        \spellinfo{Evocation [Destructive, Electricity]}{Arcane, Destruction, Nature}
        \miscastexplode
    \end{spellfooter}
\end{spellsection}

\begin{spellsection}{Chaos Hammer}[3]
    \begin{spellheader}
        \spelldesc{You unleash a multicolored explosion of leaping, ricocheting energy to smite your foe.}
    \end{spellheader}
    \begin{spellcontent}
        \begin{spelltargetinginfo}
            \spelltwocol{\spelltgt{One nonchaotic creature}}{\spellrng{\rngmed}}
        \end{spelltargetinginfo}
        \begin{spelleffects}
            \begin{spellattack}{Spellpower vs. Mental}
                \spellsuccess \spelldamage{3}{divine}.
                \spellcritical As above, and the target is \disoriented for 5 rounds.
                \spellfailure Half damage.
            \end{spellattack}
        %\spelldur 5 rounds
        \end{spelleffects}
    \end{spellcontent}
    \begin{spellfooter}
        \spellinfo{Evocation [Chaotic]}{Chaos}
        \miscastrandom
    \end{spellfooter}
\end{spellsection}

\begin{spellsection}{Charm Monster}[4]
    \begin{spellheader}
        \spelldesc{You manipulate a creature's mind so it thinks of you as a trusted friend and ally.}
    \end{spellheader}
    \begin{spellcontent}
        \begin{spelltargetinginfo}
            \spelltwocol{\spelltgt{One creature}}{\spellrng{\rngmed}}
            \spellcmp{Somatic only}
        \end{spelltargetinginfo}
        \begin{spelleffects}
            \spellspecial This spell functions like \spell{charm person}, except that it affects creatures of any type.
        \end{spelleffects}
    \end{spellcontent}
    \begin{spellfooter}
        \spellinfo{Enchantment [Delusion, Mind, Subtle]}{Enchantment}
        \miscastrandom
    \end{spellfooter}
\end{spellsection}

\begin{spellsection}[Mass]{Charm Monster}[8]
    \begin{spellheader}
        \spelldesc{You manipulate the minds of creatures so they think of you as a trusted friend and ally.}
    \end{spellheader}
    \begin{spellcontent}
        \begin{spelltargetinginfo}
            \spelltwocol{\spelltgt{Up to five creatures}}{\spellrng{\rngmed}}
            \spellcmp{Somatic only}
        \end{spelltargetinginfo}
        \begin{spelleffects}
            \spellspecial This spell functions like \spell{charm person}, except that it affects multiple creatures of any type.
        \end{spelleffects}
    \end{spellcontent}
    \begin{spellfooter}
        \spellinfo{Enchantment [Delusion, Mind, Subtle]}{Enchantment}
        \miscastyou
    \end{spellfooter}
\end{spellsection}

\begin{spellsection}{Charm Person}[2]
    \begin{spellheader}
        \spelldesc{You manipulate a person's mind so he thinks of you as a trusted friend and ally.}
    \end{spellheader}
    \begin{spellcontent}
        \begin{spelltargetinginfo}
            \spelltwocol{\spelltgt{One humanoid creature}}{\spellrng{\rngmed}}
            \spellcmp{Somatic only}
        \end{spelltargetinginfo}
        \begin{spelleffects}
            \begin{spellattack}{Spellpower vs. Mental}
                \spellspecial If the target thinks that you or your allies are threatening it, you take a \minus5 penalty on the attack.
                \spellsuccess The target sees your words and actions in the most favorable way, as a close friend or trusted ally. You cannot control it like an automaton, but you can persuade it to take particular actions with the Persuasion skill (see \pcref{Persuasion}). The target is treated as a friend (a \plus10 relationship modifier) for the purpose of Persuasion checks you make.
                \spellcritical As above, but the effect is permanent.
            \end{spellattack}
            \spelldur \durlong
        \end{spelleffects}
    \end{spellcontent}
    \begin{spellfooter}
        \spellinfo{Enchantment [Charm, Mind, Subtle]}{Enchantment}
        \spellnotes Any act by you or your apparent allies that threatens or damages the \spell{charmed} person breaks the spell.

        \subtlespellnotes

        \norepeatspellnotes
        \miscastrandom
    \end{spellfooter}
\end{spellsection}

\begin{spellsection}[Mass]{Charm Person}[6]
    \begin{spellheader}
        \spelldesc{You manipulate the minds of many people so they think of you as a trusted friend and ally.}
    \end{spellheader}
    \begin{spellcontent}
        \begin{spelltargetinginfo}
            \spelltwocol{\spelltgts{Up to five humanoid creatures}}{\spellrng{\rngmed}}
        \end{spelltargetinginfo}
        \begin{spelleffects}
            \spellspecial This spell functions like \spell{charm person}, except that it affects multiple creatures.
        \end{spelleffects}
    \end{spellcontent}
    \begin{spellfooter}
        \spellinfo{Enchantment [Charm, Mind, Subtle]}{Enchantment}
        \miscastexplode
    \end{spellfooter}
\end{spellsection}

\begin{spellsection}{Cloak of Chaos}[8]
    \begin{spellheader}
        \spelldesc{You shield your allies with a powerful aura that resembles a random pattern of color -- an affront to your lawful foes.}
    \end{spellheader}
    \begin{spellcontent}
        \begin{spelltargetinginfo}
            \spelltwocol{\spellrng{\rngclose}}{\spelltgts{Up to five creatures}}
        \end{spelltargetinginfo}
        \begin{spelleffects}
            \spelleffect The target gains spell resistance against lawful spells and spells cast by lawful creatures.
            \spelldur \durshort \dismissable
        \end{spelleffects}
    \end{spellcontent}
    \begin{spellsubcontent}
        \begin{spelltargetinginfo}
            \spelltgr{Whenever a lawful creature within 30 feet of the target makes a physical attack against it}
            \spelltgt{The attacking creature}
        \end{spelltargetinginfo}
        \begin{spelleffects}
            \begin{spellattack}{Spellpower vs. Mental}
                \spellsuccess \spelldamage{9}{divine}[d6].
            \end{spellattack}
        \end{spelleffects}
    \end{spellsubcontent}
    \begin{spellfooter}
        \spellinfo{Abjuration [Chaotic, Retributive]}{Chaos, Divine}
        \miscastexplode
    \end{spellfooter}
\end{spellsection}

\begin{spellsection}{Color Spray}[1]
    \begin{spellheader}
    \end{spellheader}
    \begin{spellcontent}
        \begin{spelltargetinginfo}
            \spellburst{\areamed cone}
            \spelltgts{All creatures in the area}
        \end{spelltargetinginfo}
        \begin{spelleffects}
            \begin{spellattack}{Spellpower vs. Mental}
                \spellsuccess The target's vision is \impaired. This affects all sight-related actions, including physical attacks and targeted spells.
                \spellcritical The target's vision is \severelyimpaired. This affects all sight-related actions, including physical attacks and targeted spells.
            \end{spellattack}
            \spelldur 5 rounds
        \end{spelleffects}
    \end{spellcontent}
    \begin{spellfooter}
        \spellinfo{Illusion [Figment, Light, Visual]}{Arcane}
        \spelldesc{You project a vivid cone of clashing colors from your outstretched hand, striking creatures in front of you.}
        \spellnotes Creatures who cannot see the light are not affected by this spell. Merely closing one's eyes is insufficient protection, however.
        \miscastexplode
    \end{spellfooter}
\end{spellsection}

\begin{spellsection}{Command}[1]
    \begin{spellheader}
        \spelldesc{You compel a foe to obey a single command of your choice.}
    \end{spellheader}
    \begin{spellcontent}
        \begin{spelltargetinginfo}
            \spelltwocol{\spelltgt{One creature}}{\spellrng{\rngmed}}
            \spellcmp{Verbal only}
        \end{spelltargetinginfo}
        \begin{spelleffects}
            \spellspecial When you cast this spell, you speak a single-word command of your choice.
            \begin{spellattack}{Spellpower vs. Mental}
                \spellsuccess The target must obey the command or be \severelyimpaired with all actions.
                \spellcritical The target must obey the command or be \stunned.
                \spellfailure The target must obey the command or be \impaired with all actions.
            \end{spellattack}
            \spelldur 1 round
        \end{spelleffects}
    \end{spellcontent}
    \begin{spellfooter}
        \spellinfo{Enchantment [Auditory, Compulsion, Mind, Speech]}*{Arcane, Divine, Law}
        \spellnotes If the target can't understand your command, the spell automatically fails. The target must obey the literal meaning of the commmand given, potentially allowing intelligent targets to subvert your intentions.
        \miscastrandom
    \end{spellfooter}
\end{spellsection}

\begin{spellsection}[Mass]{Command}[4]
    \begin{spellheader}
        \spelldesc{You compel many foes to obey your command.}
    \end{spellheader}
    \begin{spellcontent}
        \begin{spelltargetinginfo}
            \spelltwocol{\spelltgts{Five creatures}}{\spellrng{\rngmed}}
            \spellcmp{Verbal only}
        \end{spelltargetinginfo}
        \begin{spelleffects}
            \spellspecial This spell functions like \spell{command}, except that it affects multiple creatures.
        \end{spelleffects}
    \end{spellcontent}
    \begin{spellfooter}
        \spellinfo{Enchantment [Auditory, Compulsion, Mind, Speech]}{Divine, Law}
        \miscastexplode
    \end{spellfooter}
\end{spellsection}

\begin{spellsection}{Cone of Cold}[4]
    \begin{spellheader}
        \spelldesc{You create an area of extreme cold that drains heat from creatures in the area, diminishing their ability to move.}
    \end{spellheader}
    \begin{spellcontent}
        \begin{spelltargetinginfo}
            \spellburst{\areamed cone}
            \spelltgts{Everything in the area}
        \end{spelltargetinginfo}
        \begin{spelleffects}
            \begin{spellattack}{Spellpower vs. Reflex}
                \spellsuccess \spelldamage{4}{cold}[d6]. In addition, the target moves at half speed.
                \spellfailure As above, but half damage.
            \end{spellattack}
            \spelldur \durshort
        \end{spelleffects}
    \end{spellcontent}
    \begin{spellfooter}
        \spellinfo{Evocation [Cold]}{Arcane, Nature}
        \miscastexplode
    \end{spellfooter}
\end{spellsection}

\begin{spellsection}[Greater]{Cone of Cold}[7]
    \begin{spellheader}
        \spelldesc{You create a massive area of extreme cold that drains heat from creatures in the area, diminishing their ability to move.}
    \end{spellheader}
    \begin{spellcontent}
        \begin{spelltargetinginfo}
            \spellburst{\arealarge cone}
            \spelltgts{Everything in the area}
        \end{spelltargetinginfo}
        \begin{spelleffects}
            \begin{spellattack}{Spellpower vs. Reflex}
                \spellsuccess \spelldamage{7}{cold}[d6]. In addition, the target moves at half speed.
                \spellfailure As above, but half damage.
            \end{spellattack}
        \end{spelleffects}
    \end{spellcontent}
    \begin{spellfooter}
        \spellinfo{Evocation [Cold]}{Arcane, Nature}
        \miscastexplode
    \end{spellfooter}
\end{spellsection}

\begin{spellsection}{Confusion}[3]
    \begin{spellheader}
        \spelldesc{You compel a creature to act randomly, sowing confusion in your foes' ranks.}
    \end{spellheader}
    \begin{spellcontent}
        \begin{spelltargetinginfo}
            \spelltwocol{\spelltgt{One creature}}{\spellrng{\rngmed}}
        \end{spelltargetinginfo}
        \begin{spelleffects}
            \begin{spellattack}{Spellpower vs. Mental}
                \spellsuccess The target is \disoriented.
                \spellcritical The target is \confused.
            \end{spellattack}
            \spelldur \durshort
        \end{spelleffects}
    \end{spellcontent}
    \begin{spellfooter}
        \spellinfo{Enchantment [Compulsion, Mind]}{Arcane, Chaos, Trickery}
        \miscastrandom
    \end{spellfooter}
\end{spellsection}

\begin{spellsection}[Mass]{Confusion}[6]
    \begin{spellheader}
        \spelldesc{You compel a group of creatures to act randomly, sowing confusion in your foes' ranks.}
    \end{spellheader}
    \begin{spellcontent}
        \begin{spelltargetinginfo}
            \spelltwocol{\spelltgts{Up to five creatures}}{\spellrng{\rngmed}}
        \end{spelltargetinginfo}
        \begin{spelleffects}
            \begin{spellattack}{Spellpower vs. Mental}
                \spellsuccess The target is \disoriented.
                \spellcritical The target is \confused.
            \end{spellattack}
        \end{spelleffects}
    \end{spellcontent}
    \begin{spellfooter}
        \spellinfo{Enchantment [Compulsion, Mind]}{Arcane, Trickery}
        \miscastexplode
    \end{spellfooter}
\end{spellsection}

\begin{spellsection}{Create Sound}[1]
    \begin{spellheader}
        \spelldesc{You create false sounds from nowhere.}
    \end{spellheader}
    \begin{spellcontent}
        \begin{spelltargetinginfo}
            \spellrng{\rngmed}
        \end{spelltargetinginfo}
        \begin{spelleffects}
            \spelleffect You create sound from a location within range. The sound can be of any kind, but can be no louder than the sound that could be created by one human per spellpower. You can create understandable speech, but the sound is not precise enough to trigger magical effects activated by command words.
            \spelldur \durshort \dismissable
        \end{spelleffects}
    \end{spellcontent}
    \begin{spellfooter}
        \spellinfo{Illusion [Figment, Unreal]}{Illusion}
        \spellnotes Creatures can identify the illusion, as \spell{silent image}. This spell can be made permanent with a \spell{permanency} ritual.
        \miscastexplode
    \end{spellfooter}
\end{spellsection}

\begin{spellsection}{Cripple}[9]
    \begin{spellheader}
        \spelldesc{You render your foe's limbs useless.}
    \end{spellheader}
    \begin{spellcontent}
        \begin{spelltargetinginfo}
            \spelltwocol{\spelltgt{One creature}}{\spellrng{\rngmed}}
        \end{spelltargetinginfo}
        \begin{spelleffects}
            \begin{spellattack}{Spellpower vs. Fortitude}
                \spellsuccess \spelldamage{9}{life}. In addition, the target is \staggered.

                \spellcritical As above, but instead of being staggered, the target is unable to move its limbs, including any wings. Generally, that means it is \paralyzed, except that it can move its head and mouth.

                \spellfailure Half damage, and no additional effects.
            \end{spellattack}
            \spelldur \durshort
        \end{spelleffects}
    \end{spellcontent}
    \begin{spellfooter}
        \spellinfo{Vivimancy [Flesh]}{Arcane}
        \miscastrandom
    \end{spellfooter}
\end{spellsection}

\begin{spellsection}{Cure Critical Wounds}[4]
    \begin{spellheader}
        \spelldesc{You lay your hand on a creature and channel positive energy into it, healing even the most grievous injuries.}
    \end{spellheader}
    \begin{spellcontent}
        \begin{spelltargetinginfo}
            \spelltwocol{\spelltgt{One creature}}{\spellrng{\rngmed}}
        \end{spelltargetinginfo}
        \begin{spelleffects}
            \spellspecial This spell functions like \spell{cure light wounds}, except that for every 2 points of healing granted by this spell, it can instead cure 1 point of critical damage.
        \end{spelleffects}
    \end{spellcontent}
    \begin{spellfooter}
        \spellinfo{Vivimancy [Positive]}{Divine, Nature, Vitality}
        \miscastrandom
    \end{spellfooter}
\end{spellsection}

\begin{spellsection}[Mass]{Cure Critical Wounds}[8]
    \begin{spellheader}
        \spelldesc{You stretch out your hand and channel positive energy into all of your allies, healing even their most grievous injuries.}
    \end{spellheader}
    \begin{spellcontent}
        \begin{spelltargetinginfo}
            \spelltwocol{\spelltgts{Up to five creatures}}{\spellrng{\rngmed}}
        \end{spelltargetinginfo}
        \begin{spelleffects}
            \spellspecial This spell functions like \spell{cure critical wounds}, except that it affects multiple creatures.
        \end{spelleffects}
    \end{spellcontent}
    \begin{spellfooter}
        \spellinfo{Vivimancy [Positive]}{Divine, Nature, Vitality}
        \miscastexplode
    \end{spellfooter}
\end{spellsection}

\begin{spellsection}{Cure Light Wounds}[1]
    \begin{spellheader}
        \spelldesc{You lay your hand on a creature and channel positive energy into it, healing some of its wounds.}
    \end{spellheader}
    \begin{spellcontent}
        \begin{spelltargetinginfo}
            \spelltwocol{\spelltgt{One creature}}{\spellrng{\rngmed}}
        \end{spelltargetinginfo}
        \begin{spelleffects}
            \spelleffect If the target is living, it is healed for \spelldamage{1}{}.
            \begin{spellattacktriggered}{If the target is undead, make a Spellpower vs. Fortitude attack.}
                \spellsuccess \spelldamage{1}{positive}.
                \spellfailure Half damage.
            \end{spellattacktriggered}
        \end{spelleffects}
    \end{spellcontent}
    \begin{spellfooter}
        \spellinfo{Vivimancy [Positive]}{Divine, Nature}
        \miscastrandom
    \end{spellfooter}
\end{spellsection}

\begin{spellsection}[Mass]{Cure Light Wounds}[5]
    \begin{spellheader}
        \spelldesc{You stretch out your hand and channel positive energy into all of your allies, healing some of their wounds.}
    \end{spellheader}
    \begin{spellcontent}
        \begin{spelltargetinginfo}
            \spelltwocol{\spelltgts{Up to five creatures}}{\spellrng{\rngmed}}
        \end{spelltargetinginfo}
        \begin{spelleffects}
            \spellspecial This spell functions like \spell{cure light wounds}, except that it affects multiple creatures.
        \end{spelleffects}
    \end{spellcontent}
    \begin{spellfooter}
        \spellinfo{Vivimancy [Positive]}{Divine, Nature, Vitality}
        \miscastexplode
    \end{spellfooter}
\end{spellsection}

\begin{spellsection}{Cure Moderate Wounds}[2]
    \begin{spellheader}
        \spelldesc{You lay your hand on a creature and channel positive energy into it, healing its wounds.}
    \end{spellheader}
    \begin{spellcontent}
        \begin{spelltargetinginfo}
            \spelltwocol{\spelltgt{One creature}}{\spellrng{\rngmed}}
        \end{spelltargetinginfo}
        \begin{spelleffects}
            \spellspecial This spell functions like \spell{cure light wounds}, except that for every 10 points of healing granted by this spell, it can instead cure 1 point of critical damage.
        \end{spelleffects}
    \end{spellcontent}
    \begin{spellfooter}
        \spellinfo{Vivimancy [Positive]}{Divine, Nature, Vitality}
        \miscastrandom
    \end{spellfooter}
\end{spellsection}

\begin{spellsection}[Mass]{Cure Moderate Wounds}[6]
    \begin{spellheader}
        \spelldesc{You stretch out your hand and channel positive energy into all of your allies, healing their wounds.}
    \end{spellheader}
    \begin{spellcontent}
        \begin{spelltargetinginfo}
            \spelltwocol{\spelltgts{Up to five creatures}}{\spellrng{\rngmed}}
        \end{spelltargetinginfo}
        \begin{spelleffects}
            \spellspecial This spell functions like \spell{cure moderate wounds}, except that it affects multiple creatures.
        \end{spelleffects}
    \end{spellcontent}
    \begin{spellfooter}
        \spellinfo{Vivimancy [Positive]}{Divine, Nature, Vitality}
        \miscastexplode
    \end{spellfooter}
\end{spellsection}

\begin{spellsection}{Cure Serious Wounds}[3]
    \begin{spellheader}
        \spelldesc{You lay your hand on a creature and channel positive energy into it, healing even serious injuries.}
    \end{spellheader}
    \begin{spellcontent}
        \begin{spelltargetinginfo}
            \spelltwocol{\spelltgt{One creature}}{\spellrng{\rngmed}}
        \end{spelltargetinginfo}
        \begin{spelleffects}
            \spellspecial This spell functions like \spell{cure light wounds}, except that for every 5 points of healing granted by this spell, it can instead cure 1 point of critical damage.
        \end{spelleffects}
    \end{spellcontent}
    \begin{spellfooter}
        \spellinfo{Vivimancy [Positive]}{Divine, Nature, Vitality}
        \miscastrandom
    \end{spellfooter}
\end{spellsection}

\begin{spellsection}[Mass]{Cure Serious Wounds}[7]
    \begin{spellheader}
        \spelldesc{You stretch out your hand and channel positive energy into all of your allies, healing even serious injuries.}
    \end{spellheader}
    \begin{spellcontent}
        \begin{spelltargetinginfo}
            \spelltwocol{\spelltgts{Up to five creatures}}{\spellrng{\rngmed}}
        \end{spelltargetinginfo}
        \begin{spelleffects}
            \spellspecial This spell functions like \spell{cure serious wounds}, except that it affects multiple creatures.
        \end{spelleffects}
    \end{spellcontent}
    \begin{spellfooter}
        \spellinfo{Vivimancy [Positive]}{Divine, Nature, Vitality}
        \miscastexplode
    \end{spellfooter}
\end{spellsection}

\begin{spellsection}{Curse of Blood and Bone}[3]
    \begin{spellheader}
        \spelldesc{You curse your foe's body, leaving it vulnerable to attacks.}
    \end{spellheader}
    \begin{spellcontent}
        \begin{spelltargetinginfo}
            \spelltwocol{\spelltgt{One creature}}{\spellrng{\rngmed}}
        \end{spelltargetinginfo}
        \begin{spelleffects}
            \begin{spellattack}{Spellpower vs. Mental}
                \spellsuccess \spelldamage{3}{life}. In addition, the target's maximum hit points are reduced by the amount of damage it takes from this effect, to a minimum of 1 hit point, for 5 rounds.
                \spellcritical As above, but the hit point reduction lasts for 1 year.
                \spellfailure Half damage, and no additional effects.
            \end{spellattack}
            \spelldur 5 rounds or 1 year
        \end{spelleffects}
    \end{spellcontent}
    \begin{spellfooter}
        \spellinfo{Vivimancy [Curse, Flesh]}{Death, Divine, Evil, Vivimancy}
        \spellnotes \cursespellnotes
        \miscastrandom
    \end{spellfooter}
\end{spellsection}

\begin{spellsection}{Curse of Enfeeblement}[4]
    \begin{spellheader}
%\spelldesc{You fire a coruscating ray from your hand. When it strikes your foe, he becomes weaker.}
    \end{spellheader}
    \begin{spellcontent}
        \begin{spelltargetinginfo}
            \spelltwocol{\spelltgt{One creature}}{\spellrng{\rngclose}}
        \end{spelltargetinginfo}
        \begin{spelleffects}
            \begin{spellattack}{Spellpower vs. Fortitude}
                \spellsuccess The target takes a \minus4 penalty to your choice of Strength or Dexterity.
                \spellfailure As above, but the penalty is halved.
            \end{spellattack}
            \spelldur One year
        \end{spelleffects}
    \end{spellcontent}
    \begin{spellfooter}
        \spellinfo{Vivimancy [Curse, Flesh]}{Vivimancy, Death}
        \spellnotes This spell cannot reduce an attribute below \minus9. \cursespellnotes
        \miscastrandom
    \end{spellfooter}
\end{spellsection}

\begin{spellsection}{Curse of the Wayward Mind}[8]
    \begin{spellheader}
    \end{spellheader}
    \begin{spellcontent}
        \begin{spelltargetinginfo}
            \spelltwocol{\spelltgt{One creature}}{\spellrng{\rngclose}}
        \end{spelltargetinginfo}
        \begin{spelleffects}
            \begin{spellattack}{Spellpower vs. Mental}
                \spellsuccess The target is \disoriented.
                \spellcritical The target is \confused.
                \spellfailure The target is \disoriented for 5 rounds.
            \end{spellattack}
            \spelldur One year
        \end{spelleffects}
    \end{spellcontent}
    \begin{spellfooter}
        \spellinfo{Vivimancy [Curse]}{Arcane}
        \spellnotes \cursespellnotes
        \miscastrandom
    \end{spellfooter}
\end{spellsection}

\pdfbookmark[2]{D}{SpellDescriptionsD}

\begin{spellsection}{Dancing Lights}[1]
    \begin{spellheader}
        \spelldesc{You create floating lights to guide your way.}
    \end{spellheader}
    \begin{spellcontent}
        \begin{spelltargetinginfo}
            \spellrng{\rngmed}
        \end{spelltargetinginfo}
        \begin{spelleffects}
            \spelleffect This spell creates mobile sources of light. You can create up to four lights which resemble lanterns or torches, up to four glowing spheres of light, or a single glowing, vaguely humanoid shape. Regardless of their form, each light creates bright illumination in a \areamed radius, as a torch.

            As a swift action, you can move the lights as you desire through the air. They can move up to 100 feet per round, but they must always stay within range of you. Any light which goes beyond that limit winks out.
            \spelldur \durshort \dismissable
        \end{spelleffects}
    \end{spellcontent}
    \begin{spellfooter}
        \spellinfo{Illusion [Figment, Light]}{Arcane}
        \spellnotes This spell can be made permanent with a \spell{permanency} ritual.
        \miscastexplode
    \end{spellfooter}
\end{spellsection}

\begin{spellsection}{Darkvision}[2]
    \begin{spellheader}
        \spelldesc{You grant an ally the ability to see in complete darkness.}
    \end{spellheader}
    \begin{spellcontent}
        \begin{spelltargetinginfo}
            \spelltwocol{\spelltgt{One creature}}{\spellrng{\rngtouch}}
        \end{spelltargetinginfo}
        \begin{spelleffects}
            \spelleffect The target gains the ability to see 50 feet even in total darkness. Beyond 60 feet, the target can see dimly, treating areas of darkness as shadowy illumination. Darkvision does not function if a creature is in an area of bright light or is dazzled. Darkvision is black and white only, but otherwise like normal sight.
            \spelldur \durlong
        \end{spelleffects}
    \end{spellcontent}
    \begin{spellfooter}
        \spellinfo{Transmutation [Augment]}{Arcane}
        \spellnotes This spell does not grant the ability to see in magical darkness. It can be made permanent with a \spell{permanency} ritual.
        \miscastexplode
    \end{spellfooter}
\end{spellsection}

\begin{spellsection}{Death Knell}[6]
    \begin{spellheader}
        \spelldesc{You draw forth the ebbing life force of a creature and use it to fuel your own power.}
    \end{spellheader}
    \begin{spellcontent}
        \begin{spelltargetinginfo}
            \spelltwocol{\spelltgt{Living creature}}{\spellrng{\rngmed}}
        \end{spelltargetinginfo}
        \begin{spelleffects}
            \begin{spellattack}{Spellpower vs. Fortitude}
                \spellsuccess \spelldamage{6}{life}. In addition, for 5 rounds, the target automatically dies if it has no hit points remaining.

                If the target dies in this way, you gain temporary hit points equal to your spellpower. These temporary hit points last for 1 round per level the target had.

                \spellfailure Half damage, and no additional effects.
            \end{spellattack}
            \spelldur See text
        \end{spelleffects}
    \end{spellcontent}
    \begin{spellfooter}
        \spellinfo{Vivimancy [Death]}{Death, Evil, Vivimancy}
        \spellnotes If you take life damage, you lose all temporary hit points provided by this spell before applying the damage.
        \miscastrandom
    \end{spellfooter}
\end{spellsection}

\begin{spellsection}{Death Ward}[3]
    \begin{spellheader}
        \spelldesc{You shield an ally from deadly spells.}
    \end{spellheader}
    \begin{spellcontent}
        \begin{spelltargetinginfo}
            \spelltwocol{\spelltgt{One living creature}}{\spellrng{\rngclose}}
        \end{spelltargetinginfo}
        \begin{spelleffects}
            \spelleffect The target is immune to all death spells, magical death effects, energy drain, and any negative energy effects.
            \spelldur \durshort
        \end{spelleffects}
    \end{spellcontent}
    \begin{spellfooter}
        \spellinfo{Abjuration/Vivimancy [Positive, Shielding]}*{Death, Divine, Good, Protection}
        \spellnotes This spell doesn't remove negative levels that the target has already gained. It does not protect against other sorts of attacks, even if those attacks might be lethal.
        \miscastrandom
    \end{spellfooter}
\end{spellsection}

\begin{spellsection}[Mass]{Death Ward}[7]
    \begin{spellheader}
        \spelldesc{You shield your allies from deadly spells.}
    \end{spellheader}
    \begin{spellcontent}
        \begin{spelltargetinginfo}
            \spelltwocol{\spelltgts{Five living creatures}}{\spellrng{\rngmed}}
        \end{spelltargetinginfo}
        \begin{spelleffects}
            \spelleffect The target is protected, as \spell{death ward}.
            \spelldur \durshort
        \end{spelleffects}
    \end{spellcontent}
    \begin{spellfooter}
        \spellinfo{Abjuration/Vivimancy [Positive, Shielding]}*{Death, Divine}
        \miscastexplode
    \end{spellfooter}
\end{spellsection}

\begin{spellsection}{Deep Slumber}[9]
    \begin{spellheader}
        \spelldesc{You fill your foe with an overpowering urge to sleep, inevitably rendering it comatose.}
    \end{spellheader}
    \begin{spellcontent}
        \begin{spelltargetinginfo}
            \spelltwocol{\spelltgt{One creature}}{\spellrng{\rngmed}}
        \end{spelltargetinginfo}
        \begin{spelleffects}
            \begin{spellattack}{Spellpower vs. Mental}
                \spellsuccess The target is \blinded for 5 rounds.
                \spellcritical The target falls asleep. It cannot be awakened by any means for 5 rounds. After that time, it can be awoken by other creatures, but if left undisturbed, it will sleep until it dies.
                \spellfailure The target is \dazed for 5 rounds.
            \end{spellattack}
            \spelldur See text
        \end{spelleffects}
    \end{spellcontent}
    \begin{spellfooter}
        \spellinfo{Enchantment [Compulsion, Mind]}{Arcane}
        \spellnotes Creatures unable to sleep, such as elves, are immune to all effects of this spell.
        \miscastrandom
    \end{spellfooter}
\end{spellsection}

\begin{spellsection}{Deflection}[3]
    \begin{spellheader}
        \spelldesc{You shield yourself from enemy attacks, causing them to deflect away from you harmlessly.}
    \end{spellheader}
    \begin{spellcontent}
        \begin{spelltargetinginfo}
            \spelltgt{You}
        \end{spelltargetinginfo}
        \begin{spelleffects}
            \spelleffect You gain a defensive legend point. If you spend it, you get another legend point 5 rounds later.
            \spelldur \durlong
        \end{spelleffects}
    \end{spellcontent}
    \begin{spellfooter}
        \spellinfo{Abjuration [Shielding]}{Abjuration}
        \miscastexplode
    \end{spellfooter}
\end{spellsection}

\begin{spellsection}{Delay Damage}[3]
    \begin{spellheader}
        \spelldesc{You partially shift yourself into the future, delaying the impact of attacks against you.}
    \end{spellheader}
    \begin{spellcontent}
        \begin{spelltargetinginfo}
            \spelltgt{You}
        \end{spelltargetinginfo}
        \begin{spelleffects}
            \spelleffect Whenever you take damage, half of the damage (rounded down) is not dealt to you immediately. This damage is tracked separately. At the end of the spell's duration, you take all of the delayed damage at once. When this happens, any damage in excess of your hit points is dealt as critical damage.
            \spelldur \durmed
        \end{spelleffects}
        \begin{spellfooter}
            \spellinfo{Abjuration/Transmutation [Shielding, Temporal]}{Divine, Nature}
            \miscastexplode
        \end{spellfooter}
    \end{spellcontent}
\end{spellsection}

\begin{spellsection}{Delay Poison}[2]
    \begin{spellheader}
    \end{spellheader}
    \begin{spellcontent}
        \begin{spelltargetinginfo}
            \spelltwocol{\spelltgt{One creature}}{\spellrng{\rngclose}}
            \spelltime{1 swift action}
        \end{spelltargetinginfo}
        \begin{spelleffects}
            \spelleffect The target becomes temporarily immune to the effects of poison. Poisons it is exposed to do not make attacks against it. This effect does not prevent the target from becoming poisoned, and any poisons in the target's system when the spell ends will continue their effects normally. 
            \spelldur \durshort
        \end{spelleffects}
    \end{spellcontent}
    \begin{spellfooter}
        \spellinfo{Vivimancy [Flesh]}{Divine, Nature}
        \spellnotes This spell does not cure any damage that poison may have already done.
        \miscastrandom
    \end{spellfooter}
\end{spellsection}

\begin{spellsection}{Delayed Blast Fireball}[7]
    \begin{spellheader}
    \end{spellheader}
    \begin{spellcontent}
        \begin{spelltargetinginfo}
            \spelltwocol{\spellburst{\areamed radius}}{\spellrng{\rngmed}}
            \spelltgts{Everything in the area}
        \end{spelltargetinginfo}
        \begin{spelleffects}
            \spellspecial You can delay this spell's attack until up to 5 rounds after the spell is cast. You select the amount of delay upon completing the spell, and that time cannot change once it has been set unless someone touches the bead (see below). For every round that this spell is delayed, your spellpower with it increases by 2.

            If you choose a delay, a glowing bead of fire sits at the point of origin, shedding light as a torch, until it detonates. It cannot be physically harmed or moved, but it can be dispelled, which prevents it from detonating.
            \begin{spellattack}{Spellpower vs. Reflex}
                \spellsuccess \spelldamage{7}{fire}[d6]
                \spellfailure Half damage.
            \end{spellattack}
            \spelldur 5 rounds or less; see text
        \end{spelleffects}
    \end{spellcontent}
    \begin{spellfooter}
        \spellinfo{Evocation [Destructive, Fire]}{Arcane, Fire}
        \spellnotes As \spell{fireball}.
        \miscastyou
    \end{spellfooter}
\end{spellsection}

\begin{spellsection}{Destruction}[7]
    \begin{spellheader}
    \end{spellheader}
    \begin{spellcontent}
        \begin{spelltargetinginfo}
            \spelltwocol{\spelltgt{One creature}}{\spellrng{\rngclose}}
        \end{spelltargetinginfo}
        \begin{spelleffects}
            \begin{spellattack}{Spellpower vs. Fortitude}
                \spellsuccess \spelldamage{7}{divine}. In addition, the target is \staggered for 5 rounds.

                \spellcritical The target dies, and divine fire utterly consumes its body. Its equipment is unaffected.

                \spellfailure Half damage, and no additional effects.
            \end{spellattack}
        \end{spelleffects}
    \end{spellcontent}
    \begin{spellfooter}
        \spellinfo{Vivimancy [Death, Flesh]}{Destruction, Divine}
        \miscastrandom
    \end{spellfooter}
\end{spellsection}

\begin{spellsection}{Detect Alignment}[2]
    \begin{spellheader}
        \spelldesc{You sense the presence of creatures with a particular alignment.}
    \end{spellheader}
    \begin{spellcontent}
        \begin{spelltargetinginfo}
            \spellemanation{\arealarge cone from you}
        \end{spelltargetinginfo}
        \begin{spelleffects}
            \spelleffect As you cast this spell, you choose an alignment: good, evil, lawful, or chaotic. Anything within the spell's area that has the chosen alignment has a faint aura, visible only to you.

            As a swift action, you can concentrate on an aura to determine the strength of the aura. Most aligned creatures and magic items have a faint aura. Creatures that embody the alignment, such as outsiders with the appropriate subtype and undead,  have a moderate aura. Creatures that act directly on behalf of the alignment, such as paladins, have a strong aura. Extraordinary magical objects or effects, such as artifacts, can also have a strong aura.
            \spelldur Concentration
        \end{spelleffects}
    \end{spellcontent}
    \begin{spellfooter}
        \spellinfo{Divination [Detection]}{Divine}
        \spellnotes Each round, you can turn to detect objects in a new area. A detection spell can penetrate barriers, but 1 foot of stone, 1 inch of common metal, a thin sheet of lead, or 3 feet of wood or dirt blocks it.
        \miscastexplode
    \end{spellfooter}
\end{spellsection}

\begin{spellsection}{Dictum}[6]
    \begin{spellheader}
        \spelldesc{You utter a powerful command, binding your foes in place.}
    \end{spellheader}
    \begin{spellcontent}
        \begin{spelltargetinginfo}
            \spellburst{\arealarge radius centered on you}
            \spelltgts{All nonlawful creatures in the area}
            \spellcmp{Verbal only}
        \end{spelltargetinginfo}
        \begin{spelleffects}
            \begin{spellattack}{Spellpower vs. Mental}
                \spellsuccess \spelldamage{6}{divine}[d6].
                \spellcritical As above, and the target is \immobilized for 5 rounds.
                \spellfailure Half damage.
            \end{spellattack}
        \end{spelleffects}
    \end{spellcontent}
    \begin{spellfooter}
        \spellinfo{Evocation [Lawful]}{Divine, Law}
        \miscastexplode
    \end{spellfooter}
\end{spellsection}

\begin{spellsection}{Dimension Door}[4]
    \begin{spellheader}
    \end{spellheader}
    \begin{spellcontent}
        \begin{spelltargetinginfo}
            \spelltgt{You}
        \end{spelltargetinginfo}
        \begin{spelleffects}
            \spelleffect You teleport to a destination within 1,000 feet of you. You must clearly visualize the destination, but you do not need line of sight or line of effect. After arriving, you cannot act until the next action phase.

            If the destination is occupied, or dramatically different from how you visualized it, the spell fails.
        \end{spelleffects}
    \end{spellcontent}
    \begin{spellfooter}
        \spellinfo{Conjuration [Teleportation]}{Arcane, Travel}
        \miscastexplode
    \end{spellfooter}
\end{spellsection}

\begin{spellsection}[Mass]{Dimension Door}[8]
    \begin{spellheader}
    \end{spellheader}
    \begin{spellcontent}
        \begin{spelltargetinginfo}
            \spellrng{\rngmed}
            \spelltgts{Up to five willing creatures}
        \end{spelltargetinginfo}
        \begin{spelleffects}
            \spelleffect The target teleports to a destination you specify within 1,000 feet of you, as \spell{dimension door}.
        \end{spelleffects}
    \end{spellcontent}
    \begin{spellfooter}
        \spellinfo{Conjuration [Teleportation]}{Conjuration, Travel}
        \spellnotes You can choose the destinations for each target independently, within the range of the spell. 
        \miscastexplode
    \end{spellfooter}
\end{spellsection}

\begin{spellsection}{Dimension Slide}[2]
    \begin{spellheader}
    \end{spellheader}
    \begin{spellcontent}
        \begin{spelltargetinginfo}
            \spelltwocol{\spelltgt{One creature}}{\spellrng{\rngmed}}
        \end{spelltargetinginfo}
        \begin{spelleffects}
            \begin{spellattack}{Spellpower vs. Mental}
                \spelleffect The target teleports to a destination in range. The destination must be an unoccupied space on stable ground. If the destination is invalid, the spell fails.
            \end{spellattack}
        \end{spelleffects}
    \end{spellcontent}
    \begin{spellfooter}
        \spellinfo{Conjuration [Teleportation]}{Conjuration, Travel}
        \miscastrandom
    \end{spellfooter}
\end{spellsection}

\begin{spellsection}{Dimensional Anchor}[3]
    \begin{spellheader}
        \spelldesc{You sever your foe's connection to the Astral Plane, trapping it where it is.}
    \end{spellheader}
    \begin{spellcontent}
        \begin{spelltargetinginfo}
            \spelltwocol{\spelltgt{One creature}}{\spellrng{\rngmed}}
        \end{spelltargetinginfo}
        \begin{spelleffects}
            \begin{spellattack}{Spellpower vs. Mental}
                \spellsuccess  The target cannot travel extradimensionally for 5 rounds. This blocks teleportation and all planar travel abilities except planar rifts.
                \spellcritical As above, except that the effect lasts for 1 year.
            \end{spellattack}
        \end{spelleffects}
    \end{spellcontent}
    \begin{spellfooter}
        \spellinfo{Abjuration [Antimagic]}{Arcane, Divine, Magic}
        \spellnotes This spell cannot be dispelled. It can only be removed by physically travelling to the Astral Plane, such as through a planar rift or the gate created by the \spell{gate} ritual.

        This spell does not interfere with the movement of creatures already in ethereal or astral form when the spell is cast, nor does it block extradimensional perception or attack forms, such as summoning monsters. Also, it does not prevent summoned creatures from disappearing at the end of a summoning spell.
        \miscastrandom
    \end{spellfooter}
\end{spellsection}

\begin{spellsection}{Discern Lies}[4]
    \begin{spellheader}
        \spelldesc{You can discern subtle magical disturbances caused by lying.}
    \end{spellheader}
    \begin{spellcontent}
        \begin{spelltargetinginfo}
            \spellemanation{\arealarge cone from you}
        \end{spelltargetinginfo}
        \begin{spelleffects}
            \spelleffect You know when any creature in the area deliberately and knowingly speaks a lie. The spell does not reveal the truth, uncover unintentional inaccuracies, or necessarily reveal evasions.
            \spelldur Concentration
        \end{spelleffects}
    \end{spellcontent}
    \begin{spellfooter}
        \spellinfo{Divination [Detection]}{Divine, Law}
        \spellnotes Each round, you can turn to discern lies in a new area. A detection spell can penetrate barriers, but 1 foot of stone, 1 inch of common metal, a thin sheet of lead, or 3 feet of wood or dirt blocks it.
        \miscastexplode
    \end{spellfooter}
\end{spellsection}

\begin{spellsection}{Discern Vulnerability}[3]
    \begin{spellheader}
    \end{spellheader}
    \begin{spellcontent}
        \begin{spelltargetinginfo}
            \spellquicktargeting{One creature}{\rngmed}
            \spelltime{1 swift action}
        \end{spelltargetinginfo}
        \begin{spelleffects}
            \spelleffect You instantly learn all of the target's weakenesses. This includes, but is not limited to, the following information:
            \begin{itemize}
                \item Which of the target's defenses is lowest
                \item If the target has any vulnerabilities to specific damage types
                \item How to overcome the target's damage reduction, regeneration, or other similar abilities
            \end{itemize}
            \spelldur \durshort
        \end{spelleffects}
    \end{spellcontent}
    \begin{spellfooter}
        \spellinfo{Divination}{Arcane}
        \spellnotes This spell gives no information about a creature's strengths or abilities -- only its weaknesses.
        \miscastrandom
    \end{spellfooter}
\end{spellsection}

\begin{spellsection}{Discordant Song}[8]
    \begin{spellheader}
        \spelldesc{Magical music fills the air, sowing confusion among your foes.}
    \end{spellheader}
    \begin{spellcontent}
        \begin{spelltargetinginfo}
            \spelltwocol{\spellburst{\areamed radius}}{\spellrng{\rngmed}}
            \spelltgts{All creatures in the area}
        \end{spelltargetinginfo}
        \begin{spelleffects}
            \begin{spellattack}{Spellpower vs. Mental}
                \spellsuccess The target is \disoriented.

                \spellcritical The target is \confused.
            \end{spellattack}
            \spelldur \durshort
        \end{spelleffects}
    \end{spellcontent}
    \begin{spellfooter}
        \spellinfo{Enchantment [Auditory, Compulsion, Mind]}{Arcane, Chaos}
        \miscastyou
    \end{spellfooter}
\end{spellsection}

\begin{spellsection}{Disintegrate}[6]
    \begin{spellheader}
        \spelldesc{You shoot a thin, green ray from your pointing finger that completely destroys whatever it hits.}
    \end{spellheader}
    \begin{spellcontent}
        \begin{spelltargetinginfo}
            \spelltwocol{\spelltgt{One creature or object}}{\spellrng{\rngclose}}
        \end{spelltargetinginfo}
        \begin{spelleffects}
            \begin{spellattack}{Spellpower vs. Fortitude}
                \spellsuccess \spelldamage{6}{physical}. If the target has no hit points remaining, it dies. Its body is completely disintegrated, leaving behind only a pinch of fine dust. Its equipment is unaffected.
                \spellfailure As above, but half damage.
            \end{spellattack}
            \spellspecial When used against an object, this spell simply disintegrates as much as one 10-foot cube of nonliving matter. Thus, the spell disintegrates only part of any very large object or structure targeted.
        \end{spelleffects}
    \end{spellcontent}
    \begin{spellfooter}
        \spellinfo{Transmutation [Alteration]}{Arcane, Destruction}
        \spellnotes This spell affects even objects constructed entirely of force, such as \spell{wall of force}, but not magical effects such as an \spell{antimagic field}.
        \miscastrandom
    \end{spellfooter}
\end{spellsection}

\begin{spellsection}{Disjoin Magic}[9]
    \begin{spellheader}
    \end{spellheader}
    \begin{spellcontent}
        \begin{spelltargetinginfo}
            \spellspecial This spell has two versions: an area dispel, and a targetted destruction of a magic item. Its effects depend on which version is chosen.
        \end{spelltargetinginfo}
    \end{spellcontent}
    \begin{spellsubcontent}
        \begin{spelltargetinginfo}
            \spelltwocol{\spellburst{\areamed radius burst}}{\spellrng{\rngmed}}
        \end{spelltargetinginfo}
        \begin{spelleffects}
            \spelleffect All spells in the area are dispelled.
        \end{spelleffects}
    \end{spellsubcontent}
    \begin{spellsubcontent}
        \begin{spelltargetinginfo}
            \spelltgt{One magic item}
        \end{spelltargetinginfo}
        \begin{spelleffects}
            \begin{spellattack}{Spellpower vs. 10 \add the spellpower of the target object}
                \spellsuccess The target item is permanently rendered nonmagical.
                \spellfailure The target item is suppressed for 5 rounds. A suppressed object loses all its magical abilities, though it is still treated as being a magical object for the purpose of spells and effects.
                \spellspecial If the item is an artifact, there is only a 1\% chance per spellpower that the spell works. If you destroy an artifact in this way, you permanently lose the ability to cast this spell.
            \end{spellattack}
        \end{spelleffects}
    \end{spellsubcontent}
    \begin{spellfooter}
        \spellinfo{Abjuration [Shielding]}{Arcane, Magic}
        \spellnotes Destroying artifacts is dangerous, and it is likely to attract the attention of some powerful being who has an interest in or connection with the device.
        \miscastyou
    \end{spellfooter}
\end{spellsection}

\begin{spellsection}{Dispel Magic}[3]
    \begin{spellheader}
        \spelldesc{You destroy magical effects}.
    \end{spellheader}
    \begin{spellcontent}
        \begin{spelltargetinginfo}
            \spelltwocol{\spelltgts{One creature, object, or location}}{\spellrng{\rngmed}}
        \end{spelltargetinginfo}
        \begin{spelleffects}
            \begin{spellattack}{Spellpower vs. Special}
                \spelleffect For every spell affecting the target, if the attack result beats a DC equal to 10 \add the spellpower of the spell, the spell is dispelled.

                If the target is an object, and the attack result beats a DC equal to 10 \add the spellpower of the object, the object is suppressed for 5 rounds. A suppressed object loses all its magical abilities, though it is still treated as being a magical object for the purpose of spells and effects.

                If the target is an effect of an ongoing spell (such as a summoned creature), and the attack result beats a DC equal to 10 \add the spellpower of the spell, the target is treated as if the spell that created it was dispelled. This usually causes the target to disappear.
            \end{spellattack}
        \end{spelleffects}
    \end{spellcontent}
    \begin{spellfooter}
        \spellinfo{Abjuration [Antimagic]}{Arcane, Divine, Magic, Nature}
        \spellnotes When a spell is dispelled, all its effects with a duration end. Unless otherwise specified, any spell with a lasting effect can be dispelled.

        If a spell affects multiple targets, it must be dispelled individually on each target. Dispelling the effect on one target does not affect the other targets of the spell.

        You may choose to automatically succeed or fail on your attack against any spell that you cast yourself.

        Spell-like abilities are treated like spells, and this spell affects them in the same way it affects spells.

        Artifacts and deities are unaffected by mortal magic such as this.
        \miscastrandom
    \end{spellfooter}
\end{spellsection}

\begin{spellsection}[Greater]{Dispel Magic}[6]
    \begin{spellheader}
    \end{spellheader}
    \begin{spellcontent}
        \begin{spelltargetinginfo}
            \spelltwocol{\spellarea{\areamed radius limit}}{\spellrng{\rngmed}}
            \spelltgts{All creatures and unattended objects in the area}
        \end{spelltargetinginfo}
        \begin{spelleffects}
            \begin{spellattack}{Spellpower vs. Special}
                \spelleffect Spells affecting the target are dispelled, as \spell{dispel magic}.
            \end{spellattack}
        \end{spelleffects}
    \end{spellcontent}
    \begin{spellfooter}
        \spellinfo{Abjuration [Antimagic]}{Arcane, Divine, Magic, Nature}
        \spellnotes As \spell{dispel magic}.
        \miscastyou
    \end{spellfooter}
\end{spellsection}

\begin{spellsection}{Displacement}[3]
    \begin{spellheader}
        \spelldesc{You shift your ally's image, causing it to appear to be about 2 feet away from its true location.}
    \end{spellheader}
    \begin{spellcontent}
        \begin{spelltargetinginfo}
            \spelltwocol{\spelltgt{One creature}}{\spellrng{\rngclose}}
        \end{spelltargetinginfo}
        \begin{spelleffects}
            \spelleffect Targeted physical attacks against the target have a 50\% miss chance. Spells and other special attacks suffer no miss chance.
            \spelldur \durshort \dismissable
        \end{spelleffects}
    \end{spellcontent}
    \begin{spellfooter}
        \spellinfo{Illusion [Glamer]}{Arcane}
        \miscastrandom
    \end{spellfooter}
\end{spellsection}

\begin{spellsection}{Divine Favor}[1]
    \begin{spellheader}
        \spelldesc{You imbue yourself with skill in combat by calling upon the divine power of your patron.}
        \spelldur \durshort
    \end{spellheader}
    \begin{spellcontent}
        \begin{spelltargetinginfo}
            \spelltgt{You}
        \end{spelltargetinginfo}
        \begin{spelleffects}
            \spelleffect You gain a legend point.
            \spelldur \durshort \dismissable
        \end{spelleffects}
    \end{spellcontent}
    \begin{spellfooter}
        \spellinfo{Transmutation [Augment]}{Divine, Strength, War}
        \miscastexplode
    \end{spellfooter}
\end{spellsection}

\begin{spellsection}[Greater]{Divine Favor}[5]
    \begin{spellheader}
        \spelldesc{You imbue yourself with great strength and skill in combat by calling upon the divine power of your patron.}
    \end{spellheader}
    \begin{spellcontent}
        \begin{spelltargetinginfo}
            \spelltgt{You}
        \end{spelltargetinginfo}
        \begin{spelleffects}
            \spelleffect You gain a legend point. If you spend it, you get another legend point 5 rounds later.
            \spelldur \durlong \dismissable
        \end{spelleffects}
    \end{spellcontent}
    \begin{spellfooter}
        \spellinfo{Transmutation [Augment]}{Divine, Strength, War}
        \miscastexplode
    \end{spellfooter}
\end{spellsection}

\begin{spellsection}{Divine Might}[6]
    \begin{spellheader}
    \end{spellheader}
    \begin{spellcontent}
        \begin{spelltargetinginfo}
            \spelltgt{You}
        \end{spelltargetinginfo}
        \begin{spelleffects}
            \spellsuccess You become larger, as \spell{enlarge person}. In addition, you gain damage reduction against physical damage equal to your spellpower. Appropriately aligned damage ignores this damage reduction and negates it for 1 round. Evil attacks ovvercome your damage reduction if you are good or neutral, and good attacks overcome your damage reduction if you are evil.
            \spelldur \durshort \dismissable
        \end{spelleffects}
    \end{spellcontent}
    \begin{spellfooter}
        \spellinfo{Transmutation [Alteration, Augment, Sizing]}*{Divine, Good, Strength}
        \spellnotes \sizingspellnotes
        \miscastexplode
    \end{spellfooter}
\end{spellsection}


\begin{spellsection}{Dominate Monster}[7]
    \begin{spellheader}
    \end{spellheader}
    \begin{spellcontent}
        \begin{spelltargetinginfo}
            \spelltwocol{\spelltgt{One creature}}{\spellrng{\rngmed}}
        \end{spelltargetinginfo}
        \begin{spelleffects}
            \begin{spellattack}{Spellpower vs. Mental}
                \spellsuccess The target is dominated, as \spell{dominate person}, except that the effect does not depend on creature type.
            \end{spellattack}
        \end{spelleffects}
    \end{spellcontent}
    \begin{spellfooter}
        \spellinfo{Enchantment [Compulsion, Mind]}{Enchantment}
        \miscastrandom
    \end{spellfooter}
\end{spellsection}

\begin{spellsection}{Dominate Person}[5]
    \begin{spellheader}
    \end{spellheader}
    \begin{spellcontent}
        \begin{spelltargetinginfo}
            \spelltwocol{\spelltgt{One humanoid creature}}{\spellrng{\rngmed}}
        \end{spelltargetinginfo}
        \begin{spelleffects}
            \begin{spellattack}{Spellpower vs. Mental}

                \spellsuccess The target is \confused for 5 rounds.

                \spellcritical The target is dominated for 5 rounds. It obeys your commands unquestioningly, as an automaton. If you have a shared language, you can command the target to perform any task, and it will obey you immediately. If you lack a shared language, you can still issue simple commands, such as ``attack'' or ``follow''.

                When this effect's duration ends, you must make another Spellpower vs. Mental attack against the target. If you succeed, the target remains dominated for another 5 rounds. If you fail, the target breaks free of your control. If you critically succeed, the target remains dominated for an additional 24 hours. If the effect's duration is extended, this attack must be repeated each time it ends until the domination is broken.

                \spellfailure The target is \dazed for 5 rounds.
            \end{spellattack}
            \spelldur \durshort \dismissable
        \end{spelleffects}
    \end{spellcontent}
    \begin{spellfooter}
        \spellinfo{Enchantment [Compulsion, Mind, Subtle]}{Enchantment}
        \spellnotes This spell gives you no special ability to communicate with the target, except as noted above. Rituals such as \spell{telepathic bond} can be used to exert influence over a dominated creature from a distance.
        \miscastrandom
    \end{spellfooter}
\end{spellsection}

\begin{spellsection}{Drain Life}[5]
    \begin{spellheader}
    \end{spellheader}
    \begin{spellcontent}
        \begin{spelltargetinginfo}
            \spelltwocol{\spelltgt{One living creature}}{\spellrng{\rngclose}}
        \end{spelltargetinginfo}
        \begin{spelleffects}
            \begin{spellattack}{Spellpower vs. Fortitude}
                \spellsuccess \spelldamage{5}{life}. You gain temporary hit points equal to half the damage you deal. You can't gain more hit points than the target had.
                
                The temporary hit points disappear after 5 minutes. If you take life damage, you lose all temporary hit points provided by this spell before applying the damage.
                \spellfailure As above, but half damage.
            \end{spellattack}
        \end{spelleffects}
    \end{spellcontent}
    \begin{spellfooter}
        \spellinfo{Vivimancy [Life]}{Arcane}
        \miscastrandom
    \end{spellfooter}
\end{spellsection}

\pdfbookmark[2]{E}{SpellDescriptionsE}

\begin{spellsection}{Earth's Pull}[1]
    \begin{spellheader}
        \spelldesc{You intensify the pull of gravity on your foe, causing it to feel much heavier and making its movements sluggish.}
    \end{spellheader}
    \begin{spellcontent}
        \begin{spelltargetinginfo}
            \spellrng{\rngclose}
            \spelltgt{One Large or smaller creature within 10 feet of solid ground}
        \end{spelltargetinginfo}
        \begin{spelleffects}
            \begin{spellattack}{Spellpower vs. Mental}
                \spellsuccess The target is \immobilized. If it is flying, it crashes to the ground.
                \spellfailure The target moves at half speed.
            \end{spellattack}
            \spelldur \durshort
        \end{spelleffects}
    \end{spellcontent}
    \begin{spellfooter}
        \spellinfo{Evocation [Earth]}{Earth, Nature, Wild}
        \spellnotes If the target gets farther than 10 feet from the ground, the spell's effect is broken.
        \miscastrandom
    \end{spellfooter}
\end{spellsection}

\begin{spellsection}{Earthen Blade}[2]
    \begin{spellheader}
    \end{spellheader}
    \begin{spellcontent}
        \begin{spelltargetinginfo}
            \spellrng{Touch}
        \end{spelltargetinginfo}
        \begin{spelleffects}
            \spelleffect This spell creates a weapon from the ground. The weapon can be of any type you are proficient with. In addition, the weapon is magical, as the \spell{magic weapon} spell.
            \spelldur \durlong \dismissable
        \end{spelleffects}
    \end{spellcontent}
    \begin{spellfooter}
        \spellinfo{Transmutation [Alteration, Augment, Earth]}{Earth, Nature}
        \miscastexplode
    \end{spellfooter}
\end{spellsection}

\begin{spellsection}{Earth Glide}[4]
    \begin{spellheader}
    \end{spellheader}
    \begin{spellcontent}
        \begin{spelltargetinginfo}
            \spelltwocol{\spelltgt{One creature}}{\spellrng{Touch}}
        \end{spelltargetinginfo}
        \begin{spelleffects}
            \spelleffect The target gains the earth glide ability, as an earth elemental. This allows it to glide through stone, dirt, or almost any other sort of earth as if it were air. The target can walk or climb at any angle in the earth. However, the target generally cannot breathe, speak, or hear while gliding. While gliding, the target can remain partially within the earth, granting it cover.
            \spelldur \durshort
        \end{spelleffects}
    \end{spellcontent}
    \begin{spellfooter}
        \spellinfo{Transmutation [Augment, Earth]}{Earth, Nature}
        \spellnotes The target's burrowing leaves behind no tunnel or hole, nor does it create any ripple or other signs of its presence.
        \miscastexplode
    \end{spellfooter}
\end{spellsection}

\begin{spellsection}{Earthquake}[8]
    \begin{spellheader}
        \spelldesc{An intense but highly localized tremor shakes the ground. The shock knocks creatures down, and rifts open in the earth to trap unwary creatures.}
    \end{spellheader}
    \begin{spellcontent}
        \begin{spelltargetinginfo}
            \spelltwocol{\spellzone{\arealarge radius}}{\spellrng{\rngmed}}
            \spelltgts{All creatures on the ground in the area}
        \end{spelltargetinginfo}
        \begin{spelleffects}
            \begin{spellattack}{Spellpower vs. Reflex}
                \spellsuccess The target is knocked prone and trapped in a crack in the ground, causing it to be \immobilized. It can escape with a grapple or Escape Artist check against a DC equal to 10 \add your spellpower.
            \end{spellattack}
        \end{spelleffects}
    \end{spellcontent}
    \begin{spellfooter}
        \spellinfo{Evocation [Earth, Physical]}{Destruction, Divine, Earth, Nature}
        \spellnotes In terrain with unusual ground, such as rivers or swamps, this spell may have different effects.

        \physicalspellnotes
        \miscastyou
    \end{spellfooter}
\end{spellsection}

\begin{spellsection}{Earthspike}[1]
    \begin{spellheader}
        \spelldesc{You create a spike from the ground that impales your foe.} 
    \end{spellheader}
    \begin{spellcontent}
        \begin{spelltargetinginfo}
            \spellrng{\rngmed}
            \spelltgt{One creature or object within 10 feet of natural earth or stone}
        \end{spelltargetinginfo}
        \begin{spelleffects}
            \begin{spellattack}{Spellpower vs. Armor defense}
                \spellsuccess 1d6 piercing damage per spellpower.
            \end{spellattack}
        \end{spelleffects}
    \end{spellcontent}
    \begin{spellfooter}
        \spellinfo{Transmutation [Alteration, Earth, Physical]}{Earth, Nature}
        \miscastrandom
    \end{spellfooter}
\end{spellsection}

\begin{spellsection}[Mass]{Earthspike}[4]
    \begin{spellheader}
    \end{spellheader}
    \begin{spellcontent}
        \begin{spelltargetinginfo}
            \spelltwocol{\spellburst{\areasmall radius}}{\spellrng{\rngmed}}
            \spelltgts{Everything in the area within 10 feet of natural earth or stone}
        \end{spelltargetinginfo}
        \begin{spelleffects}
            \begin{spellattack}{Spellpower vs. Armor defense}
                \spellsuccess 1d6 piercing damage per two spellpower.
            \end{spellattack}
        \end{spelleffects}
    \end{spellcontent}
    \begin{spellfooter}
        \spellinfo{Transmutation [Alteration, Earth, Physical]}{Earth, Nature}
        \spellnotes This spell cannot attack more than one target within a single 5-ft. square.
        \miscastexplode
    \end{spellfooter}
\end{spellsection}

\begin{spellsection}{Energy Conversion}[8]
    \begin{spellheader}
    \end{spellheader}
    \begin{spellcontent}
        \begin{spelleffects}
            \spelleffect You gain damage reduction against all energy types (acid, cold, electricity, fire) equal to twice your spellpower.

            If you absorb damage with this spell from a single energy type that exceeds your spellpower, you gain a charge for that energy type. You can store up to 3 charges of any combination of energy types. Additional charges replace existing charges of your choice.
            \spelldur \durlong
        \end{spelleffects}
    \end{spellcontent}
    \begin{spellsubcontent}
        \begin{spelltargetinginfo}
            \spellspecial As a standard action, you can expend a charge to fire a bolt of energy.
            \spelltwocol{\spelltgt{One creature or object}}{\spellrng{\rngclose}}
        \end{spelltargetinginfo}
        \begin{spelleffects}
            \begin{spellattack}{Spellpower vs. Reflex}
                \spellsuccess \spelldamage{8}{energy}[d8]. The damage type is the same as the type of the charge expended.
                \spellfailure Half damage.
            \end{spellattack}
        \end{spelleffects}
    \end{spellsubcontent}
    \begin{spellfooter}
        \spellinfo{Abjuration/Evocation [Shielding]}{Arcane, Protection}
        \miscastexplode
    \end{spellfooter}
\end{spellsection}

\begin{spellsection}{Enervation}[4]
    \begin{spellheader}
        \spelldesc{Your foe's body loses its color momentarily as you drain its life force away.}
    \end{spellheader}
    \begin{spellcontent}
        \begin{spelltargetinginfo}
            \spelltwocol{\spelltgt{One creature}}{\spellrng{\rngclose}}
        \end{spelltargetinginfo}
        \begin{spelleffects}
            \spelleffect If the target is living, it gains two \conditionlink{negative levels}. This imposes a \minus2 penalty to the target's attacks, special defenses, and checks, and a \minus10 penalty to its current and maximum hit points.

            If the target is undead, it gains damage reduction against physical damage equal to your spellpower. Positive damage ignores this damage reduction and negates it for 1 round.
        \end{spelleffects}
    \end{spellcontent}
    \begin{spellfooter}
        \spellinfo{Vivimancy [Negative]}{Arcane, Death, Divine, Evil}
        \spellnotes These negative levels do not stack with other negative levels the target has, if any.
        \miscastrandom
    \end{spellfooter}
\end{spellsection}

\begin{spellsection}[Greater]{Enervation}[8]
    \begin{spellheader}
        \spelldesc{Your foe's body loses its color momentarily as you drain its life force away.}
    \end{spellheader}
    \begin{spellcontent}
        \begin{spelltargetinginfo}
            \spelltwocol{\spelltgt{One creature}}{\spellrng{\rngmed}}
        \end{spelltargetinginfo}
        \begin{spelleffects}
            \spelleffect If the target is undead, it gains an offensive legend point and damage reduction against physical damage equal to your spellpower. Positive damage ignores this damage reduction and negates it for 1 round.
            \begin{spellattacktriggered}{If the target is living, make a Spellpower vs. Fortitude attack.}
                \spellsuccess The target gains eight \conditionlink{negative levels}. This imposes a \minus8 penalty to the target's attacks, special defenses, and checks, and a \minus40 penalty to its current and maximum hit points.
                \spellsuccess The target gains four \conditionlink{negative levels}. This imposes a \minus4 penalty to the target's attacks, special defenses, and checks, and a \minus20 penalty to its current and maximum hit points.
            \end{spellattacktriggered}
        \end{spelleffects}
    \end{spellcontent}
    \begin{spellfooter}
        \spellinfo{Vivimancy [Negative]}{Arcane, Death, Divine, Evil}
        \spellnotes These negative levels do not stack with other negative levels the target has, if any.
        \miscastrandom
    \end{spellfooter}
\end{spellsection}

\begin{spellsection}{Enlarge Monster}[4]
    \begin{spellheader}
    \end{spellheader}
    \begin{spellcontent}
        \begin{spelltargetinginfo}
            \spelltwocol{\spelltgts{One creature (Large or smaller)}}{\spellrng{\rngmed}}
            \spelltime{Full-round action}
        \end{spelltargetinginfo}
        \begin{spelleffects}
            \spellspecial This spell functions like \spell{enlarge person}, except that the target can be a creature of any type.
        \end{spelleffects}
    \end{spellcontent}
    \begin{spellfooter}
        \spellinfo{Transmutation [Alteration, Sizing]}{Nature, Strength, Wild}
        \spellnotes \sizingspellnotes
        \miscastrandom
    \end{spellfooter}
\end{spellsection}

\begin{spellsection}[Mass]{Enlarge Monster}[7]
    \begin{spellheader}
    \end{spellheader}
    \begin{spellcontent}
        \begin{spelltargetinginfo}
            \spelltwocol{\spelltgts{Five creatures (Large or smaller)}}{\spellrng{\rngmed}}
            \spelltime{Full-round action}
        \end{spelltargetinginfo}
        \begin{spelleffects}
            \spelleffect The target is enlarged, as \spell{enlarge person}.
            \spelldur \durshort \dismissable
        \end{spelleffects}
    \end{spellcontent}
    \begin{spellfooter}
        \spellinfo{Transmutation [Alteration, Sizing]}{Strength, Transmutation}
        \spellnotes \sizingspellnotes
        \miscastexplode
    \end{spellfooter}
\end{spellsection}

\begin{spellsection}{Enlarge Person}[3]
    \begin{spellheader}
    \end{spellheader}
    \begin{spellcontent}
        \begin{spelltargetinginfo}
            \spelltwocol{\spelltgt{One humanoid creature}}{\spellrng{\rngmed}}
            \spelltime{Full-round action}
        \end{spelltargetinginfo}
        \begin{spelleffects}
            \begin{spellattack}{Spellpower vs. Fortitude}
                \spellsuccess The target and its equipment instantly grows, doubling its height and multiplying its weight by 8. This changes the creature's size category to the next larger one. This has several effects.
                \begin{itemize}
                    \item \plus10 ft. bonus to movement speed.
                    \item \plus4 bonus to maneuver attack and defense.
                    \item \minus1 penalty to other physical attacks and defenses.
                    \item \minus4 penalty to Stealth checks.
                    \item Melee weapons increase damage die size by one.
                \end{itemize}
                \par If insufficient room is available for the desired growth, the creature attains the maximum possible size and may make a Strength check (using its increased Strength) to burst any enclosures in the process. If it fails, it is constrained without harm by the materials enclosing it -- the spell cannot be used to crush a creature by increasing its size.
                \par Equipment that leaves the target's possession returns to its original size.
            \end{spellattack}
            \spelldur \durshort \dismissable
        \end{spelleffects}
    \end{spellcontent}
    \begin{spellfooter}
        \spellinfo{Transmutation [Alteration, Sizing]}{Strength, Transmutation}
        \spellnotes A typical humanoid creature whose size increases to Large has a space of 10 feet and a natural reach of 10 feet. \sizingspellnotes

        This spell can be made permanent with a \spell{permanency} ritual.
        \miscastrandom
    \end{spellfooter}
\end{spellsection}

\begin{spellsection}[Mass]{Enlarge Person}[6]
    \begin{spellheader}
    \end{spellheader}
    \begin{spellcontent}
        \begin{spelltargetinginfo}
            \spelltwocol{\spelltgts{Five humanoid creatures}}{\spellrng{\rngmed}}
            \spelltime{Full-round action}
        \end{spelltargetinginfo}
        \begin{spelleffects}
            \spelleffect The target is enlarged, as \spell{enlarge person}.
            \spelldur \durshort \dismissable
        \end{spelleffects}
    \end{spellcontent}
    \begin{spellfooter}
        \spellinfo{Transmutation [Alteration, Sizing]}{Strength, Transmutation}
        \spellnotes \sizingspellnotes
        \miscastexplode
    \end{spellfooter}
\end{spellsection}

\begin{spellsection}{Entangle}[1]
    \begin{spellheader}
        \spelldesc{Plants grow and ensnare your foe.}
    \end{spellheader}
    \begin{spellcontent}
        \begin{spelltargetinginfo}
            \spelltwocol{\spelltgt{One creature within 5 feet of plants}}{\spellrng{\rngmed}}
        \end{spelltargetinginfo}
        \begin{spelleffects}
            \begin{spellattack}{Spellpower vs. Reflex}
                \spellsuccess The target is \immobilized.
                \spellfailure The target moves at half speed.
                \spellspecial The target can break this spell's effect with a grapple or Escape Artist check against a DC equal to 10 \add your spellpower.
            \end{spellattack}
            \spelldur \durshort
        \end{spelleffects}
    \end{spellcontent}
    \begin{spellfooter}
        \spellinfo{Transmutation [Alteration, Physical]}{Nature, Wild}
        \spellnotes The effects of this spell may be altered somewhat based on the nature of the plants in the area. If no plants exist in the area, this spell has no effect.
        \miscastrandom
    \end{spellfooter}
\end{spellsection}

\begin{spellsection}[Greater]{Entangle}[4]
    \begin{spellheader}
        \spelldesc{Plants grow out of nowhere and ensnare your foe.}
    \end{spellheader}
    \begin{spellcontent}
        \begin{spelltargetinginfo}
            \spelltwocol{\spelltgt{One creature}}{\spellrng{\rnglong}}
        \end{spelltargetinginfo}
        \begin{spelleffects}
            \spellspecial This spell functions like \spell{entangle}, except that the target does not need to be near plants.
        \end{spelleffects}
    \end{spellcontent}
    \begin{spellfooter}
        \spellinfo{Transmutation [Alteration, Physical]}{Nature, Wild}
        \spellnotes The effects of this spell may be altered somewhat based on the nature of the plants in the area.
        \miscastrandom
    \end{spellfooter}
\end{spellsection}

\begin{spellsection}{Entropic Shield}[1]
    \begin{spellheader}
        \spelldesc{You surround your ally with a magical field that glows with a chaotic blast of multicolored hues. This field deflects incoming ranged attacks, causing them to randomly swerve away from their intended target.}
    \end{spellheader}
    \begin{spellcontent}
        \begin{spelltargetinginfo}
            \spelltwocol{\spelltgt{One creature}}{\spellrng{\rngclose}}
        \end{spelltargetinginfo}
        \begin{spelleffects}
            \spelleffect Each physical ranged attack directed at the target has a 50\% miss chance. Other attacks that simply work at a distance are not affected.
            \spelldur \durshort \dismissable
        \end{spelleffects}
    \end{spellcontent}
    \begin{spellfooter}
        \spellinfo{Abjuration [Shielding]}{Chaos, Divine}
        \miscastrandom
    \end{spellfooter}
\end{spellsection}

\begin{spellsection}{Ethereal Jaunt}[6]
    \begin{spellheader}
    \end{spellheader}
    \begin{spellcontent}
        \begin{spelltargetinginfo}
            \spelltgt{You}
        \end{spelltargetinginfo}
        \begin{spelleffects}
            \spelleffect You become ethereal, along with your equipment. For the duration of the spell, you are in a place called the Ethereal Plane, which overlaps the normal, physical, Material Plane. When the spell expires, you return to material existence.
            \par An ethereal creature is invisible, insubstantial, and capable of moving in any direction, even up or down, albeit at half normal speed. As an insubstantial creature, you can move through solid objects, including living creatures. An ethereal creature can see and hear on the Material Plane, but everything looks gray and ephemeral. Sight and hearing onto the Material Plane are limited to 50 feet.
            \par Force effects and abjurations affect an ethereal creature normally. Their effects extend onto the Ethereal Plane from the Material Plane, but not vice versa. An ethereal creature can't attack material creatures, and spells you cast while ethereal affect only other ethereal things. Certain material creatures or objects have attacks or effects that work on the Ethereal Plane (such as a basilisk's gaze attack). Treat other ethereal creatures and ethereal objects as if they were material. 
            \par If you end the spell and become material while inside a material object (such as a solid wall), you are shunted off to the nearest open space and take 1d6 damage per 5 feet that you so travel.
            \spelldur \durshort \dismissable
        \end{spelleffects}
    \end{spellcontent}
    \begin{spellfooter}
        \spellinfo{Conjuration [Planar]}{Arcane, Travel}
        \spellnotes If you are not on the Material Plane when you cast this spell, it has no effect.
        \miscastexplode
    \end{spellfooter}
\end{spellsection}

\begin{spellsection}{Etherealness}[9]
    \begin{spellheader}
    \end{spellheader}
    \begin{spellcontent}
        \begin{spelltargetinginfo}
            \spelltwocol{\spelltgt{You and up to five willing creatures}}{\spellrng{Touch}}
        \end{spelltargetinginfo}
        \begin{spelleffects}
            \spelleffect The target becomes ethereal, as \spell{ethereal jaunt}.
        \end{spelleffects}
    \end{spellcontent}
    \begin{spellfooter}
        \spellinfo{Conjuration [Planar]}{Arcane, Travel}
        \spellnotes If you are not on the Material Plane when you cast this spell, it has no effect. When the spell expires, all affected creatures on the Ethereal Plane return to the Material Plane.
        \miscastexplode
    \end{spellfooter}
\end{spellsection}

\begin{spellsection}{Eyebite}[4]
    \begin{spellcontent}
        \begin{spelltargetinginfo}
            \spellquicktargeting{One creature}{\rngmed}
        \end{spelltargetinginfo}
        \begin{spelleffects}
            \spellsuccess \spelldamage{4}{life}. In addition, the target's vision is \impaired. This affects all sight-related actions, including physical attacks and targeted spells.
            \spellcritical As above, except that the target's vision is \severelyimpaired.
            \spellfailure Half damage, and no additional effects.
        \end{spelleffects}
    \end{spellcontent}
    \begin{spellfooter}
        \spellinfo{Vivimancy [Flesh]}{Arcane}
        \spellnotes This spell has no effect on creatures without eyes.
        \miscastrandom
    \end{spellfooter}
\end{spellsection}

\pdfbookmark[2]{F}{SpellDescriptionsF}

\begin{spellsection}{Faerie Fire}[2]
    \begin{spellheader}
    \end{spellheader}
    \begin{spellcontent}
        \begin{spelltargetinginfo}
            \spelltwocol{\spellburst{\areasmall radius}}{\spellrng{\rngmed}}
            \spelltgts{Everything in the area}
        \end{spelltargetinginfo}
        \begin{spelleffects}
            \spelleffect A pale glow surrounds and outlines the target, causing it to shed light as a candle. This imposes a \minus20 penalty to Stealth checks, and negates invisibility, concealment, and similar effects. 
            \spelldur \durshort \dismissable
        \end{spelleffects}
    \end{spellcontent}
    \begin{spellfooter}
        \spellinfo{Illusion [Figment, Light]}{Nature}
        \spellnotes Illusory figments, such as those created by the \spell{silent image} spell, are not outlined, which may reveal their false nature. The lights continue illuminating creatures after they leave the area.
        \miscastyou
    \end{spellfooter}
\end{spellsection}

\begin{spellsection}{False Reality}[9]
    \begin{spellheader}
    \end{spellheader}
    \begin{spellcontent}
        \begin{spelltargetinginfo}
            \spellzone{1 mile radius centered on you}
        \end{spelltargetinginfo}
        \begin{spelleffects}
            \spelleffect A scripted figment of your design appears within the area, as \spell{persistent image}.
            \spelldur \durlong \dismissable
        \end{spelleffects}
    \end{spellcontent}
    \begin{spellfooter}
        \spellinfo{Illusion [Figment, Unreal]}{Illusion}
        \spellnotes Creatures can identify the illusion, as \spell{silent image}.
        \miscastexplode
    \end{spellfooter}
\end{spellsection}

\begin{spellsection}{Fear}[2]
    \begin{spellheader}
        \spelldesc{You terrify your foe.}
    \end{spellheader}
    \begin{spellcontent}
        \begin{spelltargetinginfo}
            \spelltwocol{\spelltgt{One creature}}{\spellrng{\rngclose}}
        \end{spelltargetinginfo}
        \begin{spelleffects}
            \begin{spellattack}{Spellpower vs. Mental}
                \spellsuccess The target is \frightened by you.
                \spellcritical The target is \panicked by you.
                \spellfailure The target is \shaken by you.
            \end{spellattack}
            \spelldur \durshort \dismissable
        \end{spelleffects}
    \end{spellcontent}
    \begin{spellfooter}
        \spellinfo{Enchantment [Fear, Mind]}{Arcane}
        \miscastrandom
    \end{spellfooter}
\end{spellsection}

\begin{spellsection}[Mass]{Fear}[6]
    \begin{spellheader}
        \spelldesc{You terrify your foe.}
    \end{spellheader}
    \begin{spellcontent}
        \begin{spelltargetinginfo}
            \spelltwocol{\spelltgt{Up to five creatures}}{\spellrng{\rngmed}}
        \end{spelltargetinginfo}
        \begin{spelleffects}
            \begin{spellattack}{Spellpower vs. Mental}
                \spellsuccess The target is \frightened by you.
                \spellcritical The target is \panicked by you.
                \spellfailure The target is \shaken by you.
            \end{spellattack}
            \spelldur \durshort \dismissable
        \end{spelleffects}
    \end{spellcontent}
    \begin{spellfooter}
        \spellinfo{Enchantment [Fear, Mind]}{Arcane}
        \miscastexplode
    \end{spellfooter}
\end{spellsection}

\begin{spellsection}{Feather Fall}[1]
    \begin{spellheader}
    \end{spellheader}
    \begin{spellcontent}
        \begin{spelltargetinginfo}
            \spelltwocol{One freefalling object or willing creature (Medium or smaller)}{\spellrng{\rngmed}}
            \spelltwocol{\spelltime{1 swift action}}{\spellcmp{Verbal only}}
        \end{spelltargetinginfo}
        \begin{spelleffects}
            \spelleffect The target falls at only 60 feet per round (equivalent to the end of a fall from a few feet). It takes no falling damage from falls of any length. If the object is heavy enough to deal falling damage to other creatures and objects, it deals half its normal falling damage, with no bonus for the height of the drop.
            \spelldur \durshort
        \end{spelleffects}
    \end{spellcontent}
    \begin{spellfooter}
        \spellinfo{Evocation [Air]}{Air, Arcane, Travel}
        \spellnotes This spell works only upon free-falling objects and creatures. It no special effect on ranged weapons or projectiles unless they are falling an extraordinary distance.
        \miscastrandom
    \end{spellfooter}
\end{spellsection}

\begin{spellsection}{Feeblemind}[8]
    \begin{spellheader}
    \end{spellheader}
    \begin{spellcontent}
        \begin{spelltargetinginfo}
            \spelltwocol{\spelltgt{One creature}}{\spellrng{\rngclose}}
        \end{spelltargetinginfo}
        \begin{spelleffects}
            \begin{spellattack}{Spellpower vs. Mental}
                \spellsuccess The target's Intelligence drops to \minus9 for 5 rounds, giving it roughly the intellect of a lizard. It is unable to cast spells, understand language, or communicate coherently. Still, it knows who its friends are and can follow them and even protect them.
                \spellcritical As above, except that the effect is permanent.
                \spellfailure The target is \dazed for 5 rounds.
            \end{spellattack}
        \end{spelleffects}
    \end{spellcontent}
    \begin{spellfooter}
        \spellinfo{Enchantment [Delusion, Mind]}{Arcane}
        \miscastrandom
    \end{spellfooter}
\end{spellsection}

\begin{spellsection}{Finger of Death}[7]
    \begin{spellheader}
    \end{spellheader}
    \begin{spellcontent}
        \begin{spelltargetinginfo}
            \spelltwocol{\spelltgt{One living creature}}{\spellrng{\rngclose}}
        \end{spelltargetinginfo}
        \begin{spelleffects}
            \begin{spellattack}{Spellpower vs. Fortitude}
                \spellsuccess \spelldamage{7}{life}. In addition, the target is \staggered for 5 rounds.
                \spellcritical The target dies.
                \spellfailure Half damage, and no additional effects.
            \end{spellattack}
        \end{spelleffects}
    \end{spellcontent}
    \begin{spellfooter}
        \spellinfo{Vivimancy [Death]}{Arcane, Death}
        \miscastrandom
    \end{spellfooter}
\end{spellsection}

\begin{spellsection}{Fire Seeds}[6]
    \begin{spellheader}
    \end{spellheader}
    \begin{spellcontent}
        \begin{spelltargetinginfo}
            \spelltgts{Up to five acorns or berries}
        \end{spelltargetinginfo}
        \begin{spelleffects}
            \spelleffect The targets are imbued with fiery energy capable of dealing up to \spelldamage{6}{fire}[d6]. You may freely decide the distribution of dice between the target berries.

            You must also specify at least one command word used to detonate the seeds. You can specify different command words to detonate different combinations of seeds.
            \spelldur \durext or until discharged
        \end{spelleffects}
    \end{spellcontent}
    \begin{spellsubcontent}
        \begin{spelltargetinginfo}
            \spellspecial As a standard action, you can say one of your defined command words to detonate seeds.
            \spellrng{\rngmed}
            \spellarea{\areasmall burst centered on a detonating seed}
            \spelltgts{Everything in the area}
        \end{spelltargetinginfo}
        \begin{spelleffects}
            \begin{spellattack}{Spellpower vs. Reflex}
                \spellsuccess The target takes the damage imbued into the seed.
                \spellfailure Half damage.
                \spellspecial This attack automatically succeeds against a creature that is holding a seed when it detonates.
            \end{spellattack}
        \end{spelleffects}
    \end{spellsubcontent}
    \begin{spellfooter}
        \spellinfo{Evocation/Transmutation [Destructive, Fire]}{Fire, Nature, Wild}
        \spellnotes You can only have one \spell{fire seeds} spell active at any time.
        \miscastexplode
    \end{spellfooter}
\end{spellsection}

\begin{spellsection}{Fire Shield}[3]
    \begin{spellheader}
        \spelldesc{You appear to immolate yourself in a wreath of flame that lashes out at anyone who tries to harm you.}
    \end{spellheader}
    \begin{spellcontent}
        \begin{spelleffects}
            \spelleffect You gain damage reduction against cold damage equal to twice your spellpower. In addition, you radiate light as a torch.
            \spelldur \durshort \dismissable
        \end{spelleffects}
    \end{spellcontent}
    \begin{spellsubcontent}
        \begin{spelltargetinginfo}
            \spelltgr{A creature within 30 feet of you attacks you}
            \spelltgt{The attacking creature}
        \end{spelltargetinginfo}
        \begin{spelleffects}
            \begin{spellattack}{Spellpower vs. Reflex}
                \spellsuccess \spelldamage{4}{fire}[d6]
            \end{spellattack}
        \end{spelleffects}
    \end{spellsubcontent}
    \begin{spellfooter}
        \spellinfo{Abjuration/Evocation [Fire, Retributive, Shielding]}{Arcane, Fire}
        \miscastexplode
    \end{spellfooter}
\end{spellsection}

\begin{spellsection}{Fire Storm}[8]
    \begin{spellheader}
        \spelldesc{You fill a massive area with sheets of roaring flame, burning everyone who opposes you.}
    \end{spellheader}
    \begin{spellcontent}
        \begin{spelltargetinginfo}
            \spelltwocol{\spellburst{\arealarge radius}}{\spellrng{\rngmed}}
            \spelltgts{Everything in the area, except allied creatures and plants}
        \end{spelltargetinginfo}
        \begin{spelleffects}
            \begin{spellattack}{Spellpower vs. Reflex}
                \spellsuccess \spelldamage{8}{fire}[d6]
                \spellfailure Half damage.
            \end{spellattack}
        \end{spelleffects}
    \end{spellcontent}
    \begin{spellfooter}
        \spellinfo{Evocation [Destructive, Fire]}{Destruction, Fire, Nature, War}
        \miscastyou
    \end{spellfooter}
\end{spellsection}

\begin{spellsection}{Fireball}[3]
    \begin{spellheader}
        \spelldesc{You create an explosion of flame that detonates with a low roar, damaging nearby creatures and objects.}
    \end{spellheader}
    \begin{spellcontent}
        \begin{spelltargetinginfo}
            \spelltwocol{\spellburst{\areasmall radius}}{\spellrng{\rngmed}}
            \spelltgts{Everything in the area}
        \end{spelltargetinginfo}
        \begin{spelleffects}
            \begin{spellattack}{Spellpower vs. Reflex}
                \spellsuccess \spelldamage{3}{fire}[d6]
                \spellfailure Half damage.
            \end{spellattack}
        \end{spelleffects}
    \end{spellcontent}
    \begin{spellfooter}
        \spellinfo{Evocation [Destructive, Fire]}{Arcane, Fire}
        \spellnotes \destructivespellnotes

        \firespellnotes
        \miscastyou
    \end{spellfooter}
\end{spellsection}

\begin{spellsection}{Flame Blade}[2]
    \begin{spellheader}
        \spelldesc{You create a 3 foot long beam of red-hot fire to serve you as a weapon.}
    \end{spellheader}
    \begin{spellcontent}
        \begin{spelleffects}
            \spelleffect A scimitar-like weapon appears in your hand. You can attack with it as a light melee weapon, except that you use your spellpower in place of your Strength for damage, and it deals both fire and slashing damage.

            Alternately, you can hurl flames from the weapon up to \rngmed range as if it were a thrown weapon.

            \spelldur \durmed \dismissable
        \end{spelleffects}
    \end{spellcontent}
    \begin{spellfooter}
        \spellinfo{Evocation [Fire]}{Nature, Fire}
        \spellnotes Spell resistance applies when a foe is struck by the weapon, but not when the blade is created.
        \firespellnotes
        \miscastexplode
    \end{spellfooter}
\end{spellsection}

\begin{spellsection}{Flame Strike}[5]
    \begin{spellheader}
        \spelldesc{You call a vertical column of divine fire that roars downward, consuming your unworthy foes.}
    \end{spellheader}
    \begin{spellcontent}
        \begin{spelltargetinginfo}
            \spelltwocol{\spellburst{\areamed radius cylinder, 40 ft. high}}{\spellrng{\rngclose}}
            \spelltgts{Everything in the area, except allied creatures}
        \end{spelltargetinginfo}
        \begin{spelleffects}
            \begin{spellattack}{Spellpower vs. Reflex}
                \spellsuccess \spelldamage{5}{fire and divine}[d6]
                \spellfailure Half damage.
            \end{spellattack}
        \end{spelleffects}
    \end{spellcontent}
    \begin{spellfooter}
        \spellinfo{Evocation [Destructive, Fire]}{Destruction, Divine, Fire, War}
        \spellnotes \destructivespellnotes

        \firespellnotes
        \miscastyou
    \end{spellfooter}
\end{spellsection}

\begin{spellsection}{Fly}[4]
    \begin{spellheader}
    \end{spellheader}
    \begin{spellcontent}
        \begin{spelltargetinginfo}
            \spelltwocol{\spelltgt{One creature}}{\spellrng{\rngtouch}}
        \end{spelltargetinginfo}
        \begin{spelleffects}
            \spelleffect The target gains a 30 foot fly speed with good maneuverability.
            \spelldur \durshort
        \end{spelleffects}
    \end{spellcontent}
    \begin{spellfooter}
        \spellinfo{Transmutation [Augment]}{Arcane}
        \spellnotes An unencumbered creature with a fly speed can fly through the air. See \pcref{Flying}, for more details.
        \miscastexplode
    \end{spellfooter}%priced as buff spell for range
\end{spellsection}

\begin{spellsection}{Fog Cloud}[1]
    \begin{spellheader}
        \spelldesc{You conjure a bank of fog, concealing those inside.}
    \end{spellheader}
    \begin{spellcontent}
        \begin{spelltargetinginfo}
            \spelltwocol{\spellzone{\areamed radius cylinder}}{\spellrng{\rngmed}}
        \end{spelltargetinginfo}
        \begin{spelleffects}
            \spelleffect Fog blocks sight in the area, causing all creatures within or looking through the area to treat everything they see as if it had \concealment.
            \spelldur \durshort
        \end{spelleffects}
    \end{spellcontent}
    \begin{spellfooter}
        \spellinfo{Conjuration [Creation, Fog, Physical]}{Arcane, Nature, Water}
        \spellnotes \fogspellnotes \fogwindspellnotes

        \physicalspellnotes
        \miscastyou
    \end{spellfooter}
\end{spellsection}

\begin{spellsection}{Fog Sea}[7]
    \begin{spellheader}
    \end{spellheader}
    \begin{spellcontent}
        \begin{spelltargetinginfo}
            \spellzone{500 ft. radius cylinder centered on you, 50 ft. high}
        \end{spelltargetinginfo}
        \begin{spelleffects}
            \spelleffect Fog fills the area, as \spell{fog cloud}.
        \end{spelleffects}
    \end{spellcontent}
    \begin{spellfooter}
        \spellinfo{Conjuration [Creation, Fog, Physical]}{Arcane, Nature}
        \spellnotes \fogspellnotes A severe wind disperses the fog within 1 minute, a windstorm disperses it within 5 rounds, and a hurricane disperses it within a round.

        \physicalspellnotes
        \miscastexplode
    \end{spellfooter}
\end{spellsection}

\begin{spellsection}{Fog Shield}[5]
    \begin{spellheader}
        \spelldesc{You create a bank of fog that follows you, concealing you and your allies.}
    \end{spellheader}
    \begin{spellcontent}
        \begin{spelltargetinginfo}
            \spellemanation{\areamed radius cylinder centered on you}
        \end{spelltargetinginfo}
        \begin{spelleffects}
            \spelleffect Fog blocks sight in the area, as \spell{fog cloud}. If you move, new fog does not form immediately. At the end of each round, the fog in your previous location disappears, and fog forms around your current location.
            \spelldur \durshort
        \end{spelleffects}
    \end{spellcontent}
    \begin{spellfooter}
        \spellinfo{Abjuration/Conjuration [Creation, Fog, Physical]}{Arcane, Divine, Nature, Water}
        \spellnotes \fogspellnotes \fogwindspellnotes
        \miscastexplode
    \end{spellfooter}
\end{spellsection}

\begin{spellsection}{Forcecage}[5]
    \begin{spellheader}
    \end{spellheader}
    \begin{spellcontent}
        \begin{spelltargetinginfo}
            \spelltwocol{\spelltgt{One creature or object (Large or smaller)}}{\spellrng{\rngmed}}
        \end{spelltargetinginfo}
        \begin{spelleffects}
            \begin{spellattack}{Spellpower vs. Reflex}
                \spellsuccess An immobile, invisible prison appears around the target. The prison can be a perfect sphere, a perfect cube, or a barred cage. The cage bars are an inch wide, with one inch gaps between them.
            \end{spellattack}
            \spelldur \durshort \dismissable
        \end{spelleffects}
    \end{spellcontent}
    \begin{spellfooter}
        \spellinfo{Evocation [Force, Physical]}{Evocation}
        \spellnotes As \spell{wall of force}.
        \miscastrandom
    \end{spellfooter}
\end{spellsection}

\begin{spellsection}{Foresight}[3]
    \begin{spellheader}
    \end{spellheader}
    \begin{spellcontent}
        \begin{spelleffects}
            \spelleffect You cannot be caught \unaware.
            \spelldur \durlong
        \end{spelleffects}
    \end{spellcontent}
    \begin{spellfooter}
        \spellinfo{Divination}{Divination, Knowledge}
        \miscastexplode
    \end{spellfooter}
\end{spellsection}

\begin{spellsection}[Greater]{Foresight}[9]
    \begin{spellheader}
        \spelldesc{You bestow a powerful sixth sense to your ally, giving them clear visions of any imminent danger.}
    \end{spellheader}
    \begin{spellcontent}
        \begin{spelleffects}
            \spelleffect You cannot be caught \unaware, and gain a \plus30 bonus to initiative checks.
            \spelldur \durlong
        \end{spelleffects}
    \end{spellcontent}
    \begin{spellfooter}
        \spellinfo{Divination}{Arcane, Knowledge}
        \miscastexplode
    \end{spellfooter}
\end{spellsection}

\begin{spellsection}{Forget}[1]
    \begin{spellheader}
    \end{spellheader}
    \begin{spellcontent}
        \begin{spelltargetinginfo}
            \spelltwocol{\spelltgt{One creature}}{\spellrng{\rngmed}}
        \end{spelltargetinginfo}
        \begin{spelleffects}
            \begin{spellattack}{Spellpower vs. Mental}
                \spelleffect The target forgets something simple. You can't make it forget something important, such as its name. You must know what you want it to forget. The spell does not prevent the target from learning the information again, and it can remember the information normally after the spell's duration.
            \end{spellattack}
            \spelldur \durlong
        \end{spelleffects}
    \end{spellcontent}
    \begin{spellfooter}
        \spellinfo{Enchantment [Delusion]}{Chaos, Enchantment}
        \miscastrandom
    \end{spellfooter}
\end{spellsection}

\begin{spellsection}{Freedom}[3]
    \begin{spellheader}
    \end{spellheader}
    \begin{spellcontent}
        \begin{spelltargetinginfo}
            \spelltwocol{\spelltgt{One creature}}{\spellrng{\rngclose}}
        \end{spelltargetinginfo}
        \begin{spelleffects}
            \spelleffect The target is immune to effects that restrict its mobility, such as \spell{slow} or \spell{web}. It suffers no penalties for acting underwater. In addition, it gains a \plus20 bonus to Maneuver defense against grapple attacks, as well as on grapple attacks or Escape Artist checks made to escape a grapple or a pin.
            \spelldur \durshort
        \end{spelleffects}
    \end{spellcontent}
    \begin{spellfooter}
        \spellinfo{Transmutation [Augment]}{Divine, Nature, Travel}
        \miscastrandom
    \end{spellfooter}
\end{spellsection}

\begin{spellsection}[Mass]{Freedom}[7]
    \begin{spellheader}
    \end{spellheader}
    \begin{spellcontent}
        \begin{spelltargetinginfo}
            \spelltwocol{\spelltgts{Up to five creatures}}{\spellrng{\rngmed}}
        \end{spelltargetinginfo}
        \begin{spelleffects}
            \spelleffect The target can move freely, as \spell{freedom}.
            \spelldur \durshort
        \end{spelleffects}
    \end{spellcontent}
    \begin{spellfooter}
        \spellinfo{Transmutation [Augment]}{Divine, Nature, Travel}
        \miscastexplode
    \end{spellfooter}
\end{spellsection}

\pdfbookmark[2]{G}{SpellDescriptionsG}

\begin{spellsection}{Gaseous Form}[3]
    \begin{spellheader}
        \spelldesc{The target and all its equipment becomes insubstantial, misty, and translucent.}
    \end{spellheader}
    \begin{spellcontent}
        \begin{spelltargetinginfo}
            \spelltwocol{\spelltgt{One willing corporeal creature}}{\spellrng{\rngtouch}}
            \spellcmp{Somatic only}
        \end{spelltargetinginfo}
        \begin{spelleffects}
            \spelleffect The target becomes a cloud of mist. All its equipment melds into its new form, though magical equipment retains its effects. Its Armor defense becomes 10, but it is immune to physical damage and critical hits.

            As a cloud of mist, the target cannot take any physical actions other than movement. It has a fly speed of 10 feet, with perfect maneuverability. It can pass through holes and openings as narrow as one quarter inch, but cannot enter water or similar liquids.
            \spelldur \durshort \dismissable
        \end{spelleffects}
    \end{spellcontent}
    \begin{spellfooter}
        \spellinfo{Transmutation [Alteration]}{Arcane, Air, Travel}
        \miscastexplode
    \end{spellfooter}
\end{spellsection}

\begin{spellsection}{Gentle Descent}[1]
    \begin{spellheader}
        \spelldesc{You grant your ally ephemeral wings which allow him to glide.}
    \end{spellheader}
    \begin{spellcontent}
        \begin{spelltargetinginfo}
            \spelltwocol{\spelltgt{One creature}}{\spellrng{\rngclose}}
        \end{spelltargetinginfo}
        \begin{spelleffects}
            \spelleffect The target gains a 30 foot glide speed.
            \spelldur \durshort
        \end{spelleffects}
    \end{spellcontent}
    \begin{spellfooter}
        \spellinfo{Transmutation [Air, Augment]}{Air, Nature}
        \spellnotes A creature with a glide speed can glide through the air at the indicated speed (see \pcref{Gliding}).
        \miscastrandom
    \end{spellfooter}
\end{spellsection}

\begin{spellsection}[Mass]{Gentle Descent}[4]
    \begin{spellheader}
        \spelldesc{You grant your ally ephemeral wings which allow him to glide.}
    \end{spellheader}
    \begin{spellcontent}
        \begin{spelltargetinginfo}
            \spelltwocol{\spelltgt{Up to five creatures}}{\spellrng{\rngmed}}
        \end{spelltargetinginfo}
        \begin{spelleffects}
            \spelleffect The target gains a 30 foot glide speed.
            \spelldur \durshort
        \end{spelleffects}
    \end{spellcontent}
    \begin{spellfooter}
        \spellinfo{Transmutation [Air, Augment]}{Air, Nature}
        \spellnotes A creature with a glide speed can glide through the air at the indicated speed (see \pcref{Gliding}).
        \miscastexplode
    \end{spellfooter}
\end{spellsection}

\begin{spellsection}{Ghoul Touch}[2]
    \begin{spellheader}
        \spelldesc{Your foe feels the touch of a ghoul's undead hand against its flesh.}
    \end{spellheader}
    \begin{spellcontent}
        \begin{spelltargetinginfo}
            \spelltwocol{\spelltgt{One living creature}}{\spellrng{\rngclose}}
        \end{spelltargetinginfo}
        \begin{spelleffects}
            \begin{spellattack}{Spellpower vs. Fortitude}
                \spellsuccess The target is \staggered.
                \spellfailure The target is \sickened.
            \end{spellattack}
            \spelldur \durshort
        \end{spelleffects}
    \end{spellcontent}
    \begin{spellfooter}
        \spellinfo{Vivimancy [Flesh]}{Arcane}
        \miscastrandom
    \end{spellfooter}
\end{spellsection}

\begin{spellsection}{Glitterdust}[2]
    \begin{spellheader}
    \end{spellheader}
    \begin{spellcontent}
        \begin{spelltargetinginfo}
            \spelltwocol{\spellburst{\areasmall radius}}{\spellrng{\rngmed}}
            \spelltgts{Everything in the area}
        \end{spelltargetinginfo}
        \begin{spelleffects}
            \spelleffect Golden particles surround and outline the target. This imposes a \minus20 penalty to Stealth checks, and negates invisibility, concealment, and similar effects. Illusory figments, such as those created by the \spell{silent image} spell, are not outlined, which may reveal their false nature.

            The dust can be removed from the target by drenching it with water or a similar liquid, or by dealing 5 points of acid or fire damage.
            \spelldur \durshort
        \end{spelleffects}
    \end{spellcontent}
    \begin{spellfooter}
        \spellinfo{Conjuration [Creation, Physical]}{Arcane}
        \miscastyou
    \end{spellfooter}
\end{spellsection}

\begin{spellsection}{Grease}[1]
    \begin{spellheader}
        \spelldesc{You conjure a layer of slippery grease on the ground, tripping up your foes.}
    \end{spellheader}
    \begin{spellcontent}
        \begin{spelltargetinginfo}
            \spelltwocol{\spellzone{\areasmall radius}}{\spellrng{\rngclose}}
        \end{spelltargetinginfo}
        \begin{spelleffects}
            \spelleffect The ground in the area is covered in grease for 5 rounds, making it slippery. A DC 15 Balance check is usually required to move on oily surfaces. See \pcref{Balance}, for more details.
        \end{spelleffects}
    \end{spellcontent}
    \begin{spellsubcontent}
        \begin{spelltargetinginfo}
            \spelltgts{All creatures in the area}
        \end{spelltargetinginfo}
        \begin{spelleffects}
            \begin{spellattack}{Spellpower vs. Reflex}
                \spellsuccess The target falls prone.
            \end{spellattack}
        \end{spelleffects}
    \end{spellsubcontent}
    \begin{spellfooter}
        \spellinfo{Conjuration [Creation, Physical]}{Arcane}
        \spellnotes \physicalspellnotes
        \miscastyou
    \end{spellfooter}
\end{spellsection}

\begin{spellsection}{Greater (Spell Name)}
    \par Any spell whose name begins with greater is alphabetized in this chapter according to the second word of the spell name. Thus, the description of a greater spell appears near the description of the spell on which it is based. Spell chains that have greater spells in them include those based on the spells command, dispel magic, invisibility, magic fang, magic weapon, restoration, scrying, shadow conjuration, shadow evocation, shout, and teleport.
\end{spellsection}

\begin{spellsection}{Gust of Wind}[2]
    \begin{spellheader}
        \spelldesc{You create a severe blast of air that knocks your foes flying.}
    \end{spellheader}
    \begin{spellcontent}
        \begin{spelltargetinginfo}
            \spellzone{\arealarge line from you}
            \spelltgts{Everything in the area}
        \end{spelltargetinginfo}
        \begin{spelleffects}
            \begin{spellattack}{Spellpower vs. Maneuver Defense}
                \spellsuccess The target is affected by a shove attack, pushing it back by 5 feet \add 5 feet per 5 points by which your attack exceeded its defense. If it is pushed outside the spell's area, it is not pushed farther.
            \end{spellattack}
        \end{spelleffects}
    \end{spellcontent}
    \begin{spellfooter}
        \spellinfo{Evocation [Air]}{Air, Nature}
        \spellnotes In addition to the effect noted, a \spell{gust of wind} can do anything that a sudden blast of wind would be expected to do. It can extinguish open flames, create a stinging spray of sand or dust, fan a large fire, overturn delicate awnings or hangings, heel over a small boat, and blow gases or vapors to the edge of its range.

        This spell can be made permanent with a \spell{permanency} ritual.
        \miscastexplode
    \end{spellfooter}
\end{spellsection}

\pdfbookmark[2]{H}{SpellDescriptionsH}

\begin{spellsection}{Harm}[6]
    \begin{spellheader}
        \spelldesc{You fill your foe with a massive influx of negative energy, crippling its body.}
    \end{spellheader}
    \begin{spellcontent}
        \begin{spelltargetinginfo}
            \spelltwocol{\spelltgt{One creature}}{\spellrng{\rngmed}}
        \end{spelltargetinginfo}
        \begin{spelleffects}
            \begin{spellattack}{Spellpower vs. Fortitude}
                \spelleffect If the target is undead, it is healed for 1d10 damage per two spellpower.
                \spellsuccess If the target is not undead, it takes that much negative energy damage, as well as four points of Constitution damage.
                \spellfailure As above, but both the negative energy damage and constitution damage is halved.
            \end{spellattack}
        \end{spelleffects}
    \end{spellcontent}
    \begin{spellfooter}
        \spellinfo{Vivimancy [Negative]}{Arcane, Death, Divine, Evil, Vitality}
        \miscastrandom
    \end{spellfooter}
\end{spellsection}

\begin{spellsection}[Lesser]{Haste}[2]
    \begin{spellheader}
    \end{spellheader}
    \begin{spellcontent}
        \begin{spelltargetinginfo}
            \spelltwocol{\spelltgt{One creature}}{\spellrng{\rngclose}}
        \end{spelltargetinginfo}
        \begin{spelleffects}
            \spelleffect The target gains a \plus30 foot bonus to its speed in all its movement modes, up to a maximum of double its original speed.
            \spelldur \durshort \dismissable
        \end{spelleffects}
    \end{spellcontent}
    \begin{spellfooter}
        \spellinfo{Transmutation [Temporal]}{Transmutation}
        \spellnotes As with any effect that increases your speed, this effect affects your ability to jump (see \pcref{Jump}).
        \miscastrandom
    \end{spellfooter}
\end{spellsection}

\begin{spellsection}{Haste}[5]
    \begin{spellheader}
        \spelldesc{You accelerate your ally's motions, causing her to move and act more quickly than normal.}
    \end{spellheader}
    \begin{spellcontent}
        \begin{spelltargetinginfo}
            \spelltwocol{\spelltgt{One creature}}{\spellrng{\rngclose}}
        \end{spelltargetinginfo}
        \begin{spelleffects}
            \spelleffect The target's speed increases, as \spell{lesser haste}. In addition, when it takes a full attack action, it may make an additional attack at a \minus5 penalty.
            \spelldur \durshort
        \end{spelleffects}
    \end{spellcontent}
    \begin{spellfooter}
        \spellinfo{Transmutation [Temporal]}{Transmutation}
        \spellnotes As \spell{lesser haste}. The extra attack granted does not stack with similar effects.
        \miscastrandom
    \end{spellfooter}
\end{spellsection}

\begin{spellsection}[Mass]{Haste}[8]
    \begin{spellheader}
        \spelldesc{You accelerate your allies' motions, causing them to move and act more quickly than normal.}
    \end{spellheader}
    \begin{spellcontent}
        \begin{spelltargetinginfo}
            \spelltwocol{\spelltgts{Up to five creatures}}{\spellrng{\rngclose}}
        \end{spelltargetinginfo}
        \begin{spelleffects}
            \spellspecial This spell functions like \spell{haste}, except that it affects multiple creatures.
        \end{spelleffects}
    \end{spellcontent}
    \begin{spellfooter}
        \spellinfo{Transmutation [Temporal]}{Arcane}
        \miscastexplode
    \end{spellfooter}
\end{spellsection}

\begin{spellsection}{Heal}[6]
    \begin{spellheader}
        \spelldesc{You fill an ally with a massive influx of positive energy, restoring its body to perfect health.}
    \end{spellheader}
    \begin{spellcontent}
        \begin{spelltargetinginfo}
            \spelltwocol{\spelltgt{One creature}}{\spellrng{\rngclose}}
        \end{spelltargetinginfo}
        \begin{spelleffects}
            \spelleffect If the target is living, it is healed for \spelldamage{6}{}. For every 2 points of healing granted by this spell, it can instead cure 1 point of critical damage. In addition, all of the following conditions are also removed from the target: ability damage, blinded, confused, dazed, deafened, diseased, exhausted, fatigued, nauseated, sickened, stunned, and poisoned.
            \begin{spellattacktriggered}{If the target is undead, make a Spellpower vs. Fortitude attack.}
                \spellsuccess \spelldamage{6}{positive}.
                \spellfailure Half damage.
            \end{spellattacktriggered}
        \end{spelleffects}
    \end{spellcontent}
    \begin{spellfooter}
        \spellinfo{Vivimancy [Positive]}{Divine, Good, Nature, Vitality}
        \miscastrandom
    \end{spellfooter}
\end{spellsection}

\begin{spellsection}{Heat Metal}[2]
    \begin{spellheader}
        \spelldesc{You heat your foe's armor, blistering its skin.}
    \end{spellheader}
    \begin{spellcontent}
        \begin{spelltargetinginfo}
            \spelltwocol{\spelltgt{One metal object (Medium or smaller)}}{\spellrng{\rngmed}}
        \end{spelltargetinginfo}
        \begin{spelleffects}
            \begin{spellattack}{Spellpower vs. Mental}
                \spellsuccess The target and everything touching it takes 1d6 fire damage per two spellpower immediately, and again at the end of the next round.

                A creature wearing or holding the target object is \dazed each round it takes damage from the object.
            \end{spellattack}
        \end{spelleffects}
    \end{spellcontent}
    \begin{spellfooter}
        \spellinfo{Evocation [Fire]}{Nature}
        \spellnotes This spell's attack automatically succeeds against unattended objects.

        If the target is underwater, this spell deals half damage, and boils the surrounding water. Any cold intense enough to damage the creature negates fire damage from the spell (and vice versa) on a point-for-point basis.
        \miscastrandom
    \end{spellfooter}
\end{spellsection}

\begin{spellsection}{Heroism}[4]
    \begin{spellheader}
        \spelldesc{You imbue your ally with great bravery and morale in battle.}
    \end{spellheader}
    \begin{spellcontent}
        \begin{spelltargetinginfo}
            \spelltwocol{\spelltgt{One creature}}{\spellrng{\rngclose}}
        \end{spelltargetinginfo}
        \begin{spelleffects}
            \spelleffect The target becomes immune to fear and gains temporary hit points equal to your spellpower. In addition, it gains an offensive legend point.
            \spelldur \durshort \dismissable
        \end{spelleffects}
    \end{spellcontent}
    \begin{spellfooter}
        \spellinfo{Enchantment [Mind, Morale]}{Arcane}
        \miscastrandom
    \end{spellfooter}
\end{spellsection}

\begin{spellsection}[Greater]{Heroism}[8]
    \begin{spellheader}
    \end{spellheader}
    \begin{spellcontent}
        \begin{spelltargetinginfo}
            \spelltwocol{\spelltgts{Up to five creatures}}{\spellrng{\rngmed}}
        \end{spelltargetinginfo}
        \begin{spelleffects}
            \spellspecial This spell functions like \spell{heroism}, except that it affects multiple targets.
        \end{spelleffects}
    \end{spellcontent}
    \begin{spellfooter}
        \spellinfo{Enchantment [Mind, Morale]}{Arcane}
        \miscastexplode
    \end{spellfooter}
\end{spellsection}

\begin{spellsection}{Hold Monster}[4]
    \begin{spellheader}
    \end{spellheader}
    \begin{spellcontent}
        \begin{spelltargetinginfo}
            \spelltwocol{\spelltgt{One creature}}{\spellrng{\rngclose}}
        \end{spelltargetinginfo}
        \begin{spelleffects}
            \spelleffect The target is \immobilized.
            \spelldur \durshort
        \end{spelleffects}
    \end{spellcontent}
    \begin{spellfooter}
        \spellinfo{Enchantment [Compulsion, Mind]}{Arcane, Law}
        \miscastrandom
    \end{spellfooter}
\end{spellsection}

\begin{spellsection}[Mass]{Hold Monster}[8]
    \begin{spellheader}
    \end{spellheader}
    \begin{spellcontent}
        \begin{spelltargetinginfo}
            \spelltwocol{\spelltgts{Up to five creatures}}{\spellrng{\rngclose}}
        \end{spelltargetinginfo}
        \begin{spelleffects}
            \spelleffect The target is \immobilized.
            \spelldur \durshort
        \end{spelleffects}
    \end{spellcontent}
    \begin{spellfooter}
        \spellinfo{Enchantment [Compulsion, Mind]}{Arcane, Law}
        \miscastexplode
    \end{spellfooter}
\end{spellsection}

\begin{spellsection}{Hold Person}[2]
    \begin{spellheader}
    \end{spellheader}
    \begin{spellcontent}
        \begin{spelltargetinginfo}
            \spelltwocol{\spelltgt{One humanoid creature}}{\spellrng{\rngclose}}
        \end{spelltargetinginfo}
        \begin{spelleffects}
            \spelleffect The target is \immobilized.
            \spelldur \durshort
        \end{spelleffects}
    \end{spellcontent}
    \begin{spellfooter}
        \spellinfo{Enchantment [Compulsion, Mind]}{Arcane, Divine, Law, War}
        \miscastrandom
    \end{spellfooter}
\end{spellsection}

\begin{spellsection}[Mass]{Hold Person}[6]
    \begin{spellheader}
    \end{spellheader}
    \begin{spellcontent}
        \begin{spelltargetinginfo}
            \spelltwocol{\spelltgts{Up to five creatures}}{\spellrng{\rngmed}}
        \end{spelltargetinginfo}
        \begin{spelleffects}
            \spelleffect The target is \immobilized.
            \spelldur \durshort
        \end{spelleffects}
    \end{spellcontent}
    \begin{spellfooter}
        \spellinfo{Enchantment [Compulsion, Mind]}{Arcane, Divine, Law}
        \miscastexplode
    \end{spellfooter}
\end{spellsection}

\begin{spellsection}{Holy Aura}[8]
    \begin{spellheader}
    \end{spellheader}
    \begin{spellcontent}
        \begin{spelltargetinginfo}
            \spelltwocol{\spellrng{\rngclose}}{\spelltgts{Up to five creatures}}
        \end{spelltargetinginfo}
        \begin{spelleffects}
            \spelleffect The target gains spell resistance against lawful spells and spells cast by evil creatures.
            \spelldur \durshort \dismissable
        \end{spelleffects}
    \end{spellcontent}
    \begin{spellsubcontent}
        \begin{spelltargetinginfo}
            \spelltgr{Whenever an evil creature within 30 feet of the target makes a physical attack against it}
            \spelltgt{The attacking creature}
        \end{spelltargetinginfo}
        \begin{spelleffects}
            \begin{spellattack}{Spellpower vs. Mental}
                \spellsuccess \spelldamage{9}{divine}[d6]
            \end{spellattack}
        \end{spelleffects}
    \end{spellsubcontent}
    \begin{spellfooter}
        \spellinfo{Abjuration [Good, Retributive, Shielding]}{Divine, Good}
        \miscastexplode
    \end{spellfooter}
\end{spellsection}

\begin{spellsection}{Holy Smite}[3]
    \begin{spellheader}
    \end{spellheader}
    \begin{spellcontent}
        \begin{spelltargetinginfo}
            \spelltwocol{\spelltgt{One nongood creature}}{\spellrng{\rngmed}}
        \end{spelltargetinginfo}
        \begin{spelleffects}
            \begin{spellattack}{Spellpower vs. Mental}
                \spellsuccess \spelldamage{3}{divine}.
                \spellcritical As above, and the target is \dazed for 5 rounds.
                \spellfailure Half damage, and no additional effects.
            \end{spellattack}
        \end{spelleffects}
    \end{spellcontent}
    \begin{spellfooter}
        \spellinfo{Evocation [Good]}{Good}
        \miscastrandom
    \end{spellfooter}
\end{spellsection}

\begin{spellsection}{Holy Word}[6]
    \begin{spellheader}
    \end{spellheader}
    \begin{spellcontent}
        \begin{spelltargetinginfo}
            \spellburst{\arealarge radius centered on you}
            \spellcmp{Verbal only}
        \end{spelltargetinginfo}
        \begin{spelleffects}
            \begin{spellattack}{Spellpower vs. Mental}
                \spellsuccess \spelldamage{6}{divine}[d6].
                \spellcritical As above, and the target is \dazed for 5 rounds.
                \spellfailure Half damage, and no additional effects.
            \end{spellattack}
        \end{spelleffects}
    \end{spellcontent}
    \begin{spellfooter}
        \spellinfo{Evocation [Good]}{Good, Divine}
        \miscastexplode
    \end{spellfooter}
\end{spellsection}

\begin{spellsection}{Horrid Wilting}[9]
    \begin{spellheader}
        \spelldesc{You dessicate your foes from a great distance, shriveling their bodies.}
    \end{spellheader}
    \begin{spellcontent}
        \begin{spelltargetinginfo}
            \spelltwocol{\spellburst{\arealarge burst}}{\spellrng{\rnglong}}
            \spelltgts{All enemies in the area}
        \end{spelltargetinginfo}
        \begin{spelleffects}
            \begin{spellattack}{Spellpower vs. Fortitude}
                \spellspecial You gain a \plus5 bonus to attack against plants and creatures with the water subtype.
                \spellsuccess \spelldamage{9}{physical}[d6]
                \spellfailure Half damage.
            \end{spellattack}
        \end{spelleffects}
    \end{spellcontent}
    \begin{spellfooter}
        \spellinfo{Vivimancy [Flesh]}{Arcane, Water}
        \miscastyou
    \end{spellfooter}
\end{spellsection}

\begin{spellsection}{Hypnotic Pattern}[2]
    \begin{spellheader}
        \spelldesc{You create a twisting pattern of subtle, shifting colors that weaves through the air, fascinating creatures within it.}
    \end{spellheader}
    \begin{spellcontent}
        \begin{spelltargetinginfo}
            \spelltwocol{\spellzone{\arealarge radius}}{\spellrng{\rngmed}}
        \end{spelltargetinginfo}
        \begin{spelleffects}
            \spelleffect Lights appear in the area, illuminating the surroundings like a torch.
            \spelldur \durshort
        \end{spelleffects}
    \end{spellcontent}
    \begin{spellsubcontent}
        \begin{spelltargetinginfo}
            \spelltgts{All creatures in the area}
        \end{spelltargetinginfo}
        \begin{spelleffects}
            \begin{spellattack}{Spellpower vs. Mental}
                \spellsuccess The target is \fascinated by the lights.
            \end{spellattack}
        \end{spelleffects}
    \end{spellsubcontent}
    \begin{spellfooter}
        \spellinfo{Enchantment/Illusion [Compulsion, Figment, Light, Mind, Visual]}*{Arcane}
        \miscastyou
    \end{spellfooter}
\end{spellsection}

\begin{spellsection}[Greater]{Hypnotic Pattern}[6]
    \begin{spellheader}
        \spelldesc{You create a twisting pattern of subtle, shifting colors that weaves through the air, fascinating creatures within it and leading them astray.}
    \end{spellheader}
    \begin{spellcontent}
        \begin{spelltargetinginfo}
            \spelltwocol{\spellzone{\arealarge radius}}{\spellrng{\rngext}}
        \end{spelltargetinginfo}
        \begin{spelleffects}
            \spelleffect Lights appear in the area, illuminating the surroundings like a torch. By concentrating as a swift action, you can move the lights up to 50 feet.
            \spelldur \durmed
        \end{spelleffects}
    \end{spellcontent}
    \begin{spellsubcontent}
        \begin{spelltargetinginfo}
            \spelltgts{All creatures in the area}
        \end{spelltargetinginfo}
        \begin{spelleffects}
            \begin{spellattack}{Spellpower vs. Mental}
                \spellsuccess The target is \fascinated by the lights.
            \end{spellattack}
        \end{spelleffects}
    \end{spellsubcontent}
    \begin{spellfooter}
        \spellinfo{Enchantment/Illusion [Compulsion, Figment, Light, Mind, Visual]}*{Arcane}
        \miscastyou
    \end{spellfooter}
\end{spellsection}

\pdfbookmark[2]{I}{SpellDescriptionsI}

\begin{spellsection}{Ice Storm}[4]
    \begin{spellheader}
        \spelldesc{You conjure magical hailstones that pound down, smashing and chilling creatures in their path.}
    \end{spellheader}
    \begin{spellcontent}
        \begin{spelltargetinginfo}
            \spelltwocol{\spellburst{\areasmall radius cylinder, 20 ft. high}}{\spellrng{\rngmed}}
        \end{spelltargetinginfo}
        \begin{spelleffects}
            \spelleffect The ground in the area is covered in ice for 5 rounds, making it slippery. A DC 15 Balance check is usually required to move on icy surfaces. See \pcref{Balance}, for more details.
        \end{spelleffects}
    \end{spellcontent}
    \begin{spellsubcontent}
        \begin{spelltargetinginfo}
            \spelltgts{Everything in the area}
        \end{spelltargetinginfo}
        \begin{spelleffects}
            \spelleffect \spelldamage{4}{cold and bludgeoning}[d4]
        \end{spelleffects}
    \end{spellsubcontent}
    \begin{spellfooter}
        \spellinfo{Conjuration/Evocation [Cold, Creation, Destructive]}*{Arcane, Destruction, Nature, Water}
        \spellnotes \destructivespellnotes
    % SR only applies to cold?
        \miscastyou
    \end{spellfooter}
\end{spellsection}

\begin{spellsection}{Implosion}[9]
    \begin{spellheader}
        \spelldesc{You create a destructive responance in your foe's body that destroys it from the inside out.}
    \end{spellheader}
    \begin{spellcontent}
        \begin{spelltargetinginfo}
            \spelltgr{At the end of every round}
            \spelltwocol{\spelltgt{One creature}}{\spellrng{\rngclose}}
            \spellspecial You cannot target the same creature more than once per casting of this spell.
        \end{spelltargetinginfo}
        \begin{spelleffects}
            \begin{spellattack}{Spellpower vs. Fortitude}
                \spellsuccess \spelldamage{9}{life}. In addition, the target is \staggered for 5 rounds.
                \spellcritical The target dies.
                \spellfailure Half damage, and no additional effects.
            \end{spellattack}
        \end{spelleffects}
    \end{spellcontent}
    \begin{spellfooter}
        \spellinfo{Evocation/Transmutation [Alteration]}{Destruction, Divine}
        \spellnotes This spell has no effect on creatures in \spell{gaseous form} or on incorporeal creatures.
        \miscastexplode
    \end{spellfooter}
\end{spellsection}

\begin{spellsection}{Imprisonment}[8]
    \begin{spellheader}
        \spelldesc{You teleport your foe deep beneath the earth, leaving it in stasis forever.}
    \end{spellheader}
    \begin{spellcontent}
        \begin{spelltargetinginfo}
            \spelltwocol{\spelltgt{One creature touching the ground}}{\spellrng{\rngmed}}
        \end{spelltargetinginfo}
        \begin{spelleffects}
            \begin{spellattack}{Spellpower vs. Mental}
                \spellsuccess \spelldamage{8}{physical}. In addition, the target is \slowed for 5 rounds.
                \spellcritical The target becomes permanently entombed in a state of suspended animation (as the \spell{temporal stasis} spell) in a small sphere far beneath the surface of the earth. It remains there until an \spell{emancipation} spell is cast at the location where the imprisonment took place.
                \spellfailure Half damage, and the target moves at half speed for 5 rounds.
            \end{spellattack}
        \end{spelleffects}
    \end{spellcontent}
    \begin{spellfooter}
        \spellinfo{Conjuration/Transmutation [Teleportation, Temporal]}{Arcane, Earth, Law}
        \spellnotes If the target becomes imprisoned beneath the earth, it is very difficult to find. Magical search by a crystal ball, a \spell{locate creature} spell, or some other similar divination does not reveal the fact that a creature is imprisoned, but \spell{discern location} does. A \spell{wish} or \spell{miracle} spell will not free the recipient, but will reveal where it is entombed.
        \miscastrandom
    \end{spellfooter}
\end{spellsection}

\begin{spellsection}{Inertial Shield}[2]
    \begin{spellheader}
        \spelldesc{You create a barrier around your ally that resists physical intrusion.}
    \end{spellheader}
    \begin{spellcontent}
        \begin{spelltargetinginfo}
            \spelltwocol{\spelltgt{One creature}}{\spellrng{\rngtouch}}
        \end{spelltargetinginfo}
        \begin{spelleffects}
            \spelleffect The target gains damage reduction against physical damage equal to your spellpower. Force damage ignores this damage reduction and negates it for 1 round.
            \spelldur \durshort
        \end{spelleffects}
    \end{spellcontent}
    \begin{spellfooter}
        \spellinfo{Abjuration [Shielding]}{Arcane}
        \miscastexplode
    \end{spellfooter}
\end{spellsection}

\begin{spellsection}{Inflict Critical Wounds}[4]
    \begin{spellheader}
    \end{spellheader}
    \begin{spellcontent}
        \begin{spelltargetinginfo}
            \spelltwocol{\spelltgt{One creature}}{\spellrng{\rngclose}}
        \end{spelltargetinginfo}
        \begin{spelleffects}
            \spelleffect If the target is undead, it is healed for \spelldamage{4}{}.
            \begin{spellattacktriggered}{If the target is living, make a Spellpower vs. Fortitude attack.}
                \spellsuccess \spelldamage{4}{negative}. For every 2 points of damage dealt in excess of the target's hit points, it takes 1 point of critical damage.
                \spellfailure Half damage.
            \end{spellattacktriggered}
        \end{spelleffects}
    \end{spellcontent}
    \begin{spellfooter}
        \spellinfo{Vivimancy [Negative]}{Arcane, Divine}
        \miscastrandom
    \end{spellfooter}
\end{spellsection}

\begin{spellsection}[Mass]{Inflict Critical Wounds}[8]
    \begin{spellheader}
    \end{spellheader}
    \begin{spellcontent}
        \begin{spelltargetinginfo}
            \spelltwocol{\spelltgts{Up to five creatures}}{\spellrng{\rngmed}}
        \end{spelltargetinginfo}
        \begin{spelleffects}
            \spellspecial This spell functions like \spell{inflict critical wounds}, except that it heals or inflicts \spelldamage{8}{}[d6].
        \end{spelleffects}
    \end{spellcontent}
    \begin{spellfooter}
        \spellinfo{Vivimancy [Negative]}{Arcane, Divine}
        \miscastexplode
    \end{spellfooter}
\end{spellsection}

\begin{spellsection}{Inflict Light Wounds}[1]
    \begin{spellheader}
    \end{spellheader}
    \begin{spellcontent}
        \begin{spelltargetinginfo}
            \spelltwocol{\spelltgt{One creature}}{\spellrng{\rngclose}}
        \end{spelltargetinginfo}
        \begin{spelleffects}
            \spelleffect If the target is undead, it is healed for \spelldamage{1}{}.
            \begin{spellattacktriggered}{If the target is living, make a Spellpower vs. Fortitude attack.}
                \spellsuccess \spelldamage{1}{negative}.
                \spellfailure Half damage.
            \end{spellattacktriggered}
        \end{spelleffects}
    \end{spellcontent}
    \begin{spellfooter}
        \spellinfo{Vivimancy [Negative]}{Arcane, Divine}
        \miscastrandom
    \end{spellfooter}
\end{spellsection}

\begin{spellsection}[Mass]{Inflict Light Wounds}[5]
    \begin{spellheader}
    \end{spellheader}
    \begin{spellcontent}
        \begin{spelltargetinginfo}
            \spelltwocol{\spelltgts{Up to five creatures}}{\spellrng{\rngmed}}
        \end{spelltargetinginfo}
        \begin{spelleffects}
            \spellspecial This spell functions like \spell{inflict light wounds}, except that it heals or inflicts \spelldamage{5}{}[d6].
        \end{spelleffects}
    \end{spellcontent}
    \begin{spellfooter}
        \spellinfo{Vivimancy [Negative]}{Arcane, Divine}
        \miscastexplode
    \end{spellfooter}
\end{spellsection}

\begin{spellsection}{Inflict Moderate Wounds}[2]
    \begin{spellheader}
    \end{spellheader}
    \begin{spellcontent}
        \begin{spelltargetinginfo}
            \spellquicktargeting{One creature}{\rngclose}
        \end{spelltargetinginfo}
        \begin{spelleffects}
            \spelleffect If the target is undead, it is healed for \spelldamage{2}{}.
            \begin{spellattacktriggered}{If the target is living, make a Spellpower vs. Fortitude attack.}
                \spellsuccess \spelldamage{4}{negative}. For every 10 points of damage dealt in excess of the target's hit points, it takes 1 point of critical damage.
                \spellfailure Half damage.
            \end{spellattacktriggered}
        \end{spelleffects}
    \end{spellcontent}
    \begin{spellfooter}
        \spellinfo{Vivimancy [Negative]}{Arcane, Divine}
        \miscastrandom
    \end{spellfooter}
\end{spellsection}

\begin{spellsection}[Mass]{Inflict Moderate Wounds}[6]
    \begin{spellheader}
    \end{spellheader}
    \begin{spellcontent}
        \begin{spelltargetinginfo}
            \spelltwocol{\spelltgts{Up to five creatures}}{\spellrng{\rngmed}}
        \end{spelltargetinginfo}
        \begin{spelleffects}
            \spellspecial This spell functions like \spell{inflict moderate wounds}, except that it heals or inflicts \spelldamage{6}{}[d6].
        \end{spelleffects}
    \end{spellcontent}
    \begin{spellfooter}
        \spellinfo{Vivimancy [Negative]}{Arcane, Divine}
        \miscastexplode
    \end{spellfooter}
\end{spellsection}

\begin{spellsection}{Inflict Serious Wounds}[3]
    \begin{spellheader}
    \end{spellheader}
    \begin{spellcontent}
        \begin{spelltargetinginfo}
            \spelltwocol{\spelltgt{One creature}}{\spellrng{\rngclose}}
        \end{spelltargetinginfo}
        \begin{spelleffects}
            \spelleffect If the target is undead, it is healed for \spelldamage{3}{}.
            \begin{spellattacktriggered}{If the target is living, make a Spellpower vs. Fortitude attack.}
                \spellsuccess \spelldamage{4}{negative}. For every 5 points of damage dealt in excess of the target's hit points, it takes 1 point of critical damage.
                \spellfailure Half damage.
            \end{spellattacktriggered}
        \end{spelleffects}
    \end{spellcontent}
    \begin{spellfooter}
        \spellinfo{Vivimancy [Negative]}{Arcane, Divine}
        \miscastrandom
    \end{spellfooter}
\end{spellsection}

\begin{spellsection}[Mass]{Inflict Serious Wounds}[7]
    \begin{spellheader}
    \end{spellheader}
    \begin{spellcontent}
        \begin{spelltargetinginfo}
            \spelltwocol{\spelltgts{Up to five creatures}}{\spellrng{\rngmed}}
        \end{spelltargetinginfo}
        \begin{spelleffects}
            \spellspecial This spell functions like \spell{inflict serious wounds}, except that it heals or inflicts \spelldamage{7}{}[d6].
        \end{spelleffects}
    \end{spellcontent}
    \begin{spellfooter}
        \spellinfo{Vivimancy [Negative]}{Arcane, Divine}
        \miscastexplode
    \end{spellfooter}
\end{spellsection}

\begin{spellsection}{Invisibility}[3]
    \begin{spellheader}
    \end{spellheader}
    \begin{spellcontent}
        \begin{spelltargetinginfo}
            \spelltwocol{\spelltgt{One creature or object (Large or smaller)}}{\spellrng{\rngclose}}
        \end{spelltargetinginfo}
        \begin{spelleffects}
            \spelleffect The target and its equipment become invisible. An invisible creature cannot be seen, even by darkvision. Invisible creatures can be detected with the Awareness skill (see \pcref{Awareness}).

            If the target attacks any creature, such as by casting any spell that affects an unwilling creature, it becomes visible.
            \spelldur \durshort \dismissable
        \end{spelleffects}
    \end{spellcontent}
    \begin{spellfooter}
        \spellinfo{Illusion [Glamer]}{Arcane, Trickery}
        \spellnotes This spell can be made permanent (on objects only) with a \spell{permanency} ritual.
        \miscastrandom
    \end{spellfooter}
\end{spellsection}

\begin{spellsection}[Greater]{Invisibility}[6]
    \begin{spellheader}
    \end{spellheader}
    \begin{spellcontent}
        \begin{spelltargetinginfo}
            \spelltwocol{\spelltgt{One creature or object (Large or smaller)}}{\spellrng{\rngclose}}
        \end{spelltargetinginfo}
        \begin{spelleffects}
            \spelleffect The target becomes invisible, as \spell{invisibility}. At the end of every round, if the target did not attack a creature that round, it becomes invisible again.
            \spelldur \durshort \dismissable
        \end{spelleffects}
    \end{spellcontent}
    \begin{spellfooter}
        \spellinfo{Illusion [Glamer]}{Illusion}
        \miscastrandom
    \end{spellfooter}
\end{spellsection}

\begin{spellsection}[Mass]{Invisibility}[7]
    \begin{spellheader}
    \end{spellheader}
    \begin{spellcontent}
        \begin{spelltargetinginfo}
            \spelltwocol{\spelltgts{Up to five creatures or objects (Large or smaller)}}{\spellrng{\rngmed}}
        \end{spelltargetinginfo}
        \begin{spelleffects}
            \spelleffect The target becomes invisible, as \spell{invisibility}.
            \spelldur \durshort \dismissable
        \end{spelleffects}
    \end{spellcontent}
    \begin{spellfooter}
        \spellinfo{Illusion [Glamer]}{Arcane, Trickery}
        \miscastexplode
    \end{spellfooter}
\end{spellsection}

\begin{spellsection}{Iron Body}[8]
    \begin{spellheader}
    \end{spellheader}
    \begin{spellcontent}
        \begin{spelltargetinginfo}
            \spelltgt{You}
        \end{spelltargetinginfo}
        \begin{spelleffects}
            \spelleffect This spell transforms your body into living iron, which grants you several powerful resistances and abilities.
            \par You gain damage reduction against physical damage equal to your spellpower. Adamantine weapons ignore this damage reduction and negate it for 1 round.
            \par You are immune to blindness, critical hits, attribute damage, deafness, disease, drowning, electricity, poison, stunning, and all spells or attacks that affect your physiology or respiration, because you have no physiology or respiration while this spell is in effect.
            \par Your move at half speed, and take a \minus8 armor check penalty. You cannot drink (and thus can't use potions) or play wind instruments.
            \par Your unarmed attacks deal damage equal to a warhammer sized for you (1d6 for Small characters or 1d8 for Medium characters), and you are considered armed when making unarmed attacks.
            \par Your weight increases by a factor of ten, causing you to sink in water like a stone. However, you could survive the crushing pressure and lack of air at the bottom of the ocean -- at least until the spell duration expires.
            \spelldur \durshort \dismissable
        \end{spelleffects}
    \end{spellcontent}
    \begin{spellfooter}
        \spellinfo{Transmutation [Alteration, Augment]}{Arcane, Earth, Strength}
        \miscastexplode
    \end{spellfooter}
\end{spellsection}

\begin{spellsection}{Irresistible Dance}[9]
    \begin{spellheader}
        \spelldesc{You fill your enemy with an overpowering urge to dance and caper in place. Against its will, it begins doing so, complete with foot shuffling and tapping.}
    \end{spellheader}
    \begin{spellcontent}
        \begin{spelltargetinginfo}
            \spelltwocol{\spelltgt{One creature}}{\spellrng{\rngclose}}
        \end{spelltargetinginfo}
        \begin{spelleffects}
            \begin{spellattack}{Spellpower vs. Mental}
                \spellsuccess The target must spend a standard action each round to do nothing but dance.
                \spellfailure The target must spend a move action each round to dance.
            \end{spellattack}
            \spelldur 5 rounds
        \end{spelleffects}
    \end{spellcontent}
    \begin{spellfooter}
        \spellinfo{Enchantment [Compulsion, Mind]}{Arcane, Chaos}
        \miscastrandom
    \end{spellfooter}
\end{spellsection}

\pdfbookmark[2]{J-L}{SpellDescriptionsJ-L}

\begin{spellsection}{Knock}[2]
    \begin{spellheader}
    \end{spellheader}
    \begin{spellcontent}
        \begin{spelltargetinginfo}
            \spelltwocol{\spelltgt{One object (Medium or smaller)}}{\spellrng{\rngclose}}
        \end{spelltargetinginfo}
        \begin{spelleffects}
            \spelleffect This spell telekinetically opens stuck, barred, locked, held, or arcane locked objects. If the target object is stuck or held, you can immediately make an Strength check to break it open, using your spellpower instead of your Strength. Others can aid you on this check as normal.

            In addition, if the target object is locked, you can immediately make a Disable Device check to open the lock as if you had rolled a 20 on the check. You get a bonus on the Disable Device check equal to half your spellpower.
        \end{spelleffects}
    \end{spellcontent}
    \begin{spellfooter}
        \spellinfo{Evocation [Telekinesis]}{Arcane}
        \spellnotes If this spell is cast on an \spellindirect{arcane lock}{arcane locked} door, make a spellpower check against a DC of 11 \add the spellpower of the \spell{arcane lock}. If you succeed, the \spell{arcane lock} is suppressed for 10 minutes. If you fail, you may still bypass the door with the checks above, if possible.
        \miscastrandom
    \end{spellfooter}
\end{spellsection}

\begin{spellsection}{Lesser (Spell Name)}
    \par Any spell whose name begins with lesser is alphabetized in this chapter according to the second word of the spell name. Thus, the description of a lesser spell appears near the description of the spell on which it is based. Spell chains that have lesser spells in them include those based on the spells cone of cold, dispel magic, moment of prescience, precognition, and spelltheft.
\end{spellsection}

\begin{spellsection}{Levitate}[3]
    \begin{spellheader}
    \end{spellheader}
    \begin{spellcontent}
        \begin{spelltargetinginfo}
            \spellrng{\rngclose}
            \spelltgt{One unattended object or willing creature (Large or smaller)}
        \end{spelltargetinginfo}
        \begin{spelleffects}
            \spelleffect As a swift action, you can mentally direct the target to move up or down as much as 30 feet each round. You cannot move the recipient horizontally, but the recipient could clamber along the face of a cliff, for example, or push against a ceiling to move laterally (generally at half its land speed).
            \spelldur \durshort \dismissable
        \end{spelleffects}
    \end{spellcontent}
    \begin{spellfooter}
        \spellinfo{Evocation [Telekinesis]}{Evocation}
        \miscastrandom
    \end{spellfooter}
\end{spellsection}

\begin{spellsection}{Lifeseeking Missile}[3]
    \begin{spellheader}
    \end{spellheader}
    \begin{spellcontent}
        \begin{spelltargetinginfo}
            \spelltwocol{\spelltgts{Any number of creatures}}{\spellrng{\rngmed}}
        \end{spelltargetinginfo}
        \begin{spelleffects}
            \spellspecial You create one missile per two spellpower. Each missile can deal d10 force damage to a single creature.

            Any missiles you do not explicitly target will automatically strike a living creature within the area. The missiles are able to unerringly strike creatures you cannot see or are not aware of, including invisible or concealed creatures. You can direct the missiles to avoid specific targets, allowing you to strike a hidden foe among your allies.
            \spelleffect The target is struck by as many missiles as you choose.
        \end{spelleffects}
    \end{spellcontent}
    \begin{spellfooter}
        \spellinfo{Evocation/Vivimancy [Force, Life]}{Arcane}
        \miscastexplode
    \end{spellfooter}
\end{spellsection}

\begin{spellsection}{Lightning Bolt}[3]
    \begin{spellheader}
    \end{spellheader}
    \begin{spellcontent}
        \begin{spelltargetinginfo}
            \spellburst{\arealarge line, 10 ft. wide}
            \spelltgts{Everything in the area}
        \end{spelltargetinginfo}
        \begin{spelleffects}
            \begin{spellattack}{Spellpower vs. Reflex}
                \spellsuccess \spelldamage{3}{electricity}[d6]
                \spellfailure Half damage.
            \end{spellattack}
        \end{spelleffects}
    \end{spellcontent}
    \begin{spellfooter}
        \spellinfo{Evocation [Destructive, Electricity]}{Arcane, Destruction, Nature}
        \spellnotes \destructivespellnotes
        \miscastexplode
    \end{spellfooter}
\end{spellsection}

\begin{spellsection}{Longstrider}[1]
    \begin{spellheader}
    \end{spellheader}
    \begin{spellcontent}
        \begin{spelltargetinginfo}
            \spelltgt{You}
        \end{spelltargetinginfo}
        \begin{spelleffects}
            \spelleffect You gain a \plus10 foot bonus to your speed in all your movement modes.
            \spelldur \durlong \dismissable
        \end{spelleffects}
    \end{spellcontent}
    \begin{spellfooter}
        \spellinfo{Transmutation [Augment]}{Nature, Travel}
        \miscastexplode
    \end{spellfooter}
\end{spellsection}

\pdfbookmark[2]{M}{SpellDescriptionsM}

\begin{spellsection}{Mage Armor}[1]
    \begin{spellheader}
        \spelldesc{You create an invisible but tangible field of force that shields you from attacks.}
    \end{spellheader}
    \begin{spellcontent}
        \begin{spelltargetinginfo}
            \spelltgt{You}
            \spellspecial When you cast this spell, you choose whether to create body armor or a shield.
        \end{spelltargetinginfo}
        \begin{spelleffects}
            \spelleffect You gain invisible body armor or a shield of the chosen kind, made of force. Body armor grants a \plus4 defense bonus, while a shield grants a \plus2 defense bonus.
            \par Unlike mundane armor, this armor has no armor check penalty, arcane spell failure chance, or encumbrance. If you create a shield, it floats in front of you, and does not need to be wielded actively to grant its bonus.
            \spelldur \durlong
        \end{spelleffects}
    \end{spellcontent}
    \begin{spellfooter}
        \spellinfo{Abjuration [Force]}{Arcane}
        \spellnotes If you cast this spell twice, you can gain both body armor and a shield. The armor created by this spell is treated as a separate piece or armor from any other armor the creature is wearing, so it does not stack with any existing bonuses. Since this armor is made of force, incorporeal creatures can't bypass it the way they do normal armor.
        \miscastexplode
    \end{spellfooter}
\end{spellsection}

\begin{spellsection}{Mage Hand}[1]
    \begin{spellheader}
    \end{spellheader}
    \begin{spellcontent}
        \begin{spelltargetinginfo}
            \spellrng{\rngclose}
        \end{spelltargetinginfo}
        \begin{spelleffects}
            \spelleffect By concentrating as a swift action, you can move an object within range up to 10 feet per round.

            Your effective Strength is \minus4, allowing you to hold and move objects up to 25 pounds. You cannot perform tasks requiring fine motor skills (with a DC higher than 0).
            \spelldur \durshort
        \end{spelleffects}
    \end{spellcontent}
    \begin{spellfooter}
        \spellinfo{Evocation [Telekinesis]}{Evocation}
        \miscastexplode
    \end{spellfooter}
\end{spellsection}

\begin{spellsection}{Magic Missile}[1]
    \begin{spellheader}
    \end{spellheader}
    \begin{spellcontent}
        \begin{spelltargetinginfo}
            \spelltwocol{\spelltgts{See text}}{\spellrng{\rngclose}}
        \end{spelltargetinginfo}
        \begin{spelleffects}
            \spelleffect You strike one target creature with a missile that deals d8 force damage. At 6th level, and every 4 spellpower thereafter, you create an additional missile. You can direct each missile to strike the same or different targets.
        \end{spelleffects}
    \end{spellcontent}
    \begin{spellfooter}
        \spellinfo{Evocation [Force]}{Arcane}
        \spellnotes \forcespellnotes
        \miscastrandom
    \end{spellfooter}
\end{spellsection}

\begin{spellsection}{Major Image}[4]
    \begin{spellheader}
    \end{spellheader}
    \begin{spellcontent}
        \begin{spelltargetinginfo}
            \spelltwocol{\spellzone{\arealarge radius}}{\spellrng{\rngmed}}
        \end{spelltargetinginfo}
        \begin{spelleffects}
            \spelleffect A figment of your design appears within the area, as \spell{silent image}, except that sound, smell, and thermal elements are included.
            \spelldur \durshort
        \end{spelleffects}
    \end{spellcontent}
    \begin{spellfooter}
        \spellinfo{Illusion [Figment, Unreal]}{Illusion}
        \spellnotes Creatures can identify the illusion, as \spell{silent image}.
        \miscastexplode
    \end{spellfooter}
\end{spellsection}

\begin{spellsection}[Lesser]{Manipulate Probability}[2]
    \begin{spellheader}
    \end{spellheader}
    \begin{spellcontent}
        \begin{spelltargetinginfo}
            \spelltgt{You}
        \end{spelltargetinginfo}
        \begin{spelleffects}
            \spelleffect When you cast this spell, you roll a d20 twice. Store the results in order. At spellpower 5, and every 5 spellpower thereafter, you roll an additional die.

            Each time you would roll a d20, you instead use a die result you rolled. The die results are used in the same order you rolled them. When all the die results have been used, the spell is expended.
            \spelldur \durshort or until expended
        \end{spelleffects}
    \end{spellcontent}
    \begin{spellfooter}
        \spellinfo{Divination}{Divination, Knowledge}
        \spellnotes Any die results unused when the spell duration expires are discarded.
        \miscastexplode
    \end{spellfooter}
\end{spellsection}

\begin{spellsection}{Manipulate Probability}[5]
    \begin{spellheader}
    \end{spellheader}
    \begin{spellcontent}
        \begin{spelltargetinginfo}
            \spelltgt{You}
        \end{spelltargetinginfo}
        \begin{spelleffects}
            \spelleffect When you cast this spell, you roll a d20 four times. Store the results in order. At 15th spellpower, and every 5 spellpower thereafter, you roll an additional die.

            Each time you would roll a d20, you may instead use a die result you rolled. The die results can be used in any order, and you may choose to roll instead of using a stored die result. When all the die results have been used, the spell is expended.
            \spelldur \durshort or until expended
        \end{spelleffects}
    \end{spellcontent}
    \begin{spellfooter}
        \spellinfo{Divination}{Divination, Knowledge}
        \spellnotes Any die results unused when the spell duration expires are discarded.
        \miscastexplode
    \end{spellfooter}
\end{spellsection}

\begin{spellsection}{Mass (Spell Name)}
    \par Any spell whose name begins with mass is alphabetized in this chapter according to the second word of the spell name. Thus, the description of a mass spell appears near the description of the spell on which it is based. Spell chains that have mass spells in them include those based on the spells charm monster, cure critical wounds, cure light wounds, cure moderate wounds, cure serious wounds, enlarge person, heal, hold monster, hold person, inflict critical wounds, inflict light wounds, inflict moderate wounds, inflict serious wounds, invisibility, reduce person, suggestion, totemic mind, and totemic power.
\end{spellsection}

\begin{spellsection}{Maze}[8]
    \begin{spellheader}
    \end{spellheader}
    \begin{spellcontent}
        \begin{spelltargetinginfo}
            \spelltwocol{\spelltgt{One creature}}{\spellrng{\rngclose}}
        \end{spelltargetinginfo}
        \begin{spelleffects}
            \begin{spellattack}{Spellpower vs. Mental}
                \spellsuccess The target is teleported into an extradimensional labyrinth of force planes. Each round, as a full-round action, it may attempt a DC 20 Intelligence check to escape the labyrinth. If the target doesn't escape, the maze disappears after 5 minutes, forcing the target back to the location where it was originally banished.
                \spellfailure As above, but the DC of the Intelligence check to escape is 10.
            \end{spellattack}
        \end{spelleffects}
    \end{spellcontent}
    \begin{spellfooter}
        \spellinfo{Conjuration [Planar, Teleportation]}*{Conjuration, Trickery}
        \spellnotes Spells and abilities that move a creature within a plane, such as \spell{teleport} and \spell{dimension door}, do not help a creature escape a \spell{maze} spell, although a \spell{plane shift} spell allows it to exit to whatever plane is designated in that spell. Minotaurs can escape the spell automatically.

        When leaving the maze, the target reappears where it had been when the maze spell was cast. If this location is filled with a solid object, the target appears in the nearest open space.

        \norepeatspellnotes
        \miscastrandom
    \end{spellfooter}
\end{spellsection}

\begin{spellsection}{Meld into Plants}[3]
    \begin{spellheader}
    \end{spellheader}
    \begin{spellcontent}
        \begin{spelltargetinginfo}
            \spelltgt{One plant of your size or larger}
        \end{spelltargetinginfo}
        \begin{spelleffects}
            \spellspecial This spell functions like \spell{meld into stone}, except that you meld into a plant instead of stone.
        \end{spelleffects}
    \end{spellcontent}
    \begin{spellfooter}
        \spellinfo{Transmutation [Alteration]}{Nature, Wild}
        \miscastexplode
    \end{spellfooter}
\end{spellsection}

\begin{spellsection}{Meld into Stone}[2]
    \begin{spellheader}
    \end{spellheader}
    \begin{spellcontent}
        \begin{spelltargetinginfo}
            \spelltgt{One solid stone object of your size or larger}
        \end{spelltargetinginfo}
        \begin{spelleffects}
            \spelleffect You and your equipment meld into the target block of stone. While in the stone, you can move, breathe, and speak as if the stone was air, but you cannot see or hear out of the stone unless you move your head out of the stone. In addition, you are unable to move farther than 5 feet from your original entrance point.

            Minor physical damage to the stone does not harm you, but if its size is reduced to be smaller than yours, or if it is otherwise altered to be unsuitable for the spell (such as by \spell{transmute flesh and stone}), you are expelled and take 5d6 points of damage.

            If you leave the stone completely, the spell immediately ends.

            \spelldur \durlong
        \end{spelleffects}
    \end{spellcontent}
    \begin{spellfooter}
        \spellinfo{Transmutation [Alteration, Earth]}{Earth, Nature}
        \miscastexplode
    \end{spellfooter}
\end{spellsection}

\begin{spellsection}{Message}[1]
    \begin{spellheader}
    \end{spellheader}
    \begin{spellcontent}
        \begin{spelltargetinginfo}
            \spelltwocol{\spelltgts{Up to five creatures}}{\spellrng{\rngmed}}
            \spellcmp{Somatic only}
        \end{spelltargetinginfo}
        \begin{spelleffects}
            \spelleffect Whenever you whisper, you may cause any or all of the targets to hear the message as if you were whispering in their ears.
            \spelldur \durlong
        \end{spelleffects}
    \end{spellcontent}
    \begin{spellfooter}
        \spellinfo{Divination}{Arcane}
        \spellnotes This is not telepathic communication, and observers can still read your lips. Very close observers may also hear the message.
        \miscastexplode
    \end{spellfooter}
\end{spellsection}

\begin{spellsection}{Meteor Swarm}[9]
    \begin{spellheader}
        \spelldesc{You call a swarm of meteors that streak down from the heavens, leaving a fiery trail behind them. The meteors crash into your foes, driving flying creatures to the ground and knocking creatures off their feet.}
    \end{spellheader}
    \begin{spellcontent}
        \begin{spelltargetinginfo}
            \spellrng{\rngmed}
            \spellburst{\arealarge radius cylinder, 100 ft. high}
            \spelltgts{Everything in the area}
        \end{spelltargetinginfo}
        \begin{spelleffects}
            \begin{spellattack}{Spellpower vs. Reflex}
                \spellsuccess \spelldamage{9}{fire}[d6].

                If the target is on the ground, it falls prone. If the target is in the air, and is Gargantuan or smaller, it is driven to the ground. It takes falling damage as appropriate for the distance descended.
                \spellfailure Half damage, and no additional effects.
            \end{spellattack}
        \end{spelleffects}
    \end{spellcontent}
    \begin{spellfooter}
        \spellinfo{Evocation [Destructive, Fire]}{Arcane, Destruction, Fire}
        \spellnotes \firespellnotes

        \destructivespellnotes
        \miscastyou
    \end{spellfooter}
\end{spellsection}

\begin{spellsection}{Minor Image}[3]
    \begin{spellheader}
    \end{spellheader}
    \begin{spellcontent}
        \begin{spelltargetinginfo}
            \spelltwocol{\spellzone{\areamed radius}}{\spellrng{\rngmed}}
        \end{spelltargetinginfo}
        \begin{spelleffects}
            \spelleffect A figment of your design appears within the area, as \spell{silent image}, except that sound elements are included.
            \spelldur \durshort
        \end{spelleffects}
    \end{spellcontent}
    \begin{spellfooter}
        \spellinfo{Illusion [Figment, Unreal]}{Illusion}
        \spellnotes Creatures can identify the illusion, as \spell{silent image}.
        \miscastexplode
    \end{spellfooter}
\end{spellsection}

\begin{spellsection}{Mirror Image}[2]
    \begin{spellheader}
        \spelldesc{You create illusory duplicates of yourself that mirror your every move, making it difficult for enemies to know which image to attack.}
    \end{spellheader}
    \begin{spellcontent}
        \begin{spelltargetinginfo}
            \spelltgt{You}
        \end{spelltargetinginfo}
        \begin{spelleffects}
            \spelleffect You gain one image per two spellpower. As long as you have images remaining, targeted attacks against you have a 50\% miss chance. If you run out of images, the spell is expended.
            \spelldur \durshort or until expended \dismissable
        \end{spelleffects}
    \end{spellcontent}
    \begin{spellfooter}
        \spellinfo{Illusion [Figment, Visual]}{Arcane}
        \spellnotes This spell offers no defense against creatures unable to see you or your images.
        \miscastexplode
    \end{spellfooter}
\end{spellsection}

\begin{spellsection}[Greater]{Mirror Image}[5]
    \begin{spellheader}
    \end{spellheader}
    \begin{spellcontent}
        \begin{spelltargetinginfo}
            \spelltgt{You}
        \end{spelltargetinginfo}
        \begin{spelleffects}
            \spelleffect You gain illusory duplicates, as \spell{mirror image}, except that the spell is not expended when you run out of images. At the end of each round, you gain two additional images, up to the number of images created when the spell was first cast.
            \spelldur \durshort \dismissable
        \end{spelleffects}
    \end{spellcontent}
    \begin{spellfooter}
        \spellinfo{Illusion [Figment, Visual]}{Arcane}
        \spellnotes As \spell{mirror image}.
        \miscastexplode
    \end{spellfooter}
\end{spellsection}

\begin{spellsection}{Mislead}[6]
    \begin{spellheader}
    \end{spellheader}
    \begin{spellcontent}
        \begin{spelltargetinginfo}
            \spelltgt{You}
        \end{spelltargetinginfo}
        \begin{spelleffects}
            \spelleffect You become invisible, as \spell{invisibility}. At the same time, an illusory double of you appears, as \spell{major image}.

            You can control the image of yourself as you would control any other figment with \spell{major image}. If not directed, it remains stationary.
            \spelldur \durshort \dismissable
        \end{spelleffects}
    \end{spellcontent}
    \begin{spellfooter}
        \spellinfo{Illusion [Figment, Glamer, Unreal]}{Arcane, Trickery}
        \miscastexplode
    \end{spellfooter}
\end{spellsection}

\begin{spellsection}{Missile Storm}[4]
    \begin{spellheader}
        \spelldesc{You unleash a swarm of missiles which seek out and destroy your foes.}
    \end{spellheader}
    \begin{spellcontent}
        \begin{spelltargetinginfo}
            \spelltwocol{\spelltgts{Up to five creatures}}{\spellrng{\rngmed}}
        \end{spelltargetinginfo}
        \begin{spelleffects}
            \spelleffect \spelldamage{4}{force}[d4]
        \end{spelleffects}
    \end{spellcontent}
    \begin{spellfooter}
        \spellinfo{Evocation [Force]}{Arcane}
        \spellnotes \forcespellnotes
        \miscastexplode
    \end{spellfooter}
\end{spellsection}

\begin{spellsection}[Greater]{Missile Storm}[8]
    \begin{spellheader}
        \spelldesc{You unleash an immense swarm of missiles which seek out and destroy your foes.}
    \end{spellheader}
    \begin{spellcontent}
        \begin{spelltargetinginfo}
            \spellburst{100 ft. radius centered on you}
            \spelltgts{All enemies in the area}
        \end{spelltargetinginfo}
        \begin{spelleffects}
            \spelleffect \spelldamage{8}{force}[d4]
        \end{spelleffects}
    \end{spellcontent}
    \begin{spellfooter}
        \spellinfo{Evocation [Force]}{Arcane}
        \spellnotes \forcespellnotes
        \miscastexplode
    \end{spellfooter}
\end{spellsection}

\begin{spellsection}{Moment of Prescience}[4]
    \begin{spellheader}
        \spelldesc{You extend your mind a fraction of a second into the future, allowing you to succeed where you would have failed.}
    \end{spellheader}
    \begin{spellcontent}
        \begin{spelltargetinginfo}
            \spelltgt{You}
            \spelltime{Immediate action}
            \spellspecial You can cast this spell any time you could use a legend point, even while casting another spell.
        \end{spelltargetinginfo}
        \begin{spelleffects}
            \spelleffect You gain a legend point.
            \spelldur Until the end of the round
        \end{spelleffects}
    \end{spellcontent}
    \begin{spellfooter}
        \spellinfo{Divination}{Divination, Knowledge}
        \spellnotes After using this spell, you cannot cast any \spell{moment of prescience} spell for 1 hour.
        \miscastexplode
    \end{spellfooter}
\end{spellsection}

\begin{spellsection}[Greater]{Moment of Prescience}[8]
    \begin{spellheader}
    \end{spellheader}
    \begin{spellcontent}
        \begin{spelltargetinginfo}
            \spelltgt{You}
            \spelltime{Immediate action}
            \spellspecial You can cast this spell any time you could use a legend point, even while casting another spell.
        \end{spelltargetinginfo}
        \begin{spelleffects}
            \spelleffect You gain two legend points.
            \spelldur Until the end of the round
        \end{spelleffects}
    \end{spellcontent}
    \begin{spellfooter}
        \spellinfo{Divination}{Arcane, Divination, Knowledge}
        \spellnotes After using this spell, you cannot cast any \spell{moment of prescience} spell for 1 hour.
        \miscastexplode
    \end{spellfooter}
\end{spellsection}

\pdfbookmark[2]{O-P}{SpellDescriptionsO-P}

\begin{spellsection}{Order's Wrath}[3]
    \begin{spellheader}
    \end{spellheader}
    \begin{spellcontent}
        \begin{spelltargetinginfo}
            \spelltwocol{\spelltgt{One nonlawful creature}}{\spellrng{\rngmed}}
        \end{spelltargetinginfo}
        \begin{spelleffects}
            \begin{spellattack}{Spellpower vs. Mental}
                \spellsuccess \spelldamage{3}{divine}.
                \spellcritical As above, and the target is \immobilized for 5 rounds.
                \spellfailure Half damage, and no additional effects.
            \end{spellattack}
        \end{spelleffects}
    \end{spellcontent}
    \begin{spellfooter}
        \spellinfo{Evocation [Lawful]}{Law}
        \miscastrandom
    \end{spellfooter}
\end{spellsection}

\begin{spellsection}{Persistent Image}[6]
    \begin{spellheader}
    \end{spellheader}
    \begin{spellcontent}
        \begin{spelltargetinginfo}
            \spelltwocol{\spellzone{\arealarge radius}}{\spellrng{\rngmed}}
        \end{spelltargetinginfo}
        \begin{spelleffects}
            \spelleffect A figment of your design appears within the area, as \spell{silent image}, except that sound, smell, and thermal elements are included. When you cast the spell, you set a script for the figment to follow. It follows that script without you having to concentrate on the spell.
            \spelldur \durmed \dismissable
        \end{spelleffects}
    \end{spellcontent}
    \begin{spellfooter}
        \spellinfo{Illusion [Figment]}{Illusion}
        \spellnotes Creatures can identify the illusion, as \spell{silent image}.
        \miscastexplode
    \end{spellfooter}
\end{spellsection}

\begin{spellsection}{Phantasmal Killer}[4]
    \begin{spellheader}
        \spelldesc{You create a phantasmal image of the most fearsome creature imaginable to your foe.}
    \end{spellheader}
    \begin{spellcontent}
        \begin{spelltargetinginfo}
            \spelltwocol{\spelltgt{One creature}}{\spellrng{\rngclose}}
        \end{spelltargetinginfo}
        \begin{spelleffects}
            \begin{spellattack}{Spellpower vs. Mental and Fortitude}
                \spellsuccess[Mental] The target is \frightened for 5 rounds.
                \spellcritical[Mental and Fortitude] The target dies.
                \spellfailure The target is \shaken for 5 rounds.
            \end{spellattack}
        \end{spelleffects}
    \end{spellcontent}
    \begin{spellfooter}
        \spellinfo{Enchantment/Illusion [Death, Fear, Mind]}*{Arcane, Trickery}
        \miscastrandom
    \end{spellfooter}
\end{spellsection}

\begin{spellsection}[Mass]{Phantasmal Killer}[8]
    \begin{spellheader}
    \end{spellheader}
    \begin{spellcontent}
        \begin{spelltargetinginfo}
            \spelltwocol{\spelltgts{Up to five creatures}}{\spellrng{\rngmed}}
        \end{spelltargetinginfo}
        \begin{spelleffects}
            \spellspecial This spell functions like \spell{phantasmal killer}, except that it affects multiple creatures.
        \end{spelleffects}
    \end{spellcontent}
    \begin{spellfooter}
        \spellinfo{Enchantment/Illusion [Death, Fear, Mind, Unreal]}{Arcane, Trickery}
        \miscastexplode
    \end{spellfooter}
\end{spellsection}

\begin{spellsection}{Planar Disruption}[2]
    \begin{spellheader}
        \spelldesc{You disrupt a creature's body by partially thrusting it into another plane.}
    \end{spellheader}
    \begin{spellcontent}
        \begin{spelltargetinginfo}
            \spelltwocol{\spelltgt{One creature}}{\spellrng{\rngclose}}
        \end{spelltargetinginfo}
        \begin{spelleffects}
            \begin{spellattack}{Spellpower vs. Mental}
                \spellsuccess \spelldamage{2}{physical}.
                \spellcritical If the creature is an outsider native to another plane, it is sent back to its home plane. Otherwise, it takes damage as normal.
                \spellfailure Half damage, and no additional effects.
            \end{spellattack}
        \end{spelleffects}
    \end{spellcontent}
    \begin{spellfooter}
        \spellinfo{Conjuration [Planar, Teleportation]}{Arcane, Divine}
        \miscastrandom
    \end{spellfooter}
\end{spellsection}

\begin{spellsection}[Mass]{Planar Disruption}[5]
    \begin{spellheader}
        \spelldesc{You disrupt the bodies of many creatures by partially thrusting them into another plane.}
    \end{spellheader}
    \begin{spellcontent}
        \begin{spelltargetinginfo}
            \spelltwocol{\spelltgts{Up to five creature}}{\spellrng{\rngclose}}
        \end{spelltargetinginfo}
        \begin{spelleffects}
            \begin{spellattack}{Spellpower vs. Mental}
                \spellsuccess \spelldamage{5}{physical}.
                \spellcritical If the creature is an outsider native to another plane, it is sent back to its home plane. Otherwise, it takes damage as normal.
                \spellfailure Half damage, and no additional effects.
            \end{spellattack}
        \end{spelleffects}
    \end{spellcontent}
    \begin{spellfooter}
        \spellinfo{Conjuration [Planar, Teleportation]}*{Arcane, Divine}
        \miscastexplode
    \end{spellfooter}
\end{spellsection}

\begin{spellsection}{Poison}[4]
    \begin{spellheader}
        \spelldesc{Calling upon the venomous powers of natural predators, you inject your foe with a potent poison.}
    \end{spellheader}
    \begin{spellcontent}
        \begin{spelltargetinginfo}
            \spelltwocol{\spelltgt{One creature}}{\spellrng{\rngclose}}
        \end{spelltargetinginfo}
        \begin{spelleffects}
            \begin{spellattacktriggered}{At the end of every round, you make a Spellpower vs. Fortitude against the target.}
                \spellsuccess If this is the first successful attack, the target is \sickened. If this is the second successful attack, the target is \nauseated. If this is the third successful attack, the target is \paralyzed.
                \spellfailure If this is the second failed attack, the target resists the poison. No further attacks are made, though the effects of any previous attacks linger until the end of the spell.
            \end{spellattacktriggered}
            \spelldur 5 minutes
        \end{spelleffects}
    \end{spellcontent}
    \begin{spellfooter}
        \spellinfo{Vivimancy [Flesh, Physical, Poison]}{Death, Divine, Nature}
        \spellnotes \physicalspellnotes
        \miscastrandom
    \end{spellfooter}
\end{spellsection}

\begin{spellsection}{Polar Ray}[6]
    \begin{spellheader}
        \spelldesc{You fire a blue-white ray of frigid air and ice, freezing your foe in place.}
    \end{spellheader}
    \begin{spellcontent}
        \begin{spelltargetinginfo}
            \spelltwocol{\spelltgt{One creature or object}}{\spellrng{\rngclose}}
        \end{spelltargetinginfo}
        \begin{spelleffects}
            \begin{spellattack}{Spellpower vs. Reflex}
                \spellsuccess \spelldamage{6}{cold}. In addition, the target is \slowed for 5 rounds.
                \spellfailure Half damage, and the target moves at half speed for 5 rounds.
            \end{spellattack}
        \end{spelleffects}
    \end{spellcontent}
    \begin{spellfooter}
        \spellinfo{Evocation [Cold]}{Arcane, Water}
        \miscastrandom
    \end{spellfooter}
\end{spellsection}

\begin{spellsection}{Power Word Blind}[9]
    \begin{spellheader}
    \end{spellheader}
    \begin{spellcontent}
        \begin{spelltargetinginfo}
            \spelltwocol{\spelltgt{One creature}}{\spellrng{\rngclose}}
            \spellcmp{Verbal only}
        \end{spelltargetinginfo}
        \begin{spelleffects}
            \begin{spellattack}{Spellpower vs. Fortitude}
                \spellsuccess The target is \blinded for 5 rounds.
                \spellfailure The target's vision is \impaired for 5 rounds. This affects all sight-related actions, including physical attacks and targeted spells.
            \end{spellattack}
        \end{spelleffects}
    \end{spellcontent}
    \begin{spellfooter}
        \spellinfo{Vivimancy [Flesh]}{Arcane}
        \miscastrandom
    \end{spellfooter}
\end{spellsection}

\begin{spellsection}{Power Word Confuse}[8]
    \begin{spellheader}
    \end{spellheader}
    \begin{spellcontent}
        \begin{spelltargetinginfo}
            \spelltwocol{\spelltgt{One creature}}{\spellrng{\rngclose}}
            \spellcmp{Verbal only}
        \end{spelltargetinginfo}
        \begin{spelleffects}
            \begin{spellattack}{Spellpower vs. Mental}
                \spellsuccess The target is \confused for 5 rounds.
                \spellfailure The target is \disoriented for 5 rounds.
            \end{spellattack}
        \end{spelleffects}
    \end{spellcontent}
    \begin{spellfooter}
        \spellinfo{Enchantment [Compulsion, Mind]}{Arcane}
        \miscastrandom
    \end{spellfooter}
\end{spellsection}

\begin{spellsection}{Power Word Daze}[2]
    \begin{spellheader}
    \end{spellheader}
    \begin{spellcontent}
        \begin{spelltargetinginfo}
            \spelltwocol{\spelltgt{One creature}}{\spellrng{\rngclose}}
            \spellcmp{Verbal only}
        \end{spelltargetinginfo}
        \begin{spelleffects}
            \begin{spellattack}{Spellpower vs. Mental}
                \spellsuccess The target is \dazed for 5 rounds.
            \end{spellattack}
        \end{spelleffects}
    \end{spellcontent}
    \begin{spellfooter}
        \spellinfo{Enchantment [Compulsion, Mind]}{Arcane}
        \miscastrandom
    \end{spellfooter}
\end{spellsection}

\begin{spellsection}{Power Word Impair}[5]
    \begin{spellheader}
    \end{spellheader}
    \begin{spellcontent}
        \begin{spelltargetinginfo}
            \spelltwocol{\spelltgt{One creature}}{\spellrng{\rngclose}}
            \spellcmp{Verbal only}
        \end{spelltargetinginfo}
        \begin{spelleffects}
            \begin{spellattack}{Spellpower vs. Mental}
                \spellsuccess The target is \severelyimpaired with all actions for 5 rounds.
                \spellfailure The target is \impaired with all actions for 5 rounds.
            \end{spellattack}
        \end{spelleffects}
    \end{spellcontent}
    \begin{spellfooter}
        \spellinfo{Enchantment [Compulsion, Mind]}{Arcane}
        \miscastrandom
    \end{spellfooter}
\end{spellsection}

\begin{spellsection}{Power Word Stun}[6]
    \begin{spellheader}
        \spelldesc{You utter a single word of power that instantly causes your foe to become stunned, whether the creature can hear the word or not.}
    \end{spellheader}
    \begin{spellcontent}
        \begin{spelltargetinginfo}
            \spelltwocol{\spelltgt{One creature}}{\spellrng{\rngclose}}
            \spellcmp{Verbal only}
        \end{spelltargetinginfo}
        \begin{spelleffects}
            \begin{spellattack}{Spellpower vs. Mental}
                \spellsuccess The target is \dazed and moves at half speed for 5 rounds.
                \spellcritical The target is \stunned for 5 rounds.
                \spellfailure The target moves at half speed for 5 rounds.
            \end{spellattack}
        \end{spelleffects}
    \end{spellcontent}
    \begin{spellfooter}
        \spellinfo{Enchantment [Compulsion, Mind]}{Arcane}
        \miscastrandom
    \end{spellfooter}
\end{spellsection}

\begin{spellsection}{Precognition}[5]
    \begin{spellheader}
        \spelldesc{You gain a sixth sense in relation to yourself.}
    \end{spellheader}
    \begin{spellcontent}
        \begin{spelltargetinginfo}
            \spelltgt{You}
        \end{spelltargetinginfo}
        \begin{spelleffects}
            \spelleffect You gain two legend points.
            \spelldur \durshort \dismissable
        \end{spelleffects}
    \end{spellcontent}
    \begin{spellfooter}
        \spellinfo{Divination}{Arcane, Divination}
        \miscastexplode
    \end{spellfooter}
\end{spellsection}

\begin{spellsection}[Greater]{Precognition}[9]
    \begin{spellheader}
    \end{spellheader}
    \begin{spellcontent}
        \begin{spelltargetinginfo}
            \spelltgt{You}
        \end{spelltargetinginfo}
        \begin{spelleffects}
            \spelleffect You gain three legend points.
            \spelldur \durshort \dismissable
        \end{spelleffects}
    \end{spellcontent}
    \begin{spellfooter}
        \spellinfo{Divination}{Arcane, Divination}
        \miscastexplode
    \end{spellfooter}
\end{spellsection}

\begin{spellsection}[Lesser]{Precognition}[1]
    \begin{spellheader}
    \end{spellheader}
    \begin{spellcontent}
        \begin{spelltargetinginfo}
            \spelltgt{You}
        \end{spelltargetinginfo}
        \begin{spelleffects}
            \spelleffect You gain a legend point.
            \spelldur \durshort \dismissable
        \end{spelleffects}
    \end{spellcontent}
    \begin{spellfooter}
        \spellinfo{Divination}{Arcane, Divination}
        \miscastexplode
    \end{spellfooter}
\end{spellsection}

\begin{spellsection}{Prismatic Beam}[3]
    \begin{spellheader}
    \end{spellheader}
    \begin{spellcontent}
        \begin{spelltargetinginfo}
            \spelltwocol{\spelltgt{One creature}}{\spellrng{\rngmed}}
        \end{spelltargetinginfo}
        \begin{spelleffects}
            \spellspecial The target is struck by a randomly colored beam of light. The beam color determines the effect and the defense used, as shown on \tref{Prismatic Beam Effects}. The damaging effects deal \spelldamage{3}{}.
        \end{spelleffects}
    \end{spellcontent}
    \begin{spellfooter}
        \spellinfo{Universal [Light]}{Arcane}
        \miscastrandom
    \end{spellfooter}
\end{spellsection}
\begin{dtable*}
    \lcaption{Prismatic Beam Effects}
    \begin{dtabularx}{\textwidth}{l >{\lcol}p{3.6em} l >{\lcol}X l}
        \thead{1d8} & \thead{Color of Beam} & \thead{Defense} & \thead{Success}\fn{1} & \thead{Failure} \\
        \hline
        1 & Red      & Reflex    & Fire damage and ignited for 5 rounds                       & Half damage, not ignited   \\
        2 & Orange   & Fortitude & Blinded for 1 round                                        & No effect                  \\
        3 & Yellow   & Reflex    & Electricity damage and staggered for 1 round               & Half damage, not staggered \\
        4 & Green    & Fortitude & Acid damage and sickened for 5 rounds                      & Half damage, sickened      \\
        5 & Blue     & Mental    & Slowed for 5 rounds                                        & Half speed for 5 rounds    \\
        6 & Indigo   & Mental    & Confused for 1 round                                       & No effect                  \\
        7 & Violet   & None      & Damage of all energy types (acid, cold, electricity, fire) & \x                         \\
        8 & Octarine & \x        & Struck by two beams; roll twice more, ignoring any ``8'' results.
    \end{dtabularx}
    1 See \pcref{Conditions} for a summary of the conditions imposed.
\end{dtable*}

\begin{spellsection}{Prismatic Storm}[9]
    \begin{spellheader}
    \end{spellheader}
    \begin{spellcontent}
        \begin{spelltargetinginfo}
            \spelltwocol{\spellburst{\arealarge radius}}{\spellrng{\rngmed}}
        \end{spelltargetinginfo}
        \begin{spelleffects}
            \begin{spellattack}{Spellpower vs. Special}
                \spellspecial The target is struck by a randomly colored beam of light. The beam color determines the effect and the defense used, as shown on \tref{Prismatic Beam Effects}. The damaging effects deal \spelldamage{9}{}[d6].
            \end{spellattack}
        \end{spelleffects}
    \end{spellcontent}
    \begin{spellfooter}
        \spellinfo{Universal [Light]}{Arcane}
        \miscastyou
    \end{spellfooter}
\end{spellsection}

\begin{spellsection}{Prismatic Spray}[6]
    \begin{spellheader}
        \spelldesc{This spell causes seven shimmering, intertwined, multicolored beams of light to spray from your hand.}
    \end{spellheader}
    \begin{spellcontent}
        \begin{spelltargetinginfo}
            \spellburst{\arealarge cone}
            \spelltgts{All creatures in the area}
        \end{spelltargetinginfo}
        \begin{spelleffects}
            \begin{spellattack}{Spellpower vs. Special}
                \spellspecial The target is struck by a randomly colored beam of light. The beam color determines the effect and the defense used, as shown on \tref{Prismatic Beam Effects}. The damaging effects deal \spelldamage{7}{}[d6]
            \end{spellattack}
        \end{spelleffects}
    \end{spellcontent}
    \begin{spellfooter}
        \spellinfo{Universal [Light]}{Arcane, Chaos}
        \miscastexplode
    \end{spellfooter}
\end{spellsection}

\begin{spellsection}{Prismatic Wall}[5]
    \begin{spellheader}
    \end{spellheader}
    \begin{spellcontent}
        \begin{spelltargetinginfo}
            \spelltwocol{\spellzone{\arealarge wall, 20 ft. high}}{\spellrng{\rngmed}}
        \end{spelltargetinginfo}
        \begin{spelleffects}
            \spelleffect This spell creates a shimmering, multicolored plane of light that blocks all sight.
            \spelldur \durshort \dismissable
        \end{spelleffects}
    \end{spellcontent}
    \begin{spellsubcontent}
        \begin{spelltargetinginfo}
            \spelltwocol{\spelltgr{A creature passes through the wall}}{\spelltgt{Triggering creature}}
        \end{spelltargetinginfo}
        \begin{spelleffects}
            \begin{spellattack}{Spellpower vs. Reflex}
                \spellspecial The target is struck by a randomly colored beam of light. The beam color determines the effect and the defense used, as shown on \tref{Prismatic Beam Effects}. The damaging effects deal \spelldamage{5}{}[d6]
            \end{spellattack}
        \end{spelleffects}
    \end{spellsubcontent}
    \begin{spellfooter}
        \spellinfo{Universal [Light]}{Arcane, Chaos}
        \spellnotes This spell can be made permanent with a \spell{permanency} ritual.
        \miscastexplode
    \end{spellfooter}
\end{spellsection}

\begin{spellsection}{Prohibition}[6]
    \begin{spellheader}
    \end{spellheader}
    \begin{spellcontent}
        \begin{spelltargetinginfo}
            \spellemanation{\arealarge radius centered on you}
        \end{spelltargetinginfo}
        \begin{spelleffects}
            \spelleffect You loudly declare a prohibition on a single, specific action which creatures must not take, such as ``Do not use ranged weapons'' or ``Do not lie''. You may choose any action that must be taken intentionally, but not involuntary actions or states of being, such as breathing or wearing armor. If the rule is too complicated, the spell fails.

            The spell grants all creatures that enter the area an understanding of the prohibition, even if they were unable to understand the rule as originally stated. If you break the rule, the spell ends -- after you suffer the consequences.
            \spelldur \durshort
        \end{spelleffects}
    \end{spellcontent}
    \begin{spellsubcontent}
        \begin{spelltargetinginfo}
            \spelltwocol{\spelltgr{A creature breaks the rule}}{\spelltgt{Triggering creature}}
        \end{spelltargetinginfo}
        \begin{spelleffects}
            \spelleffect \spelldamage{6}{}[d6]. You know a creature broke the rule, but not which creature.
        \end{spelleffects}
    \end{spellsubcontent}
    \begin{spellfooter}
        \spellinfo{Abjuration/Divination}*{Abjuration, Law}
        \spellnotes Mindless creatures are given no special insight into the rule. Any individual creature can only take damage for breaking the rule once per round.
        \miscastexplode
    \end{spellfooter}
\end{spellsection}

\begin{spellsection}{Project Image}[6]
    \begin{spellheader}
    \end{spellheader}
    \begin{spellcontent}
        \begin{spelltargetinginfo}
            \spellrng{\rngmed}
        \end{spelltargetinginfo}
        \begin{spelleffects}
            \spelleffect You tap energy from the Plane of Shadow to create a quasi-real version of yourself. The projected image looks, sounds, and smells like you, but is intangible. Normally, it mimics your actions perfectly, including speech.
            \par As a swift action, you can attune to the projected image. This has several effects.
            \begin{itemize}
                \item You see and hear from the image's location, rather from where your body is.
                \item Any spells you cast originate from the image instead of from you. This causes you to measure range, line of effect, and so on from the image's location, rather than from your location.
                \item You can control the image's actions independently from your own actions. Each round, it can move up to 100 feet in any direction, including vertically.
            \end{itemize}

            As a free action, you can stop attuning to the projected image, restoring your perceptions and spells to your original body.

            \spelldur \durmed \dismissable
        \end{spelleffects}
    \end{spellcontent}
    \begin{spellfooter}
        \spellinfo{Conjuration/Illusion [Planar, Unreal]}{Arcane}
        \spellnotes You must maintain line of effect to the projected image at all times. If your line of effect is obstructed, the spell ends. If you teleport or use a similar effect that breaks your line of effect, even momentarily, the spell ends.

        Since the image is not a creature, it is difficult to disrupt, and many spells have no effect on it.
        \miscastexplode
    \end{spellfooter}
\end{spellsection}

\begin{spellsection}{Protection from Alignment}[2]
    \begin{spellheader}
    \end{spellheader}
    \begin{spellcontent}
        \begin{spelltargetinginfo}
            \spelltwocol{\spelltgt{One creature}}{\spellrng{\rngclose}}
        \end{spelltargetinginfo}
        \begin{spelleffects}
            \spellspecial Choose an alignment other than neutral (chaotic, good, evil, lawful).
            \spelleffect The target is protected from attacks by creatures of the chosen alignment.
            \spelldur \durshort \dismissable
        \end{spelleffects}
    \end{spellcontent}
    \begin{spellsubcontent}
        \begin{spelltargetinginfo}
            \spelltgr{The target is attacked by a creature of the chosen alignment}
            \spelltgt{The attacking creature}
        \end{spelltargetinginfo}
        \begin{spelleffects}
            \begin{spellattack}{Spellpower vs. Mental}
                \spellsuccess \spelldamage{2}{divine}[d6]
            \end{spellattack}
        \end{spelleffects}
    \end{spellsubcontent}
    \begin{spellfooter}
        \spellinfo{Abjuration [Retributive, Shielding]}{Arcane, Chaos, Divine, Evil, Good, Law}
        \spellnotes This spell has the subtype of the alignment opposed to the chosen alignment.
        \miscastrandom
    \end{spellfooter}
\end{spellsection}

\pdfbookmark[2]{Q-R}{SpellDescriptionsQR}

\begin{spellsection}{Read Mind}[2]
    \begin{spellheader}
    \end{spellheader}
    \begin{spellcontent}
        \begin{spelltargetinginfo}
            \spelltwocol{\spelltgt{One creature}}{\spellrng{\rngmed}}
        \end{spelltargetinginfo}
        \begin{spelleffects}
            \begin{spellattack}{Spellpower vs. Mental}
                \spellsuccess You can read the target's surface thoughts. You gain a \plus4 bonus to Bluff, Persuasion, and Intimidate checks against a creature whose mind you are reading.
            \end{spellattack}
            \spelldur Concentration
        \end{spelleffects}
    \end{spellcontent}
    \begin{spellfooter}
        \spellinfo{Divination [Mind]}{Arcane, Knowledge}
        \miscastrandom
    \end{spellfooter}
\end{spellsection}

\begin{spellsection}[Greater]{Read Mind}[6]
    \begin{spellheader}
    \end{spellheader}
    \begin{spellcontent}
        \begin{spelltargetinginfo}
            \spelltwocol{\spelltgt{One creature}}{\spellrng{\rngmed}}
        \end{spelltargetinginfo}
        \begin{spelleffects}
            \spelleffect You can read the target's surface thoughts, as \spell{read mind}.
            \spelldur Concentration
        \end{spelleffects}
    \end{spellcontent}
    \begin{spellfooter}
        \spellinfo{Divination [Mind]}{Arcane, Knowledge}
        \miscastrandom
    \end{spellfooter}
\end{spellsection}

\begin{spellsection}[Mass]{Read Mind}[8]
    \begin{spellheader}
    \end{spellheader}
    \begin{spellcontent}
        \begin{spelltargetinginfo}
            \spelltwocol{\spelltgts{Up to five creatures}}{\spellrng{\rngmed}}
        \end{spelltargetinginfo}
        \begin{spelleffects}
            \spellspecial This spell functions like \spell{read mind}, except that it affects multiple creatures.
        \end{spelleffects}
    \end{spellcontent}
    \begin{spellfooter}
        \spellinfo{Divination [Mind]}{Arcane, Knowledge}
        \miscastexplode
    \end{spellfooter}
\end{spellsection}

\begin{spellsection}{Reduce Person}[1]
    \begin{spellheader}
    \end{spellheader}
    \begin{spellcontent}
        \begin{spelltargetinginfo}
            \spelltwocol{\spelltgt{One humanoid creature}}{\spellrng{\rngmed}}
            \spelltime{Full-round action}
        \end{spelltargetinginfo}
        \begin{spelleffects}
            \begin{spellattack}{Spellpower vs. Fortitude}
                \spellsuccess The target and its equipment instantly shrinks, halving its height and dividing its weight by 8. This changes the creature's size category to the next smaller one. This has several effects.
                \begin{itemize} 
                    \item \minus10 ft. penalty to movement speed.
                    \item \minus4 penalty to maneuver attack and defense.
                    \item \plus1 bonus to other physical attacks and defenses.
                    \item \plus4 bonus to Stealth checks.
                    \item Melee weapons decrease damage die size by one.
                \end{itemize}
                \par Equipment that leaves the target's possession returns to its original size.
            \end{spellattack}
            \spelldur \durshort \dismissable
        \end{spelleffects}
    \end{spellcontent}
    \begin{spellfooter}
        \spellinfo{Transmutation [Alteration, Sizing]}{Transmutation}
        \spellnotes A Small humanoid creature whose size decreases to Tiny has a space of 2-1/2 feet and a natural reach of 0 feet (meaning that it must enter an opponent's square to attack).

        \sizingspellnotes This spell can be made permanent with a \spell{permanency} ritual.
        \miscastrandom
    \end{spellfooter}
\end{spellsection}

\begin{spellsection}[Mass]{Reduce Person}[4]
    \begin{spellheader}
    \end{spellheader}
    \begin{spellcontent}
        \begin{spelltargetinginfo}
            \spelltwocol{\spelltgts{Up to five humanoid creatures}}{\spellrng{\rngmed}}
        \end{spelltargetinginfo}
        \begin{spelleffects}
            \begin{spellattack}{Spellpower vs. Fortitude}
                \spellsuccess The target shrinks, as \spell{reduce person}.
            \end{spellattack}
            \spelldur \durshort \dismissable
        \end{spelleffects}
    \end{spellcontent}
    \begin{spellfooter}
        \spellinfo{Transmutation [Alteration, Sizing]}{Transmutation}
        \spellnotes As \spell{reduce person}.
        \miscastexplode
    \end{spellfooter}
\end{spellsection}

\begin{spellsection}[Lesser]{Regeneration}[1]
    \begin{spellheader}
        \spelldesc{You grant an ally's body the ability to heal itself rapidly.}
    \end{spellheader}
    \begin{spellcontent}
        \begin{spelltargetinginfo}
            \spelltwocol{\spelltgt{One living creature}}{\spellrng{Touch}}
        \end{spelltargetinginfo}
        \begin{spelleffects}
            \spelleffect At the end of every round, the target regains one hit point per spellpower.
            \spelldur 5 rounds.
        \end{spelleffects}
    \end{spellcontent}
    \begin{spellfooter}
        \spellinfo{Transmutation [Augment]}{Divine, Nature}
        \miscastexplode
    \end{spellfooter}
\end{spellsection}

\begin{spellsection}{Regeneration}[7]
    \begin{spellheader}
        \spelldesc{You grant an ally's body the ability to heal itself rapidly.}
    \end{spellheader}
    \begin{spellcontent}
        \begin{spelltargetinginfo}
            \spelltwocol{\spelltgt{One living creature}}{\spellrng{Touch}}
        \end{spelltargetinginfo}
        \begin{spelleffects}
            \spelleffect At the end of every round, the target regains one hit point per spellpower. In addition, the target is immune to being sickened, nauseated, staggered, or poisoned.

            \par You can also use this spell to regrow lost portions of the target's body and to reattach severed limbs or body parts, if both you and the target do nothing but concentrate on regrowing the lost body part or reattaching the severed limb for the spell's duration.
            \spelldur 5 rounds.
        \end{spelleffects}
    \end{spellcontent}
    \begin{spellfooter}
        \spellinfo{Transmutation [Augment]}{Divine, Nature}
        \miscastexplode
    \end{spellfooter}
\end{spellsection}

\begin{spellsection}{Repulsion}[5]
    \begin{spellheader}
        \spelldesc{An invisible, mobile field surrounds you and prevents creatures from approaching you.}
    \end{spellheader}
    \begin{spellcontent}
        \begin{spelltargetinginfo}
            \spellemanation{\arealarge radius centered on you}
            \spelltwocol{\spelltgr{A creature in the area moves towards you}}{\spelltgt{The moving creature}}
        \end{spelltargetinginfo}
        \begin{spelleffects}
            \begin{spellattack}{Spellpower vs. Mental}
                \spellsuccess The target is unable to move towards you. It can stand still, or alter the direction of its movement to move parallel towards you or away from you.
            \end{spellattack}
            \spelldur \durshort \dismissable
        \end{spelleffects}
    \end{spellcontent}
    \begin{spellfooter}
        \spellinfo{Abjuration [Barrier]}{Arcane, Protection, Travel}
        \spellnotes Unlike most barrier spells, this spell does not collapse if you move towards a creature held at bay by the barrier. The spell continues to prevent that creature from approaching you, but the creature suffers no other ill effect.
        \miscastexplode
    \end{spellfooter}
\end{spellsection}

\begin{spellsection}{Resist Energy}[1]
    \begin{spellheader}
    \end{spellheader}
    \begin{spellcontent}
        \begin{spelltargetinginfo}
            \spellquicktargeting{One creature}{\rngclose}
        \end{spelltargetinginfo}
        \begin{spelleffects}
            \spellspecial Choose a type of energy (acid, cold, electricity, fire).
            \spelleffect The target gains damage reduction against the chosen energy type equal to twice your spellpower.
            \spelldur \durpersonallong
        \end{spelleffects}
    \end{spellcontent}
    \begin{spellfooter}
        \spellinfo{Abjuration [Shielding]}{Arcane, Divine, Nature, Protection}
        \spellnotes A character can only be affected by one \spell{resist energy} spell at once.
        \miscastexplode
    \end{spellfooter}
\end{spellsection}

\begin{spellsection}[Greater]{Resist Energy}[3]
    \begin{spellheader}
    \end{spellheader}
    \begin{spellcontent}
        \begin{spelltargetinginfo}
            \spelltwocol{\spelltgt{One creature}}{\spellrng{\rngclose}}
        \end{spelltargetinginfo}
        \begin{spelleffects}
            \spelleffect The target gains damage reduction against all energy types (acid, cold, electricity, fire) equal to twice your spellpower.
            \spelldur \durpersonallong
        \end{spelleffects}
    \end{spellcontent}
    \begin{spellfooter}
        \spellinfo{Abjuration [Shielding]}{Arcane, Divine, Nature, Protection}
        \spellnotes A character can only be affected by one \spell{resist energy} spell at once.
        \miscastexplode
    \end{spellfooter}
\end{spellsection}

\begin{spellsection}{Retributive Shield}[5]
    \begin{spellheader}
    \end{spellheader}
    \begin{spellcontent}
        \begin{spelltargetinginfo}
            \spelltwocol{\spelltgt{One creature}}{\spellrng{\rngclose}}
        \end{spelltargetinginfo}
        \begin{spelleffects}
            \spelleffect The target gains damage reduction against physical damage equal to your spellpower. Life damage ignores this damage reduction and negates it for 1 round.

            \par Any creature within \rngmed range of the target that deals damage to it takes life damage equal to the damage resisted by this damage reduction.
            \spelldur \durshort
        \end{spelleffects}
    \end{spellcontent}
    \begin{spellfooter}
        \spellinfo{Abjuration/Vivimancy [Life, Shielding]}{Arcane}
        \miscastrandom
    \end{spellfooter}
\end{spellsection}

\begin{spellsection}{Retrieve Object}[1]
    \begin{spellheader}
        \spelldesc{You teleport an object into your hand.}
    \end{spellheader}
    \begin{spellcontent}
        \begin{spelltargetinginfo}
            \spelltwocol{\spelltgt{One unattended object (Medium or smaller)}}{\spellrng{\rngmed}}
        \end{spelltargetinginfo}
        \begin{spelleffects}
            \spelleffect The target teleports into your hands.
        \end{spelleffects}
    \end{spellcontent}
    \begin{spellfooter}
        \spellinfo{Conjuration [Teleportation]}{Conjuration}
        \spellnotes This spell has no effect on attended objects or intelligent items.
        \miscastrandom
    \end{spellfooter}
\end{spellsection}

\begin{spellsection}[Greater]{Retrieve Object}[4]
    \begin{spellheader}
    \end{spellheader}
    \begin{spellcontent}
        \begin{spelltargetinginfo}
            \spelltwocol{\spelltgt{One object (Medium or smaller)}}{\spellrng{\rngmed}}
        \end{spelltargetinginfo}
        \begin{spelleffects}
            \begin{spellattack}{Spellpower vs. Mental}
                \spellsuccess The target teleports into your hands.
            \end{spellattack}
        \end{spelleffects}
    \end{spellcontent}
    \begin{spellfooter}
        \spellinfo{Conjuration [Teleportation]}{Conjuration}
        \spellnotes As \spell{retrieve object}.
        \miscastrandom
    \end{spellfooter}
\end{spellsection}

\begin{spellsection}{Revelation}[9]
    \begin{spellheader}
        \spelldesc{You grant the target a powerful vision of a possible future.}
    \end{spellheader}
    \begin{spellcontent}
        \begin{spelltargetinginfo}
            \spellquicktargeting{One creature}{\rngclose}
        \end{spelltargetinginfo}
        \begin{spelleffects}
            \spellspecial This spell has three versions. Its effects depend on which version is chosen.
            \spelleffect[\subspell{Revelation of Destruction}] You inflict a vision of a terrible future upon the target. It is \severelyimpaired with all actions as it struggles to avoid the certainty of its own doom.
            \spelleffect[\subspell{Revelation of Prowess}] You show the target a vision of its success in the combat to come. It gains the benefits of a \spell{precognition} spell.
            \spelleffect[\subspell{Revelation of Truth}] You show the target the truth of the world around it. It gains the benefits of a \spell{true seeing} spell.
            \spelldur \durshort
        \end{spelleffects}
    \end{spellcontent}
    \begin{spellfooter}
        \spellinfo{Divination}{Arcane, Knowledge}
        \spellnotes Creatures without an Intelligence are not affected by this spell.
        \miscastrandom
    \end{spellfooter}
\end{spellsection}

\begin{spellsection}{Reverse Gravity}[8]
    \begin{spellheader}
    \end{spellheader}
    \begin{spellcontent}
        \begin{spelltargetinginfo}
            \spelltwocol{\spellzone{\areamed radius cylinder, 50 ft. high}}{\spellrng{\rngmed}}
        \end{spelltargetinginfo}
        \begin{spelleffects}
            \spelleffect Gravity is reversed in the area. Everything inside falls upwards, reaching the top of the area within 1 round. If a ``falling'' object or creature strikes a solid object, such as a ceiling, it is affected in the same way as it would be during a normal fall. Otherwise, it floats at the top of the area, oscillating slightly.

            When the spell ends, everything still floating falls, potentially taking damage for the fall.
            \spelldur Concentration (up to 5 rounds)
        \end{spelleffects}
    \end{spellcontent}
    \begin{spellfooter}
        \spellinfo{Transmutation}{Air, Transmutation, Trickery}
        \spellnotes Creatures who can fly or levitate can keep themselves from falling, though the shift in gravity can be disorienting.
        \miscastyou
    \end{spellfooter}
\end{spellsection}

\begin{spellsection}{Revivify}[5]
    \begin{spellheader}
        \spelldesc{You reconnect a corpse's soul with its body before the soul has completely passed on.}
    \end{spellheader}
    \begin{spellcontent}
        \begin{spelltargetinginfo}
            \spelltwocol{\spelltgt{One dead creature}}{\spellrng{Touch}}
            \spellcmp{Verbal, Somatic, and Material}
        \end{spelltargetinginfo}
        \begin{spelleffects}
            \spelleffect If the target has been dead for no more than one round per four spellpower, it is restored to life. This functions like \spell{lesser resurrection} ritual, except that the target suffers no negative effects for having died.
        \end{spelleffects}
    \end{spellcontent}
    \begin{spellfooter}
        \spellinfo{Vivimancy [Life]}{Divine}
        \spellmat{Diamonds worth at least 500 gp.}
        \miscastexplode
    \end{spellfooter}
\end{spellsection}

\pdfbookmark[2]{S}{SpellDescriptionsS}

\begin{spellsection}{Sanctuary}[1]
    \begin{spellheader}
    \end{spellheader}
    \begin{spellcontent}
        \begin{spelltargetinginfo}
            \spelltwocol{\spelltgt{One creature}}{\spellrng{Touch}}
        \end{spelltargetinginfo}
        \begin{spelleffects}
            \spelleffect The target is protected from attacks. If it takes any actions, this spell immediately ends.
            \spelldur \durshort
        \end{spelleffects}
    \end{spellcontent}
    \begin{spellsubcontent}
        \begin{spelltargetinginfo}
            \spelltgr{A creature attacks the target}
            \spelltgt{The attacking creature}
        \end{spelltargetinginfo}
        \begin{spelleffects}
            \begin{spellattack}{Spellpower vs. Mental}
                \spellsuccess The target's attack fails, and it is unable to attack the protected creature for the duration of the spell.
            \end{spellattack}
        \end{spelleffects}
    \end{spellsubcontent}
    \begin{spellfooter}
        \spellinfo{Abjuration/Enchantment [Compulsion, Mind, Shielding]}{Arcane, Divine, Protection}
        \spellnotes This is considered a mental effect on any creature that attempts to attack the target. Creatures immune to mental effects can attack the target freely.
        \miscastexplode
    \end{spellfooter}
\end{spellsection}

\begin{spellsection}{Scorching Ray}[3]
    \begin{spellheader}
        \spelldesc{You blast your foe with a fiery ray.}
    \end{spellheader}
    \begin{spellcontent}
        \begin{spelltargetinginfo}
            \spelltwocol{\spelltgt{One creature or object}}{\spellrng{\rngclose}}
        \end{spelltargetinginfo}
        \begin{spelleffects}
            \begin{spellattack}{Spellpower vs. Reflex}
                \spellsuccess \spelldamage{3}{fire}. In addition, the target is \ignited for 5 rounds.
                \spellfailure Half damage, and no additional effects.
            \end{spellattack}
        \end{spelleffects}
    \end{spellcontent}
    \begin{spellfooter}
        \spellinfo{Evocation [Destructive, Fire]}{Arcane, Destruction, Fire}
        \miscastrandom
    \end{spellfooter}
\end{spellsection}

\begin{spellsection}{Searing Light}[3]
    \begin{spellheader}
        \spelldesc{You fire a blast of light that strikes your foe.}
    \end{spellheader}
    \begin{spellcontent}
        \begin{spelltargetinginfo}
            \spelltwocol{\spelltgt{One creature or object}}{\spellrng{\rngclose}}
        \end{spelltargetinginfo}
        \begin{spelleffects}
            \begin{spellattack}{Spellpower vs. Reflex}
                \spellspecial You gain a \plus5 bonus to attack against creatures vulnerable to bright light.
                \spellsuccess \spelldmg{3}{solar}. In addition, the target's vision is \impaired for 5 rounds. This affects all sight-related actions, including physical attacks and targeted spells.
                \spellfailure Half damage, and no additional effects.
            \end{spellattack}
        \end{spelleffects}
    \end{spellcontent}
    \begin{spellfooter}
        \spellinfo{Evocation [Light]}{Divine}
        \miscastrandom
    \end{spellfooter}
\end{spellsection}

\begin{spellsection}{See Invisibility}[1]
    \begin{spellheader}
    \end{spellheader}
    \begin{spellcontent}
        \begin{spelltargetinginfo}
            \spelltwocol{\spelltgt{One creature}}{\spellrng{Touch}}
        \end{spelltargetinginfo}
        \begin{spelleffects}
            \spelleffect The target can see any objects or beings that are invisible within its range of vision, as well as any that are ethereal, as if they were normally visible. Such creatures are visible as translucent shapes, allowing the target to easily discern the difference between visible, invisible, and ethereal creatures.
            \spelldur \durpersonallong
        \end{spelleffects}
    \end{spellcontent}
    \begin{spellfooter}
        \spellinfo{Divination/Transmutation [Augment]}{Arcane}
        \spellnotes The spell does not reveal the method used to obtain invisibility. It does not reveal illusions other than invisibility. It does not reveal creatures who are simply hiding, concealed, or otherwise hard to see.

        This spell can be made permanent with a \spell{permanency} ritual.
        \miscastexplode
    \end{spellfooter}
\end{spellsection}

\begin{spellsection}[Mass]{See Invisibility}[6]
    \begin{spellheader}
    \end{spellheader}
    \begin{spellcontent}
        \begin{spelltargetinginfo}
            \spelltwocol{\spelltgt{Up to five creatures}}{\spellrng{Touch}}
            \spelltime{Full-round action}
        \end{spelltargetinginfo}
        \begin{spelleffects}
            \spelleffect The target can invisible things, as \spell{see invisibility}.
            \spelldur \durlong
        \end{spelleffects}
    \end{spellcontent}
    \begin{spellfooter}
        \spellinfo{Divination/Transmutation [Augment]}{Arcane}
        \spellnotes As \spell{see invisibility}.
        \miscastexplode
    \end{spellfooter}
\end{spellsection}

\begin{spellsection}{Shadow Puppet}[9]
    \begin{spellheader}
    \end{spellheader}
    \begin{spellcontent}
        \spelleffect You step into the Plane of Shadow, as \spell{shadow walk}. At the same time, you create a quasi-real version of yourself, as \spell{project image}. The duplicate image appears superimposed over your body so that observers don't notice an image appearing and you disappearing. You can then control the image and cast spells through it even though you are on a different plane.
        \spelldur \durshort
    \end{spellcontent}
    \begin{spellfooter}
        \spellinfo{Conjuration/Illusion [Planar, Unreal]}{Illusion}
        \spellnotes If the image moves farther than \rnglong range away from where it was originally created, or if you leave the Plane of Shadow, the image ceases to exist.

        If you are not on the Material Plane when you cast this spell, it has no effect.
        \miscastexplode
    \end{spellfooter}
\end{spellsection}

\begin{spellsection}{Shadow Umbra}[6]
    \begin{spellheader}
        \spelldesc{You shield your ally with a dark umbra that connects directly to the Plane of Shadow.}
    \end{spellheader}
    \begin{spellcontent}
        \begin{spelltargetinginfo}
            \spelltwocol{\spelltgt{One creature}}{\spellrng{\rngclose}}
        \end{spelltargetinginfo}
        \begin{spelleffects}
            \spelleffect All attacks that would affect the creature, including magical and supernatural attacks, have a 50\% chance to be absorbed by the umbra. Attacks absorbed by the umbra have no effect on the target. The umbra is selective, and does not inhibit beneficial effects.

            Whenever the umbra absorbs an attack, it alters the creature's appearance (including smell, sound, and other senses, as appropriate) with a glamer. This causes the creature to seem as if were affected by the attack. Outside observers have no way of knowing which attacks were absorbed by the umbra unless they can recognize the illusion. The spell does not attempt to mimic the effects of extraordinary attacks which cannot be disguised, such as attacks which would destroy the creature's body.
            \spelldur \durshort
        \end{spelleffects}
    \end{spellcontent}
    \begin{spellfooter}
        \spellinfo{Abjuration/Illusion [Glamer, Planar, Shielding]}{Arcane}
        \spellnotes If you are not on the Material Plane or Plane of Shadow when you cast this spell, it has no effect.
        \miscastrandom
    \end{spellfooter}
\end{spellsection}

\begin{spellsection}{Share Pain}[2]
    \begin{spellheader}
    \end{spellheader}
    \begin{spellcontent}
        \begin{spelltargetinginfo}
            \spelltwocol{\spelltgts{Two willing creatures}}{\spellrng{\rngmed}}
        \end{spelltargetinginfo}
        \begin{spelleffects}
            \spellspecial When you cast this spell, you choose which target will be protected.
            \spelleffect When the protected creature would take hit point damage, it instead loses half that many hit points (rounded down), and the other target loses hit points equal to the other half of the damage (rounded up).

            If the targets get out of range of each other, the effect is suppressed until they return within range.
            \spelldur \durlong \dismissable
        \end{spelleffects}
    \end{spellcontent}
    \begin{spellfooter}
        \spellinfo{Abjuration/Vivimancy [Life, Shielding]}{Arcane, Divine, Protection}
        \spellnotes The loss of hit points caused by this spell is not damage, and is not affected by damage reduction or other abilities which affect damage.
        \miscastexplode
    \end{spellfooter}
\end{spellsection}

\begin{spellsection}{Shield of Faith}[1]
    \begin{spellheader}
        \spelldesc{You create a shimmmering, magical shield that protects you.}
    \end{spellheader}
    \begin{spellcontent}
        \begin{spelltargetinginfo}
            \spelltwocol{\spelltgt{One creature}}{\spellrng{\rngtouch}}
        \end{spelltargetinginfo}
        \begin{spelleffects}
            \spelleffect The target gains a floating shield with a \plus2 defense bonus. Unlike a mundane shield, this shield does not require a free hand and has no armor check penalty or arcane spell failure chance.
            \spelldur \durpersonallong
        \end{spelleffects}
    \end{spellcontent}
    \begin{spellfooter}
        \spellinfo{Abjuration [Shielding]}{Divine, Protection}
        \miscastexplode
    \end{spellfooter}
\end{spellsection}

\begin{spellsection}{Shield of Law}[8]
    \begin{spellheader}
        \spelldesc{A dim, blue glow surrounds your allies, protecting them from attacks.}
    \end{spellheader}
    \begin{spellcontent}
        \begin{spelltargetinginfo}
            \spelltwocol{\spellrng{\rngclose}}{\spelltgts{Up to five creatures}}
        \end{spelltargetinginfo}
        \begin{spelleffects}
            \spelleffect The target gains spell resistance against chaotic spells and spells cast by chaotic creatures.
            \spelldur \durshort \dismissable
        \end{spelleffects}
    \end{spellcontent}
    \begin{spellsubcontent}
        \begin{spelltargetinginfo}
            \spelltgr{Whenever a chaotic creature within 30 feet of the target makes a physical attack against it}
            \spelltgt{The attacking creature}
        \end{spelltargetinginfo}
        \begin{spelleffects}
            \begin{spellattack}{Spellpower vs. Mental}
                \spellsuccess \spelldamage{9}{divine}[d6]
            \end{spellattack}
        \end{spelleffects}
    \end{spellsubcontent}
    \begin{spellfooter}
        \spellinfo{Abjuration [Lawful, Retributive, Shielding]}{Divine, Law}
        \miscastexplode
    \end{spellfooter}
\end{spellsection}

\begin{spellsection}{Shocking Grasp}[1]
    \begin{spellheader}
        \spelldesc{You deliver a powerful electrical shock to your foe.}
    \end{spellheader}
    \begin{spellcontent}
        \begin{spelltargetinginfo}
            \spelltwocol{\spelltgt{One creature or object}}{\spellrng{Touch}}
        \end{spelltargetinginfo}
        \begin{spelleffects}
            \begin{spellattack}{Spellpower vs. Reflex}
                \spellspecial You gain a \plus5 bonus to attack if the target is wearing metal armor or otherwise has a significant quantity of metal.
                \spellsuccess \spelldamage{1}{electricity}.
                \spellcritical The target is \staggered for 5 rounds.
                \spellfailure Half damage, and no additional effects.
            \end{spellattack}
        \end{spelleffects}
    \end{spellcontent}
    \begin{spellfooter}
        \spellinfo{Evocation [Electricity]}{Arcane, Destruction}
        \miscastexplode
    \end{spellfooter}
\end{spellsection}

\begin{spellsection}{Shout}[3]
    \begin{spellheader}
        \spelldesc{You emit an ear-splitting yell that deafens and damages creatures in its path.}
    \end{spellheader}
    \begin{spellcontent}
        \begin{spelltargetinginfo}
            \spellburst{\areamed cone}
            \spelltgts{Everything in the area}
            \spellcmp{Verbal only}
        \end{spelltargetinginfo}
        \begin{spelleffects}
            \spellspecial You gain a \plus5 bonus to attack against brittle or crystalline objects and creatures.
            \spellsuccess \spelldamage{3}{sonic}[d6]. In addition, the target is \deafened.
            \spellfailure Half damage, and no additional effects.
        \end{spelleffects}
    \end{spellcontent}
    \begin{spellfooter}
        \spellinfo{Evocation [Destructive, Sonic]}{Arcane, Destruction, Strength}
        \miscastexplode
    \end{spellfooter}
\end{spellsection}

\begin{spellsection}[Greater]{Shout}[6]
    \begin{spellheader}
    \end{spellheader}
    \begin{spellcontent}
        \begin{spelltargetinginfo}
            \spellburst{\arealarge cone}
            \spelltgts{Everything in the area}
            \spellcmp{Verbal only}
        \end{spelltargetinginfo}
        \begin{spelleffects}
            \spellspecial You gain a \plus5 bonus to attack against brittle or crystalline objects and creatures.
            \spellsuccess \spelldamage{6}{sonic}[d6]. In addition, the target is \deafened.
            \spellfailure Half damage, and no additional effects.
        \end{spelleffects}
    \end{spellcontent}
    \begin{spellfooter}
        \spellinfo{Evocation [Destructive, Sonic]}{Arcane, Destruction, Strength}
        \miscastexplode
    \end{spellfooter}
\end{spellsection}

\begin{spellsection}{Shrink Item}[3]
    \begin{spellheader}
    \end{spellheader}
    \begin{spellcontent}
        \begin{spelltargetinginfo}
            \spellquicktargeting{One nonmagical object (Medium or smaller)}{\rngclose}
            \spellspecial You can target a Large object at 10th spellpower, a Huge object at 16th spellpower, or a Gargantuan object at 24th spellpower.
        \end{spelltargetinginfo}
        \begin{spelleffects}
            \spellspecial As you cast this spell, choose a command word.
            \begin{spellattack}{Spellpower vs. Mental}
                \spellsuccess The target shrinks to 1/16 of its normal size in each dimension (to about 1/4,000 the original volume and mass). This change effectively reduces its size by four size categories. If the target is physically unable to shrink, such as a ring on a finger, it shrinks as much as it can without causing harm to itself or the physical impediment.

                As a standard action, any creature can speak the command word to return the target to its original size. It must be resting on a stable surface. If the command word is spoken while the target is not stable, such as while it is in the air, it returns to its original size as soon as it finds a resting point. Restoring the target to its normal size ends the spell.
            \end{spellattack}
            \spelldur 24 hours or until discharged
        \end{spelleffects}
    \end{spellcontent}
    \begin{spellfooter}
        \spellinfo{Transmutation [Alteration]}{Transmutation}
        \spellnotes  If you recast this spell each day on an object, you can keep it at its small size indefinitely.

        This spell can be made permanent with a \spell{permanency} ritual, in which case the affected object can be shrunk and expanded an indefinite number of times, but only by the original caster.
        \miscastrandom
    \end{spellfooter}
\end{spellsection}

\begin{spellsection}{Silence}[2]
    \begin{spellheader}
    \end{spellheader}
    \begin{spellcontent}
        \begin{spelltargetinginfo}
            \spelltwocol{\spelltgt{One creature or object}}{\spellrng{\rngmed}}
        \end{spelltargetinginfo}
        \begin{spelleffects}
            \begin{spellattack}{Spellpower vs. Mental}
                \spelleffect The target becomes unable to make noise. Any noises it makes are inaudible to other creatures. When you cast the spell, you may choose whether the target can still hear itself normally, potentially causing it to be unaware of the effect of the spell.

                Extraordinarily loud noises, such as the yell of a giant, are merely muffled by the spell rather than completely silenced. The DC to hear such sounds produced by the target is increased by 40. Sonic attacks function normally.
            \end{spellattack}
            \spelldur \durshort \dismissable
        \end{spelleffects}
    \end{spellcontent}
    \begin{spellfooter}
        \spellinfo{Illusion [Glamer]}{Divine, Trickery}
        \spellnotes Spellcasters unable to hear themselves cast are treated as deafened, and suffer a 20\% chance of spell failure when casting spells with verbal components.
        \miscastrandom
    \end{spellfooter}
\end{spellsection}

\begin{spellsection}[Mass]{Silence}[5]
    \begin{spellheader}
    \end{spellheader}
    \begin{spellcontent}
        \begin{spelltargetinginfo}
            \spelltwocol{\spelltgt{Up to five creatures or objects}}{\spellrng{\rngmed}}
        \end{spelltargetinginfo}
        \begin{spelleffects}
            \spellspecial This spell functions like \spell{silence}, except that it affects multiple creatures.
        \end{spelleffects}
    \end{spellcontent}
    \begin{spellfooter}
        \spellinfo{Illusion [Glamer]}{Divine, Trickery}
        \spellnotes As \spell{silence}.
        \miscastexplode
    \end{spellfooter}
\end{spellsection}

\begin{spellsection}{Silent Image}[2]
    \begin{spellheader}
    \end{spellheader}
    \begin{spellcontent}
        \begin{spelltargetinginfo}
            \spelltwocol{\spellzone{\areamed radius}}{\spellrng{\rngmed}}
        \end{spelltargetinginfo}
        \begin{spelleffects}
            \spelleffect This spell creates the visual illusion of an object, creature, or force within the area, as determined by you. The illusion does not create sound, smell, texture, or temperature. As a standard action, you can concentrate to alter the image within the area.
            \spelldur \durshort
        \end{spelleffects}
    \end{spellcontent}
    \begin{spellfooter}
        \spellinfo{Illusion [Figment, Unreal]}{Illusion}
        \spellnotes Creatures can recognize the figment as unreal by interacting with it physically, or by making a Awareness check against a DC equal to 10 \add your spellpower. A creature gets a \plus10 bonus on this Awareness check when using senses which should be present in the figment, but which are missing.

        A creature faced with definitive proof that the figment is unreal can disbelieve it, treating it as if it were not there.
        \miscastexplode
    \end{spellfooter}
\end{spellsection}

\begin{spellsection}{Skysmite}[6]
    \begin{spellheader}
        \spelldesc{You call down lightning from the heavens, unerringly striking your foes, even if you cannot see them.}
    \end{spellheader}
    \begin{spellcontent}
        \begin{spelltargetinginfo}
            \spellrng{\rngext}
            \spellburst{\arealarge vertical line of lightning, 5 ft. wide}
            \spelltgts{Everything in the area}
            \spellspecial If no creature or objects lie in the area, the lightning strikes elsewhere instead. It strikes the occupied square within the spell's range that lies closest to its original destination. If multiple occupied squares are equally close, it strikes the largest target.
        \end{spelltargetinginfo}
        \begin{spelleffects}
            \begin{spellattack}{Spellpower vs. Reflex}
                % damage penalty for AOE lightning?
                \spellsuccess \spelldamage{6}{electricity}
                \spellfailure Half damage.
            \end{spellattack}
        \end{spelleffects}
    \end{spellcontent}
    \begin{spellfooter}
        \spellinfo{Evocation [Destructive, Electricity]}{Air, Arcane, Destruction, Nature}
        \spellnotes The lightning can unerringly identify invisible and concealed creatures, but it does not differentiate between friend, foe, and inanimate object.
        \miscastexplode
    \end{spellfooter}
\end{spellsection}

\begin{spellsection}{Sleep}[1]
    \begin{spellheader}
    \end{spellheader}
    \begin{spellcontent}
        \begin{spelltargetinginfo}
            \spelltwocol{\spelltgt{One creature}}{\spellrng{\rngmed}}
            \spellcmp{Somatic only}
        \end{spelltargetinginfo}
        \begin{spelleffects}
            \begin{spellattack}{Spellpower vs. Mental}
                \spellsuccess The target is \fatigued and attempts to go to sleep as soon as possible, though it will not stop fighting to do so. Awakening a creature put to sleep by this spell is difficult, and requires a standard action.
                \spellcritical As above, except that the target is \exhausted instead of \fatigued.
            \end{spellattack}
            \spelldur \durshort
        \end{spelleffects}
    \end{spellcontent}
    \begin{spellfooter}
        \spellinfo{Enchantment [Delusion, Mind, Sleep]}{Arcane}
        \miscastrandom
    \end{spellfooter}
\end{spellsection}

\begin{spellsection}[Mass]{Sleep}[4]
    \begin{spellheader}
    \end{spellheader}
    \begin{spellcontent}
        \begin{spelltargetinginfo}
            \spelltwocol{\spelltgts{Up to five creatures}}{\spellrng{\rngmed}}
        \end{spelltargetinginfo}
        \begin{spelleffects}
            \spellspecial This spell functions like \spell{sleep}, except that it affects multiple targets.
        \end{spelleffects}
    \end{spellcontent}
    \begin{spellfooter}
        \spellinfo{Enchantment [Delusion, Mind, Sleep]}{Arcane}
        \miscastexplode
    \end{spellfooter}
\end{spellsection}

\begin{spellsection}{Slow}[2]
    \begin{spellheader}
        \spelldesc{You decelerate your enemy's motions, causing it to move and act more slowly than normal.}
    \end{spellheader}
    \begin{spellcontent}
        \begin{spelltargetinginfo}
            \spelltwocol{\spelltgt{One creature}}{\spellrng{\rngclose}}
        \end{spelltargetinginfo}
        \begin{spelleffects}
            \begin{spellattack}{Spellpower vs. Mental}
                \spellsuccess The target is \slowed.
                \spellfailure The target moves at half speed.
            \end{spellattack}
            \spelldur \durshort
        \end{spelleffects}
    \end{spellcontent}
    \begin{spellfooter}
        \spellinfo{Transmutation [Temporal]}{Arcane}
        \miscastrandom
    \end{spellfooter}
\end{spellsection}

\begin{spellsection}[Mass]{Slow}[6]
    \begin{spellheader}
        \spelldesc{You decelerate your enemies' motions, causing them to move and act more slowly than normal.}
    \end{spellheader}
    \begin{spellcontent}
        \begin{spelltargetinginfo}
            \spelltwocol{\spelltgts{Up to five creatures}}{\spellrng{\rngmed}}
        \end{spelltargetinginfo}
        \begin{spelleffects}
            \spellspecial This spell functions like \spell{slow}, except that it affects multiple targets.
        \end{spelleffects}
    \end{spellcontent}
    \begin{spellfooter}
        \spellinfo{Transmutation [Temporal]}{Arcane}
        \miscastexplode
    \end{spellfooter}
\end{spellsection}

\begin{spellsection}{Solid Fog}[6]
    \begin{spellheader}
        \spelldesc{You conjure a bank of immensely thick fog, concealing those inside.}
    \end{spellheader}
    \begin{spellcontent}
        \begin{spelltargetinginfo}
            \spelltwocol{\spellzone{\areamed radius cylinder}}{\spellrng{\rngmed}}
        \end{spelltargetinginfo}
        \begin{spelleffects}
            \spelleffect Fog blocks sight in the area, as \spell{fog cloud}. The fog is so thick that all creatures in the area move at half speed and suffer penalties as if they were fighting underwater. Attacks entering or passing through the area are similarly penalized.
            \spelldur \durshort
        \end{spelleffects}
    \end{spellcontent}
    \begin{spellfooter}
        \spellinfo{Conjuration [Creation, Fog, Physical]}{Arcane, Nature, Water}
        \spellnotes \fogspellnotes A severe wind disperses the fog within 1 minute, and a windstorm disperses it within a round.

        \physicalspellnotes

        This spell can be made permanent with a \spell{permanency} ritual. A permanent solid fog dispersed by wind reforms in 10 minutes.
        \miscastyou
    \end{spellfooter}
\end{spellsection}

\begin{spellsection}{Sound Burst}[2]
    \begin{spellheader}
        \spelldesc{You create a cacophony of sound.}
    \end{spellheader}
    \begin{spellcontent}
        \begin{spelltargetinginfo}
            \spelltwocol{\spellburst{\areasmall radius}}{\spellrng{\rngclose}}
            \spelltgts{Everything in the area}
        \end{spelltargetinginfo}
        \begin{spelleffects}
            \begin{spellattack}{Spellpower vs. Fortitude}
                \spellsuccess \spelldamage{2}{sonic}[d6].
                \spellfailure Half damage.
            \end{spellattack}
        \end{spelleffects}
    \end{spellcontent}
    \begin{spellfooter}
        \spellinfo{Evocation [Destructive, Sonic]}{Arcane, Destruction}
        \miscastyou
    \end{spellfooter}
\end{spellsection}

\begin{spellsection}{Spell Resistance}[3]
    \begin{spellheader}
    \end{spellheader}
    \begin{spellcontent}
        \begin{spelltargetinginfo}
            \spelltwocol{\spelltgt{One creature}}{\spellrng{\rngclose}}
        \end{spelltargetinginfo}
        \begin{spelleffects}
            \spelleffect The target gains spell resistance equal to 10 \add your spellpower.
            \spelldur \durshort
        \end{spelleffects}
    \end{spellcontent}
    \begin{spellfooter}
        \spellinfo{Abjuration [Shielding]}{Abjuration, Magic, Protection}
        \spellnotes To affect a creature with spell resistance with a spell, a caster must make an attack with its spellpower. If the attack beats the creature's spell resistance, the spell works normally. Otherwise, the spell has no effect on the creature.
        \miscastrandom
    \end{spellfooter}
\end{spellsection}

\begin{spellsection}{Spelltheft}[5]
    \begin{spellheader}
    \end{spellheader}
    \begin{spellcontent}
        \spellspecial This spell functions like \spell{dispel magic}, except that you can choose to gain the effects of any spells you dispel as if they had been originally cast on you. The effects last for the remainder of their original durations or for 5 rounds, whichever is shorter. Spells that cannot be cast on you, such as spells which only affect the caster, are simply dispelled.
    \end{spellcontent}
    \begin{spellfooter}
        \spellinfo{Abjuration [Antimagic]}{Abjuration, Magic}
        \miscastrandom
    \end{spellfooter}
\end{spellsection}

\begin{spellsection}[Greater]{Spelltheft}[8]
    \begin{spellheader}
    \end{spellheader}
    \begin{spellcontent}
        \spellspecial This spell functions like \spell{greater dispel magic}, except that you can choose to gain the effects of any spells you dispel as if they had been originally cast on you. The effects last for the remainder of their original durations or for 5 rounds, whichever is shorter. Spells that cannot be cast on you, such as spells which only affect the caster, are simply dispelled.
    \end{spellcontent}
    \begin{spellfooter}
        \spellinfo{Abjuration [Antimagic]}{Abjuration, Magic}
        \miscastyou
    \end{spellfooter}
\end{spellsection}

\begin{spellsection}{Spell Turning}[7]
    \begin{spellheader}
    \end{spellheader}
    \begin{spellcontent}
        \begin{spelltargetinginfo}
            \spelltgt{You}
        \end{spelltargetinginfo}
        \begin{spelleffects}
            \spelleffect Whenever a creature targets you with a spell or spell-like ability, it targets itself instead. If the spell affects multiple targets, the other targets are affected normally. If the caster is not a valid target, the spell simply has no effect on you.

            After you reflect one spell per five spellpower, the spell ends.
            \spelldur \durlong or until expended
        \end{spelleffects}
    \end{spellcontent}
    \begin{spellfooter}
        \spellinfo{Abjuration [Shielding]}{Arcane, Magic, Protection}
        \miscastexplode
    \end{spellfooter}
\end{spellsection}

\begin{spellsection}{Spider Climb}[2]
    \begin{spellheader}
        \spelldesc{You grant your ally the ability to climb on walls and ceilings as well as a spider does.}
    \end{spellheader}
    \begin{spellcontent}
        \begin{spelltargetinginfo}
            \spelltwocol{\spelltgt{One creature}}{\spellrng{Touch}}
        \end{spelltargetinginfo}
        \begin{spelleffects}
            \spelleffect The target gains a climb speed of 20 feet. It must use at least one hand to climb in this manner.
            \spelldur \durmed
        \end{spelleffects}
    \end{spellcontent}
    \begin{spellfooter}
        \spellinfo{Transmutation [Augment]}{Arcane, Nature, Travel}
        \spellnotes See \pcref{Climbing}, for more details.
        \miscastexplode
    \end{spellfooter}
\end{spellsection}

%need to run expected damage calculation
\begin{spellsection}{Spiritual Weapon}[3]
    \begin{spellheader}
        \spelldesc{You bring into being a weapon made of pure force which attacks your foes of its own volition.}
    \end{spellheader}
    \begin{spellcontent}
        \begin{spelltargetinginfo}
            \spelltwocol{\spelltgt{One creature or object}}{\spellrng{\rngmed}}
        \end{spelltargetinginfo}
        \begin{spelleffects}
            \spelleffect This spell creates a floating weapon made of magical force. At the beginning of each round, you may spend a swift action to command the weapon. If you do, the weapon moves during the movement phase with a fly speed of 50 feet (perfect maneuverability), and attacks in the action phase. If you do not direct the weapon, it remains motionless.

            The weapon is sized for you, and can be any type of weapon you are proficient with, though the weapon's shape does not alter this spell's effects. Since it is made of force, the weapon is immune to damage and most effects.
            \spelldur \durshort \dismissable
        \end{spelleffects}
    \end{spellcontent}
    \begin{spellsubcontent}
        \begin{spelltargetinginfo}
            \spelltrigger{During the action phase, if you commanded the weapon that round}
            \spelltgt{One creature adjacent to the weapon}
        \end{spelltargetinginfo}
        \begin{spelleffects}
            \begin{spellattack}{Spellpower vs. Armor defense}
                \spellsuccess \spelldamage{3}{force}[d6]
            \end{spellattack}
        \end{spelleffects}
    \end{spellsubcontent}
    \begin{spellfooter}
        \spellinfo{Evocation [Force]}{Divine, War}
        \spellnotes Since the weapon is directed by you, its ability to interact with invisible or concealed creatures is no better than yours. Its special defenses are the same as your special defenses. If the weapon goes out of range of you, the spell ends.
        \miscastexplode
    \end{spellfooter}
\end{spellsection}

\begin{spellsection}{Stinking Cloud}[3]
    \begin{spellheader}
        \spelldesc{You create putrid vapors which obscure sight and sicken creatures.}
    \end{spellheader}
    \begin{spellcontent}
        \begin{spelltargetinginfo}
            \spelltwocol{\spellzone{\areamed radius cylinder}}{\spellrng{\rngmed}}
        \end{spelltargetinginfo}
        \begin{spelleffects}
            \spelleffect Fog fills the area, as \spell{fog cloud}, except that the fog has a putrid stench. All creatures within the area are \sickened for as long as they remain within the cloud, and for 1 round after they leave.
            \spelldur \durshort
        \end{spelleffects}
    \end{spellcontent}
    \begin{spellfooter}
        \spellinfo{Conjuration [Creation, Physical]}{Arcane}
        \spellnotes This spell can be made permanent with a \spell{permanency} ritual. A permanent \spell{stinking cloud} dispersed by wind reforms in 10 minutes. \fogspellnotes \fogwindspellnotes

        \physicalspellnotes
        \miscastyou
    \end{spellfooter}
\end{spellsection}

\begin{spellsection}{Stoneskin}[4]
    \begin{spellheader}
        \spelldesc{You dramatically toughen a creature's skin, giving it the appearance of stone.}
    \end{spellheader}
    \begin{spellcontent}
        \begin{spelltargetinginfo}
            \spelltwocol{\spelltgt{One creature}}{\spellrng{\rngclose}}
        \end{spelltargetinginfo}
        \begin{spelleffects}
            \spelleffect The target gains damage reduction against physical damage equal to half your spellpower. Adamantine weapons ignore this damage reduction and negate it for 1 round.
            \spelldur \durpersonallong
        \end{spelleffects}
    \end{spellcontent}
    \begin{spellfooter}
        \spellinfo{Transmutation [Augment, Earth]}{Arcane, Earth, Nature, Protection}
        \miscastexplode
    \end{spellfooter}
\end{spellsection}

\begin{spellsection}{Storm of Vengeance}[9]
    \begin{spellheader}
    \end{spellheader}
    \begin{spellcontent}
        \begin{spelltargetinginfo}
            \spelltwocol{\spellzone{500 ft. radius cylinder}}{\spellrng{\rnglong}}
            \spelltime{Full-round action}
        \end{spelltargetinginfo}
        \begin{spelleffects}
            \spelleffect An enormous storm cloud occupies the top 200 feet of the area, as \spell{fog cloud}. Within the area, lightning strikes and thunder rolls. Sunlight is blocked by the dark cloud. This may cause the area to have shadowy illumination, granting everything in it \concealment.

            At the end of every round, the storm has an additional effect, as shown on \tref{Storm of Vengeance Effects}. Damaging effects deal \spelldamage{9}{}[d6].

            \spelldur Concentration (maximum 10 rounds)
        \end{spelleffects}
    \end{spellcontent}
    \begin{spellfooter}
        \spellinfo{Conjuration/Evocation [Acid, Creation, Electricity, Physical]}{Air, Divine, Nature, War, Water}
        \spellnotes When the storm has multiple effects in the same round, roll a single attack and compare the result to all relevant defenses.

        \physicalspellnotes
        \miscastyou
    \end{spellfooter}
\end{spellsection}
\begin{dtable*}
    \lcaption{Storm of Vengeance Effects}
    \begin{dtabularx}{\textwidth}{l l l >{\lcol}X l}
        \thead{Rounds} & \thead{Effect} & \thead{Defense} & \thead{Success} & \thead{Failure} \\
        \hline
        Odd (1, 3, 5, 7, 9)   & Lightning  & Reflex    & Electricity damage (foes only) & Half damage \\
        Even (2, 4, 6, 8, 10) & Thunder    & Fortitude & Deafened for 5 rounds & No effect \\
        2, 6, 10              & Hail       & Reflex    & Bludgeoning damage & Half damage \\
        4, 8                  & Acid rain  & None      & Acid damage & \x \\
    \end{dtabularx}
\end{dtable*}

\begin{spellsection}{Stormlord}[6]
    \begin{spellheader}
        \spelldesc{You surround yourself in a whirlwind which deflects ranged attacks and batters your foes.} 
    \end{spellheader}
    \begin{spellcontent}
        \begin{spelltargetinginfo}
            \spelltgt{You}
        \end{spelltargetinginfo}
        \begin{spelleffects}
            \spelleffect Physical ranged attacks against you have a 50\% miss chance. Other attacks that simply work at a distance are not affected.
            \spelldur \durshort
        \end{spelleffects}
    \end{spellcontent}
    \begin{spellsubcontent}
        \begin{spelltargetinginfo}
            \spelltgr{Creature within \rnglong range of you makes a physical attack against you}
            \spelltgt{Triggering creature}
        \end{spelltargetinginfo}
        \begin{spelleffects}
            \begin{spellattack}{Spellpower vs. Fortitude}
                \spellsuccess \spelldamage{6}{bludgeoning}[d6].
                \spellfailure Half damage.
            \end{spellattack}
        \end{spelleffects}
    \end{spellsubcontent}
    \begin{spellfooter}
        \spellinfo{Abjuration/Evocation [Air, Electricity, Shielding]}{Air, Nature}
        \miscastexplode
    \end{spellfooter}
\end{spellsection}

\begin{spellsection}{Strip the Flesh}[6]
    \begin{spellheader}
        \spelldesc{You rend parts of your foe's skin off its body, inflicting grievous wounds and leaving it vulnerable.}
    \end{spellheader}
    \begin{spellcontent}
        \begin{spelltargetinginfo}
            \spelltwocol{\spelltgt{One creature}}{\spellrng{\rngclose}}
        \end{spelltargetinginfo}
        \begin{spelleffects}
            \begin{spellattack}{Spellpower vs. Fortitude}
                \spellsuccess \spelldamage{6}{slashing}. In addition, all damage the target takes is doubled for 5 rounds. This does not apply to the initial damage dealt by this spell.
                \spellcritical As above, except that the doubling of damage lasts for 1 year.
                \spellfailure Half damage, and no additional effects.
            \end{spellattack}
        \end{spelleffects}
    \end{spellcontent}
    \begin{spellfooter}
        \spellinfo{Vivimancy [Flesh, Physical]}{Arcane}
        \spellnotes The doubling of damage can be negated by a Heal check against a DC equal to 10 \add your spellpower.

        \physicalspellnotes
        \miscastrandom
    \end{spellfooter}
\end{spellsection}

\begin{spellsection}{Suggestion}[6]
    \begin{spellheader}
    \end{spellheader}
    \begin{spellcontent}
        \begin{spelltargetinginfo}
            \spelltwocol{\spelltgt{One creature}}{\spellrng{\rngclose}}
            \spellcmp{Verbal only}
        \end{spelltargetinginfo}
        \begin{spelleffects}
            \begin{spellattack}{Spellpower vs. Mental}
                \spellspecial You suggest a course of action that the target could take. The suggestion must not be longer than a couple of sentences. It must be worded in such a manner as to make the activity sound reasonable. Asking the creature to do some obviously harmful act makes the spell fail automatically. You take a \minus5 penalty to the attack roll if the target thinks it is threatened.
                \spellsuccess For 5 rounds, the target is compelled to obey your suggestion. If the suggested activity is completed during that time, the spell's effect ends.
                \spellcritical As above, except that the target will obey the suggestion indefinitely, until it completes its task.
            \end{spellattack}
        \end{spelleffects}
    \end{spellcontent}
    \begin{spellfooter}
        \spellinfo{Enchantment [Auditory, Delusion, Mind, Speech, Subtle]}{Enchantment}
        \spellnotes A very reasonable suggestion can grant a \plus2 or greater bonus on the magic attack.

        \norepeatspellnotes
        \miscastrandom
    \end{spellfooter}
\end{spellsection}

\begin{spellsection}[Mass]{Suggestion}[9]
    \begin{spellheader}
    \end{spellheader}
    \begin{spellcontent}
        \begin{spelltargetinginfo}
            \spelltwocol{\spelltgts{Up to five creatures}}{\spellrng{\rngclose}}
        \end{spelltargetinginfo}
        \begin{spelleffects}
            \spellspecial This spell functions like \spell{suggestion}, except that it affects multiple creatures.
        \end{spelleffects}
    \end{spellcontent}
    \begin{spellfooter}
        \spellinfo{Enchantment [Auditory, Delusion, Mind, Speech, Subtle]}{Enchantment}
        \spellnotes As \spell{suggestion}. All targets must receive the same suggestion.
        \miscastexplode
    \end{spellfooter}
\end{spellsection}

\begin{spellsection}{Summon Monster I}{1}\hypertarget{spell:summon monster}{}
    \begin{spellheader}
    \end{spellheader}
    \begin{spellcontent}
        \begin{spelltargetinginfo}
            \spelltwocol{\spelltime{Full-round action}}{\spellrng{\rngclose}}
        \end{spelltargetinginfo}
        \begin{spelleffects}
            \spelleffect This spell summons an extraplanar creature (typically an outsider, elemental, or magical beast native to another plane). It appears where you designate and acts on your next turn. You must spend a swift action each round to control the creature summoned by this spell. If you do, it attacks your opponents to the best of its ability. You can direct the creature not to attack, to attack particular enemies, or to perform other actions if you can communicate with it. If you do not actively control the creature summoned by this spell, it acts according to its nature.
            \par When you learn this spell, you choose two creatures from the 1st-level list on \tref{Summon Monster List}. You can summon those creatures with this or any other summon monster spell.
            \par A summoned monster cannot summon or otherwise conjure another creature, nor can it use any teleportation or planar travel abilities. Creatures cannot be summoned into an environment that cannot support them.
            \par When you use a summoning spell to summon an air, chaotic, earth, evil, fire, lawful, or water creature, it is a spell of that type.
            \spelldur \durshort \dismissable
        \end{spelleffects}
    \end{spellcontent}
    \begin{spellfooter}
        \spellinfo{Conjuration [Summoning, see text]}{Arcane, Divine}
        \miscastexplode
    \end{spellfooter}
\end{spellsection}

\begin{spellsection}{Summon Monster II}[2]
    \begin{spellheader}
    \end{spellheader}
    \begin{spellcontent}
        \begin{spelltargetinginfo}
            \spelltwocol{\spelllimit{\areamed radius}}{\spellrng{\rngclose}}
        \end{spelltargetinginfo}
        \begin{spelleffects}
            \spellspecial This spell functions like \spell{summon monster I}, except that you can summon one creature from the 2nd-level list or 1d3 creatures of the same kind from the 1st-level list. When you learn this spell, you choose two creatures from the 2nd-level list or lower on \tref{Summon Monster List}. You can summon those creatures with this or any other \spell{summon monster} spell.
            \spelldur \durshort \dismissable
        \end{spelleffects}
    \end{spellcontent}
    \begin{spellfooter}
        \spellinfo{Conjuration [Summoning, see text]}{Arcane, Divine}
        \miscastexplode
    \end{spellfooter}
\end{spellsection}

\begin{spellsection}{Summon Monster III}[3]
    \begin{spellheader}
    \end{spellheader}
    \begin{spellcontent}
        \begin{spelltargetinginfo}
            \spelltwocol{\spelllimit{\areamed radius}}{\spellrng{\rngclose}}
        \end{spelltargetinginfo}
        \begin{spelleffects}
            \spellspecial This spell functions like \spell{summon monster I}, except that you can summon one creature from the 3rd-level list or 1d3 creatures of the same kind from a lower-level list. When you learn this spell, you choose two creatures from the 3rd-level list or lower on \tref{Summon Monster List}. You can summon those creatures with this or any other \spell{summon monster} spell.
            \spelldur \durshort \dismissable
        \end{spelleffects}
    \end{spellcontent}
    \begin{spellfooter}
        \spellinfo{Conjuration [Summoning, see text]}{Arcane, Chaos, Divine, Evil, Good, Law}
        \miscastexplode
    \end{spellfooter}
\end{spellsection}

\begin{spellsection}{Summon Monster IV}[4]
    \begin{spellheader}
    \end{spellheader}
    \begin{spellcontent}
        \begin{spelltargetinginfo}
            \spelltwocol{\spelllimit{\areamed radius}}{\spellrng{\rngclose}}
        \end{spelltargetinginfo}
        \begin{spelleffects}
            \spellspecial This spell functions like \spell{summon monster I}, except that you can summon one creature from the 4th-level list or 1d3 creatures of the same kind from a lower-level list. When you learn this spell, you choose two creatures from the 4th-level list or lower on \tref{Summon Monster List}. You can summon those creatures with this or any other \spell{summon monster} spell.
            \spelldur \durshort \dismissable
        \end{spelleffects}
    \end{spellcontent}
    \begin{spellfooter}
        \spellinfo{Conjuration [Summoning, see text]}{Arcane, Divine}
        \miscastexplode
    \end{spellfooter}
\end{spellsection}

\begin{spellsection}{Summon Monster V}[4]
    \begin{spellheader}
    \end{spellheader}
    \begin{spellcontent}
        \begin{spelltargetinginfo}
            \spelltwocol{\spelllimit{\areamed radius}}{\spellrng{\rngclose}}
        \end{spelltargetinginfo}
        \begin{spelleffects}
            \spellspecial This spell functions like \spell{summon monster I}, except that you can summon one creature from the 5th-level list or 1d3 creatures of the same kind from a lower-level list. When you learn this spell, you choose two creatures from the 5th-level list or lower on \tref{Summon Monster List}. You can summon those creatures with this or any other \spell{summon monster} spell.
            \spelldur \durshort \dismissable
        \end{spelleffects}
    \end{spellcontent}
    \begin{spellfooter}
        \spellinfo{Conjuration [Summoning, see text]}{Air, Arcane, Divine, Earth, Fire, Water}
        \miscastexplode
    \end{spellfooter}
\end{spellsection}

\begin{spellsection}{Summon Monster VI}[6]
    \begin{spellheader}
    \end{spellheader}
    \begin{spellcontent}
        \begin{spelltargetinginfo}
            \spelltwocol{\spelllimit{\areamed radius}}{\spellrng{\rngclose}}
        \end{spelltargetinginfo}
        \begin{spelleffects}
            \spellspecial This spell functions like \spell{summon monster I}, except you can summon one creature from the 6th-level list or 1d3 creatures of the same kind from a lower-level list. When you learn this spell, you choose two creatures from the 6th-level list or lower on \tref{Summon Monster List}. You can summon those creatures with this or any other \spell{summon monster} spell.
            \spelldur \durshort \dismissable
        \end{spelleffects}
    \end{spellcontent}
    \begin{spellfooter}
        \spellinfo{Conjuration [Summoning, see text]}{Arcane, Chaos, Divine, Evil, Good, Law}
        \miscastexplode
    \end{spellfooter}
\end{spellsection}

\begin{spellsection}{Summon Monster VII}[7]
    \begin{spellheader}
    \end{spellheader}
    \begin{spellcontent}
        \begin{spelltargetinginfo}
            \spelltwocol{\spelllimit{\areamed radius}}{\spellrng{\rngclose}}
        \end{spelltargetinginfo}
        \begin{spelleffects}
            \spellspecial This spell functions like \spell{summon monster I}, except that you can summon one creature from the 7th-level list or 1d3 creatures of the same kind from a lower-level list. When you learn this spell, you choose two creatures from the 7th-level list or lower on \tref{Summon Monster List}. You can summon those creatures with this or any other \spell{summon monster} spell.
            \spelldur \durshort \dismissable
        \end{spelleffects}
    \end{spellcontent}
    \begin{spellfooter}
        \spellinfo{Conjuration [Summoning, see text]}{Arcane, Divine}
        \miscastexplode
    \end{spellfooter}
\end{spellsection}

\begin{spellsection}{Summon Monster VIII}[7]
    \begin{spellheader}
    \end{spellheader}
    \begin{spellcontent}
        \begin{spelltargetinginfo}
            \spelltwocol{\spelllimit{\areamed radius}}{\spellrng{\rngclose}}
        \end{spelltargetinginfo}
        \begin{spelleffects}
            \spellspecial This spell functions like \spell{summon monster I}, except that you can summon one creature from the 8th-level list or 1d3 creatures of the same kind from a lower-level list. When you learn this spell, you choose two creatures from the 8th-level list or lower on \tref{Summon Monster List}. You can summon those creatures with this or any other \spell{summon monster} spell.
            \spelldur \durshort \dismissable
        \end{spelleffects}
    \end{spellcontent}
    \begin{spellfooter}
        \spellinfo{Conjuration [Summoning, see text]}{Air, Arcane, Divine, Earth, Fire, Water}
        \miscastexplode
    \end{spellfooter}
\end{spellsection}

\begin{spellsection}{Summon Monster IX}[9]
    \begin{spellheader}
    \end{spellheader}
    \begin{spellcontent}
        \begin{spelltargetinginfo}
            \spelltwocol{\spelllimit{\areamed radius}}{\spellrng{\rngclose}}
        \end{spelltargetinginfo}
        \begin{spelleffects}
            \spellspecial This spell functions like \spell{summon monster I}, except that you can summon one creature from the 9th-level list or 1d3 creatures of the same kind from a lower-level list. When you learn this spell, you choose two creatures from the 9th-level list or lower on \tref{Summon Monster List}. You can summon those creatures with this or any other \spell{summon monster} spell.

            \spelldur \durshort \dismissable
        \end{spelleffects}
    \end{spellcontent}
    \begin{spellfooter}
        \spellinfo{Conjuration [Summoning, see text]}{Arcane, Chaos, Divine, Evil, Good, Law}
        \miscastexplode
    \end{spellfooter}
\end{spellsection}

\begin{dtable!*}
    \lcaption{Summon Monster List}
    \begin{dtabularx}{\textwidth}{>{\lcol}X c >{\lcol}X c >{\lcol}X c}
        \thead{1st Level} &  & \thead{4th Level} &  & Fiendish monstrous spider, Huge & CE \\
        \hline
        Celestial dog & LG & Archon, lantern & LG & Fiendish snake, giant constrictor & CE \\
        Celestial owl & LG & Celestial giant owl & LG &  &  \\
        Celestial giant fire beetle & NG & Celestial giant eagle & CG & \thead{7th Level} &  \\
        Celestial porpoise\fn{1} & NG & Celestial lion & CG & Celestial elephant & LG \\
        Celestial badger & CG & Mephit (any)\fn{2} & N & Avoral (guardinal) & NG \\
        Celestial monkey & CG & Fiendish dire wolf & LE & Celestial baleen whale\fn{1} & NG \\
        Fiendish dire rat & LE & Fiendish giant wasp & LE & Djinni (genie) & CG \\
        Fiendish raven & LE & Fiendish giant praying mantis & NE & Elemental, Huge (any)\fn{2} & N \\
        Fiendish monstrous centipede, Medium & NE & Fiendish shark, Large\fn{1} & NE & Invisible stalker & N \\
        Fiendish monstrous scorpion, Small & NE & Yeth hound & NE & Devil, bone & LE \\
        Fiendish hawk & CE & Fiendish monstrous spider, Large & CE & Fiendish megaraptor & LE \\
        Fiendish monstrous spider, Small & CE & Fiendish snake, Huge viper & CE & Fiendish monstrous scorpion, Huge & \\ NE
        Fiendish octopus\fn{1} & CE & Howler & CE & Babau (demon) & CE \\
        Fiendish snake, Small viper & CE &  &  & Fiendish giant octopus\fn{1} & CE \\
        &  & \thead{5th Level} &  & Fiendish girallon & CE \\
        \thead{2nd Level} &  & Archon, hound & K &  &  \\
        Celestial giant bee & LG & Celestial brown bear & LG &  &  \\
        Celestial giant bombardier beetle & NG & Celestial giant stag beetle & LG & \thead{8th Level} &  \\
        Celestial riding dog & NG & Celestial sea cat\fn{1} & NG & Celestial dire bear & LG \\
        Celestial eagle & CG & Celestial griffon & NG & Celestial cachalot whale\fn{1} & NG \\
        Lemure (devil) & LE & Elemental, Medium (any)\fn{2} & CG & Celestial triceratops & NG \\
        Fiendish squid\fn{1} & LE & Achaierai & N & Lillend & CG \\
        Fiendish wolf & LE & Devil, bearded & LE & Elemental, greater (any)\fn{2} & N \\
        Fiendish monstrous centipede, Large & NE & Fiendish deinonychus & LE & Fiendish giant squid\fn{1} & LE \\
        Fiendish monstrous scorpion, Medium & NE & Fiendish dire ape & LE & Hellcat & LE \\
        Fiendish shark, Medium\fn{1} & NE & Fiendish dire boar & LE & Fiendish monstrous centipede, Colossal & NE \\
        Fiendish monstrous spider, Medium & CE & Fiendish shark, Huge & NE & Fiendish dire tiger & CE \\
        Fiendish snake, Medium viper & CE & Fiendish monstrous scorpion, Large & NE & Fiendish monstrous spider, Gargantuan & CE \\
        &  & Shadow mastiff & NE & Fiendish tyrannosaurus & CE \\
        \thead{3rd Level} &  & Fiendish dire wolverine & NE & Vrock (demon) & CE \\
        Celestial black bear & LG & Fiendish giant crocodile & CE &  &  \\
        Celestial bison & NG & Fiendish tiger & CE &  &  \\
        Celestial dire badger & CG &  &  & \thead{9th Level} &  \\
        Celestial hippogriff & CG & \thead{6th Level} &  & Couatl & LG \\
        Elemental, Small (any)\fn{2} & N & Celestial polar bear & LG & Leonal (guardinal) & NG \\
        Fiendish ape & LE & Celestial orca whale\fn{1} & NG & Celestial roc & CG \\
        Fiendish dire weasel & LE & Bralani (eladrin) & CG & Elemental, elder (any)\fn{2} & N \\
        Hell hound & LE & Celestial dire lion & CG & Devil, barbed & LE \\
        Fiendish snake, constrictor  & LE & Elemental, Large (any)\fn{2} & N & Fiendish dire shark\fn{1} & NE \\
        Fiendish boar & NE & Janni (genie) & N & Fiendish monstrous scorpion, Gargantuan & NE \\
        Fiendish dire bat & NE & Chaos beast & CN & Night hag & NE \\
        Fiendish monstrous centipede, Huge & NE & Devil, chain & LE & Bebilith (demon) & CE \\
        Fiendish crocodile & CE & Xill & LE & Fiendish monstrous spider, Colossal & CE \\
        Dretch (demon) & CE & Fiendish monstrous centipede, Gargantuan & NE & Hezrou (demon) & CE \\
        Fiendish snake, Large viper & CE & Fiendish rhinoceros & NE & & \\
        Fiendish wolverine & CE & Fiendish elasmosaurus\fn{1} & CE & &
    \end{dtabularx}
    1 May be summoned only into an aquatic or watery environment. \\
    2 Each variety must be learned individually.
\end{dtable!*}

\begin{spellsection}{Summon Nature's Ally I}{1}\hypertarget{spell:summon nature's ally}{}
    \begin{spellheader}
    \end{spellheader}
    \begin{spellcontent}
        \begin{spelltargetinginfo}
            \spelltime{Full-round action}
            \spellrng{\rngclose}
        \end{spelltargetinginfo}
        \begin{spelleffects}
            \spelleffect This spell summons a natural creature. It appears where you designate and acts on your next turn. You must spend a swift action each round to control the creature summoned by this spell. If you do, it attacks your opponents to the best of its ability. You can direct the creature not to attack, to attack particular enemies, or to perform other actions if you can communicate with it. If you do not actively control the creature summoned by this spell, it acts according to its nature.
            \par When you learn this spell, you choose two creatures from the 1st-level list on \tref{Summon Nature's Ally List}. You can summon those creatures with this or any other \spell{summon nature's ally} spell.
            \par A summoned monster cannot summon or otherwise conjure another creature, nor can it use any teleportation or planar travel abilities. Creatures cannot be summoned into an environment that cannot support them.
            \par All the creatures on the table are neutral unless otherwise noted.
            \spelldur \durshort \dismissable
        \end{spelleffects}
    \end{spellcontent}
    \begin{spellfooter}
        \spellinfo{Conjuration [Summoning]}{Nature}
        \miscastexplode
    \end{spellfooter}
\end{spellsection}

\begin{spellsection}{Summon Nature's Ally II}[2]
    \begin{spellheader}
    \end{spellheader}
    \begin{spellcontent}
        \begin{spelltargetinginfo}
            \spelltwocol{\spelllimit{\areamed radius}}{\spellrng{\rngclose}}
        \end{spelltargetinginfo}
        \begin{spelleffects}
            \spellspecial This spell functions like \spellindirect{summon nature's ally i}{summon nature's ally I}, except that you can summon one 2nd-level creature or 1d3 1st-level creatures of the same kind. When you learn this spell, you choose two creatures from the 2nd-level list or lower on \tref{Summon Nature's Ally List}. You can summon those creatures with this or any other \spell{summon nature's ally} spell.
            \spelldur \durshort \dismissable
        \end{spelleffects}
    \end{spellcontent}
    \begin{spellfooter}
        \spellinfo{Conjuration [Summoning]}{Nature}
        \miscastexplode
    \end{spellfooter}
\end{spellsection}

\begin{spellsection}{Summon Nature's Ally III}[3]
    \begin{spellheader}
    \end{spellheader}
    \begin{spellcontent}
        \begin{spelltargetinginfo}
            \spelltwocol{\spelllimit{\areamed radius}}{\spellrng{\rngclose}}
        \end{spelltargetinginfo}
        \begin{spelleffects}
            \spellspecial This spell functions like \spellindirect{summon nature's ally i}{summon nature's ally I}, except that you can summon one 3rd-level creature, 1d3 2nd-level creatures of the same kind, or 1d4\plus1 1st-level creatures of the same kind. When you learn this spell, you choose two creatures from the 3rd-level list or lower on \tref{Summon Nature's Ally List}. You can summon those creatures with this or any other \spell{summon nature's ally} spell.
            \spelldur \durshort \dismissable
        \end{spelleffects}
    \end{spellcontent}
    \begin{spellfooter}
        \spellinfo{Conjuration [Summoning]}{Nature, Wild}
        \miscastexplode
    \end{spellfooter}
\end{spellsection}

\begin{spellsection}{Summon Nature's Ally IV}[4]
    \begin{spellheader}
    \end{spellheader}
    \begin{spellcontent}
        \begin{spelltargetinginfo}
            \spelltwocol{\spelllimit{\areamed radius}}{\spellrng{\rngclose}}
        \end{spelltargetinginfo}
        \begin{spelleffects}
            \spellspecial This spell functions like \spellindirect{summon nature's ally i}{summon nature's ally I}, except that you can summon one 4th-level creature or 1d3 creatures of the same kind from a lower-level list. When you learn this spell, you choose two creatures from the 4th-level list or lower on \tref{Summon Nature's Ally List}. You can summon those creatures with this or any other \spell{summon nature's ally} spell.
            \spelldur \durshort \dismissable
        \end{spelleffects}
    \end{spellcontent}
    \begin{spellfooter}
        \spellinfo{Conjuration [Summoning, see text]}{Nature}
        \miscastexplode
    \end{spellfooter}
\end{spellsection}

\begin{spellsection}{Summon Nature's Ally V}[5]
    \begin{spellheader}
    \end{spellheader}
    \begin{spellcontent}
        \begin{spelltargetinginfo}
            \spelltwocol{\spelllimit{\areamed radius}}{\spellrng{\rngclose}}
        \end{spelltargetinginfo}
        \begin{spelleffects}
            \spellspecial This spell functions like \spellindirect{summon nature's ally i}{summon nature's ally I}, except that you can summon one 5th-level creature or 1d3 creatures of the same kind from a lower-level list. When you learn this spell, you choose two creatures from the 5th-level list or lower on \tref{Summon Nature's Ally List}. You can summon those creatures with this or any other \spell{summon nature's ally} spell.
            \spelldur \durshort \dismissable
        \end{spelleffects}
    \end{spellcontent}
    \begin{spellfooter}
        \spellinfo{Conjuration [Summoning, see text]}{Nature}
        \miscastexplode
    \end{spellfooter}
\end{spellsection}

\begin{spellsection}{Summon Nature's Ally VI}[6]
    \begin{spellheader}
    \end{spellheader}
    \begin{spellcontent}
        \begin{spelltargetinginfo}
            \spelltwocol{\spelllimit{\areamed radius}}{\spellrng{\rngclose}}
        \end{spelltargetinginfo}
        \begin{spelleffects}
            \spellspecial This spell functions like \spellindirect{summon nature's ally i}{summon nature's ally I}, except that you can summon one 6th-level creature or 1d3 creatures of the same kind from a lower-level list. When you learn this spell, you choose two creatures from the 6th-level list or lower on \tref{Summon Nature's Ally List}. You can summon those creatures with this or any other \spell{summon nature's ally} spell.
            \spelldur \durshort \dismissable
        \end{spelleffects}
    \end{spellcontent}
    \begin{spellfooter}
        \spellinfo{Conjuration [Summoning, see text]}{Nature, Wild}
        \miscastexplode
    \end{spellfooter}
\end{spellsection}

\begin{spellsection}{Summon Nature's Ally VII}[7]
    \begin{spellheader}
    \end{spellheader}
    \begin{spellcontent}
        \begin{spelltargetinginfo}
            \spelltwocol{\spelllimit{\areamed radius}}{\spellrng{\rngclose}}
        \end{spelltargetinginfo}
        \begin{spelleffects}
            \spellspecial This spell functions like \spellindirect{summon nature's ally i}{summon nature's ally I}, except that you can summon one 7th-level creature or 1d3 creatures of the same kind from a lower-level list. When you learn this spell, you choose two creatures from the 7th-level list or lower on \tref{Summon Nature's Ally List}. You can summon those creatures with this or any other \spell{summon nature's ally} spell.
            \spelldur \durshort \dismissable
        \end{spelleffects}
    \end{spellcontent}
    \begin{spellfooter}
        \spellinfo{Conjuration [Summoning, see text]}{Nature}
        \miscastexplode
    \end{spellfooter}
\end{spellsection}

\begin{spellsection}{Summon Nature's Ally VIII}[8]
    \begin{spellheader}
    \end{spellheader}
    \begin{spellcontent}
        \begin{spelltargetinginfo}
            \spelltwocol{\spelllimit{\areamed radius}}{\spellrng{\rngclose}}
        \end{spelltargetinginfo}
        \begin{spelleffects}
            \spellspecial This spell functions like \spellindirect{summon nature's ally i}{summon nature's ally I}, except that you can summon one 8th-level creature or 1d3 creatures of the same kind from a lower-level list. When you learn this spell, you choose two creatures from the 8th-level list or lower on \tref{Summon Nature's Ally List}. You can summon those creatures with this or any other \spell{summon nature's ally} spell.
            \spelldur \durshort \dismissable
        \end{spelleffects}
    \end{spellcontent}
    \begin{spellfooter}
        \spellinfo{Conjuration [Summoning, see text]}{Nature}
        \miscastexplode
    \end{spellfooter}
\end{spellsection}

\begin{spellsection}{Summon Nature's Ally IX}[9]
    \begin{spellheader}
    \end{spellheader}
    \begin{spellcontent}
        \begin{spelltargetinginfo}
            \spelltwocol{\spelllimit{\areamed radius}}{\spellrng{\rngclose}}
        \end{spelltargetinginfo}
        \begin{spelleffects}
            \spellspecial This spell functions like \spellindirect{summon nature's ally i}{summon nature's ally I}, except that you can summon one 9th-level creature or 1d3 creatures of the same kind from a lower-level list. When you learn this spell, you choose two creatures from the 9th-level list or lower on \tref{Summon Nature's Ally List}. You can summon those creatures with this or any other \spell{summon nature's ally} spell.

            \spelldur \durshort \dismissable
        \end{spelleffects}
    \end{spellcontent}
    \begin{spellfooter}
        \spellinfo{Conjuration [Summoning, see text]}{Nature, Wild}
        \miscastexplode
    \end{spellfooter}
\end{spellsection}

\begin{dtable*}
    \lcaption{Summon Nature's Ally List}
    \begin{dtabularx}{\textwidth}{>{\lcol}X >{\lcol}X >{\lcol}X >{\lcol}X}
        \thead{1st Level} & Eagle, giant [NG] & \thead{5th Level} & \thead{7th Level} \\
        \hline
        Dire rat & Lion & Arrowhawk, adult & Arrowhawk, elder \\
        Eagle (animal) & Owl, giant [NG] & Bear, polar (animal) & Dire tiger \\
        Monkey (animal) & Satyr [CN; without pipes] & Dire lion & Elemental, greater (any)\fn{2} \\
        Octopus\fn{1} (animal) & Shark, Large\fn{1} (animal) & Elasmosaurus\fn{1} (dinosaur) & Djinni (genie) [NG] \\
        Owl (animal) & Snake, constrictor (animal) & Elemental, Large (any)\fn{2} & Invisible stalker \\
        Porpoise\fn{1} (animal) & Snake, Large viper (animal) & Griffon & Pixie\fn{3} (sprite) [NG; with sleep arrows] \\
        Snake, Small viper (animal) & Thoqqua & Janni (genie) & Squid, giant\fn{1} (animal) \\
        Wolf (animal) &  & Rhinoceros (animal) & Triceratops (dinosaur) \\
        & \thead{4th Level} & Satyr [CN; with pipes] & Tyrannosaurus (dinosaur) \\
        \thead{2nd Level} & Arrowhawk, juvenile & Snake, giant constrictor (animal) & Whale, cachalot\fn{1} (animal) \\
        Bear, black (animal) & Bear, brown (animal) & Nixie (sprite) & Xorn, elder \\
        Crocodile (animal) & Crocodile, giant (animal) & Tojanida, adult\fn{1} &  \\
        Dire badger & Deinonychus (dinosaur) & Whale, orca\fn{1} (animal) & \thead{8th Level} \\
        Dire bat & Dire ape &  & Dire shark\fn{1} \\
        Elemental, Small (any)\fn{2} & Dire boar & \thead{6th Level} & Roc \\
        Hippogriff & Dire wolverine & Dire bear & Salamander, noble [NE] \\
        Shark, Medium\fn{1} (animal) & Elemental, Medium (any)\fn{2} & Elemental, Huge (any)\fn{2} & Tojanida, elder \\
        Snake, Medium viper (animal) & Salamander, flamebrother [NE] & Elephant (animal) &  \\
        Squid\fn{1} (animal) & Sea cat\fn{1} & Girallon & \thead{9th Level} \\
        Wolverine (animal) & Shark, Huge\fn{1} (animal) & Megaraptor (dinosaur) & Elemental, elder \\
        & Snake, Huge viper (animalo) & Octopus, giant\fn{1} (animal) & Grig [NG; with fiddle] (sprite) \\
        \thead{3rd Level} & Tiger (animal) & Pixie\fn{3} (sprite) [NG; no special arrows] & Pixie\fn{4} (sprite) [NG; with sleep and memory loss arrows] \\
        Ape (animal) & Tojanida, juvenile\fn{1} & Salamander, average [NE] & Unicorn, celestial charger \\
        Dire weasel & Unicorn [CG] & Whale, baleen\fn{1} &  \\
        Dire wolf & Xorn, minor & Xorn, average & 
    \end{dtabularx}
    1 May be summoned only into an aquatic or watery environment. \\
    2 Each variety must be learned individually. \\
    3 Can't cast irresistible dance \\
    4 Can cast irresistible dance \\
\end{dtable*}

\begin{spellsection}{Summon Nature's Army}[8]
    \begin{spellheader}
    \end{spellheader}
    \begin{spellcontent}
        \begin{spelltargetinginfo}
            \spelltwocol{\spelllimit{\areamed radius}}{\spellrng{\rngclose}}
        \end{spelltargetinginfo}
        \begin{spelleffects}
            \spellspecial This spell functions like \spellindirect{summon nature's ally i}{summon nature's ally I}, except that you can summon up to one creature per spellpower from the 4th-level list or lower.
            \par When you learn this spell, you choose a creature from the 4th-level list or lower on the Summon Nature's Ally table. That is the only creature you can summon with this spell.
            \spelldur \durshort \dismissable
        \end{spelleffects}
    \end{spellcontent}
    \begin{spellfooter}
        \spellinfo{Conjuration [Summoning]}{Nature, Wild}
        \miscastexplode
    \end{spellfooter}
\end{spellsection}

\begin{spellsection}{Sunbeam}[4]
    \begin{spellheader}
        \spelldesc{You evoke a dazzling beam of intense light, blinding your foes with the power of the sun itself.}
    \end{spellheader}
    \begin{spellcontent}
        \begin{spelltargetinginfo}
            \spellburst{\arealarge line, 10 ft. wide}
            \spelltgts{Everything in the area}
        \end{spelltargetinginfo}
        \begin{spelleffects}
            \begin{spellattack}{Spellpower vs. Reflex}
                \spellspecial You gain a \plus5 bonus to attack against creatures vulnerable to sunlight.
                \spellsuccess \spelldamage{4}{solar}[d6]
                \spellcritical If the target is vulnerable to sunlight, it is also \blinded for 5 rounds.
                \spellfailure Half damage, and no additional effects.
            \end{spellattack}
        \end{spelleffects}
    \end{spellcontent}
    \begin{spellfooter}
        \spellinfo{Evocation [Light]}{Nature}
        \spellnotes This light is considered natural sunlight for the purpose of effects which depend on sunlight.
        \miscastexplode
    \end{spellfooter}
\end{spellsection}

\begin{spellsection}{Sunburst}[6]
    \begin{spellheader}
        \spelldesc{You cause a globe of searing radiance to explode silently from a point you select.}
    \end{spellheader}
    \begin{spellcontent}
        \begin{spelltargetinginfo}
            \spelltwocol{\spellburst{\areamed radius}}{\spellrng{\rngmed}}
            \spelltgts{Everything in the area}
        \end{spelltargetinginfo}
        \begin{spelleffects}
            \begin{spellattack}{Spellpower vs. Reflex}
                \spellspecial You gain a \plus5 bonus to attack against creatures vulnerable to sunlight.
                \spellsuccess \spelldamage{6}{solar}[d6]
                \spellcritical If the target is vulnerable to sunlight, it is also \blinded for 5 rounds.
                \spellfailure Half damage, and no additional effects.
            \end{spellattack}
        \end{spelleffects}
    \end{spellcontent}
    \begin{spellfooter}
        \spellinfo{Evocation [Light]}{Nature}
        \spellnotes This light is considered natural sunlight for the purpose of effects which depend on sunlight.
        \miscastyou
    \end{spellfooter}
\end{spellsection}

\pdfbookmark[2]{T}{SpellDescriptionsT}

\begin{spellsection}{Telepathy}[5]
    \begin{spellheader}
    \end{spellheader}
    \begin{spellcontent}
        \begin{spelltargetinginfo}
            \spelltgt{You}
        \end{spelltargetinginfo}
        \begin{spelleffects}
            \spelleffect You gain telepathy out to a range of 100 feet. This allows you to send mental messages to any creature within range that has a language. Non-telepathic creatures can reply mentally to your messages, but they cannot initiative a telepathic conversation with you.

            You can address multiple creatures at once with telepathy, but maintaining separate mental conversations is just as difficult as simultaneously speaking and listening to multiple creatures at the same time. 
            \spelldur \durlong
        \end{spelleffects}
    \end{spellcontent}
    \begin{spellfooter}
        \spellinfo{Divination/Transmutation [Augment]}{Arcane}
        \miscastexplode
    \end{spellfooter}
\end{spellsection}

\begin{spellsection}{Temporal Stasis}[7]
    \begin{spellheader}
    \end{spellheader}
    \begin{spellcontent}
        \begin{spelltargetinginfo}
            \spelltwocol{\spelltgt{One creature}}{\spellrng{\rngmed}}
        \end{spelltargetinginfo}
        \begin{spelleffects}
            \begin{spellattack}{Spellpower vs. Mental}
                \spellsuccess The target is placed in a state of suspended animation for 5 rounds. Time ceases to flow for it, and it cannot be altered or moved by any effect.
                \spellcritical As above, except that the effect is permanent.
                \spellfailure The targes moves at half speed for 5 rounds.
            \end{spellattack}
        \end{spelleffects}
    \end{spellcontent}
    \begin{spellfooter}
        \spellinfo{Transmutation [Temporal]}{Arcane}
        \miscastrandom
    \end{spellfooter}
\end{spellsection}

\begin{spellsection}{Time Stop}[9]
    \begin{spellheader}
        \spelldesc{This spell seems to make time cease to flow for everyone but you. In fact, you step into an alternate timestream, causing you to speed up so greatly that all other creatures seem frozen, though they are actually still moving at their normal speeds.}
    \end{spellheader}
    \begin{spellcontent}
        \begin{spelltargetinginfo}
            \spelltime{Full-round action}
        \end{spelltargetinginfo}
        \begin{spelleffects}
            % exception for Temporal effects?
            \spelleffect You can take 1d3\plus1 rounds of actions immediately. During this time, all other creatures and objects are fixed in time, and cannot be moved or altered by any effect. You can still affect yourself and create areas or new effects, such with \spell{fog cloud} or \spell{summon monster}.

            You are still vulnerable to danger, such as from heat or dangerous gases. However, you cannot be detected by any means while you travel.
        \end{spelleffects}
    \end{spellcontent}
    \begin{spellfooter}
        \spellinfo{Transmutation [Temporal]}{Arcane}
        \spellnotes Spells active on you have their normal effects, including decreasing their remaining duration as appropriate, but spells active on other creatures have no effects and do not decrease in remaining duration.

        You cannot enter an area protected by an \spell{antimagic field} while under the effect of this spell.

        Most spellcasters use the additional time to improve their defenses or flee from combat. 
        \miscastexplode
    \end{spellfooter}
\end{spellsection}

\begin{spellsection}{Totemic Mind}[2]
    \begin{spellheader}
        \spelldesc{You grant your ally the mental prowess of a totem animal.}
    \end{spellheader}
    \begin{spellcontent}
        \begin{spelltargetinginfo}
            \spelltwocol{\spelltgt{One creature}}{\spellrng{\rngtouch}}
        \end{spelltargetinginfo}
        \begin{spelleffects}
            \spelleffect The target gains a \plus2 enhancement bonus to a mental attribute: Intelligence, Perception, or Willpower. This bonus cannot increase the target's attribute above your spellpower.
            \spelldur \durpersonallong
        \end{spelleffects}
    \end{spellcontent}
    \begin{spellfooter}
        \spellinfo{Transmutation (Augment)}{Arcane, Divine, Nature}
        \spellnotes Hit points gained by a temporary increase in Willpower are not temporary hit points. They go away when the target's Willpower drops back to normal, and are not lost first as temporary hit points are.
    \end{spellfooter}
\end{spellsection}

\begin{spellsection}[Greater]{Totemic Mind}[5]
    \begin{spellheader}
    \end{spellheader}
    \begin{spellcontent}
        \begin{spelltargetinginfo}
            \spelltwocol{\spelltgt{One creature}}{\spellrng{\rngtouch}}
        \end{spelltargetinginfo}
        \begin{spelleffects}
            \spelleffect The target's mind improves, as \spell{totemic mind}, except that it gains a \plus4 enhancement bonus. Alternately, you can grant the target a \plus2 enhancement bonus to all its mental attributes.
            \spelldur \durpersonallong
        \end{spelleffects}
    \end{spellcontent}
    \begin{spellfooter}
        \spellinfo{Transmutation (Augment)}{Arcane, Divine, Nature}
    \end{spellfooter}
\end{spellsection}

\begin{spellsection}[Mass]{Totemic Mind}[4]
    \begin{spellheader}
    \end{spellheader}
    \begin{spellcontent}
        \begin{spelltargetinginfo}
            \spelltwocol{\spelltgts{Up to five creatures}}{\spellrng{\rngmed}}
        \end{spelltargetinginfo}
        \begin{spelleffects}
            \spellspecial This spell functions like \spell{totemic mind}, except that it affects multiple creatures.
            \spelldur \durshort
        \end{spelleffects}
    \end{spellcontent}
    \begin{spellfooter}
        \spellinfo{Transmutation (Augment)}{Arcane, Divine, Nature}
        \spellnotes All affected creatures must gain a bonus to the same attribute.
    \end{spellfooter}
\end{spellsection}

\begin{spellsection}{Totemic Power}[2]
    \begin{spellheader}
        \spelldesc{You grant your ally the physical prowess of a totem animal.}
    \end{spellheader}
    \begin{spellcontent}
        \begin{spelltargetinginfo}
            \spelltwocol{\spelltgt{One creature}}{\spellrng{\rngtouch}}
        \end{spelltargetinginfo}
        \begin{spelleffects}
            \spelleffect The target gains a \plus2 enhancement bonus to a physical attribute: Strength, Dexterity, or Constitution. This bonus cannot increase the target's attribute above your spellpower.
            \spelldur \durpersonallong
        \end{spelleffects}
    \end{spellcontent}
    \begin{spellfooter}
        \spellinfo{Transmutation (Augment)}{Arcane, Divine, Nature, Strength}
        \spellnotes Hit points gained by a temporary increase in Constitution are not temporary hit points. They go away when the target's Constitution drops back to normal, and are not lost first as temporary hit points are.
    \end{spellfooter}
\end{spellsection}

\begin{spellsection}[Greater]{Totemic Power}[5]
    \begin{spellheader}
    \end{spellheader}
    \begin{spellcontent}
        \begin{spelltargetinginfo}
            \spelltwocol{\spelltgt{One creature}}{\spellrng{\rngtouch}}
        \end{spelltargetinginfo}
        \begin{spelleffects}
            \spelleffect The target's body improves, as \spell{totemic power}, except that it gains a \plus4 enhancement bonus. Alternately, you can grant the target a \plus2 enhancement bonus to all its physical attributes.
            \spelldur \durpersonallong
        \end{spelleffects}
    \end{spellcontent}
    \begin{spellfooter}
        \spellinfo{Transmutation (Augment)}{Arcane, Divine, Nature, Strength}
    \end{spellfooter}
\end{spellsection}

\begin{spellsection}[Mass]{Totemic Power}[4]
    \begin{spellheader}
    \end{spellheader}
    \begin{spellcontent}
        \begin{spelltargetinginfo}
            \spelltwocol{\spelltgts{Up to five creatures}}{\spellrng{\rngmed}}
        \end{spelltargetinginfo}
        \begin{spelleffects}
            \spellspecial This spell functions like \spell{totemic power}, except that it affects multiple creatures.
            \spelldur \durshort
        \end{spelleffects}
    \end{spellcontent}
    \begin{spellfooter}
        \spellinfo{Transmutation (Augment)}{Arcane, Divine, Nature}
        \spellnotes All affected creatures must gain a bonus to the same attribute.
    \end{spellfooter}
\end{spellsection}

\begin{spellsection}{Transmute Any Object}[9]
    \begin{spellheader}
    \end{spellheader}
    \begin{spellcontent}
        \begin{spelltargetinginfo}
            \spellrng{\rngmed}
        \end{spelltargetinginfo}
        \begin{spelleffects}
            \spellspecial This spell can be used to duplicate the effects of \spell{fabricate}, \spell{passwall}, \spell{shape metal}, \spell{shape stone}, \spell{shape wood}, \spell{transmute flesh and stone}, or \spell{wall of stone}. The object or creature to be transformed must meet any requirements of the spell being duplicated, other than range.
        \end{spelleffects}
    \end{spellcontent}
    \begin{spellfooter}
        \spellinfo{Transmutation [Alteration]}{Arcane}
        \miscastexplode
    \end{spellfooter}
\end{spellsection}

\begin{spellsection}{Transmute Flesh and Stone}[6]
    \begin{spellheader}
    \end{spellheader}
    \begin{spellcontent}
        \begin{spelltargetinginfo}
            \spellspecial This spell has two versions: transmuting flesh into stone, and transmuting stone into flesh. Its effects depend on which version is chosen.
        \end{spelltargetinginfo}
    \end{spellcontent}
    \begin{spellsubcontent}
        \begin{spelltargetinginfo}
            \spelltwocol{\spelltgt{One creature (Huge or smaller)}}{\spellrng{\rngmed}}
        \end{spelltargetinginfo}
        \begin{spelleffects}
            \spellspecial If the target is not made of flesh (such as a golem), it is unaffected.
            \begin{spellattack}{Spellpower vs. Fortitude}
                \spellsuccess \spelldamage{6}{physical}. For the next 5 rounds, if the target has no hit points remaining at the end of the round, it becomes \petrified along with its equipment.
                \spellfailure Half damage, and no additional effects.
            \end{spellattack}
        \end{spelleffects}
    \end{spellsubcontent}
    \begin{spellsubcontent}
        \begin{spelltargetinginfo}
            \spelltgt{One petrified creature (Huge or smaller)}
        \end{spelltargetinginfo}
        \begin{spelleffects}
            \spelleffect The target is restored to its normal state, including its equipment. Stone which was not originally a petrified creature is unaffected.
        \end{spelleffects}
    \end{spellsubcontent}
    \begin{spellfooter}
        \spellinfo{Transmutation [Alteration]}{Arcane, Earth}
        \miscastrandom
    \end{spellfooter}
\end{spellsection}

\begin{spellsection}{Tree Shape}[2]
    \begin{spellheader}
    \end{spellheader}
    \begin{spellcontent}
        \begin{spelltargetinginfo}
            \spelltgt{You}
        \end{spelltargetinginfo}
        \begin{spelleffects}
            \spelleffect You transform into a Large tree, shrub, or dead tree trunk. In this form, you are effectively \paralyzed, but you can see around you in any direction as if you were in your normal form.
            \spelldur \durext \dismissable
        \end{spelleffects}
    \end{spellcontent}
    \begin{spellfooter}
        \spellinfo{Transmutation [Alteration]}{Nature}
        \spellnotes You can sleep comfortably in this form.
        \miscastexplode
    \end{spellfooter}
\end{spellsection}

\begin{spellsection}{Tremorsense}[1]
    \begin{spellheader}
    \end{spellheader}
    \begin{spellcontent}
        \begin{spelltargetinginfo}
            \spellquicktargeting{One creature}{\rngtouch}
        \end{spelltargetinginfo}
        \begin{spelleffects}
            \spelleffect The target gains the tremorsense ability with a range of 50 feet. If it is touching a surface, it can automatically pinpoint the location of anything within 50 feet that is in contact with the surface, including inanimate objects.
            \spelldur \durpersonallong
        \end{spelleffects}
    \end{spellcontent}
    \begin{spellfooter}
        \spellinfo{Transmutation [Augment]}{Nature, Earth}
        \spellnotes Tremorsense functions on surfaces of any kind, regardless of lighting conditions.
        \miscastexplode
    \end{spellfooter}
\end{spellsection}

\begin{spellsection}{True Seeing}[7]
    \begin{spellheader}
        \spelldesc{You grant your ally the ability to see all things as they actually are.}
    \end{spellheader}
    \begin{spellcontent}
        \begin{spelltargetinginfo}
            \spellquicktargeting{One creature}{\rngtouch}
        \end{spelltargetinginfo}
        \begin{spelleffects}
            \spelleffect The target sees through normal and magical darkness, sees the truth behind visual figments and glamers, and sees the true form of polymorphed, changed, or transmuted things. In addition, the target can see into the Ethereal Plane from the Material Plane. The effect extends out to \rngmed range.
            \spelldur \durshort
        \end{spelleffects}
    \end{spellcontent}
    \begin{spellfooter}
        \spellinfo{Divination/Transmutation [Augment]}{Arcane, Divine, Knowledge}
        \spellnotes This spell does not negate concealment, including that caused by fog and the like. It does not help against mundane disguises or concealed objects or creatures. In addition, the spell's effects cannot be further enhanced with known magic, so the benefits do not apply when seeing through a scrying effect or similar vision enhancements.
        \miscastexplode
    \end{spellfooter}
\end{spellsection}

\begin{spellsection}{True Strike}[1]
    \begin{spellheader}
        \spelldesc{You grant your ally a temporary, intuitive insight into the immediate future during their next attack.}
    \end{spellheader}
    \begin{spellcontent}
        \begin{spelltargetinginfo}
            \spellquicktargeting{One creature}{\rngmed}
        \end{spelltargetinginfo}
        \begin{spelleffects}
            \spelleffect The target gains an offensive legend point.
            \spelldur 1 round
        \end{spelleffects}
    \end{spellcontent}
    \begin{spellfooter}
        \spellinfo{Divination}{Arcane}
        \spellnotes After casting this spell, you cannot cast it again for 5 rounds.
        \miscastrandom
    \end{spellfooter}
\end{spellsection}

\pdfbookmark[2]{U-Z}{SpellDescriptionsU-Z}

\begin{spellsection}{Unholy Aura}[8]
    \begin{spellheader}
        \spelldesc{You shield your allies with malevolent darkness, protecting them from good foes.}
    \end{spellheader}
    \begin{spellcontent}
        \begin{spelltargetinginfo}
            \spelltwocol{\spellrng{\rngclose}}{\spelltgts{Up to five creatures}}
        \end{spelltargetinginfo}
        \begin{spelleffects}
            \spelleffect The target gains spell resistance against good spells and spells cast by good creatures.
            \spelldur \durshort \dismissable
        \end{spelleffects}
    \end{spellcontent}
    \begin{spellsubcontent}
        \begin{spelltargetinginfo}
            \spelltgr{Whenever a good creature within 30 feet of the target makes a physical attack against it}
            \spelltgt{The attacking creature}
        \end{spelltargetinginfo}
        \begin{spelleffects}
            \begin{spellattack}{Spellpower vs. Mental}
                \spellsuccess \spelldamage{9}{divine}[d6]
            \end{spellattack}
        \end{spelleffects}
    \end{spellsubcontent}
    \begin{spellfooter}
        \spellinfo{Abjuration [Evil, Retributive, Shielding]}{Divine, Evil}
        \miscastexplode
    \end{spellfooter}
\end{spellsection}

\begin{spellsection}{Unholy Blight}[3]
    \begin{spellheader}
    \end{spellheader}
    \begin{spellcontent}
        \begin{spelltargetinginfo}
            \spelltwocol{\spelltgt{One nonevil creature}}{\spellrng{\rngmed}}
        \end{spelltargetinginfo}
        \begin{spelleffects}
            \begin{spellattack}{Spellpower vs. Mental}
                \spellsuccess \spelldamage{3}{divine}.
                \spellcritical As above, and the target is \staggered for 5 rounds.
                \spellfailure Half damage, and no additional effects.
            \end{spellattack}
        \end{spelleffects}
    \end{spellcontent}
    \begin{spellfooter}
        \spellinfo{Evocation [Evil]}{Evil}
        \miscastrandom
    \end{spellfooter}
\end{spellsection}

\begin{spellsection}{Unliving Eyes}[5]
    \begin{spellheader}
    \end{spellheader}
    \begin{spellcontent}
        \begin{spelltargetinginfo}
            \spelltwocol{\spelltgt{One creature}}{\spellrng{Touch}}
        \end{spelltargetinginfo}
        \begin{spelleffects}
            \spelleffect The target gains the ability to ``see'' any living creatures and their equipment within 30 feet perfectly, regardless of lighting conditions, invisibility, or any other means of concealment. This cannot detect living creatures through solid walls, however.

            If the target is undead, the range of the vision is increased to 50 feet.
            \spelldur \durpersonallong
        \end{spelleffects}
    \end{spellcontent}
    \begin{spellfooter}
        \spellinfo{Transmutation/Vivimancy [Augment, Life]}{Arcane}
        \miscastexplode
    \end{spellfooter}
\end{spellsection}

\begin{spellsection}{Unliving Heart}[1]
    \begin{spellheader}
        \spelldesc{You harness the power of unlife to grant yourself a limited ability to avoid death.}
    \end{spellheader}
    \begin{spellcontent}
        \begin{spelltargetinginfo}
            \spelltgt{You}
        \end{spelltargetinginfo}
        \begin{spelleffects}
            \spelleffect You gain temporary hit points equal to your spellpower. If you take life damage, you lose all temporary hit points provided by this spell before applying the damage.

            In addition, you are treated as being undead for the purpose of spells or abilities which affect undead. This can cause some unintelligent undead, such as skeletons and zombies, to avoid attacking you.
            \spelldur \durlong
        \end{spelleffects}
    \end{spellcontent}
    \begin{spellfooter}
        \spellinfo{Vivimancy [Life]}{Death, Vivimancy}
        \miscastexplode
    \end{spellfooter}
\end{spellsection}

\begin{spellsection}{Ventriloquism}[1]
    \begin{spellheader}
    \end{spellheader}
    \begin{spellcontent}
        \begin{spelltargetinginfo}
            \spellrng{\rngmed}
            \spellcmp{Somatic only}
        \end{spelltargetinginfo}
        \begin{spelleffects}
            \spelleffect Your voice (or any sound that you can normally make vocally) originates from another location within range. As a swift action, you can change the apparent origin of your voice. If you move out of range of your designated location, the sound of your voice comes from your own mouth as normal.
            \spelldur \durshort \dismissable
        \end{spelleffects}
    \end{spellcontent}
    \begin{spellfooter}
        \spellinfo{Illusion [Figment]}{Arcane, Trickery}
        \miscastexplode
    \end{spellfooter}
\end{spellsection}

\begin{spellsection}{Wail of the Banshee}[9]
    \begin{spellheader}
        \spelldesc{You emit a terrible scream that kills anyone that hears it.}
    \end{spellheader}
    \begin{spellcontent}
        \begin{spelltargetinginfo}
            \spellburst{\areamed radius centered on you}
            \spelltgts{Everything in the area}
            \spellcmp{Verbal only}
        \end{spelltargetinginfo}
        \begin{spelleffects}
            \begin{spellattack}{Spellpower vs. Fortitude}
                \spellsuccess \spelldamage{9}{sonic}[d6]. If the target is living and has no hit points remaining, it dies.
            \end{spellattack}
        \end{spelleffects}
    \end{spellcontent}
    \begin{spellfooter}
        \spellinfo{Vivimancy [Auditory, Death]}{Death, Vivimancy}
        \miscastexplode
    \end{spellfooter}
\end{spellsection}

\begin{spellsection}{Wall of Fire}[2]
    \begin{spellheader}
    \end{spellheader}
    \begin{spellcontent}
        \begin{spelltargetinginfo}
            \spelltwocol{\spellzone{20 ft. wall, 10 ft. high}}{\spellrng{\rngmed}}
        \end{spelltargetinginfo}
        \begin{spelleffects}
            \spelleffect This spell creates a wall made of fire.
            \spelldur \durshort
        \end{spelleffects}
    \end{spellcontent}
    \begin{spellsubcontent}
        \begin{spelltargetinginfo}
            \spelltwocol{\spelltgr{A creature passes through the wall}}{\spelltgt{The moving creature}}
        \end{spelltargetinginfo}
        \begin{spelleffects}
            \begin{spellattack}{Spellpower vs. Reflex}
                \spellsuccess \spelldamage{2}{fire}[d6].
                \spellfailure Half damage.
            \end{spellattack}
        \end{spelleffects}
    \end{spellsubcontent}
    \begin{spellfooter}
        \spellinfo{Evocation [Destructive, Fire, Wall]}{Evocation,  Fire, Nature}
        \spellnotes Any part of the wall takes 5 cold damage in a single round is extinguished.

        This spell can be made permanent with a \spell{permanency} ritual. A permanent \spell{wall of fire} that is extinguished by cold damage becomes inactive for 10 minutes, then reforms at normal strength.
        \miscastexplode
    \end{spellfooter}
\end{spellsection}

\begin{spellsection}[Greater]{Wall of Fire}[5]
    \begin{spellheader}
    \end{spellheader}
    \begin{spellcontent}
        \begin{spelltargetinginfo}
            \spelltwocol{\spellzone{100 ft. wall, 20 ft. high \shapeable}}{\spellrng{\rngmed}}
        \end{spelltargetinginfo}
        \begin{spelleffects}
            \spellspecial This spell functions like \spell{wall of fire}, except that it is larger and can be shaped. The damage dealt to a creature passing through the wall is \spelldamage{5}{fire}[d6].
            \spelldur \durshort
        \end{spelleffects}
    \end{spellcontent}
    \begin{spellfooter}
        \spellinfo{Evocation [Destructive, Fire, Wall]}{Arcane, Nature, Fire}
        \spellnotes Any part of the wall takes 10 cold damage in a single round is extinguished.

        This spell can be made permanent with a \spell{permanency} ritual. A permanent \spell{greater wall of fire} that is extinguished by cold damage becomes inactive for 10 minutes, then reforms at normal strength.
        \miscastexplode
    \end{spellfooter}
\end{spellsection}

\begin{spellsection}{Wall of Force}[6]
    \begin{spellheader}
    \end{spellheader}
    \begin{spellcontent}
        \begin{spelltargetinginfo}
            \spelltwocol{\spellzone{100 ft. solid wall, 10 ft. high}}{\spellrng{\rngmed}}
        \end{spelltargetinginfo}
        \begin{spelleffects}
            \spelleffect This spell creates an invisible wall made of force. A 5-foot square of wall has 5 hit points per spellpower, and hardness equal to your spellpower.
            \spelldur \durshort \dismissable
        \end{spelleffects}
    \end{spellcontent}
    \begin{spellfooter}
        \spellinfo{Evocation [Force, Physical, Wall]}{Evocation}
        \spellnotes \forcespellnotes

        This spell can be made permanent with a \spell{permanency} ritual.
        \miscastexplode
    \end{spellfooter}
\end{spellsection}

\begin{spellsection}{Wall of Thorns}[3]
    \begin{spellheader}
    \end{spellheader}
    \begin{spellcontent}
        \begin{spelltargetinginfo}
            \spelltwocol{\spellzone{20 ft. line, 5 ft. wide, 10 ft. high \shapeable}}{\spellrng{\rngmed}}
        \end{spelltargetinginfo}
        \begin{spelleffects}
            \spelleffect This spell creates a thicket of thorns in the area. A 5-foot cube of wall has 10 hit points per spellpower. Moving into or out of a square in the area costs 20 feet of movement. The wall can be created where creatures are.

            The wall provides total cover against attacks through the wall. A creature in the wall has cover from attacks on either side of the wall.
            \spelldur \durshort
        \end{spelleffects}
    \end{spellcontent}
    \begin{spellsubcontent}
        \begin{spelltargetinginfo}
            \spelltwocol{\spelltgr{A creature enters or exits a square in the area}}{\spelltgt{The moving creature}}
        \end{spelltargetinginfo}
        \begin{spelleffects}
            \begin{spellattack}{Spellpower vs. Reflex}
                \spellsuccess \spelldamage{2}{piercing}[d6]
            \end{spellattack}
        \end{spelleffects}
    \end{spellsubcontent}
    \begin{spellfooter}
        \spellinfo{Conjuration [Creation, Physical, Wall]}{Nature, Wild}
        \spellnotes A \spell{wall of thorns} can be breached by slow work with edged weapons or fire. It has hardness 8 and 30 hit points per square foot of thickness.

        \physicalspellnotes
        \miscastexplode
    \end{spellfooter}
\end{spellsection}

\begin{spellsection}[Greater]{Wall of Thorns}[6]
    \begin{spellheader}
    \end{spellheader}
    \begin{spellcontent}
        \begin{spelltargetinginfo}
            \spelltwocol{\spellzone{100 ft. line, 5 ft. wide, 20 ft. high \shapeable}}{\spellrng{\rngmed}}
        \end{spelltargetinginfo}
        \begin{spelleffects}
            \spellspecial This spell functions like \spell{wall of thorns}, except that it is larger and can be shaped. The damage dealt to a creature passing through the wall is \spelldamage{6}{piercing}[d6].
            \spelldur \durshort
        \end{spelleffects}
    \end{spellcontent}
    \begin{spellsubcontent}
        \begin{spelltargetinginfo}
            \spelltwocol{\spelltgr{A creature enters or exits a square in the area}}{\spelltgt{The moving creature}}
        \end{spelltargetinginfo}
        \begin{spelleffects}
            \begin{spellattack}{Spellpower vs. Reflex}
                \spellsuccess \spelldamage{2}{piercing}[d6]
            \end{spellattack}
        \end{spelleffects}
    \end{spellsubcontent}
    \begin{spellfooter}
        \spellinfo{Conjuration [Creation, Physical, Wall]}{Nature, Wild}
        \spellnotes A \spell{wall of thorns} can be breached by slow work with edged weapons or fire. It has hardness 8 and 30 hit points per square foot of thickness.

        \physicalspellnotes
        \miscastexplode
    \end{spellfooter}
\end{spellsection}

\begin{spellsection}{Water Walk}[3]
    \begin{spellheader}
    \end{spellheader}
    \begin{spellcontent}
        \begin{spelltargetinginfo}
            \spelltwocol{\spelltgts{One creature}}{\spellrng{Touch}}
        \end{spelltargetinginfo}
        \begin{spelleffects}
            \spelleffect The target can tread on any liquid as if it were firm ground. Mud, oil, snow, quicksand, running water, ice, and even lava can be traversed easily, since the target's feet hover an inch or two above the surface.
            \par If the target is underwater, it rises toward the surface at 60 feet per round it can stand on it.
            \spelldur \durlong \dismissable
        \end{spelleffects}
    \end{spellcontent}
    \begin{spellfooter}
        \spellinfo{Transmutation [Augment, Water]}{Nature, Water}
        \miscastexplode
    \end{spellfooter}
\end{spellsection}

\begin{spellsection}{Waves of Fatigue}[4]
    \begin{spellheader}
    \end{spellheader}
    \begin{spellcontent}
        \begin{spelltargetinginfo}
            \spellburst{\arealarge cone}
            \spelltgts{All creatures in the area}
        \end{spelltargetinginfo}
        \begin{spelleffects}
            \spelleffect The target is \fatigued for 5 rounds.
        \end{spelleffects}
    \end{spellcontent}
    \begin{spellfooter}
        \spellinfo{Vivimancy [Flesh]}{Arcane, Death}
        \miscastexplode
    \end{spellfooter}
\end{spellsection}

\begin{spellsection}[Greater]{Waves of Fatigue}[7]
    \begin{spellheader}
    \end{spellheader}
    \begin{spellcontent}
        \begin{spelltargetinginfo}
            \spellburst{\arealarge cone}
            \spelltgts{All creatures in the area}
        \end{spelltargetinginfo}
        \begin{spelleffects}
            \begin{spellattack}{Spellpower vs. Fortitude}
                \spellsuccess The target is \exhausted for 5 rounds.
                \spellfailure The target is \fatigued for 5 rounds.
            \end{spellattack}
        \end{spelleffects}
    \end{spellcontent}
    \begin{spellfooter}
        \spellinfo{Vivimancy [Flesh]}{Arcane, Death}
        \miscastexplode
    \end{spellfooter}
\end{spellsection}

\begin{spellsection}[Lesser]{Waves of Fatigue}[1]
    \begin{spellheader}
    \end{spellheader}
    \begin{spellcontent}
        \begin{spelltargetinginfo}
            \spellburst{\areamed cone}
            \spelltgts{All creatures in the area}
        \end{spelltargetinginfo}
        \begin{spelleffects}
            \spelleffect The target is \fatigued for 5 rounds.
        \end{spelleffects}
    \end{spellcontent}
    \begin{spellfooter}
        \spellinfo{Vivimancy [Flesh]}{Arcane, Death, War}
        \miscastexplode
    \end{spellfooter}
\end{spellsection}

\begin{spellsection}{Web}[4]
    \begin{spellheader}
        \spelldesc{You create a many-layered mass of strong, stricky strands that trap creatures caught within them. The strands are similar to spider webs, but larger and tougher.}
    \end{spellheader}
    \begin{spellcontent}
        \begin{spelltargetinginfo}
            \spelltwocol{\spellzone{\areamed radius}}{\spellrng{\rngclose}}
        \end{spelltargetinginfo}
        \begin{spelleffects}
            \spelleffect The area is filled with webs, causing it to be treated as difficult terrain. The webs are thick and strong, but too widely spaced to significantly obscure sight. % info about how to destroy webs
            \spelldur \durshort \dismissable
        \end{spelleffects}
    \end{spellcontent}
    \begin{spellsubcontent}
        \begin{spelltargetinginfo}
            \spelltgts{Everything in the area}
        \end{spelltargetinginfo}
        \begin{spelleffects}
            \begin{spellattack}{Spellpower vs. Reflex}
                \spellsuccess The target is \immobilized. As a standard action, it can make a grapple attack or Escape Artist check. If its result beats your attack result, it is no longer entangled.
            \end{spellattack}
        \end{spelleffects}
    \end{spellsubcontent}
    \begin{spellfooter}
        \spellinfo{Conjuration [Creation, Physical]}{Arcane}
        \spellnotes The strands of web are flammable. Any five-foot square that takes 5 points of fire damage is destroyed. A magic flaming sword can slash them away as easily as a hand brushes away cobwebs. Any fire can set the webs alight and burn away 5 square feet in 1 round. All creatures within flaming webs take 1d6 points of fire damage from the flames.

        \physicalspellnotes

        This spell can be made permanent with a \spell{permanency} ritual. A permanent \spell{web} that is destroyed regrows in 10 minutes.
        \miscastyou
    \end{spellfooter}
\end{spellsection}

\begin{spellsection}{Windstrike}[2]
    \begin{spellheader}
        \spelldesc{You command the air to bludgeon the target, sending it flying.}
    \end{spellheader}
    \begin{spellcontent}
        \begin{spelltargetinginfo}
            \spelltwocol{\spelltgt{One creature or object}}{\spellrng{\rngmed}}
        \end{spelltargetinginfo}
        \begin{spelleffects}
            \begin{spellattack}{Spellpower vs. Fortitude}
                \spellsuccess \spelldamage{2}{bludgeoning}.
                \spellfailure Half damage.
            \end{spellattack}
            \begin{spellattack}{Spellpower vs. Maneuver defense (shove)}
                \spellsuccess You shove the target in any direction -- even vertically. Moving the target up takes twice as much movement as moving the target horizontally.
            \end{spellattack}
        \end{spelleffects}
    \end{spellcontent}
    \begin{spellfooter}
        \spellinfo{Evocation [Air]}{Air, Nature}
        \miscastrandom
    \end{spellfooter}
\end{spellsection}

\begin{spellsection}[Greater]{Windstrike}[5]
    \begin{spellheader}
        \spelldesc{You command the air to bludgeon the target with tremendous force, sending it flying.}
    \end{spellheader}
    \begin{spellcontent}
        \begin{spelltargetinginfo}
            \spelltwocol{\spelltgt{One creature or object}}{\spellrng{\rnglong}}
        \end{spelltargetinginfo}
        \begin{spelleffects}
            \begin{spellattack}{Spellpower vs. Fortitude}
                \begin{spellmargin}
                    \spellsuccess \spelldamage{5}{bludgeoning}.
                    \spellfailure Half damage.
                \end{spellmargin}
                \spellatk Spellpower \add 10 vs. Maneuver defense (shove)
                \begin{spellmargin}
                    \spellsuccess You shove the target in any direction -- even vertically.
                \end{spellmargin}
            \end{spellattack}
        \end{spelleffects}
    \end{spellcontent}
    \begin{spellfooter}
        \spellinfo{Evocation [Air]}{Air, Nature}
        \miscastrandom
    \end{spellfooter}
\end{spellsection}

\begin{spellsection}{Word of Recall}[6]
    \begin{spellheader}
    \end{spellheader}
    \begin{spellcontent}
        \begin{spelltargetinginfo}
            \spelltwocol{\spelltgt{You}}{\spellrng{Unlimited \rngunrestricted}}
            \spellcmp{Verbal only}
        \end{spelltargetinginfo}
        \begin{spelleffects}
            \spelleffect This spell teleports you instantly back to your sanctuary. You must designate the sanctuary when you ready the spell for the day, and it must be a very familiar place. The actual point of arrival is a designated area no larger than 10 feet by 10 feet. You can be transported any distance within a plane but cannot travel between planes. You can transport, in addition to yourself, any objects you carry, as long as their weight doesn't exceed your maximum load. Exceeding this limit causes the spell to fail.
        \end{spelleffects}
    \end{spellcontent}
    \begin{spellfooter}
        \spellinfo{Conjuration [Teleportation]}{Divine, Travel}
        \miscastexplode
    \end{spellfooter}
\end{spellsection}

\begin{spellsection}{Zephyr Blade}[1]
    \begin{spellheader}
        \spelldesc{You imbue a weapon with the power of the wind, allowing it to manipulate air currents as it strikes.}
    \end{spellheader}
    \begin{spellcontent}
        \begin{spelltargetinginfo}
            \spelltwocol{\spelltgt{One melee weapon}}{\spellrng{Touch}}
        \end{spelltargetinginfo}
        \begin{spelleffects}
            \spelleffect The target weapon gains an additional five feet of reach, extending the wielder's threatened area.
            \spelldur \durpersonallong
        \end{spelleffects}
    \end{spellcontent}
    \begin{spellfooter}
        \spellinfo{Evocation/Transmutation [Air, Augment]}{Air, Nature}
        \spellnotes Despite the name of the spell, it can affect melee weapons of any type, even reach weapons. The weapon's extended reach is visible, and opponents can defend themselves normally against the attacks.
        \miscastexplode
    \end{spellfooter}
\end{spellsection}

\begin{spellsection}[Greater]{Zephyr Blade}[4]
    \begin{spellheader}
        \spelldesc{You imbue a weapon with the full might of the wind, allowing it to shred opponents with nothing but the air itself.}
    \end{spellheader}
    \begin{spellcontent}
        \begin{spelltargetinginfo}
            \spelltwocol{\spelltgt{One melee weapon}}{\spellrng{Touch}}
        \end{spelltargetinginfo}
        \begin{spelleffects}
            \spellspecial This spell functions like \spell{zephyr blade}, except that the weapon can also be used to attack as a ranged weapon by expelling blasts of wind. This functions like attacking with the weapon normally, using the wielder's normal attack and damage bonuses, except that the attack is a ranged attack against any creature within 30 feet. All damage dealt when attacking in this way is bludgeoning damage instead of the attack's normal damage types. This effect does not increase the wielder's threatened area.
            \spelldur \durpersonallong
        \end{spelleffects}
    \end{spellcontent}
    \begin{spellfooter}
        \spellinfo{Evocation/Transmutation [Air, Augment]}{Air, Nature}
        \miscastexplode
    \end{spellfooter}
\end{spellsection}
