\chapter{General Rules and Glossary}
\begin{cp}%
The general rules for what to do when rounding fractions and when 
several multipliers apply to a die roll (often encountered as what to 
do when doubling something that is already doubled) are provided 
below, followed by a glossary of game terms.
\end{cp}

\section{Rounding Fractions}
In general, if you wind up with a fraction, round down, even if the 
fraction is one-half or larger.
\begin{cp}%
For example, if a \spell{fireball} deals you 17 
points of damage, but you succeed at your saving throw and only 
take half damage, you take 8 points of damage. 
\end{cp}
\subparhead{Exception:} Certain rolls, such as damage and hit points, have a 
minimum of 1. 
 
\section{Multiplying}
Sometimes a special rule makes you multiply a number or a die roll. 
As long as you're applying a single multiplier, multiply the number 
normally. When two or more multipliers apply to any abstract value 
(such as a modifier or a die roll), however, combine them into a 
single multiple, with each extra multiple adding 1 less than its value 
to the first multiple. Thus, a double (\mult2) and a double (\mult2) applied to 
the same number results in a triple (\mult3, because 2 \add 1 = 3). 
\begin{cp}%
\par For example, Tordek, a high-level dwarven fighter, deals 1d8\plus6 
points of damage with a warhammer. On a critical hit, a warhammer 
deals triple damage, so that's 3d8\plus18 damage for Tordek. A magic 
dwarven thrower warhammer deals double damage (2d8\plus12 for 
Tordek) when thrown. If Tordek scores a critical hit while throwing 
the dwarven thrower, his player rolls quadruple damage 
(4d8\plus24) because 3 \add 1 = 4. 
\par Another way to think of it is to convert the 
multiples into additions. Tordek's critical hit 
increase his damage by 2d8\plus12, and the 
dwarven thrower's doubling of damage 
increases his damage by 1d8\plus6, so both of 
them together increase his damage by 3d8\plus18 
for a grand total of 4d8\plus24. 
\end{cp}
When applying multipliers to real-world 
values (such as weight or distance), normal rules 
of math apply instead. 
\begin{cp}%
A creature whose size doubles 
(thus multiplying its weight by 8) and then is turned to 
stone (which would multiply its weight by a factor of 
roughly 3) now weighs about 24 times normal, not 10 
times normal. Similarly, a blinded creature 
attempting to negotiate difficult terrain would 
count each square as 4 squares (doubling the cost 
twice, for a total multiplier of \mult4), rather than as 
3 squares (adding 100\% twice).
\end{cp}

 \section{Glossary} 
\parhead{0-level spell:} A spell of the lowest 
possible level. Arcane spellcasters often 
call their 0-level spells ``cantrips," and 
divine spellcasters often call them ``orisons." 

\parhead{ability:} One of the six basic character qualities: Strength (Str), 
Dexterity (Dex), Constitution (Con), Intelligence (Int), Wisdom 
(Wis), and Charisma (Cha). See attribute score. 

\parhead{ability check:} A check of 1d20 \add the appropriate attribute. 

\parhead{ability damage:} A temporary loss of 1 or more attribute score 
points. Lost points return at a rate of 1 point per day unless noted 
otherwise by the condition dealing the damage. A character with 
Strength 0 falls to the ground and is helpless. A character with 
Dexterity 0 is paralyzed. A character with Constitution 0 is dead. A 
Character with Intelligence, Wisdom, or Charisma 0 is unconscious. 

\parhead{ability decrease:} A decrease in an attribute score that ends when 
the condition causing it does. 

\parhead{ability drain:} A permanent loss of 1 or more attribute score points. 
The character can only regain these points through magical means. 
A character with Strength 0 falls to the ground and is helpless. A 
character with Dexterity 0 is paralyzed. A character with 
Constitution 0 is dead. A Character with Intelligence, Wisdom, or 
Charisma 0 is unconscious. 

\parhead{attribute:} The bonus or penalty associated with a particular attribute score. Attributes apply to die rolls for character 
actions involving the corresponding abilities. 

\parhead{attribute score:} The numeric rating of one of the six character 
abilities (see ability). Some creatures lack certain attribute scores; 
others cannot be rated in particular abilities. 

\parhead{action:} A character activity. Actions are divided into the fol-
lowing categories, according to the time required to perform them 
(from most time required to least): full-round actions, standard 
actions, move actions , swift and immediate actions, and free actions. 

\parhead{adjacent:} In a square that shares a border or a corner with a 
designated square. Each square is adjacent to eight other squares on 
the board. 

\parhead{adventuring party:} A group of characters who adventure 
together. An adventuring party is composed of player characters 
plus any followers, familiars, animal companions, associates, cohorts, or hirelings they might have. 

\parhead{alignment:} One of the nine descriptors of morality for intelligent 
creatures: lawful good (LG), neutral good (NG), chaotic good (CG), lawful neutral 
(LN), neutral (N), chaotic neutral (CN), lawful 
evil (LE), neutral evil (NE), and chaotic evil (CE). 

\parhead{ally:} A creature friendly to you. In most cases, 
references to ``allies" include yourself. 

\parhead{animal:} A type of creature that includes 
all natural animals, dire animals, giant 
animals, and some other nonmagical 
vertebrate creatures (see the \booktitle{Monster Manual}). 
Animals always have an Intelligence score of 1 
or 2. 

\parhead{arcane spell failure:} The chance that a spell 
fails and is cast to no effect because the 
caster's ability to use a somatic 
component was hampered by armor. 
Bards can ignore the arcane spell failure 
chance for light armor when casting bard 
spells. 

\parhead{arcane spells:} Arcane spells involve the direct manipulation of 
mystic energies. Bards, sorcerers, and wizards cast arcane spells. 

\parhead{armor modifier:} A modifier to Armor Class granted by armor or by a spell or magical effect that mimics armor. Armor bonuses stack with 
all other bonuses to Armor Class (even with natural armor bonuses) 
except other armor bonuses. Magic armor typically grants an 
enhancement bonus to the armor's armor modifier. An armor modifier
granted by a spell or magic item typically takes the form of an 
invisible, tangible field of force around the recipient. A character's armor modifier doesn't apply against touch attacks, except for armor bonuses 
granted by force effects (such as the \spell{mage armor} spell) which apply 
against incorporeal touch attacks, such as that of a shadow. 

\parhead{Armor Class (AC):} A number representing a creature's 
ability to avoid being hit in combat. An opponent's attack 
roll must equal or exceed the target creature's Armor Class 
to hit it. Armor Class = 10 \add all modifiers that apply 
(typically armor modifier, shield modifier, Dexterity, and size modifier). 

\parhead{artifact:} A magic item of incredible power. 
Some spells do not function when targeted on an artifact. 

\parhead{Astral Plane:} An open, weightless plane 
that connects with all other planes of 
existence and is used for transportation 
among them (and is thus described as a 
transitive plane, like the Ethereal Plane 
and the Plane of Shadow). Certain 
spells (such as  \spell{astral projection}) allow 
access to this plane. 

\parhead{attack:} Any of numerous 
actions intended to harm, disable, or neutralize an opponent. 
The outcome of an attack is 
determined by an attack roll. 

\parhead{attack of opportunity:}  A single extra melee attack that a combatant can 
make when an opponent within 
reach takes an action that provokes 
attacks of opportunity. Cover prevents attacks of opportunity. 

\parhead{attack roll:} A roll to determine 
whether an attack hits. To make 
an attack roll, roll 1d20 and add 
the appropriate modifiers for the 
attack type, as follows: melee attack roll 
= 1d20 \add base attack bonus \add Strength \add size modifier; ranged attack roll = 1d20 \add 
base attack bonus \add Dexterity \add size modifier \add range penalty. In either case, the attack hits if 
the result is at least as high as the target's Armor Class. 

\parhead{barbarian (Bbn):} A class made up of ferocious warriors who use 
inborn fury and instinct to bring down foes. 

\parhead{bard (Brd):} A class made up of performers whose music and 
poetics produce magical effects. 

\parhead{base attack bonus:} An attack roll bonus derived from character 
class and level. Base attack bonuses increase at different rates for 
different character classes. A character gains a second attack when 
his or her base attack bonus reaches \plus6, a third with a base attack 
bonus of \plus11 or higher, and a fourth with a base attack bonus of \plus16 
or higher. Base attack bonuses gained from different classes, such as 
when a character is a multiclass character, stack. 

\parhead{base land speed:} The speed a character can move while unarmored. Base land speed is derived from character race. 

\parhead{base save bonus:} A saving throw modifier derived from character class and level. Base save bonuses increase at different rates for 
different character classes. Base save bonuses gained from 
different classes, such as when a character is a multiclass 
character, stack. 

\parhead{bewildered:} Mentally affected in a way that detracts from a character's ability to act. A bewildered creature takes a \minus2 penalty to attack rolls, saving throws, checks, and AC.

\parhead{blind:} Unable to see. A blind character takes a \minus2 penalty to attack rolls, Armor Class, and any checks which involve sight. In addition, it loses its Dexterity and dodge modifiers to AC (if any) and moves at half speed. All checks and activities that rely on vision (such as reading and Spot checks) automatically fail. All opponents are considered to have total concealment (50\% miss chance) relative to the blinded character. 

\parhead{bloodied:} At or below half hit points. Bloodied creatures are more vulnerable to many spells and effects.

\parhead{bonus:} An increase to the result of a die roll or to a numerical attribute. In most cases, 
multiple bonuses from the same 
source or of the same type in 
effect on the same character or 
object do not stack; only the highest 
bonus of that type applies. Bonuses 
that don't have a specific type always 
stack with all bonuses. See also modifier.

\parhead{cantrip:} An arcane 0-level spell. 

\parhead{cast a spell:} Trigger the magical or divine energy 
of a spell by means of words, gestures, focuses, and/or 
special materials. Spellcasting requires the uninterrupted 
concentration of the caster during the requisite casting 
time. Disruption of this concentration forces the caster to make a 
successful Concentration check or lose the spell. Successful casting 
brings about the spell's listed effect or effects. 

\parhead{caster level:} A measure of the power with which a spellcaster 
casts a spell. Generally, a spell's caster level is the spellcaster's class 
level. 

\parhead{caster level check:} A roll 1d20 \add the caster level (in the relevant 
class). If the result equals or exceeds the DC (or the spell resistance, 
in the case of caster level checks made for spell resistance), the 
check succeeds. 

\parhead{casting time:} The time required to cast a spell, usually either 1 
standard action, 1 round, or 1 free action. Spells with casting times 
longer than 1 round require full-round actions for all the rounds 
encompassed in the casting time. 

\parhead{Charisma (Cha):} The ability that measures a character's force of 
personality, persuasiveness, personal magnetism, ability to lead, and 
physical attractiveness. 

\parhead{character:} A fictional individual within the confines of a fantasy 
game setting. The words ``character" and ``creature" are often used 
synonymously within these rules, since almost any creature could 
be a character within the game, and every character is a creature (as 
opposed to an object). 

\parhead{character class:} One of the eleven player character types --
barbarian, bard, cleric, druid, fighter, monk, paladin, ranger, rogue, 
sorcerer, or wizard. Class defines a character's predominant talents 
and general function within an adventuring party. Character class 
may also refer to a nonplayer character class or a prestige class (see 
the \booktitle{Dungeon Master's Guide}). 

\parhead{character level:} A character's total level. For a character with 
levels in only one class, class level and character level are the same 
thing. 

\parhead{check:} A method of determining the result when a character 
attempts an action (other than an attack or a saving throw) that has a 
chance of failure. Checks are based on a relevant character ability, 
skill, or other characteristic. Most checks are either ability checks or 
skill checks, though special types such as caster 
level checks and initiative checks also exist. The 
specific name of the check usually corresponds to the skill or ability 
used. To make a check, roll 1d20 and add any relevant modifiers. 
(Higher results are always better.) If this check result equals or 
exceeds the Difficulty Class number assigned by the DM (or the 
opponent's check, if the action is opposed) the check succeeds. An effect that gives a penalty to checks affects ability checks, skill checks, initiative checks, and other similar checks, but not to caster level checks or any other level checks.

\parhead{checked:} Prevented from achieving forward motion by an 
applied force, such as wind. Checked creatures on the ground 
merely stop. Checked flying creatures move back a distance specified in the description of the effect. 

\parhead{circumstance bonus:} A bonus granted because of specific 
conditional factors favorable to the success of the task at hand. 
Circumstance bonuses stack with all other bonuses, including other 
circumstance bonuses, unless they arise from essentially the same 
benefit. For instance, a magnifying glass gives a \plus2 circumstance 
bonus on Appraise checks involving any item that is small or highly 
detailed, such as a gem. If you had a second tool that also granted a 
circumstance bonus from improved visual acuity (such as a jeweler's 
loupe), the circumstance bonuses wouldn't stack. 

\parhead{class:} See character class. 

\parhead{class feature:} Any special characteristic derived from a character 
class. 

\parhead{class level:} A character's level in a single class. Class features 
generally depend on class level rather than character level. 

\parhead{class skill:} A skill to which characters of a particular class have 
easier access than characters of other classes.

\parhead{cleric (Clr):} A class made up of characters who cast divine spells 
and are also capable in combat. 

\parhead{Colossal:} A Colossal creature is typically 64 feet or more in 
height or length and weighs 250,000 pounds or more. 

\parhead{comatose:} Effectively in a state of suspended animation. A 
comatose creature is helpless. 

\parhead{command word item:} A magic item that activates when the user 
speaks a particular word or phrase. Activating a command word item 
does not require concentration and does not provoke attacks of 
opportunity. 

\parhead{command undead:} The supernatural ability of evil clerics and 
some neutral clerics to control undead creatures by channeling 
negative energy. 

\parhead{competence bonus:} A bonus that derives from a character's experience and training, such as from feats and class features.  Multiple competence bonuses don't stack; only the highest bonus applies. 

\parhead{concealment:} Something that prevents an attacker from clearly 
seeing his or her target. Concealment creates a chance that an 
otherwise successful attack misses (a miss chance). 

\parhead{concentrate on a spell:} Concentrate to maintain an active spell's 
effect. Concentrating on a spell is a standard action and provokes an 
attack of opportunity. 

\parhead{confused:} Befuddled and unable to determine a course of action due 
to a spell or magical effect. \confusionexplanation Attackers are not at any special advantage when attacking aconfused
character. A confused character does not make attacks of 
opportunity against any creature that it is not already devoted to 
attacking (either because of its most recent action or because it has 
just been attacked).

\parhead{Constitution (Con):} The ability that represents a character's 
health and stamina. 

\parhead{continuous damage:} Damage from a single attack that continues 
to deal damage every round without the need for additional attack 
rolls. 

\parhead{copper piece (cp):} The most prevalent form of currency among 
beggars and unskilled laborers. Ten copper pieces are equivalent to 1 silver 
piece. 

\parhead{coup de grace:} A full-round action that allows an attacker to 
attempt a killing blow against a helpless opponent. A coup de grace 
can be administered with a melee weapon or with a bow or crossbow 
if the attacker is adjacent to the opponent. An attacker delivering a 
coup de grace automatically scores a critical hit, after which the 
defender must make a successful Fortitude save (DC 10 \add damage 
dealt) or die. Rogues also gain their extra sneak attack damage for 
this attack. Delivering a coup de grace provokes attacks of opportunity from threatening foes. A coup de grace is not possible against 
a creature immune to critical hits. 

\parhead{cover:} Any barrier between an attacker and defender. Such a 
barrier can be an object, a creature, or a magical force. Cover grants 
the defender a bonus to Armor Class. 

\parhead{cowering:} Frozen in fear and unable to take actions. A cowering 
character takes a \minus2 penalty to Armor Class and loses her Dexterity 
and dodge modifiers to AC. 

\parhead{creature:} A living or otherwise active being, not an object. The 
terms ``creature" and ``character" are sometimes used interchangeably. 

\parhead{creature type:} One of several broad categories of creatures. The 
creature types are aberration, animal, construct, dragon, elemental, 
fey, giant, humanoid, magical beast, monstrous humanoid, ooze, 
outsider, plant, undead, and vermin. (See the \booktitle{Monster Manual} for full 
descriptions.) 

\parhead{critical hit (crit):} A hit that strikes a vital area and therefore deals 
double damage or more. To score a critical hit, an attacker must first 
score a threat (usually a natural 20 on an attack roll) and then 
succeed on a critical roll (just like another attack roll). Critical hit 
damage is usually double damage, which means rolling damage 
twice, just as if the attacker had actually hit the defender two times. 
(Any extra damage dice, such as from a rogue's sneak attack, are not 
rolled multiple times, but are added to the total at the end of the 
calculation.) 

\parhead{critical roll:} A special second attack roll made in the event of a 
threat to determine whether a critical hit has been scored. If the 
critical roll is a hit against the target creature's AC, then the original 
attack is a critical hit. Otherwise, the original attack is a regular hit. 

\parhead{cross-class (cc) skill:} A skill that is not a class skill for a character.Characters have more difficulty mastering cross-class skills.

\parhead{cure spell:} Any spell with the word ``cure" in its name, such as 
\spell{cure minor wounds}, \spell{cure light wounds}, or \spell{mass cure critical wounds}. 

\parhead{current hit points:} A character's hit points at a given moment in 
the game. Current hit points go down when the character takes 
damage and go back up upon recovery. 

\parhead{damage:} A decrease in hit points, an attribute score, or other aspects 
of a character caused by an injury, illness, or magical effect. The 
three main categories of damage are lethal damage, nonlethal 
damage, and ability damage. In addition, wherever it is relevant, the 
type of damage an attack deals is specified, since natural abilities, 
magic items, or spell effects may grant immunity to certain types of 
damage. Damage types include weapon damage (subdivided into 
bludgeoning, slashing, and piercing) and energy damage (positive, 
negative, acid, cold, electricity, fire, and sonic). Modifiers to melee 
damage rolls apply to both subcategories of weapon damage (melee 
and unarmed). Some modifiers apply to both weapon and spell 
damage, but only if so stated. Damage points are deducted from 
whatever character attribute has been harmed -- lethal and 
nonlethal damage from current hit points, and ability damage from 
the relevant attribute score). Damage heals naturally over time, but can 
also be negated wholly or partially by curative magic. 

\parhead{damage reduction (DR):} A special defense that allows a creature 
to ignore a set amount of damage from most weapons, unarmed 
attacks, or natural weapons, but not from energy attacks, spells, 
spell-like abilities, and supernatural abilities. The number in a 
creature's damage reduction is the amount of hit points of damage 
the creature ignores. The information after the slash indicates the 
type of weapon (such as magic, silver, or evil) that overcomes the 
damage reduction. Some damage reduction, such as that of a 
barbarian, is not overcome by any type of weapon. 

\parhead{darkvision:} An extraordinary ability possessed by some creatures 
that enables them to see in the dark. Darkvision always extends out to a certain range. Beyond that range, the creature can see dimly, treating areas of darkness as shadowy illumination. Darkvision does not function if the creature is in a brightly lit area or is dazzled, and does not resume functioning until 1 round after the creature leaves the brightly lit area or stops being dazzled.
\par Darkvision is black and white only, but it is otherwise like normal sight. 

\parhead{dazed:} Unable to act normally. A dazed character can take no 
actions, but has no penalty to AC. 

\parhead{dazzled:} Unable to see well because of overstimulation of the 
eyes. A dazzled creature takes a \minus1 penalty on attack rolls, Spot 
checks, and AC. Dazzled creatures are unable to use darkvision.

\parhead{dead:} A character dies when his or her hit points drop to \minus10 or 
lower. A character also dies when his or her Constitution drops to 0, 
and certain spells or effects can also kill a character outright. Death causes the 
character's soul to leave the body and journey to an Outer Plane. 
Dead characters cannot benefit from normal or magical healing, but 
they can be restored to life via magic. A dead body decays normally 
unless magically preserved, but magic that restores a dead character 
to life also restores the body either to full health or to its condition 
at the time of death (depending on the spell or device). 

\parhead{deafened:} Unable to hear. A deafened character automatically fails Listen checks and has a 20\% chance of spell failure when casting spells with verbal components. 

\parhead{deal damage:} Cause damage to a target with a successful attack. 
How much damage is dealt is usually expressed in terms of dice (for 
example, 2d6\plus4) and may have a situational modifier as well. If the target mas have 
special defenses that negate some or all of the damage, the damage dealt is reduced.

\parhead{death attack:} A spell or special ability that instantly slays the 
target, such as  \spell{finger of death}. Neither raise dead nor reincarnation can 
grant life to a creature slain by a death attack, though resurrection and 
more powerful effects can. 

\parhead{deflection modifier:} A modifier to Armor Class granted by a spell or magic effect that makes attacks veer off harmlessly. A deflection modifier applies against touch attacks. 

\parhead{Dexterity (Dex):} The ability that measures a character's hand-eye 
coordination, agility, reflexes, and balance. 

\parhead{difficult terrain:} An area containing one or more features (such 
as rubble or undergrowth) that costs 2 squares instead of 1 square to 
move through. 

\parhead{Difficulty Class (DC):} The target number that a player must 
meet or beat for a check or saving throw to succeed. Difficulty 
Classes other than those given in specific spell or item descriptions 
are set by the DM using the skill rules as a guideline. 

\parhead{Diminutive:} A Diminutive creature is typically between 6 inches 
and 1 foot in height or length and weighs between 1/8 pound and 1 
pound. 

\parhead{direct a spell:} Direct an active spell's effect at a specific target or 
targets. Directing a spell is a move action and does not provoke an 
attack of opportunity. 

\parhead{disabled:} At exactly 0 current hit points, or in negative hit points 
but stable and conscious. A disabled character may take a single 
move action or standard action each round (but not both, nor can 
she take full-round actions). She moves at half speed. Taking move 
actions doesn't risk further injury, but performing any standard 
action (or any other strenuous action, including 
some swift actions such as casting a quickened spell) deals 1 point of 
damage after the completion of the act. Unless the action increased 
the disabled character's hit points, she is now in negative hit points 
and dying. 

\parhead{dispel:} Negate, suppress, or remove one or more existing spells 
or other effects on a creature, item, or area. Dispel usually refers to a 
\spell{dispel magic} spell, though other forms of dispelling are possible. 
Certain spells cannot be dispelled, as noted in the individual spell 
descriptions. 

\parhead{dispel check:} A roll 1d20 \add caster level of the character making 
the attempt to dispel (usually used with \spell{dispel magic}). The DC is 11 
plus the level of the spellcaster who initiated the effect being 
dispelled. 

\parhead{divine spells:} Spells of religious origin powered by faith or by a 
deity. Clerics, druids, paladins, and rangers cast divine spells. 

\parhead{dodge modifier:} A character's modifier to Armor Class resulting from physical skill at avoiding blows and other ill effects. Any situation or effect (except wearing armor) that negates a character's Dexterity bonus also negates any dodge bonuses the character may have (for instance, you lose your dodge modifier to AC when you're flat-footed). Your dodge modifier applies against touch attacks. 

\parhead{domain:} A granted power and a set of eighteen divine spells (two each of 1st through 9th level) themed around a particular concept 
and associated with one or more deities. The available domains are:
Air, Animal, Chaos, Death, Destruction, Earth, Evil, Fire, Good, 
Healing, Knowledge, Law, Magic, Plant, Protection, Strength, 
Sun, Travel, Trickery, War, and Water. 

\parhead{domain spell:} A divine spell belonging to a domain. Each domain offers two spells of each spell level. A cleric adds his domain spells to his spell list if the spells were not already on his list. In addition to the normal spells a cleric knows at each spell level, he knows two domain spells which may be selected from any of the spells his domains offer at that level. He may also select any of his domain spells with his normal selection of spells known.

\parhead{double weapon:} A weapon with two ends, blades, or heads that 
are both intended for use in combat. Any weapon for which two 
damage ranges are listed is a double weapon. Double weapons can be 
used to make an extra attack as if the wielder were fighting with two 
weapons (light weapon in the off hand).

\parhead{druid (Drd):} A class made up of characters who draw energy 
from the natural world to cast divine spells and gain special magical 
powers. 

\parhead{Dungeon Master (DM):} The player who portrays nonplayer 
characters, makes up the story setting for the other players, and 
serves as a referee. 

\parhead{dying:} Unconscious and near death. A dying character has \minus1 to \minus9 current hit points, can take no actions, and is unconscious. Each 
round on her turn, a dying character rolls d% to see whether she 
becomes stable. She has a 10\% chance of becoming stable. If she 
does not, she loses 1 hit point. If a dying character reaches \minus10 hit 
points, she is dead. 

\parhead{effective hit point increase:} Hit points gained through temporary increases in Constitution score. Unlike temporary hit points, 
points gained in this manner are not lost first, and must be 
subtracted from the character's current hit points at the time the 
Constitution increase ends. 

\parhead{electrum:} A naturally-occurring alloy of gold and silver. 

\parhead{Elemental Plane:} One of the Inner Planes consisting almost 
entirely of one type of element: air, earth, fire, or water. 

\parhead{end of round:} The point in a combat round when all the participants have completed all their allowed actions. End of round 
occurs when no one else involved in the combat has an action 
pending for that round. 

\parhead{enemy:} A creature unfriendly to you. 

\parhead{energy damage:} Damage caused by one of five types of energy 
(not counting positive and negative energy): acid, cold, electricity, 
fire, and sonic. 

\parhead{energy drain:} An attack that saps a creature's vital energy giving 
it negative levels, which might permanently drain the creature's levels. 

\parhead{Energy Plane:} An Inner Plane, either the Positive Energy Plane 
or the Negative Energy Plane. 

\parhead{engaged:} Threatening or being threatened by an enemy. 
(Unconscious, or otherwise immobilized characters are not considered engaged unless they are actually being attacked.) 

\parhead{enhancement bonus:} A bonus that represents an increase in the effectiveness of a creature or object. Enhancement bonuses almost always come from magical effects, and almost all spells grant enhancement bonuses. Multiple enhancement bonuses do not stack. Only the highest enhancement bonus 
applies.

\parhead{entangled:} Ensnared. Being entangled impedes movement, but 
does not entirely prevent it unless the bonds are anchored to an 
immobile object or tethered by an opposing force. An entangled 
creature moves at half speed, cannot run or charge, and takes a \minus2 
penalty to Strength and Dexterity. An entangled character who attempts to cast a spell must make a Concentration check (DC 10 \add double the spell's level) or lose the spell.

\parhead{ethereal:} On the Ethereal Plane. An ethereal creature is invisible 
and intangible to creatures on the Material Plane, but visible and 
corporeal to creatures on the Ethereal Plane. As such, such a 
creature is capable of moving through solid on the Material Plane, 
and in any direction (even up or down), though all movement is at 
half speed. Ethereal beings can see and hear what is happening in 
the same area of the Material Plane to a distance of 60 feet, but 
everything looks gray and insubstantial. Force effects originating on 
the Material Plane can affect items and creatures that are ethereal, 
but the reverse is not true. 

\parhead{Ethereal Plane:} A gray, foggy plane parallel to the Material Plane 
at all points. Creatures within the Ethereal Plane can see and hear 
into the Material Plane to a distance of 60 feet, though the reverse is 
not usually true. Force effects originating on the Material Plane can 
affect items and creatures on the Ethereal Plane, but the reverse is 
not true. Because the Ethereal Plane is often used for travel, it is also 
considered a transitive plane (like the Astral Plane and the Plane of 
Shadow). 

\parhead{exhausted:} Tired to the point of significant impairment. An 
exhausted character moves at half speed and takes a \minus6 penalty to 
Strength and Dexterity. After 1 hour of complete rest, an exhausted 
character becomes fatigued. A fatigued character becomes exhausted 
by doing something else that would normally cause fatigue. 

\parhead{experience points (XP):} A numerical measure of a character's 
personal achievement and advancement. Characters earn experience 
points by defeating opponents and by overcoming challenges. At 
the end of each adventure, the DM assigns experience to the 
characters based on what they have accomplished. Characters 
continue to accumulate experience points throughout their 
adventuring careers, gaining new levels in their character classes at 
certain experience totals. 

\parhead{extraordinary ability (Ex):} A nonmagical special ability (as 
opposed to a spell-like or supernatural ability). 

\parhead{extraplanar:} Native to a plane of existence other than the plane 
on which a creature is present. On the Material Plane, an outsider is 
an extraplanar creature. On an outsider's home plane, a native of the 
Material Plane is an extraplanar creature. 

\parhead{failure:} An unsuccessful result on a check, saving throw, or other 
determination involving a die roll. 

\parhead{fascinated:} Entranced by a supernatural or spell effect. A fascinated creature stands or sits quietly, taking no actions other than to 
pay attention to the fascinating effects, for as long as the effect lasts. 
It takes a \minus4 penalty on skill checks made as reactions, such as Listen 
and Spot checks. Any potential threat, such as a hostile creature 
approaching, allows the fascinated creature a new saving throw 
against the fascinating effect. Any obvious threat, such as a someone 
drawing a weapon, casting a spell, or aiming a ranged weapon at the 
fascinated creature, automatically breaks the effect. A fascinated 
creature's ally may shake it free of the effect as a standard action. 

\parhead{fatigued:} Tired to the point of impairment. A fatigued character 
can neither run nor charge and takes a \minus2 penalty to Strength and 
Dexterity. Doing anything that would normally cause fatigue causes 
the fatigued character to become exhausted. After 8 hours of 
complete rest, fatigued characters are no longer fatigued. 

\parhead{fear effect:} Any spell or magical effect that causes the victim to 
become shaken, frightened, or panicked, or to suffer from some 
other fear-based effect defined in the description of the specific spell 
or item in question. Fear effects that inflict the same level of fear usually stack, worsening the condition of the affected creature.

\parhead{fighter (Ftr):} A class made up of characters who have exceptional 
combat capability and unequalled skill with weapons. 

\parhead{Fine:} A Fine creature is typically 6 inches or less in height or 
length and weighs 1/8 pound or less. 

\parhead{flat-footed:} Especially vulnerable to attacks at the beginning of a 
battle. Characters are flat-footed until their first turns in the 
initiative cycle. A flat-footed creature loses its Dexterity and dodge modifiers to 
Armor Class (if any) and cannot make attacks of opportunity. 

\parhead{force damage:} A special type of damage dealt by force effects, 
such as a \spell{magic missile} spell. A force effect can strike incorporeal creatures without the normal miss chance associated with incorporeality. 

\parhead{Fortitude save:} A type of saving throw, related to a character's 
ability to withstand damage thanks to his physical stamina. 

\parhead{free action:} Free actions consume a negligible amount of time, 
and one or more such actions can be performed in conjunction with 
actions of other types. 

\parhead{frightened:} Fearful of a creature, situation, or object. A frightened creature flees from the source of its fear as best it can. If unable 
to flee, it may fight. A frightened creature takes a \minus2 penalty on
attack rolls, saving throws, checks, and AC. A 
frightened creature can use special abilities, including spells, to flee; 
indeed, the creature must use such means if they are the only way to 
escape. If frightened again, a frightened creature becomes panicked.

\parhead{full attack:} The action that allows a character to make their normal sequence of attacks. Making a full attack is normally a standard action. 

\parhead{full normal hit points:} An individual character's maximum hit 
points when undamaged. 

\parhead{full-round action:} Full-round actions consume all of a character's effort during a round. Unless the action involves movement, you cannot move while performing a full-round action. When using a full-round action to cast a spell whose 
casting is 1 round, the spell is not completed until the beginning of 
the caster's next turn. 

\parhead{Gargantuan:} A Gargantuan creature is between 32 and 64 feet in 
height or length and weighs between 32,000 and 250,000 pounds. 

\parhead{gold piece (gp):} The primary unit of currency used by adventurers. 

\parhead{granted power:} The special ability a cleric gain from each of his 
selected domains. 

\parhead{grappled:} Engaged in wrestling or some other form of hand-to-hand struggle with one or more attackers. For creatures, grappling can also mean 
trapping opponents in any number of ways (in a toothy maw, under 
a huge paw, and so on). While grappled, you suffer certain penalties and restrictions, as described below.
\begin{itemize*}
\item You must use one of your hands (or equivalent limbs) to grapple, preventing you from taking any actions which would require having two free hands. For example, you cannot attack with a two-handed weapon while grappling.
\item You take a \minus2 penalty to all attack rolls except those made to grapple.
\item You lose your Dexterity and dodge modifiers to AC to all opponents except the one you are grappling.
\item Anyone making a ranged attack at you has a 50\% chance to strike the other participant in the grapple instead.
\item You do not threaten any opponents except for the creature you are grappling with.
\item You take an additional \minus4 penalty to attack rolls made with one-handed weapons, since they are too large and cumbersome to be used effectively in a grapple.
\item You cannot cast spells with somatic components.
\item Casting a spell without somatic components requires a DC 20 \add double spell level Concentration check.
\item You cannot move normally.
\end{itemize*}
Other than the restrictions listed above, you can act normally. You can also try to move the grapple, escape the grapple, or pin your opponent.

\parhead{half speed:} When restricted to moving at half speed, count each 
square moved into as 2 squares, and every square of diagonal 
movement as 3 squares. If you are restricted to half speed, you can't 
run or charge.

\parhead{hardness:} A measure of an object's ability to resist damage. Only 
damage in excess of the object's hardness is actually deducted from 
the object's hit points upon a successful attack. 

\parhead{healthy:} Above half hit points. Healthy creatures can resist many spells and effects more easily.

\parhead{helpless:} Paralyzed,  held, bound, sleeping, unconscious, or otherwise completely at an opponent's mercy. A helpless target is 
treated as having a Dexterity of 0 (\minus5 modifier), and they automatically fail Reflex saves. Helpless creatures are usually also treated as being prone. An attacker can use a coup de grace against a helpless target. 

\parhead{hit:} Make a successful attack roll. 

\parhead{Hit Value/Values (HV):} In the singular form, the number of hit points a creature gains when gaining a level. In the plural form, a measure of relative power that is 
synonymous with character level for the sake of spells, magic items, 
and magical effects that affect a certain number of Hit Values of 
creatures. 

\parhead{hit points (hp):} A measure of a character's health or an object's 
integrity. Damage decreases current hit points, and lost hit points 
return with healing or natural recovery. A character's hit point total 
increases permanently with additional experience and/or permanent increases in Constitution, or temporarily through the use of 
various special abilities, spells, magic items, or magical effects (see 
temporary hit points and effective hit point increase). 

\parhead{Huge:} A Huge creature is typically between 16 and 32 feet in 
height or length and weighs between 4,000 and 32,000 pounds. 


\parhead{ignited:} An ignited creature has been set on fire. It takes a \minus2 penalty to attack rolls, saving throws, checks, and AC, and takes d6 damage per round from the fire. If the creature takes a full-round action, it can attempt a DC 15 Reflex save to put out the flames. This action provokes attacks of opportunity. Dropping prone gives a \plus4 circumstance bonus on this save. 

\parhead{immobilized:} An immobilized creature can't move out 
of the space it was in when it became immobilized. It otherwise functions normally unless it's flying. Immobilized flying creatures that have the ability to hover can maintain 
their initial altitude. All other flying creatures subjected 
to this condition descend at a rate of 20 feet per round until 
they reach the ground, taking no falling damage.


\parhead{incorporeal:} Having no physical body. Incorporeal creatures are 
immune to all nonmagical attack forms. They can be harmed only by 
other incorporeal creatures, \plus1 or better magic weapons, spells, 
spell-like effects, or supernatural effects. Even when struck by spells, 
magical effects, or magic weapons, however, they have a 50% chance 
to ignore any damage from a corporeal source. In addition, rogues 
cannot employ sneak attacks against incorporeal beings, since such 
opponents have no vital areas to target. An incorporeal creature has 
no armor or natural armor bonus (or loses any armor or natural 
armor bonus it may have when corporeal), but it gains a deflection 
bonus equal to its Charisma or \plus1, whichever is greater. 
Such creatures can move in any direction and even pass through 
solid objects at will, but not through force effects. Therefore, their 
attacks negate the bonuses provided by natural armor, armor, and 
shields, but deflection bonuses and force effects (such as \spell{mage armor}) work normally against them. Incorporeal creatures have no 
weight, do not leave footprints, have no scent and make no noise, so 
they cannot be heard with Listen checks unless they wish it. 
Incorporeal creatures cannot fall or take falling damage. 

\parhead{inflict spell:} A spell with the word ``inflict" in its name, such as 
inflict light wounds,  inflict moderate wounds, or  mass inflict critical 
wounds. 

\parhead{inherent bonus:} A bonus to an attribute score resulting from 
powerful magic, such as a  \spell{wish}. Inherent bonuses cannot be dispelled. A character is limited to a total inherent bonus of \plus5 to any 
attribute score. Multiple inherent bonuses to a particular attribute score stack.

\parhead{initiative:} A system of determining the order of actions in battle. 
Before the first round of combat, each combatant makes a single 
initiative check. Each round, the participants act in order from the 
highest initiative result to the lowest. 

\parhead{initiative check:} A check used to determine a creature's place in 
the initiative order for a combat. An initiative check is 1d20 \add Dex 
modifier \add other modifiers. 

\parhead{initiative count:} The result of an initiative check, expressed as a 
number that indicates when a character's turn comes up. 

\parhead{initiative modifier:} A character's total bonus or penalty to initiative checks. 

\parhead{Inner Plane:} One of several portions of the planar landscape that 
contain the primal forces -- those energies and elements that make 
up the building blocks of reality. The Elemental Planes and the 
Energy Planes are Inner Planes. 

\parhead{Intelligence (Int):} The ability that determines how well a 
character learns and reasons. 

\parhead{invisible:} Visually undetectable. An invisible creature gains a \plus2 
bonus on attack rolls against sighted opponents, and ignores its 
opponents' Dexterity bonus to AC (if any). (Invisibility has no effect 
against blinded or otherwise nonsighted creatures.) An invisible 
creature's location cannot be pinpointed by visual means. It has total 
concealment; even if an attacker correctly guesses the invisible 
creature's location, the attacker has a 50\% miss chance in combat. 
An invisible creature gains a \plus40 bonus on Hide checks if 
immobile, or a \plus20 bonus on Hide checks if moving. If it is not hiding, locating the 
square an invisible creature occupies requires a Spot check (DC 40 if 
the creature is immobile, DC 20 if the creature moved during its last 
turn), modified by appropriate factors (such as an armor check 
penalty or a penalty for movement). 

\parhead{kind:} A subcategory of creature type. For example, giant is a 
creature type, and hill giant is a kind of giant. 

\parhead{known spell:} A spell that a spellcaster has learned. Knowing a spell means having 
selected it when acquiring new spells as a benefit of level advancement. 

\parhead{Large:} A Large creature is typically between 8 and 16 feet in 
height or length and weighs between 500 and 4,000 pounds. 

\parhead{lethal damage:} Damage that reduces a creature's hit points. 

\parhead{level:} A measure of advancement or power applied to several 
areas of the game. See caster level, character level, class level, and 
spell level. 

\parhead{light weapon:} A weapon suitable for use in the wielder's off 
hand, such as a dagger. A light weapon is considered to be an object 
two size categories smaller than its designated wielder (for example, 
a Medium dagger -- that is, a dagger suited for a Medium creature -- is a Tiny object). 

\parhead{line of effect:} Line of effect tells you whether an effect (such as 
an explosion) can reach a creature. Line of effect is just like line of 
sight, except line of effect ignores restrictions on visual ability. For 
instance, a \spell{fireball}'s explosion doesn't care if a creature is invisible or 
hiding in darkness. 

\parhead{line of sight:} Two creatures can see each other if they have line 
of sight to each other. To determine line of sight, draw an imaginary 
line between your space and the target's space. If any such line is 
clear (not blocked), then you have line of sight to the creature (and it 
has line of sight to you). The line is clear if it doesn't intersect or 
even touch squares that block line of sight. If you can't see the target 
(for instance, if you're blind or the target is invisible), you can't have 
line of sight to it even if you could draw an unblocked line between 
your space and the target's. 

\parhead{low-light vision:} The ability to see in conditions of dim illumination as if the illumination were actually as bright as daylight. A character with low-light vision sees as if the radius bright and shadowy illumination was doubled. Additionally 


\parhead{magic level:} How good a character is at magic. This is the level used to determine the spells per day and spells known for a particular class. Characters track magic levels separately for each of their classes.

\parhead{magical class:} A class that gains magic levels. For example, wizards and rangers are magical classes.


\parhead{masterwork:} Exceptionally well-made, generally providing a \plus1 
enhancement bonus on attack rolls (if the item is a weapon or 
ammunition), reducing the armor check penalty by 1 (if the item is 
armor or a shield), or adding \plus2 to relevant skill checks (if the item is 
a tool). 

\parhead{Material Plane:} The ``normal" plane of existence. 

\parhead{Medium:} A Medium creature is typically between 4 and 8 feet in 
height or length and weighs between 60 and 500 pounds. 

\parhead{melee:} Melee combat consists of physical blows exchanged by 
opponents close enough to threaten one another's space (as opposed 
to ranged combat). 

\parhead{melee attack:} A physical attack suitable for close combat. 

\parhead{melee attack bonus:} The modifier applied to a melee attack roll. 

\parhead{melee attack roll:} An attack roll during melee combat, as 
opposed to a ranged attack roll. See attack roll. 

\parhead{melee touch attack:} A touch attack made in melee, as opposed to 
a ranged touch attack. See touch attack. 

\parhead{melee weapon:} A handheld weapon designed for close combat. 

\parhead{miss chance:} The possibility that a successful attack roll misses 
anyway because of the attacker's uncertainty about the target's 
location. See concealment. 

\parhead{miss chance roll:} A d\% to determine the success of an attack roll 
to which a miss chance applies. 

\parhead{modifier:} A change to the result of a die roll or to a numerical attribute (such as armor class). A modifier is the result of a combination of bonuses and penalties. Bonuses of the same type usually do not stack when calculating a total modifier. For most rolls, there is only one total modifier, and all bonuses and penalties are added together to form the final modifier. In some situations, such as with armor class, the final result is the result of a combination of modifiers. In that case, each modifier is calculated separately, and different modifiers track the use of bonus types separately. For example, since armor class is affected by both an armor modifier and a shield modifier, those two modifiers are calculated separately. If a character has an enhancement bonus to both his armor modifier and his shield modifier, both enhancement bonuses apply in full, rather than overlapping.

\parhead{monk (Mnk):} A class made up of characters who are masters of 
the martial arts and have a number of exotic powers. 

\parhead{move action:} An action that is the equivalent of the character 
moving his speed. Move actions include standing up from prone, 
drawing or sheathing a weapon, opening a door, loading a light 
crossbow, and moving your speed. In a typical round, a character can 
take a move action and a standard action, or he can take a second 
move action in place of his standard action. 

\parhead{mundane:} Normal, commonplace, or everyday. Also used as a 
synonym for ``nonmagical." 

\parhead{natural:} A natural result on a roll or check is the actual number 
appearing on the die, not the modified result obtained by adding 
bonuses or subtracting penalties. 

\parhead{natural ability:} A nonmagical capability, such as walking, 
swimming (for aquatic creatures), and flight (for winged creatures). 

\parhead{natural armor modifier:} A modifier to Armor Class resulting from a creature's naturally tough hide. A natural armor bonus doesn't apply against touch attacks. 

\parhead{natural reach:} The distance from which a creature can make a 
melee attack. The creature threatens all squares within that distance 
from its space. 

\parhead{nauseated:} Experiencing stomach distress. Nauseated creatures 
are unable to attack, cast spells, concentrate on spells, or do anything 
else requiring attention. The only action such a character can take is 
a single move action per turn, plus free actions. A nauseated creature is also sickened.

\parhead{negate:} Invalidate, prevent, or end an effect with respect to a 
designated area or target. 

\parhead{negative energy:} A black, crackling energy that originates on the 
Negative Material Plane. In general, negative energy heals undead 
creatures and hurts the living. 

\parhead{Negative Energy Plane:} The Inner Plane from which negative 
energy originates. 

\parhead{negative level:} A loss of vital energy resulting from energy drain, 
spells, magic items, or magical effects. For each negative level 
gained, a creature takes a \minus1 penalty on all attack rolls, saving 
throws, and checks, loses 5 hit points, and takes a 
\minus1 penalty to effective level. (That is, whenever the creature's level is 
used in a die roll or calculation, reduce its value by 1 for each 
negative level.) The hit points lost decrease the creature's maximum hit points for as long as the negative level persists. In addition, a spellcaster loses one spell slot  
from the highest spell level he or she can cast. The 
lost spell slot becomes available again as soon as the negative level is 
removed, providing the caster would be capable of using it at that 
time. A creature recovers from negative levels at a rate of one per day.
Some negative levels, such as those acquired as a result of dying, cannot be removed by normal means and will only go away after a fixed period of time or until some condition is met.

\parhead{nonintelligent:} Lacking an Intelligence score. Mind-affecting 
spells do not affect nonintelligent creatures.

\parhead{nonlethal damage:} Damage typically resulting from an unarmed 
attack, an armed attack delivered with intent to subdue, a forced 
march, or a debilitating condition such as heat or starvation. 


\parhead{nonmagical class:} A class that never gains magic levels. For example, fighters and rogues are nonmagical classes.


\parhead{nonplayer character (NPC):} A character controlled by the 
Dungeon Master rather than by one of the other players in a game 
session, as opposed to a player character. 

\parhead{off hand:} A character's weaker or less dexterous hand (usually 
the left). An attack made with the off hand incurs a \minus4 penalty on 
the attack roll. In addition, only one-half of a character's Strength 
bonus may be added to the damage dealt with a weapon held in the 
off hand.

\parhead{one-handed weapon:} A weapon designed for use in one hand, 
such as a longsword, often either along with a shield or a light 
weapon in the other hand. A one-handed weapon is considered to be 
an object one size category smaller than its designated wielder (for 
example, a Medium longsword -- a longsword sized for a Medium creaturejjs --  is a Small object). 

\parhead{orison:} A divine 0-level spell. 

\parhead{Outer Plane:} One of several planes of existence where spirits of 
mortal beings go after death. These planes are the homes of 
powerful beings, such as demons, devils, and deities. Individual 
Outer Planes typically exhibit the traits of one or two specific 
alignments associated with the beings who control them. 

\parhead{overlap:} Coexist with another effect or modifier in the same area 
or on the same target. Bonuses that do not stack with each other 
overlap instead, such that only the largest bonus provides its benefit. 

 \parhead{overwhelming hit:} An attack where the die result is treated as if it were 30. Overwhelming  hits occur on an attack roll of natural 20 or as a result of 
certain spells. A natural 20 on an attack roll is also a threat -- a possible 
critical hit. 

 \parhead{overwhelming miss:} An attack where the die result is treated as if it were \minus10.
Overwhelming  misses occur on an attack roll of natural 1. 

\parhead{paladin (Pal):} A class made up of characters who are champions 
of justice and destroyers of evil, with an array of divine powers. 

\parhead{panicked:} A panicked creature must drop anything it holds and 
flee at top speed from the source of its fear, as well as any other 
dangers it encounters. It can't take any other 
actions. In addition, the creature takes a \minus2 penalty on saving 
throws, checks, and AC. If cornered, a panicked 
creature cowers and does not attack, typically using the total defense 
action in combat. A panicked creature can use special abilities, 
including spells, to flee; indeed, the creature must use such means if 
they are the only way to escape. 

\parhead{paralyzed:} Frozen in place and unable to move or act, such as by a ghoul's touch. A paralyzed character has effective Dexterity 
and Strength scores of 0 and is helpless, but can take purely mental 
actions. A winged creature flying in the air at the time that it 
becomes paralyzed cannot flap its wings and falls. A paralyzed 
swimmer can't swim and may drown. A creature can move through a 
space occupied by a paralyzed creature -- ally or not. Each square 
occupied by a paralyzed creature, however, counts as difficult terrain.

\parhead{party:} A group of adventurers. 

\parhead{petrified:} Turned to stone. Petrified characters are considered 
unconscious. If a petrified character cracks or breaks, but the broken 
pieces are joined with the body as it returns to flesh, he is 
unharmed. Otherwise, the DM must assign some amount of permanent hit point loss and/or debilitation. 

\parhead{pinned:} Held immobile (but not helpless) in a grapple. 

\parhead{plane of existence:} One of many dimensions that may be 
accessed by spells, spell-like abilities, magic items, or specific creatures. These planes include (but are not limited to) the Astral Plane, 
the Ethereal Plane, the Inner Planes, the Outer Planes, the Plane of 
Shadow, and various other realities. The ``normal" world is part of 
the Material Plane. 

\parhead{Plane of Shadow:} A plane of existence that pervades the Material 
Plane. The Plane of Shadow may be accessed and manipulated from 
the Material Plane through shadows. Shadow spells make use of the 
substance of this plane in their casting. Since some creatures use the 
Plane of Shadow to travel from place to place, it is often described as 
a transitive plane (like the Astral Plane and Ethereal Plane). 

\parhead{platinum piece (pp):} A form of currency not in common 
circulation but occasionally found as treasure. One platinum piece is 
equivalent to 10 gold pieces. 

\parhead{player character (PC):} A character controlled by a player other 
than the Dungeon Master, as opposed to a nonplayer character. 

\parhead{point of origin:} The location in space where a spell or magical 
effect begins. The caster designates the point of origin for any spells 
in which it is variable. 

\parhead{points of damage:} A number by which an attack reduces a 
character's current hit points. 

\parhead{positive energy:} A white, luminous energy that originates on the 
Positive Material Plane. In general, positive energy heals the living 
and hurts undead creatures. 

\parhead{Positive Energy Plane:} The Inner Plane from which positive 
energy originates. 

\parhead{prerequisite:} A requirement that must be met before a given 
benefit can be gained. 

\parhead{projectile weapon:} A device, such as a bow, that uses mechanical 
force to propel a projectile toward a target. 

\parhead{prone:} Lying on the ground. An attacker who is prone has a \minus4 
penalty on melee attack rolls and cannot used a ranged weapon 
(except for a crossbow). A defender who is prone gains a \plus4 bonus to 
Armor Class against ranged attacks, but takes a \minus4 penalty to AC 
against melee attacks. 

\parhead{racial bonus:} A bonus granted because of the culture a particular 
creature was brought up in or because of innate characteristics of 
that type of creature. If a creature's race changes (for instance, if it 
dies and is reincarnated), it loses all racial bonuses it had in its 
previous form. 

\parhead{range increment:} Each full range increment of distance between an attacker using a ranged weapon and a target gives the 
attacker a cumulative \minus2 penalty on the ranged attack roll. Thrown 
weapons have a maximum range of five range increments. Projectile 
weapons have a maximum range of ten range increments. 

\parhead{range penalty:} A penalty applied to a ranged attack roll based on 
distance. See range increment. 

\parhead{ranged attack:} Any attack made at a distance with a ranged 
weapon, as opposed to a melee attack. 

\parhead{ranged attack roll:} An attack roll made with a ranged weapon. 
See attack roll. 

\parhead{ranged touch attack:} A touch attack made at range, as opposed 
to a melee touch attack. See touch attack. 

\parhead{ranged weapon:} A thrown or projectile weapon designed for 
ranged attacks. 

\parhead{ranger (Rgr):} A class made up of characters who are particularly 
skilled at adventuring in the wilderness. 

\parhead{ray:} A beam created by a spell. The caster must succeed on a 
ranged touch attack to hit with a ray. 

\parhead{reach weapon:} A long melee weapon, or one that has a long haft. 
Reach weapons allow the user to threaten or strike at opponents 10 
feet away with a melee attack roll. Most such weapons cannot be 
used to attack adjacent foes, however. 

\parhead{reaction:} Acting in response to a situation or circumstance 
beyond one's control. For example, the DM may call for a Listen 
check as a reaction to see if you hear something you weren't 
specifically trying to hear. 

\parhead{rebuke undead:} A supernatural ability to make undead cower by 
channeling negative energy. 

\parhead{redirect a spell:} Redirect an active spell's effect at a specific 
target or targets. Redirecting a spell is a move action and does not 
provoke an attack of opportunity. 

\parhead{Reflex save:} A type of saving throw, related to a character's ability 
to withstand damage thanks to his agility or quick reactions. 

\parhead{regeneration:} The ability of some creatures to regrow severed 
body parts and ruined organs, repair broken bones, and heal other 
damage. Severed body parts that are not reattached simply die, and 
the regenerating creature grows replacements at a rate specified in 
the individual spell or monster description. Most damage dealt to a 
naturally regenerating creature is treated as nonlethal damage, 
which heals at a fixed rate. However, certain attack forms (typically 
fire and acid) deal damage that does not convert to nonlethal 
damage. Such damage is not regenerated. Regeneration does not 
alter conditions that do not deal damage in hit points, such as 
poisoning or disintegration. 

\parhead{resistance to energy:} A creature with resistance to an energy 
type ignores a certain amount of damage dealt by that energy type 
each time it is dealt. For instance, a creature with fire resistance 10 
ignores the first 10 points of fire damage dealt by each attack. 
Resistance to energy doesn't affect the saving throw made against 
the attack (if any). Multiple sources of resistance to a certain energy 
type (such as a spell and a special quality of a monster) don't stack 
with each other; only the highest value applies to any given attack. If a creature takes damage from multiple attacks simultaneously, as from a \spell{scorching ray} spell, resistance to energy only applies once.

\parhead{result:} The numerical outcome of a check, attack roll, saving 
throw, or other 1d20 roll. The result is the sum of the natural die roll 
and all applicable modifiers. 

\parhead{rogue (Rog):} A class made up of characters who primarily rely on 
stealth and cunning rather than brute force or magical ability. 

\parhead{round:} A 6-second unit of game time used to manage combat. 
Every combatant may make at least one action every round. 

\parhead{saving throw (save):} A roll made to avoid (at least partially) 
damage or harm. The three types of saving throws are Fortitude, 
Reflex, and Will. 

\parhead{school of magic:} A group of related spells that work in similar 
ways. The eight schools of magic available to spellcasters are 
abjuration, conjuration, divination, enchantment, evocation, 
illusion, necromancy, and transmutation. 

\parhead{scribe:} Write a spell onto a scroll. 

\parhead{scry:} See and hear events from afar through the use of a spell or a 
magic item. 

\parhead{shaken:} Mildly fearful. A shaken character takes a \minus2 penalty on 
attack rolls, saving throws, checks, and AC. A shaken creature who becomes shaken again becomes frightened.

\parhead{shield modifier:}  A bonus to Armor Class granted by a shield or by a 
spell or magic effect that mimics a shield. A 
shield modifier granted by a spell or magic item typically takes the 
form of an invisible, tangible field of force that protects the 
recipient. A shield bonus applies against touch attacks. 

\parhead{sickened:} Mildly ill. A sickened character takes a \minus2 penalty on attack rolls, weapon damage rolls, saving throws, checks, and AC.

\parhead{silver piece (sp):} The most prevalent form of currency among 
commoners. Ten silver pieces are equivalent to 1 gold piece. 

\parhead{size:} The physical dimensions and/or weight of a creature or 
object. The sizes, from smallest to largest, are Fine, Diminutive, 
Tiny, Small, Medium, Large, Huge, Gargantuan, and Colossal. 

\parhead{size modifier:} The bonus or penalty derived from a creature's 
size category. Size modifiers of different kinds apply to Armor Class, 
attack rolls, Hide checks, combat maneuver checks, and various other checks. 

\parhead{skill:} A talent that a character acquires and improves through 
training. 

\parhead{skill check:} A check relating to use of a skill. The basic skill 
check = 1d20 \add skill rank \add the relevant attribute (or simply 
1d20 \add skill modifier). 

\parhead{skill modifier:} The bonus or penalty associated with a particular 
skill. Skill modifier = skill rank \add attribute \add miscellaneous 
modifiers. (Miscellaneous modifiers include racial bonuses, armor 
check penalty, situational modifiers, and so forth.) Skill modifiers 
apply to skill checks by characters in the course of using the 
corresponding skills. 

\parhead{skill points:} A measure of a character's ability to gain and 
improve skills.

\parhead{skill rank:} A number indicating how much training or experience a character has with a given skill. Skill rank is incorporated 
into the skill modifier, which in turn improves the chance of 
success for skill checks with that skill. 

\parhead{slowed:} A slowed creature can take only a single move action or standard action each turn, but not both, and takes a \minus2 penalty to Strength and Dexterity.

\parhead{Small:} A Small creature is typically between 2 feet and 4 feet in 
height or length and weighs between 8 pounds and 60 pounds. 

\parhead{sorcerer (Sor):} A class made up of characters who have inborn 
magical ability. 

\parhead{space:} The amount of floor space a creature requires to fight 
effectively, expressed as one dimension of a square area (for 
example, a creature with a space of 10 feet occupies a 10-foot by 10-foot area on the battle grid). Space determines how many creatures 
can fight side by side in a corridor, as well as how many creatures 
can attack a single opponent at once. A creature's space depends 
upon both its size and its body shape. Sometimes also called fighting 
space. 

\parhead{special qualities:} Characteristics possessed by certain monsters 
(and sometimes characters) that are distinctive in some way. The 
\booktitle{Monster Manual} has detailed information on all special qualities. 

\parhead{speed:} The number of feet a creature can move when taking a 
move action. 

\parhead{spell:} A one-time magical effect. The two primary categories of 
spells are arcane and divine. Clerics, druids, paladins, and rangers 
cast divine spells, while wizards, sorcerers, and bards cast arcane 
spells. Spells are further grouped into eight schools of magic. 

\parhead{spell completion item:} A magic item (typically a scroll) that 
contains a partially cast spell. Since the spell preparation step has 
already been completed, all the user need do to cast the spell is 
complete the final gestures or words normally required to trigger it. 
To use a spell completion item safely, the caster must be high 
enough level in the appropriate class to cast the spell already, 
though it need not be a known spell. A caster who does not fit this 
criterion has a chance of spell failure. Activating a spell completion 
item is a standard action and provokes attacks of opportunity just as 
casting a spell does. 

\parhead{spell failure:} The chance that a spell fails and is ruined when cast 
under less than ideal conditions; when a spell is cast to no effect. 

\parhead{spell level:} A number from 0 to 9 that indicates the general 
power of a spell. 

\parhead{spell-like ability (Sp):} A special ability with effects that resemble 
those of a spell. In most cases, a spell-like ability works just like the 
spell of the same name. 

\parhead{spell resistance (SR):} A special defensive ability that allows a 
creature or item to resist the effects of spells and spell-like abilities. 
Supernatural abilities are not subject to spell resistance. To 
overcome a creature's spell resistance, the caster of the spell or spell-
like ability must equal or exceed the creature's spell resistance with a 
caster level check. 

\parhead{spell slot:} The ``space" in a spellcaster's mind dedicated to holding a spell of a particular spell level. A spellcaster has enough spell 
slots to accommodate an entire day's allotment of spells. A spellcaster can always opt 
to cast a lower-level spell slot with a higher-level spell slot, if desired.

\parhead{spell trigger item:} A magic item (such as a wand) that produces a 
particular spell effect. Any spellcaster whose class spell list includes 
a particular spell knows how to use a spell trigger item that 
duplicates it, regardless of whether the character knows (or could 
know) that spell at the time. The user must determine what the spell 
stored in the item is before trying to use it. To activate the item, the 
user must speak a word, but no gesture or spell finishing is required. 
Activating a spell trigger item is a standard action and does not 
provoke attacks of opportunity. 

\parhead{spell version:} One of several variations of the same spell. The 
caster must select the desired version of the spell at the time of 
casting.  \spell{Transmute flesh and stone},  \spell{dispel magic,} and  \spell{plant growth} are examples of 
spells with multiple versions. 

\parhead{spellcaster:} A character capable of casting spells. 

\parhead{splash weapon:} A ranged 
weapon that splashes on impact, 
dealing damage to creatures who are 
within 5 feet of the spot where it lands as well as to targets it actually 
hits. Attacks with splash weapons are ranged touch attacks. 

\parhead{spontaneous casting:} The special ability of a cleric to spend a 
\spell{spell slot} to gain a \spell{cure} or \spell{inflict} spell of 
the same level or lower, or of a druid to spend a \spell{spell slot} to gain 
a \spell{summon nature's ally} spell of the same level or lower.

\parhead{square:} A square on the battle grid. A square is 1 inch on a side 
and represents a 5-foot by 5-foot area. The terms ``1 square" and ``5 
feet" are generally interchangeable. 

\parhead{stable:} Unconscious and at negative hit points, but not dying. A dying character who is stable regains no hit points, but stops losing them.

\parhead{stack:} Combine for a cumulative effect. In most cases, enhancement bonuses and competence bonuses do not stack with each other, but bonuses and penalties from other sources do stack. If the effects do not stack, only the best bonus or worst penalty 
applies. Spell effects that do 
not stack may overlap, coexist independently, or render one another 
irrelevant, depending on their exact effects. 

\parhead{staggered:} A staggered character may take a single move action or 
standard action each round (but not both, nor can she take full-round actions). It can still take a swift action each round. A character is staggered when she has nonlethal damage exactly equal to her current hit points.

\parhead{standard action:} The most basic type of action. Common 
standard actions include attacking with melee or ranged weapons, casting a 
spell, and using a magic item. In a typical round, a character can take 
a standard action and a move action, but he can't take a second 
standard action in place of his move action. 

\parhead{Strength (Str):} The ability that measures a character's muscle and 
physical power. 

\parhead{stunned:} A stunned creature drops everything held, can't take 
actions, takes a \minus2 penalty to AC, and loses his 
Dexterity and dodge modifiers to AC (if any). 

\parhead{subject:} A creature affected by a spell. 

\parhead{subschool:} A category of spells within a school of magic. For example, 
charm and compulsion are subschools within the school of enchantment. 

\parhead{subtype:} A subdivision  of creature type. For example, humans and elves are both  of the humanoid type, but each  of those races also constitutes its  own subtype of humanoid. 

\parhead{supernatural ability (Su):} A magical power that produces a 
particular effect, as opposed to a natural, extraordinary, or spell-like 
ability. Using a supernatural ability generally does not provoke an 
attack or opportunity. Supernatural abilities are not subject to 
dispelling, disruption or spell resistance. However, they do not 
function in areas where magic is suppressed or negated, such as 
inside an antimagic field. 

\parhead{suppress:} Cause a magical effect to cease functioning without 
actually ending it. When the supression ends, the spell effect is 
returns, provided it has not expired in the meantime. 

\parhead{surprise:} A special situation that occurs at the beginning of a 
battle if some (but not all) combatants are unaware of their opponents' presence. In this case, a surprise round happens before 
regular rounds begin. In initiative order (highest to lowest), those 
combatants who started the battle aware of their opponents each 
take a partial action during the surprise round. Creatures unaware of 
opponents are flat-footed through the entire surprise round and do 
not enter the initiative cycle until the first regular combat round.  

\parhead{take damage:} Be affected by damage (either lethals or nonlethal) 
from a successful attack. Damage dealt by an opponent does not 
necessarily equal damage taken, as various special defenses may 
reduce or negate damage from certain kinds of attacks. 

\parhead{take 10:} Assume a result of at least a 10 when rolling for certain skill checks or other rolls, treating any roll lower than 10 as if it were a 10. You normally can't take 10 on skill checks if distracted or threatened, such as during combat. 

\parhead{take 20:} To assume that a character makes sufficient retries to 
obtain the maximum possible check result (as if a 20 were rolled on 
d20). Taking 20 takes as much time as making twenty separate skill 
checks (usually at least 2 minutes). Taking 20 assumes that the 
character fails many times before succeeding, and thus can't be used 
if failure carries negative consequences. 

\parhead{target:} The intended recipient of an attack, spell, supernatural 
ability, extraordinary ability, or magical effect. If a targeted spell is 
successful, its recipient is known as the subject of the spell. 

\parhead{temporary hit points:} Hit points gained for a limited time 
through certain spells (such as \spell{aid}) and magical effects. When a 
character with temporary hit points is dealt damage, deduct the 
damage from temporary hit points first, then deduct any remaining 
damage (if any) to the character's actual (nontemporary) hit points. 
Temporary hit points can cause a character's hit point total to exceed 
its normal maximum. \spell{Temporary hit points never stack.}

\parhead{threat:} A possible critical hit. 

\parhead{threat range:} All natural die roll results that constitute in a threat 
when rolled for an attack roll. For most weapons, the threat range is 
20, but some weapons have threat ranges of 19--20 or even 18--20. 
Any attack roll that does not result in a hit is not a threat, whether or 
not it lies within the weapon's threat range. 

\parhead{threaten:} To be able to attack in melee without moving from 
your current space. A creature typically threatens all squares within 
its natural reach, even when it is not its turn to take an action. For 
Medium or Small creature this usually includes all squares adjacent 
to its space. Larger creatures threaten more squares, while smaller 
creatures may not threaten any squares except their own. 

\parhead{threatened area:} All squares within an opponent's reach. Certain actions provoke attack of opportunity when taken within a threatened area. 

\parhead{threatened square:} A square within an opponent's reach. Generally, characters threaten all adjacent squares, though reach weapons can alter this range. Certain actions provoke attack of 
opportunity when taken within a threatened square. 

\parhead{thrown weapon:} A ranged weapon that a character hurls at an 
enemy, such as a \spell{javelin}, as opposed to a projectile weapon. 

\parhead{Tiny:} A Tiny creature is typically between 1 and 2 feet in height 
or length and weighs between 1 and 8 pounds. 

\parhead{total concealment:} Attacks against a target with total concealment have a 50\% miss chance. Total concealment blocks line of sight. See concealment. 

\parhead{total cover:} Attacks against a target that has total cover automatically fail. Total cover blocks line of sight and line of effect. See cover. 

\parhead{touch attack:} An attack in which the attacker must connect with 
an opponent, but does not need to penetrate armor. Touch attacks 
may be either melee or ranged. The target's armor modifier and natural armor modifier do not apply to AC against a touch attack. 

\parhead{touch spell:} A spell that delivers its effect when the caster 
touches a target creature or object. Touch spells are delivered to 
unwilling targets by touch attacks. 

\parhead{trained:} Having at least 1 rank in a skill. Many skills can be used 
untrained by making a successful skill check using 0 skill ranks. 
Others, such as Spellcraft, can be used only by characters who are 
trained in that skill. 

\parhead{transitive plane:} A plane of existence often used to travel from 
one place (or plane) to another. The Astral Plane, the Ethereal Plane, 
and the Plane of Shadow are all transitive planes. 

\parhead{turn:} The point in the round at which you take your action(s). On 
your turn, you may perform one or more actions, as dictated by your 
current circumstances. 

\parhead{turn undead:} The supernatural ability to drive off or destroy 
undead by channeling positive energy. 

\parhead{turned:} Affected by a turn undead attempt. Turned undead flee 
for 10 rounds (1 minute) by the best and fastest means available to 
them. If they cannot flee, they cower. 

\parhead{two-handed weapon:} A weapon designed for use in two hands, 
such as a greatsword. A two-handed weapon is considered to be an 
object of the same size as its designated wielder (for example, a 
Medium greatsword -- a greatsword sized for a Medium creature -- is a Medium object). 

\parhead{type:} See creature type. 

\parhead{unarmed attack:} A melee attack made with no weapon in hand. 

\parhead{unarmed strike:} A successful blow, typically dealing nonlethal 
damage, from a character attacking without weapons. A monk can 
deal lethal damage with an unarmed strike, but others deal 
nonlethal damage. 

\parhead{unconscious:} Knocked out and helpless. Unconsciousness can 
result from having negative hit points, or from 
nonlethal damage in excess of current hit points. A character who is 
unconscious as a result of having negative hit points who becomes stable has a 10\% chance every hour to become 
conscious. A character who is unconscious as a result of having 
nonlethal damage in excess of current hit points has a 10\% chance 
every minute to wake up and be staggered. 

\parhead{untrained:} Having no ranks in a skill. Many skills can be used 
untrained by making a successful skill check using 0 skill ranks and 
including all other modifiers as normal. Other skills can be used 
only by characters who are trained in that skill. 

\parhead{use-activated item:} A magic item that activates upon typical 
usage for a normal item of its type. For example, a character can 
activate a potion by drinking it, a magic sword by swinging it, a lens 
by looking through it, or a cloak by wearing it. Characters do not 
learn what a use-activated item does just by wearing or using it 
unless the benefit occurs automatically with use. 

\parhead{Will save:} A type of saving throw, related to a character's ability 
to withstand damage thanks to his mental toughness. 

\parhead{Wisdom (Wis):} The ability that describes a character's willpower, 
common sense, perception, and intuition. 

\parhead{wizard (Wiz):} A class made up of characters who are schooled in 
the arcane arts. 
